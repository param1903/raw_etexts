\documentclass[11pt, openany]{book}
\usepackage[text={4.65in,7.45in}, centering, includefoot]{geometry}

\usepackage[table, x11names]{xcolor}
\usepackage{fontspec,realscripts}
\usepackage{polyglossia}

\usepackage{enumerate}
\pagestyle{plain}
\usepackage{fancyhdr}
\pagestyle{fancy}
\renewcommand{\headrulewidth}{0pt}
\usepackage{afterpage}
\usepackage{multirow}
\usepackage{niceframe}
\usepackage{amsmath}
\usepackage{amssymb}
\usepackage{graphicx}
\usepackage{longtable}
\usepackage{footnote}
\usepackage{perpage}
\MakePerPage{footnote}
%\usepackage{bigfoot}
%\DeclareNewFootnote[para]{default}
%\usepackage{dblfnote}
\usepackage{xspace}
%\newcommand\nd{\textsuperscript{nd}\xspace}
\usepackage{array}
\usepackage{emptypage}

\usepackage{pgf,tikz}
\usetikzlibrary{arrows}

\usepackage{hyperref}% Package for hyperlinks
\hypersetup{
colorlinks,
citecolor=black,
filecolor=black,
linkcolor=blue,
urlcolor=black
}

\usepackage[Devanagari, Latin]{ucharclasses}

\setdefaultlanguage{english}
\setotherlanguage{sanskrit}
\setmainfont[Scale=1]{Times New Roman}

\newfontfamily\s[Script=Devanagari, Scale=0.9]{Shobhika}
\newfontfamily\regular{Times New Roman}

\newcommand{\devanagarinumeral}[1]{%
	\devanagaridigits{\number \csname c@#1\endcsname}} % for devanagari page numbers

\setTransitionTo{Devanagari}{\s}
\setTransitionFrom{Devanagari}{\regular}

\XeTeXgenerateactualtext=1 % for searchable pdf

\begin{document}
\thispagestyle{empty}

\begin{center}
\niceframe{THE
\vspace{4mm}

{\Large PRINCESS OF WALES}
\vspace{6mm}

{\LARGE \textbf{SARASWATI BHAVANA TEXTS}}
\vspace{4mm}

{\large \textit{No. 57 (Part I)}}
\vspace{2mm}

\line(1,0){50}
\vspace{4mm}

{\Large EDITED BY}
\vspace{4mm}

MAHAMAHOPADHYAYA 
\vspace{2mm}

{\Large GOPI NATH KAVIRAJ, M.A. }
\vspace{2mm}

\line(1,0){50}
\vspace{4mm}

{\large THE }
\vspace{4mm}

{\huge \textbf{GANITA KAUMUDI }}
\vspace{4mm}
 
\line(1,0){50}
\vspace{6mm}

{\large Only Cover Printed by Ali Bukhsh at the Royal Printing Works, Godowlia, Benares; text printed at the Tara Printing Works, Benares; and published under the authority of the Government, United Provinces by the Superintendent,\\
Government Printing, Allahabad,}
\vspace{2mm}

\line(1,0){50}
\vspace{2mm}

1936.}
\end{center}

\newpage
\thispagestyle{empty}

\begin{center}
{\Huge \textbf{गणित कौमुदी}}
\vspace{4mm}

{\LARGE \textbf{( नारायणपण्डितकृता )}}
\vspace{4mm}

{\Large \textbf{प्रथमो भागः~।}}
\vspace{6mm}

काशीस्थराजकीयसंस्कृतमहाविद्यालयाध्यापकेन ज्यौतिषाचार्येण\\
\vspace{2mm}

पद्माकरद्विवेदिना संस्कृता~।
\vspace{6mm}

\line(1,0){50}
\vspace{6mm}

{\Large THE}
\vspace{6mm}

{\huge \textbf{GANITA KAUMUDI}}
\vspace{4mm}

{\Large (\textit{Part I})}
\vspace{6mm}

\line(1,0){50}
\vspace{8mm}

{\small \textbf{EDITED BY\textendash} }\\
\vspace{2mm}

PADMAKARA DVIVEDI JYAUTISHACHARYA,\\
\vspace{2mm}

Asst.\;Professor Government Sanskrit College,\\
\vspace{2mm}

BENARES.
\vspace{4mm}

\line(1,0){50}
\vspace{4mm}

1936
\end{center}

\newpage
\thispagestyle{empty}

\begin{center}
{\Large \textbf{भूमिका~।}}
\end{center}

\begin{sloppypar}
श्रीपूज्यपादपितृसङ्गृहीतपुस्तकेषु तेषां निधनात् पश्चान्मया श्रीनृसिंहनन्दननारायण-पण्डितरचिताया गणितकौमुद्या एकाशुद्धा हस्तलिखतप्रतिः प्राप्ता~। लुप्तप्रायस्य सुतरां~कुत्रा-प्यमुद्रितपूर्वस्य गणितकौमुदीग्रन्थस्यायं प्रथमप्रकाशनावसरः~। तत्प्रकाशनं सरस्वतीभवन-द्वारा सञ्जायते~।
\vspace{2mm}

एतद्विषयरसिकसज्जनानुरोधेन अस्य ग्रन्थस्य श्रेढीव्यवहारपर्यन्तस्य प्रथमभागस्य प्राकाश्यं तावन्मया कृतम्~। द्वितीयभागे सग्रन्थपरिचयं प्राक्कथनादिकं यथावसरं प्रकटी-करिष्यते~।
\vspace{2mm}

नेपालराज्यसरस्वतीसदनाद्गणितकौमुद्या एका प्रतिलिपिरस्मिन् वर्षे प्राप्ता~। सा मन्नि-कटवर्तिप्रतिलिपिसमाना~। अस्य मुद्रितप्रथमभागस्य २२ पृष्ठे २७ श्लोकमनु प्रेसकर्मचारि-गणासावधानतया यत् २८, २९, ३० श्लोकत्रयात्मकं पत्रं विनष्टं तत्रस्थास्ते श्लोकत्रयोऽमी यथा-स्थानं निवेशनीयाः अधोलिखिताः~।

\phantomsection \label{Ex 1.28}
\begin{quote}
{\large \textbf{{\color{purple}यूथं मत्तद्विपानां हरिपरुषरवैर्भीषणैर्भीतमस्मात्\renewcommand{\thefootnote}{१}\footnote{The reading -मास्मात् seems to be a typographical error.} \\
यूथाद्यातौ दलाङ्घ्री गिरिकटकतरं मूलदिग्भागहीनः~।\\
पञ्चांशः कच्छदेशं मदमुदितमनाः पञ्चभिर्हस्तिनीभिः\\
नागः पञ्चाननेन प्रधनवति सखे ते कियन्तः करीन्द्राः~॥~२८~॥}}}
\end{quote}

न्यासः~। {\small $\begin{matrix}
\mbox{{१ं}}\\
\vspace{-1mm}
\mbox{{४}}
\vspace{1mm}
\end{matrix}$}\; {\small $\begin{matrix}
\mbox{{१ं}}\\
\vspace{-1mm}
\mbox{{५}}
\vspace{1mm}
\end{matrix}$}~। दृश्यम् ६~। जाताः करिणः १००~।
\vspace{2mm}

ऋणस्वांशजातौ न्यासः~। १ं १ं~। मू.\,१~। दृश्यम् २५~। जातो राशिः ३६~।\\

अपि च\textendash 

\phantomsection \label{Ex 1.29}
\begin{quote}
{\large \textbf{{\color{purple}पारावतव्रजदलं कुतुकी मुमोच \\ 
शेषत्रिभागयुत\renewcommand{\thefootnote}{२}\footnote{The reading -युग- seems to be a typographical error.}शेषचतुर्थभागम्~।\\
मूलाङ्गभागरहितं च दशावशिष्टा \\
दृष्टा भुवि प्रवद ते खगगाः कति स्युः~॥~२९~॥}}}
\end{quote}
\end{sloppypar}

\afterpage{\fancyhead[RE,LO]{{\small{}}}}
\afterpage{\fancyhead[CE,CO]{({\small{~\thepage~}})}}
\afterpage{\fancyhead[LE,RO]{{\small{}}}}
\cfoot{}

\newpage
\renewcommand{\thepage}{\devanagarinumeral{page}}
\setcounter{page}{2}

\begin{sloppypar}
न्यासः~। स्वां.\;{\small $\begin{matrix}
\mbox{{१ं}}\\
\vspace{-1mm}
\mbox{{२}}
\vspace{1mm}
\end{matrix}$}~। शे.\;{\small $\begin{matrix}
\mbox{{१}}\\
\vspace{-1mm}
\mbox{{४}}
\vspace{1mm}
\end{matrix}$}~। मू.\;{\small $\begin{matrix}
\mbox{{१}}\\
\vspace{-1mm}
\mbox{{६}}
\vspace{1mm}
\end{matrix}$}~। दृश्यम् १०~। जाताः पारावताः ३६~।
\vspace{2mm}

ऋणशेषमूलजातौ न्यासः~। १ं १ं ४ं~। मू.\,१~। दृश्यम् १८~। जातो राशिः ३६~।\\

अपि च\textendash 

\phantomsection \label{Ex 1.30}
\begin{quote}
{\large \textbf{{\color{purple}सङ्ख्येऽसङ्ख्यबलात्कलिङ्गनृपतेः स्तम्बेरमाणां\renewcommand{\thefootnote}{१}\footnote{The reading -नृपते स्तम्बेरमाणं seems to be a typographical error.} दलं\\
सत्र्यंशं गदया व्यपोथयदथो भीमश्चपेटायुधः~।\\
तद्विश्लेषरसांशकं शरहतं\renewcommand{\thefootnote}{२}\footnote{The reading -रसांशकं रसहतं seems to be incorrect, as it means \,{\scriptsize $6 \times \dfrac{1}{6}$}, which is equal to 1, and it gives incorrect result.} मूलार्कभागोनितं\\
पञ्चोन्मीलितदन्तपादसुकराः स्युस्ते कतीभाः सखे~॥~३०~॥}}}
\end{quote}

न्यासः~। {\small $\begin{matrix}
\mbox{{१}}\\
\vspace{-1mm}
\mbox{{२}}
\vspace{1mm}
\end{matrix}$} \;{\small $\begin{matrix}
\mbox{{१}}\\
\vspace{-1mm}
\mbox{{१२}}
\vspace{1mm}
\end{matrix}$} \;{\small $\begin{matrix}
\mbox{{१}}\\
\vspace{-1mm}
\mbox{{६}}
\vspace{1mm}
\end{matrix}$}~। मू.\;{\small $\begin{matrix}
\mbox{{१}}\\
\vspace{-1mm}
\mbox{{१२}}
\vspace{1mm}
\end{matrix}$}~। दृश्यम् ५~। जाता गजाः १४४~।\\

अथ धनांशविमूलजातौ न्यासः~। {\small $\begin{matrix}
\mbox{{१}}\\
\vspace{-1mm}
\mbox{{३}}
\vspace{1mm}
\end{matrix}$} \;{\small $\begin{matrix}
\mbox{{१}}\\
\vspace{-1mm}
\mbox{{४}}
\vspace{1mm}
\end{matrix}$} \;{\small $\begin{matrix}
\mbox{{१}}\\
\vspace{-1mm}
\mbox{{१२}}
\vspace{1mm}
\end{matrix}$}~। मू.\,१ दृश्यम् १४~। जातो राशिः ३६~।\\

अस्य ग्रन्थस्य रचनाकालो नारयणपण्डितेन स्वयमेव लिखितस्तद्यथा

\begin{quote}
{\color{violet}गजनगरविमितशाके दुर्मुखवर्षे च बाहुले मासि~।\\
धातृतिथौ कृष्णदले गुरौ समाप्तिं गतं गणितम्~॥}
\end{quote}

अन्ते नान्यादृशं ग्रन्थस्यास्य प्रकाशनप्रकारमालोक्य कथमपि मदीयोपपत्यादिसहित-स्यास्य प्रकाशनव्यापारे ससाहसं प्रवर्त्तमानो गुणैकपक्षपातिनो गणका एव संशोधयितुं प्रयतेरन्निति मत्वा तानेव सानुनयमेतद्ग्रन्थोपरि शुभदृष्टिदानाय मुहुर्मुहुरभ्यर्थये~। यतो~नाति-शुद्धमेकमेवादर्शपुस्तकमवलम्ब्यास्य ग्रन्थस्य संशोधनं सम्पादनं च मया कृतम्~।\\

एतत्सम्पादनं हि\textendash 

\begin{quote}
{\color{violet}ईदृशं तादृशं चास्ति जल्पनेनेति किं फलम्~।\\
विद्यारत्नविदामग्रे व्यक्तदोषगुणं स्वयम्~॥}
\end{quote}
\vspace{8mm}

\begin{tabular}{ccr}
\hspace{10mm} & खजुरी & \\
 & बनारस क्याण्ट~। & \hspace{35mm} {\large \textbf{पद्माकर द्विवेदी~।}} \\ 
 & २४-४-३५
\end{tabular}
\end{sloppypar}

\newpage
\thispagestyle{empty}

\begin{center}
\textbf{श्रीजानकीवल्लभो विजयते~।}
\vspace{5mm}

{\small \textbf{अथ}}
\vspace{4mm}

{\LARGE \textbf{गणितकौमुदी~।}}
\vspace{4mm}

\line(1,0){60}
\end{center}
\vspace{2mm}

\phantomsection \label{1.1}
\begin{quote}
{\large \textbf{{\color{purple}नत्वेशं गणितार्णव-\\
वर्धनहेतुं तमोनुदं विमलाम्~।\\
बहुजनचकोरजीवन-\\
सम्पत्तिं गणितकौमुदीं वक्ष्ये~॥~१~॥}}}
\end{quote}
\vspace{-2mm}
 
\begin{center}
\textbf{अथ परिभाषा~।}
\end{center}
\vspace{-2mm}

\phantomsection \label{1.2}
\begin{quote}
{\large \textbf{{\color{purple}स्थानान्येकं दश शतमथो सहस्रायुते लक्षम्~।\\
प्रयुतमनु कोटिरर्बुदसरोजखर्वाण्यनु निखर्वम्~॥~२~॥}}}
\end{quote}
\vspace{-6mm}

\phantomsection \label{1.3}
\begin{quote}
{\large \textbf{{\color{purple}तदनु महाब्जं शङ्कुः \\
पारावारान्त्यमध्यानि~।\\ 
तस्मात्परार्धमिति दश-\\
गुणोत्तराणि क्रमेण सञ्ज्ञानि~॥~३~॥}}}
\end{quote}
\vspace{-6mm}

\phantomsection \label{1.4}
\begin{quote}
{\large \textbf{{\color{purple}नखमितकपर्दिकाभिः\\
काकिणिका चतसृभिः पणस्ताभिः~।\\
द्वादशभिस्तैर्द्रम्मः\renewcommand{\thefootnote}{१}\footnote{अत्र भास्करपाटीगणिताद्भिन्नौ द्रम्मनिष्कौ~।}\\
तैः षड्वर्गोन्मितैर्निष्कः~॥~४~॥}}}
\end{quote}

\newpage
\setcounter{page}{2}

\phantomsection \label{1.5}
\begin{quote}
{\large \textbf{{\color{purple}आहुस्तुला\renewcommand{\thefootnote}{१}\footnote{भास्करेण निजपाट्यां तुलासञ्ज्ञा न लिखिता~।}शतांशः\\
पलं पलाङ्घ्रिं तु कर्षसञ्ज्ञं च~।\\
तमपि सुवर्णं तन्नृप-\\
भागं माषं तदिषुलवं गुञ्जाम्~॥~५~॥}}}
\end{quote}
\vspace{-8mm}

\phantomsection \label{1.6}
\begin{quote}
{\large \textbf{{\color{purple}वल्लो भवेत् त्रिगुञ्जो\\
गद्याणो वल्लकैस्तु षोडशभिः~।\\
हस्तोऽङ्गुलैश्चतुर्भिः\\
षड्गुणितैर्दशकरो भवेद्दण्डः\renewcommand{\thefootnote}{२}\footnote{भास्कराचार्यस्तु चतुर्भिर्हस्तैर्दण्डं मन्यते~।}॥~६~॥}}}
\end{quote}
\vspace{-8mm}

\phantomsection \label{1.7}
\begin{quote}
{\large \textbf{{\color{purple}दण्डाष्टशतं क्रोशः\renewcommand{\thefootnote}{३}\footnote{भास्कराचार्यमते क्रोशमानं हस्ताः $=$ ४ $\times$ २००० $=$ ८००० $=$ १० $\times$ ८००~। अतो मतद्वयेऽपि क्रोशान्तःपातिहस्तसङ्ख्यायां न विप्रतिपत्तिः~।}\\
तैः क्रोशैर्योजनं चतुर्भिश्च~।\\
समचतुरस्त्रं विंशति-\\
दण्डभुजं तुल्यकर्णकं क्षेत्रम्~॥~७~॥}}}
\end{quote}
\vspace{-8mm}

\phantomsection \label{1.8}
\begin{quote}
{\large \textbf{{\color{purple}एतन्निवर्तनं स्यात् \\
समदण्डचतुःशतं\renewcommand{\thefootnote}{४}\footnote{The reading -चतुःशती seems to be a typographical error.} कोष्ठम्~।\\
हस्तोन्मितविस्तारा-\\
यामोच्छ्रायैः करोन्मितैर्गणकाः~॥~८~॥}}}
\end{quote}
\vspace{-8mm}

\phantomsection \label{1.9}
\begin{quote}
{\large \textbf{{\color{purple}घनहस्तमानमाहुः \\
नियतं तद्द्वादशास्रं यत्~। \\
सिद्धनृपभूप\textendash \,२४।१६।१६\textendash \,सङ्ख्या-\\
ङ्गुलोन्मितैर्दैर्घ्यविस्तरोच्छ्रायैः~॥~९~॥}}}
\end{quote}

\newpage
	
\phantomsection \label{1.10}
\begin{quote}
{\large \textbf{{\color{purple}मानं दृषत्करस्य हि \\
घनहस्ते तौ च साङ्घ्री २\,$\frac{{\footnotesize{\hbox{१}}}}{{\footnotesize{\hbox{४}}}}$\, स्तः\renewcommand{\thefootnote}{१}\footnote{घनहस्ताङ्गुलानि $=$ २४ $\times$ २४ $\times$ २४~। दृशत्कराङ्गुलानि $=$ २४ $\times$ १६ $\times$ १६ \;एकघनहस्ते दृषत्करमानम् $= \dfrac{{\footnotesize{\hbox{२४ $\times$ २४ $\times$ २४}}}}{{\footnotesize{\hbox{२४ $\times$ १६ $\times$ १६}}}} = \dfrac{{\footnotesize{\hbox{९}}}}{{\footnotesize{\hbox{४}}}} = \mbox{२}\dfrac{{\footnotesize{\hbox{१}}}}{{\footnotesize{\hbox{४}}}}$
\vspace{1mm}
}। \\
खारी विंशतिकुडवा \\
कुडवनृपांशेन पादिका ज्ञेया\renewcommand{\thefootnote}{२}\footnote{पादिकाया घनफलमङ्गुलात्मकम् $=$ २१६ $=$ ६$^{\scriptsize{\hbox{३}}}$, हस्तात्मकम् $= \dfrac{{\footnotesize{\hbox{६}^{\scriptsize{\hbox{३}}}}}}{{\footnotesize{\hbox{२४}^{\scriptsize{\hbox{३}}}}}} = \dfrac{{\footnotesize{\hbox{१}}}}{{\footnotesize{\hbox{४}^{\scriptsize{\hbox{३}}}}}} = \dfrac{{\footnotesize{\hbox{१}}}}{{\footnotesize{\hbox{६४}}}}$ एकस्यां खार्यां पादिकासङ्ख्या $=$ १६ $\times$ २० $=$ ३२०, अतः खार्यां घनफलम् $= \dfrac{{\footnotesize{\hbox{३२०}}}}{{\footnotesize{\hbox{६४}}}} =$ ५~। एतेनात्रोका खारी भास्करोक्ताभिः पञ्चभिर्मागधखारीभिः समेति प्रतीयते~।}॥~१०~॥}}}
\end{quote}
\vspace{-8mm}

\phantomsection \label{1.11}
\begin{quote}
{\large \textbf{{\color{purple}रसशशिनयनघनाङ्गुल\textendash \,(२१६)\textendash \\
मितिर्भवेत् पादिकायाश्च~। \\
घटिकाषष्टिर्द्युनिशं \\
मासस्तत्त्रिंशता तु तैर्मासैः~॥~११~॥\\
वर्षं द्वादशभिः स्यात्\\
इति परिभाषोदिता गणिते~॥~११\,$\frac{{\footnotesize{\hbox{१}}}}{{\footnotesize{\hbox{२}}}}$॥}}}
\end{quote}
\vspace{-2mm}
 
\begin{center}
\textbf{इति परिभाषा~।}\\
\vspace{8mm}

{\Large \textbf{अथाभिन्नपरिकर्माष्टकम्~।}} \\
\vspace{4mm}

\textbf{तत्र सङ्कलितव्यवकलितयोः करणसूत्रं गीत्यर्धम्~।}
\end{center}
\vspace{-3mm}

\phantomsection \label{1.12}
\begin{quote}
{\large \textbf{{\color{purple}स्वस्थाने समजात्योः \\
योगः कार्यो वियोगश्च~॥~१२~॥}}}
\end{quote}

\newpage

\noindent \textbf{उद्देशकः~।}

\phantomsection \label{Ex 1.1}
\begin{quote}
\textbf{{\color{red}स्तम्बेरमा जलधयो विषया महीध्राः\\ 
पञ्चेन्दवो जिनगुणा रसबाहवश्च~।\\
द्रम्माः सनिष्कयुगलाः कथयाश्वमीषां \\
योगं च तेष्वपकृतेष्वयुताच्च शेषम्~॥~१~॥}}
\end{quote}

अत्र समजात्योर्योगं कृत्वेति द्रम्मान् कृत्वा स्वस्थाने न्यासः~। ८।४।५।७।१५।३२४।२६।७२~।\\

एतेषां योगे जाता द्रम्माः ४६१~। एतेष्वयुताच्छोधितेषु जातं शेषम् ९५३९~॥
\vspace{2mm}

\begin{center}
\textbf{इति सङ्कलितव्यवकलिते~।}\\
\vspace{8mm}

\textbf{गुणने करणसूत्रमार्यात्रयम्~।}
\end{center}

\phantomsection \label{1.13}
\begin{quote}
{\large \textbf{{\color{purple}गुण्यस्याधो गुणकं \\
विन्यस्य कपाटसन्धिविधिनैव~।\\ 
गुण्यान्त्यं गुणकेना-\\
हन्यादुत्सार्य पृथगेव~॥~१३~॥}}}
\end{quote}
\vspace{-8mm}

\phantomsection \label{1.14}
\begin{quote}
{\large \textbf{{\color{purple}गुणखण्डैर्वा गुण्यो \\
रूपविभागाहतो युतिस्तु फलम्~।\\ 
स्थानविभागैर्गुणितः \\
स्वस्थानयुतः फलं वापि~॥~१४~॥}}}
\end{quote}
\vspace{-8mm}

\phantomsection \label{1.15}
\begin{quote}
{\large \textbf{{\color{purple}भक्तो येन विशुध्यति \\
तेन च लब्ध्याहतः फलं वा स्यात्~।\\ 
गुणगुण्ययोरभेदः \\
तथैव गुण्याहते गुणके~॥~१५~॥}}}
\end{quote}

\newpage

\begin{sloppypar}
\noindent \textbf{उदाहरणम्~।}

\phantomsection \label{Ex 1.2}
\begin{quote}
\textbf{{\color{red}पञ्चमहीधरनयनप्रमिता धृतिमगुणाः कति ते स्युः~।\\ 
रूपस्थानविभागजखण्डे विगुणजं तथापवर्तनजम्~॥~२~॥}}
\end{quote}

न्यासः~। गुण्यः २७५~। गुणकः १८~। गुणिते जातम् ४९५०~। अथ गुणकस्य रूपविभागे खण्डे ७।११ आभ्यां गुण्ये गुणिते जाते १९२५।३०२५~। स्वस्थानयुते जातं तदेव ४९५०~। गुणकस्य स्थानविभागे खण्डे १।८ आभ्यां गुण्ये गुणिते जातम् २७५।२२०० स्वस्थानयुते जातं तदेव ४९५०। अथवा गुणकः १८ त्रिभिर्भक्तो लब्धम् ३~। एभिस्त्रिभिश्च गुण्ये गुणिते जातं तदेव ४९५०~। अत्र गुणकगुण्ययोरभेदः~। यदि गुण्यस्य गुणकत्वं तदा गुणकस्य गुण्यत्वमिति~। यथा त्रिगुणेषु पञ्चसु पञ्चदश तथा पञ्चगुणेषु त्रिषु पञ्चदशैव~। एवं सर्वत्र गुणकारविधिः~।\\

\begin{center}
\textbf{भागहारे करणसूत्रमार्या~।}
\end{center}
\vspace{-2mm}

\phantomsection \label{1.16}
\begin{quote}
{\large \textbf{{\color{purple}भाज्यादन्त्याद्धारः \\
शुध्यति येनाहतः फलं तत् स्यात्~।\\ 
अपवर्त्य भाज्यहारौ \\
केनापि समेन वा विभजेत्~॥~१६~॥}}}
\end{quote}

\textbf{उदाहरणम्~।}\\

पूर्वगुणनफलस्य भजनार्थं न्यासः~। भाज्यः ४९५०~। भाजकः १८~। भागे हृते जातम् २७५~। अथवा भाज्यभाजकौ नवभिरपवर्त्तितौ ५५०।२ भागे हृते जातम् २७५~। \renewcommand{\thefootnote}{$\star$}\footnote{हारोत्पत्तिनामापवर्तनान्वेषणप्रकारः~। भागादानाख्य एकादशो व्यवहारेऽस्य यत्र निर्दिष्टसङ्ख्याया गुण्य-गुणकभावापन्ना दृढा अङ्काः पृथक क्रियन्ते~।}हारोत्पत्तिः पुरतो भागादाने वक्ष्ये~।

\begin{center}
\textbf{इति गुणनभजने~।}
\end{center}

\end{sloppypar}

\newpage

\begin{center}
\textbf{वर्गे करणसूत्रम्~।}
\end{center}
\vspace{-2mm}

\phantomsection \label{1.17}
\begin{quote}
{\large \textbf{{\color{purple}सदृशद्विवधो वर्गः \\
स्थाप्योऽन्त्यकृतिर्द्विसङ्गुणान्त्यगुणाः~।\\
स्वस्वोपरि च परेऽङ्काः \\
त्यक्त्वान्त्याङ्कान् मुहुः समुत्सार्य~॥~१७~॥}}}
\end{quote}
\vspace{-8mm}

\phantomsection \label{1.18}
\begin{quote}
{\large \textbf{{\color{purple}अथवाभीष्टयुतोनित-\\
राशिवधोऽभीष्टवर्गयुग्वर्गः~।\\
रूपाद्द्व्युत्तरपदयुति\renewcommand{\thefootnote}{१}\footnote{रूपाद्रूपमारभ्य द्व्युत्तराणां सङ्ख्यानां पदप्रमितानां युतिः पदवर्गः स्यादित्यर्थः~। आदि $=$ १, चयम् $=$ २, गच्छम् $=$ प, अत्र {\color{violet}'व्येकपदघ्नचयो मुखयुक् स्यात्'} इत्यादिना श्रेढीफलमानीतं सत् पदवर्गसमं भवतीति सुगमोपपत्तिः~। तद्यथा १ $+$ ३ $+$ ५ $+$ ७ $+$ ...... $+$ प~। एषां योगः यो $= \frac{{\footnotesize{\hbox{प}}}}{{\footnotesize{\hbox{२}}}}$ [२ आदि $+$ (प $-$ १) च] $= \frac{{\footnotesize{\hbox{प}}}}{{\footnotesize{\hbox{२}}}}$ [२ $\times$ १ $+$ (प $-$ १) २] $= \frac{{\footnotesize{\hbox{प}}}}{{\footnotesize{\hbox{२}}}}$ [२ $+$ २ प $-$ २] $= \frac{{\footnotesize{\hbox{प}}}}{{\footnotesize{\hbox{२}}}} \times$ २ प $=$ $\hbox{प}^{\scriptsize{\hbox{२}}}$~।
\vspace{2mm}
}-\\
रन्तरकृतियुग्वधश्चतुर्गुणितः\renewcommand{\thefootnote}{२}\footnote{यस्य वर्गः कर्तव्योऽस्ति तस्यातुल्यं खण्डद्वयं कृत्वा खण्डान्तरवर्गे खण्डयोश्चतुर्गुणिते वधे योजिते सति तस्य वर्गः स्यादित्यर्थः~।}॥~१८~॥}}}
\end{quote}

\noindent \textbf{उदाहरणम्~।}

\phantomsection \label{Ex 1.3}
\begin{quote}
\textbf{{\color{red}एकादिकानां वद मे नवानां \\
सखे दशानां च ससप्तकानाम्~।\\
सपञ्चवर्गद्विशतीयुतस्य \\
वर्गान् पृथक् चेदयुतस्य वेत्सि~॥~३~॥}}
\end{quote}

न्यासः १।२।३।४।५।६।७।८।९।१७।१०२२५~। \\

प्रकारैर्जाता वर्गाः १।४।९।१६।२५।३६।४९।६४।८१।२८९।१०४५५०६२५~। 

\begin{center}
\textbf{इति वर्गः~।}
\end{center}

\newpage

\begin{center}
\textbf{अथ वर्गमूले सूत्रम्~।}
\end{center}
\vspace{-2mm}

\phantomsection \label{1.19}
\begin{quote}
{\large \textbf{{\color{purple}विषमं सममित्यन्त्यात् \\
विषमाद्वर्गं त्यजेद्द्विगुणितेन~।\\ 
मूलेन समं विभजेत् \\
तदाद्यविषमात्त्यजेच्च लब्धकृतिम्~॥~१९~॥}}}
\end{quote}
\vspace{-8mm}

\phantomsection \label{1.20}
\begin{quote}
{\large \textbf{{\color{purple}लब्धं द्विघ्नं पङ्क्त्यां \\
विन्यस्य च तान् समुत्सार्य~।\\ 
पुनरपि विभजेदेवं \\
पङ्क्त्यङ्कदलं प्रजायते मूलम्~॥~२०~॥}}}
\end{quote}

\noindent \textbf{उदाहरणम्~।}\\
\vspace{-2mm}

पूर्ववर्गाणां ~मूलार्थं ~न्यासः~। १।४।९।१६।२५।३६।४९।६४।८१।२८९।१०४५५०६२५~। लब्धानि यथा क्रमेण मूलानि १।२।३।४।५।६।७।८।९।१७।१०२२५~।
\vspace{2mm}

\begin{center}
\textbf{इति वर्गमूलम्~।}\\
\vspace{8mm}

\textbf{अथ घने करणसूत्रमार्यात्रयम्~।}
\end{center}

\phantomsection \label{1.21}
\begin{quote}
{\large \textbf{{\color{purple}त्रिसदृशहतिर्घनः स्यात् \\
स्थाप्योऽन्त्यघनोऽन्त्यपूर्वयोर्वर्गौ~।\\ 
त्रिगुणावाद्यन्त्यगुणौ \\
क्रमशः स्थानान्तरेण संयुक्तः~॥~२१~॥}}}\renewcommand{\thefootnote}{}\footnote{\hspace{-7mm} {\color{violet}`चतुर्गुणस्य घातस्य युतिवर्गस्य चान्तरम्'} इत्येतत्पद्यवैपरीत्येनैतदुपपद्यते~। तद्यथा~। कल्प्यते कस्यापि राशेर-तुल्यं खण्डद्वयं या.\,का~। तदा राशिवर्गः $=$ रा$^{\scriptsize{\hbox{२}}} =$ (या + का)$^{\scriptsize{\hbox{२}}} =$ या$^{\scriptsize{\hbox{२}}} +$ २ याका $+$ का$^{\scriptsize{\hbox{२}}} =$ या$^{\scriptsize{\hbox{२}}} +$ २ याका $-$ ४ याका $+$ का$^{\scriptsize{\hbox{२}}} +$ ४ याका $=$ (या $-$ का)$^{\scriptsize{\hbox{२}}} +$ ४ याका~।}
\end{quote}

\newpage

\phantomsection \label{1.22}
\begin{quote}
{\large \textbf{{\color{purple}आदिघनश्च घनः स्यात् \\
अन्त्यस्याव्यवहितस्य राशेश्च~।\\
एकादिचयेनान्त्योऽ-\\
न्त्यत्रिहतोऽथवैकयुग्युतश्च घनः~॥~२२~॥}}}
\end{quote}
\vspace{-8mm}

\phantomsection \label{1.23}
\begin{quote}
{\large \textbf{{\color{purple}त्रिघ्नो राशिः खण्ड-\\
द्व्याहतः खण्डघनयुतियुतोऽथवा\renewcommand{\thefootnote}{१}\footnote{The reading युतो वा doesn’t fit in the meter of the verse.}।\\ 
राशेर्मूलस्य घनः \\
तद्वर्गो घनसमो भवति~॥~२३~॥}}}
\end{quote}

\noindent \textbf{उदाहरणम्~।}

\phantomsection \label{Ex 1.4}
\begin{quote}
\textbf{{\color{red}एकादिकानां च पृथग्नवानाम् \\
अष्टादशानामपि षट्कृतेश्च~।\\
घनं च षष्ट्येकयुतत्रिशत्या \\
वदाशु सप्ताधिकपञ्चशत्याः~॥~४~॥}}
\end{quote}

न्यासः ~१।२।३।४।५।६।७।८।९।१८।३६।३६१।५०७~। \\

जाता ~घनाः ~१।८।२७।६४।१२५।२१६।३४३।५१२।७२९।५८३२।४६६५६।४७०४५८८१।\\
१३०३२३८४३~।

\begin{center}
\textbf{इति घनः~।}\\
\vspace{8mm}

\textbf{अथ घनमूले सूत्रम्~।}
\end{center}
\vspace{-3mm}

\phantomsection \label{1.24}
\begin{quote}
{\large \textbf{{\color{purple}घनमघने द्वे च घनात् \\
अन्त्याद्घनतो घनं विशोध्य पदम्~।\\
अन्यत्र न्यस्यास्य च \\
पदस्य कृत्या त्रिसङ्गुणया~॥~२४~॥}}}
\end{quote}

\newpage

\phantomsection \label{1.25}
\begin{quote}
{\large \textbf{{\color{purple}विभजेत्तदादिमाप्तं \\
स्थाप्य तदादौ पृथक् च तद्वर्गम्~।\\
त्रिगुणान्त्यघ्नं जह्यात् \\
तत्पूर्वघने च लब्धिघनम्~॥~२५~॥}}}
\end{quote}

\noindent \textbf{उदाहरणम्~।}\\
\vspace{-2mm}

पूर्वघनानां मूलार्थं न्यासः~। १।८।२७।६४।१२५।२१६।३४३।५१२।७२९।५८३२।४६६५६।\\
४७०४५८८१।१३०३२३८४३~।\\

जातानि यथाक्रमं घनमूलानि ~१।२।३।४।५।६।७।८।९।१८।३६।३६१।५०७~।
\vspace{2mm}

\begin{center}
\textbf{इति घनमूलम्~।}\\
\vspace{4mm}

\textbf{एवमष्टाभिन्नपरिकर्माणि~।}\\
\vspace{8mm}

{\Large \textbf{अथ भिन्नपरिकर्माष्टकम्~।}} \\
\vspace{4mm}

\textbf{तत्रादावंशसवर्णनम्~। तत्रापि भागजात्यादौ सूत्रम्~।}
\end{center}
\vspace{-3mm}

\phantomsection \label{1.26}
\begin{quote}
{\large \textbf{{\color{purple}समहृतहरसङ्गुणिता-\\
वन्योन्यांशच्छिदौ समच्छित्त्यै~। \\
अहरे हारो रूपं \\
प्रभागके हरवधस्तथांशवधः~॥~२६~॥}}}
\end{quote}
\vspace{-8mm}

\phantomsection \label{1.27}
\begin{quote}
{\large \textbf{{\color{purple}भागगणं रूपेषु \\
स्वर्णं कुर्याद्धराभिगुणितेषु~।\\ 
आद्यच्छिदधिच्छिद्घ्नः \\
स्वांशयुगूनश्च स्वहरहताद्यंशाः~॥~२७~॥}}}
\end{quote}

\noindent \textbf{भागजातावुद्देशकः~।}

\phantomsection \label{Ex 1.5}
\begin{quote}
\textbf{{\color{red}द्व्यब्ध्यङ्गार्कलवानां सदृशच्छेदा भवन्ति कथमेषाम्~।\\
त्र्यंशौ त्रयः शरांशा रूपाणि च पञ्च कथमेषाम्~॥~५~॥}}
\end{quote}

\newpage

न्यासः~। ~$\dfrac{{\footnotesize{\hbox{१}}}}{{\footnotesize{\hbox{२}}}}\; \dfrac{{\footnotesize{\hbox{१}}}}{{\footnotesize{\hbox{४}}}}\; \dfrac{{\footnotesize{\hbox{१}}}}{{\footnotesize{\hbox{६}}}}\; \dfrac{{\footnotesize{\hbox{१}}}}{{\footnotesize{\hbox{१२}}}}$~। ~जाताः समच्छेदाः\; $\dfrac{{\footnotesize{\hbox{६}}}}{{\footnotesize{\hbox{१२}}}}\; \dfrac{{\footnotesize{\hbox{३}}}}{{\footnotesize{\hbox{१२}}}}\; \dfrac{{\footnotesize{\hbox{२}}}}{{\footnotesize{\hbox{१२}}}}\; \dfrac{{\footnotesize{\hbox{१}}}}{{\footnotesize{\hbox{१२}}}}$~। \\

न्यासः~। ~$\dfrac{{\footnotesize{\hbox{२}}}}{{\footnotesize{\hbox{३}}}}\; \dfrac{{\footnotesize{\hbox{३}}}}{{\footnotesize{\hbox{५}}}}\; \dfrac{{\footnotesize{\hbox{५}}}}{{\footnotesize{\hbox{१}}}}$~। ~जाताः समच्छेदाः\; $\dfrac{{\footnotesize{\hbox{१०}}}}{{\footnotesize{\hbox{१५}}}}\; \dfrac{{\footnotesize{\hbox{९}}}}{{\footnotesize{\hbox{१५}}}}\; \dfrac{{\footnotesize{\hbox{७५}}}}{{\footnotesize{\hbox{१५}}}}$~।\\
\vspace{2mm}

\textbf{अथ भागप्रभागजातावुदाहरणम्~।}

\phantomsection \label{Ex 1.6}
\begin{quote}
\textbf{{\color{red}निष्कत्र्यंशयुगस्य षड्लवदलं तत्पञ्चमांशत्रयं\\
तस्याष्टांशनवांशषोडशलवः केनापि लुब्धेन च~।\\
कस्मैचिल्लघुमार्गणाय स भृशं संप्रार्थितेनादरात्\\
दत्तस्तन्मितिमाशु कोविद वद प्रौढः प्रभागेऽसि चेत्~॥~६~॥}}
\end{quote}

न्यासः। ~$\dfrac{{\footnotesize{\hbox{१}}}}{{\footnotesize{\hbox{१}}}}$~। $\dfrac{{\footnotesize{\hbox{२}}}}{{\footnotesize{\hbox{३}}}}$~। $\dfrac{{\footnotesize{\hbox{१}}}}{{\footnotesize{\hbox{६}}}}$~। $\dfrac{{\footnotesize{\hbox{१}}}}{{\footnotesize{\hbox{२}}}}$~। $\dfrac{{\footnotesize{\hbox{३}}}}{{\footnotesize{\hbox{५}}}}$~। $\dfrac{{\footnotesize{\hbox{१}}}}{{\footnotesize{\hbox{८}}}}$~। $\dfrac{{\footnotesize{\hbox{१}}}}{{\footnotesize{\hbox{९}}}}$~। $\dfrac{{\footnotesize{\hbox{१}}}}{{\footnotesize{\hbox{१६}}}}$~। ~लब्धो वराटकः १~।\\
\vspace{2mm}

\textbf{भागानुबन्धभागापवाहयोरुदाहरणम्~।}

\phantomsection \label{Ex 1.7}
\begin{quote}
\textbf{{\color{red}रूपत्रयं पञ्चलवाधिकं च \\
त्रिभिश्च षट् सप्तलवैर्युतानि~।\\
त्र्यंशोनिते द्वे कथयाशु पञ्च \\
व्यङ्घ्रीणि भो वेत्सि सवर्णनं चेत्~॥~७~॥}}
\end{quote}

न्यासः~। ~३$\dfrac{{\footnotesize{\hbox{१}}}}{{\footnotesize{\hbox{५}}}}$~। ६$\dfrac{{\footnotesize{\hbox{३}}}}{{\footnotesize{\hbox{७}}}}$ ~सवर्णिते जातम् ~$\dfrac{{\footnotesize{\hbox{१६}}}}{{\footnotesize{\hbox{५}}}}$~। $\dfrac{{\footnotesize{\hbox{४५}}}}{{\footnotesize{\hbox{७}}}}$~। \\

न्यासः~। ~२$\dfrac{{\footnotesize{\hbox{१ं}}}}{{\footnotesize{\hbox{३}}}}$~। ५$\dfrac{{\footnotesize{\hbox{१ं}}}}{{\footnotesize{\hbox{४}}}}$ ~सवर्णिते जातम् ~$\dfrac{{\footnotesize{\hbox{५}}}}{{\footnotesize{\hbox{३}}}}$~। $\dfrac{{\footnotesize{\hbox{१९}}}}{{\footnotesize{\hbox{४}}}}$~। \\
\vspace{2mm}

\textbf{स्वांशानुबन्धस्वांशापवाहयोरुदाहरणम्~।}

\phantomsection \label{Ex 1.8}
\begin{quote}
\textbf{{\color{red}अङ्घ्रिः स्वत्रिलवाधिको निजशरांशाढ्योऽथ सप्तांशकः\\
सस्वाङ्घ्रिः स्वषडंशयुग्वद सखे कीदृक् सवर्णक्रमः~।\\
स्वार्धोनौ त्रिलवौ निजाङ्घ्रिरहितौ पञ्चांशकाः षट् च ते\\
स्वाङ्गांशेन विवर्जिताः स्वविनवांशाः सप्त भो वेत्सि चेत्~॥~८~॥}}
\end{quote}

\newpage

न्यासः \;{\small $\begin{matrix}
\vspace{1.5mm}
\mbox{{$\frac{{\footnotesize{\hbox{१}}}}{{\footnotesize{\hbox{४}}}} ~~~\frac{{\footnotesize{\hbox{१}}}}{{\footnotesize{\hbox{७}}}}$}}\\
\vspace{1.5mm}
\mbox{{$\frac{{\footnotesize{\hbox{१}}}}{{\footnotesize{\hbox{३}}}} ~~~\frac{{\footnotesize{\hbox{१}}}}{{\footnotesize{\hbox{४}}}}$}}\\
\mbox{{$\frac{{\footnotesize{\hbox{१}}}}{{\footnotesize{\hbox{५}}}} ~~~\frac{{\footnotesize{\hbox{१}}}}{{\footnotesize{\hbox{६}}}}$}}
\vspace{1mm}
\end{matrix}$}\; सवर्णिते जातम् \;$\dfrac{{\footnotesize{\hbox{२}}}}{{\footnotesize{\hbox{५}}}}$~। $\dfrac{{\footnotesize{\hbox{५}}}}{{\footnotesize{\hbox{२४}}}}$~।\\

द्वितीयोदाहरणे न्यासः~। \;{\small $\begin{matrix}
\vspace{1.5mm}
\mbox{{$\frac{{\footnotesize{\hbox{२}}}}{{\footnotesize{\hbox{३}}}} ~~~\frac{{\footnotesize{\hbox{६}}}}{{\footnotesize{\hbox{५}}}}$}}\\
\vspace{1.5mm}
\mbox{{$\frac{{\footnotesize{\hbox{१ं}}}}{{\footnotesize{\hbox{२}}}} ~~~\frac{{\footnotesize{\hbox{१ं}}}}{{\footnotesize{\hbox{६}}}}$}}\\
\mbox{{$\frac{{\footnotesize{\hbox{१ं}}}}{{\footnotesize{\hbox{४}}}} ~~~\frac{{\footnotesize{\hbox{७ं}}}}{{\footnotesize{\hbox{९}}}}$}}
\vspace{1mm}
\end{matrix}$}\; सवर्णिते जातम् \;$\dfrac{{\footnotesize{\hbox{१}}}}{{\footnotesize{\hbox{४}}}}$~। $\dfrac{{\footnotesize{\hbox{२}}}}{{\footnotesize{\hbox{९}}}}$~।\\

एवं सर्वत्र~।
\vspace{-1mm}

\begin{center}
\textbf{इति सवर्णनजातिषट्कम्~।}\\
\vspace{8mm}

\textbf{अथ भिन्नसङ्कलितव्यवकलितयोः सूत्रम्~।}
\end{center}
\vspace{-3mm}

\phantomsection \label{1.28.1}
\begin{quote}
{\large \textbf{{\color{purple}सदृशच्छेदांशानां \\
प्राग्वत् संयोजनं वियोगो वा~।}}}
\end{quote}

\noindent \textbf{उदाहरणम्~।}

\phantomsection \label{Ex 1.9}
\begin{quote}
\textbf{{\color{red}तिथ्यङ्गदिग्रामलवान् सखे मे \\
सम्पीड्य सर्वान् वद कोविदाशु~।\\
तानेव रूपाच्च विशोध्य किं स्यात् \\
शेषं विभिन्नेऽस्ति परिश्रमश्चेत्~॥~९~॥}}
\end{quote}

न्यासः~। ~$\dfrac{{\footnotesize{\hbox{१}}}}{{\footnotesize{\hbox{१५}}}}, \dfrac{{\footnotesize{\hbox{१}}}}{{\footnotesize{\hbox{६}}}}, \dfrac{{\footnotesize{\hbox{१}}}}{{\footnotesize{\hbox{१०}}}}, \dfrac{{\footnotesize{\hbox{१}}}}{{\footnotesize{\hbox{३}}}}$\; योगे जातम् \;$\dfrac{{\footnotesize{\hbox{२}}}}{{\footnotesize{\hbox{३}}}}$\; एतान् रूपाद्विशोध्य जातम् \;$\dfrac{{\footnotesize{\hbox{१}}}}{{\footnotesize{\hbox{३}}}}$~।
\vspace{2mm}

\begin{center}
\textbf{इति भिन्नसङ्कलितव्यवकलिते~।}
\end{center}

\newpage

\begin{center}
\textbf{अथ भिन्नगुणने सूत्रम्~।}
\end{center}
\vspace{-1mm}

\phantomsection \label{1.28}
\begin{quote}
{\large \textbf{{\color{purple}छेदवधेन विभक्तो-\\
ऽंशवधो भिन्ने फलं गुणने~॥~२८~॥}}}
\end{quote}

\noindent \textbf{उदाहरणम्~।}

\phantomsection \label{Ex 1.10}
\begin{quote}
\textbf{{\color{red}सत्र्यंशरूपाणि सखे चतुर्भिः साष्टांशकैः पञ्च हतानि किं स्यात्~।\\
द्वौ पञ्चमांशापचितौ विनिघ्नौ त्र्यंशद्वयाढ्येन च रूपकेण~॥~१०~॥}}
\end{quote}

न्यासः~। गुणकः \;$\dfrac{{\footnotesize{\hbox{३३}}}}{{\footnotesize{\hbox{८}}}}$\; गुण्यः \;$\dfrac{{\footnotesize{\hbox{१६}}}}{{\footnotesize{\hbox{३}}}}$~। ~गुणिते जातम् \;$\dfrac{{\footnotesize{\hbox{२२}}}}{{\footnotesize{\hbox{१}}}}$~।\\

न्यासः \;$\dfrac{{\footnotesize{\hbox{५}}}}{{\footnotesize{\hbox{३}}}}$~। $\dfrac{{\footnotesize{\hbox{९}}}}{{\footnotesize{\hbox{५}}}}$\; फलम् \;$\dfrac{{\footnotesize{\hbox{३}}}}{{\footnotesize{\hbox{१}}}}$~।
\vspace{-1mm}

\begin{center}
\textbf{इति भिन्नगुणनम्~।}\\
\vspace{8mm}

\textbf{अथ भिन्नभागहारे सूत्रम्~।}
\end{center}
\vspace{-3mm}

\phantomsection \label{1.29.1}
\begin{quote}
{\large \textbf{{\color{purple}कृत्वा भाजकराशेः \\
हरलवपरिवर्तनं विधिः प्राग्वत्~।}}}
\end{quote}

उदाहरणम्~। पूर्वगुणनफलानां स्वगुणच्छेदानां न्यासः \;$\dfrac{{\footnotesize{\hbox{३३}}}}{{\footnotesize{\hbox{८}}}}, \dfrac{{\footnotesize{\hbox{२२}}}}{{\footnotesize{\hbox{१}}}}$~। $\dfrac{{\footnotesize{\hbox{५}}}}{{\footnotesize{\hbox{३}}}}, \dfrac{{\footnotesize{\hbox{३}}}}{{\footnotesize{\hbox{१}}}}$~। भागे गृहीते जातौ गुण्यौ\; ५$\dfrac{{\footnotesize{\hbox{१}}}}{{\footnotesize{\hbox{३}}}}$~। १$\dfrac{{\footnotesize{\hbox{४}}}}{{\footnotesize{\hbox{५}}}}$
\vspace{-1mm}

\begin{center}
\textbf{इति भिन्नभागहारः~।}\\
\vspace{8mm}

\textbf{अथ भिन्नवर्गादौ सूत्रम्~।}
\end{center}
\vspace{-3mm}

\phantomsection \label{1.29}
\begin{quote}
{\large \textbf{{\color{purple}कुर्याद्धारांशकयोः \\
वर्गौ च घनौ पदे तथा प्राग्वत्~॥~२९~॥}}}
\end{quote}

\noindent \textbf{उदाहरणम्~।}

\phantomsection \label{Ex 1.11}
\begin{quote}
\textbf{{\color{red}सत्र्यंशपञ्चरूपाणां वर्गं वर्गात् पदं वद~।\\ 
घनं तस्माद्घनपदं सखे भिन्नं प्रवेत्सि चेत्~॥~११~॥}}
\end{quote}

न्यासः~। $\dfrac{{\footnotesize{\hbox{१६}}}}{{\footnotesize{\hbox{३}}}}$\; जातो वर्गः\; $\dfrac{{\footnotesize{\hbox{२५६}}}}{{\footnotesize{\hbox{९}}}}$~। अस्माद्वर्गमूलम्\; ५$\dfrac{{\footnotesize{\hbox{१}}}}{{\footnotesize{\hbox{३}}}}$~। जातो घनः\; $\dfrac{{\footnotesize{\hbox{४०९६}}}}{{\footnotesize{\hbox{२७}}}}$~। अस्माद्घन-मूलम्\; ५$\dfrac{{\footnotesize{\hbox{१}}}}{{\footnotesize{\hbox{३}}}}$~। 
\vspace{-1mm}

\begin{center}
\textbf{इति भिन्नवर्गादि~।}
\end{center}

\newpage

\begin{center}
\textbf{अथ शून्यपरिकर्मसु सूत्रम्~।}
\end{center}
\vspace{-3mm}

\phantomsection \label{1.30}
\begin{quote}
{\large \textbf{{\color{purple}राशिः खेन युतोनोऽ-\\
विकृतस्तस्मिँश्च खेन गुणिते खम्~।\\ 
खस्य वधादौ खं स्यात् \\
क्षेपसमं खं समायोगे~॥~३०~॥}}}
\end{quote}

\noindent \textbf{उदाहरणम्~।}

\phantomsection \label{Ex 1.12}
\begin{quote}
\textbf{{\color{red}किं शून्येन युते शते विरहिते तस्मिंश्च शून्याहते\\
किं स्याच्छून्यहतं च खं\renewcommand{\thefootnote}{१}\footnote{खं खहृतं सदा शून्यं न भवति\textendash \,इत्येतदर्थं चलनकलनं विलोक्यम्~।} खहृतमप्याचक्ष्व शीघ्रं मम~।\\
किं वर्गं च पदं घनं घनपदं शून्यस्य खे संयुता\\
अष्टौ कोविद शून्यकर्मणि तव प्रौढिः प्रभूतास्ति चेत्~॥~१२~॥}}
\end{quote}

न्यासः~। राशिः १००~। अयं खेन युत ऊनितोऽविकृत एव जातः १००~। न्यासः~। १००~। अस्मिन् खेन हते जातम् ०~। खं वानेन हतं जातम् ०~। खहृत् खम् ०~। अस्य वर्गः ०~। वर्गमूलम् ०~। घनः ०~। घनमूलम् ०~। अस्मिन्नष्टौ ८ युक्ता जाताः क्षेपसमाः ८~।\\

अत्र पाटीगणिते खहरे कृते लोकस्य व्यवहृतौ प्रतीतिर्नास्तीत्यतोऽत्र खहरो नोक्तः~। अस्मदीये बीजगणिते बीजोपयोगित्वात् तत्र खहरः कथितः~।

\begin{center}
\textbf{इति परिकर्माणि समाप्तानि~।}\\
\vspace{8mm}

{\Large \textbf{अथ सङ्क्रमणे सूत्रम्~।}}
\end{center}
\vspace{-3mm}

\phantomsection \label{1.31}
\begin{quote}
\renewcommand{\thefootnote}{२}\footnote{भास्करसङ्क्रमणाख्यानुरूपमेवेदम्~।}{\large \textbf{{\color{purple}योगो द्विष्ठोऽन्तरयुत-\\
हीनस्तावर्धितौ च राशी स्तः~।\\ 
आह्वा कारणस्यास्य\renewcommand{\thefootnote}{३}\footnote{The reading कारणस्यास्य च doesn’t fit in meter.} \\
सङ्क्रमणं सङ्क्रमश्च सङ्क्रामः~॥~३१~॥}}}
\end{quote}

\newpage

\begin{sloppypar}
\noindent \textbf{उदाहरणम्~।}

\phantomsection \label{Ex 1.13}
\begin{quote}
\textbf{{\color{red}राश्योर्योगे ययोः षष्टिस्त्रियुता वियुतौ नव~।\\ 
तौ राशी कोविद क्षिप्रं सङ्क्रामं वेत्सि चेद्वद~॥~१३~॥}}
\end{quote}

न्यासः~। योगः ६३~। वियोगः ९~। अनयोर्योगः ७२~। अन्तरम् ५४~। अनयोरर्धे जातौ राशी ३६।२७~।

\begin{center}
\textbf{सङ्क्रमणान्तरे सूत्रम्~।}
\end{center}
\vspace{-3mm}

\phantomsection \label{1.32}
\begin{quote}
\renewcommand{\thefootnote}{१}\footnote{{\color{violet}``वर्गान्तरं राशिवियोगभक्तम्"} इत्यादि {\color{violet}भास्करो}क्तमेवेदम्~।}{\large \textbf{{\color{purple}वर्गान्तरं तु राश्योः \\
वियोगभक्तं भवेद्योगः~।\\ 
योगहृतमन्तरं स्यात् \\
ताभ्यां सङ्क्रामतो राशी~॥~३२~॥}}}
\end{quote}

\noindent \textbf{उदाहरणम्~।}

\phantomsection \label{Ex 1.14}
\begin{quote}
\textbf{{\color{red}चतुःशती कयो राश्योर्दृष्टा वर्गान्तरं सखे~।\\ 
राश्यन्तरेऽष्टौ योगे वा शतं तौ वद वेत्सि चेत्~॥~१४~॥}}
\end{quote}

न्यासः~। राश्योर्वर्गान्तरम् ४००। राश्योरन्तरम्~। राश्योर्वर्गान्तरे राश्यन्तरहृते जातो राश्योर्योगः ५०~। \hyperref[1.31]{'योगो द्विष्ठ-'} इति जातौ राशी २९।२१~।\\

पुनर्न्यासः~। राश्योर्वर्गान्तरम् ४००। राश्योर्योगः १००। वर्गान्तरं योगहृतं जातमन्तरम् ४~। \hyperref[1.31]{'योगो द्विष्ठ-'} इति जातौ राशी ५२।४८~।
\vspace{2mm}

\begin{center}
\textbf{सङ्क्रमणान्तरे सूत्रम्~।}
\end{center}
\vspace{-3mm}

\phantomsection \label{1.33}
\begin{quote}
\renewcommand{\thefootnote}{२}\footnote{अत्रोपपत्तिः~। कल्प्यते राशी या, का~। अनयोर्वर्गयुतिः}{\large \textbf{{\color{purple}वर्गसमासाद्द्विगुणात्\\
अन्तरवर्गोनितात् पदं योगः~॥~३३~॥}}}
\end{quote}

\noindent \textbf{उदाहरणम्~।}

\phantomsection \label{Ex 1.15.1}
\begin{quote}
\textbf{{\color{red}वर्गयोगः शतं राश्योरन्तरं द्विमितं ययोः~।\\ 
तौ राशी शीघ्रमाचक्ष्व जानासि गणितं यदि~॥}}
\end{quote}
\end{sloppypar}

\newpage

\begin{sloppypar}
न्यासः~। वर्गयोगः १००~। अन्तरम् २~। वर्गयोगाद्द्विगुणात् २००~। अन्तरवर्गेण ४ ऊनितात् १९६~। पदं जातो योगः १४~। \hyperref[1.31]{'योगो द्विष्ठ-'} इति जातौ राशी ८।६~।\renewcommand{\thefootnote}{}\footnote{$=$ या$^{\scriptsize{\hbox{२}}} +$ का$^{\scriptsize{\hbox{२}}}$, द्विगुणा $=$ २ या$^{\scriptsize{\hbox{२}}} +$ २ का$^{\scriptsize{\hbox{२}}}$~। अस्या अन्तरवर्गः (या $-$ का)$^{\scriptsize{\hbox{२}}} =$ या$^{\scriptsize{\hbox{२}}} -$ २ याका $+$ का$^{\scriptsize{\hbox{२}}}$ विशोध्य जातो युतिवर्गः $=$ या$^{\scriptsize{\hbox{२}}} +$ २ याका $+$ का$^{\scriptsize{\hbox{२}}}$~। अस्य पदं योगो भवत्येव~। {\color{violet}'कर्णस्य वर्गाद्द्विगुणाद्विशोध्यः'} इत्यादि {\color{violet}भास्करो}क्तमेवेदम्~।
\vspace{2mm}
}
\vspace{2mm}

\begin{center}
\textbf{सङ्क्रमणान्तरे सूत्रम्~।}
\end{center}
\vspace{-3mm}

\phantomsection \label{1.34}
\begin{quote}
\renewcommand{\thefootnote}{१}\footnote{अत्रोपपत्तिः~। कल्प्यते राशी या, का~। तदा प्रश्नोक्त्या
\vspace{2mm}

\hspace{2mm} ब $=$ याका \hspace{7mm} $\Rightarrow$ \;ब$^{\scriptsize{\hbox{२}}} =$ या$^{\scriptsize{\hbox{२}}}$का$^{\scriptsize{\hbox{२}}}$
\vspace{2mm}

\hspace{2mm} अं $=$ या$^{\scriptsize{\hbox{२}}} -$ का$^{\scriptsize{\hbox{२}}}\; \Rightarrow$ \;अं$^{\scriptsize{\hbox{२}}} =$ या$^{\scriptsize{\hbox{४}}} -$ २\,या$^{\scriptsize{\hbox{२}}}$का$^{\scriptsize{\hbox{२}}} +$ का$^{\scriptsize{\hbox{४}}}$
\vspace{2mm}

अतः अं$^{\scriptsize{\hbox{२}}} +$ ४\,ब$^{\scriptsize{\hbox{२}}} =$ या$^{\scriptsize{\hbox{४}}} +$ २\,या$^{\scriptsize{\hbox{२}}}$का$^{\scriptsize{\hbox{२}}} +$ का$^{\scriptsize{\hbox{४}}} =$ (या$^{\scriptsize{\hbox{२}}} +$ का$^{\scriptsize{\hbox{२}}}$)$^{\scriptsize{\hbox{२}}}$
\vspace{2mm}

ततः $\sqrt{\hbox{अं}^{\scriptsize{\hbox{२}}} + \hbox{४\,ब}^{\scriptsize{\hbox{२}}}} =$ या$^{\scriptsize{\hbox{२}}} +$ का$^{\scriptsize{\hbox{२}}}$
\vspace{2mm}

\hspace{19mm} अं $=$ या$^{\scriptsize{\hbox{२}}} -$ का$^{\scriptsize{\hbox{२}}}$
\vspace{2mm}

\hspace{2mm} आभ्यां सङ्कमणतो राशिवर्गौ ततस्तन्मूलाभ्यां राशी स्त इत्युपपद्यते~।}{\large \textbf{{\color{purple}वधवर्गो वधः कल्प्यः कृत्योरन्तरमन्तरम्~।\\ 
ताभ्यां सङ्क्रमतो राशी स्यातां मूले तयोः पृथक्~॥~३४~॥}}}
\end{quote}

\noindent \textbf{उदाहरणम्~।}

\phantomsection \label{Ex 1.15.2}
\begin{quote}
\textbf{{\color{red}वर्गान्तरं ययो राश्योः पञ्चसप्तेन्दवस्तथा~।\\ 
वधः शतत्रयं वत्स तौ राशी वद वेत्सि चेत्~॥}}
\end{quote}

न्यासः~। राश्योर्वधः ३००~। वर्गान्तरम् १७५~। अत्र वधवर्गो वधः ९००००~। कृत्य-न्तरमन्तरम् १७५~। \hyperref[1.35]{'राश्योर्विवरकृतियुतात्'} इति वक्ष्यमाणसूत्रेण जातो योगः ६२५~। \hyperref[1.31]{'योगो द्विष्ठ-'} इति सूत्रेण जातौ राशी ४००।२२५~। अनयोर्मूले २०।१५ एतावेव वास्तवौ राशी~।
\end{sloppypar}

\newpage

\begin{sloppypar}
\begin{center}
\textbf{सूत्रम्~।}
\end{center}
\vspace{-5mm}

\phantomsection \label{1.35}
\begin{quote}
\renewcommand{\thefootnote}{१}\footnote{अत्रोपपत्तिः~। {\color{violet}'चतुर्गुणस्य घातस्य युतिवर्गस्य चान्तरम्'} इत्यादि {\color{violet}भास्कर}विधिना स्फुटा~।}{\large \textbf{{\color{purple}राश्योर्विवरकृतियुतात् \\
चतुराहतघाततः पदं योगः~।\\ 
योगकृतेश्चतुराहत-\\
घातोनायाः पदं विवरम्~॥~३५~॥}}}
\end{quote}

\noindent \textbf{उदाहरणम्~।}

\phantomsection \label{Ex 1.15}
\begin{quote}
\textbf{{\color{red}कयो राश्योर्वधः षष्टिरन्तरं सप्त का युतिः~।\\ 
युतितोऽन्तरमाचक्ष्व जानासि यदि सङ्क्रमम्~॥~१५~॥}}
\end{quote}

न्यासः~। राशिघातः ६०~। राश्यन्तरम् ७~। राश्यन्तरकृतिः ४९~। अनया, राशिघातात् ६० चतुर्गुणात् २४० युक्तात् २८९ मूलम् १७ जातो योगः~। \hyperref[1.31]{'योगो द्विष्ठ-'} इति क्रियया जातौ राशी १२।५~।
\vspace{2mm}

\begin{center}
\textbf{सूत्रम्~।}
\end{center}
\vspace{-5mm}

\phantomsection \label{1.36}
\begin{quote}
\renewcommand{\thefootnote}{२}\footnote{अत्रोपपत्तिः~। कल्प्यते राशी या, का तदा प्रश्नोक्त्या}{\large \textbf{{\color{purple}राश्योरन्तरकृतियुक् \\
द्विघ्नो घातः\renewcommand{\thefootnote}{३}\footnote{The reading घातश्च doesn’t fit in meter.} कृतिसमासः~।\\
तस्माद्द्विगुणवधयुतात् \\
मूलं सञ्जायते योगः~॥~३६~॥}}}
\end{quote}

\noindent \textbf{उदाहरणम्~।}

\phantomsection \label{Ex 1.16}
\begin{quote}
\textbf{{\color{red}अन्तरं पञ्च यद्राश्योर्वधस्तु त्रिशती सखे~।\\ 
कृतियोगं योगकृतिं योगं च वद वेत्सि चेत्~॥~१६~॥}}
\end{quote}

न्यासः~। राश्योरन्तरम् ५~। राश्योर्वधः ३००~। राश्यन्तरवर्गेण २५ द्विगुणो राशिवधो ६०० युतो ६२५ जातो वर्गयोगः~। वर्गयोगेन ६२५ द्विगुणे राशिवधे ६०० युते जातो योगवर्गः १२२५~। अस्य मूलम्
\end{sloppypar}

\newpage

\begin{sloppypar}
\noindent अतो योगः ३५~। अन्तरमुद्दिष्टम् ५~। \hyperref[1.31]{'योगो द्विष्ठ-'} इति जातौ राशी २०।१५~।\renewcommand{\thefootnote}{}\footnote{$\begin{matrix}
\mbox{{अं = या $-$ का}}\\
\vspace{-1mm}
\mbox{{घा = याका~~~~}}
\vspace{1mm}
\end{matrix}\; \Rightarrow \begin{matrix}
\mbox{{अं$^{\scriptsize{\hbox{२}}} +$ २\,घा = (या $-$ का)$^{\scriptsize{\hbox{२}}} +$ २\,याका}}\\
\vspace{-1mm}
\mbox{{~~~~= या$^{\scriptsize{\hbox{२}}} +$ का$^{\scriptsize{\hbox{२}}}$~।}}
\vspace{1mm}
\end{matrix}$ \\

ततः या$^{\scriptsize{\hbox{२}}} +$ का$^{\scriptsize{\hbox{२}}} +$ २\,याका $=$ (या + का)$^{\scriptsize{\hbox{२}}} =$ अं$^{\scriptsize{\hbox{२}}} +$ ४\,घा = यु$^{\scriptsize{\hbox{२}}}$
\vspace{1mm}

इति सर्व भास्करोक्तमेव~।
\vspace{2mm}
}\\
\vspace{2mm}

\textbf{सूत्रम्~।}

\phantomsection \label{1.37}
\begin{quote}
\renewcommand{\thefootnote}{१}\footnote{अत्रोपपत्त्यर्थं \hyperref[1.33]{३३} सूत्रं द्रष्टव्यम्~।}{\large \textbf{{\color{purple}वर्गसमासाद्द्विगुणात् \\
युतिकृतिहीनात् पदं विवरम्~।}}}
\end{quote}

\noindent \textbf{उदाहरणम्~।}

\phantomsection \label{Ex 1.17}
\begin{quote}
\textbf{{\color{red}शतं कृतियुती राश्योर्ययोर्योगश्चतुर्दश~।\\ 
तौ राशी कथयाशु त्वं वेत्सि सङ्क्रमणं यदि~॥~१७~॥}}
\end{quote}

न्यासः~। वर्गयोगः १००~। राशियोगः १४~। अत्रापि वर्गयोगाद्द्विगुणात् २०० राशियोग-वर्गेण १९६ हीनात् ४ पदं जातं विवरम्~। \hyperref[1.31]{"योगो द्विष्ठ-"} इति जातौ राशी ८।६~। शेषं क्षेत्रोपयोगि तत्रैव वक्ष्ये~।

\begin{center}
\textbf{इति सङ्कमणम्~।}\\
\vspace{6mm}

{\Large \textbf{अथ जातिसमुदाये सूत्रम्~।}}
\end{center}
\vspace{-3mm}

\phantomsection \label{1.38}
\begin{quote}
\renewcommand{\thefootnote}{२}\footnote{{\color{violet}'उद्देशकालापवदिष्टराशिरि'}त्यादि {\color{violet}भास्करो}क्तमेवेदम्~।}{\large \textbf{{\color{purple}उद्देशालापकवत् \\
ज्ञेयं रूपं प्रकल्प्य गुणितं वा~॥~३७~॥ \\
भक्तं सहितं रहितं \\
कृत्वा कर्मानुपातादि~।\\ 
रूपसमुत्थफलं तत् \\
दृश्ये हारो भवत्येव~॥~३८~॥}}}
\end{quote}
\end{sloppypar}

\newpage

\begin{sloppypar}
\noindent \textbf{उदाहरणम्~।}

\phantomsection \label{Ex 1.18}
\begin{quote}
\textbf{{\color{red}त्र्यब्ध्यर्कांशसमन्वितो निजनिजाब्ध्यंशस्त्रिभागान्वितः\\
सूर्यांशाङ्घ्रितदन्तरद्वययुतो राशिस्तु षष्टिर्भवेत्~।\\
व्यंशोऽर्कस्मृतिभानवोऽथ सपदः षोढा सषड् दृश्यको\\
मूलोनस्तु षडूनदृश्य इति कः स्याज्जातिषट्कत्रये~॥~१८~॥}}
\end{quote}

धनांशजातौ न्यासः ~$\dfrac{{\footnotesize{\hbox{१}}}}{{\footnotesize{\hbox{३}}}}$~। $\dfrac{{\footnotesize{\hbox{१}}}}{{\footnotesize{\hbox{४}}}}$~। $\dfrac{{\footnotesize{\hbox{१}}}}{{\footnotesize{\hbox{१२}}}}$~। दृश्यः ६०~। जातो राशिः ३६~।\\
\vspace{2mm}

धनस्वांशजातौ न्यासः \;{\small $\begin{matrix}
\vspace{0.5mm}
\mbox{{१}}\\
\vspace{2mm}
\mbox{{$\frac{{\footnotesize{\hbox{१}}}}{{\footnotesize{\hbox{४}}}}$}}\\
\mbox{{$\frac{{\footnotesize{\hbox{१}}}}{{\footnotesize{\hbox{३}}}}$}}
\vspace{1mm}
\end{matrix}$}~। दृश्यः ६०~। जातो राशिः ३६~।\\
\vspace{2mm}

विश्लेषजातौ न्यासः \;$\dfrac{{\footnotesize{\hbox{१}}}}{{\footnotesize{\hbox{१२}}}} + \dfrac{{\footnotesize{\hbox{१}}}}{{\footnotesize{\hbox{४}}}} + {\hbox{२}}\,\left(\dfrac{{\footnotesize{\hbox{१}}}}{{\footnotesize{\hbox{४}}}} - \dfrac{{\footnotesize{\hbox{१}}}}{{\footnotesize{\hbox{१२}}}}\right)$~। दृश्यः ६०~। राशिः ३६~।\\
\vspace{2mm}

व्यंशः, पूर्वोदितप्रश्नत्रये यद्यंशा ऋणास्तदा क्रमेण दृश्या अकस्मृतिभानयो द्वादशाष्टा-दश द्वादश स्युः~। यथा \\

ऋणजातौ न्यासः ~$\dfrac{{\footnotesize{\hbox{१ं}}}}{{\footnotesize{\hbox{३}}}}$~। $\dfrac{{\footnotesize{\hbox{१ं}}}}{{\footnotesize{\hbox{४}}}}$~। $\dfrac{{\footnotesize{\hbox{१ं}}}}{{\footnotesize{\hbox{१२}}}}$~। दृश्यः १२~। जातो राशिः ३६~।\\
\vspace{2mm}

भागापवाहजातौ न्यासः \;{\small $\begin{matrix}
\vspace{0.5mm}
\mbox{{१}}\\
\vspace{2mm}
\mbox{{$\frac{{\footnotesize{\hbox{१}}}}{{\footnotesize{\hbox{४}}}}$}}\\
\mbox{{$\frac{{\footnotesize{\hbox{१ं}}}}{{\footnotesize{\hbox{३}}}}$}}
\vspace{1mm}
\end{matrix}$}~। दृश्यः १८~। जातो राशिः ३६~।\\
\vspace{2mm}

ऋणविश्लेषजातौ न्यासः \;$\dfrac{{\footnotesize{\hbox{१ं}}}}{{\footnotesize{\hbox{१२}}}} + \dfrac{{\footnotesize{\hbox{१ं}}}}{{\footnotesize{\hbox{४}}}} + {\hbox{२}}\,\left(\dfrac{{\footnotesize{\hbox{१}}}}{{\footnotesize{\hbox{४}}}} - \dfrac{{\footnotesize{\hbox{१}}}}{{\footnotesize{\hbox{१२}}}}\right)$~। दृश्यः १२~। जातो राशिः ३६~।\\
\vspace{2mm}

उपरि विहितः पोढा षड्विधः प्रश्नः सपदो राशिमूलेन सहितस्तदा सषड् दृश्यकः पूर्वो-दितो दृश्यः षडधिको भवति,
\end{sloppypar}

\newpage

षट्सूदाहरणेषु क्रमेण दृश्याः ६६।६६।६६।१८।२४।१८~।\\

अथ पूर्वोदितः षड्विधः प्रश्नो मूलोनस्तदा षडूनदृश्यो भवति~। तदा षट्सूदाहरणेषु क्रमेण दृश्याः ५४।५४।५४।६।१२।६~।\\

\noindent \textbf{अपि च~।}

\phantomsection \label{Ex 1.19}
\begin{quote}
\textbf{{\color{red}अविरलदलशाले सौरभोत्पत्तिकाले \\
स्थितमलिकुलमस्याब्ध्यर्कभागौ परस्मात्~।\\
त्वरितममिलतां ते\renewcommand{\thefootnote}{१}\footnote{The reading त्वरितममिलतांस्ते is grammatically incorrect.} भूपसङ्ख्याः प्रजाताः \\
कथय गणक वृन्दं भ्रामरं प्राक्स्थितं मे~॥~१९~॥}}
\end{quote}

न्यासः $\dfrac{{\footnotesize{\hbox{१}}}}{{\footnotesize{\hbox{४}}}}$~। $\dfrac{{\footnotesize{\hbox{१}}}}{{\footnotesize{\hbox{१२}}}}$~। दृश्यः १६~। अत्र रूपमिष्टं प्रकल्प्य जातो राशिः १२~।\\

\noindent \textbf{अपि च~।}

\phantomsection \label{Ex 1.20}
\begin{quote}
\textbf{{\color{red}क्रौञ्चावली गगनवर्त्मनि सञ्चरन्ती \\
स्वत्र्यंशयुक् स्वदलयुक् च पुनः सषट्का~।\\
दृष्टाः खगास्त्रिगुणदिक्प्रमिताः सखेऽस्मिन् \\
वृन्दे कति प्रवद तेऽस्ति परिश्रमश्चेत्~॥~२०~॥}}
\end{quote}

न्यासः \;{\small $\begin{matrix}
\vspace{0.5mm}
\mbox{{१}}\\
\vspace{2mm}
\mbox{{$\frac{{\footnotesize{\hbox{१}}}}{{\footnotesize{\hbox{३}}}}$}}\\
\mbox{{$\frac{{\footnotesize{\hbox{१}}}}{{\footnotesize{\hbox{२}}}}$}}
\vspace{1mm}
\end{matrix}$}~। दृश्यः ३०~। ऋणम् ६~। दृश्ये २४ जाताः क्रौञ्चाः १२~।\\

\noindent \textbf{अपि च~।}

\phantomsection \label{Ex 1.21}
\begin{quote}
\textbf{{\color{red}शालेये नवपल्लवे सकलिके यत् कोकिलानां कुलं\renewcommand{\thefootnote}{२}\footnote{The reading कोकिलाङ्कुलं doesn’t seem to be correct, and also doesn’t fit in meter.}\\
तत्त्र्यंशः सहकारतोऽष्टमलवः किक्किल्लतोऽभ्येत्य च~।\\
विश्लेषोऽप्यमिलत्तयोस्त्रिगुणितः फुल्लादशोकात् सखे\\
जातास्ते शतमाशु कोविद कति स्युः पूर्ववृन्दे पिकाः~॥~२१~॥}}
\end{quote}

\newpage

न्यासः ~$\dfrac{{\footnotesize{\hbox{१}}}}{{\footnotesize{\hbox{३}}}}$~। $\dfrac{{\footnotesize{\hbox{१}}}}{{\footnotesize{\hbox{८}}}}$~। $\dfrac{{\footnotesize{\hbox{५}}}}{{\footnotesize{\hbox{८}}}}$~। दृश्यः १००~। अत्रेष्टं रूपं प्रकल्प्य
जाताः पिकाः ४८~।\\

\noindent \textbf{अपि च~।}

\phantomsection \label{Ex 1.22}
\begin{quote}
\textbf{{\color{red}उत्तुङ्गपीवरपयोधरभारनम्राः \\
कान्ताचये\renewcommand{\thefootnote}{१}\footnote{The reading -नम्रा कान्ता च ये seems to be a typographical error.} द्विलववेदलवाङ्गभागाः~।\\
आनद्धवंशततवाद्यविचक्षणे स्युः \\
वृन्दे कति प्रवद तत्र चला किलैका~॥~२२~॥}}
\end{quote}

न्यासः ~$\dfrac{{\footnotesize{\hbox{१ं}}}}{{\footnotesize{\hbox{२}}}}$~। $\dfrac{{\footnotesize{\hbox{१ं}}}}{{\footnotesize{\hbox{४}}}}$~। $\dfrac{{\footnotesize{\hbox{१ं}}}}{{\footnotesize{\hbox{६}}}}$~। दृश्यः १~। जाताः कान्ताः १२~।\\

\noindent \textbf{अपि च~।}

\phantomsection \label{Ex 1.23}
\begin{quote}
\textbf{{\color{red}दृष्ट्वा पद्मपरागपिङ्गसलिले हंसे सरोहर्षिते\\
वासायावरयत्यलिव्रज इतः प्रोत्थाय भीतो ययौ~।\\
जातीं पञ्चलवः सखे सचरणः शेषत्रिभागान्वितं\\
शेषार्धं कुटजं भ्रमन्ति गगने भृङ्गाश्च षट् ते कति~॥~२३~॥}}
\end{quote}

न्यासः ~$\dfrac{{\footnotesize{\hbox{१ं}}}}{{\footnotesize{\hbox{५}}}}$~। $\dfrac{{\footnotesize{\hbox{१ं}}}}{{\footnotesize{\hbox{४}}}}$~। $\dfrac{{\footnotesize{\hbox{१ं}}}}{{\footnotesize{\hbox{३}}}}$~। $\dfrac{{\footnotesize{\hbox{१ं}}}}{{\footnotesize{\hbox{२}}}}$~। दृश्यः ६~। जाता भ्रमराः ३६~।\\

\noindent \textbf{अपि च~।}

\phantomsection \label{Ex 1.24}
\begin{quote}
\textbf{{\color{red}गजचयदलं यूथाद्यातं प्रफुल्लसरोजिनीं \\
शरलवदलं रम्भाकुञ्जेषु भञ्जनलालसम्~।\\
दलितविवरस्यार्धं तुङ्गाद्रिसानुजशल्लक-\\
ग्रसनचपलं दन्तीभीभिर्युतः कति पञ्चभिः~॥~२४~॥}}
\end{quote}

न्यासः ~$\dfrac{{\footnotesize{\hbox{१ं}}}}{{\footnotesize{\hbox{२}}}}$~। $\dfrac{{\footnotesize{\hbox{१}}}}{{\footnotesize{\hbox{५}}}}\,.\,\dfrac{{\footnotesize{\hbox{१ं}}}}{{\footnotesize{\hbox{२}}}}$~। $\dfrac{{\footnotesize{\hbox{१ं}}}}{{\footnotesize{\hbox{१०}}}}$~। दृश्यः ६~। जाता गजाः २०~।

\newpage

\textbf{अत्र मूलस्वर्णजातायां सूत्रम्~।}

\phantomsection \label{1.39}
\begin{quote}
\renewcommand{\thefootnote}{१}\footnote{अत्रोपपत्तिः~। प्रश्नालापानुसारेण यदि राशिः $=$ य$^{\scriptsize{\hbox{२}}}$\; तदा
\vspace{2mm}

\hspace{2mm} य$^{\scriptsize{\hbox{२}}} + \dfrac{{\footnotesize{\hbox{य}}^{\scriptsize{\hbox{२}}}}}{{\footnotesize{\hbox{अ}}}} + \dfrac{{\footnotesize{\hbox{य}}^{\scriptsize{\hbox{२}}}}}{{\footnotesize{\hbox{क}}}} + \dfrac{{\footnotesize{\hbox{य}}^{\scriptsize{\hbox{२}}}}}{{\footnotesize{\hbox{ग}}}} \pm$ य $=$ दृ
\vspace{2mm}

\hspace{15mm} $=$ य$^{\scriptsize{\hbox{२}}} \left({\hbox{१}} + \dfrac{{\footnotesize{\hbox{१}}}}{{\footnotesize{\hbox{अ}}}} + \dfrac{{\footnotesize{\hbox{१}}}}{{\footnotesize{\hbox{क}}}} + \dfrac{{\footnotesize{\hbox{१}}}}{{\footnotesize{\hbox{ग}}}}\right) \pm$ य 
\vspace{2mm}

अत्र कोष्ठकान्तर्गता सङ्ख्या रूपेष्टे उद्देशकालापभवं फलं तदेवाचार्येण रूपोत्थं फलं कथ्यते तेन हृतं य-गुणकं रूपं पदसञ्ज्ञं तथा तेनैव हृतमग्रं दृश्यं क्रमेणान्तरवधसञ्ज्ञौ जातौ ततः पूर्वसमीकरणस्य रूपान्तरम्~।
\vspace{3mm}

\hspace{13mm} य$^{\scriptsize{\hbox{२}}} \pm$ अं.य $=$ ब
\vspace{3mm}

ततः \hspace{8mm} य $= \dfrac{{\footnotesize \mp\, {\hbox{अं}} + \sqrt{{\hbox{अं}}^{\scriptsize{\hbox{२}}} + {\hbox{४\,ब}}}}}{{\footnotesize{\hbox{२}}}}$
\vspace{2mm}

क्षयगे मूले \hspace{2mm} य $= \dfrac{{\footnotesize + {\hbox{अं}} + \sqrt{{\hbox{अं}}^{\scriptsize{\hbox{२}}} + {\hbox{४\,ब}}}}}{{\footnotesize{\hbox{२}}}}$
\vspace{2mm}

धने मूले \hspace{4mm} य $= \dfrac{{\footnotesize - {\hbox{अं}} + \sqrt{{\hbox{अं}}^{\scriptsize{\hbox{२}}} + {\hbox{४\,ब}}}}}{{\footnotesize{\hbox{२}}}}$
\vspace{3mm}

\noindent अत उपपन्नम्~।}{\large \textbf{{\color{purple}रूपोत्थहृतपदाग्रे \\
स्यातामन्तरवधौ ततस्ताभ्याम्~।\\ 
प्राग्वद्योगः साध्यः \\
स्यातां सङ्क्रामतो राशी~॥~३९~॥ \\
क्षयगे मूलेऽनल्पं तत्कृती राशिः~।}}}
\end{quote}

\noindent \textbf{अपि च~।}

\phantomsection \label{Ex 1.25}
\begin{quote}
\textbf{{\color{red}सरसि सारससङ्कुल आगतौ \\
दलचतुर्थलवौ सपदौ सखे~।\\
बिसविकाशि रदोन्मितसारसाः \\
प्रविहरन्ति च पूर्वचयं वद~॥~२५~॥}}
\end{quote}

\newpage

न्यासः ~$\dfrac{{\footnotesize{\hbox{१}}}}{{\footnotesize{\hbox{२}}}}$~। $\dfrac{{\footnotesize{\hbox{१}}}}{{\footnotesize{\hbox{४}}}}$~। मू १~। दृ ३२~। जातं सारसवृन्दम् १६~।\\

\noindent \textbf{अपि च~।}

\phantomsection \label{Ex 1.26}
\begin{quote}
\textbf{{\color{red}दुर्योधनप्रधनभूमिषु वीरवार‍-\\
दुर्वारमन्युमभिमन्युमभिप्रयाताः~।\\
सार्धं स्वपञ्चमलवैश्च पदद्वयेन \\
साकं सकण्टकभटा द्विशती कथं स्यात्~॥~२६~॥}}
\end{quote}

न्यासः ~$\dfrac{{\footnotesize{\hbox{१}}}}{{\footnotesize{\hbox{२}}}}$~। $\dfrac{{\footnotesize{\hbox{१}}}}{{\footnotesize{\hbox{५}}}}$~। मू २~। दृ २००~। जाता भटाः १००~।\\

\noindent \textbf{अपि च~।}

\phantomsection \label{Ex 1.27}
\begin{quote}
\textbf{{\color{red}आकर्ण्य ध्वनिमद्रिमूर्ध्नि शिखिनोऽब्दानां स्फुरद्विद्युतां\\
वृन्दार्धत्रिलवौ तदन्तरचतुर्भागैस्त्रिभिः संयुतौ~।\\
अध्यर्धैकपदाधिकौ ननृततुः प्रीत्याप्तवृन्दौ सखे\\
जातं तत्र शतत्रयं प्रवद मे तत्पूर्ववृन्दे कति~॥~२७~॥}}
\end{quote}

न्यासः~। $\dfrac{{\footnotesize{\hbox{१}}}}{{\footnotesize{\hbox{२}}}}$~। $\dfrac{{\footnotesize{\hbox{१}}}}{{\footnotesize{\hbox{३}}}}$~। $\dfrac{{\footnotesize{\hbox{१}}}}{{\footnotesize{\hbox{८}}}}$~। मू $\dfrac{{\footnotesize{\hbox{३}}}}{{\footnotesize{\hbox{२}}}}$~। दृ ३००~। जाताः केकिनः १४४~।\\

\noindent \textbf{अपि च~।}

\phantomsection \label{Ex 1.31}
\begin{quote}
\textbf{{\color{red}लङ्कोद्यानविमोटने प्रचलितो युद्धाय \renewcommand{\thefootnote}{१}\footnote{रक्षो गणस्याष्टांशोऽष्टादशभागश्च तत्रैव गणे मिलित इत्यर्थः~।}रक्षोगणोऽ-\\
ष्टांशाष्टादशभागयुग्विदलितः क्रीडावनीं पावनिः\renewcommand{\thefootnote}{२}\footnote{पावनिः हनुमान्~।}।\\
तेनापि स्वकठोरमूललतया बद्धे मृतिं प्रापिते\\
सप्तघ्ने तु पदे भटौ करतलेनाताडितौ ते कति~॥~३१~॥}}\renewcommand{\thefootnote}{३}\footnote{यन्त्रालयकर्मचारिगणासावधानतया २८, २९, ३० श्लोकत्रयाङ्कितमेकं पत्रं विनष्टम्~।}
\end{quote}

न्यासः ~$\dfrac{{\footnotesize{\hbox{१}}}}{{\footnotesize{\hbox{८}}}}$~। $\dfrac{{\footnotesize{\hbox{१}}}}{{\footnotesize{\hbox{१८}}}}$~। मू १ं४~। दृ २~। जाता भटाः १४४~।

\newpage

\noindent \textbf{अपि च~।}

\phantomsection \label{Ex 1.32}
\begin{quote}
\textbf{{\color{red}साङ्घ्रिस्वपञ्चांशयुताद्गजानां \\
व्रजान्मृगारेर्भयतः पदानि~।\\
यातानि पञ्चाद्रिमिभः स्विभीभिः \\
सरित्तटं ते तिसृभिः कति स्युः~॥~३२~॥}}
\end{quote}

न्यासः~। $\dfrac{{\footnotesize{\hbox{१}}}}{{\footnotesize{\hbox{४}}}}$~। $\dfrac{{\footnotesize{\hbox{१}}}}{{\footnotesize{\hbox{५}}}}$~। मू ५ं~। दृ ४~। जाता गजाः १६~।\\

धनविशेषजातौ न्यासः ~$\dfrac{{\footnotesize{\hbox{१}}}}{{\footnotesize{\hbox{१२}}}}$~। $\dfrac{{\footnotesize{\hbox{१}}}}{{\footnotesize{\hbox{४}}}}$~। $\dfrac{{\footnotesize{\hbox{१}}}}{{\footnotesize{\hbox{३}}}}$~। मू १ं~। दृ ५४~। जातो राशिः ३६~।\\

\noindent \textbf{अपि च~।}

\phantomsection \label{Ex 1.33}
\begin{quote}
\renewcommand{\thefootnote}{१}\footnote{पुष्पेषु = कामः~।}\textbf{{\color{red}पुष्पेषुकेलिभवनं वनिताजनानां \\
वृन्दं त्रिभागनवभागतदन्तरैर्युक्~।\\
तस्मात् पदानि जलकेलिकलापभाञ्जि \\
त्रीणि प्रिय प्रवद ताः प्रियषड्विशिष्टाः~॥~३३~॥}}
\end{quote}

 न्यासः~। $\dfrac{{\footnotesize{\hbox{१}}}}{{\footnotesize{\hbox{३}}}}$~। $\dfrac{{\footnotesize{\hbox{१}}}}{{\footnotesize{\hbox{९}}}}$~। $\dfrac{{\footnotesize{\hbox{२}}}}{{\footnotesize{\hbox{९}}}}$~। मूलम् ३ं~। दृ ६~। जाता वनिताः ९~।\\

ऋणांशविमूलजातौ न्यासः~। $\dfrac{{\footnotesize{\hbox{१ं}}}}{{\footnotesize{\hbox{३}}}}$~। $\dfrac{{\footnotesize{\hbox{१ं}}}}{{\footnotesize{\hbox{४}}}}$~। $\dfrac{{\footnotesize{\hbox{१ं}}}}{{\footnotesize{\hbox{१२}}}}$~। मू १ं~। दृ ६~। जातो राशिः ३६~।\\

\noindent \textbf{अपि च~।}

\phantomsection \label{Ex 1.34.1}
\begin{quote}
\textbf{{\color{red}शम्भौ पङ्कजपुञ्जपञ्चमलवोऽप्यर्धं हरावर्पितं\\
दुर्गायां दशमांशकश्च सदलं मूलं रमापादयोः~।}}
\end{quote}

\newpage

\phantomsection \label{Ex 1.34}
\begin{quote}
\textbf{{\color{red}विघ्नेशे कमलद्वयं च दिनपे युग्मं गुरौ पङ्कजं\\
ब्रूहि त्वं यदि वेत्सि वत्स सकलं राजीवपुञ्जं द्रुतम्~॥~३४~॥}}
\end{quote}

न्यासः ~$\dfrac{{\footnotesize{\hbox{१ं}}}}{{\footnotesize{\hbox{५}}}}$~। $\dfrac{{\footnotesize{\hbox{१ं}}}}{{\footnotesize{\hbox{२}}}}$~। $\dfrac{{\footnotesize{\hbox{१ं}}}}{{\footnotesize{\hbox{१०}}}}$~। मू १$\dfrac{{\footnotesize{\hbox{१}}}}{{\footnotesize{\hbox{२}}}}$~। दृ ५~। जातं राजीवमानम् १००~।\\

\noindent \textbf{अपि च~।}

\phantomsection \label{Ex 1.35}
\begin{quote}
\textbf{{\color{red}शरांशो भृङ्गाणां कुटजकुसुमे शेषचरणः \\
कदम्बे तच्छेषत्रिलव उपविष्टस्तु सरले~।\\
लवङ्ग्यां शेषार्धं\renewcommand{\thefootnote}{$\star$}\footnote{The reading शेषोऽर्धं seems to be a typographical error.} पदशरलवः फुल्लबकुले \\
लवङ्गेषु द्वन्द्वद्वयमलिकुलं मे वद सखे~॥~३५~॥}}
\end{quote}

न्यासः~। $\dfrac{{\footnotesize{\hbox{१ं}}}}{{\footnotesize{\hbox{५}}}}$~। $\dfrac{{\footnotesize{\hbox{१ं}}}}{{\footnotesize{\hbox{४}}}}$~। $\dfrac{{\footnotesize{\hbox{१ं}}}}{{\footnotesize{\hbox{३}}}}$~। $\dfrac{{\footnotesize{\hbox{१ं}}}}{{\footnotesize{\hbox{२}}}}$~। मूलम् \,$\dfrac{{\footnotesize{\hbox{१}}}}{{\footnotesize{\hbox{५}}}}$~। दृ ४~। जातो राशिः २५~।\\
     
\noindent \textbf{अपि च~।}

\phantomsection \label{Ex 1.36}
\begin{quote}
\textbf{{\color{red}कामिन्यो निवसन्ति राजभवने तासां त्रिभागः सखे\\
च्छन्दस्यङ्घ्रिरपि प्रबन्धकरणे द्विघ्नो विशेषः स्वरे~।\\
\renewcommand{\thefootnote}{१}\footnote{{\color{violet}द्विघ्नमूलं मूलार्धसहितम्} इति~।}अध्यर्धे स्वपदे सुगीतिचतुरे तिस्रः कथाकोविदाः\\
\renewcommand{\thefootnote}{२}\footnote{द्वे कञ्चुक्यौ एका महिषी राज्ञी तिस्त्रः कथाकोविदा इति षड् दृश्याः~।}कञ्चुक्यौ महिषी च ताः कति वद प्रौढोऽसि पाट्यां यदि~॥~३६~॥}}
\end{quote}

न्यासः ~$\dfrac{{\footnotesize{\hbox{१ं}}}}{{\footnotesize{\hbox{३}}}}$~। $\dfrac{{\footnotesize{\hbox{१ं}}}}{{\footnotesize{\hbox{४}}}}$~। $\dfrac{{\footnotesize{\hbox{१ं}}}}{{\footnotesize{\hbox{६}}}}$~। मूलम \,$\dfrac{{\footnotesize{\hbox{५ं}}}}{{\footnotesize{\hbox{२}}}}$~। दृ ६~। जाताः कामिन्यः १४४~।
\vspace{1mm}

\begin{center}
\textbf{इति सङ्कीर्णाष्टजातयः~।}
\end{center}
\vspace{2mm}

\noindent \textbf{अथादृश्यजातावुद्देशकः~।}

\phantomsection \label{Ex 1.37}
\begin{quote}
\textbf{{\color{red}लीलाम्बुजेनाम्बुजलोचना च कान्ता स्वकान्तं निजघान मूर्ध्नि~।\\
शीर्णानि पर्णानि नले दलाङ्घ्री पदे पदव्यां पतिते कति स्युः~॥}}
\end{quote}

\newpage

न्यासः~। $\dfrac{{\footnotesize{\hbox{१ं}}}}{{\footnotesize{\hbox{२}}}}$~। $\dfrac{{\footnotesize{\hbox{१ं}}}}{{\footnotesize{\hbox{४}}}}$~। मू २ं~। दृ ०~। जातानि कमलपत्राणि ६४~।\\

\noindent \textbf{निरंशजातावुद्देशकः~।}

\phantomsection \label{Ex 1.38}
\begin{quote}
\renewcommand{\thefootnote}{$\star$}\footnote{
\vspace{-4mm}
\begin{quote}
{\color{red}बालिकासङ्कुले चित्रशालिकां च पदत्रयम्~।\\
मालिकां कुरुते मूलं केलिकां पञ्चकं प्रिये~॥}
\end{quote}

\hspace{40mm} इति पाठान्तरम्~।
\vspace{2mm}
}\textbf{{\color{red}बालिकानां कुले चित्रशालिकायां पदत्रयम्~।\\ 
मालिकां कुरुते मूलं कालिकां पञ्चकं ययौ~॥}}
\end{quote}

न्यासः~। मू ४ं~। दृ ५~। जाता बालिकाः २५~।\\

\noindent \textbf{अन्यत् सूत्रं गीत्यर्धम्~।}

\phantomsection \label{1.40.1}
\begin{quote}
\renewcommand{\thefootnote}{१}\footnote{अत्रोपपत्तिः~। विलोमविधिनान्त्यादुक्तवत् कर्म कर्त्तव्यमिति सुगमा~।
\vspace{1mm}

\hspace{2mm} यथाचार्योक्तोदाहरणे प्रथमं राशौ राशिमूलत्रयोनिते शेषं दृश्यम् $=$ ४~।
\vspace{1mm}

\hspace{2mm} अतः, \hyperref[1.39]{३९} सूत्रेण अन्तरम् $=$ ३, वधः $=$ ४~।
\vspace{1mm}

\hspace{2mm} ततो योगः $= \sqrt{{\hbox{अं}}^{\scriptsize{\hbox{२}}} + {\hbox{४\,ब}}} = \sqrt{{\hbox{९}} + {\hbox{१६}}} =$ ५~।
\vspace{2mm}

\hspace{2mm} सङ्क्रमणतो जातौ राशी ४।१ं~। \hyperref[1.39]{'क्षयगे मूलेऽनल्पम्'} इत्यादि विशेषवाक्येन राशिः $=$ ४$^{\scriptsize{\hbox{२}}} =$ १६~।
\vspace{1mm}

\hspace{2mm} पुनः 'स को राशिर्यः स्वत्रिलवोनः षोडश स्युः' इत्यत्र स्व ~$\dfrac{{\footnotesize{\hbox{१ं}}}}{{\footnotesize{\hbox{३}}}}$~। दृ १६~। विलोमविधिना {\color{violet}'अथ स्वांशाधिकोने तु लवाढ्योनो हरो हरः'} इत्यादिना जातो राशिः २४~। एवं सर्वत्र~।}{\large \textbf{{\color{purple}उक्तनिजविधिवदन्त्यात् \\
शेषविधौ जायते राशिः~।}}}
\end{quote}

\noindent \textbf{शेषमूलजातावुद्देशकः~।}

\phantomsection \label{Ex 1.39.1}
\begin{quote}
\textbf{{\color{red}याते नृपे मृगयुभिर्मृगयार्थमाशु \\
पाशान् प्रसारयति तत्त्रिलवोऽप्यटव्याम्~।}}
\end{quote}

\newpage

\phantomsection \label{Ex 1.39}
\begin{quote}
\textbf{{\color{red}शेषस्य घोरतरकेसरिपीडितानि \\
त्रीणि प्रचक्ष्व सचतुष्कपदानि विद्वन्~॥}}
\end{quote}

न्यासः~। $\dfrac{{\footnotesize{\hbox{१ं}}}}{{\footnotesize{\hbox{३}}}}$~। शेमू ३ं~। दृ ४~। जाता मृगयवः २४~।\\

\textbf{अपि च~।}

\phantomsection \label{Ex 1.40}
\begin{quote}
\renewcommand{\thefootnote}{$\star$}\footnote{अत्राप्यन्तात् कर्म कर्त्तव्यम्~।
\vspace{2mm}
}\textbf{{\color{red}कान्तायाः सुरतप्रसङ्गसमये भिन्ना च मुक्तावली\\
मुक्तानां च पदद्वयं विचरणं शय्यापटस्योपरि~।\\
तच्छेषस्य पदं त्रिभागयुगलेनाढ्यं प्रियेणाहृतं\\
तच्छेषस्य पदं क्षितौ निपतितं सूत्रे द्वयं किं वद~॥}}
\end{quote}

 न्यासः~। स्वमू २$\dfrac{{\footnotesize{\hbox{१ं}}}}{{\footnotesize{\hbox{४}}}}$~। शेमू १$\dfrac{{\footnotesize{\hbox{२}}}}{{\footnotesize{\hbox{३}}}}$~। शेमू १~। दृ २~। 
\vspace{2mm}

जातानि मौक्तिकानि १६~।\\
\vspace{2mm}

\noindent \textbf{सूत्रमार्यार्धम्~।}

\phantomsection \label{1.40}
\begin{quote}
\renewcommand{\thefootnote}{१}\footnote{अत्रोपपत्तिः~। कल्प्यते राशिः $=$ य$^{\scriptsize{\hbox{२}}}$~। आद्याग्रम् $=$ आ~। अन्त्याग्रम् $=$ अं~। तदालापानुसारेण
\vspace{1mm}

\hspace{2mm} या$^{\scriptsize{\hbox{२}}} -$ आ, अत्र कर्मणि कृते रूपोत्थफलं गुणको भवति, अतः
\vspace{1mm}

\hspace{11mm} रूफ (य$^{\scriptsize{\hbox{२}}} -$ आ) $-$ य $=$ अं
\vspace{1mm}

\hspace{2mm} वा ~~रूफ.य$^{\scriptsize{\hbox{२}}} -$ आ.रूफ $-$ य $=$ अं
\vspace{1mm}

\hspace{2mm} समशोधनेन ~~रूफ.य$^{\scriptsize{\hbox{२}}} -$ य $=$ अं + आ.रूफ
\vspace{1mm}

\hspace{2mm} अस्मात् \hyperref[1.39]{३९} सूत्रविधिना राशिज्ञानं सुगमम्~। इति~।}{\large \textbf{{\color{purple}रूपोत्थघ्नाद्याग्रं \\
योज्यान्त्याग्रे विधिः प्राग्वत्~।}}}
\end{quote}

\noindent \textbf{उदाहरणम्~।}

\newpage

\phantomsection \label{Ex 1.41}
\begin{quote}
\renewcommand{\thefootnote}{१}\footnote{अत्राद्याग्रम् = १ = आ~। अन्त्याग्रम् = १ = अ । रूपोत्थं फलम्
$= \dfrac{{\footnotesize{\hbox{१}}}}{{\footnotesize{\hbox{५}}}}$
\vspace{1mm}

\hspace{2mm} तत्र सूत्रानुसारेण अग्रमानम् ~~रूफआ $+$ अ $= \dfrac{{\footnotesize{\hbox{१}}}}{{\footnotesize{\hbox{५}}}} +$ १ $= \dfrac{{\footnotesize{\hbox{६}}}}{{\footnotesize{\hbox{५}}}}$~।
\vspace{1mm}

(ततो \hyperref[1.39]{रूपोत्थहृतपदाग्रे} इत्यादिना)
\vspace{1mm}

\hspace{2mm} अन्तरम् $= \dfrac{{\footnotesize{\hbox{१}}}}{{\footnotesize{\hbox{रूफ}}}} =$ ५, वधः $= \dfrac{{\footnotesize{\hbox{६}}}}{{\footnotesize{\hbox{५}}}} \div$ रूफ $=$ ६~।
\vspace{1mm}

\hspace{2mm} ततो योगः $= \sqrt{{\hbox{अं}}^{\scriptsize{\hbox{२}}} + {\hbox{४\,ब}}} = \sqrt{{\hbox{२५}} + {\hbox{२४}}} = \sqrt{{\hbox{४९}}} =$ ७~।
\vspace{2mm}

जातौ राशी ६।१~। \hyperref[1.39]{क्षयगे मूले}, इति विशेषवाक्येन राशिः $=$ ६$^{\scriptsize{\hbox{२}}} =$ ३६~।
\vspace{2mm}
}\textbf{{\color{red}गणेशं पद्मेन त्रिनयनहरिब्रह्मदिनपान् \\
विलोमैः शेषांशैर्विषयलवपूर्वैश्च कमलाम्~।\\
पदेनापूज्यैकेन च गुरुपदाम्भोजयुगलं \\
सरोजेनाचक्ष्व द्रुतमखिलमम्भोजनिचयम्~॥}}
\end{quote}

न्यासः~। १~। $\dfrac{{\footnotesize{\hbox{१ं}}}}{{\footnotesize{\hbox{५}}}}$~। $\dfrac{{\footnotesize{\hbox{१ं}}}}{{\footnotesize{\hbox{४}}}}$~। $\dfrac{{\footnotesize{\hbox{१ं}}}}{{\footnotesize{\hbox{३}}}}$~। $\dfrac{{\footnotesize{\hbox{१ं}}}}{{\footnotesize{\hbox{२}}}}$~। मू १ं~। दृ १~। 
\vspace{2mm}

जातं पङ्कजमानम् ३६~।\\

\noindent \textbf{सूत्रम्~।}

\phantomsection \label{1.41}
\begin{quote}
{\large \textbf{{\color{purple}गुणकपदे तु पदाग्रे \\
हत्वा गुणकेन पूर्वविधिनात्र\renewcommand{\thefootnote}{$\star$}\footnote{पूर्वविधिना च, इति पाठान्तरम्~।
\vspace{2mm}
}।\\
उत्पन्नं तं राशिं \\
समुद्धरेत्तेन गुणकेन~॥}}}\renewcommand{\thefootnote}{२}\footnote{अत्रोपपत्तिः~। येन गुणेन राशिर्मूलदो भवति स गुणकसञ्ज्ञः~। मूलं यद्गुणं स मूलगुणकोऽत्र ज्ञेयः~।
\vspace{1mm}

\hspace{2mm} ततो राशिः $\dfrac{{\footnotesize{\hbox{य}}^{\scriptsize{\hbox{२}}}}}{{\footnotesize{\hbox{गु}}}}$~। आलापानुसारेण}
\end{quote}

\newpage

\noindent \textbf{गुणमूलजातावुद्देशकः~।}\renewcommand{\thefootnote}{}\footnote{\hspace{6mm} $\dfrac{{\footnotesize{\hbox{य}}^{\scriptsize{\hbox{२}}}}}{{\footnotesize{\hbox{गु}}}} - \dfrac{{\footnotesize{\hbox{अ.य}}^{\scriptsize{\hbox{२}}}}}{{\footnotesize{\hbox{क.गु}}}} \mp \dfrac{{\footnotesize{\hbox{ग}}}}{{\footnotesize{\hbox{घ}}}}\,{\hbox{य}} =$ दृ
\vspace{2mm}

\hspace{4mm} $\therefore {\hbox{य}}^{\scriptsize{\hbox{२}}} - \dfrac{{\footnotesize{\hbox{अ}}}}{{\footnotesize{\hbox{क}}}}\,{\hbox{य}}^{\scriptsize{\hbox{२}}} \mp \dfrac{{\footnotesize{\hbox{ग}}}}{{\footnotesize{\hbox{घ}}}}\,{\hbox{गु.य}} =$ दृ.गु 
\vspace{2mm}

वा ~~य$^{\scriptsize{\hbox{२}}} \left({\hbox{१}} - \dfrac{{\footnotesize{\hbox{अ}}}}{{\footnotesize{\hbox{क}}}}\right) \mp \dfrac{{\footnotesize{\hbox{ग}}}}{{\footnotesize{\hbox{घ}}}}\,{\hbox{गु.य}} =$ दृ.गु 
\vspace{2mm}

अतो पदाग्रस्य दृश्याग्रस्य स्थाने दृ.गु, पदगुणकस्य स्थाने च\, $\dfrac{{\footnotesize{\hbox{ग}}}}{{\footnotesize{\hbox{घ}}}}$गु\; इदं संस्थाप्य पूर्वविधिना राशिः $= {\hbox{य}}^{\scriptsize{\hbox{२}}}$ भवति~। ततोऽभीष्टराशिश्चायं\, $\dfrac{{\footnotesize{\hbox{य}}^{\scriptsize{\hbox{२}}}}}{{\footnotesize{\hbox{गु}}}}$ प्रसिद्धो भविष्यतीति~।
\vspace{2mm}
}

\phantomsection \label{Ex 1.42}
\begin{quote}
\textbf{{\color{red}उद्याने कनकावदातगरुतं हंसं विलोक्याङ्गना-\\
वृन्दे वृन्दगजांशको नगगुणः षड्भागमूलैस्त्रिभिः~।\\
पादोनैः सहितो गतस्तत इतो धर्तुं च तं भीमजा\renewcommand{\thefootnote}{$\star$}\footnote{भीमजा = दमयन्ती~।}\\
मञ्जीरध्वनिमञ्जुलालमगतिर्गत्वागृहीत्ते\renewcommand{\thefootnote}{१}\footnote{The reading -त्वा गृहीतः seems to be a typographical error.} कति~॥}}
\end{quote}

न्यासः~। $\dfrac{{\footnotesize{\hbox{७ं}}}}{{\footnotesize{\hbox{८}}}}$~। गुणकः $\dfrac{{\footnotesize{\hbox{१ं}}}}{{\footnotesize{\hbox{६}}}}$~। मू गु $\dfrac{{\footnotesize{\hbox{१ं१}}}}{{\footnotesize{\hbox{४}}}}$~। दृ १~। 
\vspace{2mm}

जाता अङ्गनाः ९६~।\\

\noindent \textbf{अपि च~।}

\phantomsection \label{Ex 1.43}
\begin{quote}
\textbf{{\color{red}त्रिघ्नस्य यूथस्य पदानि यूथात् \\
गतानि च त्रीणि पयोजखण्डे~।\\
सार्धानि याताः\renewcommand{\thefootnote}{२}\footnote{The reading यातं seems to be a typographical error.} कुटजं वदाशु \\
त्रयस्त्रयस्तेऽप्यलयः कति स्युः~॥}}
\end{quote}

\newpage

न्यासः~। गुणकः ३~। मू गु\, ३$\dfrac{{\footnotesize{\hbox{१}}}}{{\footnotesize{\hbox{२}}}}$~। दृ ६~। 
\vspace{2mm}

जाता अलयः ४८~।\\

\noindent \textbf{सूत्रं सार्धगीतिः}

\phantomsection \label{1.43}
\begin{quote}
\renewcommand{\thefootnote}{१}\footnote{अत्रोपपत्तिः~। कल्यते राशिः = या~। ततः प्रश्नोक्त्या
\vspace{2mm}

\hspace{2mm} या $- \left(\dfrac{{\footnotesize{\hbox{अं.या}}}}{{\footnotesize{\hbox{छे}}}} - {\hbox{ही}}\right)^{\scriptsize{\hbox{२}}} =$ या $- \left(\dfrac{{\footnotesize{\hbox{या}}}}{\dfrac{{\footnotesize{\hbox{छे}}}}{{\footnotesize{\hbox{अं}}}}} - {\hbox{ही}}\right)^{\scriptsize{\hbox{२}}} =$ या $- \left(\dfrac{{\footnotesize{\hbox{या}}}}{{\footnotesize{\hbox{रूह}}}} - {\hbox{ही}}\right)^{\scriptsize{\hbox{२}}}$
\vspace{2mm}

\hspace{6mm} $=$ या $- \left(\dfrac{{\footnotesize{\hbox{या}} - {\hbox{ही.रूह}}}}{{\footnotesize{\hbox{रूह}}}}\right)^{\scriptsize{\hbox{२}}}$
\vspace{2mm}

\hspace{6mm} $= \dfrac{{\footnotesize{\hbox{रूह}}^{\scriptsize{\hbox{२}}}\,{\hbox{या}} - {\hbox{या}}^{\scriptsize{\hbox{२}}} + {\hbox{२\,ही\,रूह\,या}} - {\hbox{ही}}^{\scriptsize{\hbox{२}}}\,{\hbox{रूह}}^{\scriptsize{\hbox{२}}}}}{{\footnotesize{\hbox{रूह}}^{\scriptsize{\hbox{२}}}}} =$ दृ
\vspace{2mm}

\hspace{2mm} ततः\; या$^{\scriptsize{\hbox{२}}} - {\hbox{या}}\,({\hbox{रूह}}^{\scriptsize{\hbox{२}}} + {\hbox{२\,ही\,रूह}}) = - {\hbox{रूह}}^{\scriptsize{\hbox{२}}} ({\hbox{ही}}^{\scriptsize{\hbox{२}}} + {\hbox{दृ}})$
\vspace{2mm}

\hspace{2mm} वा~~~~ या$^{\scriptsize{\hbox{२}}} -$ यो या $= -$ घा
\vspace{2mm}

\hspace{6mm} $\therefore$ या $= \dfrac{{\footnotesize {\hbox{यो}} \pm \sqrt{{\hbox{यो}}^{\scriptsize{\hbox{२}}} - {\hbox{४\,घा}}}}}{{\footnotesize{\hbox{२}}}}$
\vspace{2mm}

\hspace{2mm} अतोऽत्र \hyperref[1.35]{३५} सूत्रस्य क्रियोत्पद्यत इत्युपपन्नं सर्वम्~।}{\large \textbf{{\color{purple}स्वांशकभक्तश्छेदो \\
रूपहराख्यो द्विनिघ्नहीनहतः~॥~४२~॥ \\
रूपहरवर्गयुक्तो \\
योगः स्याद्धीनवर्गयुतदृश्यः~।\\ 
रूपहरवर्गगुणितो \\
घातो राशिर्द्विधा प्राग्वत्~॥~४३~॥}}}
\end{quote}

\noindent \textbf{हीनवर्गजातावुद्देशकः~।}

\phantomsection \label{Ex 1.44}
\begin{quote}
\textbf{{\color{red}मल्लाहवे सदसि मत्स्यमहीपतेश्च \\
षट्कोनिताङ्घ्रिकृतिमङ्घ्रितलेन भीमः~।\\
कक्षाद्वयेन युगलं निजघान मल्लम् \\
एकं निपीडितगलं वद ते कति स्युः~॥}}
\end{quote}

\newpage

न्यासः~। $\dfrac{{\footnotesize{\hbox{१}}}}{{\footnotesize{\hbox{४}}}}$~। ही ६~। व~। दृ ३~।
\vspace{2mm}

जाता मल्लाः ५२ वा १२~।
\vspace{2mm}

अत्र द्वितीयो राशिर्न ग्राह्योऽनुपपन्नत्वात्~।\\

क्वचिद्ग्राह्य एव\textendash 
\vspace{2mm}

\noindent \textbf{उदाहरणम्~।}

\phantomsection \label{Ex 1.45}
\begin{quote}
\textbf{{\color{red}यूथाद्विंशांशकस्यैकवर्जितस्य कृतिः सखे~।\\
प्रयाता मानसं हंसाः खं पञ्चोनशतं कति~॥}}
\end{quote}

न्यासः~। $\dfrac{{\footnotesize{\hbox{१}}}}{{\footnotesize{\hbox{२०}}}}$~। ही १ व~। दृ ९५~।
\vspace{2mm}

जाता हंसाः ३२० वा १२०~।
\vspace{2mm}

अत्र द्वावपि राशी ग्राह्यौ~।\\

\noindent \textbf{अंशवर्गजातावुद्देशकः~।}

\phantomsection \label{Ex 1.46}
\begin{quote}
\textbf{{\color{red}शालालवालजललालसबालहंस-\\
कोलाहलादलिकुलस्य दशांशवर्गः~।\\
गुञ्जन् ययौ सुरभिपुञ्जितजम्बुभूजम् \\
अम्भोजिनीं जिनमिताः कति तेऽलयः स्युः~॥}}
\end{quote}

न्यासः~। $\dfrac{{\footnotesize{\hbox{१}}}}{{\footnotesize{\hbox{१०}}}}$~। ही ०। दृ २४~। जाता अलयः ६० वा २४~।\\
\vspace{2mm}

"अत्र करणम्~। यथा
\vspace{2mm}

प्रथमोदाहरणे न्यासः~। $\dfrac{{\footnotesize{\hbox{१}}}}{{\footnotesize{\hbox{४}}}}$\, स्वांशकभक्तश्छेदः $\dfrac{{\footnotesize{\hbox{४}}}}{{\footnotesize{\hbox{१}}}}$~।
\vspace{-2mm}
  
\begin{quote}
अयं रूपहराख्यः~। हीनः ६~। द्विगुणः १२। अनेन गुणितो रूपहरः ४८~। रूपहरवर्गः १६~। अनेन युतो जातः ६४ (योगः)~। गुणितं जातो घातः ६२४~। एवं जातौ योगघातौ ६४।६२४ आभ्यां सङ्क्रमणविधिना जातौ राशी ५२।१२"
\end{quote}

\newpage

\noindent \textbf{सूत्रम्~।}

\phantomsection \label{1.44.1}
\begin{quote}
\renewcommand{\thefootnote}{१}\footnote{अत्रोपपत्तिः~। कल्प्यते राशिः = या~। ततः प्रश्नोक्त्या
\vspace{2mm}

\hspace{11mm} या $- \dfrac{{\footnotesize{\hbox{प्रअं.द्विअं}}}}{{\footnotesize{\hbox{प्रछे.द्विछे}}}}\, {\hbox{या}}^{\scriptsize{\hbox{२}}} =$ दृ
\vspace{2mm}

\hspace{2mm} ततः~~~ ${\hbox{या}}^{\scriptsize{\hbox{२}}} - \dfrac{{\footnotesize{\hbox{प्रछे.द्विछे}}}}{{\footnotesize{\hbox{प्रअं.द्विअं}}}}\, {\hbox{या}} = - \dfrac{{\footnotesize{\hbox{प्रछे.द्विछे}}}}{{\footnotesize{\hbox{प्रअं.द्विअं}}}}\,$ दृ
\vspace{2mm}

\hspace{2mm} वा~~~ ${\hbox{या}}^{\scriptsize{\hbox{२}}} - \dfrac{{\footnotesize{\hbox{१}}}}{\dfrac{{\footnotesize{\hbox{प्रअं}}}}{{\footnotesize{\hbox{प्रछे}}}}}.\dfrac{{\footnotesize{\hbox{१}}}}{\dfrac{{\footnotesize{\hbox{द्विअं}}}}{{\footnotesize{\hbox{द्विछे}}}}}\, {\hbox{या}} = - \dfrac{{\footnotesize{\hbox{दृ}}}}{\dfrac{{\footnotesize{\hbox{प्रअं}}}}{{\footnotesize{\hbox{प्रछे}}}}.\dfrac{{\footnotesize{\hbox{द्विअं}}}}{{\footnotesize{\hbox{द्विछे}}}}}$
\vspace{2mm}

\hspace{20mm} या$^{\scriptsize{\hbox{२}}} -$ यो या $= -$ घा
\vspace{3mm}

\hspace{2mm} ततः~~~ या $= \dfrac{{\footnotesize {\hbox{यो}} \pm \sqrt{{\hbox{यो}}^{\scriptsize{\hbox{२}}} - {\hbox{४\,घा}}}}}{{\footnotesize{\hbox{२}}}}$
\vspace{2mm}

\hspace{2mm} अत्राचार्येण $\dfrac{{\footnotesize{\hbox{प्रअं}}}}{{\footnotesize{\hbox{प्रछे}}}} =$ प्रथमांशः~। $\dfrac{{\footnotesize{\hbox{द्विअं}}}}{{\footnotesize{\hbox{द्विछे}}}} =$ द्वितीयांशः कल्पित इत्युपपन्नम्~।
\vspace{3mm}
}{\large \textbf{{\color{purple}दृश्येऽंशाभ्यां भक्ते \\
घातो रूपे च योगः स्यात्~।}}}
\end{quote}

\noindent \textbf{भागसङ्गुण्यजातावुद्देशकः~।}

\phantomsection \label{Ex 1.47}
\begin{quote}
\textbf{{\color{red}षडंशकघ्नोऽर्कलवः कपीनाम् \\
अधित्यकायां\renewcommand{\thefootnote}{$\star$}\footnote{The reading अधीत्यकायां seems to be a typographical error.
\vspace{1mm}
} विचरत्यगस्य~।\\
दृष्टा नितम्बे झरवारिकेलि-\\
व्यग्राः सखे षोडश ते कियन्तः~॥}}
\end{quote}

न्यासः\, $\dfrac{{\footnotesize{\hbox{१}}}}{{\footnotesize{\hbox{६}}}}$ भा~। $\dfrac{{\footnotesize{\hbox{१}}}}{{\footnotesize{\hbox{१२}}}}$ भा~। दृ १६~। जाताः कपयः ४८ वा २४~। अत्र द्वावेव राशी ग्राह्यौ~।\\
\vspace{2mm}

\noindent \textbf{सूत्रं गीत्यर्धम्~।}

\phantomsection \label{1.44}
\begin{quote}
\renewcommand{\thefootnote}{२}\footnote{अत्रोपपत्तिः~। कल्प्यते राशिः या~। ततः प्रश्नोक्त्या}{\large \textbf{{\color{purple}रूपे दृश्यांशोनेंऽ-\\
शकयोर्घातेन भाजितो राशिः~॥~४४~॥}}}
\end{quote}

\newpage

\noindent \textbf{भिन्नसंदृश्यजातावुद्देशकः~।}\renewcommand{\thefootnote}{}\footnote{\hspace{2mm} या $- \dfrac{{\footnotesize{\hbox{या}}}}{{\footnotesize{\hbox{प्रअं}}}}.\dfrac{{\footnotesize{\hbox{या}}}}{{\footnotesize{\hbox{द्विअं}}}} = \dfrac{{\footnotesize{\hbox{या}}}}{{\footnotesize{\hbox{दृअं}}}}$~।\; यावत्तावत्तापवर्तिते
\vspace{2mm}

\hspace{5mm} १ $- \dfrac{{\footnotesize{\hbox{१}}}}{{\footnotesize{\hbox{दृअं}}}} = \dfrac{{\footnotesize{\hbox{दृअं}} - {\hbox{१}}}}{{\footnotesize{\hbox{दृअं}}}} = \dfrac{{\footnotesize{\hbox{या}}}}{{\footnotesize{\hbox{प्रअं.द्विअं}}}}$
\vspace{2mm}

\hspace{2mm} $\therefore$ या\: $= \dfrac{{\footnotesize{\hbox{प्रअं.द्विअं}}\,({\hbox{दृअं}} - {\hbox{१}})}}{{\footnotesize{\hbox{दृअं}}}}$
\vspace{2mm}

\hspace{9mm} $= \dfrac{{\footnotesize{\hbox{प्रअं.द्विअं}}\,\left({\hbox{१}} - \dfrac{{\footnotesize{\hbox{१}}}}{{\footnotesize{\hbox{दृअं}}}}\right)}}{{\footnotesize{\hbox{१}}}}$
\vspace{2mm}

\hspace{9mm} $= \dfrac{{\footnotesize{\hbox{१}} - \dfrac{{\footnotesize{\hbox{१}}}}{{\footnotesize{\hbox{दृअं}}}}}}{\dfrac{{\footnotesize{\hbox{१}}}}{{\footnotesize{\hbox{प्रअं}}}}.\dfrac{{\footnotesize{\hbox{१}}}}{{\footnotesize{\hbox{द्विअं}}}}}$ 
\vspace{2mm}

\hspace{2mm} अत्राचार्येणा\textendash \,$\dfrac{{\footnotesize{\hbox{१}}}}{{\footnotesize{\hbox{प्रअं}}}}$\textendash \,स्य दृश्यांशसञ्ज्ञा तथा\textendash \,$\dfrac{{\footnotesize{\hbox{१}}}}{{\footnotesize{\hbox{प्रअं}}}},\,\dfrac{{\footnotesize{\hbox{१}}}}{{\footnotesize{\hbox{द्विअं}}}}$\textendash \,नयारंशसञ्ज्ञे कृते इत्युपपन्नम्~।
\vspace{3mm}
}

\phantomsection \label{Ex 1.48}
\begin{quote}
\renewcommand{\thefootnote}{$\star$}\footnote{शालशालदलनीलकोकिलासङ्कुलस्य कलमीलितोऽलिनाम्~।~~ इति पाठान्तरम्~।}\textbf{{\color{red}शालतालदलनीपकोकिला-\\
सङ्कुलस्य कलभीतितोऽलिनाम्~।\\
षड्लवोऽङ्कलवसङ्गुणोऽब्जिनीं \\
पाटलीं गुणलवोऽगमत् कति~॥}}
\end{quote}

न्यासः~। $\dfrac{{\footnotesize{\hbox{१}}}}{{\footnotesize{\hbox{६}}}}$ भा~। $\dfrac{{\footnotesize{\hbox{१}}}}{{\footnotesize{\hbox{९}}}}$ भा~। दृ $\dfrac{{\footnotesize{\hbox{१}}}}{{\footnotesize{\hbox{३}}}}$~। जाता अलयः ३६~। \\
\vspace{2mm}

\noindent \textbf{परिभाषा~।}

\newpage

\begin{center}
{\large \textbf{अथ कृतौ किञ्चित् कुतूहलमुच्यते~।}}
\end{center}

\noindent \textbf{सूत्रमार्या~।}

\phantomsection \label{1.45}
\begin{quote}
\renewcommand{\thefootnote}{१}\footnote{अत्रोपपत्तिः~। कल्प्येते राशी या, का~। ततः प्रश्नालापेन
\vspace{1mm}

\hspace{3mm} या$^{\scriptsize{\hbox{२}}} \pm$ का$^{\scriptsize{\hbox{२}}} +$ १\; अयं वर्गः~। अतो मूलानयनविधिना यदि
\vspace{1mm}

\hspace{3mm} या$^{\scriptsize{\hbox{२}}} +$ का$^{\scriptsize{\hbox{२}}} + {\hbox{१}} = ({\hbox{या}} \pm {\hbox{१}})^{\scriptsize{\hbox{२}}} =$ या$^{\scriptsize{\hbox{२}}} \pm$ २\,या $+$ १
\vspace{1mm}

\hspace{3mm} ततो यावत्तावन्मानम् $=$ या $= \dfrac{{\footnotesize{\hbox{का}}^{\scriptsize{\hbox{२}}}}}{{\footnotesize{\hbox{२}}}}$~।
\vspace{2mm}

\hspace{3mm} अतः प्रथमः कालकस्तद्वर्गदलसमोऽपरो जायत इत्युपपद्यते~।
\vspace{2mm}
}{\large \textbf{{\color{purple}इष्टः प्रथमो राशिः \\
तद्वर्गदलं प्रजायते चान्यः~।\\
अनयोः कृतियुतिवियुती \\
रूपयुते मूलदे भवतः~॥~४५~॥}}}
\end{quote}

\noindent \textbf{उदाहरणम्~।}

\phantomsection \label{Ex 1.49}
\begin{quote}
\textbf{{\color{red}वर्गयोगवियोगौ च ययो रूपयुतौ कृती~।\\ 
\renewcommand{\thefootnote}{$\star$}\footnote{{\color{red}'तौ वदानेकधा विद्वन् वेत्सि चेत् कृतिकौतुकम्~।'} इति पाठान्तरम्~।}बहुधा तौ वद क्षिप्रं वेत्सि वर्गचमत्कृतिम्~॥}}
\end{quote}

अत्र द्विकेनेष्टेन जातौ राशी २,२~। त्रिकेण ३, $\dfrac{{\footnotesize{\hbox{९}}}}{{\footnotesize{\hbox{२}}}}$~।
\vspace{2mm}

चतुष्केण ४, ८~। सत्र्यंशद्वयेन $\dfrac{{\footnotesize{\hbox{७}}}}{{\footnotesize{\hbox{३}}}}, \dfrac{{\footnotesize{\hbox{४९}}}}{{\footnotesize{\hbox{१८}}}}$~।
\vspace{2mm}

सार्धद्वयेन $\dfrac{{\footnotesize{\hbox{५}}}}{{\footnotesize{\hbox{२}}}}, \dfrac{{\footnotesize{\hbox{२५}}}}{{\footnotesize{\hbox{८}}}}$~। एवमिष्टवशादानन्त्यम्~।
\vspace{2mm}

\renewcommand{\thefootnote}{$\dag$}\footnote{अन्यथा यावत्तावन्महान् राशिरेव लघुर्भवति~।}अत्रेष्टं रूपद्वयादूनं न प्रकल्पयेत्~।

\newpage

\noindent \textbf{सूत्रमार्या~।}

\phantomsection \label{1.46}
\begin{quote}
\renewcommand{\thefootnote}{१}\footnote{अत्रोपपत्तिः~। इष्टस्य वर्गवर्गो घनश्चेत्यादि भास्करसूत्रे
यदीष्टमानम् $= \dfrac{{\footnotesize{\hbox{इ}}}}{{\footnotesize{\hbox{२}}}}$\, तदा राशी, ८\,$\left(\dfrac{{\footnotesize{\hbox{इ}}}}{{\footnotesize{\hbox{२}}}}\right)^{\scriptsize{\hbox{४}}} +$ १, ८\,$\left(\dfrac{{\footnotesize{\hbox{इ}}}}{{\footnotesize{\hbox{२}}}}\right)^{\scriptsize{\hbox{३}}}$ वा\, $\dfrac{{\footnotesize{\hbox{इ}}^{\scriptsize{\hbox{४}}}}}{{\footnotesize{\hbox{२}}}} +$ १,\, इ$^{\scriptsize{\hbox{३}}}$~।
\vspace{2mm}

\hspace{2mm} अत उपपद्यते सूत्रम्~।
\vspace{2mm}
}{\large \textbf{{\color{purple}आद्योऽभीष्टघनः स्यात् \\
कृतिकृतिदलमेकयुग् भवेदन्यः~।\\
अनयोः कृतियुतिवियुती \\
रूपोने मूलदे स्याताम्~॥~४६~॥}}}
\end{quote}

\noindent \textbf{उदाहरणम्~।}

\phantomsection \label{Ex 1.50}
\begin{quote}
\textbf{{\color{red}वर्गयोगवियोगौ च ययो रूपोनितौ कृती~।\\
बहुधा तौ वद क्षिप्रं वेत्सि चेत् कृतिकौतुकम्~॥}}
\end{quote}

अत्रैकेष्टेन जातौ राशी १, $\dfrac{{\footnotesize{\hbox{३}}}}{{\footnotesize{\hbox{२}}}}$~। द्वाभ्यां जातौ ८, ९~। त्रिभिः
२७, $\dfrac{{\footnotesize{\hbox{८३}}}}{{\footnotesize{\hbox{२}}}}$~।\\

अर्धेन $\dfrac{{\footnotesize{\hbox{१}}}}{{\footnotesize{\hbox{८}}}}, \dfrac{{\footnotesize{\hbox{३३}}}}{{\footnotesize{\hbox{३२}}}}$~। त्र्यंशेन $\dfrac{{\footnotesize{\hbox{१}}}}{{\footnotesize{\hbox{२७}}}}, \dfrac{{\footnotesize{\hbox{८२}}}}{{\footnotesize{\hbox{८१}}}}$~। एवमिष्टवशादनेकधा~।\\

\noindent \textbf{सूत्रम्~।}

\phantomsection \label{1.47}
\begin{quote}
\renewcommand{\thefootnote}{२}\footnote{अत्रोपपत्तिः~। कल्प्येते राशी\, २\,(या$^{\scriptsize{\hbox{२}}} +$ का$^{\scriptsize{\hbox{२}}}$), २\,(या$^{\scriptsize{\hbox{२}}} -$ का$^{\scriptsize{\hbox{२}}}$)~। अत्रालापद्वयं घटत एव~। अनयोर्घातः~। सैकः~ $=$\, ४\,या$^{\scriptsize{\hbox{४}}} -$ ४\,का$^{\scriptsize{\hbox{४}}} +$ १~ अयं वर्गः~। ~~अतो मूलानयनविधिना~~ २ $\times$ २\,या$^{\scriptsize{\hbox{२}}} \times$ १ $=$ ४\,या$^{\scriptsize{\hbox{२}}} =$ ४\,का$^{\scriptsize{\hbox{२}}}$~। $\therefore$ या $=$ का$^{\scriptsize{\hbox{२}}}$~। उत्थापनेन जातौ राशी\; २\,(का$^{\scriptsize{\hbox{४}}} +$ का$^{\scriptsize{\hbox{२}}}$), २\,(का$^{\scriptsize{\hbox{४}}} -$ का$^{\scriptsize{\hbox{२}}}$)~। अत उपपन्नं सूत्रम्~।}{\large \textbf{{\color{purple}इष्टवर्गकृतिर्द्विष्ठा वर्गोनाढ्या द्विसङ्गुणा~।\\ तयोर्योगान्तरे वर्गो घाते रूपयुते भवेत्~॥~४७~॥}}}
\end{quote}

\newpage

\noindent \textbf{उदाहरणम्~।}

\phantomsection \label{Ex 1.51}
\begin{quote}
\textbf{{\color{red}ययोर्योगे वियोगे च वर्गो घाते सरूपके~।\\ 
तौ वदास्ति तवालं चेदभ्यासः कृतिकौतुके~॥}}
\end{quote}

द्विकेनेष्टेन जातौ राशी ४०,२५~। त्रिकेण १८०,१४४~। सार्धेनैकेन $\dfrac{{\footnotesize{\hbox{११७}}}}{{\footnotesize{\hbox{८}}}}, \dfrac{{\footnotesize{\hbox{४५}}}}{{\footnotesize{\hbox{८}}}}$~। 
\vspace{2mm}

एवमिष्टवशादनेकधा~।\\

\noindent \textbf{सूत्रमार्या~।}

\phantomsection \label{1.48}
\begin{quote}
\renewcommand{\thefootnote}{१}\footnote{अत्रोपपत्तिः स्फुटैव यतः\; या$^{\scriptsize{\hbox{२}}} +$ का$^{\scriptsize{\hbox{२}}} + {\hbox{२\,या.का}} = ({\hbox{या}} + {\hbox{का}})^{\scriptsize{\hbox{२}}}$~।
\vspace{1mm}

\hspace{2mm} तथा\; या$^{\scriptsize{\hbox{२}}} +$ का$^{\scriptsize{\hbox{२}}} - {\hbox{२\,या.का}} = ({\hbox{या}} - {\hbox{का}})^{\scriptsize{\hbox{२}}}$~।
\vspace{1mm}

\hspace{2mm} अतः प्रथमो राशिः $=$ या$^{\scriptsize{\hbox{२}}} +$ का$^{\scriptsize{\hbox{२}}}$~। द्वितीयश्च $=$ २\,या.का~।
\vspace{1mm}

\hspace{2mm} अत उपपद्यते~।
\vspace{2mm}
}{\large \textbf{{\color{purple}वर्गयुतिः प्रथमः स्यात् \\
अभीष्टयोराहतिर्द्विगुणितान्यः~।\\
संयोगे च वियोगे \\
पृथक्तयोर्जायते वर्गः~॥~४८~॥}}}
\end{quote}

\noindent \textbf{उदाहरणम्~।}

\phantomsection \label{Ex 1.52}
\begin{quote}
\textbf{{\color{red}ययोर्योगे वियोगे च सखे वर्गः प्रजायते~।\\ 
तौ कौ वद त्वयात्यर्थं यदि वर्गे कृतः श्रमः~॥}}
\end{quote}

इष्टे १,२ आभ्यां जातौ राशी ५,४~। अथवेष्टे २,३ आभ्यां जातौ राशी १२,१३~। वेष्टे १,३ अाभ्यां जातौ राशी ६.१०~। एवमिष्टवशादनेकधा~।\\

\noindent \textbf{सूत्रमार्या~।}

\phantomsection \label{1.49.1}
\begin{quote}
\renewcommand{\thefootnote}{२}\footnote{अत्रोपपत्तिः~। पूर्वसूत्रेण यदि राशी\; या$^{\scriptsize{\hbox{२}}} +$ का$^{\scriptsize{\hbox{२}}}$, २\,या.का~। 
\vspace{1mm}

केनापीष्टवर्गेण नीलकवर्गेण गुणौ\; नी$^{\scriptsize{\hbox{२}}}$\,(या$^{\scriptsize{\hbox{२}}} +$ का$^{\scriptsize{\hbox{२}}}$),}{\large \textbf{{\color{purple}प्रागुक्तौ यौ च तयोः \\
वधकृतिभक्तेष्टघनकृतिहतौ तौ~।}}}
\end{quote}

\newpage

\phantomsection \label{1.49}
\begin{quote}
{\large \textbf{{\color{purple}राश्योर्योगे विवरे \\
वर्गो घाते घनो भवति~॥~४९~॥}}}\renewcommand{\thefootnote}{}\footnote{२\,या.का.नी$^{\scriptsize{\hbox{२}}}$\, कल्प्येते तदाप्यालापद्वयं घटते~। अथानयोर्घातः $=$ नी$^{\scriptsize{\hbox{२}}} \times$ २\,या.का (या$^{\scriptsize{\hbox{२}}} +$ का$^{\scriptsize{\hbox{२}}}$)~। 
\vspace{1mm}

\hspace{2mm} अयं घनोऽतो यदि
\vspace{1mm}

\hspace{8mm} नी$^{\scriptsize{\hbox{२}}} = \dfrac{{\footnotesize({\hbox{इ}}^{\scriptsize{\hbox{३}}})^{\scriptsize{\hbox{२}}}}}{{\footnotesize({\hbox{२\,या.का}})^{\scriptsize{\hbox{२}}}.({\hbox{या}}^{\scriptsize{\hbox{२}}} + {\hbox{का}}^{\scriptsize{\hbox{२}}})^{\scriptsize{\hbox{२}}}}}$
\vspace{2mm}

\hspace{2mm} तदा\; नी$^{\scriptsize{\hbox{४}}} = \dfrac{{\footnotesize({\hbox{इ}}^{\scriptsize{\hbox{३}}})^{\scriptsize{\hbox{४}}}}}{{\footnotesize({\hbox{२\,या.का}})^{\scriptsize{\hbox{४}}}.({\hbox{या}}^{\scriptsize{\hbox{२}}} + {\hbox{का}}^{\scriptsize{\hbox{२}}})^{\scriptsize{\hbox{४}}}}}$
\vspace{2mm}

अतो~~~ घातः $= \dfrac{{\footnotesize({\hbox{इ}}^{\scriptsize{\hbox{३}}})^{\scriptsize{\hbox{४}}}}}{{\footnotesize({\hbox{२\,या.का}})^{\scriptsize{\hbox{३}}}.({\hbox{या}}^{\scriptsize{\hbox{२}}} + {\hbox{का}}^{\scriptsize{\hbox{२}}})^{\scriptsize{\hbox{३}}}}}$\; अयं घनो भवत्येव यतोऽयम् 
\vspace{2mm}

\hspace{14mm} $= \dfrac{{\footnotesize{\hbox{इ}}^{\scriptsize{\hbox{१२}}}}}{{\footnotesize[{\hbox{२\,या.का}}.({\hbox{या}}^{\scriptsize{\hbox{२}}} + {\hbox{का}}^{\scriptsize{\hbox{२}}})]^{\scriptsize{\hbox{३}}}}}$
\vspace{2mm}

\hspace{14mm} $= \left[\dfrac{{\footnotesize{\hbox{इ}}^{\scriptsize{\hbox{४}}}}}{{\footnotesize{\hbox{२\,या.का}}.({\hbox{या}}^{\scriptsize{\hbox{२}}} + {\hbox{का}}^{\scriptsize{\hbox{२}}})}}\right]^{\scriptsize{\hbox{३}}}$
\vspace{2mm}

अथ~~ नी$^{\scriptsize{\hbox{२}}} = \dfrac{{\footnotesize({\hbox{इ}}^{\scriptsize{\hbox{३}}})^{\scriptsize{\hbox{२}}}}}{{\footnotesize({\hbox{२\,या.का}})^{\scriptsize{\hbox{२}}}.({\hbox{या}}^{\scriptsize{\hbox{२}}} + {\hbox{का}}^{\scriptsize{\hbox{२}}})^{\scriptsize{\hbox{२}}}}}$
\vspace{2mm}

\hspace{12mm} $= \dfrac{{\footnotesize({\hbox{इ}}^{\scriptsize{\hbox{३}}})^{\scriptsize{\hbox{२}}}}}{{\footnotesize[{\hbox{२\,या.का}}.({\hbox{या}}^{\scriptsize{\hbox{२}}} + {\hbox{का}}^{\scriptsize{\hbox{२}}})]^{\scriptsize{\hbox{२}}}}}$
\vspace{3mm}

\hspace{2mm} अत एतद्गुणा पूर्वोक्तौ राशी कल्प्येतां तदालापत्रयं घटत इत्युपपद्यते सर्वम्~।}
\end{quote}

\noindent \textbf{उदाहरणम्~।}

\phantomsection \label{Ex 1.53}
\begin{quote}
\textbf{{\color{red}राश्योर्योगे वियोगे च वर्गो घाते घनो भवेत्~।\\ 
सखे यदि विजानासि वद तौ त्वरितं मम~॥}}
\end{quote}

अत्र प्राग्वत् राशी ४, ५~। आभ्यामिष्टदशकघनेन जातौ राशी १००००,१२५००~। पञ्च-घनेन\, $\dfrac{{\footnotesize{\hbox{६२५}}}}{{\footnotesize{\hbox{४}}}}, \dfrac{{\footnotesize{\hbox{३१२५}}}}{{\footnotesize{\hbox{४}}}}$~। अथवा राशी १२,१३~।

आभ्यामिष्टदशकघनेन\, $\dfrac{{\footnotesize{\hbox{२५००००}}}}{{\footnotesize{\hbox{५०७}}}}, \dfrac{{\footnotesize{\hbox{६२५००}}}}{{\footnotesize{\hbox{११७}}}}$~। पञ्चकघनेन\, $\dfrac{{\footnotesize{\hbox{१५६२५}}}}{{\footnotesize{\hbox{२०२८}}}}$,

\newpage

\noindent $\dfrac{{\footnotesize{\hbox{१५६२५}}}}{{\footnotesize{\hbox{१८७२}}}}$~। एवमिष्टवशादनेकधा~।\\
\vspace{2mm}

\noindent \textbf{सूत्रमार्या~।}

\phantomsection \label{1.50}
\begin{quote}
\renewcommand{\thefootnote}{१}\footnote{अत्रोपपत्तिः~। कल्प्येते राशी\, $\dfrac{{\footnotesize{\hbox{इ}}^{\scriptsize{\hbox{६}}}}}{{\footnotesize{\hbox{का}}^{\scriptsize{\hbox{२}}}}}$, $\dfrac{{\footnotesize{\hbox{या.इ}}^{\scriptsize{\hbox{६}}}}}{{\footnotesize{\hbox{का}}^{\scriptsize{\hbox{२}}}}}$~।
\vspace{2mm}

अनयोर्वर्गयुतिः\, $= {\hbox{इ}}^{\scriptsize{\hbox{१२}}}\,\left(\dfrac{{\footnotesize{\hbox{१}} + {\hbox{या}}^{\scriptsize{\hbox{२}}}}}{{\footnotesize{\hbox{का}}^{\scriptsize{\hbox{४}}}}}\right)$~। अत्र प्रथमखण्डोऽयं\, ${\hbox{इ}}^{\scriptsize{\hbox{१२}}}$\, घनो भवत्येवातो यदि\, $\dfrac{{\footnotesize{\hbox{१}} + {\hbox{या}}^{\scriptsize{\hbox{२}}}}}{{\footnotesize{\hbox{का}}^{\scriptsize{\hbox{२}}}}}$\, अयं घनस्तदालापो घटते~।
कल्प्यते
\vspace{2mm}

$\dfrac{{\footnotesize{\hbox{१}} + {\hbox{या}}^{\scriptsize{\hbox{२}}}}}{{\footnotesize{\hbox{का}}^{\scriptsize{\hbox{४}}}}} = \dfrac{{\footnotesize{\hbox{१}}}}{{\footnotesize{\hbox{का}}^{\scriptsize{\hbox{३}}}}} \hspace{4mm} \therefore \, {\hbox{१}} + {\hbox{या}}^{\scriptsize{\hbox{२}}} =$ का
\vspace{2mm}

$\therefore \, {\hbox{या}}^{\scriptsize{\hbox{२}}} =$ का $-$ १ \hspace{4mm} ततः\; या = $\sqrt{{\hbox{का}} - {\hbox{१}}}$
\vspace{2mm}

ततो राशी\, $\dfrac{{\footnotesize{\hbox{इ}}^{\scriptsize{\hbox{६}}}}}{{\footnotesize{\hbox{का}}^{\scriptsize{\hbox{२}}}}}$, $\dfrac{{\footnotesize{\hbox{इ}}^{\scriptsize{\hbox{६}}} \sqrt{{\hbox{का}} - {\hbox{१}}}}}{{\footnotesize{\hbox{का}}^{\scriptsize{\hbox{२}}}}}$~। अनयोर्घनयुतिः
\vspace{2mm}

\hspace{4mm} $= \dfrac{{\footnotesize{\hbox{इ}}^{\scriptsize{\hbox{१८}}}}}{{\footnotesize{\hbox{का}}^{\scriptsize{\hbox{६}}}}}\,\left[{\hbox{१}} + ({\hbox{का}} - {\hbox{१}})^{\frac{{\scriptsize{\hbox{३}}}}{{\scriptsize{\hbox{२}}}}}\right]$~। अत्र यदि १ $+ ({\hbox{का}} - {\hbox{१}})^{\frac{{\scriptsize{\hbox{३}}}}{{\scriptsize{\hbox{२}}}}}$ 
\vspace{2mm}

अयं वर्गो भवेत्तदा द्वितीयालापश्च घटते~। अथ यदि का = ५, तदा
\vspace{1mm}

या $= \sqrt{{\hbox{का}} - {\hbox{१}}} =$ २, तथा १ + $({\hbox{का}} - {\hbox{१}})^{\frac{{\scriptsize{\hbox{३}}}}{{\scriptsize{\hbox{२}}}}} =$ १ $+$ ८ $=$ ९ अयं स्वत एव वर्गो भवत्यतस्तदुत्थापनेन जातौ राशी\, $\dfrac{{\footnotesize{\hbox{इ}}^{\scriptsize{\hbox{६}}}}}{{\footnotesize{\hbox{२५}}}}$, $\dfrac{{\footnotesize{\hbox{२\,इ}}^{\scriptsize{\hbox{६}}}}}{{\footnotesize{\hbox{२५}}}}$~।
\vspace{2mm}

अत उपपद्यते~।}{\large \textbf{{\color{purple}इष्टघनवर्ग एको \\
द्विघ्नोऽन्यः पञ्चकृतिहृतौ राशी~।\\
वर्गयुतौ च घनः स्यात् \\
तयोर्भवेद्घनयुतौ वर्गः~॥~५०~॥}}}
\end{quote}

\newpage

\noindent \textbf{उदाहरणम्~।}

\phantomsection \label{Ex 1.54}
\begin{quote}
\textbf{{\color{red}घनयोगे कयोर्वर्गो वर्गयोगे घनो भवेत्~।\\ 
तौ वदाशु सखे वर्गकौतुके कुशलोऽसि चेत्~॥}}
\end{quote}

एकेनेष्टेन जातौ राशी $\dfrac{{\footnotesize{\hbox{१}}}}{{\footnotesize{\hbox{२५}}}}$, $\dfrac{{\footnotesize{\hbox{२}}}}{{\footnotesize{\hbox{२५}}}}$~। द्विकेन $\dfrac{{\footnotesize{\hbox{६४}}}}{{\footnotesize{\hbox{२५}}}}$, $\dfrac{{\footnotesize{\hbox{१२८}}}}{{\footnotesize{\hbox{२५}}}}$~।\\

पञ्चकेन ६२५, १२५०~। अर्धेन $\dfrac{{\footnotesize{\hbox{१}}}}{{\footnotesize{\hbox{१६००}}}}$, $\dfrac{{\footnotesize{\hbox{१}}}}{{\footnotesize{\hbox{८००}}}}$~।\\

त्र्यंशेन $\dfrac{{\footnotesize{\hbox{१}}}}{{\footnotesize{\hbox{१८२२५}}}}$, $\dfrac{{\footnotesize{\hbox{२}}}}{{\footnotesize{\hbox{१८२२५}}}}$~। एवमिष्टवशादनेकधा~।\\
\vspace{2mm}

\noindent \textbf{सूत्रमार्या~।}

\phantomsection \label{1.51}
\begin{quote}
\renewcommand{\thefootnote}{१}\footnote{अत्रोपपत्तिः~। कल्प्यन्ते क्रमेण गुणकौ $=$ अ, क, १ $=$ क्षे,
\vspace{1mm}

\hspace{2mm} राशिः $=$ य, मूले $=$ र, ल~। अतः समीकरणद्वयम्
\vspace{1mm}

\hspace{2mm} अय $+$ क्षे $= {\hbox{र}}^{\scriptsize{\hbox{२}}}$ ..... (१)~।~~ क.य $+$ क्षे $= {\hbox{ल}}^{\scriptsize{\hbox{२}}}$ ..... (२)
\vspace{1mm}

\hspace{2mm} अनयोरन्तरेण~~ (अ $-$ क) य $=$ (${\hbox{र}}^{\scriptsize{\hbox{२}}} - {\hbox{ल}}^{\scriptsize{\hbox{२}}}$) $=$ (र $-$ ल) (र $+$ ल)
\vspace{1mm}

\hspace{2mm} अत्र~~ र $-$ ल $=$ इ (अ $-$ क)~~ प्रकल्पनेन, र $+$ ल $= \dfrac{{\footnotesize{\hbox{य}}}}{{\footnotesize{\hbox{इ}}}}$~।
\vspace{2mm}

\hspace{2mm} आभ्यां सङ्क्रमणेन~~ र $= \dfrac{{\footnotesize{\hbox{१}}}}{{\footnotesize{\hbox{२}}}} \left[ \dfrac{{\footnotesize{\hbox{य}}}}{{\footnotesize{\hbox{इ}}}} + {\hbox{इ}}\,({\hbox{अ}} - {\hbox{क}}) \right] = \dfrac{{\footnotesize{\hbox{य}} + {\hbox{इ}}^{\scriptsize{\hbox{२}}}\,({\hbox{अ}}- {\hbox{क}})}}{{\footnotesize{\hbox{२\,इ}}}}$
\vspace{2mm}
  
\hspace{2mm} वर्गकरणेन~~ $\dfrac{{\footnotesize{\hbox{य}}^{\scriptsize{\hbox{२}}} + {\hbox{२\,इ}}^{\scriptsize{\hbox{२}}}.{\hbox{य}}\,({\hbox{अ}}- {\hbox{क}}) - {\hbox{इ}}^{\scriptsize{\hbox{२}}}.({\hbox{अ}}- {\hbox{क}})^{\scriptsize{\hbox{२}}}}}{{\footnotesize{\hbox{४\,इ}}^{\scriptsize{\hbox{२}}}}} =$ अय $+$
क्षे~।
\vspace{2mm}

द्रष्टव्यं (१) समीकरणम्~।}{\large \textbf{{\color{purple}गुणितो राशिर्याभ्यां \\
द्विष्ठो रूपान्वितो भवेद्वर्गः~।\\ 
तद्युतिरष्टविगुणिता \\
विवरकृतिविभाजिता राशिः~॥~५१~॥}}}
\end{quote}

\newpage

\noindent \textbf{उदाहरणम्~।}\renewcommand{\thefootnote}{}\footnote{छेदगमेन~~ ${\hbox{य}}^{\scriptsize{\hbox{२}}} + {\hbox{२\,इ}}^{\scriptsize{\hbox{२}}}$\,य\,(अ $-$ क) $+ {\hbox{इ}}^{\scriptsize{\hbox{४}}}$\,(अ $-$ क)$^{\scriptsize{\hbox{२}}} =$ ४\,अ.इ$^{\scriptsize{\hbox{२}}}$\,य $+ {\hbox{४\,इ}}^{\scriptsize{\hbox{२}}}$ क्षे~।
\vspace{2mm}

\hspace{2mm} समशोधनेन~~ ${\hbox{य}}^{\scriptsize{\hbox{२}}} - {\hbox{२\,इ}}^{\scriptsize{\hbox{२}}}$\,य\,(अ $+$ क) $= {\hbox{इ}}^{\scriptsize{\hbox{२}}}$ क्षे $- {\hbox{इ}}^{\scriptsize{\hbox{४}}}$\,(अ $-$ क)$^{\scriptsize{\hbox{२}}}$~।
\vspace{2mm}

\hspace{2mm} वर्गपूरणेन~~ ${\hbox{य}}^{\scriptsize{\hbox{२}}} - {\hbox{२\,इ}}^{\scriptsize{\hbox{२}}}$\,य\,(अ $+$ क) $+ {\hbox{इ}}^{\scriptsize{\hbox{४}}}$\,(अ $+$ क)$^{\scriptsize{\hbox{२}}} = {\hbox{४\,इ}}^{\scriptsize{\hbox{२}}}$\,(अक${\hbox{इ}}^{\scriptsize{\hbox{२}}} +$ क्षे)~।
\vspace{1mm}

\hspace{2mm} मूलग्रहणेन~~ य $- {\hbox{इ}}^{\scriptsize{\hbox{२}}}$\,(अ $+$ क) $= \pm$ २\,इ $\sqrt{{\hbox{अक.इ}}^{\scriptsize{\hbox{२}}} + {\hbox{क्षे}}}$
\vspace{1mm}

\footnotemark \hyperref[f1]{*}\,अतः~~ य $= {\hbox{इ}}^{\scriptsize{\hbox{२}}}$\,(अ $+$ क) $\pm$ २\,इ $\sqrt{{\hbox{अक.इ}}^{\scriptsize{\hbox{२}}} + {\hbox{क्षे}}}$
\vspace{1mm}

\hspace{2mm} प्रकृते क्षे $=$ १ तेन परमाल्पं राशिमानं शून्यं भवितुमर्हति तच्च तदैव सम्पद्यते
\vspace{1mm}

\hspace{2mm} यदा\; ${\hbox{इ}}^{\scriptsize{\hbox{२}}}$\,(अ $+$ क) $=$ २\,इ $\sqrt{{\hbox{अक.इ}}^{\scriptsize{\hbox{२}}} + {\hbox{१}}}$\, स्यात्~। तथा च वर्गकरणेन
\vspace{1mm}

\hspace{6mm} ${\hbox{इ}}^{\scriptsize{\hbox{४}}}$\,(अ $+$ क)$^{\scriptsize{\hbox{२}}} = {\hbox{४\,अक.इ}}^{\scriptsize{\hbox{४}}} + {\hbox{४\,इ}}^{\scriptsize{\hbox{२}}}$~।
\vspace{2mm}

\hspace{2mm} समशोधनेन~~ ${\hbox{इ}}^{\scriptsize{\hbox{४}}}$\,(अ $-$ क)$^{\scriptsize{\hbox{२}}} = {\hbox{४\,इ}}^{\scriptsize{\hbox{२}}}$
\vspace{1mm}

\hspace{2mm} अतः~~ ${\hbox{इ}}^{\scriptsize{\hbox{२}}} = \dfrac{{\footnotesize{\hbox{४}}}}{{\footnotesize({\hbox{अ}} - {\hbox{क}})^{\scriptsize{\hbox{२}}}}}$~। तदा च शून्याधिकं परमाल्पराशिमानम् $= {\hbox{२\,इ}}^{\scriptsize{\hbox{२}}}$\,(अ $+$ क)
\vspace{2mm}

\hspace{2mm} उत्थापनेन तादृशराशिमानम् $= \dfrac{{\footnotesize{{\hbox{८}}\,({\hbox{अ}} + {\hbox{क}})}}}{{\footnotesize({\hbox{अ}} - {\hbox{क}})^{\scriptsize{\hbox{२}}}}}$
\vspace{2mm}

इत्युपपन्नम्~।

\noindent\rule{5cm}{0.5pt}
\vspace{1mm}
}
\label{f1} \footnotetext{* कृतिप्रकृत्या निर्दिष्टक्षेपे ज्येष्ठकनिष्ठके~।

\hspace{4mm} साध्ये गुणकयोर्घातं प्रकल्प्य प्रकृतिं ततः~॥

\hspace{4mm} द्विगुणज्येष्ठगुणितकनिष्ठेन युतोनितः~।

\hspace{4mm} गुणयोगहतो ह्रस्ववर्गो राशिर्भवेदिह~॥~~ इति प्रकारान्तरम्~।}

\phantomsection \label{Ex 1.55}
\begin{quote}
\textbf{{\color{red}पञ्चभिस्त्रिभिरभ्यस्तः को राशिः पृथगेकयुग्~।\\ 
मूलदो जायते तं मे वद कोविद सत्वरम्~॥}}
\end{quote}

गुणौ ५,३~। जातो राशिः १६~।

\newpage

\noindent \textbf{उदाहरणम्~।}

\phantomsection \label{Ex 1.56}
\begin{quote}
\textbf{{\color{red}को राशिर्निगमैः शैलैः पृथक् क्षुण्णस्त्रिवर्जितः~।\\ 
मूलदो जायते ब्रूहि कृतिकौतुककोविद~॥}}
\end{quote}

उक्तवज्जातो राशिः १~। २१ वा १०५७~।\\

\noindent \textbf{सूत्रमार्या~।}

\phantomsection \label{1.52}
\begin{quote}
\renewcommand{\thefootnote}{१}\footnote{अत्रोपपत्तिः~। कल्प्यते राशिः $=$ या~। ततः प्रश्नोक्त्या
\vspace{2mm}

\hspace{4mm} $\begin{matrix}
\mbox{{या $+$ क्षे$_{\scriptsize{\hbox{१}}} =$ का$^{\scriptsize{\hbox{२}}}$}}\\
\vspace{-1mm}
\mbox{{या $-$ क्षे$_{\scriptsize{\hbox{२}}} =$ नी$^{\scriptsize{\hbox{२}}}$}}
\vspace{1mm}
\end{matrix}\; \Rightarrow $ ~अनयोरन्तरेण
\vspace{2mm}

\hspace{2mm} क्षे$_{\scriptsize{\hbox{१}}} +$ क्षे$_{\scriptsize{\hbox{२}}} =$ का$^{\scriptsize{\hbox{२}}} -$ नी$^{\scriptsize{\hbox{२}}} =$ (का $+$ नी)\;(का $-$ नी)~।\; अत्र यदि\; का $-$ नी $=$ इ\; तदा
\vspace{2mm}

\hspace{2mm} $\dfrac{{\footnotesize{\hbox{क्षे}}_{\scriptsize{\hbox{१}}} + {\hbox{क्षे}}_{\scriptsize{\hbox{२}}}}}{{\footnotesize{\hbox{इ}}}} =$ का $+$ नी $=$ फ
\vspace{1mm}

\hspace{19mm} का $-$ नी $=$ इ
\vspace{3mm}

\hspace{15mm} $\therefore$\; नी $= \dfrac{{\footnotesize{\hbox{फ}} - {\hbox{इ}}}}{{\footnotesize{\hbox{२}}}}$
\vspace{2mm}

\hspace{2mm} ततः~~ या $- {\hbox{क्षे}}_{\scriptsize{\hbox{२}}} = {\hbox{नी}}^{\scriptsize{\hbox{२}}} \hspace{4mm} \therefore \;$ या $= {\hbox{नी}}^{\scriptsize{\hbox{२}}} + {\hbox{क्षे}}_{\scriptsize{\hbox{२}}}$~।
\vspace{2mm}

अत उपपद्यते सर्वम्~।}{\large \textbf{{\color{purple}राशिर्येन समेतो \\
येनोनः स्यात् कृतिस्तयोरैक्यम्~।\\ 
इष्टहृदूनं दलितं \\
स्वघ्नं हीनेन युग्राशिः~॥~५२~॥}}}
\end{quote}

\noindent \textbf{उदाहरणम्~।}

\phantomsection \label{Ex 1.57}
\begin{quote}
\textbf{{\color{red}द्विष्ठः सप्तदशाढ्यः कश्चतुर्भिर्वर्जितः कृतिः~।\\
तं गाणितिकवर्याशु वद वेत्सि कृतिं यदि~॥~५७~॥}}
\end{quote}

१७।४ एकेनेष्टेन जातो राशिः १०४~। त्रिकेण ८~।\\

अर्धेन $\dfrac{{\footnotesize{\hbox{६९५३}}}}{{\footnotesize{\hbox{१६}}}}$~। एवमिष्टवशादनेकधा~।

\newpage

\noindent \textbf{सूत्रमार्या~।}

\phantomsection \label{1.53}
\begin{quote}
\renewcommand{\thefootnote}{१}\footnote{अत्रोपपत्तिः~। कल्प्यते राशिः $=$ या~। ततः प्रश्नोक्त्या
\vspace{1mm}

\hspace{6mm} या $+$ क्षे$_{\scriptsize{\hbox{१}}} =$ का$^{\scriptsize{\hbox{२}}}$
\vspace{1mm}

\hspace{6mm} या $+$ क्षे$_{\scriptsize{\hbox{२}}} =$ नी$^{\scriptsize{\hbox{२}}}$
\vspace{1mm}

\hspace{2mm} अन्तरेण~~ क्षे$_{\scriptsize{\hbox{१}}} -$ क्षे$_{\scriptsize{\hbox{२}}} =$ का$^{\scriptsize{\hbox{२}}} -$ नी$^{\scriptsize{\hbox{२}}} =$ (का $+$ नी)\;(का $-$ नी)
\vspace{1mm}

\hspace{2mm} यदि\; का $-$ नी $=$ इ\; तदा
\vspace{2mm}

\hspace{6mm} $\dfrac{{\footnotesize{\hbox{क्षे}}_{\scriptsize{\hbox{१}}} - {\hbox{क्षे}}_{\scriptsize{\hbox{२}}}}}{{\footnotesize{\hbox{इ}}}} =$ फ $=$ का $+$ नी
\vspace{1mm}

\hspace{23mm} इ $=$ का $-$ नी
\vspace{2mm}

\hspace{2mm} ततः~~ फ $+$ इ $=$ २\,का
\vspace{3mm}

\hspace{2mm} $\therefore$\; का $= \dfrac{{\footnotesize{\hbox{फ}} + {\hbox{इ}}}}{{\footnotesize{\hbox{२}}}}$
\vspace{2mm}

\hspace{2mm} तथा~~ या $= {\hbox{का}}^{\scriptsize{\hbox{२}}} -{\hbox{क्षे}}_{\scriptsize{\hbox{१}}}$~।\; अत्र यदि\, इ $=$ १\, तदाचार्योक्तमुपपद्यते~।
\vspace{2mm}
}{\large \textbf{{\color{purple}द्विष्ठो राशिर्याभ्यां \\
सहितो वर्गो भवेत्तयोर्विवरः~।\\ 
सैको द्विहृतः स्वघ्नोऽ-\\
नल्पविहीनो भवेद्राशिः~॥~५३~॥}}}
\end{quote}

\noindent \textbf{उदाहरणम्~।}

\phantomsection \label{Ex 1.58}
\begin{quote}
\textbf{{\color{red}पृथक् समन्वितो राशिस्त्रिभिश्च दशभिः सखे~।\\ 
मूलदो जायते तं मे वद वर्गेऽसि चेत् पटुः~॥}}
\end{quote}

१०।३ जातो राशिः ६~।\\

\noindent \textbf{सूत्रमार्या~।}

\phantomsection \label{1.54.1}
\begin{quote}
\renewcommand{\thefootnote}{२}\footnote{अत्रोपपत्तिः~। कल्प्यते राशिः $=$ या~। ततः प्रश्नोक्त्या
\vspace{1mm}

\hspace{6mm} या $-$ क्षे$_{\scriptsize{\hbox{१}}} =$ का$^{\scriptsize{\hbox{२}}}$
\vspace{1mm}

\hspace{6mm} या $-$ क्षे$_{\scriptsize{\hbox{२}}} =$ नी$^{\scriptsize{\hbox{२}}}$
\vspace{1mm}
}{\large \textbf{{\color{purple}राशिर्द्विष्ठो याभ्यां \\
रहितः कृतितां प्रयाति\renewcommand{\thefootnote}{$\star$}\footnote{The reading प्रयान्ति seems to be a typographical error.} तद्विवरः~।}}}
\end{quote}

\newpage

\phantomsection \label{1.54}
\begin{quote}
{\large \textbf{{\color{purple}व्येको दलितः स्वघ्नोऽ-\\
नल्पसमेतो भवेद्राशिः~॥~५४~॥}}}\renewcommand{\thefootnote}{}\footnote{अन्तरेण~~ क्षे$_{\scriptsize{\hbox{१}}} -$ क्षे$_{\scriptsize{\hbox{२}}} =$ का$^{\scriptsize{\hbox{२}}} -$ नी$^{\scriptsize{\hbox{२}}} =$ (का $+$ नी)\;(का $-$ नी)
\vspace{1mm}

\hspace{2mm} अत्रापि यदि\; का $-$ नी $=$ इ\; तदा
\vspace{2mm}

\hspace{6mm} $\dfrac{{\footnotesize{\hbox{क्षे}}_{\scriptsize{\hbox{१}}} - {\hbox{क्षे}}_{\scriptsize{\hbox{२}}}}}{{\footnotesize{\hbox{इ}}}} =$ फ $=$ का $+$ नी
\vspace{1mm}

\hspace{23mm} इ $=$ का $-$ नी
\vspace{2mm}

\hspace{2mm} $\therefore\; \dfrac{{\footnotesize{\hbox{फ}} - {\hbox{इ}}}}{{\footnotesize{\hbox{२}}}} =$ नी
\vspace{2mm}

\hspace{2mm} तथा~~ या $= {\hbox{नी}}^{\scriptsize{\hbox{२}}} + {\hbox{क्षे}}_{\scriptsize{\hbox{२}}}$~।
\vspace{2mm}

\hspace{2mm} अत्र यदि\, इ $=$ १\, तदाचार्योक्तमुपपद्यते~।
\vspace{2mm}
}
\end{quote}

\noindent \textbf{उदाहरणम्~।}

\phantomsection \label{Ex 1.59}
\begin{quote}
\textbf{{\color{red}को राशिः सप्तविंशत्या चतुर्भिर्वर्जितः पृथक्~।\\ 
मूलदो विद्धि तं पाट्यां पाटवं तेऽस्ति चेत् सखे~॥}}
\end{quote}

न्यासः ४ं।२ं७ जातो राशिः १४८~।\\

\noindent \textbf{सूत्रमार्या~।}

\phantomsection \label{1.55}
\begin{quote}
\renewcommand{\thefootnote}{१}\footnote{अत्रोपपत्तिः~। कल्प्येते राशी या, का~। ततः प्रश्नोक्त्या
\vspace{2mm}

\hspace{8mm} या$^{\scriptsize{\hbox{२}}} + {\hbox{या.का}} +$ का$^{\scriptsize{\hbox{२}}} = {\hbox{नी}}^{\scriptsize{\hbox{२}}}$~।
\vspace{1mm}

\hspace{2mm} वा~~ $\left[{\hbox{या}} + \dfrac{{\footnotesize{\hbox{का}}}}{{\footnotesize{\hbox{२}}}}\right]^{\scriptsize{\hbox{२}}} + \dfrac{{\footnotesize{\hbox{३\,का}}^{\scriptsize{\hbox{२}}}}}{{\footnotesize{\hbox{४}}}} = {\hbox{नी}}^{\scriptsize{\hbox{२}}}$
\vspace{2mm}

\hspace{2mm} वा~~ $\dfrac{{\footnotesize{\hbox{३\,का}}^{\scriptsize{\hbox{२}}}}}{{\footnotesize{\hbox{४}}}} = {\hbox{नी}}^{\scriptsize{\hbox{२}}} - \left[{\hbox{या}} + \dfrac{{\footnotesize{\hbox{का}}}}{{\footnotesize{\hbox{२}}}}\right]^{\scriptsize{\hbox{२}}}$}{\large \textbf{{\color{purple}यत्किञ्चित् प्रथमस्तत्-\\ 
वर्गः स्वाङ्घ्र्यून इष्टभक्तोनः~।\\ 
आद्योनो दलितोऽन्यो \\
वर्गैक्याढ्यो वधो वर्गः~॥~५५~॥}}}
\end{quote}

\newpage

\noindent \textbf{उदाहरणम्~।}\renewcommand{\thefootnote}{}\footnote{$= \left[{\hbox{नी}} + \left({\hbox{या}} + \dfrac{{\footnotesize{\hbox{का}}}}{{\footnotesize{\hbox{२}}}}\right) \right]\;\left[{\hbox{नी}} - \left({\hbox{या}} + \dfrac{{\footnotesize{\hbox{का}}}}{{\footnotesize{\hbox{२}}}}\right) \right]$ 
\vspace{2mm}

\hspace{2mm} अत्र यदि\; ${\hbox{नी}} - \left({\hbox{या}} + \dfrac{{\footnotesize{\hbox{का}}}}{{\footnotesize{\hbox{२}}}}\right) =$ इ
\vspace{2mm}
 
\hspace{2mm} तदा\; $\dfrac{\dfrac{{\footnotesize{\hbox{३\,का}}^{\scriptsize{\hbox{२}}}}}{{\footnotesize{\hbox{४}}}}}{{\footnotesize{\hbox{इ}}}} = {\hbox{नी}} + \left({\hbox{या}} + \dfrac{{\footnotesize{\hbox{का}}}}{{\footnotesize{\hbox{२}}}}\right) =$ फ
\vspace{2mm}

\hspace{2mm} सङ्क्रमेण\; $\dfrac{{\footnotesize{\hbox{फ}} - {\hbox{इ}}}}{{\footnotesize{\hbox{२}}}} = {\hbox{या}} + \dfrac{{\footnotesize{\hbox{का}}}}{{\footnotesize{\hbox{२}}}}$
\vspace{2mm}
  
\hspace{2mm} $\therefore$\; या $= \dfrac{{\footnotesize{\hbox{फ}} - {\hbox{इ}} - {\hbox{का}}}}{{\footnotesize{\hbox{२}}}}$~।
\vspace{2mm}

अत उपपद्यते~।
\vspace{2mm}
}

\phantomsection \label{Ex 1.60}
\begin{quote}
\textbf{{\color{red}वर्गयोगः कयो राश्योर्घाताढ्यः स्यात् पदप्रदः~।\\
तावाशु\renewcommand{\thefootnote}{$\star$}\footnote{The reading तमाशु seems to be a typographical error.
\vspace{1mm}
} वद चेद्वर्गकुतुकेऽसि कुतूहली~॥}}
\end{quote}

यत्किंचित् प्रथम इति कल्पितः १२~। एकेनेष्टेन जातौ राशी १२, $\dfrac{{\footnotesize{\hbox{९५}}}}{{\footnotesize{\hbox{२}}}}$~। एतावभिन्नार्थं द्विगुणितौ २४।९५~। द्विकेनेष्टेन जातौ १२,२०~। एतौ सति संभवे चतुर्भिरपवर्तितौ ३,५~। एवमिष्टवशादानन्त्यम्~।
\vspace{3mm}

शेषं क्षेत्रोपयोगि तत्रैव वक्ष्ये~।

\begin{center}
\textbf{इति कृतिकौतूहलम्~।}
\vspace{6mm}

{\Large \textbf{अथ घातसमासादिसाम्यमुच्यते~।}}
\end{center}

\noindent \textbf{सूत्रमार्या~।}

\phantomsection \label{1.56.1}
\begin{quote}
\renewcommand{\thefootnote}{१}\footnote{अत्रोपपत्तिः~। कल्प्येते राशी या, का~। ततः प्रश्नोक्त्या
\vspace{1mm}
  
\hspace{5mm} या $+$ का $=$ याका \hspace{3mm} $\therefore$\; या $=$ याका $-$ का $=$ का (या $-$ १)
\vspace{1mm}
  
\hspace{2mm} $\therefore$\; का $= \dfrac{{\footnotesize{\hbox{या}}}}{{\footnotesize{\hbox{या}} - {\hbox{१}}}}$~। अनेन प्रकारान्तरमुपपद्यते~।}{\large \textbf{{\color{purple}इष्टद्वयसंयोगो \\
द्विष्ठस्ताविष्टभाजितौ राशी~।}}}
\end{quote}

\newpage

\phantomsection \label{1.56}
\begin{quote}
{\large \textbf{{\color{purple}अभ्यासेन समासः}}}\renewcommand{\thefootnote}{}\footnote{अथ~~ या $+$ का $=$ या\,का \hspace{4mm} $\therefore\; \dfrac{{\footnotesize{\hbox{या}} + {\hbox{का}}}}{{\footnotesize{\hbox{या}}}} =$ का
\vspace{2mm}

\hspace{2mm} तथा~~ $\dfrac{{\footnotesize{\hbox{या}} + {\hbox{का}}}}{{\footnotesize{\hbox{का}}}} =$ या
\vspace{2mm}

\hspace{2mm} अतो यावत्तावत्कालकौ कावपीष्टौ कल्प्यौ~।
\vspace{1mm}

\hspace{2mm} अनेन प्रथमः प्रकार उपपद्यते~।
\vspace{1mm}
} \\
{\large \textbf{{\color{purple}समस्तयोर्जायते नियतम्}}}\renewcommand{\thefootnote}{$\star$}\footnote{इष्टः प्रथमो राशिर्व्येकेष्टहृतः स एवान्यः~। प्रकारान्तरमेतत्~।
\vspace{1mm}
}\,{\large \textbf{{\color{purple}॥~५६~॥}}}
\end{quote}

\noindent \textbf{उदाहरणम्~।}

\phantomsection \label{Ex 1.61}
\begin{quote}
\textbf{{\color{red}समे समाससंहती ययोश्च तावनेकधा~।\\ 
वद द्रुतं त्वया परिश्रमः कृतोऽत्र कर्मणि~॥~६१~॥}}
\end{quote}

इष्टे १।१ आभ्यां जातौ राशी २,२~। अथवेष्टे १।२
\vspace{2mm}

आभ्याम् ३, $\dfrac{{\footnotesize{\hbox{३}}}}{{\footnotesize{\hbox{२}}}}$\, वा ३।४ आभ्याम्\, $\dfrac{{\footnotesize{\hbox{७}}}}{{\footnotesize{\hbox{३}}}}$, $\dfrac{{\footnotesize{\hbox{७}}}}{{\footnotesize{\hbox{४}}}}$~।\\

एवमिष्टवशादानन्त्यम्~।\\

\noindent \textbf{सूत्रमार्या~।}

\phantomsection \label{1.57}
\begin{quote}
\renewcommand{\thefootnote}{१}\footnote{अत्रोपपत्तिः~। कल्प्येते राशी या, का~। ततः प्रश्नोक्त्या
\vspace{1mm}

\hspace{2mm} या$^{\scriptsize{\hbox{२}}} +$ का$^{\scriptsize{\hbox{२}}} = $ या $+$ का
\vspace{2mm}

\hspace{2mm} वा \hspace{4mm} या\,(या$^{\scriptsize{\hbox{२}}} +$ का$^{\scriptsize{\hbox{२}}}$) $=$ या\,(या $+$ का)  \hspace{4mm} $\therefore\;$ या $= \dfrac{{\footnotesize{\hbox{या}}\,({\hbox{या}} + {\hbox{का}})}}{{\footnotesize{\hbox{या}}^{\scriptsize{\hbox{२}}} + {\hbox{का}}^{\scriptsize{\hbox{२}}}}}$
\vspace{2mm}

\hspace{2mm} एवम्~~ का\,(या$^{\scriptsize{\hbox{२}}} +$ का$^{\scriptsize{\hbox{२}}}$) $=$ का\,(या $+$ का)  \hspace{4mm} $\therefore\;$ का $= \dfrac{{\footnotesize{\hbox{का}}\,({\hbox{या}} + {\hbox{का}})}}{{\footnotesize{\hbox{या}}^{\scriptsize{\hbox{२}}} + {\hbox{का}}^{\scriptsize{\hbox{२}}}}}$
\vspace{1mm}

\hspace{2mm} अत उपपद्यते सूत्रम्~।
\vspace{1mm}
}{\large \textbf{{\color{purple}इष्टे योगेन गुणे\renewcommand{\thefootnote}{२}\footnote{इष्टे तद्युतिनिघ्ने~।
\vspace{1mm}
} \\
तत्कृतियोगोद्धृते च राशी स्तः~।\\
वर्गसमासेन तयोः \\
संयोगो जायते तुल्यः\renewcommand{\thefootnote}{३}\footnote{योगः सञ्जायते तुल्यः~। इति पाठान्तरम्~।}॥~५७~॥}}}
\end{quote}

\newpage

\noindent \textbf{उदाहरणम्~।}

\phantomsection \label{Ex 1.62}
\begin{quote}
\textbf{{\color{red}ययोश्च वर्गसंयुतेः समा भवेद्युतिः सखे~।\\ 
वदाशु तावनेकधा यदीह तेऽस्ति पाटवम्~॥}}
\end{quote}

इष्टे १।२ आभ्यां जातौ राशी\, $\dfrac{{\footnotesize{\hbox{३}}}}{{\footnotesize{\hbox{५}}}}$, $\dfrac{{\footnotesize{\hbox{६}}}}{{\footnotesize{\hbox{५}}}}$~।\\

अथवेष्टे २।३ आभ्यां जातौ राशी\, $\dfrac{{\footnotesize{\hbox{१०}}}}{{\footnotesize{\hbox{१३}}}}$, $\dfrac{{\footnotesize{\hbox{१५}}}}{{\footnotesize{\hbox{१३}}}}$~।\\

एवमिष्टवशादानन्त्यम्~।\\

\noindent \textbf{सूत्रमार्या~।}

\phantomsection \label{1.58}
\begin{quote}
\renewcommand{\thefootnote}{१}\footnote{अत्रोपपत्तिः~। कल्प्येते राशी या, का~। ततः प्रश्नोक्त्या
\vspace{1mm}

\hspace{13.5mm} या$^{\scriptsize{\hbox{३}}} +$ का$^{\scriptsize{\hbox{३}}} = $ या$^{\scriptsize{\hbox{२}}} +$ का$^{\scriptsize{\hbox{२}}}$
\vspace{1mm}

\hspace{2mm} वा,~~ या\,(या$^{\scriptsize{\hbox{३}}} +$ का$^{\scriptsize{\hbox{३}}}) =$ या\,(या$^{\scriptsize{\hbox{२}}} +$ का$^{\scriptsize{\hbox{२}}}$)
\vspace{2mm}

\hspace{2mm} $\therefore\; \dfrac{{\footnotesize{\hbox{या}}\,({\hbox{या}}^{\scriptsize{\hbox{२}}} + {\hbox{का}}^{\scriptsize{\hbox{२}}})}}{{\footnotesize{\hbox{या}}^{\scriptsize{\hbox{३}}} + {\hbox{का}}^{\scriptsize{\hbox{३}}}}} =$ या
\vspace{2mm}

\hspace{2mm} एवम्,~~ का\,(या$^{\scriptsize{\hbox{३}}} +$ का$^{\scriptsize{\hbox{३}}}) =$ का\,(या$^{\scriptsize{\hbox{२}}} +$ का$^{\scriptsize{\hbox{२}}}$)
\vspace{2mm}

\hspace{5mm} $\therefore\; \dfrac{{\footnotesize{\hbox{का}}\,({\hbox{या}}^{\scriptsize{\hbox{२}}} + {\hbox{का}}^{\scriptsize{\hbox{२}}})}}{{\footnotesize{\hbox{या}}^{\scriptsize{\hbox{३}}} + {\hbox{का}}^{\scriptsize{\hbox{३}}}}} =$ का~।
\vspace{2mm}

\hspace{2mm} एवं द्वितीयप्रश्ने~~ $\dfrac{{\footnotesize{\hbox{या}}\,({\hbox{या}} + {\hbox{का}})^{\scriptsize{\hbox{२}}}}}{{\footnotesize{\hbox{या}}^{\scriptsize{\hbox{३}}} + {\hbox{का}}^{\scriptsize{\hbox{३}}}}} =$ या
\vspace{2mm}

\hspace{16mm} $\therefore\; \dfrac{{\footnotesize{\hbox{का}}\,({\hbox{या}} + {\hbox{का}})^{\scriptsize{\hbox{२}}}}}{{\footnotesize{\hbox{या}}^{\scriptsize{\hbox{३}}} + {\hbox{का}}^{\scriptsize{\hbox{३}}}}} =$ का~।
\vspace{2mm}

\hspace{2mm} एवमन्यप्रश्नेष्वपि राशी भवत इति~।
\vspace{1mm}
}{\large \textbf{{\color{purple}घनयुतिभक्ते कृतियुति-\\
युतिकृतिघाताहते त्विष्टे\renewcommand{\thefootnote}{$\dag$}\footnote{घनयुतियुतिघनभक्ते कृतियुतियुतिकृतिवधाहतेऽभीष्टे~। इति पाठान्तरम्~।}।\\
घनयुतियुतिघनतुल्या \\
कृतियुतियुतिकृतिवधां राश्योः~॥~५८~॥}}}
\end{quote}

\newpage

\noindent \textbf{उद्देशकषट्कम्~।}

\phantomsection \label{Ex 1.63}
\begin{quote}
\textbf{{\color{red}ययोश्च वर्गसंयुतिर्युतेः कृतिर्वधस्तथा~।\\ 
घनैक्यतुल्यतां ययोर्युतेर्घनस्य तौ वद~॥}}
\end{quote}

प्रथमोदाहरणे घनयोगार्थमिष्टे १।२ कृतियुतौ जातौ राशी\, $\dfrac{{\footnotesize{\hbox{५}}}}{{\footnotesize{\hbox{९}}}}$, $\dfrac{{\footnotesize{\hbox{१०}}}}{{\footnotesize{\hbox{९}}}}$~।
\vspace{2mm}

इष्टे १।२ युतिकृतौ राशी १२~।
\vspace{2mm}

इष्टे १।२ घाते जातौ राशी\, $\dfrac{{\footnotesize{\hbox{२}}}}{{\footnotesize{\hbox{९}}}}$, $\dfrac{{\footnotesize{\hbox{४}}}}{{\footnotesize{\hbox{९}}}}$~।\\ 

अथवा योगघनार्थमिष्टे १।२ कृतियुतौ जातौ राशी\, $\dfrac{{\footnotesize{\hbox{५}}}}{{\footnotesize{\hbox{२७}}}}$, $\dfrac{{\footnotesize{\hbox{१७}}}}{{\footnotesize{\hbox{२७}}}}$~।\\

इष्टे १।२ युतिकृतौ राशी\, $\dfrac{{\footnotesize{\hbox{१}}}}{{\footnotesize{\hbox{३}}}}$, $\dfrac{{\footnotesize{\hbox{२}}}}{{\footnotesize{\hbox{३}}}}$~।\\

इष्टे १।२ घाते जातौ राशी\, $\dfrac{{\footnotesize{\hbox{२}}}}{{\footnotesize{\hbox{२७}}}}$, $\dfrac{{\footnotesize{\hbox{४}}}}{{\footnotesize{\hbox{२७}}}}$~।\\ 

एवं स्वबुद्ध्यान्तरादिराशी ज्ञेयौ~।

\begin{center}
\textbf{इति योगादितुल्यसाधनम्~।}
\end{center}
\vspace{2mm}

{\large \textbf{अथ व्यस्तविधौ सूत्रमार्या~।}}

\phantomsection \label{1.59}
\begin{quote}
\renewcommand{\thefootnote}{$\star$}\footnote{छेदं गुणं गुणं छेदमित्यादि भास्करोक्तानुरूपमेवेदम्~।}{\large \textbf{{\color{purple}कृतिपदयोर्हरगुणयोः \\
धनर्णयोर्दृश्यतो विपर्यासः~।\\ 
स्वांशाधिके विहीने \\
रूपाढ्योनांशहृद्राशिः~॥~५९~॥}}}
\end{quote}

\noindent \textbf{उदाहरणम्~।}

\phantomsection \label{Ex 1.64.1}
\begin{quote}
\textbf{{\color{red}(को राशिश्चतुराहतो निजचतुर्भागैस्त्रिभिर्वर्जितः\\
षड्युक्तः स्वहतो रसेन विहृतः स्वत्र्यंशयुग्मोनितः~।)}}\renewcommand{\thefootnote}{$\dag$}\footnote{हस्तलिखितपुस्तके श्लोकत्रुटिः~। अतः कोष्ठान्तर्गतो भागो निवेशितः~।}
\end{quote}

\newpage

\phantomsection \label{Ex 1.64}
\begin{quote}
\textbf{{\color{red}को राशिश्चतुराहतो निजचतुर्भागैस्त्रिभिर्वर्जितः\\
षड्युक्तः स्वहतश्च षष्टिविहृतः स्वत्र्यंशयुग्मान्वितः~।\\
तन्मूले द्विविवर्जिते यदि सखे शिष्टं च रूपं द्रुतं\\
राशिं तं वद कोविदास्ति गणिताभ्यासः प्रभूतस्तव~॥}}
\end{quote}

न्यासः~। गुणः ४~। ऋ\, $\dfrac{{\footnotesize{\hbox{३}}}}{{\footnotesize{\hbox{४}}}}$~। ध ६ वर्गः हारः ६०। ध\, $\dfrac{{\footnotesize{\hbox{२}}}}{{\footnotesize{\hbox{३}}}}$~। मू ऋ २~।\\

दृ १~। जातो राशिः १२~।\\
\vspace{4mm}

\noindent {\large \textbf{अथ त्रैराशिके सूत्रमार्या~।}}

\phantomsection \label{1.60}
\begin{quote}
\renewcommand{\thefootnote}{$\dag$}\footnote{{\color{violet}'प्रमाणमिच्छा च समानजाती'} इत्यादि {\color{violet}भास्करो}क्तमेवेदम्~।}{\large \textbf{{\color{purple}आद्यान्त्ययोः प्रमाणे-\\
च्छे समजाती फलं त्वितरजाति~।\\ 
मध्ये तदन्तताडितम् \\
आदिहृदिच्छाफलं भवति~॥~६०~॥}}}
\end{quote}

\noindent \textbf{उदाहरणम्~।}
 
\phantomsection \label{Ex 1.65}
\begin{quote}
\textbf{{\color{red}नारिकेलफलान्यष्टौ लभ्यन्ते पञ्चभिः पणैः~।\\ 
चत्वारिंशत् फलानां किं मौल्यं वद सखे मम~॥}}
\end{quote}

न्यासः ८।५।४० जातौ द्रम्मौ\renewcommand{\thefootnote}{$\star$}\footnote{अत्र द्वादशभिः पणैरेको द्रम्मस्तदर्थं परिभाषा द्रष्टव्या~।} २ पणश्च १~।\\

\noindent \textbf{अपि च~।}

\phantomsection \label{Ex 1.66}
\begin{quote}
\textbf{{\color{red}सदलानि पलान्यष्टौ कुङ्कुमस्य त्रिभिः पणैः~।\\ 
सपादैस्तत्षडंशोनैः पणैः किं दशभिर्वद~॥}}
\end{quote}

न्यासः\; ३\,$\dfrac{{\footnotesize{\hbox{१}}}}{{\footnotesize{\hbox{४}}}}$~। ८\,$\dfrac{{\footnotesize{\hbox{१}}}}{{\footnotesize{\hbox{२}}}}$~। १०\,$\dfrac{{\footnotesize{\hbox{१ं}}}}{{\footnotesize{\hbox{६}}}}$~।\\

जातानि पलानि २५ कर्षौ २ माषाः १२~।\\

पुञ्जाः ४ गुञ्जाभागाश्च\, $\dfrac{{\footnotesize{\hbox{२९}}}}{{\footnotesize{\hbox{३९}}}}$~।

\newpage

\noindent \textbf{अपि च~।}

\phantomsection \label{Ex 1.67}
\begin{quote}
\textbf{{\color{red}त्रिलवाधिकस्य गुग्गुलपलाष्टकस्यार्धयुक् पणत्रितयम्~।\\ 
तत् किं सगुञ्जकस्य प्रवद सखे पलशतस्याशु~॥}}
\end{quote}

न्यासः\; ८\,$\dfrac{{\footnotesize{\hbox{१}}}}{{\footnotesize{\hbox{३}}}}$~। ३\,$\dfrac{{\footnotesize{\hbox{३}}}}{{\footnotesize{\hbox{२}}}}$~। १००\,$\dfrac{{\footnotesize{\hbox{१}}}}{{\footnotesize{\hbox{३२०}}}}$~।\\

जाता द्रम्माः ३ पणाः ६ काकिणी ० वराटकः ० वराटकभागाश्च\, $\dfrac{{\footnotesize{\hbox{२१}}}}{{\footnotesize{\hbox{३२०}}}}$~।\\

\noindent \textbf{अपि च~।}

\phantomsection \label{Ex 1.68}
\begin{quote}
\textbf{{\color{red}द्रम्मैश्चतुर्भिर्दशभागहीनैः \\
त्रिपादिकोनं कुडवत्रयं चेत्~।\\
अवाप्यते तत्कुडवाधिकायाः \\
किं खारिकायाः कथयाशु मौल्यम्~॥}}
\end{quote}

न्यासः\; ३\,$\dfrac{{\footnotesize{\hbox{१ं}}}}{{\footnotesize{\hbox{३}}}}$~। ४\,$\dfrac{{\footnotesize{\hbox{१ं}}}}{{\footnotesize{\hbox{१०}}}}$~। १\,$\dfrac{{\footnotesize{\hbox{१}}}}{{\footnotesize{\hbox{२०}}}}$~।\\

आद्यन्तयोः पादिकाः कृताः\, $\dfrac{{\footnotesize{\hbox{४५}}}}{{\footnotesize{\hbox{१}}}}$~। ४\,$\dfrac{{\footnotesize{\hbox{१}}}}{{\footnotesize{\hbox{१०}}}}$~। $\dfrac{{\footnotesize{\hbox{३३६}}}}{{\footnotesize{\hbox{१}}}}$~।\\

जाता द्रमाः २६ पणः १ काकिणी १ वराटकाश्च १५\,$\dfrac{{\footnotesize{\hbox{१}}}}{{\footnotesize{\hbox{५}}}}$~।\\

\noindent \textbf{अपि च~।}

\phantomsection \label{Ex 1.69}
\begin{quote}
\textbf{{\color{red}अङ्गुलैश्च सदलैः करं त्रिभिः \\
संयुतं दिनदलेन याति चेत्~।\\
सर्पिणी च समयेन केन सा \\
योजनानि सदलानि पञ्च च~॥}}
\end{quote}

\newpage

\phantomsection \label{Ex 1.64.2}
\begin{quote}
\textbf{{\color{red}(तत्खण्डस्य पदं विवर्जितमथो द्वाभ्यां च रूपं द्रुतम्~।\\
राशिं तं वद कोविदोऽस्ति गणिताभ्यासः प्रभूतस्तव~॥)}}
\end{quote}
\vspace{4mm}

न्यासः~। गुणः ४~। स्व $\dfrac{{\footnotesize{\hbox{३ं}}}}{{\footnotesize{\hbox{४}}}}$~। ध ६\; व.भा ६~। स्व $\dfrac{{\footnotesize{\hbox{२ं}}}}{{\footnotesize{\hbox{३}}}}$~। भा २~। मू.ऋ २ं~। दृ १~।
\vspace{2mm}

जातो राशिः १२~।\\

न्यासः~। $\dfrac{{\footnotesize{\hbox{१}}}}{{\footnotesize{\hbox{३}}}}$~। $\dfrac{{\footnotesize{\hbox{१}}}}{{\footnotesize{\hbox{२}}}}$~। ५\,$\dfrac{{\footnotesize{\hbox{१}}}}{{\footnotesize{\hbox{२}}}}$~। आद्यन्तयोरङ्गुलानि\, $\dfrac{{\footnotesize{\hbox{५५}}}}{{\footnotesize{\hbox{२}}}}$~। $\dfrac{{\footnotesize{\hbox{१}}}}{{\footnotesize{\hbox{२}}}}$~। $\dfrac{{\footnotesize{\hbox{४२२४०००}}}}{{\footnotesize{\hbox{१}}}}$~। \\

जातानि वर्षाणि २१३, मासाः ४~।

\begin{center}
\textbf{इति त्रैराशिकम्~।}

\noindent\rule{2cm}{0.7pt}
\vspace{6mm}

{\large \textbf{व्यस्तत्रैराशिके सूत्रमार्या~।}}
\end{center}
\vspace{-3mm}

\phantomsection \label{1.61}
\begin{quote}
{\large \textbf{{\color{purple}मौल्याऽसूनां वयसा \\
हेम्नो वर्णस्य तुल्यधरणेऽपि~।\\
धान्यादीनां कुडवा-\\
दिकस्य मानान्तरे व्यस्तम्~॥~६१~॥}}}
\end{quote}

\noindent \textbf{उदाहरणम्~।}

\phantomsection \label{Ex 1.70}
\begin{quote}
\textbf{{\color{red}प्राप्नोति भावपरिहासविलासरम्या \\
चेत् षोडशाब्दवनिता दशनिष्कभाटीम्~।\\
तन्मे द्रुतं प्रवद विंशतिवत्सरायाः \\
का भाटिका गणकवर्य विटेन\renewcommand{\thefootnote}{१}\footnote{नटेन, इति पाठान्तरम्~।} देया~॥}}
\end{quote}

न्यासः\; १६~। १०~। २०~। जाता भाटिका निष्काः ८~।\\

\noindent \textbf{अपि च~।}

\phantomsection \label{Ex 1.71}
\begin{quote}
\textbf{{\color{red}द्विधूरुक्षाष्टभिर्निष्कैर्यदि सम्प्राप्यते सखे~।\\
स पञ्चधूरवाप्येत निष्कैश्च कतिभिर्वद~॥}}
\end{quote}

\newpage

न्यासः~। २~। ८~। ५~। जाता निष्काः ३\,$\dfrac{{\footnotesize{\hbox{१}}}}{{\footnotesize{\hbox{५}}}}$~।\\

\noindent \textbf{अपि च~।}

\phantomsection \label{Ex 1.72}
\begin{quote}
\textbf{{\color{red}षड्गुञ्जिकेन माषेण सुवर्णत्रिशती सखे~।\\ 
तुलिता यदि सा पञ्चमाषेण वद किं भवेत्~॥}}
\end{quote}

न्यासः ६~। ३०००~। ५~। जाताः सुवर्णाः ३६०~।\\

\noindent \textbf{अपि च~।}

\phantomsection \label{Ex 1.73}
\begin{quote}
\textbf{{\color{red}मापितं कुडवेनैव पादिकाषोडशेन चेत्~।\\
खारिकाशतषट्कं स्यात् पादिकाष्टादशेन किम्~॥}}
\end{quote}

न्यासः १६~। ३००~। १८~। जाताः खार्यः ५३३, कुडवाः ६, पादिकाश्च १२\,$\dfrac{{\footnotesize{\hbox{१}}}}{{\footnotesize{\hbox{३}}}}$~।
\vspace{2mm}

\begin{center}
\textbf{इति व्यस्तत्रैराशिकम्~।}
\end{center}
\vspace{2mm}

{\large \textbf{पञ्चराशिकादौ सूत्रमार्या~।}}

\phantomsection \label{1.62}
\begin{quote}
{\large \textbf{{\color{purple}व्यत्यासं हरफलयोः\\
\renewcommand{\thefootnote}{$\star$}\footnote{`र्मिथो विधायाल्पराशिघाताप्ते' इति पाठोऽनुमीयते~। प्रकारश्च {\color{violet}'पञ्चसप्तनवराशिकादिके'} इत्यादि {\color{violet}भास्करो}-क्तानुरूप एव~।}कृत्वाल्पराशिघाताप्ते~।\\ 
बहु‍राशि‍वधे च स्यात् \\
पञ्चादिषु राशिकेषु फलम्~॥~६२~॥}}}
\end{quote}

\noindent \textbf{उदाहरणम्~।}

\phantomsection \label{Ex 1.74}
\begin{quote}
\textbf{{\color{red}मासेन पञ्चक‍शतेन हि वत्सरेण \\
किं स्यात्फलं द्रुततरं वद पञ्च‍षष्टेः~।\\
मूलं फलात्कथय मूल‍कलान्तराभ्यां \\
कालं प्रचक्ष्व यदि कोविद वेत्सि पाटीम्~॥}}
\end{quote}

\newpage

न्यासः~। \begin{tabular}{c|c}
१ & १२ \\
१०० & ६५ \\
५ &
\end{tabular} जातं कलान्तरम् ३९~।\\
\vspace{2mm}

पुनर्न्यासः~। \begin{tabular}{c|c}
१ & १२ \\
१०० &  \\
५ & ३९
\end{tabular} जातं मूलधनम् ६५~।\\
\vspace{2mm}

पुनर्न्यासः~। \begin{tabular}{c|c}
१ &  \\
१०० & ६५ \\
५ & ३९
\end{tabular} जाता मासाः १२~।\\
\vspace{2mm}

एवं प्रमाणकालादि~।\\

\noindent \textbf{अपि च~।}

\phantomsection \label{Ex 1.75}
\begin{quote}
\textbf{{\color{red}व्यङ्घ्रेः शतस्य हि फलं सदलद्वयं यत् \\
पक्षेण पक्षसहितैर्दशभिश्च मासैः~।\\
षण्णां\renewcommand{\thefootnote}{$\star$}\footnote{The reading षस्मां doesn’t seem to be correct.} तदा सदलषष्टिसमन्वितानां \\
किं स्यात्फलं प्रवद भो यदि वेत्सि पाटीम्~॥}}
\end{quote}

न्यासः~ \begin{tabular}{c|c}
$\dfrac{{\footnotesize{\hbox{१}}}}{{\footnotesize{\hbox{२}}}}$ & १०\,$\dfrac{{\footnotesize{\hbox{१}}}}{{\footnotesize{\hbox{२}}}}$ \\
 & \\
१००\,$\dfrac{{\footnotesize{\hbox{१}}}}{{\footnotesize{\hbox{४}}}}$ & ६६\,$\dfrac{{\footnotesize{\hbox{१}}}}{{\footnotesize{\hbox{२}}}}$ \\
 & \\
२\,$\dfrac{{\footnotesize{\hbox{१}}}}{{\footnotesize{\hbox{२}}}}$ & 
\end{tabular} ~जातं कलान्तरम् ३५~।\\
\vspace{2mm}

\noindent \textbf{अपि च~।}

\phantomsection \label{Ex 1.76}
\begin{quote}
\textbf{{\color{red}प्राप्यते च पणसप्तकं त्रिभिः \\
वासरैर्भृतिभुजा नरेण च~।\\
वासरैर्द्विगुणितैस्तु पञ्चभिः \\
षड्भिरत्र पुरुषैः किमाप्यते~॥}}
\end{quote}

न्यासः~। \begin{tabular}{c|c}
१ & ६ \\
३ & १० \\
७ & 
\end{tabular} जाता द्रम्माः ११, पणाः ८~।

\newpage

\noindent \textbf{सप्तराशिकोदाहरणम्~।}

\phantomsection \label{Ex 1.77}
\begin{quote}
\textbf{{\color{red}त्रिहस्तविस्तारपटी त्रिवर्ग-\\
दैर्घ्या पुराणैर्दशभिः क्रीता चेत्\renewcommand{\thefootnote}{$\star$}\footnote{The reading कतिंश्चेत् seems to be grammatically incorrect or a typographical error, as it doesn't make any sense in the verse.}।\\
अवाप्यते द्वादशहस्तदैर्घ्ये \\
पटीत्रये पञ्चकविस्तृतौ किम्~॥}}
\end{quote}

न्यासः~। \begin{tabular}{c|c}
३ & ५ \\
९ & १२ \\
१० & ३ \\
१० & 
\end{tabular} जाता निष्कः १ द्रम्माः ३ पणाः ८~।\\
\vspace{2mm}

\noindent \textbf{नवराशिकोद्देशकः~।}

\phantomsection \label{Ex 1.78}
\begin{quote}
\textbf{{\color{red}विस्तारे च करद्वयं नवकरा दैर्घ्ये च पिण्डे करो\\
दारोर्यस्य स लभ्यते गजमितैर्द्रम्मैस्तु दारुद्वयम्~।\\
विस्तारे त्रिकरं दिवाकरकरायामं द्विपिण्डं द्रुतं\\
मौल्यं तस्य कियद्वदामलमते त्वं वेत्सि पाटीं यदि~॥}}
\end{quote}

न्यासः~। \begin{tabular}{c|c}
२ & ३ \\
९ & १२ \\
१ & २ \\
१ & २ \\
८ & 
\end{tabular} जातौ निष्कौ २ द्रम्माः ४~।\\
\vspace{2mm}

\noindent \textbf{एकादशराशिकोदाहरणम्~।}

\phantomsection \label{Ex 1.79}
\begin{quote}
\textbf{{\color{red}स्तम्भः पञ्चदशोच्छ्रयो गुणकरव्यासोऽर्धपिण्डश्च तत्\\
पञ्चक्रोशविचालनाय शकटी प्राप्नोति चेत् षट् पणान्~।\\
क्रोशाष्टानयनाय दारुदशकं माने त्रियुक्ते सखे\\
तस्मिन् किं वद भाटकं द्रुततरं त्वं वेत्सि पाटीं यदि~॥}}
\end{quote}

न्यासः \begin{tabular}{c|c}
१५ & १८ \\
३ & ६ \\ 
$\frac{{\footnotesize{\hbox{१}}}}{{\footnotesize{\hbox{२}}}}$ & ३\,$\frac{{\footnotesize{\hbox{१}}}}{{\footnotesize{\hbox{२}}}}$ \\
१ & १० \\
५ & ८ \\
५ & 
\end{tabular} जातं भाटकं निष्काः ३ द्रम्माः २६ पणाः ४ काकिण्यः ३ वराटकाः~४~।\\

\begin{center}
\textbf{इति पञ्चराशिकादिचतुष्टयम्~।}
\end{center}

\newpage

{\large \textbf{विनिमये सूत्रम्~।}}

\phantomsection \label{1.63.1}
\begin{quote}
\renewcommand{\thefootnote}{१}\footnote{{\color{violet}`तथैव भाण्डप्रतिभाण्डके विधिः'} इत्यादि {\color{violet}भास्करो}क्तानुरूपमेवेदम्~।
\vspace{1mm}

\hspace{2mm} `व्यत्ययविहिते मौल्ये भाण्डप्रतिभाण्डके विधिस्तद्वत्~।' इति पाठान्तरम्~।}{\large \textbf{{\color{purple}व्यस्तस्वहृते मौल्ये\\
भाण्डप्रतिभाण्डको भवति विधिः प्राग्वत्~।}}}
\end{quote}

\noindent \textbf{उदाहरणम्~।}

\phantomsection \label{Ex 1.80}
\begin{quote}
\textbf{{\color{red}द्रम्मेण दाडिमफलत्रिशती पणेन \\
पञ्चोनितं च विपणौ शतमाम्रकाणाम्~।\\
ब्रूह्याम्रकाणि दशकेन हि दाडिमानां \\
दक्षोऽसि चेद्विनिमये कति मित्र तानि~॥}}
\end{quote}

न्यासः \begin{tabular}{c|c}
१ & १२ \\
३०० & ९५ \\
१० & 
\end{tabular} जातान्याम्रकाणि ३८~।\\

\begin{center}
\textbf{इति विनिमयविधिः~।}
\end{center}
\vspace{2mm}

{\large \textbf{अथ जीवविक्रये सूत्रम्~।}}

\phantomsection \label{1.63}
\begin{quote}
{\large \textbf{{\color{purple}विहिते तु वैपरीत्ये \\
वयसोः प्राणिक्रये विधिः प्राग्वत्~॥~६३~॥}}}
\end{quote}

\noindent \textbf{उदाहरणम्~।}

\phantomsection \label{Ex 1.81}
\begin{quote}
\textbf{{\color{red}चेत् षोडशाब्दवनितायुगलस्य निष्क-\\
षष्टिर्भवेत् कथय मे नखवत्सराणाम्~।\\
षण्णां नितम्बभरमन्थरगामिनीनां \\
मौल्यं च किं गणकवर्य सुलोचनानाम्~॥}}
\end{quote}

\newpage

\begin{sloppypar}
न्यासः \begin{tabular}{c|c}
१६ & २० \\
२ & ६ \\
६० & 
\end{tabular} लब्धं निष्काः १४४~।\\
\vspace{2mm}

पञ्चराशिके द्विवारं त्रैराशिकं सप्तराशिके त्रिवारं नवराशिके चतुर्वारमेकादशराशिके पञ्चवारं भाण्डप्रतिभाण्डजीवविक्रययोर्व्यस्तत्रैराशिकं त्रैराशिकं चेति~।

\begin{center}
\textbf{इति सकलकलानिधिनरसिंहनन्दनगणितविद्याचतुराननश्रीनारायणपण्डित-\\
विरचितायां गणितपाट्यां कौमुद्यां प्रकीर्णकानि समाप्तानि~।}
\vspace{4mm}

\textbf{इति प्रकीर्णकव्यवहारः प्रथमः~।}
\end{center}
\vspace{4mm}

{\Large \textbf{अथ मिश्रव्यवहारः~।}}
\vspace{4mm}

\noindent \textbf{तत्र सूत्रम्~।}

\phantomsection \label{2.1.1}
\begin{quote}
\renewcommand{\thefootnote}{$\star$}\footnote{{\color{violet}`प्रक्षेपका मिश्रहताः'} इत्यादि {\color{violet}भास्करो}क्तानुरूपमेवेदम्~।}{\large \textbf{{\color{purple}प्रक्षेपास्तद्युतिहृत-\\
मिश्रेण हताः पृथक् फलानि स्युः~।}}}
\end{quote}

\noindent \textbf{उदाहरणम्~।}

\phantomsection \label{Ex 2.1}
\begin{quote}
\textbf{{\color{red}त्रिपञ्चसप्ताङ्कमितानि येषां \\
नवाहतान्यादिधनानि विद्वन्~।\\
चतुःशती षोडशवर्जिता च \\
जाता पृथग्लाभमितिं वदाशु~॥}}
\end{quote}

न्यासः २७।४५।६३।८१ मिश्रधनम् ३८४~। जातानि पृथग्लाभमूलानि ४८।८०।११२। १४४~। एतानि पूर्वमूलैरूनितानि जाता लाभाः २१।३५।४९।६३~। अथवा मूलधनैक्यं २१६ मिश्रधनैक्यादपास्य जातो लाभयोगः १६८ अस्मात् प्राग्वज्जाता लाभास्त एव २१।३५।४९। ६३~।
\end{sloppypar}

\newpage

\noindent \textbf{अपि च~।}

\phantomsection \label{Ex 2.2}
\begin{quote}
\textbf{{\color{red}मध्वाज्यदुग्धदधिभिः दशगुणितत्र्यादिचयफलप्रमितैः~।\\
आलोड्यैकत्र शिवं संस्नाप्यापूरितेषु कलशेषु~।\\
मध्वादिपलानां मे पृथक् पृथक् तेषु का सङ्ख्या~॥}}
\end{quote}

\begin{sloppypar}
न्यासः ३०।६०।९०।१२० अत्र मध्वानयने मिश्रम् ३०~। प्रक्षेपकरणे मधुघटे जातानि मध्वादीनां पलानि ३।६।९।१२ एतावन्त्येव सर्वघटेषु मधुपलानि~। एवमाज्यघटे मध्वादीनां पलानि ६।१२।१८।२४ एतावन्त्येव सर्वघटेष्वाज्यपलानि~। एवं क्षीरघटे मध्वादीनां पलानि ९।१८।२७।३६ एतावन्त्येव सर्वघटेषु क्षीरपलानि~। एवं दधिघटे मध्वादीनां पलानि १२।२४। ३६।४८ एतावन्त्येव सर्वघटे दधिपलानि~।\\
\end{sloppypar}

\noindent \textbf{अपि च~।}

\phantomsection \label{Ex 2.3}
\begin{quote}
\textbf{{\color{red}सखे चतुर्णां वणिजां क्रमेण \\
पञ्चादिकाश्वैर्दिवसैः षडाद्यैः\renewcommand{\thefootnote}{$\star$}\footnote{The reading षडश्वैः doesn’t correctly form a sentence.}।\\
सञ्चारितैः क्षेत्रधनं पणानां \\
सहस्रमेकं वद किं पृथक् स्यात्~॥}}
\end{quote}

न्यासः \;{\small $\begin{matrix}
\mbox{{५~ ६~ ७~ ८}}\\
\vspace{-1mm}
\mbox{{६~ ७~ ८~ ९}}
\vspace{1mm}
\end{matrix}$}\; मिश्रपणाः १००~।\\

निजदिनगुणिततुरङ्गा जाताः प्रक्षेपकाः ३०।४२।५६।७२~।
\vspace{2mm}

अतो जाताः पृथक् पृथक् पणाः १५०।२१०।२८०।३६०~।

\begin{center}
\textbf{इति प्रक्षेपकाः~।}\\
\vspace{6mm}

{\large \textbf{अथ क्रयविक्रयविधानम्~।}}
\end{center}

\noindent \textbf{सूत्रम्~।}

\phantomsection \label{2.1}
\begin{quote}
{\large \textbf{{\color{purple}मिश्रं भवेत् क्रयार्घः \\
क्रयश्च मिश्रं तु विक्रयो मूलम्~॥~१~॥ \\
मूलं च विक्रयार्घो \\
विक्रयहीनः क्रयो लाभः~।}}}
\end{quote}

\newpage

\phantomsection \label{2.2}
\begin{quote}
{\large \textbf{{\color{purple}ज्ञेयमनुपातविधिना \\
यद्यदविदितं फलं तत्तत्~॥~२~॥ \\
प्राग्वद्रूपसमुत्थित-\\
मिश्रात् प्रक्षेपकरणेन~।}}}
\end{quote}

\noindent \textbf{उदाहरणम्~।}

\phantomsection \label{Ex 2.4}
\begin{quote}
\textbf{{\color{red}गृहित्वाष्ट क्रयेणैव शालितण्डुलखारिकाः~।\\
विक्रीताः पञ्चकेनात्र लाभः षष्टिर्धनं वद~॥}}
\end{quote}

\begin{sloppypar}
न्यासः~। क्रयः ८ विक्रयः ५ लाभः ६०~। अत्र \hyperref[2.1]{'मिश्रं भवेत् क्रयार्घ'}\textendash \,इत्यादिविधिना क्रयो मिश्रधनाख्यः~। विक्रयोनः क्रयो लाभः~। यदि लाभस्यास्य ३ मिश्रधनमूले ८।५ तदा षष्टेः ६० क इति जाते मिश्रमूलधने १६०।१००~॥
\vspace{3mm}

अथाज्ञातक्रये न्यासः~। क्रयः ८ विक्रयः ५ मिश्र-धनमूलधने १६०।१००~। अत्रानुपातः~। यदि विक्रयस्यास्य १०० अयं क्रयः १६० तदास्य ५ क इति जातः क्रयः ८~॥
\vspace{3mm}

अज्ञाते विक्रये न्यासः~। क्रयः ८ विक्रयः ० मिश्रधनमूलधने १६०।१००~। अत्रानुपातः~। यदि क्रयस्यास्य १६० विक्रयोऽयं १०० तदास्य ८ क इति जातो विक्रयः ५~॥\\
\end{sloppypar}

\noindent \textbf{अपि च~।}

\phantomsection \label{Ex 2.5}
\begin{quote}
\textbf{{\color{red}शालिगोधूमकुल्माषखार्यः सखे \\
रामबाणाद्रिसङ्ख्याः क्रयाः सप्ततिः~।\\
रूपहीना धनं विक्रया भूकरा-\\
ग्न्युन्मितास्तुल्यलाभं धनं किं पृथक्~।\\
लाभयुक्तानि तुल्यानि वित्तानि वा \\
लाभहीनानि वा स्युः कथं ब्रूहि मे~॥}}
\end{quote}

न्यासः \;{\small $\begin{matrix}
\mbox{{३~ ५~ ७}}\\
\mbox{{१~ २~ ३}}\\
\vspace{-1mm}
\mbox{{१~ १~ १}}
\vspace{1mm}
\end{matrix}$}\; मूलानां मिश्रधनम् ६९~। अत्रानुपाताद्रूपलाभ-

\newpage

\begin{sloppypar}
\noindent मूलानि \;{\small $\begin{matrix}
\mbox{{१~ २~ ३}}\\
\vspace{-1mm}
\mbox{{२~ ३~ ४}}
\vspace{1mm}
\end{matrix}$}\; प्रक्षेपकरणेन जातानि मूलधनानि १८।२४।२७~। समलाभाः ३६~।\\

अथ द्वितीयोदाहरणे सलाभमूलधने आलापिते सलाभमूलधनानि \;{\small $\begin{matrix}
\mbox{{१~ २~ ३}}\\
\vspace{-1mm}
\mbox{{३~ ५~ ७}}
\vspace{1mm}
\end{matrix}$}\; प्रक्षेप-करणेन जातानि मूलधनानि \;{\small $\begin{matrix}
\mbox{{१९~~ २३~~ ~२५}}\\
\mbox{{९७~~ ९२~~ ~५५}}\\
\vspace{-1mm}
\mbox{{१२२~ १२२~ १२२}}
\vspace{1mm}
\end{matrix}$}~। तृतीयोदाहरणे विलाभमूलधने आलापिते तदुदाहरणम्~। \\

\noindent \textbf{अपि च~।}

\phantomsection \label{Ex 2.6}
\begin{quote}
\textbf{{\color{red}रसाद्रीभाः क्रया बाणरसशैलाश्च विक्रयाः~।\\
मिश्रेऽत्यष्ट्यश्विनो मूलं विलाभं सदृशं कथम्~॥}}
\end{quote}

न्यासः~। {\small $\begin{matrix}
\mbox{{६~ ७~ ८}}\\
\vspace{-1mm}
\mbox{{५~ ६~ ७}}
\vspace{1mm}
\end{matrix}$}\; मूलानां मिश्रधनम् २१७~।\\

अत्रानुपाताद्विलाभरूपाज्जातानि मूलधनानि ७५।७२।७०~। लाभोनमूलधनं समधनम् ६०~।\\

\noindent \textbf{सूत्रम्~।}

\phantomsection \label{2.3}
\begin{quote}
\renewcommand{\thefootnote}{$\star$}\footnote{{\color{violet}पण्यैः स्वमूल्यानि भजेदि}त्यादि {\color{violet}भास्करो}क्तानुरूपमेवेदम्~।}{\large \textbf{{\color{purple}निजभागहते मूल्ये \\
स्वपण्यभक्ते विधिः प्राग्वत्~॥~३~॥}}}
\end{quote}

\noindent \textbf{उदाहरणम्~।}

\phantomsection \label{Ex 2.7}
\begin{quote}
\textbf{{\color{red}द्रम्मेण पादयुतमेकपलं च शुण्ठ्या \\
द्वाभ्यां त्रिभागयुतमेकपलं च हिङ्गोः~।\\
द्रम्मैस्त्रिभिर्मगधिकापलमेकमाढ्यं\renewcommand{\thefootnote}{१}\footnote{The reading -माद्यं seems to be a typographical error.} \\
पञ्चांशकेन वद किं दशभिः समं मे~॥}}
\end{quote}

न्यासः~। {\small $\begin{matrix}
\mbox{{१~ १~ १}}\\
\vspace{-1mm}
\mbox{{१~ २~ ३}}
\vspace{1mm}
\end{matrix}$}\; मिश्रम् १०~। रूपसमभागमौल्यानि \;{\small $\begin{matrix}
\mbox{{४~ ३~ ५}}\\
\vspace{-1mm}
\mbox{{५~ २~ २}}
\vspace{1mm}
\end{matrix}$}~। अतः प्राग्वत् प्रक्षेप-करणेन जातानि शुण्ठ्यादीनां मौल्यानि \;{\small $\begin{matrix}
\mbox{{५~ २५~ १२५}}\\
\vspace{-1mm}
\mbox{{३~ ~८~ ~~२४}}
\vspace{1mm}
\end{matrix}$}~।
\end{sloppypar}

\newpage

\begin{sloppypar}
\noindent शुण्ठ्यादीनां समपलानि \;{\small $\begin{matrix}
\mbox{{२५}}\\
\vspace{-1mm}
\mbox{{१२}}
\vspace{1mm}
\end{matrix}$}~।\\

\noindent \textbf{अपि च~।}

\phantomsection \label{Ex 2.8}
\begin{quote}
\textbf{{\color{red}निष्केण चन्दनपलद्वयमाप्यते चेत् \\
द्वाभ्यां च कुङ्कुमपलं शरसम्मितं च~।\\
ईशार्चनाय मम देहि सुचन्दनं च \\
सत्कुङ्कुमं द्विगुणितं दशभिश्च निष्कैः~॥}}
\end{quote}

न्यासः~। {\small $\begin{matrix}
\mbox{{१~ २}}\\
\mbox{{१~ १}}\\
\mbox{{२~ ५}}\\
\vspace{-1mm}
\mbox{{१~ १}}
\vspace{1mm}
\end{matrix}$}\; मिश्रधनम् १०~।\\

एकद्विरूपभागमूल्यम् \;{\small $\begin{matrix}
\mbox{{१~ ४}}\\
\vspace{-1mm}
\mbox{{२~ ५}}
\vspace{1mm}
\end{matrix}$}~। प्रक्षेपकरणेन जाते चन्दनकुङ्कुममूल्ये \;{\small $\begin{matrix}
\mbox{{५०~ ८०}}\\
\vspace{-1mm}
\mbox{{१३~ १३}}
\vspace{1mm}
\end{matrix}$}\; अतः चन्दनकुङ्कुमपलमाने \;{\small $\begin{matrix}
\mbox{{१००~ २००}}\\
\vspace{-1mm}
\mbox{{~१३~~ ~१३~}}
\vspace{1mm}
\end{matrix}$}~।\\
\vspace{2mm}

\noindent \textbf{अपि च~।}

\phantomsection \label{Ex 2.9}
\begin{quote}
\textbf{{\color{red}द्वित्रिचतुर्भिर्निष्कैस्त्रिचतुःपञ्चार्घकेण च क्रीत्वा~।\\
चणगोधूमतिलानां तुल्यास्ते राशयश्च पुनः~॥ \\
पञ्चाङ्गनगप्रमितैर्निष्कैरङ्गागवसुमितार्घेण~।\\
विक्रीतो लाभानां योगे सप्तोनिता त्रिशती~॥ \\
तत्र पृथक्त्वाल्लाभश्चणकादीनां पृथक् च मूलधनम्~।\\
वद यदि गणिताहङ्कृतिरस्ति कृतिन् कौशलं तव चेत्~॥}}
\end{quote}

न्यासः~।\, {\small $\begin{matrix}
\mbox{{१~ १~ १}}\\
\mbox{{२~ ३~ ४}}\\
\mbox{{३~ ४~ ५}}\\
\mbox{{५~ ६~ ७}}\\
\vspace{-1mm}
\mbox{{६~ ७~ ८}}
\vspace{1mm}
\end{matrix}$}\; लाभमिश्रम् २९३~।\\

अत्रापि \hyperref[2.3]{'निजभागहते मूल्ये'} इत्यादिना रूपसमराशीनां सलाभमूलधनानि\, $\frac{{\footnotesize{\hbox{५}}}}{{\footnotesize{\hbox{६}}}}\, \frac{{\footnotesize{\hbox{६}}}}{{\footnotesize{\hbox{७}}}}\, \frac{{\footnotesize{\hbox{७}}}}{{\footnotesize{\hbox{८}}}}$~। एभ्यः पूर्वमूलान्यपास्य जाताः पृथग्लाभाः
\end{sloppypar}

\newpage

\begin{sloppypar}
\noindent $\frac{{\footnotesize{\hbox{१}}}}{{\footnotesize{\hbox{६}}}}\, \frac{{\footnotesize{\hbox{३}}}}{{\footnotesize{\hbox{२८}}}}\, \frac{{\footnotesize{\hbox{३}}}}{{\footnotesize{\hbox{४०}}}}$~। अत्र प्रक्षेपकरणेन जाताः पृथग्लाभाः १४०।९०।६३~। अत्र त्रैराशिकम्~। यदि लाभयोगस्यास्य\, $\frac{{\footnotesize{\hbox{२९३}}}}{{\footnotesize{\hbox{८४०}}}}$\, एतानि मूलधनानि\, $\frac{{\footnotesize{\hbox{२}}}}{{\footnotesize{\hbox{३}}}}\, \frac{{\footnotesize{\hbox{३}}}}{{\footnotesize{\hbox{४}}}}\, \frac{{\footnotesize{\hbox{४}}}}{{\footnotesize{\hbox{५}}}}$\, सलाभानि\, $\frac{{\footnotesize{\hbox{५}}}}{{\footnotesize{\hbox{६}}}}\, \frac{{\footnotesize{\hbox{६}}}}{{\footnotesize{\hbox{७}}}}\, \frac{{\footnotesize{\hbox{७}}}}{{\footnotesize{\hbox{८}}}}$\, तदा लाभयोगस्य २९३ कानीति जातानि मूलधनानि ५६०।६३०।६७२~। सलाभानि च ७००।७२०।७३५~।\\

\noindent \textbf{अपि च~।}

\phantomsection \label{Ex 2.10}
\begin{quote}
\textbf{{\color{red}द्रम्मेण दाडिमफलं शतपञ्चकोनं \\
विद्वन् पणेन सहकारफलानि पञ्च~।\\
लब्धानि यैर्वद फलानि पणैश्च तेषां \\
योगेन सप्तरहितं च शतं कथं स्यात्~॥}}
\end{quote}

न्यासः \;{\small $\begin{matrix}
\mbox{{१२~ १}}\\
\vspace{-1mm}
\mbox{{९५~ ५}}
\vspace{1mm}
\end{matrix}$}\; मिश्रम् ९३~। अत्रानुपातेन सममौल्यफलानि ज्ञेयानि~। तद्यथा~। यद्येकेन पणेन पञ्च सहकारफलानि तदा पणद्वादशकेन किमिति जाते पणद्वादशफलमाने ९५।६०~। प्रक्षेपकरणेन जाते सहकारदाडिमफलमाने ३६।५७~। अथवा द्वादशपणानां दाडिमानां पञ्चोनशतं तदैकस्य पणस्य किमिति जाते एकपणफलमाने\, $\frac{{\footnotesize{\hbox{९५}}}}{{\footnotesize{\hbox{१२}}}}\, \frac{{\footnotesize{\hbox{५}}}}{{\footnotesize{\hbox{१}}}}$~। प्रक्षेपकरणेन त एव दाडिमसहकारफलमाने ५७।३६~।\\

\noindent \textbf{अपि च~।}

\phantomsection \label{Ex 2.11}
\begin{quote}
\textbf{{\color{red}लभ्यन्ते नवकेकिनोऽलसमणिप्रौढैः सखे पञ्चभिः\\
हंसैर्हंसशरांशकैस्त्रिभिरपि प्राप्ताश्च ये केकिनः~।\\
तेषां शिष्टमरालबालवयसां योगाश्चतुर्वर्जिता\\
जाता मे त्रिशती वदाशु कति ते हंसा मयूराः पृथक्~॥}}
\end{quote}

\end{sloppypar}

\newpage

\begin{sloppypar}
न्यासः~।\, {\small $\begin{matrix}
\mbox{{हं\, ५}}\\
\vspace{-1mm}
\mbox{{म\, ९}}
\vspace{1mm}
\end{matrix}$}\; हंसानां मिश्रम् १९६~।\\

अत्र त्रैराशिकम्~। यदि पञ्चभिर्मरालैर्नव केकिनो लभ्यन्ते तदा पञ्चमांशैस्त्रिभिः कतीति न्यासः\, ५, ९, $\frac{{\footnotesize{\hbox{३}}}}{{\footnotesize{\hbox{५}}}}$\, लब्धम्\, $\frac{{\footnotesize{\hbox{२७}}}}{{\footnotesize{\hbox{२५}}}}$~। पञ्चमांशत्रयं रूपादपास्य शेषम्\, $\frac{{\footnotesize{\hbox{२}}}}{{\footnotesize{\hbox{५}}}}$~। अनेन लब्धमेतद्युतं जातम्\, $\frac{{\footnotesize{\hbox{३७}}}}{{\footnotesize{\hbox{२५}}}}$~। यद्यनेन मिश्रेण\, $\frac{{\footnotesize{\hbox{३७}}}}{{\footnotesize{\hbox{२५}}}}$\, हंसप्रमाणं रूपं १ तदादिष्टमिश्रेण २९६ किमिति जातं हंसयूथमानम् २००~।

\begin{center}
\textbf{इति क्रयविक्रयौ~।}\\
\vspace{6mm}

{\large \textbf{अथ वृद्धिधनम्~।}}
\end{center}

\noindent \textbf{तत्र रूपोत्थितोद्देशः~।}

\phantomsection \label{Ex 2.12}
\begin{quote}
\textbf{{\color{red}शतेन मासेन च पञ्चकेन वर्षेण साष्टं शतमेव जातम्~।\\
सवृद्धिमूलं वद किं पृथग्मे कलान्तरं मूलधनं च मित्र~॥}}
\end{quote}

न्यासः~। {\small $\begin{matrix}
\mbox{{~~१~~ ~१२}}\\
\mbox{{१००~~ १}}\\
\vspace{-1mm}
\mbox{{५ ~~~}}
\vspace{1mm}
\end{matrix}$}\; मिश्रम् १०८~। अत्र रूपं मूलं कृत्वा जाते मूलकलान्तरे\, $\frac{{\footnotesize{\hbox{१}}}}{{\footnotesize{\hbox{२}}}}\, \frac{{\footnotesize{\hbox{३}}}}{{\footnotesize{\hbox{५}}}}$~। प्रक्षेपकरणेन जाते मूलकलान्तरे\, $\frac{{\footnotesize{\hbox{१३५}}}}{{\footnotesize{\hbox{२}}}}\, \frac{{\footnotesize{\hbox{८१}}}}{{\footnotesize{\hbox{२}}}}$~।\\
\vspace{2mm}

\noindent \textbf{अपि च~।}

\phantomsection \label{Ex 2.13}
\begin{quote}
\textbf{{\color{red}मासेन निष्कयुगलेन धनं गृहीत्वा \\
दत्तं हि पञ्चकशतेन पुनस्तदेव~।\\
लाभस्तु पञ्च नवसङ्गुणिताः प्रजातो \\
मासैः सखे वद धनं दशभिश्च किं स्यात्~॥}}
\end{quote}

न्यासः~। \begin{small}\begin{tabular}{c|c}
~~१~ ~~१० & ~~१~ ~~१० \\
१००~~ १ & १००~~ १ \\
२~~~~~ & ५~~~~~
\end{tabular}\end{small} लाभयोगः ४५~। रूपमूलधनलाभः\, $\frac{{\footnotesize{\hbox{३}}}}{{\footnotesize{\hbox{१०}}}}$

\end{sloppypar}

\newpage

\begin{sloppypar}
\noindent अतोऽनुपातः~। यद्यस्य लाभस्येदं मूलधनं रूपं तदोद्दिष्टलाभस्य किमिति न्यासः\, $\frac{{\footnotesize{\hbox{३}}}}{{\footnotesize{\hbox{१०}}}}$।१।४५~। जातं मूलधनम् १५०। अतः पञ्चराशिकेन जाते कलान्तरे ३०।७५~।\\

\noindent \textbf{अपि च~।}

\phantomsection \label{Ex 2.14}
\begin{quote}
\textbf{{\color{red}दत्तं द्विकत्रिकचतुष्कफलेन वित्तं \\
खण्डैस्त्रिभिश्च शतमाशु वदार्यवर्य~।\\
मासेषु पञ्चगजदिक्-प्रमितेषु तस्मिन् \\
खण्डत्रयेऽपि च फलं सदृशं कथं स्यात्~।\\
यद्वा फलेन सहितं निजखण्डवित्तं \\
तुल्यं भवेदपि फलोनधनं समं वा~॥}}
\end{quote}

प्रथमोदाहरणे समफलमालापितं रूपं फलं प्रकल्प्य न्यासः\\ 
\begin{small}\begin{tabular}{c|c|c|}
~~१~ ~~~५ & ~~१~ ~~~८ & ~~१~ ~~~१० \\
१००~~  & १००~~  & १००~~~~  \\
~~२~~ ~१ & ~~३~~ ~१ & ~~४~~ ~१~
\end{tabular}\end{small} \,अत्र रूपफलानां मूलधनानि\, $\frac{{\footnotesize{\hbox{१०}}}}{{\footnotesize{\hbox{१}}}}$~। $\frac{{\footnotesize{\hbox{२५}}}}{{\footnotesize{\hbox{६}}}}$~। $\frac{{\footnotesize{\hbox{५}}}}{{\footnotesize{\hbox{२}}}}$~।\\
\vspace{2mm}

प्रक्षेपकरणेन जातानि मूलधनानि ६०।२५।१५~। समकलान्तरम् ६~। \\

अथ द्वितीयोदाहरणे सफलं सममूलमालापितं तत्र रूपं सममूलं प्रकल्प्य न्यासः \begin{small}\begin{tabular}{c|c|c|}
~~१~~ ~~५ & ~~१~~ ~~८ & ~~१~~ ~~१० \\
१००~~ १ & १००~~ १ & १००~~ १ \\
२~~~~~  & ३~~~~~ & ४~~~~~
\end{tabular}\end{small}\, मिश्रधनम् १००~। \\
\vspace{2mm}

जातानि रूपमूलधनस्य
कलान्तराणि\, $\frac{{\footnotesize{\hbox{११}}}}{{\footnotesize{\hbox{१०}}}}$~। $\frac{{\footnotesize{\hbox{३१}}}}{{\footnotesize{\hbox{१५}}}}$~। $\frac{{\footnotesize{\hbox{७}}}}{{\footnotesize{\hbox{५}}}}$~। यदि सकलान्तरस्य\, $\frac{{\footnotesize{\hbox{११}}}}{{\footnotesize{\hbox{१०}}}}$\, मूलधनं १ तदा रूपस्य किमिति जातं प्रथममूलधनम्\, $\frac{{\footnotesize{\hbox{१०}}}}{{\footnotesize{\hbox{११}}}}$~। एवमन्ययोः\, $\frac{{\footnotesize{\hbox{२५}}}}{{\footnotesize{\hbox{३१}}}}$~। $\frac{{\footnotesize{\hbox{५}}}}{{\footnotesize{\hbox{७}}}}$~। प्रक्षेपकरणेन जातानि मूलधनानि\, $\frac{{\footnotesize{\hbox{२१७०}}}}{{\footnotesize{\hbox{५८}}}}$~। $\frac{{\footnotesize{\hbox{१९२५}}}}{{\footnotesize{\hbox{५८}}}}$~। $\frac{{\footnotesize{\hbox{१७०५}}}}{{\footnotesize{\hbox{५८}}}}$~।
\end{sloppypar}

\newpage

\begin{sloppypar}
\noindent एभ्योऽनुपातेन कलान्तराणि\, $\frac{{\footnotesize{\hbox{२१७}}}}{{\footnotesize{\hbox{५८}}}}$~। $\frac{{\footnotesize{\hbox{४६२}}}}{{\footnotesize{\hbox{५८}}}}$~। $\frac{{\footnotesize{\hbox{६८२}}}}{{\footnotesize{\hbox{५८}}}}$~। सकलान्तरं सममूलधनम् $\frac{{\footnotesize{\hbox{२३८७}}}}{{\footnotesize{\hbox{५८}}}}$~। \\

तृतीयोदाहरणे विकलान्तराणि समधनानि न्यासः~। \begin{small}\begin{tabular}{c|c|c|}
~१~~ ~~५ & ~१~~ ~~८ & ~~१~~ ~~१० \\
१००~~~~  & १००~~~~  & १००~~~~~  \\
२~~~~~  & ३~~~~~ & ४~~~~~
\end{tabular}\end{small}\, मिश्रधनम् १००~। अत्र प्राग्वद्रूपमूलानि\, $\frac{{\footnotesize{\hbox{१}}}}{{\footnotesize{\hbox{१०}}}}$~। $\frac{{\footnotesize{\hbox{६}}}}{{\footnotesize{\hbox{२५}}}}$~। $\frac{{\footnotesize{\hbox{२}}}}{{\footnotesize{\hbox{५}}}}$~।\; एतानि पृथग्रूपादपास्य शेषम्\, $\frac{{\footnotesize{\hbox{९}}}}{{\footnotesize{\hbox{१०}}}}$~। $\frac{{\footnotesize{\hbox{१९}}}}{{\footnotesize{\hbox{२५}}}}$~। $\frac{{\footnotesize{\hbox{३}}}}{{\footnotesize{\hbox{५}}}}$~।\; त्रैराशिकम्~। यदि शेषस्यास्य\, $\frac{{\footnotesize{\hbox{९}}}}{{\footnotesize{\hbox{१०}}}}$\, मूलधनं रूपं तदा रूपस्य किमिति जातं मूलधनम्\, $\frac{{\footnotesize{\hbox{१०}}}}{{\footnotesize{\hbox{९}}}}$~। एवमन्ययोः\, $\frac{{\footnotesize{\hbox{२५}}}}{{\footnotesize{\hbox{१९}}}}$~। $\frac{{\footnotesize{\hbox{५}}}}{{\footnotesize{\hbox{३}}}}$~।\, प्रक्षेपकरणाज्जातानि मूलधनानि\, $\frac{{\footnotesize{\hbox{१९०}}}}{{\footnotesize{\hbox{७}}}}$~। $\frac{{\footnotesize{\hbox{२२५}}}}{{\footnotesize{\hbox{७}}}}$~। $\frac{{\footnotesize{\hbox{२८५}}}}{{\footnotesize{\hbox{७}}}}$~।\, एभ्योऽनुपातेन कलान्तराणि\, $\frac{{\footnotesize{\hbox{१९}}}}{{\footnotesize{\hbox{७}}}}$~। $\frac{{\footnotesize{\hbox{५४}}}}{{\footnotesize{\hbox{७}}}}$~। $\frac{{\footnotesize{\hbox{११४}}}}{{\footnotesize{\hbox{७}}}}$~।\, फलोनसमधनम्\, $\frac{{\footnotesize{\hbox{१७१}}}}{{\footnotesize{\hbox{७}}}}$~।\\
\vspace{2mm}
 
\noindent \textbf{अपि च~।}

\phantomsection \label{Ex 2.15}
\begin{quote}
\textbf{{\color{red}दत्तं धनं धनवता कियदाद्यवर्षे \\
विप्राय केनचिदपि द्विगुणं च विद्वन्~।\\
वर्षं प्रति प्रवद पञ्चभिरेव वर्षैः \\
जातं शतं च षडशीत्यधिकं किमाद्यम्~॥}}
\end{quote}

अत्र प्रथमवर्षे क्रियावतरणार्थं रूपमेव कल्पितं परेषु यथोत्तरं द्विगुणम्~। न्यासः १।२।
४।८।१६~। मिश्रधनम् १८६~। प्रक्षेपकरणेन जातानि धनानि ६।१२।२४।४८।९६~।\\

\noindent \textbf{सूत्रम्~।}

\phantomsection \label{2.4.1}
\begin{quote}
{\large \textbf{{\color{purple}इष्टकलान्तरहीना-\\
धिकेष्टभक्तेष्टमिह भवेन्नीवी~।}}}
\end{quote}

\end{sloppypar}

\newpage

\phantomsection \label{2.4}
\begin{quote}
\renewcommand{\thefootnote}{१}\footnote{अत्रोपपत्तिः~। कल्प्यन्ते प्रथमप्रमाणमास-प्रमाणधन-प्रमाणफल-नियतमासाः कमेण प्रमा, प्रध, प्रफ, निमा~। एवं द्वितीयप्रमाणमासप्रमाणधनप्रमाणफलनियतमासाः क्रमेण प्र'मा, प्र'ध, प्र'फ, नि'मा~। प्रथममूल-धनप्रमाणम् $=$ या, द्वितीयमूलधनप्रमाणम् $=$ का~।
\vspace{2mm}

\hspace{2mm} तदा प्रश्नानुसारेण~~ ${\hbox{का}} \pm \dfrac{{\footnotesize{\hbox{प्रफ.या}} \times {\hbox{निमा}}}}{{\footnotesize{\hbox{प्रध}} \times {\hbox{प्रमा}}}} = {\hbox{या}} \pm \dfrac{{\footnotesize{\hbox{प्र'फ.का}} \times {\hbox{नि'मा}}}}{{\footnotesize{\hbox{प्र'ध}} \times {\hbox{प्र'मा}}}}$ 
\vspace{2mm}

\hspace{2mm} पक्षान्तरानयनेन~~ ${\hbox{का}} \left({\hbox{१}} \mp \dfrac{{\footnotesize{\hbox{प्र'फ.नि'मा}}}}{{\footnotesize{\hbox{प्र'ध.प्र'मा}}}} \right) = {\hbox{या}} \left({\hbox{१}} \mp \dfrac{{\footnotesize{\hbox{प्रफ.निमा}}}}{{\footnotesize{\hbox{प्रध.प्रमा}}}} \right)$ 
\vspace{2mm}

\hspace{2mm} वा इष्टवर्गेण पक्षौ सङ्गुण्य
\vspace{1mm}

\hspace{13mm} ${\hbox{इ.का}} \left({\hbox{इ}} \mp \dfrac{{\footnotesize{\hbox{प्र'फ.इ}}}}{{\footnotesize{\hbox{प्र'ध}}}} \times \dfrac{{\footnotesize{\hbox{नि'मा}}}}{{\footnotesize{\hbox{प्र'मा}}}} \right) = {\hbox{इ.या}} \left({\hbox{इ}} \mp \dfrac{{\footnotesize{\hbox{प्रफ.इ}}}}{{\footnotesize{\hbox{प्रध}}}} \times \dfrac{{\footnotesize{\hbox{निमा}}}}{{\footnotesize{\hbox{प्रमा}}}} \right)$ 
\vspace{2mm}

\hspace{8mm} वा, \hspace{7mm} $\dfrac{{\footnotesize{\hbox{इ}}}}{{\hbox{इ}} \mp \dfrac{{\footnotesize{\hbox{प्रफ.इ}}}}{{\footnotesize{\hbox{प्रध}}}} \times \dfrac{{\footnotesize{\hbox{निमा}}}}{{\footnotesize{\hbox{प्रमा}}}}}\, {\hbox{का}} = \dfrac{{\footnotesize{\hbox{इ}}}}{{\hbox{इ}} \mp \dfrac{{\footnotesize{\hbox{प्र'फ.इ}}}}{{\footnotesize{\hbox{प्र'ध}}}} \times \dfrac{{\footnotesize{\hbox{नि'मा}}}}{{\footnotesize{\hbox{प्र'मा}}}}}\, {\hbox{या}}$
\vspace{2mm}

\hspace{2mm} अत्र यदि~~ ${\hbox{या}} = \dfrac{{\footnotesize{\hbox{इ}}}}{{\hbox{इ}} \mp \dfrac{{\footnotesize{\hbox{प्रफ.इ}}}}{{\footnotesize{\hbox{प्रध}}}} \times \dfrac{{\footnotesize{\hbox{निमा}}}}{{\footnotesize{\hbox{प्रमा}}}}}\;,~~ {\hbox{का}} = \dfrac{{\footnotesize{\hbox{इ}}}}{{\hbox{इ}} \mp \dfrac{{\footnotesize{\hbox{प्र'फ.इ}}}}{{\footnotesize{\hbox{प्र'ध}}}} \times \dfrac{{\footnotesize{\hbox{नि'मा}}}}{{\footnotesize{\hbox{प्र'मा}}}}}$
\vspace{2mm}

\hspace{2mm} तदा पक्षद्वयसाम्यं स्यात्~। अत उपपद्यते~। कलान्तरानयनमतिसुगममिति~।
\vspace{2mm}
}{\large \textbf{{\color{purple}इष्टान्तरिता नीवी \\
कलान्तरं जायते नियतम्~॥~४~॥ \\
अन्योऽन्यकलान्तरयुत-\\
हीने निव्यौ समे भवतः~।}}}
\end{quote}

\noindent \textbf{उदाहरणम्~।}

\phantomsection \label{Ex 2.16}
\begin{quote}
\textbf{{\color{red}मासेन पञ्चकचतुष्कशतेन दत्तं \\
द्वैधं हि मासदशसप्त\renewcommand{\thefootnote}{२}\footnote{'नगपङ्क्ति' इति पाठः साधुः~।}समुद्भवाभ्याम्~।\\
अन्योऽन्यसम्मिलितमेव कलान्तराभ्यां \\
तुल्यं धनं भवति चापि विवर्जिताभ्याम्~॥}}
\end{quote}

\newpage

\begin{sloppypar}
प्रथमोदाहरणस्य न्यासः~। \begin{small}\begin{tabular}{c|c|}
~~१~ ~~~७ & ~~१~ ~~१० \\
१००~~ १ & १००~~ १ \\
५~~~~~ & ४~~~~~
\end{tabular}\end{small}\; अत्र करणमिष्टम् १~। अस्य कला-न्तरम्\, $\frac{{\footnotesize{\hbox{७}}}}{{\footnotesize{\hbox{२०}}}}$~। $\frac{{\footnotesize{\hbox{२}}}}{{\footnotesize{\hbox{५}}}}$~। \hyperref[2.4.1]{'हीनाधिकेष्ट-'} इति, इष्टाद्रूपादपास्य शेषे\, $\frac{{\footnotesize{\hbox{१३}}}}{{\footnotesize{\hbox{२०}}}}$~। $\frac{{\footnotesize{\hbox{३}}}}{{\footnotesize{\hbox{५}}}}$~। आभ्यां पृथगिष्टे भक्ते जाते मूलधने\, $\frac{{\footnotesize{\hbox{२०}}}}{{\footnotesize{\hbox{१३}}}}$~। $\frac{{\footnotesize{\hbox{५}}}}{{\footnotesize{\hbox{३}}}}$~। एते पृथग्रूपान्तरिते जाते कलान्तरे\, $\frac{{\footnotesize{\hbox{७}}}}{{\footnotesize{\hbox{२०}}}}$~। $\frac{{\footnotesize{\hbox{२}}}}{{\footnotesize{\hbox{५}}}}$~। अभिन्नार्थम् एकोनचत्वारिंशता गुणिते जाते मूलधने ६०।६५ कलान्तरे च २६।२१~। समधने ८६।८६~।\\

द्वितीयोदाहरणे अन्योऽन्यकलान्तरवर्जिते समे मूलधने आलापिते तदर्थं न्यासः~।
\vspace{2mm}

\noindent \begin{small}\begin{tabular}{c|c|}
~~१~ ~~~७ & ~~१~ ~~१० \\
१००~~ १ & १००~~ १ \\
५~~~~~ & ४~~~~~
\end{tabular}\end{small}\; अत्र करणम्~। प्राग्वत् कलान्तरे\, $\frac{{\footnotesize{\hbox{७}}}}{{\footnotesize{\hbox{२०}}}}$~। $\frac{{\footnotesize{\hbox{२}}}}{{\footnotesize{\hbox{५}}}}$~। \hyperref[2.4.1]{'हीनाधिकेष्ट-'} इति रूपाधिके\, $\frac{{\footnotesize{\hbox{२७}}}}{{\footnotesize{\hbox{२०}}}}$~। $\frac{{\footnotesize{\hbox{७}}}}{{\footnotesize{\hbox{५}}}}$~। आभ्यां पृथगिष्टे हृते जाते मूलधने\, $\frac{{\footnotesize{\hbox{२०}}}}{{\footnotesize{\hbox{२७}}}}$~। $\frac{{\footnotesize{\hbox{५}}}}{{\footnotesize{\hbox{७}}}}$~। एते रूपान्तरिते जाते कलान्तरे\, $\frac{{\footnotesize{\hbox{७}}}}{{\footnotesize{\hbox{२७}}}}$~। $\frac{{\footnotesize{\hbox{२}}}}{{\footnotesize{\hbox{७}}}}$~। अन्योऽन्यकलान्तरवर्जिते मूलधने समे\, $\frac{{\footnotesize{\hbox{८६}}}}{{\footnotesize{\hbox{१८९}}}}$~। $\frac{{\footnotesize{\hbox{८६}}}}{{\footnotesize{\hbox{१८९}}}}$~। अभिन्नार्थम् एकोननवत्यधिकशतगुणिते जाते अभिन्ने १४०।१३५~। कलान्तरे ५४।४९~। समधने ८६।८६~।\\
\end{sloppypar}

\noindent \textbf{सूत्रम्~।}

\phantomsection \label{2.5}
\begin{quote}
{\large \textbf{{\color{purple}विहिते विपरीतफले \\
पक्षद्वितये मिथस्तु फलयोर्वा~॥~५~॥}}}
\end{quote}

\newpage

\phantomsection \label{2.6}
\begin{quote}
\renewcommand{\thefootnote}{१}\footnote{अत्रोपपत्तिः~। कल्प्यन्ते प्रमाणधन-प्रमाणकाल-प्रमाणफल-इच्छामास-इच्छाधनमानानि क्रमेण प्रध, प्रमा, प्रफ, इमा, इध, तदा त्रैराशिकद्वितयेन इच्छाधनस्य फलम् $= {\hbox{इफ}} = \dfrac{{\footnotesize{\hbox{प्रफ.इध.इमा}}}}{{\footnotesize{\hbox{प्रध}} \times {\hbox{प्रमा}}}}$
\vspace{1mm}

\hspace{2mm} अतः प्रश्नालापानुसारेण
\vspace{1mm}

\hspace{2mm} इफ $+$ प्रफ $=$ मिश्र $= \dfrac{{\footnotesize{\hbox{प्रफ.इध.इमा}} + {\hbox{प्रफ.प्रध.प्रमा}}}}{{\footnotesize{\hbox{प्रध.प्रमा}}}}$
\vspace{2mm}

\hspace{24mm} $= \dfrac{{\footnotesize{\hbox{प्रफ}}\,({\hbox{इध.इमा}} + {\hbox{प्रध.प्रमा}})}}{{\footnotesize{\hbox{प्रध.प्रमा}}}}$
\vspace{2mm}

\hspace{5mm} $\therefore$\; प्रफ $= \dfrac{{\footnotesize{\hbox{मि}}\,({\hbox{प्रध.प्रमा}})}}{{\footnotesize{\hbox{इध.इमा}} + {\hbox{प्रध.प्रमा}}}}$
\vspace{2mm}

\hspace{2mm} तथा~~ इफ $= \dfrac{{\footnotesize{\hbox{मि}}\,({\hbox{इध.इमा}})}}{{\footnotesize{\hbox{इध.इमा}} + {\hbox{प्रध.प्रमा}}}}$
\vspace{2mm}

\hspace{2mm} एवमन्यान्यपि भवन्तीति सर्वमुपपद्यते~।}{\large \textbf{{\color{purple}धनयोश्च कालयोर्वा \\
धनफलयोः कालफलयोर्वा~।\\
मूलधनकालयोर्वा \\
मिश्रं यदि दृश्यते तदा तत्र~॥~६~॥ \\
तत्पक्षयोश्च घातौ \\
ताभ्यां प्रक्षेपतो जाती~।}}}
\end{quote}

\noindent \textbf{उदाहरणम्~।}

\phantomsection \label{Ex 2.17}
\begin{quote}
\textbf{{\color{red}मासेन शतस्य कियत् षष्टेर्वर्षस्य यत्फलं फलयोः~।\\
योगे चत्वारिंशद्रूपयुतं मे फलं कथय~। \\
ताभ्यां पक्षद्वितये मिथो विमिश्रे पृथक्कृते ब्रूहि~॥}}
\end{quote}

न्यासः \begin{small}\begin{tabular}{c|c|}
~~१~~ ~~१२ & ~१ ~~१२ \\
१००~~ ६० & ०~~~ ० \\
~०~~~~ ० & ~५~~ ३६
\end{tabular}\end{small}\; मिश्रधनम् ४१~।\\

अत्र \hyperref[2.5]{'विहिते विपरीतफले-'} इति पक्षयोर्घातौ १००।७२०~। प्रक्षेपकरणेन जाते कलान्तरे ५।३६~।

\newpage

धनयोर्न्यासः \begin{small}\begin{tabular}{c|c}
१ & १२ \\
० & ~० \\
५ & ३६
\end{tabular}\end{small}\; मिश्रम् १६०~। जाते धने १००।६०~।\\
\vspace{2mm}

कालयोगे न्यासः \begin{small}\begin{tabular}{c|c}
~०~ & १२ \\
१०० & ~० \\
~५~ & ३६
\end{tabular}\end{small}\; मिश्रम् ६१~। जाते कालधने १।६०~।\\

\noindent \textbf{सूत्रम्~।}

\phantomsection \label{2.7}
\begin{quote}
\renewcommand{\thefootnote}{१}\footnote{अत्रोपपत्तिः~। पूर्ववदत्रापि सङ्केतेन
\vspace{2mm}

\hspace{7mm} इध $= \dfrac{{\footnotesize{\hbox{प्रमा.इफ.प्रध}}}}{{\footnotesize{\hbox{इमा.प्रफ}}}}$
\vspace{2mm}

\hspace{2mm} प्रफ.इध $= \dfrac{{\footnotesize{\hbox{प्रमा.इफ.प्रध}}}}{{\footnotesize{\hbox{इमा}}}}$\; अयं वधो जातः~।
\vspace{2mm}

\hspace{2mm} योगस्तु प्रमाणफलेच्छाधनयोर्ज्ञात एव ततः प्राग्वदिच्छाधनम्~। एवमपरत्र च बहुराशिघातो भाज्यः परवधश्च हारो भवति लब्धिश्च तद्घातः~। इत्युपपद्यते सर्वम्~।}{\large \textbf{{\color{purple}विहिते विपरीतफले \\
पक्षद्वितये स्वपक्षफलधनयोः~॥~७~॥ \\
फलकालयोश्च धनका-\\
लयोर्यदा दृश्यते मिश्रम्~।\\
बहुराशिपक्षघाते \\
परवधभक्ते वधो विधिः प्राग्वत्~॥~८~॥}}}
\end{quote}

\noindent \textbf{उदाहरणम्~।}

\phantomsection \label{Ex 2.18}
\begin{quote}
\textbf{{\color{red}मासि शतस्य फलं यद्वर्षेण च षट्कृतिः फलं यस्य~। \\
तद्योगे पञ्चगुणास्त्रयोदश सखे पृथक् कथय~।\\
काले निजधनयुक्ते शतफलयुक्तेऽथवा गणितम्~॥}}
\end{quote}

न्यासः \begin{small}\begin{tabular}{c|c|}
~१~ & १२ \\
१०० & ~० \\
~०~ & ३६
\end{tabular}\end{small}\; मिश्रम् ६५~। अत्र \hyperref[2.7]{'विहिते विपरीतफले-'} इति बहुराशिघातः ३६०० अल्पराशिघातेन १२ भक्तो जातो घातः ३००~।

\newpage

\noindent योगः ६५~। \hyperref[1.35]{'योगकृतेश्च-'} इति जातमन्तरम् ५५~। \hyperref[1.31]{'योगो द्विष्ठ-'} इति जाते प्रमाणफलमूलधने ५।६०~।\\

प्रमाणफलकालयोगे न्यासः~। \begin{small}\begin{tabular}{c|}
~१~~ ~~० \\
१००~ ६० \\
~~०~~ ~३६
\end{tabular}\end{small}\; मिश्रम् १७~। प्राग्वज्जातो घातः ६०~। अतो जातौ प्रमाणकालौ ५।१२~। कालधनयोगे न्यासः \begin{small}\begin{tabular}{c|c|}
~१~ & ~० \\
१०० & ~० \\
~५~ & ३६
\end{tabular}\end{small}\; मिश्रम् ७२~। जातो घातः ७२०~। अतो जातौ मूलधनकालौ ६०।१२~।\\

\noindent \textbf{सूत्रम्~।}

\phantomsection \label{2.9}
\begin{quote}
\renewcommand{\thefootnote}{१}\footnote{अत्रोपपत्तिः~। अत्रापि पूर्ववत् सङ्केतेन~। प्रमाणफलसमेन इच्छाधनेन या कल्पनेन~।

\vspace{2mm}

\hspace{7mm} इफ $= \dfrac{{\footnotesize{\hbox{इमा.या}}^{\scriptsize{\hbox{२}}}}}{{\footnotesize{\hbox{प्रमा.प्रध}}}}$
\vspace{2mm}

\hspace{2mm} ततः~~ इफ $+$ या $=$ मिश्र $= \dfrac{{\footnotesize{\hbox{इमा.या}}^{\scriptsize{\hbox{२}}} + {\hbox{प्रमा.प्रध.या}}}}{{\footnotesize{\hbox{प्रमा.प्रध}}}}$
\vspace{2mm}

\hspace{2mm} वा~~ इमा.या$^{\scriptsize{\hbox{२}}} +$ प्रमा.प्रध.या $-$ मि.प्रमा.प्रध $=$ ०
\vspace{2mm}

\hspace{2mm} वा \hspace{4mm} या$^{\scriptsize{\hbox{२}}} + \dfrac{{\footnotesize{\hbox{प्रमा.प्रध}}}}{{\footnotesize{\hbox{इमा}}}}$\,या $- \dfrac{{\footnotesize{\hbox{मि.प्रमा.प्रध}}}}{{\footnotesize{\hbox{इमा}}}} =$ ०
\vspace{2mm}

\hspace{2mm} अत्र वर्गसमीकरणसिद्धान्तेन यावत्तावन्मानयोर्वधः
\vspace{2mm}

\hspace{4mm} $= - \dfrac{{\footnotesize{\hbox{मि.प्रमा.प्रध}}}}{{\footnotesize{\hbox{इमा}}}}$\, ऋणात्मकः~। अत एकमानमृणमन्यच्च धनमतस्तयोर्योगो धनर्णयोरन्तरमेव योग इत्यादिना तयोरन्तरम् $= \dfrac{{\footnotesize{\hbox{प्रमा.प्रध}}}}{{\footnotesize{\hbox{इमा}}}}$~। अतोऽन्तरवधाभ्यां यावत्तावन्मानद्वयं व्यक्तम्~। तत्र धनमानमेव मूलं धनं तत्समं प्रमाणफलं चेति~।}{\large \textbf{{\color{purple}निजकालगुणे मूले \\
परकालहृते भवेत् कयोर्वियुतिः~।\\
स द्वेधा मिश्रहता-\\
घातस्ताभ्यां तु धनमूलम्~॥~९~॥}}}
\end{quote}

\newpage

\begin{sloppypar}
\noindent \textbf{उदाहरणम्~।}

\phantomsection \label{Ex 2.19}
\begin{quote}
\textbf{{\color{red}पञ्चाशदुत्तरशतस्य कलान्तरं यत् \\
मासैश्चतुर्भिरपि तद्धनिना प्रदत्तम्~।\\
मासैस्त्रिभिर्वद सखे फलमूलयोगे \\
जातं धनं द्विगुणितेशमितं पृथक् किम्~॥}}
\end{quote}

न्यासः \begin{small}\begin{tabular}{c|}
~~४~~ ~३ \\
१५०~~ 
\end{tabular}\end{small}\; मिश्रम् २२~। अत्र धनं १५० निजकाल\textendash \,४\textendash \,हतं ६०० परकाल\textendash \,३\textendash \,हृतं २०० कयो राश्योरन्तरम्~। एतस्मिन् मिश्र\textendash \,२२\textendash \,हते जातो घातः ४४००~।~\hyperref[1.35]{'राश्योर्विवर-कृतियुताद्-'} इत्यादिना जातो योगः २४०~। सङ्क्रमेण जातौ राशी २२०।२०~। मूलोनिते कलान्तरं भवेदिति धनराशिरल्पो ग्राह्यः~।\\

\noindent \textbf{सूत्रम्~।}

\phantomsection \label{2.10}
\begin{quote}
\renewcommand{\thefootnote}{१}\footnote{अत्रोपपत्तिः~। कल्प्यते या मासो निर्मुक्तकालः~। तदा प्रश्नोक्त्या
\vspace{2mm}

\hspace{6mm} $\dfrac{{\footnotesize{\hbox{प्रफ.इध.या}}}}{{\footnotesize{\hbox{प्रध.प्रमा}}}} + {\hbox{इध}} = \dfrac{{\footnotesize{\hbox{इध}}\,({\hbox{प्रफ.या}} + {\hbox{प्रध.प्रमा}})}}{{\footnotesize{\hbox{प्रध.प्रमा}}}} = \dfrac{{\footnotesize{\hbox{स्कध.या}}}}{{\footnotesize{\hbox{स्कका}}}}$
\vspace{2mm}

\hspace{2mm} ततः~~ स्कध.प्रध.प्रमा.या $=$ स्कका.इध.प्रफ.या $+$ स्कका.इध.प्रध.प्रमा
\vspace{1mm}

\hspace{4mm} $\therefore$\; या\,(स्कध.प्रध.प्रमा $-$ स्कका.इध.प्रफ) $=$ स्कका.इध.प्रध.प्रमा
\vspace{2mm}

\hspace{2mm} या $= \dfrac{{\footnotesize{\hbox{स्कका.इध.प्रध.प्रमा}}}}{{\footnotesize{\hbox{स्कध.प्रध.प्रमा}} - {\hbox{स्कका.इध.प्रफ}}}} = \dfrac{{\footnotesize{\hbox{स्कका.इध}}}}{{\footnotesize{\hbox{स्कध}} -} \dfrac{{\footnotesize{\hbox{स्कका.इध.प्रफ}}}}{{\footnotesize{\hbox{प्रध.प्रमा}}}}}$
\vspace{1mm}

\hspace{2mm} अत उपपद्यते~।}{\large \textbf{{\color{purple}स्कन्धककालकलान्तर-\\
हीनस्कन्धेन भाजिते वित्ते~।\\
स्कन्धककालविगुणिते \\
नियतं निर्मुक्तकालः स्यात्~॥~१०~॥}}}
\end{quote}

\end{sloppypar}

\newpage

\begin{sloppypar}
अधमर्णो येन नियतकालेन यन्नियतधनमुत्तमर्णाय ददाति तत्स्कन्धमानम्~। नियत-कालश्च स्कन्धकालः कथ्यते~। अत्रोत्तमर्णः सर्वदा स्वप्रथमदत्तमूलधनस्यैव कलान्तरं गृह्णाति~।\\

\noindent \textbf{उदाहरणम्~।}

\phantomsection \label{Ex 2.20}
\begin{quote}
\textbf{{\color{red}मासेन पञ्चकशतेन शतं दशोनं \\
दत्तं कुसङ्कट इहाप्यधमर्णकाय~।\\
मासद्वयं प्रति सखे दशपञ्चयुक्तात् \\
स्कन्धं द्रुतं कथय मे परिमुक्तकालम्~॥}}
\end{quote}

न्यासः~। \begin{small}\begin{tabular}{c|c|}
~१~ &  \\
१०० & ९० \\
~५~ & 
\end{tabular}\end{small}\; स्कन्धमासौ २ स्कन्धमानम् १५~। जातो निर्मुक्तकालः ३०~। अत्र करणम्~। स्कन्धकालः २ नवतेः कलान्तरम् ९~। एतत् स्कन्धधनाद्विशोध्य शेषम् ६~। अनेन स्कन्धकाल\textendash \,२\textendash \,गुणिते धने १८० भक्ते जातो निर्मुक्तकालः ३०~।\\

\noindent \textbf{सूत्रम्~।}

\phantomsection \label{2.11.1}
\begin{quote}
\renewcommand{\thefootnote}{१}\footnote{अत्रोपपत्तिः~। मूलधनमानम् $=$ या~। तदा पूर्वसूत्रेण यदि निर्मुक्तकालः $=$ मुका~।
\vspace{2mm}

\hspace{2mm} मुका $= \dfrac{{\footnotesize{\hbox{स्कका.या.प्रध.प्रमा}}}}{{\footnotesize{\hbox{स्कध.प्रध.प्रमा}} - {\hbox{स्कका.या.प्रफ}}}}$
\vspace{2mm}

\hspace{2mm} अतः~~ मुका.स्कध.प्रध.प्रमा $-$ मुका.स्कका.प्रफ.या $=$ स्कका.प्रध.प्रमा.या
\vspace{2mm}

\hspace{4mm} $\therefore$\; या $=$ मू $= \dfrac{{\footnotesize{\hbox{मुका.स्कध.प्रध.प्रमा}}}}{{\footnotesize{\hbox{मुका.स्कका.प्रफ}} + {\hbox{स्कका.प्रध.प्रमा}}}}$
\vspace{2mm}

\hspace{17mm} $= \dfrac{{\footnotesize{\hbox{मुका.स्कध}}}}{{\footnotesize{\hbox{स्कका}}}\,\left(\dfrac{{\footnotesize{\hbox{मुका.प्रफ}}}}{{\footnotesize{\hbox{प्रध.प्रमा}}}} + {\hbox{१}}\right)}$ }{\large \textbf{{\color{purple}निर्मुक्तकालवृद्ध्या \\
रूपस्य हि सैकया हतेन भजेत्~।}}}
\end{quote}

\end{sloppypar}

\newpage

\begin{sloppypar}
\phantomsection \label{2.11}
\begin{quote}
{\large \textbf{{\color{purple}स्कन्धककालेन च गत-\\
कालस्कन्धाहतिर्मूलम्~॥~११~॥}}}\renewcommand{\thefootnote}{}\footnote{$\dfrac{{\footnotesize{\hbox{मुका.प्रफ}}}}{{\footnotesize{\hbox{प्रध.प्रमा}}}}$\; इदं रूपस्य निर्मुक्तकालसंबन्धिकलान्तरमर्थाद्वृद्धिः~। अत उपपद्यते सूत्रम्~।}
\end{quote}

गतकालस्कन्धाहतिर्निर्मुक्तकालस्कन्धधनयोराहतिः~।\\

\noindent \textbf{उदाहरणम्~।}

\phantomsection \label{Ex 2.21}
\begin{quote}
\textbf{{\color{red}पञ्चकशतेन वित्तं मासद्वितयेन सदलेन~।\\
स्कन्धः पञ्चदशाथ त्रिंशन्मासा विनिर्मुक्तः~।\\
कालस्त्विह वद मूलं किं वृद्धिः का च यदि वेत्सि~॥}}
\end{quote}

न्यासः~। \begin{small}\begin{tabular}{l|}
~~१ ~~~३० \\
१०० \\
~~५
\end{tabular}\end{small}\; स्कन्धकालः\, $\frac{{\footnotesize{\hbox{५}}}}{{\footnotesize{\hbox{२}}}}$\, स्कन्धधनम् १५~। जातं मूलधनम् ७२~।\\

करणम्~। रूपस्य निर्मुक्तकालवृद्धिः\, $\frac{{\footnotesize{\hbox{३}}}}{{\footnotesize{\hbox{२}}}}$\, सैका\textendash \,$\frac{{\footnotesize{\hbox{५}}}}{{\footnotesize{\hbox{२}}}}$\textendash \,नया स्कन्धकालो\, $\frac{{\footnotesize{\hbox{५}}}}{{\footnotesize{\hbox{२}}}}$\, गुणितः\, $\frac{{\footnotesize{\hbox{२५}}}}{{\footnotesize{\hbox{४}}}}$~। अनेन निर्मुक्तः कालः ३०~। स्कन्धधनम् १५~। अनयोराहति\textendash \,४५०\textendash \,र्भक्ता जाता मूलधनम् ७२~।\\

\noindent \textbf{सूत्रम्~।}

\phantomsection \label{2.12}
\begin{quote}
{\large \textbf{{\color{purple}स्कन्धकभक्तं वित्तं \\
लब्धं पदसञ्ज्ञकं च शेषांशः~।\\
अग्राख्यः पदवर्गः \\
पदयुक् स्कन्धार्धसङ्गुणो युक्तः~॥~१२~॥}}}
\end{quote}
\vspace{-8mm}

\phantomsection \label{2.13}
\begin{quote}
{\large \textbf{{\color{purple}अग्रांशघ्नधनेन \\
प्रजायते मूलपिण्डाख्यः~।\\
तस्य स्कन्धककालात् \\
समानयेद्वृद्धिमानमथ~॥~१३~॥}}}
\end{quote}

\end{sloppypar}

\newpage

\phantomsection \label{2.14.1}
\begin{quote}
\renewcommand{\thefootnote}{१}\footnote{कलान्तरमूलधनमिश्रितमत्र मूलधनमुच्यते तथा स्कन्धकाले स्कन्धधनं देयमित्युभयोः प्रतिज्ञा~। एवं निर्मुक्तकालः $= \dfrac{{\footnotesize{\hbox{स्कका.मूध}}}}{{\footnotesize{\hbox{स्कध}}}}$~। अथ कल्प्यते निर्मुक्तकाले वास्तवमूलधनस्य वृद्धिरर्थात् कलान्तरम् $=$ या~। तदा वास्तवमूलधनम् $=$ मू $-$ या~। अस्य निर्मुक्तकाले कलान्तरम्
\vspace{2mm}

\hspace{6mm} $=$ या $= \dfrac{{\footnotesize{\hbox{प्रफ}}\,({\hbox{मू}} - {\hbox{या}})}}{{\footnotesize{\hbox{प्रध.प्रमा}}}} \times \dfrac{{\footnotesize{\hbox{मू.स्कका}}}}{{\footnotesize{\hbox{स्कध}}}}$~।
\vspace{2mm}

\hspace{6mm} $\therefore$\; या $= \dfrac{{\footnotesize{\hbox{मू}}^{\scriptsize{\hbox{२}}}.{\hbox{प्रफ.स्कका}}}}{{\footnotesize{\hbox{प्रफ.मू.स्कका}} + {\hbox{प्रध.प्रमा.स्कध}}}}$~। 
\vspace{2mm}

\hspace{2mm} स्कन्धकाले यस्य धनस्य मूलपिण्डाख्यस्य कलान्तरमिदं तन्मानं त्रैराशिकेन~।
\vspace{2mm}

\hspace{4mm} मूपि $= \dfrac{{\footnotesize{\hbox{मू}}^{\scriptsize{\hbox{२}}}.{\hbox{प्रध.प्रमा}}}}{{\footnotesize{\hbox{प्रफ.मू.स्कका}} + {\hbox{प्रध.प्रमा.स्कध}}}}$~। मन्मते वास्तवं मूलपिण्डमिदमेव~। अत्र हरेण यदि भाज्यो विभज्यते तर्हि शेषत्यागेन स्थूला लब्धिः
\vspace{2mm}

\hspace{6mm} $= \dfrac{{\footnotesize{\hbox{मू}}^{\scriptsize{\hbox{२}}}.{\hbox{प्रध.प्रमा}}}}{{\footnotesize{\hbox{प्रध.प्रमा.स्कध}} + {\hbox{प्रफ.मू.स्कका}}}} = \dfrac{{\footnotesize{\hbox{मू}}^{\scriptsize{\hbox{२}}}}}{{\footnotesize{\hbox{स्कध}}}}$~।
\vspace{2mm}

\hspace{2mm} यदि\; मू $=$ स्कध (प $+$ शे)\; तदा
\vspace{2mm}

\hspace{4mm} $\dfrac{{\footnotesize{\hbox{मू}}^{\scriptsize{\hbox{२}}}}}{{\footnotesize{\hbox{स्कध}}}} = \dfrac{{\footnotesize{\hbox{स्कध}}^{\scriptsize{\hbox{२}}}({\hbox{प}} + {\hbox{शे}})^{\scriptsize{\hbox{२}}}}}{{\footnotesize{\hbox{स्कध}}}} =$ स्कध\,(प + शे)$^{\scriptsize{\hbox{२}}}$
\vspace{2mm}

\hspace{11mm} $=$ स्कध\,(प$^{\scriptsize{\hbox{२}}} +$ २\,प.शे $+$ शे$^{\scriptsize{\hbox{२}}}$) $=$ स्कध\,(प$^{\scriptsize{\hbox{२}}} +$ प.शे $+$ प.शे $+$ शे$^{\scriptsize{\hbox{२}}}$)
\vspace{1mm}

\hspace{11mm} $=$ स्कध\,(प$^{\scriptsize{\hbox{२}}} +$ प.शे) $+$ शे.स्कध\,(प $+$ शे)
\vspace{1mm}

\hspace{11mm} $=$ स्कध\,(प$^{\scriptsize{\hbox{२}}} +$ प.शे) $+$ शे.मू
\vspace{1mm}
 
\hspace{2mm} अत्र शेषमानं सदा रूपाल्पम्~। आचार्येण पदगुणस्य शेषस्य मानं परमं रूपं स्थूलं कल्पितम्~। अतः 'स्कन्धेन सङ्गुणा युक्तः' इति पाठः साधुर्भवति~।}{\large \textbf{{\color{purple}स्कन्धककालघ्नधने \\
स्कन्धहृते मुख्यकालः स्यात्~।}}}
\end{quote}

\newpage

\begin{sloppypar}
\noindent \textbf{उदाहरणम्~।}\renewcommand{\thefootnote}{}\footnote{\hspace{-4mm} अत्र,
\vspace{-6mm}

\begin{quote}
मूलधनस्य च वर्गः\\
स्कन्धहतो भवति वृद्धिमितिः~।\\
स्कन्धककालघ्नधने\\
स्कन्धहृते मुख्यकालः स्यात्~॥
\end{quote}

इति लाघवेन सूत्रं भवति~। आचार्येण गुरुकल्पना स्थूला च किमर्थं कृतेति सुधियो विभावयन्तु~॥
\vspace{1mm}
}

\phantomsection \label{Ex 2.22}
\begin{quote}
\textbf{{\color{red}पञ्चकशतेन दत्त्वा पञ्चयुताः सप्ततिरङ्केन\renewcommand{\thefootnote}{१}\footnote{The reading सप्ततिः केन seems to be a typographical error, as केन doesn’t give the number $9$ as mentioned in the commentary.}।\\
स्कन्धेन च प्रयच्छति मासाभ्यां ग्राहवृद्धिभयात्\renewcommand{\thefootnote}{२}\footnote{The reading ग्राहकवृद्धिभयात् doesn’t fit in the meter of the verse.}।\\
वृद्धिं विमुक्तिकालं कथय सखे त्वं पुरा वेत्सि~॥}}
\end{quote}

न्यासः \begin{small}\begin{tabular}{l|}
~~१ \\
१०० ~~७५\\
~~५
\end{tabular}\end{small}\; स्कन्धमासौ २~। स्कन्धधनम् ९~। जातं कलान्तरम् ३४\,$\frac{{\footnotesize{\hbox{९}}}}{{\footnotesize{\hbox{१०}}}}$~। निर्मुक्तकालो मासाः १६ दिनानि २०~। अथ करणं मूलधनं ७५ स्कन्धधनेन ९ भक्तं लब्धं पदसञ्ज्ञं ८ शेषमग्रसञ्ज्ञम्\, $\frac{{\footnotesize{\hbox{१}}}}{{\footnotesize{\hbox{३}}}}$~। पदवर्गः ६४ पदयुतः ७२ स्कन्धधनार्धेन\, $\frac{{\footnotesize{\hbox{९}}}}{{\footnotesize{\hbox{२}}}}$\, हतः ३२४~। अग्रांशघ्न\textendash \,$\frac{{\footnotesize{\hbox{१}}}}{{\footnotesize{\hbox{३}}}}$\textendash \,धनेन २५ युक्तः पूर्वराशिरयं ३२४ जातो मूलपिण्डः ३४९~। अस्य स्कन्धककालेन २ कलान्तरम् ३४\,$\frac{{\footnotesize{\hbox{९}}}}{{\footnotesize{\hbox{१०}}}}$~। अथ स्कन्धकालेन २ गुणिते धने १५० स्कन्धधनेन ९ भक्ते जातो निर्मुक्तकालो मासाः १६ दिनानि २०~।\\

\noindent \textbf{सूत्रम्~।}

\phantomsection \label{2.14}
\begin{quote}
{\large \textbf{{\color{purple}स्कन्धककालोपनयात् \\
मूलं मूलात् पृथक् पृथक् त्यक्त्वा~॥~१४~॥}}}
\end{quote}
\end{sloppypar}

\newpage

\phantomsection \label{2.15}
\begin{quote}
\renewcommand{\thefootnote}{१}\footnote{अत्र प्रथमस्कन्धधनदानसमये स्कन्धकाले स्कन्धधनं मूलकलान्तरमिलितं प्रकल्प्य तत् संबन्धि मूल-धनमेव धनिना लभ्यते~। एवं द्वितीयस्कन्धधनदानसमये द्विगुणस्कन्धकाले सकलान्तरं मूलधनं स्कन्धधनं प्रकल्प्य यन्मूलधनं तदेव धनिना मूलधने लब्धमित्येवमग्रेऽपि~। अन्ते यदवशिष्टं तत् संबन्धि गतस्कन्धकालि-ककलान्तरमानीय तदन्त्यावशिष्टधने संयोज्य मिश्रधनं कृत्वा ततोऽवशिष्टधनस्य माससंबन्धिकलान्तरं स्कन्ध-धनात् संशोध्य शेषस्कन्धधनेनैको मासस्तदा पूर्वानीतमिश्रधनेन किं लब्धं गतमासयुक्तं निर्मुक्तकाल आनीत इति~।}{\large \textbf{{\color{purple}अवशिष्टस्य च मासिक-\\
फलं त्यजेन्मासिकोपनयात्~।\\
शेषेण मासिकफलं \\
मासघ्नं मूलशेषयुग्विभजेत्~॥~१५~॥ \\
लब्धं गतमासयुतं \\
धनस्य निर्मुक्तकालः स्यात्~।}}}
\end{quote}

\noindent \textbf{उदाहरणम्~।}

\phantomsection \label{Ex 2.23}
\begin{quote}
\textbf{{\color{red}दशकशतेन तु दत्तं शतं च पञ्चाशदुत्तरं धनिना~।\\
प्रतिमासमृणी सफलं पञ्चाशत् स्कन्धकं प्रयच्छति च~।\\
अनृणी कालेन सखे प्रजायते केन कथयाशु~॥}}
\end{quote}

\begin{sloppypar}
न्यासः \begin{small}\begin{tabular}{l|}
~~१ \\
१०० ~~१५०\\
~१०
\end{tabular}\end{small}\; स्कन्धकालः १ सकलान्तरं स्कन्धधनम् ५०~। लब्धो विमुक्तकालः ३$\frac{{\footnotesize{\hbox{५४४७}}}}{{\footnotesize{\hbox{८१६१}}}}$~। अत्र करणम्~। स्कन्धधनं मित्रं कृत्वा स्कन्धकालस्य मूलकलान्तरे साध्ये~। तद्यथा~। मिश्रम् ५० अत्र मूलकलान्तरे\, $\frac{{\footnotesize{\hbox{५००}}}}{{\footnotesize{\hbox{११}}}}$~। $\frac{{\footnotesize{\hbox{५०}}}}{{\footnotesize{\hbox{११}}}}$~। मूलधनमेतत्\, $\frac{{\footnotesize{\hbox{५००}}}}{{\footnotesize{\hbox{११}}}}$\, पूर्वमूलादपास्य शेषम्
\end{sloppypar}

\newpage

\begin{sloppypar}
\noindent $\frac{{\footnotesize{\hbox{११५०}}}}{{\footnotesize{\hbox{११}}}}$~। पुनर्द्वितीयस्कन्धे पञ्चाशन्मिश्रान्मासद्वयमूलकलान्तरे\, $\frac{{\footnotesize{\hbox{१२५}}}}{{\footnotesize{\hbox{३}}}}$~। $\frac{{\footnotesize{\hbox{२५}}}}{{\footnotesize{\hbox{३}}}}$~। मूलं\, $\frac{{\footnotesize{\hbox{१२५}}}}{{\footnotesize{\hbox{३}}}}$\, पूर्वमूलादस्मात्\, $\frac{{\footnotesize{\hbox{११५०}}}}{{\footnotesize{\hbox{११}}}}$\, अपास्य शेषम्\, $\frac{{\footnotesize{\hbox{२०७५}}}}{{\footnotesize{\hbox{४३}}}}$~। पुनस्तृतीयस्कन्धे प्राग्वन्मूलकलान्तरे\, $\frac{{\footnotesize{\hbox{५००}}}}{{\footnotesize{\hbox{१३}}}}$~। $\frac{{\footnotesize{\hbox{५०}}}}{{\footnotesize{\hbox{१३}}}}$~। पूर्वमूलमेतच्छेषादपास्य शेषम्\, $\frac{{\footnotesize{\hbox{१०४७५}}}}{{\footnotesize{\hbox{४२९}}}}$~। अस्य मासकलान्तरम्\, $\frac{{\footnotesize{\hbox{२०९५}}}}{{\footnotesize{\hbox{८५८}}}}$~। एतत्स्कन्धधनादस्मात् ५० विशोध्य शेषम्\, $\frac{{\footnotesize{\hbox{४०८०५}}}}{{\footnotesize{\hbox{८५८}}}}$~। अयं छेदः~। मासफलम्\, $\frac{{\footnotesize{\hbox{२०९५}}}}{{\footnotesize{\hbox{८५८}}}}$\, गतमासैस्त्रिभिर्गुणितं पूर्वमूलशेषेणानेन\, $\frac{{\footnotesize{\hbox{१०४७५}}}}{{\footnotesize{\hbox{४२९}}}}$\, युतं जातम्\, $\frac{{\footnotesize{\hbox{२७२३५}}}}{{\footnotesize{\hbox{८५८}}}}$~। पूर्वच्छेदेनानेन\, $\frac{{\footnotesize{\hbox{४०८०५}}}}{{\footnotesize{\hbox{८५८}}}}$\, हृतं जातं लब्धम्\, $\frac{{\footnotesize{\hbox{५४४७}}}}{{\footnotesize{\hbox{८१६१}}}}$\, एतद्गतमासत्रययुतं जातो निर्मुक्तकालः ३$\frac{{\footnotesize{\hbox{५४४७}}}}{{\footnotesize{\hbox{८१६१}}}}$~।\\

अस्योपपत्तिः पञ्चराशिकेन~। न्यासो यथा \\

\begin{minipage}{0.2\textwidth}
\begin{small}\begin{tabular}{l|}
~~$\frac{{\footnotesize{\hbox{१}}}}{{\footnotesize{\hbox{१}}}} \hspace{5mm} \frac{{\footnotesize{\hbox{२९९३०}}}}{{\footnotesize{\hbox{८१६१}}}}$\\
 \\
$\frac{{\footnotesize{\hbox{१००}}}}{{\footnotesize{\hbox{१}}}} ~~~~\frac{{\footnotesize{\hbox{१०४७५}}}}{{\footnotesize{\hbox{४२९}}}}$\\
 \\
~$\frac{{\footnotesize{\hbox{१०}}}}{{\footnotesize{\hbox{१}}}}$ \hspace{6mm} ०
\end{tabular}\end{small}
\end{minipage} 
\hfill
\begin{minipage}[c]{0.7\textwidth} 
लब्धं कलान्तरम्\, $\frac{{\footnotesize{\hbox{३१३५१६७५}}}}{{\footnotesize{\hbox{३५०१०६९}}}}$~। एतन्मूलधनेनानेन\, $\frac{{\footnotesize{\hbox{१०४७५}}}}{{\footnotesize{\hbox{४२९}}}}$\, युतम्~।
\vspace{2mm}

अत्र त्रैराशिकम्~। यदि पञ्चाशत्स्कन्धेनैको मासस्तदानेन किमिति\, ५०।१। $\frac{{\footnotesize{\hbox{११६८३८१५०}}}}{{\footnotesize{\hbox{३५०१०६९}}}}$\, लब्धम्\, $\frac{{\footnotesize{\hbox{५४४७}}}}{{\footnotesize{\hbox{८१६१}}}}$
\end{minipage} 
\end{sloppypar}

\newpage

\begin{sloppypar}
\noindent एतद्गतमासयुतं जातो निर्मुक्तिकालः ३$\frac{{\footnotesize{\hbox{५४४७}}}}{{\footnotesize{\hbox{८१६१}}}}$~।\\

\noindent \textbf{सूत्रम्~।}

\phantomsection \label{2.16}
\begin{quote}
\renewcommand{\thefootnote}{१}\footnote{अत्रोपपत्तिराचार्यन्यासेनैव स्फुटा~।}{\large \textbf{{\color{purple}प्रतिमासिकफलशुद्धौ \\
मूलं मूलात् पृथक् पृथग् जह्यात्~॥~१६~॥ \\
शेषस्य मासिकफलं \\
विशोधयेद्मासिकोपनयात्~।\\
शेषेणानेन मूल-\\
विशेषमाप्तं तु मासयुक् कालः~॥~१७~॥}}}
\end{quote}

\noindent \textbf{उदाहरणम्~।}

\phantomsection \label{Ex 2.24}
\begin{quote}
\textbf{{\color{red}दत्तं दशकशतेन च शतं च कस्यापि केनचिद्धनिना~।\\
प्रतिमासिकफलसहिता पञ्चाशत् स्कन्धकं प्रयच्छति च~।\\
अनृणी कालेन सखे केन भवेद्ग्राहकस्य वद~॥}}
\end{quote}

न्यासः \begin{small}\begin{tabular}{l|}
~~१~~~~~~ ० \\
१०० ~~१००\\
~१०~~~~~ ०
\end{tabular}\end{small}\; स्कन्धकालः १ प्रतिमासं फलशुद्धिस्कन्धधनम् ५० लब्धो विमुक्तिकालः २$\frac{{\footnotesize{\hbox{४०}}}}{{\footnotesize{\hbox{१२१}}}}$~।\\

अत्र करणम्~। शतस्य मासेन कलान्तरम् १० स्कन्धादपास्य शेषम् ४० एतन्मूल-धनादस्मात् १०० अपास्य शेषम् ६० मूलधनम्~। द्वितीयस्कन्धे एभ्यो मासफलम् ६~। एतत् स्कन्धादपास्य ४४ एतन्मूलधनादस्मात् ६० अपास्य शेषं मूलधनम् १६। अस्य मूलशेषस्य मासफलम्\, $\frac{{\footnotesize{\hbox{८}}}}{{\footnotesize{\hbox{५}}}}$~। इदं स्कन्धधनादपास्य शेषम्\, $\frac{{\footnotesize{\hbox{२४२}}}}{{\footnotesize{\hbox{५}}}}$~। अनेन पूर्वधनमेतत् १६ भक्तं जातः शेषकालः\, $\frac{{\footnotesize{\hbox{४०}}}}{{\footnotesize{\hbox{१२१}}}}$\, गतमासद्वययुतो

\end{sloppypar}

\newpage

\noindent जातो निर्मुक्तिकालः २$\frac{{\footnotesize{\hbox{४०}}}}{{\footnotesize{\hbox{१२१}}}}$~।

\begin{center}
\textbf{इति वृद्धिज्ञानम्~।}\\
\vspace{6mm}

{\large \textbf{अथ सुवर्णगणितम्~।}}
\end{center}

\noindent \textbf{तत्र सूत्रम्~।}

\phantomsection \label{2.18}
\begin{quote}
\renewcommand{\thefootnote}{१}\footnote{{\color{violet}'सुवर्णवर्णाहतियोगराशौ स्वर्णैक्यभक्ते कनकैक्यवर्णः'} इत्यादि {\color{violet}भास्करो}क्तानुरूपमेवेदम्~।}{\large \textbf{{\color{purple}निजवर्णस्वर्णाहति-\\
योगं विभजेत् सुवर्णयोगेन~।\\
वर्णः स्यादथ पक्वैः \\
स्वर्णैर्वर्णस्तु\renewcommand{\thefootnote}{$\star$}\footnote{The reading पक्वैर्वर्णैः स्वर्णस्तु seems to be a typographical error as it doesn't give a relevant meaning in the context.} वर्णकैः स्वर्णम्~॥~१८~॥}}}
\end{quote}

\noindent \textbf{उदाहरणम्~।}

\phantomsection \label{Ex 2.25}
\begin{quote}
\textbf{{\color{red}रन्ध्रेशविश्वतिथिवर्णसुवर्णमाषा \\
नेत्राङ्कतर्कदहनप्रमिताश्च ते तु~।\\
आवर्तिताः कथय कोविद वर्णमानम् \\
अत्राशु हेमगणितेऽस्ति तव श्रमश्चेत्~।\\
तच्छोधने कथय मे तिथिवर्णकानां \\
सङ्ख्यामितिं\renewcommand{\thefootnote}{$\dag$}\footnote{The reading मितिं seems to be a typographical error, as it doesn't fit in the meter of the verse. Two long syllables are needed before मितिं.} नृपसुवर्णजवर्णकानाम्~॥}}
\end{quote}

न्यासः~। $\frac{{\footnotesize{\hbox{९}}}}{{\footnotesize{\hbox{२}}}}$~। $\frac{{\footnotesize{\hbox{११}}}}{{\footnotesize{\hbox{९}}}}$~। $\frac{{\footnotesize{\hbox{१३}}}}{{\footnotesize{\hbox{६}}}}$~। $\frac{{\footnotesize{\hbox{१५}}}}{{\footnotesize{\hbox{३}}}}$~। जातः समावर्तने वर्णः १२ माषाः २०~। यदि शोधने कृते पञ्चदशवर्णहेम्नां सङ्ख्या १५ तर्हि शोधने कृते षोडशसुवर्णकानां जाता वर्णसङ्ख्या १५~।\\

\noindent \textbf{सूत्रम्~।}

\phantomsection \label{2.19.1}
\begin{quote}
\renewcommand{\thefootnote}{२}\footnote{{\color{violet}'स्वर्णैक्यनिघ्नाद्युतिजातवर्णात् सुवर्णतद्वर्णवधैक्यहीनात्~।'} इत्यादि {\color{violet}भास्करो}क्तानुरूपमेवेदम्~।}{\large \textbf{{\color{purple}कनकयुतिताडिताग्निज-\\
वर्णे स्वर्णघ्नवर्णयुतिविहीने~।}}}
\end{quote}

\newpage

\phantomsection \label{2.19}
\begin{quote}
{\large \textbf{{\color{purple}अज्ञातवर्णहेम्नो-\\
द्धृते तदज्ञातवर्णमितिः~॥~१९~॥}}}
\end{quote}

\noindent \textbf{उदाहरणम्~।}

\phantomsection \label{Ex 2.26}
\begin{quote}
\textbf{{\color{red}त्रिद्व्येकमाषा दशभानुशक्र-\\
वर्णाश्च विश्वप्रमितोऽग्निजातः~।\\
अज्ञातवर्णस्य च पञ्चमाषाः \\
तद्वर्णसङ्ख्यां वद कोविदाशु~॥}}
\end{quote}

न्यासः\, $\frac{{\footnotesize{\hbox{१०}}}}{{\footnotesize{\hbox{३}}}}$~। $\frac{{\footnotesize{\hbox{१२}}}}{{\footnotesize{\hbox{२}}}}$~। $\frac{{\footnotesize{\hbox{१४}}}}{{\footnotesize{\hbox{१}}}}$~। $\frac{{\footnotesize{\hbox{०}}}}{{\footnotesize{\hbox{५}}}}$~। अग्निजवर्णः १३~। 
\vspace{3mm}

ज्ञातोऽज्ञातवर्णः १५~।\\

\noindent \textbf{सूत्रम्~।}

\phantomsection \label{2.20}
\begin{quote}
\renewcommand{\thefootnote}{१}\footnote{{\color{violet}'स्वर्णैक्यनिघ्नो युतिजातवर्णः'} इत्यादि {\color{violet}भास्करो}क्तानुरूपमिदम्~।}{\large \textbf{{\color{purple}स्वर्णैक्याग्निजघाते \\
स्वर्णाहतवर्णयोगविश्लेषे~।\\
ज्ञातकनकवर्णाग्निज-\\
वर्णान्तरहृते कनकमानम्\renewcommand{\thefootnote}{$\star$}\footnote{The reading कनकम् seems to be a typographical error, as it is short of three \textit{mātrās} to fit in a meter.}॥}}}
\end{quote}

\noindent \textbf{उदाहरणम्~।}

\phantomsection \label{Ex 2.27}
\begin{quote}
\textbf{{\color{red}वर्णा महीपतिथिभानुमिताः सुवर्णा \\
नेत्रानलाम्बुनिधयोऽत्र हि वह्निजातः~।\\
ईशोन्मितो वद सखे नववर्णहेम्नां \\
सङ्ख्यां प्रवेत्सि कनकव्यवहारमार्गम्~॥}}
\end{quote}

न्यासः\, $\frac{{\footnotesize{\hbox{१६}}}}{{\footnotesize{\hbox{२}}}}$~। $\frac{{\footnotesize{\hbox{१५}}}}{{\footnotesize{\hbox{३}}}}$~। $\frac{{\footnotesize{\hbox{१४}}}}{{\footnotesize{\hbox{९}}}}$~। $\frac{{\footnotesize{\hbox{९}}}}{{\footnotesize{\hbox{०}}}}$~। अग्निजवर्णः ११~। जातमज्ञातसुवर्णमानम् १३~।\\

\noindent \textbf{सूत्रम्~।}

\phantomsection \label{2.21.1}
\begin{quote}
\renewcommand{\thefootnote}{२}\footnote{अत्रोपपत्तिरतिसुगमा~।}{\large \textbf{{\color{purple}एकं मुक्त्वा हेम्नोऽ-\\
न्येषां मानं सुवर्णमानानि~।}}}
\end{quote}

\newpage
\begin{sloppypar}

\phantomsection \label{2.21}
\begin{quote}
{\large \textbf{{\color{purple}परिकल्प्येष्टानि ततः \\
प्राग्वज्ज्ञेयं यदज्ञातम्~॥~२१~॥}}}
\end{quote}

\noindent \textbf{उदाहरणम्~।}

\phantomsection \label{Ex 2.28}
\begin{quote}
\textbf{{\color{red}गजदशतिथिभूपा वर्णसङ्ख्याग्निजातो \\
भवति मनुमितश्चानेकधा ब्रूहि हेम्नाम्~।\\
गणक परिमितिं मे प्रौढतास्ति प्रभूता \\
कनकगणितकर्मण्यर्कसङ्ख्योऽग्निजो वा~॥}}
\end{quote}

प्रथमन्यासः\, $\frac{{\footnotesize{\hbox{८}}}}{{\footnotesize{\hbox{०}}}}$~। $\frac{{\footnotesize{\hbox{१०}}}}{{\footnotesize{\hbox{०}}}}$~। $\frac{{\footnotesize{\hbox{१५}}}}{{\footnotesize{\hbox{०}}}}$~। $\frac{{\footnotesize{\hbox{१६}}}}{{\footnotesize{\hbox{०}}}}$~। अग्निजः १४~। अत्राद्यं मुक्त्वान्येषां कल्पितानि सुवर्णमानानि\, $\frac{{\footnotesize{\hbox{८}}}}{{\footnotesize{\hbox{०}}}}$~। $\frac{{\footnotesize{\hbox{१०}}}}{{\footnotesize{\hbox{१}}}}$~। $\frac{{\footnotesize{\hbox{१५}}}}{{\footnotesize{\hbox{२}}}}$~। $\frac{{\footnotesize{\hbox{१६}}}}{{\footnotesize{\hbox{३}}}}$~। प्राग्वज्जातं सुवर्णमानम्\, $\frac{{\footnotesize{\hbox{२}}}}{{\footnotesize{\hbox{३}}}}$\, अभिन्नार्थं त्रिभिर्गुणिता जाताः २।३।६।९ षड्गुणा वा ४।६।१२।१८~।\\

अथवा द्वितीयं तैरेवेष्टैर्जातं सुवर्णमानम् $\frac{{\footnotesize{\hbox{१}}}}{{\footnotesize{\hbox{२}}}}$~। सर्वे द्विगुणिता जाता अभिन्नाः २।१।४।६~। एवं तृतीयचतुर्थे कृत्वा स्वर्णमानानि साध्यानि~।\\

द्वितीयोदाहरणे न्यासः~। $\frac{{\footnotesize{\hbox{८}}}}{{\footnotesize{\hbox{०}}}}$~। $\frac{{\footnotesize{\hbox{१०}}}}{{\footnotesize{\hbox{०}}}}$~। $\frac{{\footnotesize{\hbox{१५}}}}{{\footnotesize{\hbox{०}}}}$~। $\frac{{\footnotesize{\hbox{१६}}}}{{\footnotesize{\hbox{०}}}}$~। अग्निजः १२ आद्यं मुक्त्वा तैरेवे-ष्टैर्जातानि हेम्नां मानानि ४।१।२।३ एते स्वयमेवाभिन्नाः~। द्वाभ्यां गुणिता जाताः ८।२।४।६ एवमिष्टवशादानन्त्यम्~।\\

\noindent \textbf{सूत्रम्~।}

\phantomsection \label{2.22}
\begin{quote}
\renewcommand{\thefootnote}{१}\footnote{अत्रोपपत्तिः~। कल्प्यते, अधमवर्णमानम् $=$ अव, उत्तमवर्णमानम्}{\large \textbf{{\color{purple}अधमोत्तमवर्णान्तरम् \\
इष्टांशाप्तं विरूपकं कृत्वा~।\\
एकैकोनं च पृथक् \\
यावद्रूपान्त्यमुपयाति~॥~२२~॥}}}
\end{quote}

\end{sloppypar}

\newpage

\phantomsection \label{2.23.1}
\begin{quote}
{\large \textbf{{\color{purple}न्यग्वर्णेः स्वर्णास्ते \\
व्यस्ता अप्युत्तमवर्णस्य~।}}}\renewcommand{\thefootnote}{}\footnote{$=$ उव~। सुवर्णैक्यम् $=$ स्थि, युतिजवर्णमानम् $=$ अव $+ \dfrac{{\footnotesize{\hbox{अ}}}}{{\footnotesize{\hbox{क}}}}$~। तदाधमस्य स्वर्णमानम् $=$ या~।
\vspace{1mm}

\hspace{2mm} तदोत्तमस्य स्थि $-$ या~। ततः
\vspace{2mm}

\hspace{10mm} $\dfrac{{\footnotesize{\hbox{या.अव}} + {\hbox{उव\,(स्थि}} - {\hbox{या)}}}}{{\footnotesize{\hbox{स्थि}}}} =$ अव $+ \dfrac{{\footnotesize{\hbox{अ}}}}{{\footnotesize{\hbox{क}}}}$
\vspace{2mm}

\hspace{2mm} वा~~~ या.अव $+$ स्थि.उव $-$ उव.या $=$ अव $-$ स्थि $+ \dfrac{{\footnotesize{\hbox{स्थि.अ}}}}{{\footnotesize{\hbox{क}}}}$~।
\vspace{2mm}

\hspace{2mm} पक्षान्तरेण~~~ या\,(उव $-$ अव) $=$ स्थि $\left({\hbox{उव}} - {\hbox{अव}} - \dfrac{{\footnotesize{\hbox{अ}}}}{{\footnotesize{\hbox{क}}}}\right)$
\vspace{2mm}

\hspace{35mm} $= \dfrac{{\footnotesize{\hbox{स्थि.अ}}}}{{\footnotesize{\hbox{क}}}} \left(\dfrac{{\footnotesize{\hbox{उव}} - {\hbox{अव}}}}{\dfrac{{\footnotesize{\hbox{अ}}}}{{\footnotesize{\hbox{क}}}}} - {\hbox{१}}\right)$
\vspace{2mm}

\hspace{2mm} अत्राचार्येण~ स्थि\,$\dfrac{{\footnotesize{\hbox{अ}}}}{{\footnotesize{\hbox{क}}}} =$ उव $-$ अव~। इति कल्पितम्~।
\vspace{2mm}

\hspace{2mm} तदा~~ या $= \left(\dfrac{{\footnotesize{\hbox{उव}} - {\hbox{अव}}}}{\dfrac{{\footnotesize{\hbox{अ}}}}{{\footnotesize{\hbox{क}}}}} - {\hbox{१}}\right)$~।~~ स्थि $= \dfrac{{\footnotesize{\hbox{उव}} - {\hbox{अव}}}}{\dfrac{{\footnotesize{\hbox{अ}}}}{{\footnotesize{\hbox{क}}}}}$~। 
\vspace{2mm}

\hspace{2mm} $\dfrac{{\footnotesize{\hbox{अ}}}}{{\footnotesize{\hbox{क}}}}$\, इत्यस्य स्थाने\, $\dfrac{{\footnotesize{\hbox{२\,अ}}}}{{\footnotesize{\hbox{क}}}}, \dfrac{{\footnotesize{\hbox{३\,अ}}}}{{\footnotesize{\hbox{क}}}}$\, इत्यादि प्रकल्प्य\; अव $+ \dfrac{{\footnotesize{\hbox{२\,अ}}}}{{\footnotesize{\hbox{क}}}}$~। अव $+ \dfrac{{\footnotesize{\hbox{३\,अ}}}}{{\footnotesize{\hbox{क}}}}$\, इत्यादि युतिजवर्ण-संबन्धिन्यग्वर्णसुवर्णमानं ज्ञेयं तत् स्थिरात् संशोध्योत्तमसुवर्णस्य मानं भवतीत्युपपद्यते~।}
\end{quote}

अत्र स्वर्णैक्यमानं सदा स्थिरं युतिजवर्णमानमधमवर्णमानतश्चैकादिगुणितेष्टांशसमम् अधिकम्~।

\newpage

\noindent \textbf{उदाहरणम्~।}

\phantomsection \label{Ex 2.29}
\begin{quote}
\textbf{{\color{red}द्वादशषोडशवर्णे\\
गुटिके ताभ्यां सुवर्णतुलिताश्च~।\\
वर्णशलाका अधमा\\
वर्णाश्चरणोक्त्या कथय~।}}
\end{quote}

\begin{sloppypar}
न्यासः~। $\frac{{\footnotesize{\hbox{१२}}}}{{\footnotesize{\hbox{०}}}}$~। $\frac{{\footnotesize{\hbox{१६}}}}{{\footnotesize{\hbox{०}}}}$\, इष्टवर्णांशः\, $\frac{{\footnotesize{\hbox{१}}}}{{\footnotesize{\hbox{४}}}}$\, तौल्यमाषः १ अत्र वर्णान्तरम् ४ इष्टांशेनानेन\, $\frac{{\footnotesize{\hbox{१}}}}{{\footnotesize{\hbox{४}}}}$\, भक्तम् १६ व्येकम् १५ एकोनं कृत्वा रूपान्तम्\textendash
\vspace{2mm}

१५ । १४ । १३ । १२ । ११ । १० । ९ । ८ । ७ । ६ । ५ । ४ । ३ । २ । १ उत्तमस्य व्यस्ताः १ । २ । ३ । ४ । ५ । ६ । ७ । ८ । ९ । १० । ११ । १२ । १३ । १४ । १५ । एते प्रक्षेपकरणेन माषतुल्या जाताः~।
\vspace{2mm}

{\small $\begin{matrix}
\mbox{{१५}}\\
\vspace{-1mm}
\mbox{{१६}}
\vspace{1mm}
\end{matrix}$}~। {\small $\begin{matrix}
\mbox{{१४}}\\
\vspace{-1mm}
\mbox{{१६}}
\vspace{1mm}
\end{matrix}$}~। {\small $\begin{matrix}
\mbox{{१३}}\\
\vspace{-1mm}
\mbox{{१६}}
\vspace{1mm}
\end{matrix}$}~। {\small $\begin{matrix}
\mbox{{१२}}\\
\vspace{-1mm}
\mbox{{१६}}
\vspace{1mm}
\end{matrix}$}~। {\small $\begin{matrix}
\mbox{{११}}\\
\vspace{-1mm}
\mbox{{१६}}
\vspace{1mm}
\end{matrix}$}~। {\small $\begin{matrix}
\mbox{{१०}}\\
\vspace{-1mm}
\mbox{{१६}}
\vspace{1mm}
\end{matrix}$}~। {\small $\begin{matrix}
\mbox{{९}}\\
\vspace{-1mm}
\mbox{{१६}}
\vspace{1mm}
\end{matrix}$}~। {\small $\begin{matrix}
\mbox{{८}}\\
\vspace{-1mm}
\mbox{{१६}}
\vspace{1mm}
\end{matrix}$}~। {\small $\begin{matrix}
\mbox{{७}}\\
\vspace{-1mm}
\mbox{{१६}}
\vspace{1mm}
\end{matrix}$}~। {\small $\begin{matrix}
\mbox{{६}}\\
\vspace{-1mm}
\mbox{{१६}}
\vspace{1mm}
\end{matrix}$}~। {\small $\begin{matrix}
\mbox{{५}}\\
\vspace{-1mm}
\mbox{{१६}}
\vspace{1mm}
\end{matrix}$}~। {\small $\begin{matrix}
\mbox{{४}}\\
\vspace{-1mm}
\mbox{{१६}}
\vspace{1mm}
\end{matrix}$}~। {\small $\begin{matrix}
\mbox{{३}}\\
\vspace{-1mm}
\mbox{{१६}}
\vspace{1mm}
\end{matrix}$}~। {\small $\begin{matrix}
\mbox{{२}}\\
\vspace{-1mm}
\mbox{{१६}}
\vspace{1mm}
\end{matrix}$}~। {\small $\begin{matrix}
\mbox{{१}}\\
\vspace{-1mm}
\mbox{{१६}}
\vspace{1mm}
\end{matrix}$}~। \\

\noindent उत्तमस्य\textendash
\vspace{2mm}

{\small $\begin{matrix}
\mbox{{१}}\\
\vspace{-1mm}
\mbox{{१६}}
\vspace{1mm}
\end{matrix}$}~। {\small $\begin{matrix}
\mbox{{२}}\\
\vspace{-1mm}
\mbox{{१६}}
\vspace{1mm}
\end{matrix}$}~। {\small $\begin{matrix}
\mbox{{३}}\\
\vspace{-1mm}
\mbox{{१६}}
\vspace{1mm}
\end{matrix}$}~। {\small $\begin{matrix}
\mbox{{४}}\\
\vspace{-1mm}
\mbox{{१६}}
\vspace{1mm}
\end{matrix}$}~। {\small $\begin{matrix}
\mbox{{५}}\\
\vspace{-1mm}
\mbox{{१६}}
\vspace{1mm}
\end{matrix}$}~। {\small $\begin{matrix}
\mbox{{६}}\\
\vspace{-1mm}
\mbox{{१६}}
\vspace{1mm}
\end{matrix}$}~। {\small $\begin{matrix}
\mbox{{७}}\\
\vspace{-1mm}
\mbox{{१६}}
\vspace{1mm}
\end{matrix}$}~। {\small $\begin{matrix}
\mbox{{८}}\\
\vspace{-1mm}
\mbox{{१६}}
\vspace{1mm}
\end{matrix}$}~। {\small $\begin{matrix}
\mbox{{९}}\\
\vspace{-1mm}
\mbox{{१६}}
\vspace{1mm}
\end{matrix}$}~। {\small $\begin{matrix}
\mbox{{१०}}\\
\vspace{-1mm}
\mbox{{१६}}
\vspace{1mm}
\end{matrix}$}~। {\small $\begin{matrix}
\mbox{{११}}\\
\vspace{-1mm}
\mbox{{१६}}
\vspace{1mm}
\end{matrix}$}~। {\small $\begin{matrix}
\mbox{{१२}}\\
\vspace{-1mm}
\mbox{{१६}}
\vspace{1mm}
\end{matrix}$}~। {\small $\begin{matrix}
\mbox{{१३}}\\
\vspace{-1mm}
\mbox{{१६}}
\vspace{1mm}
\end{matrix}$}~। {\small $\begin{matrix}
\mbox{{१४}}\\
\vspace{-1mm}
\mbox{{१६}}
\vspace{1mm}
\end{matrix}$}~। {\small $\begin{matrix}
\mbox{{१५}}\\
\vspace{-1mm}
\mbox{{१६}}
\vspace{1mm}
\end{matrix}$}~। \\
\end{sloppypar}
\vspace{2mm}

\noindent \textbf{सूत्रम्~।}

\phantomsection \label{2.23}
\begin{quote}
\renewcommand{\thefootnote}{१}\footnote{अत्रोपपत्तिः~। मिश्रसुवर्णमानम् $=$ मि, प्रथमस्य वर्णः $=$ व$_{\scriptsize{\hbox{१}}}$~। द्वितीयस्य व$_{\scriptsize{\hbox{२}}}$~। पक्कवर्णमानम् $=$ प~। पक्कसुवर्णमानम् $=$ युव~। प्रथमसुवर्णमानम् $=$ या~। तदा द्वितीयसुवर्णमानम् $=$ मि $-$ या~।}{\large \textbf{{\color{purple}वर्णौ मिश्रविगुणितौ \\
पक्वस्वर्णघ्नवर्णकान्तरितौ~॥~२३~॥}}}
\end{quote}

\newpage

\phantomsection \label{2.24}
\begin{quote}
{\large \textbf{{\color{purple}वर्णान्तरेण (भक्तौ)\\
सुवर्णमाने मिथो भवतः~॥}}}\renewcommand{\thefootnote}{}\footnote{तदा~~ ${\hbox{व}}_{\scriptsize{\hbox{१}}}$.या $+ {\hbox{व}}_{\scriptsize{\hbox{२}}}$\,(मि $-$ या) $=$ युव.प
\vspace{1mm}

\hspace{2mm} वा $=$ या (${\hbox{व}}_{\scriptsize{\hbox{१}}} - {\hbox{व}}_{\scriptsize{\hbox{२}}}$) $=$ युव.प $- {\hbox{व}}_{\scriptsize{\hbox{२}}}$.मि
\vspace{2mm}

\hspace{4mm} $\therefore$\; या $= \dfrac{{\footnotesize{\hbox{युव.प}} - {\hbox{व}}_{\scriptsize{\hbox{२}}}.{\hbox{मि}}}}{{\footnotesize{\hbox{व}}_{\scriptsize{\hbox{१}}} - {\hbox{व}}_{\scriptsize{\hbox{२}}}}}$~~ तथा~~ मि $-$ या $= \dfrac{{\footnotesize{\hbox{मि.व}}_{\scriptsize{\hbox{१}}} - {\hbox{युव.प}}}}{{\footnotesize{\hbox{व}}_{\scriptsize{\hbox{१}}} - {\hbox{व}}_{\scriptsize{\hbox{२}}}}}$
\vspace{2mm}

अत उपपद्यते~।
\vspace{1mm}}
\end{quote}

\noindent \textbf{उदाहरणम्~।}

\phantomsection \label{Ex 2.30}
\begin{quote}
\textbf{{\color{red}दशमनुवर्णौ खोटौ मिश्रे स्वर्णं तु पञ्चपञ्चाशत्~।\\
त्रियुतदशवर्णपक्वाः पञ्चाशन्मे पृथक् पृथक् कथय~॥}}
\end{quote}

न्यासः~। $\frac{{\footnotesize{\hbox{१०}}}}{{\footnotesize{\hbox{०}}}}$~। $\frac{{\footnotesize{\hbox{१४}}}}{{\footnotesize{\hbox{०}}}}$\, मिश्रम् ५५ पक्कः १३~। जाते स्वर्णमाने ३०।२५~।\\

\noindent \textbf{सूत्रम्~।}

\phantomsection \label{2.25}
\begin{quote}
\renewcommand{\thefootnote}{१}\footnote{अत्रोपपत्तिः~। अत्र द्वयोः सुवर्णयोर्वर्णमाने ${\hbox{व}}_{\scriptsize{\hbox{१}}}$, ${\hbox{व}}_{\scriptsize{\hbox{२}}}$, तत्सुवर्णमाने च ${\hbox{सु}}_{\scriptsize{\hbox{१}}}$, ${\hbox{सु}}_{\scriptsize{\hbox{२}}}$, तदा प्रश्नोक्त्या}{\large \textbf{{\color{purple}छेदसमाने गुटिके \\
अंशसमाने च खण्डकाख्ये स्तः~॥~२४~॥ \\
निजहेमाहतवर्णे \\
स्वखण्डपरहेमयुतिगुणिते\renewcommand{\thefootnote}{२}\footnote{The reading -योगगुणिते doesn’t fit in the meter of the verse.}।\\
स्वकखण्डकपरवर्णा-\\
हतिनिजहेमान्यखण्डयुतिघातम्~॥~२५~॥ \\
त्यक्त्वा शेषं हेम्नोः \\
वधेन खण्डाभिघातहीनेन~।\\
विभजेत् परवर्णमितिः \\
विज्ञेयैवं द्वितीयस्य~॥~२६~॥}}}
\end{quote}

\newpage

\noindent \textbf{उदाहरणम्~।}\renewcommand{\thefootnote}{}\footnote{\hspace{4mm} $\dfrac{\dfrac{{\footnotesize{\hbox{व}}_{\scriptsize{\hbox{१}}}.{\hbox{सु}}_{\scriptsize{\hbox{१}}}.{\hbox{अ}}}}{{\footnotesize{\hbox{क}}}} + {\footnotesize{\hbox{व}}_{\scriptsize{\hbox{२}}}.{\hbox{सु}}_{\scriptsize{\hbox{२}}}}}{{\footnotesize{\hbox{सु}}_{\scriptsize{\hbox{१}}}}\dfrac{{\footnotesize{\hbox{अ}}}}{{\footnotesize{\hbox{क}}}} + {\footnotesize{\hbox{सु}}_{\scriptsize{\hbox{२}}}}} = {\hbox{युव}}_{\scriptsize{\hbox{१}}}$
\vspace{2mm}

\hspace{9mm} $\therefore \hspace{2mm} {\hbox{व}}_{\scriptsize{\hbox{१}}}.{\hbox{सु}}_{\scriptsize{\hbox{१}}}.{\hbox{अ}} + {\hbox{व}}_{\scriptsize{\hbox{२}}}.{\hbox{सु}}_{\scriptsize{\hbox{२}}}.{\hbox{क}} = ({\hbox{सु}}_{\scriptsize{\hbox{१}}}.{\hbox{अ}} + {\hbox{सु}}_{\scriptsize{\hbox{२}}}.{\hbox{क}})\,{\hbox{युव}}_{\scriptsize{\hbox{१}}}$
\vspace{2mm}

\hspace{2mm} एवम् \hspace{4mm} ${\hbox{व}}_{\scriptsize{\hbox{१}}}.{\hbox{सु}}_{\scriptsize{\hbox{१}}}.{\hbox{क'}} + {\hbox{व}}_{\scriptsize{\hbox{२}}}.{\hbox{सु}}_{\scriptsize{\hbox{२}}}.{\hbox{अ'}} = ({\hbox{सु}}_{\scriptsize{\hbox{२}}}.{\hbox{अ'}} + {\hbox{सु}}_{\scriptsize{\hbox{१}}}.{\hbox{क'}})\,{\hbox{युव'}}_{\scriptsize{\hbox{१}}}$
\vspace{2mm}

\hspace{2mm} वा~~ ${\hbox{व}}_{\scriptsize{\hbox{१}}}.{\hbox{सु}}_{\scriptsize{\hbox{१}}}.{\hbox{अ.क'}} + {\hbox{व}}_{\scriptsize{\hbox{२}}}.{\hbox{सु}}_{\scriptsize{\hbox{२}}}.{\hbox{क.क'}} = ({\hbox{सु}}_{\scriptsize{\hbox{१}}}.{\hbox{अ.क'}} + {\hbox{सु}}_{\scriptsize{\hbox{२}}}.{\hbox{क.क'}})\,{\hbox{युव}}_{\scriptsize{\hbox{१}}}$
\vspace{2mm}

\hspace{2mm} वा~~ ${\hbox{व}}_{\scriptsize{\hbox{१}}}.{\hbox{सु}}_{\scriptsize{\hbox{१}}}.{\hbox{अ.क'}} + {\hbox{व}}_{\scriptsize{\hbox{२}}}.{\hbox{सु}}_{\scriptsize{\hbox{२}}}.{\hbox{अ.अ'}} = ({\hbox{सु}}_{\scriptsize{\hbox{२}}}.{\hbox{अ.अ'}} + {\hbox{सु}}_{\scriptsize{\hbox{१}}}.{\hbox{अ.क'}})\,{\hbox{युव'}}_{\scriptsize{\hbox{१}}}$
\vspace{2mm}

\hspace{2mm} $\therefore\; {\hbox{व}}_{\scriptsize{\hbox{२}}}.{\hbox{सु}}_{\scriptsize{\hbox{२}}}\,({\hbox{क.क'}} - {\hbox{अ.अ'}}) = ({\hbox{सु}}_{\scriptsize{\hbox{१}}}.{\hbox{अ.क'}} + {\hbox{सु}}_{\scriptsize{\hbox{२}}}.{\hbox{क.क'}})\,{\hbox{युव}}_{\scriptsize{\hbox{१}}} - ({\hbox{सु}}_{\scriptsize{\hbox{२}}}.{\hbox{अ.अ'}} + {\hbox{सु}}_{\scriptsize{\hbox{१}}}.{\hbox{अ.क'}})\,{\hbox{युव'}}_{\scriptsize{\hbox{१}}}$
\vspace{2mm}

\hspace{2mm} $\therefore\; {\hbox{व}}_{\scriptsize{\hbox{२}}}.{\hbox{सु}}_{\scriptsize{\hbox{२}}} = \dfrac{{\footnotesize({\hbox{सु}}_{\scriptsize{\hbox{१}}}.{\hbox{अ.क'}} + {\hbox{सु}}_{\scriptsize{\hbox{२}}}.{\hbox{क.क'}})\,{\hbox{युव}}_{\scriptsize{\hbox{१}}} - ({\hbox{सु}}_{\scriptsize{\hbox{२}}}.{\hbox{अ.अ'}} + {\hbox{सु}}_{\scriptsize{\hbox{१}}}.{\hbox{अ.क'}})\,{\hbox{युव'}}_{\scriptsize{\hbox{१}}}}}{{\footnotesize{\hbox{क.क'}} - {\hbox{अ.अ'}}}}$
\vspace{2mm}

\hspace{2mm} अत्र\; ${\hbox{सु}}_{\scriptsize{\hbox{२}}} =$ क',\; ${\hbox{सु}}_{\scriptsize{\hbox{१}}} =$ क\; इति प्रकल्प्य जातम्
\vspace{2mm}

\hspace{7mm} ${\hbox{व}}_{\scriptsize{\hbox{२}}}.{\hbox{क'}} = \dfrac{{\footnotesize({\hbox{अ.क.क'}} + {\hbox{क.क'}}^{\scriptsize{\hbox{२}}})\,{\hbox{युव}}_{\scriptsize{\hbox{१}}} - ({\hbox{अ.अ'.क'}} + {\hbox{अ.क.क'}})\,{\hbox{युव'}}_{\scriptsize{\hbox{१}}}}}{{\footnotesize{\hbox{क.क'}} - {\hbox{अ.अ'}}}}$
\vspace{2mm}

\hspace{5mm} वा,~~ ${\hbox{व}}_{\scriptsize{\hbox{२}}} = \dfrac{{\footnotesize({\hbox{अ.क}} + {\hbox{क.क'}})\,{\hbox{युव}}_{\scriptsize{\hbox{१}}} - ({\hbox{अ.अ'}} + {\hbox{अ.क}})\,{\hbox{युव'}}_{\scriptsize{\hbox{१}}}}}{{\footnotesize{\hbox{क.क'}} - {\hbox{अ.अ'}}}}$
\vspace{2mm}

\hspace{15mm} $= \dfrac{{\footnotesize({\hbox{अ}} + {\hbox{क'}})\,{\hbox{क.युव}}_{\scriptsize{\hbox{१}}} - ({\hbox{अ'}} + {\hbox{क}})\,{\hbox{अ.युव'}}_{\scriptsize{\hbox{१}}}}}{{\footnotesize{\hbox{क.क'}} - {\hbox{अ.अ'}}}}$
\vspace{2mm}

\hspace{2mm} एवम्,~~ ${\hbox{व}}_{\scriptsize{\hbox{१}}} = \dfrac{{\footnotesize({\hbox{अ'}} + {\hbox{क}})\,{\hbox{क'.यु'व}}_{\scriptsize{\hbox{१}}} - ({\hbox{अ}} + {\hbox{क'}})\,{\hbox{अ'.युव}}_{\scriptsize{\hbox{१}}}}}{{\footnotesize{\hbox{क.क'}} - {\hbox{अ.अ'}}}}$
\vspace{2mm}

\hspace{2mm} एवं सर्वमुपपद्यते~।}

\phantomsection \label{Ex 2.31.1}
\begin{quote}
\textbf{{\color{red}अज्ञातवर्णहाटकखोटौ प्रथमत्रिभागयुगलेन~।\\
आवर्तिते परस्मिन् निखिले निष्पद्यतेऽर्कवर्णमितिः~॥}}
\end{quote}

\newpage

\begin{sloppypar}
\phantomsection \label{Ex 2.31}
\begin{quote}
\textbf{{\color{red}इतरस्य पञ्चमांशैश्चतुर्भिराद्येऽखिलेऽपि सम्मिलिते~।\\
रुद्रमितवर्णिका स्याद्वद मे वर्णौ च खोटौ च~॥}}
\end{quote}

न्यासः~। १२।११ जातौ हेमवर्णौ\, {\small $\begin{matrix}
\mbox{{७}}\\
\vspace{-1mm}
\mbox{{३}}
\vspace{1mm}
\end{matrix}$}~। {\small $\begin{matrix}
\mbox{{१४}}\\
\vspace{-1mm}
\mbox{{५}}
\vspace{1mm}
\end{matrix}$}~।\\

अत्र करणम्~। अंशौ\, $\frac{{\footnotesize{\hbox{२}}}}{{\footnotesize{\hbox{३}}}}$।$\frac{{\footnotesize{\hbox{४}}}}{{\footnotesize{\hbox{५}}}}$ अत्र छेदौ हेममाने ३।५, अंशौ खण्डसञ्ज्ञौ २।४,\, $\frac{{\footnotesize{\hbox{२}}}}{{\footnotesize{\hbox{३}}}}$।$\frac{{\footnotesize{\hbox{४}}}}{{\footnotesize{\hbox{५}}}}$ निज-हेमहतौ वर्णाविति हेममाने ३।५ आभ्यां वर्णौ १२।११ गुणितौ ३६।५५ स्वखण्डं २ परहेम ५ अनयोर्योगः ७~। परखण्डं ४ स्वहेम ३ अनयोर्योगः ७~। एवं जातौ योगौ ७।७ आभ्यामेतौ ३६।५५ गुणितौ २५२।३८५ स्वखण्ड\textendash \,२\textendash \,परवर्णौ ११ अनयोराहतिः २२~। परखण्ड\textendash \,४\textendash \,स्ववर्णौ १२ अनयोराहतिः ४८~। एवमाहती २२।४८~। निजहेमा\textendash \,३\textendash \,न्यखण्ड\textendash \,४\textendash \,युति\textendash \,७\textendash \,घातः १५४~। हेम\textendash \,५\textendash \,स्वखण्ड\textendash \,२\textendash \,युति\textendash \,७\textendash \,घातः ३३६~। एवमेतौ घातौ १५४।३३६ पूर्वराश्योरेतयोः २५२।३८५ क्रमेणापास्य शेषे ९८।४९ एते हेमवधः १५ खण्डवधः ८ अन्तरेण ७ भक्ते जाते १४।७ ज्ञाते व्यत्ययेन वर्णमाने ७।१४~। 

\begin{center}
\textbf{इति सुवर्णगणितम्~।}
\end{center}

\noindent \textbf{सूत्रम्~।}

\phantomsection \label{2.27.1}
\begin{quote}
\renewcommand{\thefootnote}{१}\footnote{{\color{violet}'नरघ्नदानोनितरत्नशेषैरिष्टे हृते स्युः खलु मूल्यसङ्ख्या'} इति {\color{violet}भास्करा}नुरूपमिदम्~।}{\large \textbf{{\color{purple}नरहतदानविहीनै \\
रत्नैरिष्टे विभाजिते मौल्यम्~।}}}
\end{quote}

\noindent \textbf{उदाहरणम्~।}

\phantomsection \label{Ex 2.32}
\begin{quote}
\textbf{{\color{red}नीलगोमेदवैदूर्यवज्राः सखे \\
सायकाङ्गाद्रिनागोन्मिताः स्युः क्रमात्~।\\
ते तु दत्त्वैकमेकं मिथः स्वाद्धनात् \\
तुल्यवित्ता मणेर्ब्रूहि मौल्यं द्रुतम्~॥}}
\end{quote}

न्यासः~। नी ५ गो ६ वै ७ व ८ दानम् १~। एकेनेष्टेन जातानि रत्न-
\end{sloppypar}

\newpage

\begin{sloppypar}
\noindent मूल्यानि\, {\small $\begin{matrix}
\mbox{{१~ १~ १~ १}}\\
\vspace{-1mm}
\mbox{{१~ २~ ३~ ४}}
\vspace{1mm}
\end{matrix}$}~। समधनं\, $\frac{{\footnotesize{\hbox{३७}}}}{{\footnotesize{\hbox{१२}}}}$~। द्वादशमितेनेष्टेन जातान्यभिन्नानि १२।६।४।३~।~सम-धनम् ३७~। एवमिष्टशादनेकधा~।\\

\noindent \textbf{सूत्रम्~।}

\phantomsection \label{2.27}
\begin{quote}
\renewcommand{\thefootnote}{१}\footnote{अत्रोपपत्तिः~। यदि वणिजां क्रमेण प्र, द्वि, तृ, च, धनानि तदा प्रश्नानुसारेण~।
\vspace{1mm}

\hspace{2mm} प्रगु.प्र $+$ द्वि $+$ तृ $+$ च $+$ ... $=$ रत्नमूल्यम्~।
\vspace{1mm}

\hspace{2mm} $=$ प्र $+$ द्विगु.द्वि $+$ तृ $+$ च $+$ ... $=$ रत्नम्~।
\vspace{1mm}

\hspace{2mm} सर्वाणि समानि, अतः प्र $+$ द्वि $+$ तृ $+$ च, शोधनेन
\vspace{1mm}

\hspace{2mm} प्र (प्रगु $-$ १) $=$ द्वि (द्विगु $-$ १) $=$ तृ (तृगु $-$ १) $=$ च (चगु $-$ १) $=$ इ
\vspace{2mm}

\hspace{2mm} अतः~~ प्र $= \dfrac{{\footnotesize{\hbox{इ}}}}{{\footnotesize{\hbox{प्रगु}} - {\hbox{१}}}}$, द्वि $= \dfrac{{\footnotesize{\hbox{इ}}}}{{\footnotesize{\hbox{द्विगु}} - {\hbox{१}}}}$, तृ $= \dfrac{{\footnotesize{\hbox{इ}}}}{{\footnotesize{\hbox{तृगु}} - {\hbox{१}}}}$, च $= \dfrac{{\footnotesize{\hbox{इ}}}}{{\footnotesize{\hbox{चगु}} - {\hbox{१}}}}$\; इत्युपपन्नम्~।}{\large \textbf{{\color{purple}रूपोनितगुणशेषैः \\
इष्टे भक्ते पृथग्धनानि स्युः~॥~२७~॥}}}
\end{quote}

\noindent \textbf{उदाहरणम्~।}

\phantomsection \label{Ex 2.33}
\begin{quote}
\textbf{{\color{red}चत्वारो वणिजो गता जलनिधिं तस्मिंश्च तान् पृष्टवान्\\
रत्नस्यास्य च नायकः किमिति भो मौल्यं तदैक्यं वद~।\\
एषां सर्वधनेन मे त्रिगुणितं युक्तं च मौल्यं मणेः\\
अन्येऽङ्घ्रीषुरसाहतं त्विति जगुस्तेषां धनं किं वद~॥}}
\end{quote}

गुणाः ३।४।५।६ एकेनेष्टेन जातानि पृथग्धनानि\, $\frac{{\footnotesize{\hbox{१}}}}{{\footnotesize{\hbox{२}}}}$।$\frac{{\footnotesize{\hbox{१}}}}{{\footnotesize{\hbox{३}}}}$।$\frac{{\footnotesize{\hbox{१}}}}{{\footnotesize{\hbox{४}}}}$।$\frac{{\footnotesize{\hbox{१}}}}{{\footnotesize{\hbox{५}}}}$~। रत्नमूल्यम्\, $\frac{{\footnotesize{\hbox{१३७}}}}{{\footnotesize{\hbox{६०}}}}$~।~षष्टि-मितेनाभिन्नानि~। ३०।२०।१५।१२ रत्नमौल्यम् १३७~।

\end{sloppypar}
\newpage

\noindent \textbf{सूत्रम्~।}

\phantomsection \label{2.28}
\begin{quote}
\renewcommand{\thefootnote}{१}\footnote{अत्रोपपत्तिः~। अत्र यदि नराणां क्रमेण धनानि प्र, द्वि, तृ, च, तदा प्रश्नानुसारेण
\vspace{2mm}

\hspace{8mm} ${\hbox{भा}}_{\scriptsize{\hbox{१}}} =$ द्वि $+$ तृ $+$ च
\vspace{1mm}

\hspace{8mm} ${\hbox{भा}}_{\scriptsize{\hbox{२}}} =$ प्र $+$ तृ $+$ च
\vspace{1mm}

\hspace{8mm} ${\hbox{भा}}_{\scriptsize{\hbox{३}}} =$ प्र $+$ द्वि $+$ च
\vspace{1mm}

\hspace{8mm} ${\hbox{भा}}_{\scriptsize{\hbox{४}}} =$ प्र $+$ द्वि $+$ तृ
\vspace{2mm}

\hspace{2mm} सर्वयोगेन,~~ ${\hbox{भा}}_{\scriptsize{\hbox{१}}} + {\hbox{भा}}_{\scriptsize{\hbox{२}}} + {\hbox{भा}}_{\scriptsize{\hbox{३}}} + {\hbox{भा}}_{\scriptsize{\hbox{४}}} =$ व्येन (प्र $+$ द्वि $+$
तृ $+$ च)
\vspace{2mm}

\hspace{2mm} $\therefore$\; प्र $+$ द्वि $+$ तृ $+$ च $=$ यो $= \dfrac{{\footnotesize{\hbox{भा}}_{\scriptsize{\hbox{१}}} + {\hbox{भा}}_{\scriptsize{\hbox{२}}} + {\hbox{भा}}_{\scriptsize{\hbox{३}}} + {\hbox{भा}}_{\scriptsize{\hbox{४}}}}}{{\footnotesize{\hbox{व्येन}}}}$
\vspace{2mm}

\hspace{2mm} तथा~~ प्र $=$ यो $- {\hbox{भा}}_{\scriptsize{\hbox{१}}}$~।\; द्वि $=$ यो $- {\hbox{भा}}_{\scriptsize{\hbox{२}}}$~।\; तृ $=$ यो $- {\hbox{भा}}_{\scriptsize{\hbox{३}}}$~।\; च $=$ यो $- {\hbox{भा}}_{\scriptsize{\hbox{४}}}$~।
\vspace{2mm}

\hspace{2mm} अत उपपद्यते सर्वम्~।}{\large \textbf{{\color{purple}वञ्चितभाण्डसमासे \\
व्येकनराप्ते प्रजायते योगः~।\\
तस्मिन्नुक्तविहीने \\
(पृथग्धनानि प्रजायन्ते)~॥~२८~॥}}}
\end{quote}

\noindent \textbf{उदाहरणम्~।}

\phantomsection \label{Ex 2.34}
\begin{quote}
\textbf{{\color{red}(हययूथे ये मिलिता नृणां चतुर्णां तुरङ्गमास्तेषाम्~।\\
शुल्कार्थं मे सङ्ख्याः पृथक् पृथक् तूर्णमाचक्ष्व~॥)\\
इति शुल्किकेन पृष्टा निजनिजतुरगान् विगृह्य जगुः~।\\
पञ्चदशादिद्व्यधिकास्तदा सखे कति हयास्तेषाम्~॥}}
\end{quote}

न्यासः~। १५।१७।१९।२१ सर्वतुरगाः २४~। पृथक् पृथक् तुरगाः ९।७।५।३

\newpage

\noindent \textbf{सूत्रम्~।}

\phantomsection \label{2.29}
\begin{quote}
\renewcommand{\thefootnote}{१}\footnote{अत्रोपपत्तिः~। कल्प्यते वणिजां धनानि\, प्र, द्वि, तृ, च,\; निधिमानम् $=$ नि\; तदा प्रश्नानुसारेण
\vspace{2mm}

\hspace{4mm} नि $+$ प्र $= {\hbox{गु}}_{\scriptsize{\hbox{१}}}$\,(द्वि $+$ तृ $+$ च)~।~~ नि $+$ द्वि $= {\hbox{गु}}_{\scriptsize{\hbox{२}}}$\,(प्र $+$ तृ $+$ च)~।
\vspace{1mm}

\hspace{4mm} नि $+$ तृ $= {\hbox{गु}}_{\scriptsize{\hbox{३}}}$\,(प्र $+$ द्वि $+$ च)~।~~ नि $+$ च $= {\hbox{गु}}_{\scriptsize{\hbox{४}}}$\,(प्र $+$ द्वि $+$
तृ)~।
\vspace{2mm}

\hspace{2mm} ततः~~ नि $+$ प्र $+$ द्वि $+$ तृ $+$ च $=$ नि $+$ धयो $=$ (${\hbox{गु}}_{\scriptsize{\hbox{१}}} +$ १)\;(द्वि $+$ तृ $+$ च)
\vspace{2mm}

\hspace{2mm} अतः~~ $\dfrac{{\footnotesize{\hbox{नि}} + {\hbox{धयो}}}}{{\footnotesize{\hbox{गु}}_{\scriptsize{\hbox{१}}} + {\hbox{१}}}} =$ द्वि $+$ तृ $+$ च $= {\hbox{भा}}_{\scriptsize{\hbox{१}}}$
\vspace{2mm}

\hspace{2mm} एवम्~~ $\dfrac{{\footnotesize{\hbox{नि}} + {\hbox{धयो}}}}{{\footnotesize{\hbox{गु}}_{\scriptsize{\hbox{२}}} + {\hbox{१}}}} =$ प्र $+$ तृ $+$ च $= {\hbox{भा}}_{\scriptsize{\hbox{२}}}$
\vspace{2mm}

\hspace{10mm} $\dfrac{{\footnotesize{\hbox{नि}} + {\hbox{धयो}}}}{{\footnotesize{\hbox{गु}}_{\scriptsize{\hbox{३}}} + {\hbox{१}}}} =$ प्र $+$ द्वि $+$ च $= {\hbox{भा}}_{\scriptsize{\hbox{३}}}$
\vspace{2mm}

\hspace{10mm} $\dfrac{{\footnotesize{\hbox{नि}} + {\hbox{धयो}}}}{{\footnotesize{\hbox{गु}}_{\scriptsize{\hbox{४}}} + {\hbox{१}}}} =$ प्र $+$ द्वि $+$ तृ $= {\hbox{भा}}_{\scriptsize{\hbox{४}}}$
\vspace{2mm}

\hspace{2mm} एषां योगेन~~ ${\hbox{भा}}_{\scriptsize{\hbox{१}}} + {\hbox{भा}}_{\scriptsize{\hbox{२}}} + {\hbox{भा}}_{\scriptsize{\hbox{३}}} + {\hbox{भा}}_{\scriptsize{\hbox{४}}} =$ व्येन (प्र $+$ द्वि $+$
तृ $+$ च) $=$ व्येन $\times$ धयो
\vspace{2mm}

\hspace{6mm} $\therefore\; \dfrac{{\footnotesize{\hbox{भा}}_{\scriptsize{\hbox{१}}} + {\hbox{भा}}_{\scriptsize{\hbox{२}}} + {\hbox{भा}}_{\scriptsize{\hbox{३}}} + {\hbox{भा}}_{\scriptsize{\hbox{४}}}}}{{\footnotesize{\hbox{व्येन}}}} =$ धयो~। 
\vspace{2mm}

\hspace{2mm} अत्र\; नि $+$ धयो $=$ इष्टं\; कल्पितमाचार्येण ततो ${\hbox{भा}}_{\scriptsize{\hbox{१}}}, {\hbox{भा}}_{\scriptsize{\hbox{२}}}$ इत्यादि वञ्चितभाण्डरूपधनानि संसाध्य पूर्व-सूत्रेण धनयोगमानं साधितम्~। ततः प्र $=$ धयो $- {\hbox{भा}}_{\scriptsize{\hbox{१}}}$, द्वि $=$ धयो $- {\hbox{भा}}_{\scriptsize{\hbox{२}}}$~। एवं सर्वेषां धनमानान्यानीतानि~। अथ\; नि $+$ धयो $=$ इ~~ $\therefore\;$ नि $=$ इ $-$ धयो~।}{\large \textbf{{\color{purple}सैकगुणाप्तमभीष्टं \\
विचिन्त्य विधिना नृणां धनानि स्युः~।\\
तद्योगोनमभीष्टं \\
निधिमानं जायते नूनम्~॥~२९~॥}}}
\end{quote}

\newpage
\begin{sloppypar}

\noindent \textbf{उदाहरणम्~।}

\phantomsection \label{Ex 2.35}
\begin{quote}
\textbf{{\color{red}निधिः प्राप्तः पुंभिः क्वचिदपि चतुर्भिश्च पुरुषः \\
समाचष्टे चैको मम धनयुतोऽयं यदि निधिः~।\\
चतुर्घ्नं युष्माकं धनत इति चैवं शररसा-\\
द्रिसंनिघ्नं चान्ये जगुरिह पृथङ्मे वद धनम्~॥}}
\end{quote}

न्यासः~। गुणाः ४।५।६।७ अत्र सैकगुणादि ५।६।७।८ एभी रूपं पृथक् पृथग्भक्तं जातम्\, $\frac{{\footnotesize{\hbox{१}}}}{{\footnotesize{\hbox{५}}}}$\,।\,$\frac{{\footnotesize{\hbox{१}}}}{{\footnotesize{\hbox{६}}}}$\,।\,$\frac{{\footnotesize{\hbox{१}}}}{{\footnotesize{\hbox{७}}}}$\,।\,$\frac{{\footnotesize{\hbox{१}}}}{{\footnotesize{\hbox{८}}}}$~। अत्र वञ्चितभाण्डसमासे इत्यादिकरणेन जातानि चतुर्णां धनानि\, $\frac{{\footnotesize{\hbox{२९}}}}{{\footnotesize{\hbox{२५२०}}}}$\,।\,$\frac{{\footnotesize{\hbox{११३}}}}{{\footnotesize{\hbox{२५२०}}}}$\,।\,$\frac{{\footnotesize{\hbox{१७३}}}}{{\footnotesize{\hbox{२५२०}}}}$\,।\,$\frac{{\footnotesize{\hbox{२१८}}}}{{\footnotesize{\hbox{२५२०}}}}$~। एषां योगः\, $\frac{{\footnotesize{\hbox{५३३}}}}{{\footnotesize{\hbox{२५२०}}}}$~। एतदिष्टरूपादपास्य शेषं निधिमानम्\, $\frac{{\footnotesize{\hbox{१९८७}}}}{{\footnotesize{\hbox{२५२०}}}}$~। छेदसमेनेष्टेन गुणितानि जातान्यभिन्नानि २९।११३।१७३।२१८~। निधानम् १९८७~।\\

\noindent \textbf{विषमपोटले सूत्रम्~।}

\phantomsection \label{2.30}
\begin{quote}
\renewcommand{\thefootnote}{१}\footnote{अत्रोपपत्तिः~। कल्प्यते निधिमानम् = नि, नरसङ्ख्या = न, नराणां धनानि क्रमेण ${\hbox{ध}}_{\scriptsize{\hbox{१}}}, {\hbox{ध}}_{\scriptsize{\hbox{२}}}, {\hbox{ध}}_{\scriptsize{\hbox{३}}}, {\hbox{ध}}_{\scriptsize{\hbox{४}}}$~। निधिभागाश्च क्रमेण\, ${\hbox{भा}}_{\scriptsize{\hbox{१}}}, {\hbox{भा}}_{\scriptsize{\hbox{२}}}, {\hbox{भा}}_{\scriptsize{\hbox{३}}}, {\hbox{भा}}_{\scriptsize{\hbox{४}}}$~। गुणकाश्च\, ${\hbox{गु}}_{\scriptsize{\hbox{१}}}, {\hbox{गु}}_{\scriptsize{\hbox{२}}}, {\hbox{गु}}_{\scriptsize{\hbox{३}}}, {\hbox{गु}}_{\scriptsize{\hbox{४}}}$~। तदा प्रश्नोक्त्या
\vspace{1mm}

\hspace{6mm} ${\hbox{ध}}_{\scriptsize{\hbox{१}}} +$ नि.${\hbox{भा}}_{\scriptsize{\hbox{१}}} = {\hbox{गु}}_{\scriptsize{\hbox{१}}}\,({\hbox{ध}}_{\scriptsize{\hbox{२}}} + {\hbox{ध}}_{\scriptsize{\hbox{३}}} + {\hbox{ध}}_{\scriptsize{\hbox{४}}})$
\vspace{1mm}

\hspace{2mm} वा,~~ ${\hbox{ध}}_{\scriptsize{\hbox{१}}} + {\hbox{ध}}_{\scriptsize{\hbox{२}}} + {\hbox{ध}}_{\scriptsize{\hbox{३}}} + {\hbox{ध}}_{\scriptsize{\hbox{४}}} +$ नि.${\hbox{भा}}_{\scriptsize{\hbox{१}}} =$ यो $+$ नि.${\hbox{भा}}_{\scriptsize{\hbox{१}}}$}{\large \textbf{{\color{purple}परभागाः स्वगुणघ्ना \\
निजभागयुताः परैः सरूपैस्तैः~।\\
गुणकैर्विहृतो योगो \\
नेत्रोननराहतैः स्वांशैः~॥~३०~॥}}}
\end{quote}

\end{sloppypar}

\newpage

\phantomsection \label{2.31}
\begin{quote}
{\large \textbf{{\color{purple}हीनो निजगुणकेन च \\
सैकेन हृतो मुहुर्धनानि स्युः~।\\
परधनयोगो निजगुण-\\
केन हतः स्वस्ववर्जितो विभजेत्~॥~३१~॥\\
निजभागेन निधानं \\
प्रजायते विषमपूर्वं तत्~।\\
कृतसमहरलुप्तहराः \\
तेऽंशास्तेभ्यस्तु पूर्ववद्वापि~॥~३२~॥}}}\renewcommand{\thefootnote}{}\footnote{\hspace{2mm} $= ({\hbox{गु}}_{\scriptsize{\hbox{१}}} + {\hbox{१}})\,({\hbox{ध}}_{\scriptsize{\hbox{२}}} + {\hbox{ध}}_{\scriptsize{\hbox{३}}} + {\hbox{ध}}_{\scriptsize{\hbox{४}}})$
\vspace{2mm}

\hspace{2mm} अतः~~ $\dfrac{{\footnotesize{\hbox{यो}}}}{{\footnotesize{\hbox{गु}}_{\scriptsize{\hbox{१}}} + {\hbox{१}}}} + \dfrac{{\footnotesize{\hbox{नि.भा}}_{\scriptsize{\hbox{१}}}}}{{\footnotesize{\hbox{गु}}_{\scriptsize{\hbox{१}}} + {\hbox{१}}}} = {\hbox{ध}}_{\scriptsize{\hbox{२}}} + {\hbox{ध}}_{\scriptsize{\hbox{३}}} + {\hbox{ध}}_{\scriptsize{\hbox{४}}}$ ...... (१)
\vspace{2mm}

\hspace{2mm} एवम्~~ $\dfrac{{\footnotesize{\hbox{यो}}}}{{\footnotesize{\hbox{गु}}_{\scriptsize{\hbox{२}}} + {\hbox{१}}}} + \dfrac{{\footnotesize{\hbox{नि.भा}}_{\scriptsize{\hbox{२}}}}}{{\footnotesize{\hbox{गु}}_{\scriptsize{\hbox{२}}} + {\hbox{१}}}} =$ प्र $+$ तृ $+$ च
\vspace{2mm}

\hspace{10mm} $\dfrac{{\footnotesize{\hbox{यो}}}}{{\footnotesize{\hbox{गु}}_{\scriptsize{\hbox{३}}} + {\hbox{१}}}} + \dfrac{{\footnotesize{\hbox{नि.भा}}_{\scriptsize{\hbox{३}}}}}{{\footnotesize{\hbox{गु}}_{\scriptsize{\hbox{३}}} + {\hbox{१}}}} =$ प्र $+$ द्वि $+$ च
\vspace{2mm}

\hspace{10mm} $\dfrac{{\footnotesize{\hbox{यो}}}}{{\footnotesize{\hbox{गु}}_{\scriptsize{\hbox{४}}} + {\hbox{१}}}} + \dfrac{{\footnotesize{\hbox{नि.भा}}_{\scriptsize{\hbox{४}}}}}{{\footnotesize{\hbox{गु}}_{\scriptsize{\hbox{४}}} + {\hbox{१}}}} =$ प्र $+$ द्वि $+$ तृ
\vspace{2mm}

\hspace{2mm} सर्वयोगेन
\vspace{2mm}

\hspace{4mm} ${\hbox{यो}}\,\left(\dfrac{{\footnotesize{\hbox{१}}}}{{\footnotesize{\hbox{गु}}_{\scriptsize{\hbox{१}}} + {\hbox{१}}}} + \dfrac{{\footnotesize{\hbox{१}}}}{{\footnotesize{\hbox{गु}}_{\scriptsize{\hbox{२}}} + {\hbox{१}}}} + \dfrac{{\footnotesize{\hbox{१}}}}{{\footnotesize{\hbox{गु}}_{\scriptsize{\hbox{३}}} + {\hbox{१}}}} + \dfrac{{\footnotesize{\hbox{१}}}}{{\footnotesize{\hbox{गु}}_{\scriptsize{\hbox{४}}} + {\hbox{१}}}}\right)$
\vspace{2mm}

\hspace{12mm} $+\, {\hbox{नि}}\,\left(\dfrac{{\footnotesize{\hbox{भा}}_{\scriptsize{\hbox{१}}}}}{{\footnotesize{\hbox{गु}}_{\scriptsize{\hbox{१}}} + {\hbox{१}}}} + \dfrac{{\footnotesize{\hbox{भा}}_{\scriptsize{\hbox{२}}}}}{{\footnotesize{\hbox{गु}}_{\scriptsize{\hbox{२}}} + {\hbox{१}}}} + \dfrac{{\footnotesize{\hbox{भा}}_{\scriptsize{\hbox{३}}}}}{{\footnotesize{\hbox{गु}}_{\scriptsize{\hbox{३}}} + {\hbox{१}}}} + \dfrac{{\footnotesize{\hbox{भा}}_{\scriptsize{\hbox{४}}}}}{{\footnotesize{\hbox{गु}}_{\scriptsize{\hbox{४}}} + {\hbox{१}}}}\right)$
\vspace{2mm}

\hspace{18mm} $=$ (न $-$ १) यो
\vspace{2mm}

\hspace{2mm} $\therefore\; {\hbox{यो}}\,\left[({\hbox{न}} - {\hbox{१}}) - \dfrac{{\footnotesize{\hbox{१}}}}{{\footnotesize{\hbox{गु}}_{\scriptsize{\hbox{१}}} + {\hbox{१}}}} - \dfrac{{\footnotesize{\hbox{१}}}}{{\footnotesize{\hbox{गु}}_{\scriptsize{\hbox{२}}} + {\hbox{१}}}} - \dfrac{{\footnotesize{\hbox{१}}}}{{\footnotesize{\hbox{गु}}_{\scriptsize{\hbox{३}}} + {\hbox{१}}}} - \dfrac{{\footnotesize{\hbox{१}}}}{{\footnotesize{\hbox{गु}}_{\scriptsize{\hbox{४}}} + {\hbox{१}}}}\right]$}
\end{quote}

\newpage

\noindent \textbf{उदाहरणम्~।}\renewcommand{\thefootnote}{}\footnote{\hspace{2mm} $= {\hbox{नि}}\,\left(\dfrac{{\footnotesize{\hbox{भा}}_{\scriptsize{\hbox{१}}}}}{{\footnotesize{\hbox{गु}}_{\scriptsize{\hbox{१}}} + {\hbox{१}}}} + \dfrac{{\footnotesize{\hbox{भा}}_{\scriptsize{\hbox{२}}}}}{{\footnotesize{\hbox{गु}}_{\scriptsize{\hbox{२}}} + {\hbox{१}}}} + \dfrac{{\footnotesize{\hbox{भा}}_{\scriptsize{\hbox{३}}}}}{{\footnotesize{\hbox{गु}}_{\scriptsize{\hbox{३}}} + {\hbox{१}}}} + \dfrac{{\footnotesize{\hbox{भा}}_{\scriptsize{\hbox{४}}}}}{{\footnotesize{\hbox{गु}}_{\scriptsize{\hbox{४}}} + {\hbox{१}}}}\right)$
\vspace{2mm}

\hspace{2mm} ततः~~ यो $= {\hbox{नि}}\,\left(\dfrac{\dfrac{{\footnotesize{\hbox{भा}}_{\scriptsize{\hbox{१}}}}}{{\footnotesize{\hbox{गु}}_{\scriptsize{\hbox{१}}} + {\hbox{१}}}} + \dfrac{{\footnotesize{\hbox{भा}}_{\scriptsize{\hbox{२}}}}}{{\footnotesize{\hbox{गु}}_{\scriptsize{\hbox{२}}} + {\hbox{१}}}} + \dfrac{{\footnotesize{\hbox{भा}}_{\scriptsize{\hbox{३}}}}}{{\footnotesize{\hbox{गु}}_{\scriptsize{\hbox{३}}} + {\hbox{१}}}} + \dfrac{{\footnotesize{\hbox{भा}}_{\scriptsize{\hbox{४}}}}}{{\footnotesize{\hbox{गु}}_{\scriptsize{\hbox{४}}} + {\hbox{१}}}}}{({\hbox{न}} - {\hbox{१}}) - \dfrac{{\footnotesize{\hbox{१}}}}{{\footnotesize{\hbox{गु}}_{\scriptsize{\hbox{१}}} + {\hbox{१}}}} - \dfrac{{\footnotesize{\hbox{१}}}}{{\footnotesize{\hbox{गु}}_{\scriptsize{\hbox{२}}} + {\hbox{१}}}} - \dfrac{{\footnotesize{\hbox{१}}}}{{\footnotesize{\hbox{गु}}_{\scriptsize{\hbox{३}}} + {\hbox{१}}}} - \dfrac{{\footnotesize{\hbox{१}}}}{{\footnotesize{\hbox{गु}}_{\scriptsize{\hbox{४}}} + {\hbox{१}}}}}\right)$
\vspace{2mm}

\hspace{2mm} (१) समीकरणेऽस्योत्थापनेन तथा\, ${\hbox{ध}}_{\scriptsize{\hbox{२}}} + {\hbox{ध}}_{\scriptsize{\hbox{३}}} + {\hbox{ध}}_{\scriptsize{\hbox{४}}}$\, एतन्मानं योगादपास्य जातम्
\vspace{2mm}

\hspace{2mm} ${\hbox{ध}}_{\scriptsize{\hbox{१}}} = \dfrac{{\footnotesize{\hbox{नि}}}}{{\footnotesize{\hbox{ह}}({\hbox{गु}}_{\scriptsize{\hbox{१}}} + {\hbox{१}})}}\,\left[\dfrac{{\footnotesize{\hbox{भा}}_{\scriptsize{\hbox{२}}}.{\hbox{गु}}_{\scriptsize{\hbox{१}}} + {\hbox{भा}}_{\scriptsize{\hbox{१}}}}}{{\footnotesize{\hbox{गु}}_{\scriptsize{\hbox{२}}} + {\hbox{१}}}} + \dfrac{{\footnotesize{\hbox{भा}}_{\scriptsize{\hbox{३}}}.{\hbox{गु}}_{\scriptsize{\hbox{१}}} + {\hbox{भा}}_{\scriptsize{\hbox{१}}}}}{{\footnotesize{\hbox{गु}}_{\scriptsize{\hbox{३}}} + {\hbox{१}}}} + \dfrac{{\footnotesize{\hbox{भा}}_{\scriptsize{\hbox{४}}}.{\hbox{गु}}_{\scriptsize{\hbox{१}}} + {\hbox{भा}}_{\scriptsize{\hbox{१}}}}}{{\footnotesize{\hbox{गु}}_{\scriptsize{\hbox{४}}} + {\hbox{१}}}} - {\hbox{भा}}_{\scriptsize{\hbox{१}}}\,({\hbox{न}} - {\hbox{२}})\right]$
\vspace{2mm}

\hspace{2mm} तत आलापानुसारेण व्यस्तविधिना निधिमानं जायते~। अत्र ${\hbox{ध}}_{\scriptsize{\hbox{१}}}, {\hbox{ध}}_{\scriptsize{\hbox{२}}}$, मानानयनार्थमाचार्येण प्रथमं निधिमानं हरसमं प्रकल्पितम्~। ततोऽभिन्नार्थं यथेच्छमिष्टगुणानि तानि बहुधा नराणां धनमानानि स्युरित्युपपद्यते सर्वम्~।
\vspace{1mm}
}

\phantomsection \label{Ex 2.36}
\begin{quote}
\textbf{{\color{red}प्राप्तं निधानं धनिभिश्चतुर्भिः \\
\renewcommand{\thefootnote}{$\star$}\footnote{The reading तत्पञ्चषट्-सप्त seems to be an error, as it gives incorrect results.}तत्सप्तषट्-पञ्चचतुर्थभागैः~।\\
पृथग्युतास्ते परवित्तयोगात् \\
द्वित्र्यब्धिपञ्चप्रगुणश्च ते स्युः~॥}}
\end{quote}

न्यासः~। गुणाः २$\frac{{\footnotesize{\hbox{१}}}}{{\footnotesize{\hbox{७}}}}$\,।\,३$\frac{{\footnotesize{\hbox{१}}}}{{\footnotesize{\hbox{६}}}}$\,।\,४$\frac{{\footnotesize{\hbox{१}}}}{{\footnotesize{\hbox{५}}}}$\,।\,५$\frac{{\footnotesize{\hbox{१}}}}{{\footnotesize{\hbox{४}}}}$\,। \\

अत्र करणम्~। परभागाः\, $\frac{{\footnotesize{\hbox{१}}}}{{\footnotesize{\hbox{६}}}}$\,।\,$\frac{{\footnotesize{\hbox{१}}}}{{\footnotesize{\hbox{५}}}}$\,।\,$\frac{{\footnotesize{\hbox{१}}}}{{\footnotesize{\hbox{४}}}}$\, स्वगुणघ्नाः

\newpage

\begin{sloppypar}
\noindent $\frac{{\footnotesize{\hbox{२}}}}{{\footnotesize{\hbox{६}}}}$\,।\,$\frac{{\footnotesize{\hbox{२}}}}{{\footnotesize{\hbox{५}}}}$\,।\,$\frac{{\footnotesize{\hbox{२}}}}{{\footnotesize{\hbox{४}}}}$\, निजभाग\textendash \,$\frac{{\footnotesize{\hbox{१}}}}{{\footnotesize{\hbox{७}}}}$\textendash \,युताः\, $\frac{{\footnotesize{\hbox{२०}}}}{{\footnotesize{\hbox{४२}}}}$\,।\,$\frac{{\footnotesize{\hbox{१९}}}}{{\footnotesize{\hbox{३५}}}}$\,।\,$\frac{{\footnotesize{\hbox{१८}}}}{{\footnotesize{\hbox{२८}}}}$~। एते परैः स्वरूपैर्गुणकैः ४।५।६ भक्ता जाताः\, $\frac{{\footnotesize{\hbox{२०}}}}{{\footnotesize{\hbox{१६८}}}}$\,।\,$\frac{{\footnotesize{\hbox{१९}}}}{{\footnotesize{\hbox{१७५}}}}$\,।\,$\frac{{\footnotesize{\hbox{१८}}}}{{\footnotesize{\hbox{१६८}}}}$~। एषां योगः\,$\frac{{\footnotesize{\hbox{१४०६}}}}{{\footnotesize{\hbox{४२००}}}}$~। निजगुणकेन सैकेन ३ हृतो जातः\, $\frac{{\footnotesize{\hbox{७०३}}}}{{\footnotesize{\hbox{६३००}}}}$~। एतत् प्रथमधनम्~। एवमन्येषां त्रयाणां धनानि\, $\frac{{\footnotesize{\hbox{७१९}}}}{{\footnotesize{\hbox{१६८००}}}}$\,।\,$\frac{{\footnotesize{\hbox{२३}}}}{{\footnotesize{\hbox{४२०}}}}$\,।\,$\frac{{\footnotesize{\hbox{११५}}}}{{\footnotesize{\hbox{२०१६}}}}$~। परधनयोगः\,$\frac{{\footnotesize{\hbox{४८७}}}}{{\footnotesize{\hbox{३१५०}}}}$~। निजगुणकेनानेन २ हतः\,$\frac{{\footnotesize{\hbox{४८७}}}}{{\footnotesize{\hbox{१५७५}}}}$~। \hyperref[2.31]{'स्वस्ववर्जितः'} इति निजधनेनानेन\, $\frac{{\footnotesize{\hbox{७०३}}}}{{\footnotesize{\hbox{६३००}}}}$\, वर्जितः\, $\frac{{\footnotesize{\hbox{४१}}}}{{\footnotesize{\hbox{१४०}}}}$~। भिन्नभागेनानेन\, $\frac{{\footnotesize{\hbox{१}}}}{{\footnotesize{\hbox{७}}}}$\, हृतो जातो निजधनम्\, $\frac{{\footnotesize{\hbox{४१}}}}{{\footnotesize{\hbox{२०}}}}$~। एवं पृथक् पृथक् परधनेभ्योऽपि निधानं सममेव~। अभिन्नार्थं समच्छेदी कृताः~। $\frac{{\footnotesize{\hbox{८२४}}}}{{\footnotesize{\hbox{५०४००}}}}$\,।\,$\frac{{\footnotesize{\hbox{२१५७}}}}{{\footnotesize{\hbox{५०४००}}}}$\,।\,$\frac{{\footnotesize{\hbox{२७६०}}}}{{\footnotesize{\hbox{५०४००}}}}$\,।\,$\frac{{\footnotesize{\hbox{२८७५}}}}{{\footnotesize{\hbox{५०४००}}}}$\,। निधानं च १०३३२०~। छेदसमेनेष्टेन जातान्यभिन्नानि ८२४।२१५७।२७६०।२८७५~। निधानं च १०३३२०~॥~अथवांशाः\, $\frac{{\footnotesize{\hbox{१}}}}{{\footnotesize{\hbox{७}}}}$\,।\,$\frac{{\footnotesize{\hbox{१}}}}{{\footnotesize{\hbox{६}}}}$\,।\,$\frac{{\footnotesize{\hbox{१}}}}{{\footnotesize{\hbox{५}}}}$\,।\,$\frac{{\footnotesize{\hbox{१}}}}{{\footnotesize{\hbox{४}}}}$\, कृतसमहरलोपाः ६०।७०।८४।१०५~। एभ्यः पूर्ववत् तान्येव धनानि साध्यानि~।\\
\end{sloppypar}

\noindent \textbf{सूत्रम्~।}

\phantomsection \label{2.33}
\begin{quote}
\renewcommand{\thefootnote}{१}\footnote{अत्रोपपत्तिः~। यदि पूर्ववत् ${\hbox{ध}}_{\scriptsize{\hbox{१}}}, {\hbox{ध}}_{\scriptsize{\hbox{२}}}, {\hbox{ध}}_{\scriptsize{\hbox{३}}}, {\hbox{ध}}_{\scriptsize{\hbox{४}}}$...\,धनमानानि ${\hbox{गु}}_{\scriptsize{\hbox{१}}}, {\hbox{गु}}_{\scriptsize{\hbox{२}}}$...\,गुणमानानि तथा प्रथमस्य प्राप्त-वित्तमाने ${\hbox{प्रा}}_{\scriptsize{\hbox{१}}}, {\hbox{प्रा'}}_{\scriptsize{\hbox{१}}}$, द्वितीयस्य ${\hbox{प्रा}}_{\scriptsize{\hbox{२}}}, {\hbox{प्रा'}}_{\scriptsize{\hbox{२}}}$, तृतीयस्य ${\hbox{प्रा}}_{\scriptsize{\hbox{३}}}, {\hbox{प्रा'}}_{\scriptsize{\hbox{३}}}$ तदा प्रश्नोक्त्या}{\large \textbf{{\color{purple}प्राप्तान्विताः सरूपैः\\
गुणकैर्निहतास्तु\renewcommand{\thefootnote}{$\star$}\footnote{The reading -हितास्तु seems to be a typographical error.} विषमपोटलवत्~।\\
निजनिजगुणकाः सैकैः\\
गुणकैर्विहृताश्च तद्योगः~॥~३३~॥}}}
\end{quote}

\newpage

\phantomsection \label{2.34.1}
\begin{quote}
{\large \textbf{{\color{purple}रूपोनेन हृताः स्युः \\
धनानि तेषां पृथक् (पृथक्) तानि~।}}}\renewcommand{\thefootnote}{}\footnote{\hspace{2mm} ${\hbox{ध}}_{\scriptsize{\hbox{१}}} + {\hbox{प्रा}}_{\scriptsize{\hbox{१}}} + {\hbox{प्रा'}}_{\scriptsize{\hbox{१}}} = {\hbox{गु}}_{\scriptsize{\hbox{१}}}\,[{\hbox{ध}}_{\scriptsize{\hbox{२}}} + {\hbox{ध}}_{\scriptsize{\hbox{३}}} - ({\hbox{प्रा}}_{\scriptsize{\hbox{१}}} + {\hbox{प्रा'}}_{\scriptsize{\hbox{१}}})]$
\vspace{1mm}

\hspace{2mm} समशोधनेन,~~ ${\hbox{ध}}_{\scriptsize{\hbox{१}}} + ({\hbox{गु}}_{\scriptsize{\hbox{१}}} + {\hbox{१}})\,({\hbox{प्रा}}_{\scriptsize{\hbox{१}}} + {\hbox{प्रा'}}_{\scriptsize{\hbox{१}}}) = {\hbox{गु}}_{\scriptsize{\hbox{१}}}\,({\hbox{ध}}_{\scriptsize{\hbox{२}}} + {\hbox{ध}}_{\scriptsize{\hbox{३}}})$
\vspace{1mm}

\hspace{2mm} वा,~~ ${\hbox{ध}}_{\scriptsize{\hbox{१}}} + {\hbox{ध}}_{\scriptsize{\hbox{२}}} + {\hbox{ध}}_{\scriptsize{\hbox{३}}} + ({\hbox{गु}}_{\scriptsize{\hbox{१}}} + {\hbox{१}})\,({\hbox{प्रा}}_{\scriptsize{\hbox{१}}} + {\hbox{प्रा'}}_{\scriptsize{\hbox{१}}}) = ({\hbox{गु}}_{\scriptsize{\hbox{१}}} + {\hbox{१}})\,\,({\hbox{ध}}_{\scriptsize{\hbox{२}}} + {\hbox{ध}}_{\scriptsize{\hbox{३}}})$
\vspace{2mm}

\hspace{2mm} $\therefore\; \dfrac{{\footnotesize{\hbox{यो}}}}{{\footnotesize{\hbox{गु}}_{\scriptsize{\hbox{१}}} + {\hbox{१}}}} + \dfrac{{\footnotesize({\hbox{गु}}_{\scriptsize{\hbox{१}}} + {\hbox{१}})\,({\hbox{प्रा}}_{\scriptsize{\hbox{१}}} + {\hbox{प्रा'}}_{\scriptsize{\hbox{१}}})}}{{\footnotesize{\hbox{गु}}_{\scriptsize{\hbox{१}}} + {\hbox{१}}}} = {\hbox{ध}}_{\scriptsize{\hbox{२}}} + {\hbox{ध}}_{\scriptsize{\hbox{३}}}$
\vspace{2mm}

\hspace{2mm} अतो यदि पूर्वसूत्रोपपत्तौ\, नि $=$ १,\, (गु $+$ १)\,$({\hbox{प्रा}}_{\scriptsize{\hbox{१}}} + {\hbox{प्रा'}}_{\scriptsize{\hbox{१}}}) = {\hbox{भ}}_{\scriptsize{\hbox{१}}}$,\, (गु $+$ १)\,$({\hbox{प्रा}}_{\scriptsize{\hbox{२}}} + {\hbox{प्रा'}}_{\scriptsize{\hbox{२}}}) = {\hbox{भ}}_{\scriptsize{\hbox{२}}}$\, इत्यादि कल्प्यते तदा धनमानम्\textendash
\vspace{2mm}

\hspace{2mm} ${\hbox{ध}}_{\scriptsize{\hbox{१}}} = \dfrac{{\footnotesize{\hbox{१}}}}{{\footnotesize{\hbox{ह}}}} \times \dfrac{{\footnotesize{\hbox{१}}}}{{\footnotesize{\hbox{गु}}_{\scriptsize{\hbox{१}}} + {\hbox{१}}}}\,\left[\dfrac{{\footnotesize{\hbox{भा}}_{\scriptsize{\hbox{२}}}.{\hbox{गु}}_{\scriptsize{\hbox{१}}} + {\hbox{भा}}_{\scriptsize{\hbox{१}}}}}{{\footnotesize{\hbox{गु}}_{\scriptsize{\hbox{२}}} + {\hbox{१}}}} + \dfrac{{\footnotesize{\hbox{भा}}_{\scriptsize{\hbox{३}}}.{\hbox{गु}}_{\scriptsize{\hbox{१}}} + {\hbox{भा}}_{\scriptsize{\hbox{१}}}}}{{\footnotesize{\hbox{गु}}_{\scriptsize{\hbox{३}}} + {\hbox{१}}}} - {\hbox{भा}}_{\scriptsize{\hbox{१}}}\,({\hbox{न}} - {\hbox{२}})\right]$
\vspace{2mm}

\hspace{2mm} अथ पूर्वसाधितो हरः $=$ ह $= ({\hbox{न}} - {\hbox{१}}) - \dfrac{{\footnotesize{\hbox{१}}}}{{\footnotesize{\hbox{गु}}_{\scriptsize{\hbox{१}}} + {\hbox{१}}}} - \dfrac{{\footnotesize{\hbox{१}}}}{{\footnotesize{\hbox{गु}}_{\scriptsize{\hbox{२}}} + {\hbox{१}}}} - \dfrac{{\footnotesize{\hbox{१}}}}{{\footnotesize{\hbox{गु}}_{\scriptsize{\hbox{३}}} + {\hbox{१}}}}$
\vspace{2mm}

\hspace{32mm} $= {\hbox{१}} - \dfrac{{\footnotesize{\hbox{१}}}}{{\footnotesize{\hbox{गु}}_{\scriptsize{\hbox{१}}} + {\hbox{१}}}} + {\hbox{१}} - \dfrac{{\footnotesize{\hbox{१}}}}{{\footnotesize{\hbox{गु}}_{\scriptsize{\hbox{२}}} + {\hbox{१}}}} + {\hbox{१}} - \dfrac{{\footnotesize{\hbox{१}}}}{{\footnotesize{\hbox{गु}}_{\scriptsize{\hbox{३}}} + {\hbox{१}}}} - {\hbox{१}}$
\vspace{2mm}

\hspace{32mm} $= \dfrac{{\footnotesize{\hbox{गु}}_{\scriptsize{\hbox{१}}}}}{{\footnotesize{\hbox{गु}}_{\scriptsize{\hbox{१}}} + {\hbox{१}}}} + \dfrac{{\footnotesize{\hbox{गु}}_{\scriptsize{\hbox{२}}}}}{{\footnotesize{\hbox{गु}}_{\scriptsize{\hbox{२}}} + {\hbox{१}}}} + \dfrac{{\footnotesize{\hbox{गु}}_{\scriptsize{\hbox{३}}}}}{{\footnotesize{\hbox{गु}}_{\scriptsize{\hbox{३}}} + {\hbox{१}}}} - {\hbox{१}}$~।\; अत उपपद्यते सर्वम्~।}
\end{quote}

\noindent \textbf{उदाहरणम्~।}

\phantomsection \label{Ex 2.37}
\begin{quote}
\textbf{{\color{red}पुरुषास्त्रयोऽपि वणिजस्तेषु प्रथमः प्रवक्ति सोत्साहम्~।\\
यदि यच्छतो युवां मे षड्वाष्टौ द्विगुणितोऽस्म्यहं युवयोः~॥\\
अन्यो मेऽष्टौ सप्त प्रयच्छतस्त्रिगुणितो भवामीति~।\\
सप्तनवैव तथान्यं पञ्चगुणोऽस्मीति वित्तता युवयोः~॥\\
तेषां धनानि वद यदि गणितेऽहङ्कारता तेऽस्ति~॥}}
\end{quote}

न्यासः~। गुणाः\, {\small $\begin{matrix}
\mbox{{२}}\\
\vspace{-1mm}
\mbox{{$\frac{{\footnotesize{\hbox{६}}}}{{\footnotesize{\hbox{८}}}}$}}
\vspace{1mm}
\end{matrix}$}~। {\small $\begin{matrix}
\mbox{{३}}\\
\vspace{-1mm}
\mbox{{$\frac{{\footnotesize{\hbox{८}}}}{{\footnotesize{\hbox{७}}}}$}}
\vspace{1mm}
\end{matrix}$}~। {\small $\begin{matrix}
\mbox{{५}}\\
\vspace{-1mm}
\mbox{{$\frac{{\footnotesize{\hbox{७}}}}{{\footnotesize{\hbox{९}}}}$}}
\vspace{1mm}
\end{matrix}$}~। अत्र करणम्~। स्वप्राप्तवित्तयोगहतिः प्राप्तयोगः १४।१५।१६~। सैकगुणाः प्राप्ताः ३।४।६~।

\newpage

हताः ४२।६०।९६~। एभ्यः पोटलवद्धनानि\, $\frac{{\footnotesize{\hbox{२५}}}}{{\footnotesize{\hbox{२}}}}$\,।\,$\frac{{\footnotesize{\hbox{१५}}}}{{\footnotesize{\hbox{१}}}}$\,।\,$\frac{{\footnotesize{\hbox{३५}}}}{{\footnotesize{\hbox{१२}}}}$\,।\\

गुणाः २।३।५~। सैकगुणाः\, ३।४।६\, एभ्यो भक्ताः\, $\frac{{\footnotesize{\hbox{२}}}}{{\footnotesize{\hbox{३}}}}$\,।\,$\frac{{\footnotesize{\hbox{३}}}}{{\footnotesize{\hbox{४}}}}$\,।\,$\frac{{\footnotesize{\hbox{५}}}}{{\footnotesize{\hbox{६}}}}$\,।\\

एषां योगः\, $\frac{{\footnotesize{\hbox{९}}}}{{\footnotesize{\hbox{४}}}}$~। रूपोनेनानेन\, $\frac{{\footnotesize{\hbox{५}}}}{{\footnotesize{\hbox{४}}}}$\, पूर्वधनान्येतानि\, $\frac{{\footnotesize{\hbox{२५}}}}{{\footnotesize{\hbox{२}}}}$।$\frac{{\footnotesize{\hbox{१५}}}}{{\footnotesize{\hbox{१}}}}$।$\frac{{\footnotesize{\hbox{३५}}}}{{\footnotesize{\hbox{१२}}}}$\, भक्तानि जातानि धनानि १०।१२।१४~।\\

\noindent \textbf{सूत्रम्~।}

\phantomsection \label{2.34}
\begin{quote}
\renewcommand{\thefootnote}{१}\footnote{अत्रोपपत्तिः~। अत्र पक्षिणः क्रमेण\, ${\hbox{प}}_{\scriptsize{\hbox{१}}}, {\hbox{प}}_{\scriptsize{\hbox{२}}}, {\hbox{प}}_{\scriptsize{\hbox{३}}}$,...\;तन्मूल्यानि\, ${\hbox{मू}}_{\scriptsize{\hbox{१}}}, {\hbox{मू}}_{\scriptsize{\hbox{२}}}, {\hbox{मू}}_{\scriptsize{\hbox{३}}}$,... इच्छापक्षियोगः $=$ पयो~। इच्छाद्रम्मयोगः $=$ द्रयो~। जीवमूल्यगुणकाश्च क्रमेण\, ${\hbox{गु}}_{\scriptsize{\hbox{१}}}, {\hbox{गु}}_{\scriptsize{\hbox{२}}}, {\hbox{गु}}_{\scriptsize{\hbox{३}}}$,... तदा प्रश्नोक्त्या
\vspace{1mm}

\hspace{4mm} ${\hbox{प}}_{\scriptsize{\hbox{१}}}.{\hbox{गु}}_{\scriptsize{\hbox{१}}} + {\hbox{प}}_{\scriptsize{\hbox{२}}}.{\hbox{गु}}_{\scriptsize{\hbox{२}}} + {\hbox{प}}_{\scriptsize{\hbox{३}}}.{\hbox{गु}}_{\scriptsize{\hbox{३}}} + {\hbox{प}}_{\scriptsize{\hbox{४}}}.{\hbox{गु}}_{\scriptsize{\hbox{४}}} =$ पयो
\vspace{1mm}

\hspace{4mm} ${\hbox{मू}}_{\scriptsize{\hbox{१}}}.{\hbox{गु}}_{\scriptsize{\hbox{१}}} + {\hbox{मू}}_{\scriptsize{\hbox{२}}}.{\hbox{गु}}_{\scriptsize{\hbox{२}}} + {\hbox{मू}}_{\scriptsize{\hbox{३}}}.{\hbox{गु}}_{\scriptsize{\hbox{३}}} + {\hbox{मू}}_{\scriptsize{\hbox{४}}}.{\hbox{गु}}_{\scriptsize{\hbox{४}}} =$ द्रयो
\vspace{1mm}

\hspace{2mm} अत्र यदि चतुर्थपक्षिण एकस्य मूल्यमिदं\, $\dfrac{{\footnotesize{\hbox{मू}}_{\scriptsize{\hbox{४}}}}}{{\footnotesize{\hbox{प}}_{\scriptsize{\hbox{४}}}}} =$ मू\, सर्वतोऽधिकं तदा तेन प्रथमसमीकरणं गुणितं जातम्
\vspace{1mm}

\hspace{4mm} ${\hbox{प}}_{\scriptsize{\hbox{१}}}.{\hbox{गु}}_{\scriptsize{\hbox{१}}}.{\hbox{मू}} + {\hbox{प}}_{\scriptsize{\hbox{२}}}.{\hbox{गु}}_{\scriptsize{\hbox{२}}}.{\hbox{मू}} + {\hbox{प}}_{\scriptsize{\hbox{३}}}.{\hbox{गु}}_{\scriptsize{\hbox{३}}}.{\hbox{मू}} + {\hbox{मू}}_{\scriptsize{\hbox{४}}}.{\hbox{गु}}_{\scriptsize{\hbox{४}}} =$ मू.पयो  ......... (१)
\vspace{1mm}

\hspace{4mm} ${\hbox{मू}}_{\scriptsize{\hbox{१}}}.{\hbox{गु}}_{\scriptsize{\hbox{१}}} + {\hbox{मू}}_{\scriptsize{\hbox{२}}}.{\hbox{गु}}_{\scriptsize{\hbox{२}}} + {\hbox{मू}}_{\scriptsize{\hbox{३}}}.{\hbox{गु}}_{\scriptsize{\hbox{३}}} + {\hbox{मू}}_{\scriptsize{\hbox{४}}}.{\hbox{गु}}_{\scriptsize{\hbox{४}}} =$ द्रयो ....................... (२)
\vspace{1mm}

\hspace{2mm} प्रथमं द्वितीयादपास्य जातम्
\vspace{1mm}

\hspace{4mm} ${\hbox{गु}}_{\scriptsize{\hbox{१}}}\,({\hbox{मू}}_{\scriptsize{\hbox{१}}} - {\hbox{प}}_{\scriptsize{\hbox{१}}}.{\hbox{मू}}) + {\hbox{गु}}_{\scriptsize{\hbox{२}}}\,({\hbox{मू}}_{\scriptsize{\hbox{२}}} - {\hbox{प}}_{\scriptsize{\hbox{२}}}.{\hbox{मू}}) + {\hbox{गु}}_{\scriptsize{\hbox{३}}}\,({\hbox{मू}}_{\scriptsize{\hbox{३}}} - {\hbox{प}}_{\scriptsize{\hbox{३}}}.{\hbox{मू}})+ {\hbox{०}} =$ द्रयो $-$ मू.पयो ........ (३)
\vspace{1mm}
}{\large \textbf{{\color{purple}अधिकैकमौल्यगुणिता \\
जीवा इच्छास्वमूल्यहीनाश्च~॥~३४~॥\\
शेषाणीष्टैर्गुणकैः \\
स्वधिया गुणयेद्यथेच्छया तुल्यम्~।\\
गुणकहतजीवमूल्यो-\\
नेच्छा विगुणार्घहृदगुणे\renewcommand{\thefootnote}{$\star$}\footnote{The reading विगुणार्धहृदगुणे seems to be a typographical error.} तु गुणः~॥~३५~॥}}}
\end{quote}

\newpage

\noindent \textbf{उदाहरणम्~।}\renewcommand{\thefootnote}{}\footnote{(३) अस्मिन् यथासम्भवमपवर्त्तनं दत्त्वा तथेष्टानि\, ${\hbox{गु}}_{\scriptsize{\hbox{१}}}, {\hbox{गु}}_{\scriptsize{\hbox{२}}}, {\hbox{गु}}_{\scriptsize{\hbox{३}}}$\, मानानि कल्प्यानि यथा पक्षौ समौ स्तः~। ततो व्यक्तगुणकानां मानानि प्रथमसमीकरणद्वये समुत्थाप्याव्यक्तगुणकस्य मानं व्यक्तं ज्ञेयम्~। अत्रा-धिकैकमौल्यमित्युपलक्षणं तेन कस्याप्यभीष्टस्य मूल्यं गुणकं प्रकल्प्य कर्म कार्यम्~। तस्यैव गुणकश्चान्तिम-कर्मणा व्यक्तो भवतीति~।}

\phantomsection \label{Ex 2.38}
\begin{quote}
\textbf{{\color{red}लभ्यन्ते पञ्चहंसास्त्रिभिरलसचराः पञ्चभिः सप्तकीरा\\
वाचाला बर्हभाजो नवतुरगमितैर्द्रम्मकैः पण्यवीथ्याम्~।\\
मन्दं गुञ्जत्पिकानां त्रयमपि नवभिर्भूपसम्प्रीतिहेतोः\\
आदायागच्छ विद्वन् खगशतमपि भो द्रम्मकाणां शतेन~॥}}
\end{quote}

\begin{sloppypar}
न्यासः~।\, {\small $\begin{matrix}
\mbox{{३}}\\
\vspace{-1mm}
\mbox{{५}}
\vspace{1mm}
\end{matrix}$}~। {\small $\begin{matrix}
\mbox{{५}}\\
\vspace{-1mm}
\mbox{{७}}
\vspace{1mm}
\end{matrix}$}~। {\small $\begin{matrix}
\mbox{{७}}\\
\vspace{-1mm}
\mbox{{९}}
\vspace{1mm}
\end{matrix}$}~। {\small $\begin{matrix}
\mbox{{९}}\\
\vspace{-1mm}
\mbox{{३}}
\vspace{1mm}
\end{matrix}$}~। {\small $\begin{matrix}
\mbox{{१००}}\\
\vspace{-1mm}
\mbox{{१००}}
\vspace{1mm}
\end{matrix}$}~। अत्र सर्वेषामेकजीवस्याधिकमूल्यम् ३ अनेन जीवा गुणिताः १५।२१।२७।९।३००~। स्वस्वमूल्यहीनाः १२।१६।२०।०।२०० एतानि चतुर्भिः अपवर्तितानि ३।४।५।०।५०~। एभ्यो जाताः कुट्टकाः १।८।३, वा १।३।७, वा २।४।६, वा ३।९।१, वा ४।२।६, वा ४।७।२, वा ५।५।३, वा ६।३।४, वा ७।१।१५, वा ७।६।१, वा ८।४।२, वा ९।२।३, वा ११।३।१, वा १२।११।१~। एभिरुद्दिष्टानि मौल्यानि जीवान् वा विनिहत्य धनतो जीवेभ्यो वा विशोध्य शेषे कुट्टकस्थानधनेन जीवैर्वा भक्ते जातः कुट्टकस्थानगुणकः\textendash
\vspace{2mm}

\begin{tabular}{c|c|c|c|c|c|c|c|c|c|c|c|c|c|c|}
४ & ११ & ११ & ४ & ४ & १३ & ४ & १३ & १३ & १३ & १४ & १४ & ६ & ५ & ४ \\
२ & ३ & ३ & १ & १ & ३ & १ & ३ & ३ & ३ & ३ & १ & ५ & १ & १
\end{tabular}
\vspace{2mm}

एवं जाताः प्रथमकुट्टके गुणकाः १।८।३।४ एभिर्गुणिता मूल्यजीवाः\, {\small $\begin{matrix}
\mbox{{३}}\\
\vspace{-1mm}
\mbox{{५}}
\vspace{1mm}
\end{matrix}$}~। {\small $\begin{matrix}
\mbox{{४०}}\\
\vspace{-1mm}
\mbox{{५६}}
\vspace{1mm}
\end{matrix}$}~। {\small $\begin{matrix}
\mbox{{२१}}\\
\vspace{-1mm}
\mbox{{२७}}
\vspace{1mm}
\end{matrix}$}~। {\small $\begin{matrix}
\mbox{{३६}}\\
\vspace{-1mm}
\mbox{{१२}}
\vspace{1mm}
\end{matrix}$}~।\, एवमनेकधा~।
\end{sloppypar}

\newpage

\noindent \textbf{सूत्रम्~।}

\phantomsection \label{2.36.1}
\begin{quote}
\renewcommand{\thefootnote}{१}\footnote{{\color{violet}'भजेच्छिदोऽंशैरथ तैर्विमिश्रै रूपं भवेत् स्यात् परिपूर्त्तिकालः~।'} इति {\color{violet}भास्करो}क्तानुरूपमिदम्~।
\vspace{1mm}
}{\large \textbf{{\color{purple}अंशहृतरूपसंयुति-\\
भक्ते रूपे प्रपूर्तिकालः स्यात्~।}}}
\end{quote}

\noindent \textbf{उदाहरणम्~।}

\phantomsection \label{Ex 2.39}
\begin{quote}
\textbf{{\color{red}दिनदिनदलत्रिभागाङ्घ्रिभिः पृथक् पूरयन्ति ये वापीम्~।\\
ते निर्झराश्च युगपन्मुक्ता वद केन भागेन~॥}}
\end{quote}

न्यासः~। $\frac{{\footnotesize{\hbox{१}}}}{{\footnotesize{\hbox{१}}}}$\,।$\frac{{\footnotesize{\hbox{१}}}}{{\footnotesize{\hbox{२}}}}$\,।\,$\frac{{\footnotesize{\hbox{१}}}}{{\footnotesize{\hbox{३}}}}$\,।\,$\frac{{\footnotesize{\hbox{१}}}}{{\footnotesize{\hbox{४}}}}$\,। लब्धो वापी पूरणकालो दिनभागः\, $\frac{{\footnotesize{\hbox{१}}}}{{\footnotesize{\hbox{१०}}}}$~।\\

\noindent \textbf{सूत्रम्~।}

\phantomsection \label{2.36}
\begin{quote}
\renewcommand{\thefootnote}{२}\footnote{अत्रोपपत्तिः~। यदि क्रयमानम् $=$ क्र~। विक्रयमानम् $=$ वि~। शेषविक्रयमानम् $=$ शे~। मूलधनानि क्रमेण ${\hbox{ध}}_{\scriptsize{\hbox{१}}}, {\hbox{ध}}_{\scriptsize{\hbox{२}}}, {\hbox{ध}}_{\scriptsize{\hbox{३}}}, {\hbox{ध}}_{\scriptsize{\hbox{४}}}$,\, यत्र\, ${\hbox{ध}}_{\scriptsize{\hbox{४}}}$\, सर्वाधिकम्~। प्रथमविक्रये च क्रमेण लब्धयः\, ${\hbox{या}}_{\scriptsize{\hbox{१}}}, {\hbox{या}}_{\scriptsize{\hbox{२}}}, {\hbox{या}}_{\scriptsize{\hbox{३}}}, {\hbox{या}}_{\scriptsize{\hbox{४}}}$,\, मूलधनगुण-कश्च $=$ गु तदा प्रश्नानुसारेण प्रथमलाभः
\vspace{1mm}

\hspace{11mm} $=$ शे\,(क्र.${\hbox{ध}}_{\scriptsize{\hbox{१}}} -$ वि.${\hbox{या}}_{\scriptsize{\hbox{१}}}$) $+ {\hbox{या}}_{\scriptsize{\hbox{१}}} -$ गु.${\hbox{ध}}_{\scriptsize{\hbox{१}}}$
\vspace{1mm}

\hspace{11mm} $= {\hbox{ध}}_{\scriptsize{\hbox{१}}}$\,(क्र.शे $-$ गु) $- {\hbox{या}}_{\scriptsize{\hbox{१}}}$\,(वि.शे $-$ १)
\vspace{1mm}

\hspace{2mm} एवम्,~~\, $= {\hbox{ध}}_{\scriptsize{\hbox{२}}}$\,(क्र.शे $-$ गु) $- {\hbox{या}}_{\scriptsize{\hbox{२}}}$\,(वि.शे $-$ १)
\vspace{1mm}

\hspace{11mm} $= {\hbox{ध}}_{\scriptsize{\hbox{३}}}$\,(क्र.शे $-$ गु) $- {\hbox{या}}_{\scriptsize{\hbox{३}}}$\,(वि.शे $-$ १)
\vspace{1mm}

\hspace{11mm} $= {\hbox{ध}}_{\scriptsize{\hbox{४}}}$\,(क्र.शे $-$ गु) $- {\hbox{या}}_{\scriptsize{\hbox{४}}}$\,(वि.शे $-$ १)
\vspace{2mm}

\hspace{2mm} अत्र ${\hbox{ध}}_{\scriptsize{\hbox{१}}}, {\hbox{ध}}_{\scriptsize{\hbox{२}}}$,... इत्यादीनां समापवर्त्तनम् $=$ स, तेनापवर्त्तने धनमानानि\, ${\hbox{ध'}}_{\scriptsize{\hbox{१}}}, {\hbox{ध'}}_{\scriptsize{\hbox{२}}}$,\, इत्यादि तदा
\vspace{1mm}

\hspace{6mm} ${\hbox{ध'}}_{\scriptsize{\hbox{१}}}\,({\hbox{क्र.शे}} - {\hbox{गु}}) - {\hbox{या}}_{\scriptsize{\hbox{१}}}\,\left(\dfrac{{\footnotesize{\hbox{वि.शे}} - {\hbox{१}}}}{{\footnotesize{\hbox{स}}}}\right) = {\hbox{ध'}}_{\scriptsize{\hbox{२}}}\,({\hbox{क्र.शे}} - {\hbox{गु}})$}{\large \textbf{{\color{purple}अधिकधनमिष्टयुक्तं \\
विक्रयमानं तु तेन सङ्गुणितः~॥~३६~॥}}}
\end{quote}

\newpage

\phantomsection \label{2.37.1}
\begin{quote}
{\large \textbf{{\color{purple}शेषार्घगुणकयोगो \\
रूपविहीनः क्रयो भवति~।}}}\renewcommand{\thefootnote}{}\footnote{\hspace{10mm} $- {\hbox{या}}_{\scriptsize{\hbox{२}}}\,\left(\dfrac{{\footnotesize{\hbox{वि.शे}} - {\hbox{१}}}}{{\footnotesize{\hbox{स}}}}\right)$
\vspace{2mm}

\hspace{6mm} $= {\hbox{ध'}}_{\scriptsize{\hbox{३}}}\,({\hbox{क्र.शे}} - {\hbox{गु}}) - {\hbox{या}}_{\scriptsize{\hbox{३}}}\,\left(\dfrac{{\footnotesize{\hbox{वि.शे}} - {\hbox{१}}}}{{\footnotesize{\hbox{स}}}}\right)$
\vspace{2mm}

\hspace{6mm} $= {\hbox{ध'}}_{\scriptsize{\hbox{४}}}\,({\hbox{क्र.शे}} - {\hbox{गु}}) - {\hbox{या}}_{\scriptsize{\hbox{४}}}\,\left(\dfrac{{\footnotesize{\hbox{वि.शे}} - {\hbox{१}}}}{{\footnotesize{\hbox{स}}}}\right) = {\hbox{ला}}$
\vspace{2mm}

\hspace{2mm} अत्र यदि~~ क्र $= \dfrac{{\footnotesize{\hbox{वि}}\,({\hbox{शे}} + {\hbox{गु.स}}) - {\hbox{१}}}}{{\footnotesize{\hbox{स}}}}$,\; तदा
\vspace{2mm}

\hspace{6mm} ${\hbox{या}}_{\scriptsize{\hbox{१}}} = {\hbox{ध'}}_{\scriptsize{\hbox{१}}}\,({\hbox{शे}} + {\hbox{गु.स}}) - {\hbox{१}}$\; यदि\, ${\hbox{वि}} > {\hbox{ध'}}_{\scriptsize{\hbox{१}}}$
\vspace{1mm}

\hspace{6mm} ${\hbox{या}}_{\scriptsize{\hbox{२}}} = {\hbox{ध'}}_{\scriptsize{\hbox{२}}}\,({\hbox{शे}} + {\hbox{गु.स}}) - {\hbox{१}}$\; यदि\, ${\hbox{वि}} > {\hbox{ध'}}_{\scriptsize{\hbox{२}}}$
\vspace{1mm}

\hspace{6mm} ${\hbox{या}}_{\scriptsize{\hbox{३}}} = {\hbox{ध'}}_{\scriptsize{\hbox{३}}}\,({\hbox{शे}} + {\hbox{गु.स}}) - {\hbox{१}}$\; यदि\, ${\hbox{वि}} > {\hbox{ध'}}_{\scriptsize{\hbox{३}}}$
\vspace{1mm}

\hspace{6mm} ${\hbox{या}}_{\scriptsize{\hbox{४}}} = {\hbox{ध'}}_{\scriptsize{\hbox{४}}}\,({\hbox{शे}} + {\hbox{गु.स}}) - {\hbox{१}}$\; यदि\, ${\hbox{वि}} > {\hbox{ध'}}_{\scriptsize{\hbox{४}}}$
\vspace{2mm}

\hspace{2mm} तथा~~ ${\hbox{ला}} = {\hbox{ध'}}_{\scriptsize{\hbox{१}}}\,({\hbox{क्र.शे}} - {\hbox{गु}}) - {\hbox{या}}_{\scriptsize{\hbox{१}}}\,\left(\dfrac{{\footnotesize{\hbox{वि.शे}} - {\hbox{१}}}}{{\footnotesize{\hbox{स}}}}\right)$
\vspace{2mm}

\hspace{13mm} $= {\hbox{ध'}}_{\scriptsize{\hbox{१}}}\,\left(\dfrac{{\footnotesize{\hbox{वि.शे}}\,({\hbox{शे}} + {\hbox{गु.स}}) - {\hbox{शे}} - {\hbox{स.गु}}}}{{\footnotesize{\hbox{स}}}}\right) - [{\hbox{ध'}}_{\scriptsize{\hbox{१}}}\,({\hbox{शे}} + {\hbox{गु.स}}) - {\hbox{१}}]\,\left(\dfrac{{\footnotesize{\hbox{वि.शे}} - {\hbox{१}}}}{{\footnotesize{\hbox{स}}}}\right)$
\vspace{2mm}

\hspace{13mm} $= {\hbox{ध'}}_{\scriptsize{\hbox{१}}}\,({\hbox{शे}} + {\hbox{स.गु}})\,\left(\dfrac{{\footnotesize{\hbox{वि.शे}} - {\hbox{१}}}}{{\footnotesize{\hbox{स}}}}\right) - [{\hbox{ध'}}_{\scriptsize{\hbox{१}}}\,({\hbox{शे}} + {\hbox{स.गु}}) - {\hbox{१}}]\,({\hbox{वि.शे}} - {\hbox{१}})$
\vspace{2mm}

\hspace{13mm} $= \dfrac{{\footnotesize{\hbox{वि.शे}} - {\hbox{१}}}}{{\footnotesize{\hbox{स}}}}$~।\; इदं लाभमानं सर्वत्र समानमेव~।
\vspace{2mm}

\hspace{2mm} अतो वि मानं तथा कल्प्यं तथा\; $\dfrac{{\footnotesize{\hbox{वि.शे}} - {\hbox{१}}}}{{\footnotesize{\hbox{स}}}}$\, इदमभिन्नमथ च}
\end{quote}

\newpage

\noindent \textbf{उदाहरणम्~।}\renewcommand{\thefootnote}{}\footnote{\hspace{-6.5mm} ${\hbox{वि}} > {\hbox{ध'}}_{\scriptsize{\hbox{४}}}$~। यदि शे भाज्यः ऋणरूपं क्षेपः समापवर्त्तनं हरो भवेत्तदा कुट्टकविधिना बहुधा गुणमानं स्यात्~। आचार्येण स $=$ १, रूपं प्रकल्प्य कुट्टकमन्तरैव क्रयविक्रयमाने साधिते तदा\, ${\hbox{वि}} > {\hbox{ध}}_{\scriptsize{\hbox{४}}}$~।
\vspace{1mm}

\hspace{6mm} क्र $= {\hbox{वि}}\,({\hbox{शे}} + {\hbox{गु}}) - {\hbox{१}}$~।
\vspace{2mm}

\hspace{2mm} मत्प्रकारेण~~ क्र $= \dfrac{{\footnotesize{\hbox{वि}}\,({\hbox{शे}} + {\hbox{गु.स}}) - {\hbox{१}}}}{{\footnotesize{\hbox{स}}}} = \dfrac{{\footnotesize{\hbox{वि.शे}} - {\hbox{१}}}}{{\footnotesize{\hbox{स}}}} +$ वि.गु
\vspace{2mm}

\hspace{2mm} अर्थात् कुट्टकविधिना गुणो विक्रयमानं लब्धिर्गुणगुणितविक्रयमानयुता क्रयमानमिति सिध्यति~। 
\vspace{1mm}

\hspace{2mm} परन्तु यत्र शे, स एतौ मिथो न दृढौ तत्र मत्प्रकारेऽपि सन्मानं रूपमेव प्रकल्प्यम्~। यथेहैवाचार्योक्तोदाहरणे धनानां समापवर्तनं २, शेषार्घश्च ६ मिथो न दृढावतोऽत्र समापवर्त्तनं रूपमेव प्रकल्प्यम्~। आचार्योक्तोदाहरणे यदि शेषार्घः ५ भवेत्तदा तृतीयोदाहरणे मत्प्रकारेण
\vspace{2mm}

\hspace{4mm} $\dfrac{{\footnotesize{\hbox{भा ५ क्षे १}}}}{{\footnotesize{\hbox{हा २}}}} = $ {\footnotesize $\begin{matrix}
\mbox{{ल = २}}\\
\vspace{-1mm}
\mbox{{गु = १}}
\vspace{1mm}
\end{matrix}$}~~ पञ्चविंशतिसमेष्टेन}

\phantomsection \label{Ex 2.40}
\begin{quote}
\textbf{{\color{red}षट्-दश-पञ्चांशच्छतपणैर्गृहीत्वा फलानि कदलीनाम्~।\\
विक्रीय समार्घेणावशेषमेकैकमिह पणैः षड्भिः~॥\\
स्युस्ते सलाभतुल्याः सखे क्रयः कोऽत्र विक्रयश्च वद~।\\
निजनिजमूलोनो वा द्विगुणितमूलोनिताश्च वा तुल्याः~।\\
गुणिताहङ्कारगिरेः शिखरं प्राप्तोऽसि चेद्गणक~॥}}
\end{quote}

\begin{sloppypar}
प्रथमस्य न्यासः~। ६।१०।५०।१०० शेषार्घः ६~। गुणकः ०~। अत्रैकेनेष्टेन जातो विक्रयः १०१ क्रयश्च ६०५~। द्विकेन विक्रयः १०२ क्रयः ६११~।
\vspace{2mm}

द्वितीयोदाहरणे न्यासः~। ६।१०।५०।१०० शेषार्घः ६ गुणकः १ एकेनेष्टेन जातौ विक्रय-क्रयौ १०१।७०६ द्विकेनेष्टेन जातौ विक्रयक्रयौ १०२।७१३~। 
\vspace{2mm}

तृतीयोदाहरणे न्यासः~। ६।१०।५०।१०० शेषार्घः ६ गुणकः २~। एकेनेष्टेन जातौ विक्रय-क्रयौ १०२।८०७ द्विकेनेष्टेन विक्रयक्रियौ १००।८१५~। एवमिष्टवशादानन्त्यम्~।
\end{sloppypar}

\newpage

\noindent \textbf{सूत्रम्~।}\renewcommand{\thefootnote}{}\footnote{गुणः $=$ ५१~। लब्धिः $=$ १२७~। अतः विक्रयः $=$ ५१~। 
\vspace{1mm}

\hspace{2mm} क्रयः $=$ ल $+$ वि.गु $=$ १२७ $+$ २ $\times$ ५१ $=$ १२७ $+$ १०२ $=$ २२९~। 
\vspace{1mm}

\hspace{2mm} आचार्यमतेन रूपमिष्टं प्रकल्प्य विक्रयः $=$ १०१, क्रयः $=$ १०१\,(५ $+$ २) $-$ १ $=$ ७०६~। 
\vspace{1mm}

\hspace{2mm} आचार्यप्रकारेण क्रयविक्रययोर्महती सङ्ख्या भवति~। 
\vspace{1mm}}

\phantomsection \label{2.37}
\begin{quote}
\renewcommand{\thefootnote}{१}\footnote{अत्रोपपत्तिः~। अतिसुगमा यतः समहरवित्तानां ये लवास्तत्तुल्यधनेषु पूर्ववत् क्रयविक्रयमाने ये तयोः क्रयः चेत् समहरगुणः क्रयः कल्प्यते तदा भिन्नधनगुणनेन राशिः पूर्वक्रयगुणितलवसमः स्याद्यत्र विक्रयः पूर्वानीतसम एव भवतीति~।}{\large \textbf{{\color{purple}कृतसमहरवित्तानां \\
छेदगमे विक्रयक्रयौ प्राग्वत्~॥~३७~॥\\
आनीय समच्छेद-\\
क्रयाहतिः स्यात् क्रयो भिन्ने~।}}}
\end{quote}

\noindent \textbf{उदाहरणम्~।}

\phantomsection \label{Ex 2.41}
\begin{quote}
\textbf{{\color{red}अर्धत्रिभागपञ्चमचरणै रम्भाफलानि च क्रीत्वा~।\\
विक्रीय समार्घेणावशेषमेकैकमिह चतुर्भिश्च~।\\
जातास्ते समवित्ता विद्वन् क्रयविक्रयौ कथय~॥}}
\end{quote}

\begin{sloppypar}
न्यासः~। $\frac{{\footnotesize{\hbox{१}}}}{{\footnotesize{\hbox{२}}}}$\,।\,$\frac{{\footnotesize{\hbox{१}}}}{{\footnotesize{\hbox{३}}}}$\,।\,$\frac{{\footnotesize{\hbox{१}}}}{{\footnotesize{\hbox{५}}}}$\,।\,$\frac{{\footnotesize{\hbox{१}}}}{{\footnotesize{\hbox{४}}}}$\,। शेषार्घः ४~। अत्र करणम्~। कृतसमच्छेदानि\, $\frac{{\footnotesize{\hbox{३०}}}}{{\footnotesize{\hbox{६०}}}}$\,।\,$\frac{{\footnotesize{\hbox{२०}}}}{{\footnotesize{\hbox{६०}}}}$\,।\,$\frac{{\footnotesize{\hbox{१२}}}}{{\footnotesize{\hbox{६०}}}}$\,। $\frac{{\footnotesize{\hbox{१५}}}}{{\footnotesize{\hbox{६०}}}}$\,। छेदगमे जातानि ३०।२०।१२।१५~। प्राग्वदेकेनेष्टेन विक्रयक्रयौ ३१।१२३ एतयोः क्रयः १२३ अयं समच्छेदहरेणा\textendash \,६०\textendash \,नेन गुणितो जातो भिन्नधनानां क्रयः ७३८० एवं विक्रयः १८६०~। द्विकेनेष्टेन विक्रयक्रयौ ३२।७६२०~। 
\vspace{2mm}

एवमिष्टवशादानन्त्यम्~। 
\end{sloppypar}

\newpage

\noindent \textbf{सूत्रम्~।}

\phantomsection \label{2.38}
\begin{quote}
\renewcommand{\thefootnote}{१}\footnote{अत्रोपपत्तिः~। अत्र कृष्णदैवज्ञप्रकारेण (द्रष्टव्या श्रीमज्जनककृता भास्करबीजटिप्पणी पृ.\,१३३) अभिन्न-मानार्थं ~भिन्नधनानां ~~हरलघुतमापवर्त्त्यमानं ~वा ~~सर्वहरापवर्त्त्यसङ्ख्यासमं ~समहारं ~~प्रकल्प्य ~तत्रेष्टं ~किमपि प्रक्षिप्य चरमविक्रयमानस्यार्थात् शेषविक्रयमानस्या\textendash \,$\dfrac{{\footnotesize{\hbox{अ}}}}{{\footnotesize{\hbox{क}}}}$\textendash \,स्य हरेण सङ्गुण्य [(शु $+$ इ)\,क] विक्रयं प्रकल्प्य \,$\dfrac{{\footnotesize{\hbox{अ}}}}{{\footnotesize{\hbox{क}}}}$\,[(शु $+$ इ)\,क] $-$ १\, क्रयमानमानीयाभिन्नार्थं समहरेण 'शु'-सञ्ज्ञकेन चरमहरेण च सङ्गुण्य क्रयमानम् 
\vspace{2mm}

\hspace{6mm} $= {\hbox{क.शु.}}\,\left[\dfrac{{\footnotesize{\hbox{अ}}}}{{\footnotesize{\hbox{क}}}}\,[({\hbox{शु}} + {\hbox{इ}})\,{\hbox{क}}] - {\hbox{१}}\right] = {\hbox{क.शु.सध}}$~। 
\vspace{2mm}

यतः समधनमानम् $= \dfrac{{\footnotesize{\hbox{अ}}}}{{\footnotesize{\hbox{क}}}}$\,[(शु $+$ इ)\,क] $-$ १
\vspace{2mm}

\hspace{2mm} अत्र यः शेषविक्रयः $= \dfrac{{\footnotesize{\hbox{अ}}}}{{\footnotesize{\hbox{क}}}}$, तेन रूपं विहृतं कृष्णदैवज्ञीयः शेषविक्रयः $= \dfrac{{\footnotesize{\hbox{क}}}}{{\footnotesize{\hbox{अ}}}}$\, इति बुद्धिमता ज्ञेयम्~। इदं प्रकारान्तरं भिन्नेऽभिन्ने सर्वत्रैव घटत इत्युपपन्नं सर्वम्~। अस्मादेव प्रकारात् कृष्णदैवज्ञप्रकार उत्पद्यते तदुपपत्त्यर्थं श्रीमज्जनकसम्पादितस्य भास्करबीजगणितस्य पूर्वोक्तं पृ.\,१३३ विलोक्यमिति~। तत्र समापवर्त्तनं रूपं प्रकल्प्यम्~।}{\large \textbf{{\color{purple}यः शुद्धिमेति हि हरैर्विहृतः स शुद्धोऽ-\\
भीष्टान्वितश्चरमविक्रयहारनिघ्नः~।\\
स्याद्विक्रयोऽथ चरमार्घहतो निरेकः \\
तुल्यं धनं भवति तच्चरमच्छिदा च~॥~३८~॥\\
शुद्धेन चाभिगुणिते नियतं क्रयार्घोऽ-\\
भिन्नेऽथवापि निरभिन्नधनेऽपि नूनम्~।}}}
\end{quote}

\newpage

\begin{sloppypar}
\noindent \textbf{उदाहरणम्~।}

\phantomsection \label{Ex 2.42}
\begin{quote}
\textbf{{\color{red}अर्धत्रिभागचरणेषुलवैर्गृहीत्वा \\
रम्भाफलानि सदृशेन च विक्रितानि~।\\
एकैकमङ्घ्रिसहितेन पणेन शेषं \\
दत्तं समाः स्युरिह कौ क्रयविक्रयौ च~॥}}
\end{quote}
 
 न्यासः~। $\frac{{\footnotesize{\hbox{१}}}}{{\footnotesize{\hbox{२}}}}$\,।\,$\frac{{\footnotesize{\hbox{१}}}}{{\footnotesize{\hbox{३}}}}$\,।\,$\frac{{\footnotesize{\hbox{१}}}}{{\footnotesize{\hbox{४}}}}$\,।\,$\frac{{\footnotesize{\hbox{१}}}}{{\footnotesize{\hbox{५}}}}$\,। शेषार्घः \,$\frac{{\footnotesize{\hbox{५}}}}{{\footnotesize{\hbox{४}}}}$\,। अत्र करणम्~। सर्वैश्छेदैः शुद्ध्यति यथा तथा कल्पित इष्टराशिः शुद्धाख्यः ६०~। इष्टः १ युतः ६१ \hyperref[2.38]{'चरमविक्रयहारनिघ्न'} इति चरमविक्रय \,$\frac{{\footnotesize{\hbox{५}}}}{{\footnotesize{\hbox{४}}}}$\, अस्य हारेणानेन ४ हतो जातो विक्रयः २४४~। अथायम् २४४~। चरमार्घेण \,$\frac{{\footnotesize{\hbox{४}}}}{{\footnotesize{\hbox{५}}}}$\, गुणितः ३०५ निरेकः ३०४ जातं समधनप्रमाणम् ३०४~। एतच्छुद्धेन ६० चरमार्घहरेण च ४ गुणितं जातः क्रयार्घः ७२९६०~। 
\vspace{2mm}

द्विकेनेष्टेन जातौ क्रयविक्रयौ ७४१६०।२४८~। त्रिकेण क्रयविक्रयौ ७५३६०।२५२~। \\

\noindent \textbf{अपि च~।}

\phantomsection \label{Ex 2.43}
\begin{quote}
\textbf{{\color{red}एकयुग्मत्रिवेदा लवाश्छेदका \\
द्व्यादयो यत्र वित्तं चतुर्णां सखे~।\\
तत्र तुल्यः क्रयो विक्रयः को भवेत् \\
पश्चिमार्घो नवाष्टांशकाः स्युः समाः~॥}}
\end{quote}

न्यासः~। $\frac{{\footnotesize{\hbox{१}}}}{{\footnotesize{\hbox{२}}}}$\,।\,$\frac{{\footnotesize{\hbox{२}}}}{{\footnotesize{\hbox{३}}}}$\,।\,$\frac{{\footnotesize{\hbox{३}}}}{{\footnotesize{\hbox{४}}}}$\,।\,$\frac{{\footnotesize{\hbox{४}}}}{{\footnotesize{\hbox{५}}}}$\,। शेषार्घः \,$\frac{{\footnotesize{\hbox{९}}}}{{\footnotesize{\hbox{८}}}}$\,। अत्रैकेनेष्टेन जातौ क्रयविक्रयौ २६३०४०।४४८~। द्विकेन\, २६७३६०।४९६~।
\end{sloppypar}

\newpage

\noindent \textbf{सूत्रम्~।}

\phantomsection \label{2.39}
\begin{quote}
\renewcommand{\thefootnote}{१}\footnote{कल्प्यते प्रथमस्य गतिः $= {\hbox{ग}}_{\scriptsize{\hbox{१}}}$~। द्वितीयस्य गतिः $= {\hbox{ग}}_{\scriptsize{\hbox{२}}}$~। पुरान्तरम् $=$ अ~। तदा गतियोगेनैकदिनं तदा पुरान्तरेण किं लब्धः प्रथमसमागमकालः $= \dfrac{{\footnotesize{\hbox{अ}}}}{{\footnotesize{\hbox{ग}}_{\scriptsize{\hbox{१}}} + {\hbox{ग}}_{\scriptsize{\hbox{२}}}}}$~। एतावता कालेन प्रथमस्य गमनम् $= \dfrac{{\footnotesize{\hbox{ग}}_{\scriptsize{\hbox{१}}}.{\hbox{अ}}}}{{\footnotesize{\hbox{ग}}_{\scriptsize{\hbox{१}}} + {\hbox{ग}}_{\scriptsize{\hbox{२}}}}}$~। द्वितीयस्य गमनम् $= \dfrac{{\footnotesize{\hbox{ग}}_{\scriptsize{\hbox{२}}}.{\hbox{अ}}}}{{\footnotesize{\hbox{ग}}_{\scriptsize{\hbox{१}}} + {\hbox{ग}}_{\scriptsize{\hbox{२}}}}}$~। अथ प्रथमसमागमानन्तरं द्वितीयसमागमकालो यदि 'या' कल्प्यते तदैतावता कालेन प्रथमस्य गमनम् $= {\hbox{ग}}_{\scriptsize{\hbox{१}}}{\hbox{या}}$\, अस्माद्द्वितीयस्य गमनं\, $\dfrac{{\footnotesize{\hbox{ग}}_{\scriptsize{\hbox{२}}}.{\hbox{अ}}}}{{\footnotesize{\hbox{ग}}_{\scriptsize{\hbox{१}}} + {\hbox{ग}}_{\scriptsize{\hbox{२}}}}}$\, विशोध्य जातं प्रथमस्य परावर्त्तनगमनम्\, ${\hbox{ग}}_{\scriptsize{\hbox{१}}}{\hbox{या}} - \dfrac{{\footnotesize{\hbox{ग}}_{\scriptsize{\hbox{२}}}.{\hbox{अ}}}}{{\footnotesize{\hbox{ग}}_{\scriptsize{\hbox{१}}} + {\hbox{ग}}_{\scriptsize{\hbox{२}}}}}$~। एवं द्वितीयस्य परावर्त्तनगमनमानम् $= {\hbox{ग}}_{\scriptsize{\hbox{२}}}{\hbox{या}} - \dfrac{{\footnotesize{\hbox{ग}}_{\scriptsize{\hbox{१}}}.{\hbox{अ}}}}{{\footnotesize{\hbox{ग}}_{\scriptsize{\hbox{१}}} + {\hbox{ग}}_{\scriptsize{\hbox{२}}}}}$~। द्वयोः परावर्त्तनगमनयोगः पुरान्तरसमः~। 
\vspace{2mm}

\hspace{2mm} अतः~~ ${\hbox{ग}}_{\scriptsize{\hbox{१}}}{\hbox{या}} - \dfrac{{\footnotesize{\hbox{ग}}_{\scriptsize{\hbox{२}}}.{\hbox{अ}}}}{{\footnotesize{\hbox{ग}}_{\scriptsize{\hbox{१}}} + {\hbox{ग}}_{\scriptsize{\hbox{२}}}}} + {\hbox{ग}}_{\scriptsize{\hbox{२}}}{\hbox{या}} - \dfrac{{\footnotesize{\hbox{ग}}_{\scriptsize{\hbox{१}}}.{\hbox{अ}}}}{{\footnotesize{\hbox{ग}}_{\scriptsize{\hbox{१}}} + {\hbox{ग}}_{\scriptsize{\hbox{२}}}}}$
\vspace{2mm}

\hspace{14mm} $= ({\hbox{ग}}_{\scriptsize{\hbox{१}}} + {\hbox{ग}}_{\scriptsize{\hbox{२}}})\,{\hbox{या}} - {\hbox{अ}} = {\hbox{अ}} \hspace{4mm} \therefore\; {\hbox{या}} = \dfrac{{\footnotesize{\hbox{२\,अ}}}}{{\footnotesize{\hbox{ग}}_{\scriptsize{\hbox{१}}} + {\hbox{ग}}_{\scriptsize{\hbox{२}}}}}$\; अत उपपन्नम्~।}{\large \textbf{{\color{purple}अध्वनि गतियोगहृते \\
प्रजायते प्रथमसङ्गमे कालः~।\\
तस्मिन् योगे द्विगुणे \\
योगात् तस्मात् पुनर्योगः~॥}}}
\end{quote}

\noindent \textbf{उदाहरणम्~।}

\phantomsection \label{Ex 2.44}
\begin{quote}
\textbf{{\color{red}योजनत्रिशती पन्थाः पुरयोरन्तरं कयोः~।\\
एकादशगतिस्त्वेको नवयोजनगः परः~॥\\
युगपन्निर्गतौ स्वस्वपुरतो लिपिवाहकौ~।\\
समागमद्वयं ब्रूहि गच्छतोश्च निवृत्तयोः~॥\\
दिवसैकगतिः शीघ्रं वद कोविद वेत्सि चेत्~॥}}
\end{quote}

\newpage

न्यासः~। नगरयोरन्तरयोजनप्रमाणं ३०० नियतगतो ११।९ जाताः प्रथमदिवसाः १५~। प्रथमसमागमाद्द्वितीयसमागमदिवसाः ३०~। \\

\noindent \textbf{अपि च~।}

\phantomsection \label{Ex 2.45}
\begin{quote}
\textbf{{\color{red}एकः प्रयाति नियतं नवयोजनानि \\
लेखावहः प्रतिदिनं च परोऽपि पञ्च~।\\
गम्यो दशाधिकशतद्वययोजनाध्वा \\
घस्रैस्तयोर्वद सखे कतिभिश्च योगः~॥}}
\end{quote}

न्यासः~। गती ९।५ गम्योऽध्वा २१०~। 
\vspace{2mm}

\renewcommand{\thefootnote}{$\dag$}\footnote{एकस्मात्स्थानाद्युगपच्चलितयोर्गती क्रमेण ${\hbox{ग}}_{\scriptsize{\hbox{१}}}, {\hbox{ग}}_{\scriptsize{\hbox{२}}}$, यत्र\, ${\hbox{ग}}_{\scriptsize{\hbox{१}}}$\,<\,${\hbox{ग}}_{\scriptsize{\hbox{२}}}$ पुरान्तरम् $=$ अ~। तदा प्रथमसमा-गमस्तत्काल एव~। द्वितीयसमागमकालः या कल्प्यते तदा\, ${\hbox{ग}}_{\scriptsize{\hbox{२}}}{\hbox{या}} - {\hbox{अ}}$\, द्वितीयपरावर्त्तनगमनम्~। ${\hbox{ग}}_{\scriptsize{\hbox{१}}}$या प्रथमगमनं स्वस्थानात्~। द्वयोर्योगः
\vspace{2mm}

\hspace{6mm} $= {\hbox{या}}\,({\hbox{ग}}_{\scriptsize{\hbox{१}}} + {\hbox{ग}}_{\scriptsize{\hbox{२}}}) - {\hbox{अ}} = {\hbox{अ}}$~।
\vspace{2mm}

\hspace{2mm} अतः~~ ${\hbox{या}} = \dfrac{{\footnotesize{\hbox{२\,अ}}}}{{\footnotesize{\hbox{ग}}_{\scriptsize{\hbox{१}}} + {\hbox{ग}}_{\scriptsize{\hbox{२}}}}}$~।\; अत उपपन्नम्~। 
\vspace{1mm}
}एकस्मान्नगरान्निर्गतवतो द्वितीयसङ्गम एव समागमकालो दिवसाः ३०~। \\

\noindent \textbf{सूत्रम्~।}

\phantomsection \label{2.40.1}
\begin{quote}
\renewcommand{\thefootnote}{१}\footnote{अत्रोपपत्तिः~। कल्प्यते पुरपरितः परिधिः $=$ प~। द्वयोर्दिनगती क्रमेण ${\hbox{ग}}_{\scriptsize{\hbox{१}}}, {\hbox{ग}}_{\scriptsize{\hbox{२}}}$~। ततोऽनुपातो यदि गत्य-न्तरेणैकं दिनं तदा परिधिना किं लब्धः सङ्गमकालः $= \dfrac{{\footnotesize{\hbox{प}}}}{{\footnotesize{\hbox{ग}}_{\scriptsize{\hbox{१}}} \backsim {\hbox{ग}}_{\scriptsize{\hbox{२}}}}}$~।\; अत उपन्नम्~।}{\large \textbf{{\color{purple}सङ्गमकालः परिधौ \\
गत्यन्तरभाजिते भवति~।}}}
\end{quote}

\noindent \textbf{उदाहरणम्~।}

\phantomsection \label{Ex 2.46.1}
\begin{quote}
\textbf{{\color{red}प्रयाति योजनान्यष्टावेकः पञ्च परस्तथा~।\\
वृत्ते देशस्य परिधिः शतं विंशतिसंयुतम्~॥}}
\end{quote}

\newpage
\begin{sloppypar}

\phantomsection \label{Ex 2.46}
\begin{quote}
\textbf{{\color{red}तयोश्च गच्छतोः स्वेन पथा पथिकयोर्वद~।\\
\renewcommand{\thefootnote}{१}\footnote{स्वपुराद्युगपन्निर्गतवतोः परिधौ भ्रमतोः कतिभिर्दिवसैः पुनः स्वपुरप्राप्तिः\textendash \,इति~।
\vspace{1mm}
}युगपत् स्वपुरप्राप्तिः कतिभिर्दिवसैर्वद~॥}}
\end{quote}

न्यासः~। नियते गती ८।५ देशे परितः परिधिः १२०। जाता निजपुरप्राप्तिकाले दिवसाः ४०~। \\

\noindent \textbf{सूत्रम्~।}

\phantomsection \label{2.40}
\begin{quote}
\renewcommand{\thefootnote}{२}\footnote{अत्रोपपत्तिः~। कल्प्यते प्रथमं भाण्डे द्रव्यमानं $=$ द्र~। क्षरणादन्ते शेषमानम् $=$ शे~। 
\vspace{1mm}

\begin{minipage}{0.01\textwidth}

\end{minipage} 
\hfill
\begin{minipage}{0.15\textwidth}
\begin{tikzpicture}[scale=0.55]
\draw [line width=0.4pt] (0,0)-- (2,0);
\draw [line width=0.4pt] (2,0)-- (2,3);
\draw [line width=0.4pt] (0,3)-- (2,3);
\draw [line width=0.4pt] (0,3)-- (2,2);
\draw [line width=0.4pt] (0,3)-- (0,0);
\draw (-0.65,3.3) node[anchor=north west] {ग};
\draw (-0.65,0.14) node[anchor=north west] {अ};
\draw (2,0.12) node[anchor=north west] {क};
\draw (2,3.3) node[anchor=north west] {च};
\draw (2,2.2) node[anchor=north west] {घ};
\end{tikzpicture}
\end{minipage} 
\hfill
\begin{minipage}[c]{0.7\textwidth} 
ततः क्षरणमानम् $=$ क्ष $=$ द्र $-$ शे~। अध्वपरिमाणम् $=$ अ, भाटकमानम् $=$ भा~। अथ अध्वपरिमाणम् $=$ अक, द्रव्यपरिमाणम् $=$ अग तदा अग.अक, क्षेत्रफलरूपकर्मणो भाटकमानम् $=$ भा~। अथ तदीयकर्ममानं च\, अ क घ ग\, क्षेत्रफलरूपम्~। यत्र कघ $=$ शेषद्रव्यमानम्~। अस्य क्षेत्रफलम्
\vspace{2mm}
\end{minipage}  
\vspace{2mm}

\hspace{10mm} $= \dfrac{{\footnotesize{\hbox{अक}}\,({\hbox{अग}} + {\hbox{कघ}})}}{{\footnotesize{\hbox{२}}}} = \dfrac{{\footnotesize{\hbox{अक}}\,({\hbox{द्र}} + {\hbox{शे}})}}{{\footnotesize{\hbox{२}}}} = \dfrac{{\footnotesize{\hbox{अक}}\,({\hbox{क्ष}} + {\hbox{शे}} + {\hbox{शे}})}}{{\footnotesize{\hbox{२}}}}$ 
\vspace{2mm}

\hspace{10mm} $= \dfrac{{\footnotesize{\hbox{अक}}\,({\hbox{क्ष}} + {\hbox{२\,शे}})}}{{\footnotesize{\hbox{२}}}} = {\hbox{अक}}\left(\dfrac{{\footnotesize{\hbox{क्ष}}}}{{\footnotesize{\hbox{२}}}} + {\hbox{शे}}\right)$~।
\vspace{2mm}

\hspace{2mm} ततोऽनुपातो यदि\, अक.द्र\, एतावता कर्मणा भाटकमानं भा तदा\, ${\hbox{अक}}\left(\dfrac{{\footnotesize{\hbox{क्ष}}}}{{\footnotesize{\hbox{२}}}} + {\hbox{शे}}\right)$\, एतावता कर्मणा किं लब्धं भाटकमानम् 
\vspace{2mm}

\hspace{10mm} $= \dfrac{{\hbox{भा}}\left(\dfrac{{\footnotesize{\hbox{क्ष}}}}{{\footnotesize{\hbox{२}}}} + {\hbox{शे}}\right)}{{\footnotesize{\hbox{द्र}}}}$~~ अत उपपन्नम्~।}{\large \textbf{{\color{purple}प्रस्कन्नभाण्डदलयुत-\\
शेषस्य च भाटकं भवेत् क्षरणे~॥}}}
\end{quote}

प्रस्कन्नभाण्डदलं गलितद्रव्यमानदलं तेन युतं शेषद्रव्यं यत्तस्य त्रैराशिकादिना भाटक-मानमेव क्षरणे भाटकमानं भवेत्~। 
\end{sloppypar}

\newpage

\noindent \textbf{उदाहरणम्~।}

\phantomsection \label{Ex 2.47}
\begin{quote}
\textbf{{\color{red}तैलपलत्रिशतभृते कुतपे दशयोजनानि नीते चेत्~।\\
तद्भाटकजपणानां नवकं कुतपाच्च सुषिरतः स्कन्नम्\renewcommand{\thefootnote}{$\star$}\footnote{\label{F Ex 2.47} स्कन्नं क्षरितं गलितम्\textendash \,इत्यर्थः~।
\vspace{2mm}
}\,।\\
शेषं पलं च षष्टिः किं देयं भाटकं कथय~॥}}
\end{quote}

न्यासः~। १०।३००।९~। शेषम् ६०~। स्कन्नम्$^{\hyperref[F Ex 2.47]{{\scriptsize{\star}}}}$ २४०~। गलिततैलार्धम् १२०~। शेषेण ६० युतम् १८०~। अस्य भाटकं पञ्चराशिकेन\textendash \begin{small}\begin{tabular}{c|c}
१० & १० \\
३०० & १८०\\
९ & 
\end{tabular}\end{small} जाता भाटके पणाः\, ५$\frac{{\footnotesize{\hbox{२}}}}{{\footnotesize{\hbox{५}}}}$~।\\

\noindent \textbf{अन्तर्भाटके करणसूत्रम्~।}

\phantomsection \label{2.41}
\begin{quote}
\renewcommand{\thefootnote}{१}\footnote{अत्र~~ $\dfrac{{\footnotesize{\hbox{भा.अप्र}}}}{{\footnotesize{\hbox{गम्य}}}} =$ यो 
\vspace{1mm}

\hspace{2mm} भाटक\,(यो $-$ भाण्ड) $=$ (यो $-$ य)\,य 
\vspace{1mm}

\hspace{2mm} इति समीकरणं सूत्रानुसारेण सिध्यति~। तत्र य-मानमेवेष्टान्तरभाटकमाचार्येण कथ्यते~।}{\large \textbf{{\color{purple}अध्वप्रमाणगुणितं \\
भाण्डं गम्याध्वभाजितं योगः~।\\
भाण्डोनितयोगघ्नं \\
भाटकमानं प्रजायते घातः~॥\\
योगवधाभ्यां विवरं \\
तत्सङ्क्रामेण भाटकं चाल्पम्~॥}}}
\end{quote}

\noindent \textbf{उदाहरणम्~।}

\phantomsection \label{Ex 2.48}
\begin{quote}
\textbf{{\color{red}तद्बीजपूरशतकस्य दिनेशतुल्यैः \\
क्रोशैर्नगाग्निमितमन्तरभाटकं चेत्~।\\
क्रोशैस्त्रिभिर्गणकवर्य वदाशु देयं \\
किं भाटकं गणकसंसदि वन्दितोऽसि~॥}}
\end{quote}

\newpage
\begin{sloppypar}

न्यासः~। \begin{small}\begin{tabular}{c|c}
१२ & ३ \\
१०० & १००\\
३७ & ०
\end{tabular}\end{small} जातान्यन्तर्भाटके मातुलिङ्गफलानि १०~। अत्र करणम्~। योज-नानि ३ प्रमाणयोजनेभ्यः १२ अपास्य शेषं गम्ययोजनानि ९~। अध्वप्रमाणेन १२ भाण्डं १०० गुणितं १२०० गम्ययोजनै\textendash \,९\textendash \,र्भक्तं\, $\frac{{\footnotesize{\hbox{४००}}}}{{\footnotesize{\hbox{३}}}}$\, अयं योगः~। अस्मात्\, $\frac{{\footnotesize{\hbox{४००}}}}{{\footnotesize{\hbox{३}}}}$\, भाण्ड\textendash \,१००\textendash \,मपास्य शेषम्\, $\frac{{\footnotesize{\hbox{१००}}}}{{\footnotesize{\hbox{३}}}}$\, एतद्भाटकेनानेन ३७ गुणितं जातो घातः\, $\frac{{\footnotesize{\hbox{३७००}}}}{{\footnotesize{\hbox{३}}}}$~। एवं जातौ योगघातौ\, $\frac{{\footnotesize{\hbox{४००}}}}{{\footnotesize{\hbox{३}}}}$\,।\, $\frac{{\footnotesize{\hbox{३७००}}}}{{\footnotesize{\hbox{३}}}}$~। \hyperref[1.35]{योगकृतेश्चतुराहतघातोनायाः पदं विवरम्'} इति जातं विवरम्\, $\frac{{\footnotesize{\hbox{३४०}}}}{{\footnotesize{\hbox{३}}}}$\,~। सङ्क्रमणेन जातौ राशी\, $\frac{{\footnotesize{\hbox{३७०}}}}{{\footnotesize{\hbox{३}}}}$\,।१०~। अनयोरल्पं ग्राह्यम्~। \\
\end{sloppypar}

\noindent \textbf{सत्यानृते सूत्रम्~।}

\phantomsection \label{2.43}
\begin{quote}
{\large \textbf{{\color{purple}सैकेष्टगुणाः पुरुषा \\
द्विगुणेष्टोना भवन्त्यसत्यानि~।\\
तैरूना पुरुषकृतिः \\
शेषं सत्यानि वचनानि~॥}}}
\end{quote}

\noindent \textbf{उदाहरणम्~।}

\phantomsection \label{Ex 2.49}
\begin{quote}
\textbf{{\color{red}कामुकाः पञ्च पारस्त्रियस्तेषु\renewcommand{\thefootnote}{$\star$}\footnote{The reading परायस्त्रियस्तेषु doesn't fit in the meter of the verse.} च \\
द्वौ प्रियावप्रियास्तत्त्रयस्तान् पृथक्~।\\
त्वं प्रियोऽसीति मे भाषमाणाद्भुतं \\
कानि सत्यान्यसत्यानि शीघ्रं वद~॥}}
\end{quote}

\newpage

न्यासः~। पुरुषाः ५~। प्रियौ २~। अप्रियाः ३~। एकेनेष्टेन जातान्यसत्यानि ८ सत्यवचनानि १७~। द्विकेनेष्टेनासत्यानि ११, एवमिष्टवशादनेकधा~। 

\begin{center}
\textbf{इति सकलकलानिधिनरसिँहनन्दनगणितविद्याचतुरानननारायणपण्डित-\\
विरचितायां गणितकौमुद्यां मिश्रव्यवहारः~॥}
\vspace{7mm}

{\Large \textbf{अथ श्रेढीव्यवहारः~।}}
\end{center}

\noindent \textbf{सूत्रम्~।}

\phantomsection \label{3.1}
\begin{quote}
\renewcommand{\thefootnote}{१}\footnote{अत्रोपपत्तिः~। प्रथमसूत्रस्य {\color{violet}'व्येकपदघ्नचयो मुखयुक् स्यात्'} इति {\color{violet}भास्कर}प्रकारोपपत्त्या स्फुटा~। द्वितीय-सूत्रपूर्वार्धस्य च प्रथमसूत्रानुसारेण
\vspace{2mm}

\hspace{8mm} मध्यधनम् $= \dfrac{{\footnotesize{\hbox{मु}} + {\hbox{च}}\,({\hbox{प}} - {\hbox{१}}) + {\hbox{मु}}}}{{\footnotesize{\hbox{२}}}} = {\hbox{मु}} + \dfrac{{\footnotesize{\hbox{च}}}}{{\footnotesize{\hbox{२}}}}\,({\hbox{प}} - {\hbox{१}})$~।
\vspace{2mm}

\hspace{2mm} ततः~~ सर्वधनम् $=$ प $\times$ मध $= {\hbox{प}}\left({\hbox{मु}} + \dfrac{{\footnotesize{\hbox{च}}}}{{\footnotesize{\hbox{२}}}}\,({\hbox{प}} - {\hbox{१}})\right)$\; इत्यनेन स्फुटा~।}{\large \textbf{{\color{purple}व्येकपदघ्नचयो मुख-\\
युक्तोऽन्त्यधनं तु तत्पुनः सादि~।\\
दलितं मध्यधनं तत् \\
पदगुणितं जायते गणितम्~॥~१~॥\\
व्येकपदार्धघ्नचयः \\
सादिः पदसङ्गुणो भवेद्गणितम्~।}}}
\end{quote}

\noindent \textbf{उदाहरणम्~।}

\phantomsection \label{Ex 3.1.1}
\begin{quote}
\textbf{{\color{red}आद्येऽहनि द्विजवराय धनी ददाति \\
निष्कत्रयं प्रतिदिनं द्विचयेन यावत्~।}}
\end{quote}

\newpage

\begin{sloppypar}
\phantomsection \label{Ex 3.1}
\begin{quote}
\textbf{{\color{red}मासार्धमत्र कथयान्त्यधनं च मध्यं \\
सर्वं धनं यदि सखे गणितं तवास्ति~॥}}
\end{quote}

न्यासः~। आदिः ३~। उत्तरः २~। गच्छः १५~। जातमन्त्यधनम् ३१~। मध्यधनम् १७~।~सर्व-धनम् २५५~। अस्य गणितसञ्ज्ञा कृता~। \\

\noindent \textbf{उदाहरणम्~।}

\phantomsection \label{Ex 3.2}
\begin{quote}
\textbf{{\color{red}द्व्यादिना त्रिचयेनाशु दिनैः षड्भिः समर्जितम्~।\\
वणिजा केनचिन्मध्यमन्त्यं च गणितं वद~॥}}
\end{quote}

न्यासः~। आदिः २~। त्रयः ३~। गच्छः ६~। जातमन्त्यधनम् १७~। मध्यधनम्\, $\frac{{\footnotesize{\hbox{१९}}}}{{\footnotesize{\hbox{२}}}}$~।~गणितम् ५७~। अत्र समदिनगच्छे मध्यदिनाभावात् तत्प्रागपरदिनदत्तधनयोर्योगार्धं मध्यधनं भवतीति छात्राणां प्रतीतिरुत्पादनीया~। \\

\noindent \textbf{आद्यानयने सूत्रम्~।}

\phantomsection \label{3.2}
\begin{quote}
\renewcommand{\thefootnote}{१}\footnote{इदं {\color{violet}'गच्छहृते गणिते वदनं स्यात्'} इत्यादि {\color{violet}भास्करा}नुरूपमेव~।}{\large \textbf{{\color{purple}वदनं पदभक्तफले \\
व्येकपदघ्नोत्तरार्धोने~॥~२~॥}}}
\end{quote}

\noindent \textbf{उदाहरणम्~।}

\phantomsection \label{Ex 3.3}
\begin{quote}
\textbf{{\color{red}वासरैः सप्तभिस्त्र्युत्तरेणाध्वगः \\
संययौ योजनान्यष्टषष्ट्युत्तरम्~।\\
ब्रूहि विद्वञ्छतं वादिना केन भोः \\
श्रेढिकौतुहले प्रौढता तेऽस्ति चेत्~॥}}
\end{quote}

न्यासः~। आदिः ०~। चयः ३~। गच्छः ७~। श्रेढीफलम् १६८~। ज्ञात आदिः १५~। \\

\noindent \textbf{उदाहरणम्~।}

\phantomsection \label{Ex 3.4}
\begin{quote}
\textbf{{\color{red}आदिश्चतुष्टयं विद्वन् फलं षोडशसंयुते~।\\
द्वे शते नवभिर्ब्रूहि दिनैः केनोत्तरेण मे~॥}}
\end{quote}
\end{sloppypar}

\newpage

न्यासः~। आदिः ४~। उत्तरः ०~। गच्छः ९~। श्रेढीफलम् २१६~। ज्ञातः प्रचयः ५~। \\

\noindent \textbf{गच्छानयने सूत्रम्~।}

\phantomsection \label{3.3}
\begin{quote}
\renewcommand{\thefootnote}{१}\footnote{{\color{violet}'श्रेढीफलादुत्तरलोचनघ्नात्~।'} इत्यादि {\color{violet}भास्करो}क्तानुरूपमेवेदम्~।}{\large \textbf{{\color{purple}द्विगुणचयघ्नाद्गणितात्\renewcommand{\thefootnote}{$\star$}\footnote{The reading गुणितात् seems to be a typographical error.} \\
चयदलमुखविवरवर्गसंयुक्तात्~॥~३~॥\\
मूले विमुखे चयदल-\\
युक्ते चयभाजिते गच्छः~।}}}
\end{quote}

\noindent \textbf{उदाहरणम्~।}

\phantomsection \label{Ex 3.5}
\begin{quote}
\textbf{{\color{red}अष्टावादिश्चयः षट् च गणितत्रिशती सखे~।\\
द्वादशोना वद क्षिप्रं जाता केन पदेन मे~॥}}
\end{quote}

न्यासः~। आदिः ८~। चयः ६~। गच्छः ०~। श्रेढीफलम् २८८~। ज्ञातो गच्छः ९~। \\

\noindent \textbf{सूत्रम्~।}

\phantomsection \label{3.4}
\begin{quote}
\renewcommand{\thefootnote}{२}\footnote{अत्रोपपत्तिः स्फुटैव~।}{\large \textbf{{\color{purple}गच्छोऽभीष्टः कार्यः \\
तथा मुखं तच्चयोऽथवाभीष्टः~।\\
यदविज्ञातं प्राग्वत् \\
स्वसूत्रविधिनैव विज्ञेयम्~॥~४~॥}}}
\end{quote}

\noindent \textbf{उदाहरणम्~।}

\phantomsection \label{Ex 3.6}
\begin{quote}
\textbf{{\color{red}श्रेढीफलं शतं येषु विस्मृतेषु मुखादिषु~।\\
गणिते ब्रूहि गच्छादि\renewcommand{\thefootnote}{$\dag$}\footnote{The reading वक्रादि is unintelligible in the given context.} श्रेढीमार्गेऽसि कोविदः~॥}}
\end{quote}

न्यासः~। आ ०~। च ०~। ग ०~। श्रेफ १००~। अत्रादिगच्छाविष्टौ कल्पितौ~। आ १~। च ०~। ग १०~। श्रे.फ १००~। जात उत्तरसूत्रेणोत्तरः २~। अथवोत्तरगच्छौ कल्पितौ, आ ०~। उ १~। ग १०~। श्रे.फ १००~। 

\newpage

\noindent जात आदिसूत्रेणादिः $\frac{{\footnotesize{\hbox{११}}}}{{\footnotesize{\hbox{२}}}}$~। \\

\noindent \textbf{सूत्रम्~।}

\phantomsection \label{3.5}
\begin{quote}
\renewcommand{\thefootnote}{१}\footnote{अत्रोपपत्तिः~। आद्युत्तरगच्छानां घातेऽभीष्टमुखहृते लब्धश्चयगच्छवधः क्रमाख्यः~। अथ चेच्चयमानम् $=$ च~। तदा पूर्वविधिना 
\vspace{2mm}

\hspace{2mm} श्रेढीफलम् $= {\hbox{फ}} = \dfrac{{\footnotesize{\hbox{ग}}}}{{\footnotesize{\hbox{२}}}}\,[{\hbox{२\,मु}} + {\hbox{च}}\,({\hbox{ग}} - {\hbox{१}})]$
\vspace{2mm}

\hspace{21mm} $= \dfrac{{\footnotesize{\hbox{च.ग}}}}{{\footnotesize{\hbox{२\,च}}}}\,[{\hbox{२\,मु}} + {\hbox{च.ग}} - {\hbox{च}}] = \dfrac{{\footnotesize{\hbox{क्र}}}}{{\footnotesize{\hbox{२\,च}}}}\,[{\hbox{२\,मु}} + {\hbox{क्र}} - {\hbox{च}}]$
\vspace{2mm}

\hspace{2mm} अतः~~ २\,च.फ $=$ २\,मु.क्र $+$ क्र$^{\scriptsize{\hbox{२}}} -$ च.क्र
\vspace{1mm}

\hspace{2mm} समशोधनेन~~ च\,(२\,फ $+$ क्र) $=$ क्र\,(२ मु $+$ क्र)
\vspace{2mm}

\hspace{4mm} $\therefore\; {\hbox{च}} = \dfrac{{\footnotesize{\hbox{क्र}}\,({\hbox{२\,मु}} + {\hbox{क्र}})}}{{\footnotesize{\hbox{२\,फ}} + {\hbox{क्र}}}} = \dfrac{\dfrac{{\footnotesize{\hbox{क्र}}\,({\hbox{२\,मु}} + {\hbox{क्र}})}}{{\footnotesize{\hbox{२}}}}}{{\footnotesize{\hbox{फ}}} + \dfrac{{\footnotesize{\hbox{क्र}}}}{{\footnotesize{\hbox{२}}}}}$~।
\vspace{2mm}

\hspace{2mm} अथ~~ क्र $=$ च.ग \hspace{2mm} $\therefore$\; ग $= \dfrac{{\footnotesize{\hbox{क्र}}}}{{\footnotesize{\hbox{च}}}}$~।~ अत उपपद्यते सर्वम्~।}{\large \textbf{{\color{purple}घातस्त्वभीष्टवदनेन हृतः क्रमाख्यो \\
द्विघ्नं मुखं क्रमयुतघ्नमथार्धितं तत्~।\\
भक्तं क्रमार्धसहितेन फलेन वृद्धिः \\
वृद्ध्या हृता क्रममितिः पदमत्र तत् स्यात्~॥}}}
\end{quote}

\noindent \textbf{उदाहरणम्~।}

\phantomsection \label{Ex 3.7}
\begin{quote}
\textbf{{\color{red}आदिगच्छोत्तराणां वधे द्वादश \\
श्रेढिवित्तं दश ब्रूहि तस्मिन् सखे~।\\
आदिगच्छोत्तराणां मितिः का भवेत् \\
स्याद्वधो रूपमेकं फलं वा समम्~॥}}
\end{quote}

न्यासः~। आद्युत्तरपदानां घातः १२~। श्रेढीफलम् १०~। एकेनेष्टेन

\newpage

\noindent जाता आद्युत्तरगच्छाः\, $\frac{{\footnotesize{\hbox{१}}}}{{\footnotesize{\hbox{१}}}}$\,।\,$\frac{{\footnotesize{\hbox{२१}}}}{{\footnotesize{\hbox{४}}}}$\,।\,$\frac{{\footnotesize{\hbox{१६}}}}{{\footnotesize{\hbox{७}}}}$\,।\, द्विकेन\, $\frac{{\footnotesize{\hbox{२}}}}{{\footnotesize{\hbox{१}}}}$\,।\,$\frac{{\footnotesize{\hbox{३०}}}}{{\footnotesize{\hbox{१३}}}}$\,।\,$\frac{{\footnotesize{\hbox{१३}}}}{{\footnotesize{\hbox{५}}}}$\,।\, त्रिकेण $\frac{{\footnotesize{\hbox{३}}}}{{\footnotesize{\hbox{१}}}}$\,।\,$\frac{{\footnotesize{\hbox{५}}}}{{\footnotesize{\hbox{३}}}}$\,।\,$\frac{{\footnotesize{\hbox{१}}}}{{\footnotesize{\hbox{५}}}}$\,।\\ 

द्वितीयोदाहरणे~। आद्युत्तरपदानां घातः १२~। गणितम् १~। अर्धेन जाता आद्युत्तरगच्छाः\, $\frac{{\footnotesize{\hbox{१}}}}{{\footnotesize{\hbox{२}}}}$\,।\,$\frac{{\footnotesize{\hbox{३}}}}{{\footnotesize{\hbox{२}}}}$\,।\,$\frac{{\footnotesize{\hbox{४}}}}{{\footnotesize{\hbox{३}}}}$\,।\, त्र्यंशेन\, $\frac{{\footnotesize{\hbox{१}}}}{{\footnotesize{\hbox{३}}}}$\,।\,$\frac{{\footnotesize{\hbox{११}}}}{{\footnotesize{\hbox{५}}}}$\,।\,$\frac{{\footnotesize{\hbox{१५}}}}{{\footnotesize{\hbox{११}}}}$\,।\, एवमिष्टवशादानन्त्यम्~। \\
\vspace{2mm}

\noindent \textbf{सूत्रम्~।}

\phantomsection \label{3.6}
\begin{quote}
\renewcommand{\thefootnote}{१}\footnote{अत्रोपपत्तिः~। कल्प्यते प्रथममुखम् $= {\hbox{मु}}_{\scriptsize{\hbox{१}}}$~। चयः $= {\hbox{च}}_{\scriptsize{\hbox{१}}}$~। गच्छः $= {\hbox{ग}}_{\scriptsize{\hbox{१}}}$~। एवं द्वितीयस्य मुखचयौ ${\hbox{मु}}_{\scriptsize{\hbox{२}}}$~। ${\hbox{च}}_{\scriptsize{\hbox{२}}}$~। ततः प्रश्नालापानुसारेण यदि गच्छमानम् $=$ ग\, तदा 
\vspace{2mm}

\hspace{6mm} $\dfrac{{\footnotesize{\hbox{ग}}}}{{\footnotesize{\hbox{२}}}}\,[\,{\hbox{२\,मु}}_{\scriptsize{\hbox{१}}} + {\hbox{च}}_{\scriptsize{\hbox{१}}}\,({\hbox{ग}} - {\hbox{१}})] = \dfrac{{\footnotesize{\hbox{ग}}}}{{\footnotesize{\hbox{२}}}}\,[\,{\hbox{२\,मु}}_{\scriptsize{\hbox{२}}} + {\hbox{च}}_{\scriptsize{\hbox{२}}}\,({\hbox{ग}} - {\hbox{१}})]$ 
\vspace{2mm}

\hspace{2mm} समशोधनादिना~~ ${\hbox{२}}\,({\hbox{मु}}_{\scriptsize{\hbox{१}}} - {\hbox{मु}}_{\scriptsize{\hbox{२}}}) - {\hbox{च}}_{\scriptsize{\hbox{१}}} + {\hbox{च}}_{\scriptsize{\hbox{२}}} = {\hbox{ग}}\,({\hbox{च}}_{\scriptsize{\hbox{२}}} - {\hbox{च}}_{\scriptsize{\hbox{१}}})$
\vspace{2mm}

\hspace{6mm} $\therefore$\; ग $= \dfrac{{\footnotesize{\hbox{२}}\,({\hbox{मु}}_{\scriptsize{\hbox{१}}} - {\hbox{मु}}_{\scriptsize{\hbox{२}}}) + {\hbox{च}}_{\scriptsize{\hbox{२}}} - {\hbox{च}}_{\scriptsize{\hbox{१}}}}}{{\footnotesize{\hbox{च}}_{\scriptsize{\hbox{२}}} - {\hbox{च}}_{\scriptsize{\hbox{१}}}}} = \dfrac{{\footnotesize{\hbox{२}}\,({\hbox{मु}}_{\scriptsize{\hbox{१}}} - {\hbox{मु}}_{\scriptsize{\hbox{२}}})}}{{\footnotesize{\hbox{च}}_{\scriptsize{\hbox{२}}} - {\hbox{च}}_{\scriptsize{\hbox{१}}}}}$ + १
\vspace{2mm}

\hspace{13mm} $= \dfrac{{\footnotesize{\hbox{मु}}_{\scriptsize{\hbox{१}}} - {\hbox{मु}}_{\scriptsize{\hbox{२}}}}}{\dfrac{{\footnotesize{\hbox{च}}_{\scriptsize{\hbox{२}}} - {\hbox{च}}_{\scriptsize{\hbox{१}}}}}{{\footnotesize{\hbox{२}}}}} + {\hbox{१}} = \dfrac{{\footnotesize{\hbox{मु}}_{\scriptsize{\hbox{१}}} - {\hbox{मु}}_{\scriptsize{\hbox{२}}}}}{\dfrac{{\footnotesize{\hbox{च}}_{\scriptsize{\hbox{२}}}}}{{\footnotesize{\hbox{२}}}} - \dfrac{{\footnotesize{\hbox{च}}_{\scriptsize{\hbox{१}}}}}{{\footnotesize{\hbox{२}}}}} + {\hbox{१}}$~।~ इत्युपपन्नम्~।
\vspace{2mm}

\hspace{2mm} अत्र यदि\, ${\hbox{मु}}_{\scriptsize{\hbox{१}}}$\,>\,${\hbox{मु}}_{\scriptsize{\hbox{२}}}$\, तदा\, ${\hbox{च}}_{\scriptsize{\hbox{२}}}$\,>\,${\hbox{च}}_{\scriptsize{\hbox{१}}}$, यदि स्यात्तदैव धनात्मिका लब्धिरन्यथा नेति स्पष्टम्~।}{\large \textbf{{\color{purple}एको बृहदाद्यल्प-\\
प्रचयस्त्वपरो मुखं बृहत्प्रचयः~।\\
तन्मुखविवरे चयदल-\\
वियोगभक्ते सरूपके गच्छः~॥}}}
\end{quote}

\newpage

\noindent \textbf{उदाहरणम्~।}

\phantomsection \label{Ex 3.8}
\begin{quote}
\textbf{{\color{red}आद्ये दिने निधिमितानि च योजनानि \\
पञ्चोत्तरेण पथिको नियमेन याति~।\\
अन्यः प्रयाति युगलं दिवसे तथाद्ये \\
सप्तोत्तरेण दिवसैर्वद कैश्च योगः~॥}}
\end{quote}

न्यासः~। \begin{small}\begin{tabular}{l|}
आ ९ च ५ ग ० \\
आ २ च ७ ग ०
\end{tabular}\end{small}~ जातो गच्छः ८~। समपथिकयोगयोजनानि २१२~। \\
\vspace{2mm}

\noindent \textbf{सूत्रम्~।}

\phantomsection \label{3.7}
\begin{quote}
\renewcommand{\thefootnote}{१}\footnote{अत्रोपपत्तिः~।\; प्रथमस्य मुखचयगच्छाः\, ${\hbox{मु}}_{\scriptsize{\hbox{१}}}, {\hbox{च}}_{\scriptsize{\hbox{१}}}, {\hbox{ग}}_{\scriptsize{\hbox{१}}}$~।\; द्वितीयस्य मुखचयगच्छाः\, ${\hbox{मु}}_{\scriptsize{\hbox{२}}}, {\hbox{च}}_{\scriptsize{\hbox{२}}}, {\hbox{ग}}_{\scriptsize{\hbox{२}}}$~।\; तदा\, ${\hbox{ग}}_{\scriptsize{\hbox{१}}} + {\hbox{ग}}_{\scriptsize{\hbox{२}}}$ गच्छे प्रथमस्य गणितम् 
\vspace{2mm}

\hspace{6mm} $= \dfrac{{\footnotesize{\hbox{ग}}_{\scriptsize{\hbox{१}}} + {\hbox{ग}}_{\scriptsize{\hbox{२}}}}}{{\footnotesize{\hbox{२}}}}\,[\,{\hbox{२\,मु}}_{\scriptsize{\hbox{१}}} + {\hbox{च}}_{\scriptsize{\hbox{१}}}\,({\hbox{ग}}_{\scriptsize{\hbox{१}}} + {\hbox{ग}}_{\scriptsize{\hbox{२}}} - {\hbox{१}})]$~। 
\vspace{2mm}

\hspace{2mm} तथा ${\hbox{ग}}_{\scriptsize{\hbox{२}}}$ गच्छे द्वितीयस्य गणितम् 
\vspace{2mm}

\hspace{6mm} $= \dfrac{{\footnotesize{\hbox{ग}}_{\scriptsize{\hbox{२}}}}}{{\footnotesize{\hbox{२}}}}\,[\,{\hbox{२\,मु}}_{\scriptsize{\hbox{२}}} + {\hbox{च}}_{\scriptsize{\hbox{२}}}\,({\hbox{ग}}_{\scriptsize{\hbox{२}}} - {\hbox{१}})]$~। \vspace{2mm}

\hspace{2mm} सङ्गमे द्वयोर्गणिते समाने स्तः~। अथ यदि\, ${\hbox{मु}}_{\scriptsize{\hbox{१}}}, {\hbox{च}}_{\scriptsize{\hbox{१}}}, {\hbox{ग}}_{\scriptsize{\hbox{१}}}, {\hbox{च}}_{\scriptsize{\hbox{२}}}, {\hbox{ग}}_{\scriptsize{\hbox{२}}}$\, व्यक्तास्तदा प्रथमगणितं च व्यक्तं तदेव द्वितीयगणितं प्रकल्प्य ज्ञाततच्चयगच्छाभ्यां द्वितीयमुखज्ञानं पूर्वविधिना सुलभम्~। ततोऽग्रे प्रथमस्य प्रथमदिनगतिः $= {\hbox{मु}}_{\scriptsize{\hbox{१}}} + {\hbox{च}}_{\scriptsize{\hbox{१}}}\,({\hbox{ग}}_{\scriptsize{\hbox{१}}} + {\hbox{ग}}_{\scriptsize{\hbox{२}}})$~।\, द्वितीयस्य प्रथमदिनगतिः $= {\hbox{मु}}_{\scriptsize{\hbox{२}}} + {\hbox{च}}_{\scriptsize{\hbox{२}}}{\hbox{ग}}_{\scriptsize{\hbox{२}}}$~।\, एतौ द्वितीय-योगसाधनार्थं नरयोर्मुखमानं प्रकल्प्यानन्तरोक्तसूत्रेण द्वितीययोगमानं साधनीयमिति~।}{\large \textbf{{\color{purple}प्रथमस्यानल्पचयो \\
मुखमिष्टं पदमितिर्द्वितीयस्य~।\\
इष्टाद्यस्य फलादेः \\
कल्प्यं वदनं द्वितीयस्य~॥~७~॥\\
पदचयघातौ समुखौ \\
द्वितीययोगे च मुखमिती भवतः~।}}}
\end{quote}

\newpage

\phantomsection \label{3.8}
\begin{quote}
{\large \textbf{{\color{purple}ताभ्यां च पूर्वविधिना \\
द्वितीययोगे च युतिदिवसाः~॥}}}
\end{quote}

\noindent \textbf{उदाहरणम्~।}

\phantomsection \label{Ex 3.9}
\begin{quote}
\textbf{{\color{red}केनाप्यादिचयेन याति च पुरस्त्वेको नरोऽष्टौ दिना-\\
न्यन्यः केनचिदादिना द्विकचयेनानूपसर्पन् क्रमात्~।\\
मार्गे मित्र तयोर्द्विवारमभवत् सङ्गो दिनैः कैर्वद\\
श्रेढीवेदिकरीन्द्रवारणरणप्रौढो हरीन्द्रोऽसि चेत्~॥}}
\end{quote}

\begin{sloppypar}
अत्र करणम्~। तत्र प्रथमस्यादिर्द्वयम् २ उत्तरश्चतुष्टयं ४ कल्पितम्~। आ २, उ ४, ग ०~। पुरतो दिनानि ८~। द्वितीयस्य आ ०~। उ २ गच्छोऽभीष्टः कल्पितः ४~। पुरोगदितदिनयुतो जातः प्रथमस्य गच्छः १२~। न्यासः~। आ २, उ ४, ग १२~। अस्य गणितम् २८८~। एतत्प्रथमसङ्गमे द्वितीयस्य गणितम्~। अथ प्राग्वद्द्वितीयस्यादिः ६९~। प्रथमस्य~पदचयघातो मुखयुतः ५०~। द्वितीयस्य पदचयघातो मुखयुतः ७७~। एतावादी कल्पितौ \begin{small}\begin{tabular}{l|}
आ ५० उ ४ ग ० \\
आ ७७ उ २ ग ०
\end{tabular}\end{small}\; \hyperref[3.6]{'तन्मुखविवरे'} इत्यादिना जातः प्रथमसङ्गमादग्रतो द्वितीयसङ्गमकालो दिवसाः २८~। एतत्पूर्वगच्छयोरेतयोः १२।४ पृथक् पृथक् संयोज्य जातौ गच्छौ ४०।३२ समानि पथिकयोर्योजनानि ३२००~। \\

अथ द्वितीयसमागमकालश्चेदृणगतस्तदा गच्छाभ्यां संशोध्य शेषं प्रथमसमागमकालः स्यात्~। तद्यथा\textendash \\

तस्मिन्नेवोदाहरणे द्वितीयस्य कल्पितो गच्छः षोडशमितः १६ पुरतो दिनयुतोऽयं गम्यस्य गच्छः २४~। एवं जातः प्रथमः~। आ २, उ ४, ग २४~। द्वितीयः आ ०, उ २, ग १६~। प्रथमस्य गणितम् ११५२~। अतो द्वितीयस्यादिः ५७~। \hyperref[3.7]{'पदचयघातौ समुखौ'} इत्यादिना न्यासः~। \begin{small}\begin{tabular}{l|}
आ ९८ उ ४ ग ० \\
आ ८९ उ २ ग ०
\end{tabular}\end{small}\; अत्रैको बृहदाद्यल्प इत्यस्मिन्नुपलक्षणं न दृश्यते यतोऽष्टनवतिम् एकोननवतेर्विशोध्य शेषम् ९ं~। एतच्च पददलवियोगेनानेन १ भक्तं लब्धं ९ं सरूपकमिति ऋणत्वाल्लब्धस्यैकोनम् ८ं~।
\end{sloppypar}

\newpage

\noindent एतत्पूर्वगच्छाभ्यां २४।१६ युक्तमिति पृथक् पृथगन्तरे जातौ गच्छौ १६।८~। एते प्रथमसङ्गमे पूर्वगच्छे द्वितीयसङ्कलनावशादनेकधा~। \\

\noindent \textbf{सूत्रम्~।}

\phantomsection \label{3.9.1}
\begin{quote}
\renewcommand{\thefootnote}{१}\footnote{अत्रोपपत्तिः~। कल्प्यते प्रथमायाः समानगतिः $=$ स, दूत्या आदिमानम् $=$ मु~। चयमानम् $=$ च~। कल्प्यते युतिदिनम् $=$ ग~। तदा 
\vspace{1mm}

\hspace{2mm} प्रथमाया गमनमानम् $=$ स.ग~। 
\vspace{1mm}

\hspace{2mm} दूत्या गमनमानम् $= \dfrac{{\footnotesize{\hbox{ग}}}}{{\footnotesize{\hbox{२}}}}\,[{\hbox{२\,मु}} + {\hbox{च}}\,({\hbox{ग}} - {\hbox{१}})]$~~ एतद्द्वयं समानम्~। 
\vspace{2mm}

\hspace{2mm} अतः~~ $\dfrac{{\footnotesize{\hbox{ग}}}}{{\footnotesize{\hbox{२}}}}\,[{\hbox{२\,मु}} + {\hbox{च}}\,({\hbox{ग}} - {\hbox{१}})] =$ स.ग 
\vspace{2mm}

\hspace{2mm} वा~~ २\,मु $+$ च (ग $-$ १) $=$ २\,स 
\vspace{2mm}

\hspace{2mm} $\therefore\; {\hbox{ग}} = \dfrac{{\footnotesize{\hbox{२\,स}} - {\hbox{२\,मु}}}}{{\footnotesize{\hbox{च}}}} + {\hbox{१}} = \dfrac{{\footnotesize{\hbox{स}} - {\hbox{मु}}}}{\dfrac{{\footnotesize{\hbox{च}}}}{{\footnotesize{\hbox{२}}}}} + {\hbox{१}}$~।
\vspace{2mm}
}{\large \textbf{{\color{purple}नियतगतिर्वदनोना \\
चयदलहृद्रूपसंयुता गच्छः~।}}}
\end{quote}

\noindent \textbf{उदाहरणम्~।}

\phantomsection \label{Ex 3.10}
\begin{quote}
\textbf{{\color{red}संस्मृत्य कान्तं रमणी स्मरातुरा \\
प्रयाति नित्यं दशयोजनानि ताम्~।\\
त्र्यादिद्विवृद्ध्यानुचचार शम्भली \\
समागमः कैर्दिवसैस्तयोर्भवेत्~॥}}
\end{quote}

न्यासः~। कामिनी दिनगतिः १०~। \renewcommand{\thefootnote}{$\star$}\footnote{{\color{violet}'कुट्टिनी शम्भली समे'} इति {\color{violet}अमरः}~।}शम्भलीगतिः आ ३ उ २~। जाताः संयोगदिवसाः ८~। 

\newpage

\noindent \textbf{सङ्कलितवर्गघनानामुपलक्षणसूत्रम्~।}

\phantomsection \label{3.9}
\begin{quote}
\renewcommand{\thefootnote}{१}\footnote{सङ्कलिते रूपं वर्गे द्वे चयः कल्प्यः~। तयोर्द्वयोरादिस्तु रूपमेव~। घने पदसम आदिः~। द्विगुणादिश्च चयः कल्प्यः~। एवं यद्यद्योगानयनमभीष्टं तत्तन्मुखयोगो मुखं तत्तच्चययोगश्च चयः कल्प्यः~। 
\vspace{1mm}

\hspace{2mm} अत्रोपपत्तिः~। कल्प्यते प पदस्य सङ्कलितं वर्गो घनश्चापेक्षितः~। 
\vspace{1mm}

तदा~~ सं $=$ १ $+$ २ $+$ ३ $+$ ... $+$ प~। अतः मु $=$ १~। चयः $=$ १~। 
\vspace{1mm}

\hspace{6mm} व $=$ प$^{\scriptsize{\hbox{२}}} = \dfrac{{\footnotesize{\hbox{प}}}}{{\footnotesize{\hbox{२}}}}$\,(२\,प) $= \dfrac{{\footnotesize{\hbox{प}}}}{{\footnotesize{\hbox{२}}}}$\,(२ $+$ २\,प $-$ २) $= \dfrac{{\footnotesize{\hbox{प}}}}{{\footnotesize{\hbox{२}}}}$\,[१ $\times$ २ $+$ २\,(प $-$ १)]
\vspace{2mm}

\hspace{2mm} अतोऽत्र~~ मु $=$ १~। च $=$ २~। तथा 
\vspace{2mm}

\hspace{6mm} घ $=$ प$^{\scriptsize{\hbox{३}}} = \dfrac{{\footnotesize{\hbox{प}}}}{{\footnotesize{\hbox{२}}}}$\,(२\,प$^{\scriptsize{\hbox{२}}}$) $= \dfrac{{\footnotesize{\hbox{प}}}}{{\footnotesize{\hbox{२}}}}$\,(२\,प $+$ २\,प$^{\scriptsize{\hbox{२}}} -$ २\,प) $= \dfrac{{\footnotesize{\hbox{प}}}}{{\footnotesize{\hbox{२}}}}$\,[प $\times$ २ $+$ २\,प\,(प $-$ १)]
\vspace{2mm}

\hspace{2mm} अतोऽत्र~~ मु $=$ प~। च $=$ २~। एवम् 
\vspace{2mm}

\hspace{6mm} सं $+$ व $= \dfrac{{\footnotesize{\hbox{प}}}}{{\footnotesize{\hbox{२}}}}$\,(प $+$ १) $+ \dfrac{{\footnotesize{\hbox{प}}}}{{\footnotesize{\hbox{२}}}}$\,[१ $\times$ २ $+$ २\,(प $-$ १)]
\vspace{2mm}

\hspace{15mm} $= \dfrac{{\footnotesize{\hbox{प}}}}{{\footnotesize{\hbox{२}}}}$\,\{[१ $\times$ २ $+$ १\,(प $-$ १)] $+$ [१ $\times$ २ $+$ २\,(प $-$ १)]\}
\vspace{2mm}

\hspace{15mm} $= \dfrac{{\footnotesize{\hbox{प}}}}{{\footnotesize{\hbox{२}}}}$\,[(१ $+$ १) $\times$ २ $+$ (१ $+$ २)\,(प $-$ १)]
\vspace{2mm}

\hspace{2mm} अतोऽत्र~~ मु $=$ १ $+$ १~। च $=$ १ $+$ २~। एवं सर्वत्र ज्ञेयमित्युपपन्नम्~।}{\large \textbf{{\color{purple}रूपं द्वे रूपे च \\
प्रचयः सङ्कलितवर्गयोरादिः~।}}}
\end{quote}

\newpage
\begin{sloppypar}

\phantomsection \label{3.10}
\begin{quote}
{\large \textbf{{\color{purple}रूपं घने तु पदसम \\
आदिः प्रचयो द्विसङ्गुणितः~।\\
यद्यद्योगानयनं \\
तन्मुखयोगश्च चययोगः~॥}}}
\end{quote}

\noindent \textbf{उदाहरणम्~।}

\phantomsection \label{Ex 3.11}
\begin{quote}
\textbf{{\color{red}सङ्कलितं पञ्चानां वर्गं च घनं च मे पृथक्\renewcommand{\thefootnote}{$\star$}\footnote{The reading पृथक् पृथक् doesn’t fit in the meter of the verse.
\vspace{1mm}
} कथय~।\\
द्वन्द्वयुतिं सर्वयुतिं श्रेढीविधिना सखे शीघ्रम्~॥}}
\end{quote}

न्यासः~। आ १ उ १ ग ५~। वर्गे, आ १ उ २ ग ५~। घने, आ ५ उ १० ग ५~। जाता यथाक्रमं सङ्कलितवर्गघनाः १५।२५।१२५~। 
\vspace{2mm}

सङ्कलितवर्गैक्ये, आ २ उ ३ ग ५~। सङ्कलितघनैक्ये, आ ६ उ ११ ग ५~। वर्गघनैक्ये, आ ६ उ १२ ग ५~। सङ्कलितवर्गघनैक्ये, आ ७ उ १३ ग ५~। जातानि ४०।१४०।१५०।१६५ एतत्क्रियाचमत्कृतये श्रेढीक्रमेण दर्शितम्~। \\

\noindent \textbf{सूत्रम्~।}

\phantomsection \label{3.11}
\begin{quote}
\renewcommand{\thefootnote}{१}\footnote{अत्रोपपत्तिः~। सङ्कलितं भास्करोक्तानुरूपमेव~। 
\vspace{2mm}

\hspace{2mm} कृतिसङ्कलितैक्यम् $=$ प$^{\scriptsize{\hbox{२}}}  + \dfrac{{\footnotesize{\hbox{प}}}}{{\footnotesize{\hbox{२}}}}$\,(प $+$ १) $= \dfrac{{\footnotesize{\hbox{प}}}}{{\footnotesize{\hbox{२}}}}$\,(२\,प $+$ प $+$ १)
\vspace{2mm}

\hspace{23mm} $= \dfrac{{\footnotesize{\hbox{प}}}}{{\footnotesize{\hbox{२}}}}$\,(३\,प $+$ ३ $-$ २) $= \dfrac{{\footnotesize{\hbox{३\,प}}}}{{\footnotesize{\hbox{२}}}}$\,(प $+$ १) $-$ प~।~~ अत उपपन्नम्~।}{\large \textbf{{\color{purple}सैकपदघ्नपदार्धं \\
सङ्कलितं तस्य वाय इति सञ्ज्ञा~।\\
आयस्त्रिगुणो विपदः \\
कृतिसङ्कलितैक्यतुल्यं स्यात्~॥}}}
\end{quote}

अत्र पूर्वोक्तोदाहरणे सङ्कलितवर्गयोगार्थं न्यासः~। पदम् ५~। जातं सङ्कलितम् १५~। सङ्क-लितवर्गैक्यम् ४०~। 
\end{sloppypar}

\newpage

\noindent \textbf{सूत्रम्~।}

\phantomsection \label{3.12}
\begin{quote}
\renewcommand{\thefootnote}{१}\footnote{अत्रोपपत्तिः~। पूर्वप्रकारेण सङ्कलितम् $= \dfrac{{\footnotesize{\hbox{प}}}}{{\footnotesize{\hbox{२}}}}\,({\hbox{प}} + {\hbox{१}})$
\vspace{2mm}

\hspace{2mm} ततः~~ ${\hbox{सं}} + {\hbox{प}}^{\scriptsize{\hbox{३}}} = \dfrac{{\footnotesize{\hbox{प}}}}{{\footnotesize{\hbox{२}}}}\,({\hbox{प}} + {\hbox{१}}) + {\hbox{प}}^{\scriptsize{\hbox{३}}} = \dfrac{{\footnotesize{\hbox{९\,प}}^{\scriptsize{\hbox{३}}}} + \dfrac{{\footnotesize{\hbox{९\,प}}^{\scriptsize{\hbox{२}}}}}{{\footnotesize{\hbox{२}}}} + \dfrac{{\footnotesize{\hbox{९\,प}}}}{{\footnotesize{\hbox{२}}}}}{{\footnotesize{\hbox{९}}}}$
\vspace{2mm}

\hspace{20mm} $= \dfrac{{\footnotesize{\hbox{३\,प}}^{\scriptsize{\hbox{३}}} + {\hbox{६\,प}}^{\scriptsize{\hbox{३}}} + {\hbox{३\,प}}^{\scriptsize{\hbox{२}}}} + \dfrac{{\footnotesize{\hbox{३\,प}}^{\scriptsize{\hbox{२}}}}}{{\footnotesize{\hbox{२}}}} + \dfrac{{\footnotesize{\hbox{प}}}}{{\footnotesize{\hbox{२}}}} + {\footnotesize{\hbox{४\,प}}}}{{\footnotesize{\hbox{९}}}}$
\vspace{3mm}

\hspace{20mm} $= \dfrac{{\footnotesize{\hbox{३\,प}}^{\scriptsize{\hbox{३}}} + {\hbox{३\,प}}^{\scriptsize{\hbox{२}}} + {\hbox{६\,प}}^{\scriptsize{\hbox{३}}}} + \dfrac{{\footnotesize{\hbox{प}}^{\scriptsize{\hbox{२}}}}}{{\footnotesize{\hbox{२}}}} + \dfrac{{\footnotesize{\hbox{प}}}}{{\footnotesize{\hbox{२}}}} + {\footnotesize{\hbox{प}}^{\scriptsize{\hbox{२}}} + {\hbox{४\,प}}}}{{\footnotesize{\hbox{९}}}}$
\vspace{3mm}

\hspace{20mm} $= \dfrac{{\footnotesize{\hbox{६\,प}}^{\scriptsize{\hbox{२}}}}\,\left(\dfrac{{\footnotesize{\hbox{प}} + {\hbox{१}}}}{{\footnotesize{\hbox{२}}}}\right) + {\footnotesize{\hbox{६\,प}}^{\scriptsize{\hbox{३}}} + {\hbox{प}}}\,\left(\dfrac{{\footnotesize{\hbox{प}} + {\hbox{१}}}}{{\footnotesize{\hbox{२}}}}\right) + {\footnotesize{\hbox{प}}^{\scriptsize{\hbox{२}}} + {\hbox{४\,प}}}}{{\footnotesize{\hbox{९}}}}$
\vspace{3mm}

\hspace{20mm} $= \dfrac{{\footnotesize{\hbox{६\,प.सं}} + {\hbox{६\,प}}^{\scriptsize{\hbox{३}}} + {\hbox{सं}} + {\hbox{प}}^{\scriptsize{\hbox{२}}} + {\hbox{४\,प}}}}{{\footnotesize{\hbox{९}}}} = \dfrac{{\footnotesize{\hbox{६\,प}}\,({\hbox{सं}} + {\hbox{प}}^{\scriptsize{\hbox{२}}}) + ({\hbox{सं}} + {\hbox{प}}^{\scriptsize{\hbox{२}}}) + {\hbox{४\,प}}}}{{\footnotesize{\hbox{९}}}}$
\vspace{3mm}

\hspace{20mm} $= \dfrac{{\footnotesize({\hbox{६\,प}} + {\hbox{१}})\,({\hbox{सं}} + {\hbox{प}}^{\scriptsize{\hbox{२}}}) + {\hbox{४\,प}}}}{{\footnotesize{\hbox{९}}}}$~।~~ इत्युपपद्यते~।}{\large \textbf{{\color{purple}सङ्कलितवर्गयोगं \\
षड्गुणपदसैकताडितं कृत्वा~।\\
चतुराहतपदयुक्तं \\
नवहृत् सङ्कलितघनयोगः~॥}}}
\end{quote}

\newpage

सङ्कलितघनैक्यार्थं न्यासः~। पदम् ५~। जातं सङ्कलितघनैक्यम् १४०~। \\

\noindent \textbf{सूत्रम्~।}

\phantomsection \label{3.13.1}
\begin{quote}
\renewcommand{\thefootnote}{१}\footnote{अत्रोपपत्तिः~। सङ्कलितम् $= \dfrac{{\footnotesize{\hbox{प}}\,({\hbox{प}} + {\hbox{१}})}}{{\footnotesize{\hbox{२}}}}$~। अत्र पदघनवर्गयोगे जाता सङ्ख्या 
\vspace{2mm}

\hspace{6mm} $= \dfrac{{\footnotesize{\hbox{प}}\,({\hbox{प}} + {\hbox{१}}) + {\hbox{२\,प}}^{\scriptsize{\hbox{२}}} + {\hbox{२\,प}}^{\scriptsize{\hbox{३}}}}}{{\footnotesize{\hbox{२}}}}$
\vspace{2mm}

\hspace{6mm} $= \dfrac{{\footnotesize{\hbox{प}}\,({\hbox{प}} + {\hbox{१}}) + {\hbox{२\,प}}\,({\hbox{प}} + {\hbox{१}})}}{{\footnotesize{\hbox{२}}}} = \dfrac{{\footnotesize{\hbox{प}}\,({\hbox{प}} + {\hbox{१}})}}{{\footnotesize{\hbox{२}}}}\, ({\hbox{२\,प}} + {\hbox{१}})$
\vspace{2mm}

\hspace{6mm} $=$ सं\,(२\,प $+$ १)~।~~ अत उपपन्नम्~।
\vspace{2mm}
}{\large \textbf{{\color{purple}द्विगुणितपदं सरूपं \\
सङ्कलितघ्नं \renewcommand{\thefootnote}{$\star$}\footnote{The reading घनाद्य- seems to be a typographical error.
\vspace{1mm}
}घनायकृतियोगः~।}}}
\end{quote}

सङ्कलितवर्गघनैक्यार्थं न्यासः~। पदम् ५~। जातमाद्यकृतिघनैक्यम् १६५~।\\

\noindent \textbf{सूत्रम्~।}

\phantomsection \label{3.13}
\begin{quote}
\renewcommand{\thefootnote}{२}\footnote{अत्रोपपत्तिः~। पूर्वार्धस्य {\color{violet}"सा द्वियुतेन पदेन विनिघ्नी स्यात् त्रिहता खलु सङ्कलितैक्यम्"} इति {\color{violet}भास्करो}क्तेन स्फुटा~। 
\vspace{1mm}

\hspace{2mm} पूर्वसूत्रेण सङ्कलितकृतिघनैक्यम् $=$ सं\,(२\,प $+$ १) $ = \dfrac{{\footnotesize{\hbox{प}}\,({\hbox{प}} + {\hbox{१}})}}{{\footnotesize{\hbox{२}}}}\, ({\hbox{२\,प}} + {\hbox{१}})$
\vspace{2mm}

\hspace{2mm} इदं त्रिहृतं फलम् $ = \dfrac{{\footnotesize{\hbox{प}}\,({\hbox{प}} + {\hbox{१}})}}{{\footnotesize{\hbox{२}}}} \times \dfrac{{\footnotesize({\hbox{२\,प}} + {\hbox{१}})}}{{\footnotesize{\hbox{३}}}} = \dfrac{{\footnotesize{\hbox{सं}}\,({\hbox{२\,प}} + {\hbox{१}})}}{{\footnotesize{\hbox{३}}}}$
\vspace{2mm}

\hspace{2mm} इदं {\color{violet}'द्विघ्नपदं कुयुतं त्रिविभक्तं सङ्कलितेन हतं कृतियोगः'} इति {\color{violet}भास्करो}क्तेन रूपादिवर्गैक्यम्~। 
\vspace{1mm}

\hspace{2mm} घनसमासाख्योपपत्तिश्च {\color{violet}'सङ्कलितस्य कृतेः सममेकाद्यङ्कघनैक्यमुदीरितमाद्यैः'} इति {\color{violet}भास्करो}क्तेन स्फुटा~।}{\large \textbf{{\color{purple}द्विकयुतपदगुणिताये \\
त्रिहृते रूपादिकायसंयोगः~॥}}}
\end{quote}

\newpage
\begin{sloppypar}

\phantomsection \label{3.14}
\begin{quote}
{\large \textbf{{\color{purple}सङ्कलितकृतिघनैक्यं \\
त्रिहृतं रूपादिवर्गयुतिः~।\\
सङ्कलितस्य च वर्गो \\
रूपादेर्घनसमासः स्यात्~॥}}}
\end{quote}

\noindent \textbf{उदाहरणम्~।}

\phantomsection \label{Ex 3.12}
\begin{quote}
\textbf{{\color{red}रूपादिपञ्चपर्यन्तमायैक्यं वद कोविद~।\\
वर्गैक्यं च घनैक्यं च श्रेढीमार्गे क्षमोऽसि चेत्~॥}}
\end{quote}

न्यासः १।२।३।४।५ जातं सङ्कलितैक्यम् १।४।१०।२०।३५~। जातानि वर्गैक्यानि १।५। १४।३०।५५~। घनैक्यानि १।९।३६।१००।२२५~। \\

\noindent \textbf{सूत्रम्~।}

\phantomsection \label{3.15}
\begin{quote}
\renewcommand{\thefootnote}{१}\footnote{अत्रोपपत्तिः~। यदि मुखम् $=$ मु~। चयः $=$ च~। गच्छः $=$ ग~। 
\vspace{1mm}

\hspace{2mm} अथ येषां युतिरपेक्षिता तेषामन्तिमसङ्ख्या यदि सं$_{\scriptsize{\hbox{ग}}}$ तदा प्रश्नानुसारेण~। 
\vspace{2mm}

\hspace{6mm} सं$_{\scriptsize{\hbox{ग}}} = \dfrac{{\footnotesize[{\hbox{च}}\,({\hbox{ग}} - {\hbox{१}}) + {\hbox{मु}} + {\hbox{१}}]\,[{\hbox{च}}\,({\hbox{ग}} - {\hbox{१}}) + {\hbox{मु}}]}}{{\footnotesize{\hbox{२}}}}$
\vspace{2mm}

\hspace{11mm} $= \dfrac{{\footnotesize[{\hbox{च}}\,({\hbox{ग}} - {\hbox{१}}) + {\hbox{मु}}]^{\scriptsize{\hbox{२}}} + [{\hbox{च}}\,({\hbox{ग}} - {\hbox{१}}) + {\hbox{मु}}]}}{{\footnotesize{\hbox{२}}}}$}{\large \textbf{{\color{purple}समुखचयायमुखाया-\\
न्तरमेकोनितपदायसङ्गुणितम्~॥\\
मुखसङ्कलितपदाहति-\\
युक्तमथो द्व्यूनगच्छस्य~।\\
आयैक्येनोत्तरकृति-\\
गुणितेन युगायसंयोगः~॥}}}
\end{quote}
\end{sloppypar}

\newpage

\noindent \textbf{उदाहरणम्~।}\renewcommand{\thefootnote}{}\footnote{\hspace{4mm} $= \dfrac{{\footnotesize{\hbox{च}}^{\scriptsize{\hbox{२}}}({\hbox{ग}} - {\hbox{१}})^{\scriptsize{\hbox{२}}}}}{{\footnotesize{\hbox{२}}}} + {\hbox{चमु}}\,({\hbox{ग}} - {\hbox{१}}) + \dfrac{{\footnotesize{\hbox{मु}}^{\scriptsize{\hbox{२}}}}}{{\footnotesize{\hbox{२}}}} + \dfrac{{\footnotesize{\hbox{मु}}}}{{\footnotesize{\hbox{२}}}} + {\hbox{च}}\,({\hbox{ग}} - {\hbox{१}})$
\vspace{2mm}

\hspace{2mm} अत्र ग स्थाने ग $-$ १, ग $-$ २, इत्याद्युत्थापनेन सं $_{{\scriptsize{\hbox{ग}} - {\hbox{१}}}}$, सं$_{{\scriptsize{\hbox{ग}} - {\hbox{२}}}}$,..., सं$_{\scriptsize{\hbox{१}}}$ इत्यादि मानमानीय तासां 
\vspace{2mm}

\hspace{2mm} योगः $= \dfrac{{\footnotesize{\hbox{च}}^{\scriptsize{\hbox{२}}}}}{{\footnotesize{\hbox{२}}}}\,[({\hbox{ग}} - {\hbox{१}})^{\scriptsize{\hbox{२}}} + ({\hbox{ग}} - {\hbox{२}})^{\scriptsize{\hbox{२}}} + ... + {\hbox{१}}^{\scriptsize{\hbox{२}}}] + {\hbox{चमु}}\,[({\hbox{ग}} - {\hbox{१}}) + ({\hbox{ग}} - {\hbox{२}}) + ... + {\hbox{२}} + {\hbox{१}}]$
\vspace{2mm}

\hspace{16mm} $+ \dfrac{{\footnotesize{\hbox{च.ग}}\,({\hbox{ग}} - {\hbox{१}})}}{{\footnotesize{\hbox{२}}}} + {\hbox{ग.}}\dfrac{{\footnotesize{\hbox{मु}}}}{{\footnotesize{\hbox{२}}}}\,({\hbox{मु}} + {\hbox{१}})$
\vspace{2mm}

\hspace{9mm} $= \dfrac{{\footnotesize{\hbox{च}}^{\scriptsize{\hbox{२}}}}}{{\footnotesize{\hbox{१.२.३}}}}\,{\hbox{ग}}\,({\hbox{ग}} - {\hbox{१}})\,({\hbox{२\,ग}} - {\hbox{१}}) + \dfrac{{\footnotesize{\hbox{चमु.ग}}\,({\hbox{ग}} - {\hbox{१}})}}{{\footnotesize{\hbox{२}}}} + {\hbox{ग.}}\dfrac{{\footnotesize{\hbox{मु}}}}{{\footnotesize{\hbox{२}}}}\,({\hbox{मु}} + {\hbox{१}}) + \dfrac{{\footnotesize{\hbox{चग}}\,({\hbox{ग}} - {\hbox{१}})}}{{\footnotesize{\hbox{२}}}}$
\vspace{2mm}

\hspace{9mm} $= \dfrac{{\footnotesize{\hbox{ग}}\,({\hbox{ग}} - {\hbox{१}})}}{{\footnotesize{\hbox{२}}}}\,\left[\dfrac{{\footnotesize{\hbox{च}}^{\scriptsize{\hbox{२}}}}}{{\footnotesize{\hbox{२.३}}}}\,({\hbox{२\,ग}} - {\hbox{१}}) + {\hbox{चमु}} + \dfrac{{\footnotesize{\hbox{च}}}}{{\footnotesize{\hbox{२}}}}\right] + {\hbox{ग.}}\dfrac{{\footnotesize{\hbox{मु}}}}{{\footnotesize{\hbox{२}}}}\,({\hbox{मु}} + {\hbox{१}})$
\vspace{2mm}

\hspace{9mm} $= \dfrac{{\footnotesize{\hbox{ग}}\,({\hbox{ग}} - {\hbox{१}})}}{{\footnotesize{\hbox{२}}}}\,\left[\dfrac{{\footnotesize{\hbox{२\,च}}^{\scriptsize{\hbox{२}}}{\hbox{ग}}}}{{\footnotesize{\hbox{२.३}}}} - \dfrac{{\footnotesize{\hbox{च}}^{\scriptsize{\hbox{२}}}}}{{\footnotesize{\hbox{२.३}}}} + {\hbox{चमु}} + \dfrac{{\footnotesize{\hbox{च}}}}{{\footnotesize{\hbox{२}}}}\right] + {\hbox{ग.}}\dfrac{{\footnotesize{\hbox{मु}}}}{{\footnotesize{\hbox{२}}}}\,({\hbox{मु}} + {\hbox{१}})$
\vspace{2mm}

\hspace{9mm} $= \dfrac{{\footnotesize{\hbox{ग}}\,({\hbox{ग}} - {\hbox{१}})}}{{\footnotesize{\hbox{२}}}}\,\left[\dfrac{{\footnotesize{\hbox{च}}^{\scriptsize{\hbox{२}}}{\hbox{ग}}}}{{\footnotesize{\hbox{३}}}} - \dfrac{{\footnotesize{\hbox{च}}^{\scriptsize{\hbox{२}}}}}{{\footnotesize{\hbox{६}}}} + {\hbox{चमु}} + \dfrac{{\footnotesize{\hbox{च}}}}{{\footnotesize{\hbox{२}}}}\right] + {\hbox{ग.}}\dfrac{{\footnotesize{\hbox{मु}}}}{{\footnotesize{\hbox{२}}}}\,({\hbox{मु}} + {\hbox{१}})$
\vspace{2mm}

\hspace{9mm} $= \dfrac{{\footnotesize{\hbox{ग}}\,({\hbox{ग}} - {\hbox{१}})}}{{\footnotesize{\hbox{२}}}}\,\left[\dfrac{{\footnotesize{\hbox{च}}^{\scriptsize{\hbox{२}}}{\hbox{ग}} - {\hbox{२\,च}}^{\scriptsize{\hbox{२}}}}}{{\footnotesize{\hbox{३}}}} + \dfrac{{\footnotesize{\hbox{२\,च}}^{\scriptsize{\hbox{२}}}}}{{\footnotesize{\hbox{३}}}} - \dfrac{{\footnotesize{\hbox{च}}^{\scriptsize{\hbox{२}}}}}{{\footnotesize{\hbox{६}}}} + {\hbox{चमु}} + \dfrac{{\footnotesize{\hbox{च}}}}{{\footnotesize{\hbox{२}}}}\right] + {\hbox{ग.}}\dfrac{{\footnotesize{\hbox{मु}}}}{{\footnotesize{\hbox{२}}}}\,({\hbox{मु}} + {\hbox{१}})$}

\phantomsection \label{Ex 3.13}
\begin{quote}
\textbf{{\color{red}त्र्यादिचतुरुत्तराणां सङ्कलितैक्यं पदेषु नवसु सखे~।\\
वर्गैक्यं च घनैक्यं वद यदि गणितेऽस्ति ते पटुता~॥}}
\end{quote}

\newpage

\begin{sloppypar}
न्यासः~। आ ३ उ ४ ग ९~। अत्र करणम्~। समुखचयः ७ मुखम् ३ अनयोरायाविति सङ्कलिते २८।६ अनयोरन्तरम् २२~। व्येकपदसङ्कलितेन ३६ हतम् ७९२~। मुखायः ६ गच्छः ९ अनयोराहत्या ५४ युतम् ८४६~। द्व्यूनगच्छस्य ७ सङ्कलितैक्येन ८४ वृद्धि\textendash \,४\textendash \,वर्ग\textendash \,१६\textendash \,गुणितेन १३४४ तद्\textendash \,८४६\textendash \,युतं जातं सङ्कलितम् २१९०~।\renewcommand{\thefootnote}{}\footnote{\hspace{-3mm} $= \dfrac{{\footnotesize{\hbox{ग}}\,({\hbox{ग}} - {\hbox{१}})\,({\hbox{ग}} - {\hbox{२}})}}{{\footnotesize{\hbox{१.२.३}}}}\,{\hbox{च}}^{\scriptsize{\hbox{२}}} + \dfrac{{\footnotesize{\hbox{ग}}\,({\hbox{ग}} - {\hbox{१}})}}{{\footnotesize{\hbox{२}}}}\left[\dfrac{{\footnotesize{\hbox{३\,च}}^{\scriptsize{\hbox{२}}}}}{{\footnotesize{\hbox{६}}}} + {\hbox{चमु}} + \dfrac{{\footnotesize{\hbox{च}}}}{{\footnotesize{\hbox{२}}}}\right] + {\hbox{ग.}}\dfrac{{\footnotesize{\hbox{मु}}}}{{\footnotesize{\hbox{२}}}}\,({\hbox{मु}} + {\hbox{१}})$
\vspace{2mm}

$= \dfrac{{\footnotesize{\hbox{ग}}\,({\hbox{ग}} - {\hbox{१}})\,({\hbox{ग}} - {\hbox{२}})}}{{\footnotesize{\hbox{१.२.३}}}}\,{\hbox{च}}^{\scriptsize{\hbox{२}}} + \dfrac{{\footnotesize{\hbox{ग}}\,({\hbox{ग}} - {\hbox{१}})}}{{\footnotesize{\hbox{२}}}}\left[\dfrac{{\footnotesize{\hbox{च}}^{\scriptsize{\hbox{२}}} + {\hbox{२\,चमु}} + {\hbox{च}}}}{{\footnotesize{\hbox{२}}}}\right] + {\hbox{ग.}}\dfrac{{\footnotesize{\hbox{मु}}}}{{\footnotesize{\hbox{२}}}}\,({\hbox{मु}} + {\hbox{१}})$
\vspace{2mm}

$= \dfrac{{\footnotesize{\hbox{ग}}\,({\hbox{ग}} - {\hbox{१}})\,({\hbox{ग}} - {\hbox{२}})}}{{\footnotesize{\hbox{१.२.३}}}}\,{\hbox{च}}^{\scriptsize{\hbox{२}}} + \dfrac{{\footnotesize{\hbox{ग}}\,({\hbox{ग}} - {\hbox{१}})}}{{\footnotesize{\hbox{२}}}}\left[\dfrac{{\footnotesize({\hbox{च}} + {\hbox{मु}})^{\scriptsize{\hbox{२}}} + {\hbox{च}} + {\hbox{मु}}}}{{\footnotesize{\hbox{२}}}} - \dfrac{{\footnotesize({\hbox{मु}} + {\hbox{मु}}^{\scriptsize{\hbox{२}}})}}{{\footnotesize{\hbox{२}}}}\right] + {\hbox{ग.}}\dfrac{{\footnotesize{\hbox{मु}}}}{{\footnotesize{\hbox{२}}}}\,({\hbox{मु}} + {\hbox{१}})$
\vspace{2mm}

$= \dfrac{{\footnotesize{\hbox{ग}}\,({\hbox{ग}} - {\hbox{१}})\,({\hbox{ग}} - {\hbox{२}})}}{{\footnotesize{\hbox{१.२.३}}}}\,{\hbox{च}}^{\scriptsize{\hbox{२}}} + \dfrac{{\footnotesize{\hbox{ग}}\,({\hbox{ग}} - {\hbox{१}})}}{{\footnotesize{\hbox{२}}}}\left[\dfrac{{\footnotesize({\hbox{च}} + {\hbox{मु}})\,({\hbox{च}} + {\hbox{मु}} + {\hbox{१}})}}{{\footnotesize{\hbox{२}}}} - \dfrac{{\footnotesize{\hbox{मु}}}}{{\footnotesize{\hbox{२}}}}\,({\hbox{मु}} + {\hbox{१}})\right] + {\hbox{ग.}}\dfrac{{\footnotesize{\hbox{मु}}}}{{\footnotesize{\hbox{२}}}}\,({\hbox{मु}} + {\hbox{१}})$
\vspace{1mm}

अत उपपन्नं सूत्रोक्तम्~।
\vspace{2mm}
} \\

\noindent \textbf{सूत्रम्~।}

\phantomsection \label{3.17}
\begin{quote}
\renewcommand{\thefootnote}{१}\footnote{अत्रोपपत्तिः~। अत्रापि पूर्वसूत्रोपपत्तिवद्येषां युतिरपेक्षिता तेषामन्तिमसङ्ख्या
\vspace{1mm}

\hspace{6mm} $= {\hbox{सं}}_{\scriptsize{\hbox{ग}}} =$ [मु $+$ च\,(ग $-$ १)]$^{\scriptsize{\hbox{२}}}$}{\large \textbf{{\color{purple}द्विगुणचयोत्थे गणिते \\
मुखगुणिते विगतरूपगच्छस्य~॥~१७~॥}}}
\end{quote}
\end{sloppypar}

\newpage
\begin{sloppypar}

\phantomsection \label{3.18.1}
\begin{quote}
{\large \textbf{{\color{purple}वर्गैक्येनोत्तरकृति-\\
गुणितेन युते तु वर्गयुतिः~।}}}\renewcommand{\thefootnote}{}\footnote{\hspace{2mm} $= {\hbox{मु}}^{\scriptsize{\hbox{२}}} + {\hbox{२\,मुच}}\,({\hbox{ग}} - {\hbox{१}}) + {\hbox{च}}^{\scriptsize{\hbox{२}}}({\hbox{ग}} - {\hbox{१}})^{\scriptsize{\hbox{२}}}$
\vspace{2mm}

\hspace{2mm} ग स्थाने ग $-$ १, ग $-$ २, इत्यादि समुत्थाप्य सं $_{{\scriptsize{\hbox{ग}} - {\hbox{१}}}}$, सं$_{{\scriptsize{\hbox{ग}} - {\hbox{२}}}}$ इत्यादि मानं विज्ञाय तेषां योगः 
\vspace{2mm}

\hspace{4mm} $= {\hbox{गमु}}^{\scriptsize{\hbox{२}}} + {\hbox{२\,मुच}}\,[({\hbox{ग}} - {\hbox{१}}) + ({\hbox{ग}} - {\hbox{२}}) + ... + {\hbox{२}} + {\hbox{१}}] + {\hbox{च}}^{\scriptsize{\hbox{२}}}[({\hbox{ग}} - {\hbox{१}})^{\scriptsize{\hbox{२}}} + ({\hbox{ग}} - {\hbox{२}})^{\scriptsize{\hbox{२}}} + ... ]$
\vspace{2mm}

\hspace{4mm} $= {\hbox{गमु}}^{\scriptsize{\hbox{२}}} + {\hbox{मु.च.ग}}\,({\hbox{ग}} - {\hbox{१}}) + {\hbox{च}}^{\scriptsize{\hbox{२}}}[({\hbox{ग}} - {\hbox{१}})^{\scriptsize{\hbox{२}}} + ({\hbox{ग}} - {\hbox{२}})^{\scriptsize{\hbox{२}}} + ... + {\hbox{२}}^{\scriptsize{\hbox{२}}} + {\hbox{१}}^{\scriptsize{\hbox{२}}}]$
\vspace{2mm}

\hspace{4mm} $= \dfrac{{\footnotesize{\hbox{ग}}}}{{\footnotesize{\hbox{२}}}}\,\{{\hbox{मु}}\,[{\hbox{२\,मु}} + {\hbox{२\,च}}\,({\hbox{ग}} - {\hbox{१}})]\} + {\hbox{च}}^{\scriptsize{\hbox{२}}}[({\hbox{ग}} - {\hbox{१}})^{\scriptsize{\hbox{२}}} + ({\hbox{ग}} - {\hbox{२}})^{\scriptsize{\hbox{२}}} + ... + {\hbox{२}}^{\scriptsize{\hbox{२}}} + {\hbox{१}}^{\scriptsize{\hbox{२}}}]$
\vspace{2mm}

\hspace{4mm} $= {\hbox{मु}}\,\left(\dfrac{{\footnotesize{\hbox{ग}}}}{{\footnotesize{\hbox{२}}}}\,[{\hbox{२\,मु}} + {\hbox{२\,च}}\,({\hbox{ग}} - {\hbox{१}})]\right)$}
\end{quote}

वर्गैक्यार्थं पूर्वोक्तोदाहरणे न्यासः~। आ ३ उ ४ ग ९~। अत्र करणम्~। द्विगुणचयः, आ ३ च ८ ग ९~। गणितं ३१५ मुख\textendash \,३\textendash \,गुणितं ९४५ व्येकपदस्यास्य ८ वर्गैक्यं २०४ चय\textendash \,४\textendash \,वर्ग\textendash \,१६\textendash \,गुणितं ३२६४ युक्तं जातं श्रेढीवर्गैक्यम् ४२०९~।
\end{sloppypar}

\newpage

\noindent \textbf{सूत्रं मङ्गलगीतिः~।}\renewcommand{\thefootnote}{}\footnote{\hspace{2mm} $+ {\hbox{च}}^{\scriptsize{\hbox{२}}}[({\hbox{ग}} - {\hbox{१}})^{\scriptsize{\hbox{२}}} + ({\hbox{ग}} - {\hbox{२}})^{\scriptsize{\hbox{२}}} + ... + {\hbox{२}}^{\scriptsize{\hbox{२}}} + {\hbox{१}}^{\scriptsize{\hbox{२}}}]$
\vspace{2mm}

\hspace{2mm} अत उपपद्यते सर्वम्~।
\vspace{2mm}
}

\phantomsection \label{3.18}
\begin{quote}
\renewcommand{\thefootnote}{१}\footnote{अत्रोपपत्तिः~। अत्रापि पूर्ववद्येषां युतिरपेक्षिता तदन्तिमसङ्ख्या $= {\hbox{सं}}_{\scriptsize{\hbox{ग}}}$
\vspace{2mm}

\hspace{4mm} $= [{\hbox{मु}} + {\hbox{च}}\,({\hbox{ग}} - {\hbox{१}})]^{\scriptsize{\hbox{३}}}$
\vspace{2mm}

\hspace{4mm} $= {\hbox{मु}}^{\scriptsize{\hbox{३}}} + {\hbox{३\,मु}}^{\scriptsize{\hbox{२}}}{\hbox{च}}\,({\hbox{ग}} - {\hbox{१}}) + {\hbox{३\,मु\,च}}^{\scriptsize{\hbox{२}}}\,({\hbox{ग}} - {\hbox{१}})^{\scriptsize{\hbox{२}}} + {\hbox{च}}^{\scriptsize{\hbox{३}}}({\hbox{ग}} - {\hbox{१}})^{\scriptsize{\hbox{३}}}$
\vspace{2mm}

\hspace{2mm} अत्र ग स्थाने ग $-$ १, ग $-$ २, इत्यादि समुत्थाप्य सं $_{{\scriptsize{\hbox{ग}} - {\hbox{१}}}}$, सं$_{{\scriptsize{\hbox{ग}} - {\hbox{२}}}}$ इत्यादि मानं विज्ञाय तद्युतिः 
\vspace{2mm}

\hspace{4mm} $= {\hbox{गमु}}^{\scriptsize{\hbox{३}}} + {\hbox{३\,मु}}^{\scriptsize{\hbox{२}}}{\hbox{च}}\,[({\hbox{ग}} - {\hbox{१}}) + ({\hbox{ग}} - {\hbox{२}}) + ... + {\hbox{२}} + {\hbox{१}}] + {\hbox{३\,मु\,च}}^{\scriptsize{\hbox{२}}}\,[({\hbox{ग}} - {\hbox{१}})^{\scriptsize{\hbox{२}}} + ({\hbox{ग}} - {\hbox{२}})^{\scriptsize{\hbox{२}}} + ... ]$
\vspace{2mm}

\hspace{10mm} $+ {\hbox{च}}^{\scriptsize{\hbox{३}}}\,[({\hbox{ग}} - {\hbox{१}})^{\scriptsize{\hbox{३}}} + ({\hbox{ग}} - {\hbox{२}})^{\scriptsize{\hbox{३}}} + ... + {\hbox{२}}^{\scriptsize{\hbox{३}}} + {\hbox{१}}^{\scriptsize{\hbox{३}}}]$
\vspace{2mm}

\hspace{4mm} $= {\hbox{गमु}}^{\scriptsize{\hbox{३}}} + \dfrac{{\footnotesize{\hbox{३}}}}{{\footnotesize{\hbox{२}}}}\,{\hbox{मु}}^{\scriptsize{\hbox{२}}}{\hbox{च.ग}}\,({\hbox{ग}} - {\hbox{१}}) + \dfrac{{\footnotesize{\hbox{मु\,च}}^{\scriptsize{\hbox{२}}}}}{{\footnotesize{\hbox{२}}}}{\hbox{ग}}\,({\hbox{ग}} - {\hbox{१}})\,({\hbox{२\,ग}} - {\hbox{१}}) + {\hbox{च}}^{\scriptsize{\hbox{३}}}\,\left(\dfrac{{\footnotesize{\hbox{ग}}\,({\hbox{ग}} - {\hbox{१}})}}{{\footnotesize{\hbox{२}}}}\right)^{\scriptsize{\hbox{२}}}$
\vspace{2mm}

\hspace{4mm} $= {\hbox{गमु}}^{\scriptsize{\hbox{३}}} + \dfrac{{\footnotesize{\hbox{३}}}}{{\footnotesize{\hbox{२}}}}\,{\hbox{मु}}^{\scriptsize{\hbox{२}}}{\hbox{च.ग}}\,({\hbox{ग}} - {\hbox{१}}) + \dfrac{{\footnotesize{\hbox{च.ग}}^{\scriptsize{\hbox{२}}}}}{{\footnotesize{\hbox{४}}}}\,\left[{\hbox{२\,मु.च}}\,({\hbox{ग}} - {\hbox{१}})\left({\hbox{२}} - \dfrac{{\footnotesize{\hbox{१}}}}{{\footnotesize{\hbox{ग}}}}\right) + {\hbox{च}}^{\scriptsize{\hbox{२}}}({\hbox{ग}} - {\hbox{१}})^{\scriptsize{\hbox{२}}}\right]$
\vspace{1mm}
}{\large \textbf{{\color{purple}फलवर्गप्रचयगुणो \\
मुखचयविवरहतादिगुणितफलेन\renewcommand{\thefootnote}{$\star$}\footnote{The reading विवराहतादिह फलेन seems to be a typographical error.}॥}}}
\end{quote}

\newpage

\phantomsection \label{3.19.1}
\begin{quote}
{\large \textbf{{\color{purple}हीनो युक्तः प्रचयात् \\
अल्पेऽनल्पे मुखे तु च घनैक्यम्~॥}}}\renewcommand{\thefootnote}{}\footnote{$= {\hbox{गमु}}^{\scriptsize{\hbox{३}}} + \dfrac{{\footnotesize{\hbox{३}}}}{{\footnotesize{\hbox{२}}}}\,{\hbox{मु}}^{\scriptsize{\hbox{२}}}{\hbox{च.ग}}\,({\hbox{ग}} - {\hbox{१}})$
\vspace{2mm}

\hspace{8mm} $+ \dfrac{{\footnotesize{\hbox{च.ग}}^{\scriptsize{\hbox{२}}}}}{{\footnotesize{\hbox{४}}}}\,\left[{\hbox{४\,मु}}^{\scriptsize{\hbox{२}}} + {\hbox{४\,मु.च}}\,({\hbox{ग}} - {\hbox{१}}) + {\hbox{च}}^{\scriptsize{\hbox{२}}}({\hbox{ग}} - {\hbox{१}})^{\scriptsize{\hbox{२}}} - \dfrac{{\footnotesize{\hbox{२\,मु.च}}\,({\hbox{ग}} - {\hbox{१}})}}{{\footnotesize{\hbox{ग}}}} - {\hbox{४\,मु}}^{\scriptsize{\hbox{२}}}\right]$
\vspace{2mm}

\hspace{2mm} $= {\hbox{गमु}}^{\scriptsize{\hbox{३}}} + \dfrac{{\footnotesize{\hbox{३}}}}{{\footnotesize{\hbox{२}}}}\,{\hbox{मु}}^{\scriptsize{\hbox{२}}}{\hbox{च.ग}}\,({\hbox{ग}} - {\hbox{१}})$
\vspace{2mm}

\hspace{8mm} $+ {\hbox{च}}\,\left( \left[\dfrac{{\footnotesize{\hbox{ग}}}}{{\footnotesize{\hbox{२}}}}\,\{{\hbox{२\,मु}} + {\hbox{च}}\,({\hbox{ग}} - {\hbox{१}})\}\right]^{\scriptsize{\hbox{२}}} - \left[\dfrac{{\footnotesize{\hbox{मु.च}}^{\scriptsize{\hbox{२}}}{\hbox{ग}}\,({\hbox{ग}} - {\hbox{१}})}}{{\footnotesize{\hbox{२}}}} + {\hbox{ग}}^{\scriptsize{\hbox{२}}}{\hbox{चमु}}^{\scriptsize{\hbox{२}}}\right]\right)$
\vspace{2mm}

\hspace{2mm} $= {\hbox{मु}}\,\left({\hbox{गमु}}^{\scriptsize{\hbox{२}}} + {\hbox{मु.च.ग}}\,({\hbox{ग}} - {\hbox{१}}) + \dfrac{{\footnotesize{\hbox{मु.च.ग}}\,({\hbox{ग}} - {\hbox{१}})}}{{\footnotesize{\hbox{२}}}} - \dfrac{{\footnotesize{\hbox{च}}^{\scriptsize{\hbox{२}}}{\hbox{ग}}\,({\hbox{ग}} - {\hbox{१}})}}{{\footnotesize{\hbox{२}}}} - {\hbox{च.ग}}^{\scriptsize{\hbox{२}}}{\hbox{मु}}\right) + {\hbox{च.फ}}^{\scriptsize{\hbox{२}}}$
\vspace{2mm}

\hspace{2mm} $= {\hbox{मु}}\,\left({\hbox{गमु}}^{\scriptsize{\hbox{२}}} + {\hbox{मु.च.ग}}\,({\hbox{ग}} - {\hbox{१}} - {\hbox{ग}}) + \dfrac{{\footnotesize{\hbox{च.ग}}\,({\hbox{ग}} - {\hbox{१}})}}{{\footnotesize{\hbox{२}}}}\,({\hbox{मु}} - {\hbox{च}})\right) + {\hbox{च.फ}}^{\scriptsize{\hbox{२}}}$
\vspace{2mm}

\hspace{2mm} $= {\hbox{मु}}\,\left({\hbox{गमु}}^{\scriptsize{\hbox{२}}} - {\hbox{मु.च.ग}} + \dfrac{{\footnotesize{\hbox{च.ग}}\,({\hbox{ग}} - {\hbox{१}})}}{{\footnotesize{\hbox{२}}}}\,({\hbox{मु}} - {\hbox{च}})\right) + {\hbox{च.फ}}^{\scriptsize{\hbox{२}}}$
\vspace{2mm}

\hspace{2mm} $= {\hbox{मु}}\,\left({\hbox{गमु}}\,({\hbox{मु}} - {\hbox{च}}) + \dfrac{{\footnotesize{\hbox{च.ग}}\,({\hbox{ग}} - {\hbox{१}})}}{{\footnotesize{\hbox{२}}}}\,({\hbox{मु}} - {\hbox{च}})\right) + {\hbox{च.फ}}^{\scriptsize{\hbox{२}}}$
\vspace{2mm}

\hspace{2mm} $= {\hbox{मु}}\,({\hbox{मु}} - {\hbox{च}})\left({\hbox{गमु}} + \dfrac{{\footnotesize{\hbox{च.ग}}\,({\hbox{ग}} - {\hbox{१}})}}{{\footnotesize{\hbox{२}}}}\right) + {\hbox{च.फ}}^{\scriptsize{\hbox{२}}}$}
\end{quote}

\newpage
\begin{sloppypar}

घनैक्यार्थं पूर्वोदाहरणस्य न्यासः~। आ ३ उ ४ ग ९~। अत्र करणम्~। फलम् १७१ अस्य वर्गः २९२४१ प्रचयगुणः ११६९६४~।
\vspace{2mm}

अयं मुख\textendash \,३\textendash \,चय\textendash \,४\textendash \,विवर\textendash \,१\textendash \,हृतादि\textendash \,३\textendash \,गुणितफलेना\textendash \,५१३ \textendash \,नेन प्रचयादल्पे मुखे हीनः~। अनल्पे मुखे सति युक्त एवं कृते जातं घनैक्यम् ११६४५१~।\renewcommand{\thefootnote}{}\footnote{$= {\hbox{मु}}\,({\hbox{मु}} - {\hbox{च}})\left(\dfrac{{\footnotesize{\hbox{ग}}}}{{\footnotesize{\hbox{२}}}}\,[{\hbox{२\,मु}} + {\hbox{च}}\,({\hbox{ग}} - {\hbox{१}})]\right) + {\hbox{च.फ}}^{\scriptsize{\hbox{२}}}$
\vspace{2mm}

\hspace{2mm} $= {\hbox{मु}}\,({\hbox{मु}} - {\hbox{च}})\,{\hbox{फ}} + {\hbox{च.फ}}^{\scriptsize{\hbox{२}}}$~।~~ अत्र यदि\, मु\,>\,च\, तदा प्रथमखण्डफलं धनमन्यथा ऋणमिति सर्वं निरवद्यम्~। \vspace{-2mm} } \\

\noindent \textbf{सूत्रम्~।}

\phantomsection \label{3.19}
\begin{quote}
\renewcommand{\thefootnote}{१}\footnote{पदसङ्कलितमेकवारसङ्कलितम्~। सङ्कलितैक्यं द्विवारसङ्कलितम्~। सङ्कलितैक्यैक्यं त्रिवारसङ्कलितम्~। एवमग्रेऽपि वारसङ्कलितं बोध्यम्~। 
\vspace{1mm}

\hspace{2mm} अत्रोपपत्तिः~। भास्करलीलावत्यां मज्जनकशोधितायां श्रेढीव्यवहारे द्रष्टव्ये ३२\textendash ३३ पृष्ठे तत्र तत्सूत्रानुसारेण 
\vspace{-1mm}

\hspace{4mm} एकवारसङ्कलितम् $= \dfrac{{\footnotesize{\hbox{प}}}}{{\footnotesize{\hbox{१}}}}.\dfrac{{\footnotesize{\hbox{प}} + {\hbox{१}}}}{{\footnotesize{\hbox{२}}}}$
\vspace{2mm}

\hspace{4mm} द्विवारसङ्कलितम् $= \dfrac{{\footnotesize{\hbox{प}}}}{{\footnotesize{\hbox{१}}}}.\dfrac{{\footnotesize{\hbox{प}} + {\hbox{१}}}}{{\footnotesize{\hbox{२}}}}.\dfrac{{\footnotesize{\hbox{प}} + {\hbox{२}}}}{{\footnotesize{\hbox{३}}}}$
\vspace{2mm}

\hspace{4mm} त्रिवारसङ्कलितम् $= \dfrac{{\footnotesize{\hbox{प}}}}{{\footnotesize{\hbox{१}}}}.\dfrac{{\footnotesize{\hbox{प}} + {\hbox{१}}}}{{\footnotesize{\hbox{२}}}}.\dfrac{{\footnotesize{\hbox{प}} + {\hbox{२}}}}{{\footnotesize{\hbox{३}}}}.\dfrac{{\footnotesize{\hbox{प}} + {\hbox{३}}}}{{\footnotesize{\hbox{४}}}}$
\vspace{2mm}

\hspace{4mm} एवं न-वारसङ्कलितम् $= \dfrac{{\footnotesize{\hbox{प}}}}{{\footnotesize{\hbox{१}}}}.\dfrac{{\footnotesize{\hbox{प}} + {\hbox{१}}}}{{\footnotesize{\hbox{२}}}}.\dfrac{{\footnotesize{\hbox{प}} + {\hbox{२}}}}{{\footnotesize{\hbox{३}}}}.\dfrac{{\footnotesize{\hbox{प}} + {\hbox{३}}}}{{\footnotesize{\hbox{४}}}}\, ...\, \dfrac{{\footnotesize{\hbox{प}} + {\hbox{न}}}}{{\footnotesize{\hbox{न}} + {\hbox{१}}}}$
\vspace{2mm}

\hspace{2mm} एतेनोपपद्यते सूत्रम्~।}{\large \textbf{{\color{purple}एकाधिकवारमिताः \\
पदादिरूपोत्तराः पृथक् तेंऽशाः~॥~१९~॥\\ 
एकाद्येकचयहराः \\
तद्घातो वारसङ्कलितम्~।}}}
\end{quote}
\end{sloppypar}

\newpage

\noindent \textbf{उदाहरणम्~।}

\phantomsection \label{Ex 3.14}
\begin{quote}
\textbf{{\color{red}एकादिचयेन सखे वद षण्णां\renewcommand{\thefootnote}{$\star$}\footnote{The reading षस्मां doesn’t seem to be correct.
\vspace{1mm}
} मे त्रिवारसङ्कलितम्~।\\
यदि गणितोदन्वन्तं दोर्भ्यां तर्तुं समर्थोऽसि~॥}}
\end{quote}

\begin{sloppypar}
न्यासः~। पदं ६ वाराः ३~। एकाधिकवाराः ४~। पदादिरूपोत्तरा इति पदं ६ रूपोत्तराः सैकवारमिता अंशाः ६।७।८।९~। एषामेकादिचया हारा न्यस्ता जाता\, $\dfrac{{\footnotesize{\hbox{६}}}}{{\footnotesize{\hbox{१}}}}\, \dfrac{{\footnotesize{\hbox{७}}}}{{\footnotesize{\hbox{२}}}}\, \dfrac{{\footnotesize{\hbox{८}}}}{{\footnotesize{\hbox{३}}}}\, \dfrac{{\footnotesize{\hbox{९}}}}{{\footnotesize{\hbox{४}}}}$~।\, एषां घाते जातं त्रिवारसङ्कलितम् १२६~। \\
\end{sloppypar}

\noindent \textbf{सूत्रम्~।}

\phantomsection \label{3.20}
\begin{quote}
\renewcommand{\thefootnote}{१}\footnote{${\hbox{मु}} + ({\hbox{मु}} + {\hbox{च}}) + ({\hbox{मु}} + {\hbox{२\,च}}) + [{\hbox{मु}} + {\hbox{च}}\,({\hbox{प}} - {\hbox{१}})]$\; एतेषां योग एकवारजं गणितम्~। 
\vspace{2mm}

\hspace{2mm} ${\hbox{मु}} + ({\hbox{२\,मु}} + {\hbox{च}}) + ({\hbox{३\,मु}} + {\hbox{४\,च}}) + \dfrac{{\footnotesize{\hbox{प}}}}{{\footnotesize{\hbox{२}}}}\,[{\hbox{२\,मु}} + {\hbox{च}}\,({\hbox{प}} - {\hbox{१}})]$\; एतेषां योगो द्विवारजं गणितम्~। 
\vspace{2mm}

\hspace{2mm} एतद्योगार्थमन्तिमसङ्ख्या = सं$_{\scriptsize{\hbox{प}}} =$ मु.प $+$ च\,$\dfrac{{\footnotesize{\hbox{प}}\,({\hbox{प}} - {\hbox{१}})}}{{\footnotesize{\hbox{२}}}} =$ मु.प $+$ च.सं~।\vspace{2mm}

\hspace{2mm} अत्र प स्थाने प $-$ १, प $-$ २ इत्यादि समुत्थाप्य सं $_{{\scriptsize{\hbox{प}} - {\hbox{१}}}}$, सं$_{{\scriptsize{\hbox{प}} - {\hbox{२}}}}$ मानं विज्ञाय तद्योगः
\vspace{2mm}

\hspace{6mm} $= [{\hbox{प}} + ({\hbox{प}} - {\hbox{१}}) + ({\hbox{प}} - {\hbox{२}}) + ... + {\hbox{२}} + {\hbox{१}}]$
\vspace{1mm}
}{\large \textbf{{\color{purple}रूपोनितपदवारज-\\
सङ्कलितं स्याच्च यो गुणः\renewcommand{\thefootnote}{$\dag$}\footnote{The reading ये गुणाः seems to be a typographical error as it gives incorrect meaning.} स पृथक्~॥~२०~॥\\
एकाधिकवारघ्नो \\
व्येकपदाप्तो मुखे गुणो भवति~।\\
स्वगुणघ्नाद्युत्तरयोः \\
योगः स्याद्वारजं गणितम्~॥~२१~॥}}}
\end{quote}

\newpage

\noindent \textbf{उदाहरणम्~।}\renewcommand{\thefootnote}{}\footnote{\hspace{4mm} $+$ च (एकोनपदसं $+$ द्व्यूनपदसं $+$ ... $+$ एकसं)
\vspace{2mm}

\hspace{2mm} $= {\hbox{मु}}\,\dfrac{{\footnotesize{\hbox{प}}\,({\hbox{प}} + {\hbox{१}})}}{{\footnotesize{\hbox{२}}}} +$ च
\vspace{2mm}

\hspace{2mm} एकोनपदजसङ्कलितैक्ये द्विवारजे गणितेऽयं योग एवान्तिमसङ्ख्या तत्रापि प स्थाने प $-$ १, प $-$ २, इत्यादि समुत्थाप्य सर्वाः सङ्ख्या विज्ञाय 
\vspace{2mm}

\hspace{2mm} तद्योगः $=$ मु.पदसङ्कलितैक्य $+$ च.एकोनपदसङ्कलितैक्यैक्य
\vspace{1mm}

\hspace{10mm} $=$ मु.पदजद्विवारसं $+$ च.एकोनपदजत्रिवारसङ्कलित 
\vspace{2mm}

\hspace{2mm} एवं न-वारजं गणितम्\textendash
\vspace{2mm}

\hspace{10mm} $=$ मु\,[\,पदज\,(न $-$ १)\,वारसङ्कलितम्\,] + च\,(एकोनपदज-न-वारसङ्कलितम्)
\vspace{2mm}

\hspace{10mm} $= {\hbox{मु}}\,\dfrac{{\footnotesize{\hbox{प}}\,({\hbox{प}} + {\hbox{१}})\,({\hbox{प}} + {\hbox{२}})\, ...\, ({\hbox{प}} + {\hbox{न}} - {\hbox{१}})}}{{\footnotesize{\hbox{१.२.३\, ...\, न}}}} + {\hbox{च}}.\dfrac{{\footnotesize{\hbox{प}} - {\hbox{१}}}}{{\footnotesize{\hbox{१}}}}.\dfrac{{\footnotesize{\hbox{प}}}}{{\footnotesize{\hbox{२}}}}.\dfrac{{\footnotesize{\hbox{प}} + {\hbox{१}}}}{{\footnotesize{\hbox{३}}}}\, ...\, \dfrac{{\footnotesize{\hbox{प}} + {\hbox{न}} - {\hbox{१}}}}{{\footnotesize{\hbox{न}} + {\hbox{१}}}}$
\vspace{2mm}

\hspace{10mm} $= {\hbox{मु}}.\dfrac{{\footnotesize{\hbox{१}}}}{{\footnotesize{\hbox{प}} - {\hbox{१}}}}.\dfrac{{\footnotesize{\hbox{प}} - {\hbox{१}}}}{{\footnotesize{\hbox{१}}}}.\dfrac{{\footnotesize{\hbox{प}}}}{{\footnotesize{\hbox{२}}}}.\dfrac{{\footnotesize{\hbox{प}} + {\hbox{१}}}}{{\footnotesize{\hbox{३}}}}\, ...\, \dfrac{{\footnotesize{\hbox{प}} + {\hbox{न}} - {\hbox{१}}}}{{\footnotesize{\hbox{न}} + {\hbox{१}}}}.({\hbox{न}} + {\hbox{१}}) + {\hbox{च}} \times {\hbox{चयगुण}}$
\vspace{2mm}

\hspace{10mm} $= {\hbox{मु}}.\dfrac{{\footnotesize{\hbox{चयगुण}}}}{{\footnotesize{\hbox{प}} - {\hbox{१}}}}.({\hbox{न}} + {\hbox{१}}) + {\hbox{च}} \times {\hbox{चयगुण}}$
\vspace{2mm}

\hspace{2mm} अत उपपद्यते सर्वम्~।}

\phantomsection \label{Ex 3.15}
\begin{quote}
\textbf{{\color{red}आदिः समीरणमितः प्रचयस्त्रिसङ्ख्यो \\
गच्छेषु सप्तसु वदाशु परार्द्ध्यबुद्धे~।\\
वारैः पयोनिधिमितैः परिवर्तनेन \\
स्यात् किं फलं गणितमत्सरतास्ति ते चेत्~॥}}
\end{quote}

न्यासः~। आ ५~। उ ३~। ग ७~। वा ४~। जातं चतुर्वारश्रेढीगणितम् १८०६~। अत्र करणम्~। व्येकपदम् ६ अस्य चतुर्वारसङ्कलितम् 

\newpage
\begin{sloppypar}

\noindent २५२ अयमुत्तरगुणकारः~। पृथक् २५२ सैकवारेण ५ हतः १२६० व्येकपदेन ६ भक्तः २१० अयमादेर्गुणकारः~। स्वस्वगुणाभ्यामाभ्याम् २१०।२५२ आद्युत्तरौ ५।३ गुणितौ १०५०।७५६ अनयोर्योगश्चतुर्वारश्रेढीगणितम् १८०६~।\\

\noindent \textbf{सूत्रम्~।}

\phantomsection \label{3.21}
\begin{quote}
\renewcommand{\thefootnote}{१}\footnote{अत्रोपपत्तिः~।~ अब्दास्तर्ण्यब्दोनाः ~प्रथमतर्णकीप्रसवसङ्ख्या $=$ अ $-$ तअ~।~ द्वितीयतर्णकीप्रसवसङ्ख्या $=$ अ $-$ तअ $-$ १, एवं सप्तदशतर्णकीप्रसवसङ्ख्या $=$ १~। तद्युतिः (अ $-$ तअ)~। अस्य पदस्यैकवारसङ्कलितम्~। एवं\, अ $-$ २\,त\,अ\, एतत् पदस्य द्विवारसङ्कलितं तर्णकीसमुद्भूततर्णकीमानम्~। एवमग्रेऽपि~। सर्वसङ्कलितानाम् ऐक्यं प्रथमगोतर्णकीमानैरब्दमितैः सहितं प्रथमगोमानेन रूपसमेन चोपेतं सर्वगोसङ्ख्या~।
\vspace{1mm}

\hspace{2mm} अत्र कोष्ठकान्तर्गतवृत्तार्धस्य त्रुटिरस्ति सोदाहरणन्यासेन मया योजितापि बुद्धिमद्भिश्चिन्त्या~। \vspace{1mm} }{\large \textbf{{\color{purple}अब्दास्तर्ण्यब्दोनाः \\
पृथक् पृथग्यावदल्पतां यान्ति~।\\
तानि क्रमशश्चैका-\\
दिकवाराणां पदानि स्युः~॥~२१~॥\\
(सङ्कलितानामैक्यं साब्दं रूपान्वितं तु गोसङ्ख्या~।)}}}
\end{quote}

\noindent \textbf{उदाहरणम्~।}

\phantomsection \label{Ex 3.16}
\begin{quote}
\textbf{{\color{red}प्रतिवर्षं गौः सूते वर्षत्रितयेन तर्णकी\renewcommand{\thefootnote}{२}\footnote{{\color{violet}'सद्योजातस्तु तर्णकः'} इति {\color{violet}अमरकोशः}~। द्विका, वै.\,व.\,श्लोकः ९४७~।} तस्याः~।\\
विद्वन् विंशतिवर्षैर्गोरेकस्याश्च सन्ततिं कथय~॥}}
\end{quote}

न्यासः~। गौः १ वर्षाणि २० तर्णकीप्रसववर्षाणि ३~। अत्र करणम्~। अब्दाः २०~एते तर्णकीप्रसववर्षैस्त्रिभिः ३ पुनः पुनरूनिता एकद्वित्रिचतुष्पञ्चषड्वाराणां जातानि पदानि\, $\dfrac{{\footnotesize{\hbox{१७}}}}{{\footnotesize{\hbox{१}}}}\, \dfrac{{\footnotesize{\hbox{१४}}}}{{\footnotesize{\hbox{२}}}}\, \dfrac{{\footnotesize{\hbox{११}}}}{{\footnotesize{\hbox{३}}}}\, \dfrac{{\footnotesize{\hbox{८}}}}{{\footnotesize{\hbox{४}}}}\, \dfrac{{\footnotesize{\hbox{५}}}}{{\footnotesize{\hbox{५}}}}\, \dfrac{{\footnotesize{\hbox{२}}}}{{\footnotesize{\hbox{६}}}}$~।
\end{sloppypar}

\newpage
\begin{sloppypar}

\noindent अथ सप्तदशानाम् एकवारं सङ्कलितम् १५३~। चतुर्दशानां द्विवारम् ५६०~। एकादशानां त्रिवारम् १००१~। अष्टानां चतुर्वारम् ७९२~। पञ्चानां पञ्चवारम् २१०~। द्वयोः षड्वारम् ८~। जातानि सङ्कलितानि १५३।५६०।१००१।७९२।२१०।८ एषां योगः २७२४~। साब्दं रूपान्वि-तमित्यब्दविंशत्या सरूपया २१ युतो जाता गोसङ्ख्या २७४५~। 
\vspace{2mm}

अथवाङ्कपाशछन्दोलक्षणमेरूणा सिध्यति~। तत् पुरतो वक्ष्ये~। \\

\noindent \textbf{अथ सूत्रम्~।}

\phantomsection \label{3.23}
\begin{quote}
\renewcommand{\thefootnote}{१}\footnote{अत्रोपपत्त्यर्थं {\color{violet}भास्करस्य 'विषमे गच्छे व्येके गुणकः स्थाप्यः'} इत्यस्योपपत्तिर्मज्जनकशोधित{\color{violet}भास्कर-लीलावत्यां} ३६ पृष्ठेऽवलोक्या~।}{\large \textbf{{\color{purple}विषमे पदे विरूपे \\
गुणः समेऽर्धीकृते कृतिश्चान्त्यात्~।\\
गुणवर्गफलं व्येकं \\
व्येकगुणाप्तं मुखाहतं गणितम्~॥}}}
\end{quote}

\noindent \textbf{उदाहरणम्~।}

\phantomsection \label{Ex 3.17}
\begin{quote}
\textbf{{\color{red}आद्ये वराटकयुगं दिवसे द्विजाय \\
त्रिघ्नोत्तरेण धनवान् प्रददौ च कश्चित्~।\\
मासेन मे गणकवर्य कियद्धनं स्यात् \\
ब्रूह्याशु तेऽस्ति गणिते यदि मत्सरोऽत्र~॥}}
\end{quote}

न्यासः~। आ २~। गुउ ३~। ग ३०~। गणितं वराटकाः २०५८९११३२०९४६४८ एषां जाता निष्काः ५९५७४९८०३५ द्रम्माः ५ पणाः ३ काकिणी ० वराटकाः ८~।\\

\noindent \textbf{अपि च~।}

\phantomsection \label{Ex 3.18}
\begin{quote}
\textbf{{\color{red}यस्मिन्नादित्रयं द्विघ्नोत्तरः सप्तसु किं वद~।\\
पदेषु गणितं तस्मादादिं गुणोत्तरं\renewcommand{\thefootnote}{$\star$}\footnote{The reading च गणितं seems to be a redundancy error. गुणोत्तरं is what need to be found out.} पदम्~॥}}
\end{quote}

न्यासः~। आ ३~। उ २~। ग ७~। जातं गणितम् ३८१~। 
\end{sloppypar}

\newpage

\noindent \textbf{अत्राद्यानयने सूत्रम्~।}

\phantomsection \label{3.24}
\begin{quote}
\renewcommand{\thefootnote}{१}\footnote{अत्रोपपत्तिः~। पूर्वसूत्रेण गणितम् $= \dfrac{{\footnotesize{\hbox{आ}}\,({\hbox{गु}}^{\scriptsize{\hbox{ग}}} - {\hbox{१}})}}{{\footnotesize{\hbox{गु}} - {\hbox{१}}}}$
\vspace{2mm}
 
\hspace{2mm} $\therefore$\; आ $= \dfrac{{\footnotesize{\hbox{गणित}}\,({\hbox{गु}} - {\hbox{१}})}}{{\footnotesize{\hbox{गु}}^{\scriptsize{\hbox{ग}}} - {\hbox{१}}}}$~।~~ इत्युपपद्यते~। \vspace{2mm} }{\large \textbf{{\color{purple}रूपोनगुणकाभ्यस्ते गणितेऽत्र विभाजिते~।\\
गुणवर्गफलेनैकोनितेन वदनं भवेत्~॥}}}
\end{quote}

मुखेऽज्ञाते न्यासः~। आ ०~। गुउ २~। ग ७~। गणितम् ३८१~। अतो ज्ञात आदिः ३~। \\

\noindent \textbf{उत्तरानयने सूत्रम्~।}

\phantomsection \label{3.25}
\begin{quote}
\renewcommand{\thefootnote}{२}\footnote{अत्रोपपत्तिः~। पूर्वप्रकारेण गणितम् $= \dfrac{{\footnotesize{\hbox{आ}}\,({\hbox{गु}}^{\scriptsize{\hbox{ग}}} - {\hbox{१}})}}{{\footnotesize{\hbox{गु}} - {\hbox{१}}}}$
\vspace{2mm}

\hspace{3mm} अतः~~ $\dfrac{{\footnotesize{\hbox{गणि}}}}{{\footnotesize{\hbox{आ}}}} = \dfrac{{\footnotesize{\hbox{गु}}^{\scriptsize{\hbox{ग}}} - {\hbox{१}}}}{{\footnotesize{\hbox{गु}} - {\hbox{१}}}} =$ ल
\vspace{2mm}

\hspace{11mm} $\dfrac{{\footnotesize{\hbox{गणि}}}}{{\footnotesize{\hbox{आ}}}} - {\hbox{१}} = \dfrac{{\footnotesize{\hbox{गु}}^{\scriptsize{\hbox{ग}}} - {\hbox{गु}}}}{{\footnotesize{\hbox{गु}} - {\hbox{१}}}}$
\vspace{2mm}

\hspace{11mm} $\dfrac{\dfrac{{\footnotesize{\hbox{गणि}}}}{{\footnotesize{\hbox{आ}}}} - {\hbox{१}}}{{\footnotesize{\hbox{गु}}}} = \dfrac{{\footnotesize{\hbox{गु}}^{\scriptsize{\hbox{ग}} - {\hbox{१}}} - {\hbox{१}}}}{{\footnotesize{\hbox{गु}} - {\hbox{१}}}} = \dfrac{{\footnotesize{\hbox{ल}}}}{{\footnotesize{\hbox{गु}}}} = {\hbox{ल}}_{\scriptsize{\hbox{१}}}$
\vspace{1mm}
}{\large \textbf{{\color{purple}मुखहृद्गणितं रूपो-\\
ज्झितं\renewcommand{\thefootnote}{$\star$}\footnote{The reading रूपोन्वितं seems to be a typographical error as $1$ is to be subtracted as given in the footnote commentary to get the result.} यथा शुद्धिमेति येनाप्तम्~।\\
फलमेकोनं मुहुरपि \\
यावद्रूपं हरो भवेद्गुणकः~॥}}}
\end{quote}

पूर्वोदाहरणेऽज्ञातगुणोत्तरज्ञानार्थं न्यासः~। आ ३~। गुउ ०~। ग ७~। गणितम् ३८१~। ज्ञातो गुणोत्तरः २~। 

\newpage
\thispagestyle{empty}

\begin{center}
{\Large \textbf{THE ~PRINCESS ~OF ~WALES}}
\vspace{2mm}

{\Large \textbf{SARASWATI ~BHAVANA ~TEXTS}}
\vspace{2mm}

Edited by
\vspace{2mm}

{\large \textbf{GOPINATH ~KAVIRAJ, M.A.}}
\end{center}

\begin{longtable}{ p{.11\textwidth} p{.68\textwidth} p{.12\textwidth} } 
No.\,1\textendash & The Kiranavali Bhaskara, \textbf{(किरणावलीभास्कर) [वैशे-षिक]}, a Commentary on Udayana's Kiranavali, Dravya section, by Padmanabha Misra. \newline Ed.\,with introduction and Index by Gopinath Kaviraj, M.A. & Rs.\,1\textendash 12 \\
No.\,2\textendash & The Advaita Chintamani, \textbf{(अद्वैतचिन्तामणि) [वेदान्त]}, by Rangoji Bhatta, \newline Ed.\,with Introduction etc.\,by Narayana Sastri Khiste Sahityacharya. & Rs.\,1\textendash 12 \\
No.\,3\textendash & The Vedanta Kalpalatika, \textbf{(वेदान्तकल्पलतिका) [वेदान्त]}, by Madhusudana Saraswati. \newline Edited with Introduction etc.\,by Ramajna Pandeya Vyakaranacharya. & Rs.\,1\textendash 12 \\
No.\,4\textendash & The Kusumanjali Bodhani, \textbf{(कुसुमाञ्जलिबोधनी) [न्याय]}, a commentary on Udayana's Theistic Tract, Nyaya Kusumanjali, by Varadaraja. \newline Ed.\,with introduction etc.\,by Gopinath Kaviraj, M.A. & Rs.\,2\textendash 0 \\
No.\,5\textendash & The Rasasara \textbf{(रससार) [वैशेषिक]}, a commentary on Udayana's Kiranavali, Guna Section, by Bhatta Vadindra. \newline Ed.\,with Introduction etc.\,by Gopinath Kaviraj, M.A. & Rs.\,1\textendash 2 \\
No.\,6\textendash  \newline (Part I) & The Bhavana Viveka \textbf{(भावनाविवेक) [मीमांसा]}, by Mandana\; Misra,\; with\, a\, Commentary\; by\; Bhatta Umbeka. \newline Ed.\,with Introduction etc.\,by M.M.\,Ganganatha Jha, M.A., D.\,Litt. & Rs.\,0\textendash 12 \\
No.\,6\textendash  \newline (Part II) & Ditto \newline Ditto & Rs.\,0\textendash 12
\end{longtable}

\newpage
\renewcommand{\thepage}{\arabic{page}}
\setcounter{page}{2}

\begin{longtable}{ p{.11\textwidth} p{.68\textwidth} p{.12\textwidth} } 
No.\,7\textendash  \newline (Part I) & The Yoginihridaya dipika \textbf{(योगिनीहृदयदीपिका) [तन्त्र]}, by Amritananda Nath, being a commentary on Yogi-nihridaya, a part of Vamakesvara Tantra. \newline Ed.\,with Introduction etc.\,by Gopinath Kaviraj, M.A. & Rs.\,1\textendash 8\\
No.\,7\textendash  \newline (Part II) & Ditto \newline Ditto & Rs.\,1\textendash 4\\
No.\,8\textendash  & The Kavyadakini \textbf{(काव्यडाकिनी) [काव्यशास्त्र]}, by Gangananda Kavindra. \newline Ed.\,with Introduction etc.\,by Jagannatha Sastri Hoshing Sahityopadhyaya. & Rs.\,0\textendash 10\\
No.\,9\textendash  \newline (Part I) & The Bhakti Chandrika \textbf{(भक्तिचन्द्रिका) [भक्ति]}, a commentary on Sandilyas Bhaktisutras, by Narayana Tirtha. \newline Ed.\,with a Prefatory Note by Gopinath Kaviraj, M.A. & Rs.\,0\textendash 15\\
No.\,10\textendash  \newline (Part I) & The Siddhantaratna, \textbf{(सिद्धान्तरत्न) [गौडीयवैष्णवदर्शन]}, by Baladeva Vidyabhusana. \newline Ed.\,with a Prefatory Note by Gopinath Kaviraj, M.A. & Rs.\,1\textendash 2\\
No.\,10\textendash  \newline (Part II) & Do. \newline Do. & Rs.\,2\textendash 12\\
No.\,11\textendash  & The Sri Vidya Ratna Sutras, \textbf{(श्रीविद्यारत्नसूत्र) [तन्त्र]}, by Gaudapada, with a Commentary by Sankararanya. \newline Ed.\,with Introduction etc.\,by Narayana Sastri Khiste, Sahityacharya. & Rs.\,0\textendash 9 \\
No.\,12\textendash  & The Rasapradipa, \textbf{(रसप्रदीप) [अलङ्कार]}, by Prabha-kara Bhatta. \newline Ed.\,with Introduction etc.\,by Narayana Sastri Khiste Sahityacharya. & Rs.\,1\textendash 2\\
No.\,13\textendash  & The Siddhasiddhanta Sangraha, \textbf{(सिद्धसिद्धान्तसङ्ग्रह) [नाथमार्ग]}, by Balabhadra. &
\end{longtable}

\newpage

\begin{longtable}{ p{.115\textwidth} p{.68\textwidth} p{.12\textwidth} } 
 & Ed.\,with Introduction by Gopinath Kaviraj, M.A. & Rs.\,0\textendash 14 \\
No.\,14\textendash  & The Trivenika, by \textbf{(त्रिवेणिका) [अलङ्कार]}, by Asadhara Bhatta. \newline Ed.\,with Introduction by Batukanatha Sarma Sahityopadhyaya, M.A. and Jagannatha Sastri Hoshing Sahityopadhyaya. & Rs.\,0\textendash 14 \\
No.\,15\textendash  \newline (Part I) & The Tripurarahasya, (Jnana Khanda) \textbf{(त्रिपुरारहस्य, ज्ञानखण्ड) [तान्त्रिकदर्शन]}, \newline Ed.\,with a Prefatory Note by Gopinath Kaviraj, M.A. & Rs.\,0\textendash 14\\
No.\,15\textendash  \newline (Part II) & Do. \newline Do. & Rs.\,2\textendash 4\\
No.\,15\textendash  \newline (Part III) & Do. \newline Do. & Rs.\,2\textendash 0\\
No.\,15\textendash  \newline (Part IV) & Do.\;with Introduction, etc.\,by Gopinath Kaviraj, M.A. & \\
No.\,16\textendash & The Kavya Vilasa, \textbf{(काव्यविलास) [अलङ्कार]}, by Chiranjiva Bhattacharya. \newline Ed.\,with Introduction etc.\,by Batukanatha Sarma Sahityopadhyaya M.A.\,and Jagannatha Sastri Hoshing Sahityopadhyaya. & Rs.\,1\textendash 2\\
No.\,17\textendash & The Nyaya Kalika.\,\textbf{(न्यायकलिका) [न्याय]}, by Bhatta Jayanta. \newline Ed.\,with Introduction by M.M.\,Ganganatha Jha, M.A.\,D.Litt. & Rs.\,0\textendash 14\\
No.\,18\textendash \newline (Part I) & The Goraksa Siddhanta Sangraha, \textbf{(गोरक्षसिद्धान्त-सङ्ग्रह) [नाथमार्ग]}, \newline Ed.\,with a Prefatory Note by Gopinath Kaviraj M.A. & Rs.\,0\textendash 14\\
No.\,19\textendash \newline (Part I) & The Prakrita Prakasa \textbf{(प्राकृतप्रकाश) (प्राकृतव्याकरण)}, by Vararuchi with the Prakrita Sanjivani by Vasantaraja and the Subodhini by Sadananda. \newline Ed.\,with Prefatory note etc.\,by & \\
\end{longtable}

\newpage

\begin{longtable}{ p{.115\textwidth} p{.68\textwidth} p{.12\textwidth} } 
 & Batuk Nath Sarma, M.A.\,and Baladeva Upadhaya M.A. & Rs.\,2\textendash 4\\
No.\,19\textendash  \newline (Part II) & Ditto \newline Ditto & Rs.\,2\textendash 12\\
No.\,19\textendash  \newline (Part III) & Introduction etc.\,(In Preparation.) & \\
No.\,20\textendash & The Mansatattvaviveka \textbf{(मांसतत्त्वविवेक) [धर्मशास्त्र]}, by Visvanatha Nyayapanchanana Bhattacharya. \newline Edited with Introduction etc.\,by Pandit Jagannatha Sastri Hoshing Sahityopadhyaya, with a Foreword by Pandit Gopi Nath Kaviraja, M.A., Principal, Government Sanskrit College, Benares. & Rs.\,0\textendash 12\\
No.\,21\textendash  \newline (Part I) & The Nyaya Siddhanta Mala \textbf{(न्यायसिद्धान्तमाला) [न्याय]}, by Jayarama Nyaya Panchanan Bhatta-charya. \newline Edited with Introduction etc.\,by Dr.\,Mangal Deva Sastri, M.A., D.Phil.\,(Oxon), Librarian, Govt. Sanskrit Library, Saraswati Bhavana, Benares. & Rs.\,1\textendash 4\\
No.\,21\textendash  \newline (Part II) & Ditto \newline Ditto & Rs.\,2\textendash 0\\
No.\,22\textendash  & The Dharmanubandhi Slokachaturdasi \textbf{(धर्मानुबन्धि-श्लोकचतुर्दशी) [धर्मशास्त्र]}, by Sri Sesa Krisna with a Commentary by Rama Pandit. \newline Edited with Introduction etc.\,by Narayana Sastri Khiste Sahityacharya, Assistant Librarian, Government Sanskrit Library, Saraswati Bhavana, Benares. & Rs.\,1\textendash 0\\
No.\,23\textendash & The Navaratrapradipa \textbf{(नवरात्रप्रदीप) [धर्मशास्त्र]}, by Nanda Pandit Dharmadhikari. \newline Ed.\,with Introduction etc.\,by Vaijanatha Sastri Vara-kale, Dharmasastra-Sastri, Sadho- & 
\end{longtable}

\newpage

\begin{longtable}{ p{.115\textwidth} p{.68\textwidth} p{.12\textwidth} } 
 & lal Research Scholar, Sanskrit College, Benares, with a Foreword by Pandit Gopi Nath Kaviraj, M.A.\,Principal, Government Sanskrit College, Benares. & Rs.\,2\textendash 0\\
No.\,24\textendash & The Sri Ramatapiniyopanisad \textbf{(रामतापिनीयोपनिषद्) [उपनिषद्]}, with the commentary called Rama Kasika in purvatapini and Anandanidhi in Uttaratapini by Anandavana. \newline Ed.\,with Introduction etc.\,by Anantarama Sastri Vetala Sahityopadhyaya, Post-Acharya Scholar, Govt.\,Sanskrit College, Benares, with a Foreword by Pandit Gopi Nath Kaviraj, M.A.\,Principal, Government Sanskrit College, Benares. & Rs.\,3\textendash 12\\
No.\,25\textendash & The Sapindyakalpalatika \textbf{(सापिण्ड्यकल्पलतिका) [धर्मशास्त्र]}, by Sadasivadeva alias Apadeva with a commentary by Narayana Deva. \newline Edited with introduction etc.\,by Jagannatha Sastri Hosinga, Sahityopadhaya, Sadholal Research Scholar, Govt.\,Sanskrit College, Benares. & Rs.\,1\textendash 4\\
No.\,26\textendash & The Mrigankalekha Natika \textbf{(मृगाङ्कलेखानाटिका) [नाटिका]}, by Visvanatha Deva Kavi. \newline Edited with Introduction etc.\,by Narayana Sastri Khiste Sahityacharya, Asst.\,Librarian, Government Sanskrit Library, Benares. & Rs.\,1\textendash 0\\
No.\,27\textendash & The Vidvachcharita Panchakam \textbf{(विद्वचरितपञ्चकम्) [निबन्ध]}, By Narayana Sastri Khiste Sahityacharya, Assistant Librarian, Govt.\,Sanskrit College, Sara-swati Bhavana Library, Benares. With an Introduction by Gopinath &
\end{longtable}

\newpage

\begin{longtable}{ p{.115\textwidth} p{.68\textwidth} p{.12\textwidth} } 
 & Kaviraj, M.A., Principal, Govt.\,Sanskrit College, Benares. & Rs.\,2\textendash 0\\
No.\,28\textendash & The Vrata Kosa \textbf{(व्रतकोश) [धर्मशास्त्र]}, by Jagannatha Sastri\; Hosinga\; Sahityopadhyaya,\; late\; Sadholal\; Research Scholar, Sanskrit College, Benares. With a Foreword by Sri Gopinath Kaviraj, M.A., Principal, Govt.\,Sanskrit College, Benares. & Rs.\,4\textendash 0\\
No.\,29\textendash & The Vritti dipika \textbf{(वृत्तिदीपिका) [व्याकरण]}, By Mauni Sri Krisna Bhatta. \newline Edited with Introduction etc.\,by Pt.\,Gangadhara Sastri Bharadvaja, Professor, Govt.\,Sanskrit College, Benares. & Rs.\,1\textendash 2\\
No.\,30\textendash & The Padartha Mandanam \textbf{(पदार्थमण्डन) [वैशेषिक]} By Sri Venidatta. \newline Edited with Introduction etc.\,by Pandit Gopala Sastri Nene Professor Govt.\,Sanskrit College, Benares. & Rs.\,0\textendash 14\\
No.\,31\textendash  \newline (Part I) & The Tantratna \textbf{(तन्त्ररत्न) [मीमांसा]}, by Partha Sarathi Misra. \newline Edited by M.M.\,Dr.\,Ganganatha Jha, M.A., D.Litt. Vice Chancellor, Allahabad University, Allahabad. & Rs.\,1\textendash 14\\
No.\,31\textendash  \newline (Part II) & Ditto. \hspace{6mm} Ditto. \newline Edited\; by\, Pt.\, Gopal\; Sastri\; Nene,\; Govt.\, Sanskrit College, Benares. & \\
No.\,32\textendash & The\; Tattvasara\; \textbf{(तत्त्वसार)\; [न्याय]},\, by\, Rakhaldasa Nyayaratna. \newline Edited with Introduction etc.\,by Harihara Sastri, Benares Hindu University. & Rs.\,1\textendash 0\\
No.\,33\textendash  \newline (Part I) & The\; Nyaya\; Kaustubha\; \textbf{(न्यायकौस्तुभ)\; [न्याय]},\, by Mahadeva Puntamkara. \newline Edited with introduction etc.\,by & 
\end{longtable}

\newpage

\begin{longtable}{ p{.115\textwidth} p{.68\textwidth} p{.12\textwidth} } 
 & Umesa Misra, M.A., Allahabad University, Allahabad. & Rs.\,3\textendash 4\\
No.\,34\textendash  \newline (Part I) & The Advaita Vidyatilakam \textbf{(अद्वैतविद्यातिलकम्) [शाङ्करवेदान्त]}, by Sri Samarapungava Diksita. \newline With a Commentary by Sri Dharmayya Diksita. \newline Edited with Introduction etc.\,by Ganapati Lal Jha, M.A., Sadholal Research Scholar, Govt.\,Sanskrit Library, Benares. & Rs.\,1\textendash 4 \\
No.\,35\textendash & The Dharma Vijaya Nataka \textbf{(धर्मविजयनाटक) [नाटक]}, by Bhudeva Sukla. \newline Edited with Introduction etc.\,by Pandit Narayana Sastri Khiste, Asst.\,Librarian, Govt.\,Sanskrit Lib-rary, Benares. & Rs.\,1\textendash 4\\
No.\,36\textendash  & The Ananda Kanda Champu \textbf{(आनन्दकन्दचम्पू) [चम्पू]}, by Mitra Misra. \newline Edited, with a Foreword by Gopinath Kaviraj, M.A., by Nanda Kishore Sahityacharya, Research Scholar, Sanskrit College, Benares. & Rs.\,3\textendash 4\\
No.\,37\textendash & The Upanidana Sutra \textbf{(उपनिदानसूत्रम्) [वेद]}. \newline Edited with Introduction by Dr.\,Mangaldeva Sastri M.A.\,D.Phil. & Rs.\,1\textendash 0\\
No.\,38\textendash & The Kiranavali prakasa didhiti (Guna), \textbf{(किरणावली-प्रकाशदीधिति) [वैशेषिक]}, by Raghunath Siromani. \newline Edited by Pandit Badrinath Sastri, M.A., Lucknow University. & Rs.\,1\textendash 12\\
No.\,39\textendash & The Rama Vijaya Mahakavya \textbf{(रामविजयमहाकाव्य) [काव्य]}, by Rupanatha. \newline Edited by Pt.\,Ganapatilal Jha, M.A. & Rs 2\textendash 0
\end{longtable}

\newpage

\begin{longtable}{ p{.115\textwidth} p{.68\textwidth} p{.12\textwidth} } 
No.\,40\textendash  \newline (Part I) & The Kalatattva Vivechana \textbf{(कालतत्त्वविवेचन) [धर्म-शास्त्र]} by Raghunath Bhatta. \newline Edited, with a Foreword by Gopinath Kaviraj, M.A., by Nanda Kishore Sarma Sahityacharya, Research Scholar, Sanskrit College, Benares. & Rs.\,4\textendash 0\\
No.\,40\textendash  \newline (Part II) & Do. \newline Do. & \\
No.\,41\textendash  \newline (Part I) & The Siddhanta Sarvabhauma \textbf{(सिद्धान्तसार्वभौम) [ज्योतिष]}, by Sri Munisvara. \newline Edited with Introduction etc.\,by Jyautisacharya Pandit Murlidhar Thakkura, late Sadholal Scholar, Sanskrit College, Benares & Rs.\,3\textendash 0\\
No.\,42\textendash  & The Bheda Siddhi \textbf{(भेदसिद्धि) [न्याय]}, by Visvanath Panchanana Bhattacharya. \newline Edited with notes etc.\,by Nyaya Vyakaranacharya Pandit\; Surya\; Narayana\; Sukla,\; Professor,\; Govt. Sanskrit College, Benares. & \\
No.\,43\textendash  \newline (Part I) & The Smartollasa \textbf{(स्मार्तोल्लास) [कर्मकाण्ड]} by Siva Prasada. \newline Edited with Introduction, notes etc.\,by Vedacharya Pandit\; Bhagavat\; Prasad\; Misra,\; Professor,\; Govt. Sanskrit College, Benares. & \\
No.\,44\textendash  \newline (Part I) & Sudrachara Siromani \textbf{(शूद्राचारशिरोमणि) [धर्मशास्त्र]}. \newline Edited by Sahityacharya Pandit Narayana Sastri Khiste. & \\
No.\,45\textendash  \newline (Part I) & Kiranavali Prakasa (Guna) \textbf{(किरणावली प्रकाश) (गुण) [वैशेषिक]}, by Vardhamana. & 
\end{longtable}

\newpage

\begin{longtable}{ p{.115\textwidth} p{.68\textwidth} p{.12\textwidth} } 
 & Edited, with a Foreword by Pandit Gopinath Kaviraj, M.A., by Pandit Badrinath Sastri M.A.\,Lucknow University. & Rs.\,1\textendash 8\\
No.\,45\textendash  \newline (Part II) & Do. \newline Do. & \\
No.\,46\textendash  \newline (Part I) & Kavya prakasa dipika \textbf{(काव्यप्रकाशदीपिका) [अलङ्कार]} by Sri Chandi Dass. \newline Edited\; by\; Sri\; Sivaprasada\; Bhattacharya,\; M.A., Professor Presidency College, Calcutta. & Rs.\,1\textendash 12\\
No.\,47\textendash  & Bhedajayasri \textbf{(भेदजयश्री) [माध्ववेदान्त]}, by Sri Tarka-vagisa Bhatta Venidattacharya \newline Edited with Introduction etc.\,by Pandit Tribhuvan prasad Upadhyaya, M.A., Inspector of Sanskrit Pathshalas, United provinces, Benares. & Rs.\,1\textendash 4 \\
No.\,48\textendash  & Samyak Sambuddha bhasitam Buddhapratimala-ksanam \textbf{(सम्यक्संबुद्धभाषितं प्रतिमालक्षणम्) [शिल्प-शास्त्रम्]}. \newline With the Commentary Sambuddhabhasita-pratima-laksana Vivarani. Critically edited with Introduction etc.\,by Haridas Mitra, M.A. Visvabharati, Santini-ketana & Rs.\,1\textendash 4\\
No.\,49\textendash  & Bhedaratna \textbf{(भेदरत्न) [न्याय]} by Sankara Misra. \newline Edited with Introduction etc., by Pandit Suryanarayana Sukla, Professor, Govt.\,Sanskrit College, Benaras. & Rs.\,1\textendash 8\\
No.\,50\textendash  & Matrika Chakra Viveka \textbf{(मातृकाचक्रविवेक) [तन्त्र]}, by Svatantrananda Natha, with a Commentary. \newline Edited by Pandit Lalita Prasad Dabral Vyakarana-charya. \newline With a Foreword by Pt.\,Gopinath Kaviraj, M.A., Principal, Govt.\,Sanskrit College, Benares. & Rs.\,2\textendash 0
\end{longtable}

\newpage

\begin{longtable}{ p{.14\textwidth} p{.8\textwidth} } 
No.\,51-52. & Advaita Siddhanta Vidyotana \textbf{(अद्वैतसिद्धान्तविद्योतन) [वेदान्त]}, by Brahmananda Saraswati \newline \hspace{10mm} and \newline Nrisimha Vijnapana \textbf{(नृसिंहविज्ञापन) [वेदान्त]}, by Nrisimhasrama. \newline Edited with notes, Introduction etc.\,by Pandit Surya Narayana Sukla, Professor, Govt.\,Sanskrit College, Benares.\\
No.\,53 & Nrisimha-Prasada-Vyavaharasara \textbf{(नृसिंहप्रसादव्यवहारसार) [धर्मशास्त्र]}, by Dalapati Raja. \newline Edited with Introduction etc.\,by Pandit Vinayaka Sastri Tillu, Research Scholar, Sanskrit College, Benares.\\
No.\,54 & Nrisimha-Prasada-Prayaschitta-Sara \textbf{(नृसिंहप्रसादप्रायश्चित्तसार) [धर्मशास्त्र]}, by Sri Dalapathi Raja. \newline Edited by Pandit Nanda Kishora Sharma and Nanda Kumar Sharma Sahityacharya.\\
No.\,55 & Nrisimha-Prasada-Sraddha-Sara \textbf{(नृसिंहप्रसादश्राद्धसार) [धर्म-शास्त्र]}, \newline Edited by Pandit Vidyadhara Misra, College of Oriental Learning, Benares Hindu University, Benares.\\
No.\,56 & Bhagavannama Mahatmya Samgraha \textbf{(भगवन्नाममाहात्म्यसङ्ग्रह) [भक्तिशास्त्र]}, by Raghunathendra Yati, with Com.\,by Ananta Sastri Phadke. \newline Edited by Pt Ananta Sastri Phadke.\\
No.\,57 \newline (Part I) & Ganita Kaumudi \textbf{(गणित कौमुदी) [गणित]} by Narayana pandit. \newline Edited by Pt.\,Padmakara Dvivedi, Professor, Govt.\,Sanskrit College, Benares.\\
No.\,58 & Khyativada \textbf{(ख्यातिवाद) [वेदान्त]} by Sankara Chaitanya Bharati.
\end{longtable}

\newpage

\begin{longtable}{ p{.14\textwidth} p{.8\textwidth} } 
 & Edited by Sankara Chaitanya Bharati, with Foreword by M.\,Gopinath Kaviraj M.A. \\
No.\,59\textendash & Sankhya Tattvaloka \textbf{(साङ्ख्यतत्त्वालोक) [साङ्ख्य]}, by Harihara-nanda Aranya. \newline Edited with Introduction by Jajneswar Ghosh M.A. \newline With a Foreword by M.\,Pt.\,Gopinath Kaviraj, M.A. \\
No.\,60\textendash   \newline (Part I) & Sandilya Samhita \textbf{(शाण्डिल्यसंहिता) [पाञ्चरात्र]} \newline Edited by Pt.\,Ananta Gopal Phadke, Professor, Govt.\,Sanskrit College, Benares. \\
No.\,62\textendash  & Nrisinha-Prasada-Tirtha-Sara \textbf{(नृसिंहप्रसादतीर्थसारः) [धर्मशास्त्र]}. \newline Edited by Pt.\,Surya Narayan Sukla. \\
No.\,63\textendash  & Bhaktyadhikarana mala \textbf{(भक्त्यधिकरणमाला) [भक्तिशास्त्र]} by Narayana Tirtha. \newline Edited by Pt.\,Anant Gopal Phadke.\\
No.\,64\textendash  & Vasistha Darsana \textbf{(वासिष्ठदर्शनम्) [वेदान्त]}. \newline Compiled by Dr.\,B.L.\,Atreya. \newline Edited by Dr B.L.\,Atreya, M.A., Ph D., Lecturer, Benares Hindu University.\\
\begin{tabular}{l}
\hspace{-3mm} No.\,65-67 \\
 \\
 \\
 \\
 \\
 \\
 \\
\end{tabular} & \begin{tabular}{ll}
(a) & Tirthendusekhara \textbf{(तीर्थेन्दुशेखरः) [धर्मशास्त्र]} by Nagisa.\\
(b) & Tristhali-Setu~~ \textbf{(त्रिस्थलीसेतुः)~~ [धर्मशास्त्र]}~~ by~ Bhattoji \\
 & Diksita.\\
(c) & Kasimoksavichara \textbf{(काशीमोक्षविचारः) [वेदान्त]} by\, Sure- \\
 & swara Acharya. \\
 & Edited with Introduction by Pt.\,Suryanarayan Sukla, \\
 & Prof. Govt.\,Sanskrit College, Benares.
\end{tabular} \\
No.\,68 & Madhvamukhalankara \textbf{(मध्वमुखालङ्कारः) [मध्ववेदान्त]} by Vanamali Misra.
\end{longtable}

\newpage

\begin{longtable}{ p{.14\textwidth} p{.8\textwidth} } 
 & Edited with Introduction by Pt.\,Narasinha Varakhedkar. \newline With a Foreword by M.\,Pandit Gopinath Kaviraj M.A.
\end{longtable}
\vspace{1mm}

\begin{center}
\textbf{WORKS ~IN ~THE ~PRESS}
\end{center}
\vspace{-2mm}

\begin{longtable}{ p{.13\textwidth} p{.81\textwidth} } 
No.\,1\textendash & Daksinamurti Samhita \textbf{(दक्षिणामूर्तिसंहिता) [तन्त्र]}. \newline Edited by Pt.\,Narayana Sastri Khiste.\\
No.\,2\textendash & Asvalayana Srauta Sutra with Siddhanti Bhasya \textbf{(सिद्धान्ति-भाष्यसहित आश्वलायनश्रौतसूत्र) [वेद]}. \newline Edited by Dr.\,M.D.\,Sastri, M.A., D.Phil.\\
No.\,3\textendash & Niti manjari \textbf{(नीतिमञ्जरी) [वेद]}, by Dya Dvivedi. \newline Edited by Dr.\,Mangaldeva Sastri, M.A., D.Phil.\\
No.\,4\textendash  \newline (Part II) & Nyaya-Kaustubha-Anumanakhanda \textbf{(न्यायकौस्तुभ-अनुमानखण्ड) [न्याय]}, by Mahadeva Puntamkar. \newline Edited by Pt.\,Goswami Damodara Sastri.\\
No.\,5\textendash & Mimansa Chandrika \textbf{(मीमांसाचन्द्रिका) [मीमांसा]}, by Brahma-nanda Saraswati. \newline Edited by Pt.\,Haran Chandra Bhattacharya Sastri\\
No.\,6\textendash  \newline (Part III) & Tantraratna \textbf{(तन्त्ररत्न) [मीमांसा]}, by Partha Sarathi. \newline Edited by Pt.\,Gopal Sastri Nene.\\
No.\,7\textendash  \newline (Part II) & Kavya-prakasa-dipika \textbf{(काव्यप्रकाशदीपिका) [अलङ्कार]}, by Sri Chandidass. \newline Edited by Pt.\,Sivaprasada Bhattacharya, M.A.\\
No.\,8\textendash  \newline (Part III) & Kalatattvavivechana \textbf{(कालतत्त्वविवेचन) [धर्मशास्त्र]}, by Raghunatha Bhatta. \newline Edited by Pt.\,Nanda Kishore Sharma.\\
No.\,9\textendash  \newline (Part III) & Siddhanta-Sarvabhauma \textbf{(सिद्धान्तसा-} 
\end{longtable}

\newpage

\begin{longtable}{ p{.13\textwidth} p{.81\textwidth} } 
 & \textbf{र्वभौम) [ज्यौतिष]}, by Munisvara. \newline Edited by Pt.\,Murlidhar Thakkur.\\
No.\,10\textendash  & Upendra Vijnana Sutra \textbf{(उपेन्द्रविज्ञानसूत्र) [दर्शन]}. \newline Edited by Dr.\,M.D.\,Shastri.\\
No.\,11\textendash  & Nyayamrita Saurabha \textbf{(न्यायामृतसौरभ) [माध्ववेदान्त]}, by Vanamali. \newline Edited by Pt.\,Nrisimha Acharya.\\
No.\,12\textendash  & Isvara\, pratyabhijna\, karikas of Utpala\, with\, the\, Vimarsini of Abhinava Gupta and Com.\,on Vimarsini by Bhaskara Kantha. \newline Edited by K.\,Subrahmanya Iyer, M.A., and K.C.\,Pandey, M.A., Ph.D.
\end{longtable}
\vspace{-3mm}

\begin{center}
\line(4,0){70}
\end{center}

\newpage
\thispagestyle{empty}

\begin{center}
{\Large \textbf{THE ~PRINCESS ~OF ~WALES}}
\vspace{2mm}

{\Large \textbf{SARASWATI ~BHAVANA ~STUDIES}}
\vspace{2mm}

Edited by
\vspace{2mm}

\textbf{GOPINATH ~KAVIRAJ, M.A.}
\end{center}

\begin{sloppypar}
\noindent Vol.\,I\textendash 
\vspace{-2mm}

\begin{enumerate}[\indent(a)]
\setlength{\itemsep}{0pt}
\setlength{\parskip}{0pt}
\item Studies in Hindu Law (1): its Evolution, by Ganganatha Jha.
\item The View point of Nyaya Vaisesika Philosophy, by Gopinath Kaviraj.
\item Nirmana Kaya, by Gopinath Kaviraj.
\end{enumerate}
\vspace{-4mm}

\hfill Rs.\,1\textendash \,12.\\

\noindent Vol II\textendash 
\vspace{-2mm}

\begin{enumerate}[\indent(a)]
\setlength{\itemsep}{0pt}
\setlength{\parskip}{0pt}
\item Parasurama Misra alias Vani Rasala Raya, by Gopinath Kaviraj.
\item Index to Sabara's Bhasya, by the Late Col.\,G.A.\,Jacob.
\item Studies in Hindu Law (2):\textendash \,its sources, by Ganganath Jha
\item A New Bhakti Sutra, by Gopinath Kaviraj.
\item The system of Chakras according to Goraksa natha, by Gopinath Kaviraj.
\item Theism in Ancient India, by Gopinath Kaviraj.
\item Hindu Poetics, by Batuka natha Sarma.
\item A Seventeenth Century Astrolabe, by Padmakara Dvivedi
\item Some aspects of Vira Saiva Philosophy, by Gopinath Kaviraj.
\item Nyaya Kusumanjali (English Translation), by Gopinath Kaviraj.
\item The Definition of Poetry by Narayana Sastri Khiste.
\item Sondala Upadhyaya, by Gopinath Kaviraj.
\end{enumerate}
\vspace{-4mm}

\hfill Rs.\,5.\\

\noindent Vol III\textendash 
\vspace{-2mm}

\begin{enumerate}[\indent(a)]
\setlength{\itemsep}{0pt}
\setlength{\parskip}{0pt}
\item Index to Sabara's Bhasya, by the Late Col.\,G.A.\,Jacob.
\item Studies in Hindu Law (3): Judicial Procedure: by Ganganatha Jha.
\item Theism in Ancient India, by Gopinath Kaviraj.
\item History and Bibliography of Nyaya Vaisesika Literature by Gopinath Kaviraj.
\item Naisadha and Sri Harsa by Nilkamal Bhattacharya.
\item Indian Dramaturgy, by P.N.\,Patankar.
\end{enumerate}
\vspace{-4mm}

\hfill Rs.\,5.
\end{sloppypar}

\newpage
\renewcommand{\thepage}{\arabic{page}}
\setcounter{page}{2}

\begin{sloppypar}
\noindent Vol.\,IV\textendash 
\vspace{-2mm}

\begin{enumerate}[\indent(a)]
\setlength{\itemsep}{0pt}
\setlength{\parskip}{0pt}
\item Studies in Hindu 'Law' (4): Judicial Procedure: by Ganganatha Jha.
\item History and Bibliography of Nyaya Vaisesika Literature, by Gopinath Kaviraj.
\item Analysis of the Contents of the Rigveda-Pratisakhya, by Mangala Deva Sastri.
\item Narayana's Ganita kaumudi, by Padmakara Dvivedi.
\item 'Food and Drink in the Ramayanik Age', by Manmatha natha Roy.
\item Satkaryavada: Casualty in Sankhya, by Gopinath Kaviraj.
\item Discipline by Consequences, by G.L.\,Sinha.
\item History of the origin and expansion of the Aryans, by A.C.\,Ganguly.
\item Punishments in Ancient Indian Schools, by G.L.\,Sinha.
\end{enumerate}
\vspace{-4mm}

\hfill Rs.\,5.\\

\noindent Vol V\textendash 
\vspace{-2mm}

\begin{enumerate}[\indent(a)]
\setlength{\itemsep}{0pt}
\setlength{\parskip}{0pt}
\item Ancient Home of the Aryans and their migration to India by A.C.\,Ganguly.
\item A Satrap Coin, by Shamalal Mehr.
\item An Estimate of ~the Civilization of the Vanaras~ as depicted in the Ramayana, by Manmatha nath Roy.
\item A comparison of the Contents of the Rgveda, Vajasaneya, Tattiriya and Atharvaveda Pratisakhyas, by Mangala Deva Sastri.
\item Formal Training and the Ancient Indian Thought, by G.L.\,Sinha.
\item History and Bibliography of Nyaya Vaisesika Literature by Gopinath Kaviraj.
\item A descriptive Index to the names in the Ramayana, by Manmatha nath Roy.
\item Notes and Queries: (1) Virgin Worship, by Gopinath Kaviraj.
\end{enumerate}
\vspace{-4mm}

\hfill Rs.\,5.
\end{sloppypar}

\newpage

\begin{sloppypar}
\noindent Vol.\,VI \textendash 
\vspace{-2mm}

\begin{enumerate}[\indent(a)]
\setlength{\itemsep}{0pt}
\setlength{\parskip}{0pt}
\item Index to Sabara's Bhasya, by the late Col.\,G.A.\,Jacob.
\item Some Aspects of the History and Doctrines of the Nathas, by Gopinath Kaviraj.
\item An Index etc.\,the Ramayana, by Manmatha natha Roy.
\item Studies in Hindu Law by M.M.\,Ganganath Jha.
\item The Mimamsa manuscripts in the Govt.\,Sanskrit Library (Benares), by Gopinath Kaviraj.
\item Notes and Queries, by Gopinath Kaviraj.
\end{enumerate}

\noindent Vol VII\textendash 
\vspace{-2mm}

\begin{enumerate}[\indent(a)]
\setlength{\itemsep}{0pt}
\setlength{\parskip}{0pt}
\item Bhamaha and his Kavyalankar, by Batukanath Sarma and Baladeva Upadhyaya.
\item Some variants in the readings of the Vaisesika Sutras, by Gopinath Kaviraj.
\item History and Bibliography of Nyaya Vaisesika Literature, by Gopinath Kaviraj.
\item An attempt to arrive at the correct meaning of some obscure Vedic words, by Sita Ram Joshi.
\item A comparison of the contents of the Rig Veda, Vajasaneya Taittiriya, and Atharva Veda (Chaturadhyayika) Pratisakhyas, by Mangal Deva Shastri.
\item An index to the Ramayana, by Manmatha Natha Roy.
\item An Index to Sabara's Bhasya, by the late Col.\,G.A.\,Jacob.
\item Gleanings from the Tantras by Gopinath Kaviraja.
\item The date of Madhususdana Saraswati, by Gopinath Kaviraj.
\item Descriptive notes on Sanskrit Manuscripts, by Gopinath Kaviraj.
\item A note on the meaning of the word Parardha, by Umesa Misra. 
\end{enumerate}
\vspace{-4mm}

\hfill Rs.\,5.\\

\noindent Vol VIII\textendash 
\vspace{-2mm}

\begin{enumerate}[\indent(a)]
\setlength{\itemsep}{0pt}
\setlength{\parskip}{0pt}
\item Indian Philosophy, by Tarakanath Sanyal.
\item An Index to the Ramayana, by Manmatha Nath Roy.
\item Index to Sabara's Bhasya, by the late Col.\,G.A.\,Jacob.
\end{enumerate}

\end{sloppypar}

\newpage

\begin{sloppypar}
\begin{enumerate}[\indent(a)]
\setlength{\itemsep}{0pt}
\setlength{\parskip}{0pt}
\addtocounter{enumi}{3}
\item Hari Swami, the commentator of Satpatha Brahmana and the date of Skanda Svami ~the commentator of the Rigveda, by Mangaladeva Sastri.
\item Mysticism in Veda, ~by Gopinath Kaviraj.
\item The Deva dasi\textendash \,a brief history of the Institution, by Manmath Natha Roy.
\end{enumerate}
\vspace{-4mm}

\hfill Rs.\,5\\

\noindent Vol IX\textendash 
\vspace{-2mm}

\begin{enumerate}[\indent(a)]
\setlength{\itemsep}{0pt}
\setlength{\parskip}{0pt}
\item The Life of a Yogin, by Gopinatha Kaviraj.
\item On the antiquity of the Indian art Canons, by Haridas Mitra.
\item Prachya Vargikarana paddhati, by Shatis Chandra Guha.
\item Yoga Vasishtha and some of the minor Upanishads, by B.L.\,Atreya.
\item An index to the proper names occurring in Valmiki's Ramayan, by Manmath nath Roy.
\item The Philosophy of Tripura Tantra, by Gopinath Kaviraj.
\item Notes on Pasupata Philosophy, by Gopinath Kaviraj.
\end{enumerate}
\vspace{4mm}
\end{sloppypar}

\begin{center}
\line(4,0){70}
\end{center}

\newpage
\thispagestyle{empty}

\begin{center}
{\Large \textbf{THE ~PRINCESS ~OF ~WALES}}
\vspace{2mm}

{\Large \textbf{SARASWATI ~BHAVANA ~STUDIES}}
\vspace{2mm}

\textbf{(SANSKRIT)}
\vspace{2mm}

{\large \textbf{SARASVATALOKA}}
\vspace{2mm}

Edited by
\vspace{2mm}

\textbf{GOPINATH ~KAVIRAJ, M.A.}
\end{center}

\begin{sloppypar}
\noindent Kirana 1 (In progress).
\vspace{-2mm}

\begin{enumerate}[\indent(a)]
\setlength{\itemsep}{0pt}
\setlength{\parskip}{0pt}
\item Mangalam, etc, by Narayana Sastri Khiste.
\item Mimansaka mata samgraha, by Haranchandra Bhattacharya.
\item Srimad Acharya Mandana Misra by Chinna Swami Sastri.
\item Bhagavato Buddhasya Chaitama Upadesascha by Gopinath Kaviraj.
\end{enumerate}

\noindent Kirana I (Supplemental)\\
\vspace{-2mm}

Sanskrita Kavi Parichaya\textendash \,(Bharavi) by Nanda Kishore Sharma.\\

\noindent Kirana II (In progress)
\vspace{-2mm}

\begin{enumerate}[\indent(a)]
\setlength{\itemsep}{0pt}
\setlength{\parskip}{0pt}
\item Sarada Prasadanam by Nārāyaṇa Śāstri Khiste.
\item Chūdamani Darsanam by Sasadhara Tarkachudamani.
\end{enumerate}
\vspace{30mm}
\end{sloppypar}

\begin{flushright}
To be had of \hspace{8mm} ~ \\
The Superintendent \hspace{2mm} ~ \\
Government Press, U.P.\\
Allahabad. \hspace{7mm} ~ 
\end{flushright}

\end{document}
