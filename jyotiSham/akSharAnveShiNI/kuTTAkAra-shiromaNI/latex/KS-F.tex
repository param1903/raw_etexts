\documentclass[11pt, openany]{book}
\usepackage[text={4.65in,7.45in}, centering, includefoot]{geometry}

\usepackage[table, x11names]{xcolor}
%\include{alias}

\usepackage{fontspec,realscripts}
\usepackage{polyglossia}

\setdefaultlanguage{sanskrit}
\setotherlanguage{english}
%\setmainfont[Scale=1]{Times New Roman}
%\newfontfamily\regular[Scale=1]{Times New Roman}
\defaultfontfeatures[Scale=MatchUppercase]{Ligatures=TeX} 
\newfontfamily\sanskritfont[Script=Devanagari]{Shobhika}
\newfontfamily\englishfont[Language=English, Script=Latin]{Linux Libertine O}
\newfontfamily\ks[Script=Devanagari, Color=purple]{Shobhika-Bold}
\newfontfamily\qt[Script=Devanagari, Scale=1, Color=violet]{Shobhika-Regular}
\newfontfamily\ku[Script=Devanagari, Color=red]{Shobhika-Bold} 
%\newfontfamily\s[Script=Devanagari, Scale=0.9]{Shobhika-Regular}
\newfontfamily\s[Script=Devanagari, Scale=0.9]{Shobhika-Regular}
\newcommand{\devanagarinumeral}[1]{%
	\devanagaridigits{\number\csname c@#1\endcsname}}
\usepackage{fancyhdr}
\pagestyle{fancy}
\renewcommand{\headrulewidth}{0pt}
%\newfontfamily\e[Scale=0.8]{Shobhika-Regular}
\XeTeXgenerateactualtext=1
\usepackage{enumerate}
%\pagestyle{plain}
%\pagestyle{empty}
\usepackage{afterpage}
\usepackage[document]{ragged2e}
\usepackage{amsmath}
\usepackage{amssymb}
\usepackage{tikz}
\usepackage{graphicx}
\usepackage{longtable}
\usepackage{multirow}
\usepackage{footnote}
%\usepackage{dblfnote} 
\usepackage{xspace}
%\newcommand\nd{\textsuperscript{nd}\xspace}
\usepackage{array}
\usepackage{emptypage}
\usepackage{hyperref}   % Package for hyperlinks
\hypersetup{
	colorlinks,
	citecolor=black,
	filecolor=black,
	linkcolor=blue,
	urlcolor=black
}

\begin{document}
\thispagestyle{empty}
\begin{center}
\vspace{2cm}
{\huge \textbf{आनन्दाश्रमसंस्कृतग्रन्थावलिः~।}}\\
{\large ग्रन्थाङ्कः~१२५~।\\}
{\vspace{2mm}}
\textbf{स्वोपज्ञया महालक्ष्मीमुक्तावल्याख्यव्याख्यया संवलितः\\}
{\vspace{2mm}}
\textbf{श्रीदेवराजविरचितः }\\
\textbf{\Huge
कुट्टाकारशिरोमणिः~।}\\
{\vspace{2mm}}
\textbf{सोऽयम्}\\
{\vspace{2mm}}
\textbf{आपटेकुलोत्पन्नेन दत्तात्रेयसूनुना बलवन्तरायेण\\
आनन्दाश्रमस्थपण्डितानां साहाय्येन\\
संशोधितः~।\\}
{\vspace{2mm}}
\rule{0.2\linewidth}{1.0pt}\\
{\vspace{2mm}}
\textbf{स च\\
रावबहादुर इत्युपपदधारिभिः\\}
{\vspace{2mm}}
\textbf{\Huge गंगाधर बापूराव काळे}\\
\textbf{जे. पी.\\
इत्येतैः\\
पुण्याख्यपत्तने\\
श्रीमन् ' महादेव चिमणाजी आपटे ' इत्यभिधेय \textendash\\
महाभागप्रतिष्ठापिते\\}
{\vspace{2mm}}
{\Huge \textbf{आनन्दाश्रममुद्रणालये}}\\
{\vspace{2mm}}
\textbf{आयसाक्षरैर्मुद्रयित्वा\\}
{\vspace{2mm}}
प्रकाशितः।\\
{\vspace{2mm}}
\textbf{शालिवाहनशकाब्दाः~१८६६~।}\\
{\vspace{2mm}}
ख्रिस्ताब्दाः~१९४४~।\\
{\vspace{2mm}}
(~अस्य सर्वेऽधिकारा राजशासनानुसारेण स्वायत्तीकृताः~)~।\\
{\vspace{2mm}}
{\large \textbf{मूल्यमाणकनवकम्~ ( ९~आणा )}}
\end{center}

\newpage
\thispagestyle{fancy}
\fancyhf{}
\chead{\textbf{कुट्टाकारशिरोमणेः परिचयः~।}}

\justifying

\indent
भो अभिज्ञमणय आनन्दाश्रमप्रकाशितशास्त्रीयग्रन्थकुसुमामोदात्राणप्रणयिनो ज्योतिर्विद्वरा गाणितिकधौरेयाश्च किंचिदिहावधत्त~। महालक्ष्यीमुक्तावलीसंवलितं कुट्टाकारशिरोमाणिनामानमपूर्वग्रन्थं पण्डितशिरोमणीनां श्रीमतां हस्ते समर्पयितुं नितान्तं प्रमोदते नः स्वान्तम्~। उत्सहते च गणकाग्रणीनां प्रकृतविषये जिज्ञासोत्पत्त्यर्थं ग्रन्थशिरोमणिं कुट्टाकारं परिचाययितुम्~। अतस्तत्स्वरूपं संक्षेपतः प्रकाश्यते \textendash\\

\indent
छेदनार्थात्कुट्टधातोर्निष्पन्नः कुट्टशब्दो लीलावत्यादौ प्रसिद्धं कुट्टकशब्दावबोध्यं भाज्यभाजकादिगणनं यत्र तादृशं गणितविशेषं ब्रूते~। 'भाज्यो हारः क्षेपकश्चापवर्त्यः केनाप्यादौ संभवे कुट्टकार्थम्' इत्युक्तत्वात्~। आकारशब्दः स्वरूपवचनः~। तथाच कुट्टस्य गणितविशेषस्य आकारः स्वरूपप्रदर्शको ग्रन्थः कुट्टाकारः~। स च शिरोमणिरिवेत्युपमितसमासेन कुट्टाकारशिरोमणिरिति प्रकृतग्रन्थस्य नामधेयं संपन्नम्~। तदुक्तमेतद्ग्रन्थाभ्यासफलं प्रदर्शयता देवराजेनैव मूले\textendash  कुट्टाकारशिरोमाणिमिममभ्यसति दृढेन यो मनसा~। तरणेः प्रसादतोऽयं सहसा तान्त्रिकशिरोमणिर्भवति~।।~इति~।\\

\indent
तस्यैतस्य कुट्टशब्दवाच्यस्य गणितविशेषस्य साग्रकुट्टाकारो निरग्रकुट्टाकारः संश्लिष्टकुट्टाकारो मिश्रश्रेढीमिश्रकुट्टाकारश्चेत्येवं स्थूलतोऽत्र चत्वारो भेदा दृग्गोचरी भवन्ति~। वारकुट्टाकारवेलाकुट्टाकारौ तु निरग्रकुट्टाकारस्यैवावान्तरभेदौ~।संश्लिष्टकुट्टाकारस्त्रिप्रकारक इति मूल एव स्पष्टम्~। प्रकृतग्रन्थनिर्मात्रा देवराजेनात्र परिच्छेदत्रयं प्रकल्प्य यस्मिन् परिच्छेदे यस्य यस्य कुट्टाकारस्य लक्षणोदाहरणोपपत्तिसहितं विवेचनं कृतमस्ति तस्य तस्य परिच्छेदस्य तत्तत्कुट्टाकारनामघटितमभिधेयं प्रदत्तमित्यालक्ष्यते~। यथा प्रथमपरिच्छेदे साग्रकुट्टाकारस्य विवृतत्वात् ' साग्रपरिच्छेदः प्रथमः ' इति~। अस्मिन् परिच्छेदे द्वाविंशतिरार्याः संप्रदृश्यन्ते~। एवं निरग्रपरिच्छेदो द्वितीयः~।अत्र निरग्रुकुट्टाकारादेः संश्लिष्टकुट्टाकारस्य च त्रिप्रकारस्य वर्णनादार्याणामेकोना नवतिः संलक्ष्यते~। तृतीये मिश्रश्रेढीमिश्रकुट्टाकारपरिच्छेदे चैकोनविंशतिसंख्याका आर्याः प्रणीताः सन्ति~। एवं ग्रन्थरचना संपादिताऽस्ति~। ग्रन्थान्तरे तु वल्लिकाकुट्टाकारः स्थिरकुट्टाकारश्चेत्यपरौ द्वौ भेदौ लक्ष्येते~। तयोर्मध्ये वल्लिकाकुट्टाकारस्य

\newpage
\thispagestyle{fancy}
\fancyhf{}
\chead{\textbf{[२]}}

\noindent
विचारो वृद्धार्यभटस्य साक्षाच्छिष्येण भास्करेण वेदाब्धियुगमिते~(४४४)~शकसमये निर्मिते महाभास्करीयादिग्रन्थे विस्तरशो वर्तते~। सोऽसौ महाभास्करीयग्रन्थोऽपि सटीकोऽचिरेणैव कालेन प्रकाश्येताऽनन्दाश्रमसंस्थया~।\\

\indent
भरतखण्डेऽस्मिन् ब्राह्म आर्यः सौर इत्येवं त्रयः प्रामुख्येण ज्योतिषसिद्धान्तग्रन्थाः सन्ति~। तेषां मध्ये कस्मिन्देशे कः सिद्धान्त आदरणीय इत्येतत्प्रदर्शितमग्रिमश्लोके \textendash

\begin{quote}
{\qt गोदावरीविन्ध्यनगान्तराले श्रीब्रह्मपक्षाद्गणितं विधेयम् ~।\\
गोदावरीदक्षिण आर्यपक्षो विन्ध्याचलादुत्तरतो हि सौरः ~॥~इति~।}
\end{quote}


\indent
तत्राऽऽर्यभटाचार्येण भूनेत्राब्धि~(४२१)~मिते शकसमये लिखित आर्यभटीयग्रन्थे गणितपादे~(३२\textendash\ ३३)~श्लोकयोः कुट्टकनामा विषयोऽयं त्रोटकेन कथितः~। परं त्वनेन देवराजेनात्र निरुक्तकुट्टाकारभेदा नियमपरिबद्धया गणितशास्त्रीयपद्धत्या सविस्तरं वर्णिताः~। अथ च ते ते प्रश्नाः कथमुपपादनीयास्तदपि मूलप्रमाणं धृत्वा नियमबद्धां रीतिमनुसृत्य प्रदर्शितम्~।तत्तत्कुट्टाकारविषयकगणितप्रक्रियाया मध्यमाद्यधिकारिणामपि यथावदर्थबोधो भूयादित्यनुसंधाय स्वेनैव स्वग्रन्थोपरि महालक्ष्मीमुक्तावली नाम व्याख्या व्यरचि~। अतो नवानवानेकाङ्कुराेत्पादनेन बुद्धिमुल्लासयन्तं प्रकृतं ग्रन्थाशिरोमाणिं कुट्टाकारं वारं वारं समभिवाच्य प्राचीनानां गणितशास्त्रतत्त्वविदामाचार्याणां सैद्धान्तिकीं सरलां सरसां सुखावबोधां च सुव्यवस्थितनियमनियन्त्रितां गणितप्रक्रियाप्रस्थापनचातुरीमास्वादयन्त्वधुनातना अर्वाचीनगणितरीत्यभिज्ञा अपि गणकोत्तंसा विद्वांस इत्यभ्यर्थये~।\\



\indent
स एष निरुक्तः कुट्टाकारशिरोमाणिर्नैव स्वतन्त्राे ग्रन्थः, अपि त्वार्यभटाचार्यप्रणीतं कुट्टाकारविषयकं द्वात्रिंशत्रयस्त्रिंशश्लोकात्मकं यत्सूत्रयुगलं तस्य व्याख्यानविशेष एवेत्यवगन्तव्यम्~। 

तदुक्तं स्वेनैव स्वग्रन्थे मूल एव\textendash

\begin{quote}
{\qt आचार्यार्यभटोदितकुट्टाकारार्थसूत्रयुगलस्य~।\\
कुट्टाकारशिरोमणिनामा व्याख्याविशेष एवैषः ~॥~ इति~॥}\\
\end{quote}

\indent
अथैष प्रकृतग्रन्थप्रणेता देवराजोऽत्रिकुलाभरणस्य स्कन्धत्रयवेदिनः सिद्धान्तवल्लभ इति प्रसिद्धापरनाम्नः श्रीवरदराजाचार्यस्य तनय इति ग्रन्थसमाप्तौ तदुल्लेखादवसीयते~। परं त्वनेनायं कुट्टाकारग्रन्थः कदा निरमायीति विशेषतो ज्ञातुं यद्यपि न शक्यं तथाऽपि द्वितीयपरिच्छेदे शेषे लीलावती \textendash

\newpage
\thispagestyle{fancy}
\fancyhf{}
\chead{\textbf{[३]}}
\noindent
गतानां निरग्रकुट्टाकारविषयाणां कतीनांचन श्लोकानामुद्धृतत्वात्सिद्धान्तशिरोमणिकाराद् भास्कराचार्यादनन्तरं नाम शा.~( १०७२ ) समये प्रकृतग्रन्थः प्रादुरासीदिति निश्चयेनानुमातुं न कियानपि कश्चिदपि प्रतिबन्ध इत्यवधेयम्~।\\

\indent
तस्यास्य सटीककुट्टाकारशिरोमणेः संशोधने पुस्तकद्वयं सहायभूतमभूत्~। तत्रैकमादावन्ते च भूयसांऽशेन त्रुटितम्~। द्वितीयं समग्रमप्यतीव स्थूलतः शुद्धियुतम् ~। तदेतेत्पुस्तकद्वयं भाण्डारकरप्राच्यविद्यामन्दिराधिकृतैः 'गोडे ' इत्युपाह्वप्राध्यापकैर्महीशूरनगरात्तञ्जावरनगराच्च महता प्रयासेन नागर्यां
लिप्यां प्रतिच्छायीकृतं संपाद्य मुद्रणार्थं प्रदत्तमिति तेषामविस्मरणार्हाः
खलूपकृतीः शिरसा वहामि~।\\

\indent
तदेवमतिप्रौढविषयस्यान्यत्र क्वाप्यमुद्रितस्य कुट्टाकारशिरोमणिग्रन्थस्य संशोधने प्रथमत एव प्रवृत्तस्य कै. पितृचरणप्रसादलब्धप्राचीनगणितशास्त्ररीतिदृष्टेरपि मम तादृशबुद्धिप्रागल्भ्याभावाद्यावदपेक्षितसाधनसामग्र्यभावाच्च पदे पदे च्युतयः संभवेयुः~। तासामावेदनेन मामनुगृह्णन्तु परोपकारनिरता ग्राहकमहाशया इति प्रार्थयते\textendash\\
\begin{table}[h!]
    \centering
    \begin{tabular}{ccc}
  पुण्यपत्तनम् ~। श.~१८६६    &\multirow{2}{*}& आपटेइत्युपाह्वदत्तात्रेयसूनुर्बलवन्तरायः~।, \\
     का.~ शु.~९~ बुधवासरः ~। &   & (~आनन्दाश्रमव्यवस्थापकः~)~।\\
    \end{tabular}
\end{table}

\centering
\rule{0.1\linewidth}{1.0pt}

\newpage
\thispagestyle{empty}
.\\
\vspace{6cm}
\rule{1.0\linewidth}{2.0pt}\\
\rule{1.0\linewidth}{1.0pt}\\

\vspace{3mm}
\textbf{\Large
प्रथमावृत्तौ पुस्तकानि (~२५०~)\\}

\vspace{3mm}
\rule{1.0\linewidth}{1.0pt}\\
\rule{1.0\linewidth}{2.0pt}\\




\newpage
\thispagestyle{fancy}
ॐ तत्सद्ब्रह्मणे नमः ~।\\
\vspace{2mm}
\textbf{श्रीदेवराजविरचितः}\\
\vspace{2mm}
(~स्वोपज्ञया महालक्ष्मीमुक्तावल्याख्यव्याख्यया संवलितः~)\\
\vspace{2mm}
\textbf{\Huge
कुट्टाकारशिरोमणिः~।}\\
\rule{0.6\linewidth}{1.0pt}\\


\vspace{2mm}

\textbf{ नत्वा श्रीरमणं गुरूनपि मया प्रोक्तो मुदे धीमता \textendash \\
\hspace{2cm}
माचार्यार्यभटोक्तकुट्टयुगलीव्याख्याविशेषोऽमु (सु) कः~।\\
कुट्टाकारशिरोमणिः स्फुटपदैरङ्केश्च सांकेतिकै \textendash\\
\hspace{2cm}
र्वाक्यैरप्यनुयोगवृत्तसहितैर्व्याख्यायते विस्तरात्~॥}\\

\justifying
\indent
अथायं देवराजः प्रारिप्सितस्य ग्रन्थस्य निर्विघ्नेन परिसमाप्तये प्रचयगमनाय
चाऽऽद्यया गीत्या स्वाभिमतदेवताप्रणिपातं कृत्वा ग्रन्थारम्भं प्रतिजानीत \textendash\\
\begin{quote}
\ks 
नत्वा रमाधरण्यौ वरदार्यसुतेन देवराजेन~।\\
आर्यभटाचार्यकृतः कुट्टाकारः प्रकाश्यते स्पष्टम्~॥~१~॥
\end{quote}

\indent
रमाधरण्यौ लक्ष्मीभुमिदेव्यौ~। नत्वा प्रणिपत्य~। वरदार्यसुतेन देवराजेनात्रिकुलतिलकस्य सिद्धान्तवल्लभ इति प्रसिद्धापरनाम्नः श्रीवरदार्यस्य तनयेन देवराजेन~। देवराज इति ग्रन्थकर्तुरभिधानम्~। अनेनाऽऽर्यभटाचार्यकृत आर्यभटविरचितः कुट्टाकारः साग्रनिरग्ररूपेण द्विविधः खण्डनात्मको गणितविशेषः~। कुट्ट च्छेदनभर्त्सनयाेः ( पा.धा.च. उभ. सेट्)~। कुट्यते छिद्यते खण्ड्यतेऽनेनेति करणे [\textsuperscript{*}वञ्~]~। टाप्प्रत्ययः~। अत्राऽऽकारशब्दः स्वरूपवचनः~। अतः कुट्टास्वरूपः कुट्टाकार इति~। स्पष्टं व्यक्तं प्रकाश्यते व्यज्यते~। अस्य ग्रन्थस्य कुट्टाकारशिरोमणिरिति नाम विज्ञायते~। एतद्यन्योपसंहारगीत्यां ' कुट्टाकारशिरोमणिभिममभ्यस्यति' इति वचनदर्शनात्~। अत्र च तन्नामकरणमिति ज्ञातत्वादेतद्ग्रन्थार्थावबोधेनाऽऽचार्यार्यभटप्रणीतकुट्टाकारसूत्रयुगलस्याप्यर्थः सम्यग्व्यक्ति भविष्यतीति तस्यायं व्याख्याविशेषो भवतीत्युपपन्नम् ~॥~१~॥\\
\indent
अथ कुट्टाकारस्य द्वैविध्यं तदुक्तक्रमं चाऽऽर्ययाऽह \textendash
\begin{quote}
\ks
कुट्टाकारौ भवतः साग्रनिरग्राविहाऽऽदितः साग्रम्~।\\
संक्षिप्य दर्शयित्वा तताे निरग्रः प्रदर्श्यते व्यक्तम्~॥~२~॥
\end{quote}
\indent
कुट्टाकारौ साग्रनिरग्रौ भवतः~। इह साग्रकुट्टाकारनिरग्रकुट्टाकारयोर्मध्य
\footnotetext{
\textsuperscript{*}करणाधिकरणयाेश्च~(~पा.~सू. ~३~।~३~।~११७ )~।}

\newpage
\thispagestyle{fancy}
\fancyhf{}
\chead{\textbf{महालक्ष्मीमुक्तावलीसहितः\textendash}}
\lhead{\textbf{२}}
\noindent
आदितः प्रथमतः साग्रं साग्रकुट्टाकारं संक्षिप्य समस्य दर्शयित्वा~। उक्त्वेत्यर्थः\textendash \ ततः साग्रकथनानन्तरं निरग्रो निरग्रकुट्टाकारो व्यक्तं स्फुटं
प्रदर्श्यते~। उच्यत इति यावत्~॥~२~॥\\
\indent
अथात्र कं सिद्धान्तमवलम्ब्य भागहारभाज्योक्तिरिति न ज्ञायत इति
तज्ज्ञानार्थमार्ययाऽऽह \textendash
\begin{quote}
\ks
अत्र हरभाज्यराशी भास्करसिद्धान्तमार्गमवलम्ब्य~।\\
उच्येते दैवज्ञैरार्यभटीये च तौ समारोप्यौ ~॥~३~॥
\end{quote}

\indent
अत्रेह ग्रन्थे हरभाज्यराशी हरराशिश्चतुर्युगसंबन्धिरविमासचान्द्रदिनसावनदिवसादिः~। भाज्यराशिश्चतुर्युगसंबन्ध्यधिमासावनदिनग्रहभगणादिः ~। तौ भास्करसिद्धान्तमार्गं श्रीसूर्यसिद्धान्तस्य पन्थानमवलम्ब्योच्येते तौ पूर्वोक्तौ हरभाज्यराशी दैवज्ञैर्ज्यौतिषिकैरार्यभटीये चाऽऽर्यभटसिद्धान्तेऽपि समारोप्यौ~। अथ श्रीसूर्यसिद्धान्ताश्रयणेन यद्यदुच्यते तत्र चाऽऽर्यभटसिद्धान्ताश्रयणे चोन्नेयमित्यर्थः~॥ ३~॥\\
\indent
अथाऽऽर्यया साग्रविषयं प्रश्नं स्पष्टयति \textendash

\begin{quote}
\ks
उद्दिष्टहारकद्वयसंभक्तविभाज्यराशिशेषाभ्याम्~।\\
यः सपदि भाज्यराशिं कथयति भूमौ स साग्रविद्भवति ~॥~४~॥
\end{quote}
\indent
\textbf{उद्दिष्टहारकद्वयसंभक्तविभाज्यराशिशेषाभ्यामिति~}। विभक्तव्यो राशिर्विभाज्यराशिः~। हारकयोर्द्वयं हारकद्वयम्~। उद्दिष्टं च तद्धारकद्वयं चोद्दिष्टहारकद्वयम्~। तेषु संभक्त उद्दिष्टहारकद्वयसंभक्तः~। उद्दिष्टहारकद्वयसंभक्तश्चायं विभाज्यराशिश्चोद्दिष्टहारकद्वयसंभक्तविभाज्यराशिः~। तस्य शेषौ तथोक्तौ~। ताभ्यामुद्दिष्टाभ्यां भाज्यराशिं यः सपदि कथयति स भूमौ साग्रविद्भवति~। अत्रेदं
प्रश्नस्वरूपं \textendash\ यस्मिन्कस्मिंश्चिद्भाज्यराशावुद्दिष्टाभ्यां भिन्नसंख्याभ्यां याभ्यां काभ्यां भागहाराभ्यां पृथक्पृथग्विभक्ते ये शिष्येते ताभ्यामतुल्यसंख्याभ्यां कथिताभ्यां भाज्यराशिः सद्योऽन्वेषणीय इति~। एतदुपलक्षणम्~। एकस्मिन्नेव भाज्यराशौ पृथक्पृथग्बहुभिरुद्दिष्टैर्द्विविधैर्भागहारैर्विभक्ते शेषाभ्यां विलक्षणाभ्यामुक्ताभ्यां
भाज्यराशिरन्वेष्टव्य इत्येवमादौ प्रश्ने (श्नौ)~। अत्र भागहाराणां भक्तशेषसाम्यवैषम्यवशाद्द्वैविध्यम्~। एतच्चोत्तरत्र व्यक्ती भविष्यति ~॥~४~॥

\newpage
\thispagestyle{fancy}
\fancyhf{}
\chead{\textbf{कुट्टाकारशिरोमणिः~।}}
\rhead{\textbf{३}}
\indent
अथोक्तप्रश्नभङ्गार्थमार्यामाह \textendash
\begin{quote}
\ks
अधिकाग्राद्यार्यायुगमाचार्यार्यभटविरचितं साग्रे~।\\
आरोप्य तत्र जातैः कर्मकदम्बैः प्रसाधयेद्भाज्यम्~॥~५~॥
\end{quote}

\indent
आचार्यार्यभटविरचितमधिकाग्राद्यार्यायुगं साग्र आरोप्य तत्र जातैः कर्मकदम्बैर्भाज्यं प्रसाधयेत्~॥~५~॥\\

\indent
अत्रैतदार्यभटाचार्यप्रणीतं कुट्टाकारयुग्मविषयमधिकाग्रादिसूत्रद्वयम् \textendash
\begin{quote}
\ks
{\textsuperscript{*}}अधिकाग्रभागहारं छिन्द्यादूनाग्रभागहारेण~।\\
शेषपरस्परभक्तं मतिगुणमग्रान्तरे क्षिप्तम् ~॥~६~॥\\
अध उपरि गुणितमन्त्ययुगूनाग्रच्छेदभाजिते शेषम्~।\\
अधिकाग्रच्छेदगुणं द्विच्छेदाग्रमधिकाग्रयुतम्~॥~७~॥~इति~।
\end{quote}
\indent
एतत्सूत्रद्वयं प्रश्नभङ्गार्थे साग्रमधिकृत्य व्याख्यायते~। अधिकाग्रभागहारमिति~। अत्राग्रशब्दः शेषवचनः~। साग्रकोटिरितिवत्~। अधिकमग्रं यस्य सोऽधिकाग्रः~। भागहारो भाजकः~। अधिकाग्रश्चासौ भागहारश्चाधिकाग्रभागहारः~। येन भाज्यराशौ भक्ते शेषमन्यस्माच्छेषादधिकसंख्यं भवति सोऽधिकाग्रभागहार इत्यर्थः \textendash \ ऊनाग्रभागहारेणोनमितरस्माच्छेषाद्धीनमग्रं शेषं यस्य स ऊनाग्रः~। ऊनाग्रश्चासौ भागहारश्च स तथोक्तः \textendash\ तेन च्छिन्द्याद्विभजेत्~। एतच्च सति संभवे कर्तव्यम्~। अधिकाग्रभागहारमूनाग्रभागहारेण च्छिन्द्यादिति पृथग्वाक्यकरणादत्र लब्धेन प्रयोजनं नास्तीति ज्ञाप्यते~। अथ शेषपरस्परभक्तम्~। भक्तं भजनम्~। भावे निष्ठा~। अत्र
कार्यमिति पदमध्याहार्यम्~। शेषयोः परस्परभजनं कार्यम्~। एतदुक्तं भवति~। ऊनाग्रभागहारेणाधिकाग्रभागहारे हृते यच्छिष्यते तेनोनाग्रभागहारं विभजेत्~। तत्र

\footnotetext{
\textsuperscript{*} (~आर्यभटीये~(~शा. ~४२१) गणितपादे श्लो.~३२\textendash\ ३३)।\\
तथा \textendash\  ब्राह्मस्फुटसिद्धान्ते (~शा.~५५०)~कुट्टाकाध्याये श्लो.~३\textendash\ ५~पृ.~२९४~।\\
महावीराचार्य कृतगणितसारसंग्रहे~(~शा.~७७५)~वल्लिकाकुट्टीकारे श्लो.~११५~॥\\
पृ.~८०)~। लघुआर्यभटीये~(~शा.~८७५) कुट्टकाध्याये श्लो.~३~पृ.~२२४~।\\ सिद्धान्तशिरोमणिकार~(~१०७२)~भास्कराचार्यकृते कुट्टकविवरणे श्लो.~५१~ पृ.~५९~ ।}

\newpage
\thispagestyle{fancy}
\fancyhf{}
\chead{\textbf{महालक्ष्मीमुक्तावलीसहितः\textendash}}
\lhead{\textbf{४}}
\noindent
लब्धं फलत्वेन कुत्रचित्स्थापयेत्~। ततस्तच्छेषेणाधिकाग्रभागहारं शेषं विभजेत्~। लब्धं पूर्वं स्थापितस्य फलस्याधः स्थापयेत्~। ततस्तच्छेषेणोनाग्रभागहारशेषं विभजेत्~। लब्धं पूर्वनिहितयोः फलयोरधो निदध्यात्~। एवं तावत्कुर्याद्यावत्फलपङ्क्तिः समा भवति~। शेषाववप्यल्पाविति~। अथ मतिगुणं मत्या गुणितं कार्यम्~। एतदुक्तं भवति~। परस्परभक्तशेषयोर्यद्यल्पमुपरि स्थितं तद्यथासंख्यमग्रान्तरेण क्षिप्तं चाधः ' स्थितेनान्येन भक्तं शुध्यति~। सा मतिः~। तथाऽल्पमुपरि स्थितं परस्परभक्तशेषं गुणितं कार्यमिति~। अत्रैतदनुसंधेयम्~। पूर्वमुपर्यधोभावेन निहितानां समपङ्क्तीनां फलानामधो मतिसंज्ञं गुणमपि स्थापयेदिति~। एतच्च मतिगुणमित्यस्यार्थान्तराङ्गीकरणेन सिध्यति~। तद्यथा \textendash \ मतिगुणं गुणो गुणकारः~। ' गुणकारस्तु गुणको गुणो वर्धक इत्यपि' इति सिद्धान्तदर्पणकारो गोविन्दभट्टयज्वा~। मतिरेव गुणो मतिगुणः~। मत्याख्यो गुण इत्यर्थः \textendash\  तं स्थापयोदिति~। नन्वत्र स्थापयेदिति नोक्तं तत्कथम्~। एवमत्रोच्यते \textendash\  मतिगुणमिति कर्मणः श्रवणादत्र सकर्मकं किंचित्क्रियापदमिति विज्ञाप्यते~। तत्र स्थापयेदित्यनवद्यम्~। अग्रान्तरे~। अग्रयोरन्तरमग्रान्तरम्~। उद्दिष्टाभ्यां भागहाराभ्यां भाज्यराशौ पृथक्पृथग्विभक्ते ये शिष्येते तयोर्विवरमिति यावत्~। तस्मिन्निक्षिप्तं निहितं कार्यम्~। अत्र मतिगुणितं प्रकृतम्~। तस्मान्मतिगुणितमत्राग्रान्तरे निधेयमिति~। अत्रैतदनुसंधेयम्~। यदत्र मतिगुणितमग्रान्तरे क्षिप्तं च तत्स्वाधःस्थितेनान्येन विभज्य लब्धं फलं मतेरधस्तान्निदध्यादिति~। एतच्च मतिगुणमित्यादिना सूचितं सूत्रकृतेत्यवगन्तव्यम्~। अध उपरि गुणितम्~। अत्र अधःशब्देनोपर्यधोभावेन विन्यस्तेषु समतिषु फलपदेषु मत्यादिकमुपान्त्यमुच्यते~। उपरिशब्देनोपान्त्यस्योर्ध्वपङ्क्तिस्थम्~। गुणशब्देन गुणत्वम्~। तस्मादुपान्त्यस्य तदुपरि फलस्य चान्योन्यगुणनं कार्यम्~। एतदुक्तं भवति \textendash\ मत्यादिकेनोपान्त्येन तदुपरि फलं गुणयेदिति~। ननु चात्र मुख्यवृत्त्या सर्वाधःस्थितस्यान्त्यस्य फलस्य वाचकोऽधःशब्दः कथमुपान्त्यमाह \textendash\ अत्रोच्यते \textendash\ अन्त्ययुगित्यस्य विषयाभावात्~। अन्त्ययुगुपान्त्येन गुणितं तदुपरिफलमत्र प्रकृतम्~। तदन्त्येन युक्तं कार्यम्~। एवं तावत्कार्यं यावद्द्वावेव राशी भवतः~। राशिद्वयमात्रे त्वधउपरिगुणितमित्येतन्न प्रवर्तते~। तस्मादत्राधोराशिना प्रयोजनं नास्तीति स त्याज्यः~। ऊनाग्रच्छेदभाजिते~। छेदो भागहारः~।~''भागहारो हरो हारो हारकश्छेदभाजकौ'' इति गोविन्दभट्टसोमसुतः~। अत्रोपरिराशिः प्रकृतः~।

\newpage
\thispagestyle{fancy}
\fancyhf{}
\chead{\textbf{कुट्टाकारशिरोमणिः~।}}
\rhead{\textbf{५}}
\noindent
तस्मिन्सति संभवे पूर्वोक्तेनोनाग्रभागहारेण भक्ते सति शेषं यच्छिष्टं तत्~।अत्रैतदनुसंधेयम्~। यत्रोनाग्रच्छेदभाजिते शेषमित्यत्र शेषं नास्ति तत्रोनाग्रच्छेदमात्रं निधेयमिति \textendash\ अधिकाग्रच्छेदगुणम्~। पूर्वोक्तेनाधिकाग्रभागहारेण गुणितम्~।अधिकाग्रयुतम्~। अत्राप्यग्रशब्दः शेषवचनः~। अधिकं चैतदग्रं चाधिकाग्रम् । तेन युक्तं द्विच्छेदाग्रम् । इहाप्यग्रशब्दः शेषवचनः~। द्वौ छेदौ द्विच्छेदौ~।अत्राष्टादशपुराणानामितिवचन इव संख्यावाचिना द्विशब्देन च्छेदौ विशेष्येते इति विशेषणसमासः~। तयोरग्रे द्विच्छेदौ यस्य स्त इति मत्वर्थीयोऽच्प्रत्ययः~।द्विच्छेदाग्रवद्भाज्यराश्यन्तरं भवतीत्यर्थः \textendash\ अयं भाज्यराशिः प्रथमः~। अत उत्तरेषामपरिमितानां भाज्यराशीनामानयनप्रकारः प्रदर्श्यते~। तद्यथा \textendash\ अधिकाग्रभागहारमित्येतदादिनोनाग्रच्छेदभाजिते शेषमित्यन्तेन गणितेन यन्निष्पद्यते तदध्रुवम्~। तत्रैकगुणितमूनाग्रभागहारं निक्षिप्यधिकाग्रभागहारेण गुणयित्वा तत्राधिकाग्रं युञ्ज्यात्~। ततस्तादृशः कश्चिद्भाज्यराशिर्भवति~। ध्रुवे द्विगुणितमूनाग्रभागहारं दत्त्वा तदधिकाग्रभागहारेण~(हत्वा)~तत्राधिकमग्रं निदध्यात्~। ततस्तदपरः कश्चित्तादृशो भाज्यराशिर्भवति~। एवं त्र्यादिषु गुणितमूनाग्र~[भाग]~हारं ध्रुवे निक्षिप्योक्तप्रक्रियया
भाज्यराशिसहस्रमानयेत्~। अथात्र केषुचित्प्रश्नविशेषेषु मत्यन्तराण्यङ्गीकृत्योक्तध्रुवाणि
कृत्वा तद्वशेनाप्युक्तप्रक्रिययाऽपरिमितान् भाज्यराशीनानयेत्~। अत्रेदमनुसंधेयम्~। अ(य)त्रोनाग्रभागहारेणाधिकाग्रभागहारः शुध्यति तत्राधिकाग्रयुक्तोऽधिकाग्रभागहारः प्रथमो भाज्यराशिर्भवति~। तत्रैकादीष्टसंख्यागुणितोऽग्रभागहारः स्वाग्रयुक्तस्तदनन्तरादिस्तदुत्तरो भाज्यराशिर्भवतीति~। इहैतदप्यनुसंधेयम् \textendash\ शेषपरस्परभक्तमित्यत्र यत्र फलिपङ्क्तिरेका भवति तदा शेषयोरप्येकः
शुध्यति~। यत्र च फलपङ्क्ती द्वे भवतस्तदा शेषयोश्चैकः शुध्यति~। तत्रोभयत्र प्रथमत एव प्रत्यासन्नां मतिं कल्पयित्वा तयाऽधिकाग्रभागहारं हत्वा तत्राधिकमग्रमर्पयेत्~। एवं कृते प्रथमो भाज्यराशिर्भवति~। अत्र प्रथमत एव कल्पितां प्रत्यासन्नां मतिमेकादीष्टसंख्यागुणितोनाग्रभागहारयुक्तां कृत्वाऽधिकाग्रभागहारेण हत्वा तत्राधिकमग्रं च दत्त्वा तदपरान् भाज्यशीनानयेत्~। अत्रापि केषुचित्प्रश्नविशेषेषु प्रथमत एव मत्यन्तराण्युररीकृत्य तद्वशादनन्तरोक्तप्रक्रिययाऽन्यानपि भाज्यराशीनानयेदिति\textendash\  एतानि सर्वाणि परे विशेषाश्च क्रमेणोदाहरणमुखेन स्पष्टी क्रियन्ते~॥~६~॥~७~॥

\newpage
\thispagestyle{fancy}
\fancyhf{}
\chead{\textbf{महालक्ष्मीमुक्तावलीसहितः\textendash}}
\lhead{\textbf{६}}

उद्देशकः \textendash 
\begin{quote}
{\ku अष्टादशभिरेकाग्रो यो भवेद्राशिरुद्धृतः~।\\
एकोनत्रिंशता चाऽऽप्तः सप्ताग्रस्तं सुधीर्वद~॥~१~॥~(८)~॥}
\end{quote}

\indent
अत्राधिकाग्रभागहार एकोनत्रिंशत्~(२९)~। अस्मिन्नूनाग्रभागहारेणाष्टादशकेन
~(१८)~च्छिन्ने शिष्टराश्योरुपरिराशिरेकादशसंख्यः~(११)~। अधोराशिरष्टादशसंख्यः~(१८)~। एतावुपर्यधोभावेन स्थापितौ~(११)~। अथोपरिराशिनैकादशकेन\\
\indent
\hspace{5.7cm}
१८~।\\
\noindent
(११)~स्वाधोराशावष्टादशके~(१८)~विभक्ते लब्धमेकम्~(१)~। ततस्तच्छेषेण सप्तकेन~(७)~स्वोपरिराशावेकादशके~(११)~भक्ते लब्धमेकम्~(१)~। इदं पूर्वफलस्याधो निहितम्~। तच्छेषेण चतुष्केण~(४)~स्वाधोराशौ सप्तके~(७)~विभक्ते लब्धमेकम्~(१)~। इदं पूर्वफलयोरधो दत्तम्~१,~१,~१~। ततस्तच्छेषेण त्रिकेण~(३)~स्वोपरिराशौ चतुष्के~(४)~विभक्ते लब्धमेकम्~(१)~। तत्पूर्वफलानामधो विन्यस्तम्~१,~१,~१,~१~। इत्थं चतस्रः फलपक्तयः~। शेषयोरुपर्येकं ~(१)~अधस्त्रीणि~१,~३~। अग्रान्तर षट्कम्~(६)~। अत्र युग्माः फलपक्तयः~। शेषावपि लघूकृताविति मतिः कल्प्यते~। अयं रूपात्मक उपरिराशिः कया संख्यया गुणितः षट्कमितेनाग्रान्तरेण युक्तश्च स्वाधोराशिना त्रिकेण शुध्यतीति लब्धा प्रत्यासन्ना मतिस्रिसंख्याका~(३)~। एषा चतुर्णां पूर्वेषां फलानामधो निहिता १,~१,~१,~१,~३~। अथोपरिराशौ मतिगुणितेऽग्नान्तरेण च युक्ते नवकं~(९)~जातम्~। अस्मिन्स्वाधोराशिना त्रिकेण~(३)~हृते लब्धं त्रयम्~(३)~। एतन्मतेरधस्ताद्दत्तम् १,~१,~१,~१,~३,~३~। एषा फलवल्लीत्युच्यते~। अथैतेषु समतिषूपान्त्येन मतिसंज्ञकेन त्रिकेण गुणेन गुणिते रूपात्मके तदुपरिफलेऽन्त्येन त्रिकेण च युक्ते षट्कं~(६)~ जातम्~। ततः शिष्टेषु पञ्चसु पदेषूपान्त्येन षट्केन गुणिते रूपात्मके तदुपरिफलेऽन्त्येन मतिसंज्ञकेन त्रिकेण च युक्ते नवकं~(९)~जातम्~। ततः शिष्टेषु चतुर्षूपान्त्येन नवकेन गुणिते रूपात्मके तदुपरिफलेऽन्त्येन षट्केन च युक्ते पञ्चदशकं~(१५)~जातम्~। ततः शिष्टेषु त्रिकेषूपान्त्येन पञ्चदशकेन गुणिते रूपात्मके तदुपरिफलेऽन्त्येन नवकेन युक्ते चतुर्विंशतिः~(२४)~जाता~। एवं वल्ल्युपसंहारः कृतः~। अथात्र पदत्र्याभावादधउपरिगुणितमन्त्ययुगित्येतन्न प्रवर्तत इत्यत्राधोराशिः{\textsuperscript{१}} प्रमृष्टः~।\\ 
अथ चतुर्विंशति \textendash

\footnotetext{
१~क.  शिर प्रकृतः~।}

\newpage
\thispagestyle{fancy}
\fancyhf{}
\chead{\textbf{कुट्टाकारशिरोमणिः~।}}
\rhead{\textbf{७}}
\noindent
(२४)~संख्याकोपरिराशावूनाग्रच्छेदेनाष्टादशकेन~(१८)~भक्ते षट्कं~(६)~शिष्यते । अस्मिन्नधिकाग्रच्छेदेनैकोनत्रिंशता~(२९)~ गुणितेऽधिकाग्रेण सप्तकेन~(७)~च युक्त एकाशीत्युत्तरं शतम्~(१८१)~। अयं द्विच्छेदाग्रः प्रथमाे भाज्यराशिः अयमष्टादशभि~(१८)~र्विभक्त एकाग्र एकोनत्रिंशता~२९~सप्ताग्रः~७~। अथान्येऽपि भाज्यराशय उदाह्रियन्ते~। अत्रोनाग्रभक्तशेषं षट्कम्~६~। अस्मिन्नेकगुणितोनाग्रभागहारेणाष्टादशकेन~१८~युक्ते चतुर्विंशति~२४~र्जाता~।
अस्यामधिकाग्रच्छेदेनैकोनत्रिंशता~२९~गुणितायामधिकाग्रेण सप्तकेन~७~च युक्तायां त्र्यधिकं शतसप्तकं~७०३~जातम्~। अयं चात्रान्यो भाज्यराशिः~। अथोनाग्रच्छेदभक्तशेषे~६~द्विगुणेनोनाग्रच्छेदेन~३६~युक्ते~४२~अधिकाग्रच्छेदं~२९~गुणयित्वा तत्राधिकाग्रं सप्तक~७~युक्तं पञ्चविंशत्युत्तरशतद्वयाधिकसहस्र~१२२५~जातम्~। अयं चापरो भाज्यराशिः~। एतौ चानन्तरौ भाज्यराशी अष्टादशभिः~१८~एकाग्रौ~। एकोनत्रिंशता ~२९~सप्ताग्रौ~। एवं प्रकारेण भाज्यराशिसहस्रमानेतव्यम्~॥~८~॥\\
\indent
अथ प्रश्नविशेष उदाह्रियते~।\\ 
उद्देशकः \textendash
\begin{quote}
{\ku
आकृत्यै~(२२)~कादशाग्रा~(११)~ये सप्ताग्रा~(७)~मनुभि~(१४)~र्हृताः~।\\
आसन्ना राशयोऽष्टौ तानाचक्ष्व मतिमद्वर~॥~२~॥~९~॥}
\end{quote}

\indent
अत्र पूर्वोक्तप्रक्रियया रूपात्मिकायाः प्रथमाया मतेराश्रयणेनाऽऽनीताः प्रथमतृतीयपञ्चमा विषमा भाज्यराशयः क्रमेण लिख्यन्ते~। तत्र प्रथमः सप्तसप्तति~(७७)~संख्यः~। तृतीयः पञ्चाशीत्युत्तरशतत्रय~(३८५)~संख्यः~। पञ्चमः सप्तोनशतसप्तक~(६९३)~संख्यः~। सप्तमः सैकसहस्र~(१००१)~संख्यः~। अत्र द्वितीयादिसमभाज्यराशयः प्रथममत्यवलम्बनेन सेत्स्यन्तीति तदनन्तरां चतुष्टयात्मिकां मतिमवलम्ब्य पूर्वोक्तप्रक्रिययाऽऽनीतो द्वितीयो भाज्यराशिरेकत्रिंशदुत्तरशतद्वयः~२३१संख्यः~। चतुर्थ एकोनचत्वारिंशदधिकपञ्चशत~(५३९)~संख्यः~। षष्ठः सप्तचत्वारिंशदुत्तरशताष्ट~(८४७)~संख्यः~। अष्टमः पञ्चपञ्चाशदुत्तरैकादशशतः~(११५५)~संख्यः~। एवमेतेऽष्टापि
राशयो द्वाविंशत्या~(२२)~विभक्ता एकादश (११)~शेषा भवन्ति~। चतुर्दशभिः सप्त शेषाः~। एवमेवंविधप्रश्नविषये भाज्यराशय आनेतव्याः~॥~९~॥\\

\indent
अथ यत्रोनाग्रभागहारेणाधिकाग्रभागहारः शुध्यति तत्रोदाह्रियते~।\\
उद्देशकः \textendash
\begin{quote}
{\ku
यस्त्र्यग्रः सप्तभिर्भक्तः सप्तत्या दशकाग्रकः~।\\
जानासि यदि तं राशिं ब्रूहि सांवत्सराग्रणीः~॥~३~॥~१०~॥}
\end{quote}

\newpage
\thispagestyle{fancy}
\fancyhf{}
\chead{\textbf{महालक्ष्मीमुक्तावलीसहितः\textendash\\ }}
\lhead{\textbf{८}}
\indent
अत्रोनाग्रभागहारेण सप्तकेनाधिकाग्रभागहारः सप्ततिसंख्यः शुध्यति~। तस्मादत्र सप्ततिसंख्याकेऽधिकाग्रभागहारेऽधिकाग्रेण दशकेन युक्तेऽशीति~(८०)~र्जाता~। अयमत्र प्रथमो भाज्यराशिः~। अयं सप्तभिर्भक्तस्त्र्यग्रो भवति~। सप्तत्या दशकाग्रः~। अथ द्विगुणेऽधिकाग्रभागहारे ~(१४०)~। अधिकाग्रेण~(१०)~युक्ते पञ्चाशदुत्तरशतं~(१५०)~जातम्~। अयमत्र द्वितीयो भाज्यराशिः~। एवं भाज्यराशिसहस्रमानेतव्यम्~॥~१०~॥\\
\indent
अथ शेषपरस्परभक्तमित्यत्र फलपङ्क्तिरेका भवति~। तदा शेषयोरप्येकः
शुध्यति~। तत्रोदाह्रियते~।\\
उद्देशकः\textendash
\begin{quote}
{\ku
यो दिग्भि~(१०)~र्विहतस्त्र्यग्रो रूपाग्रस्त्रिभिरुद्धृतः~।\\
तं राशिं शीघ्रमाचक्ष्व वेत्सि चेद्गणकोत्तम~॥~४~॥~११~॥}
\end{quote}

\indent
अत्रोनाग्रभागहारेण त्रिकेणाधिकाग्रभागहारे दशके हृत उपरिराशिरेकसंख्यः~। अधोराशिस्त्रि~(३)~संख्यः~। अत्रोपरिराशिनाऽधोराशिः शुध्यति~। फलपङ्क्तिप्येका भवति~। अतोऽत्राऽऽदावेव मतिः कल्प्यते~। अयं रूपात्मक उपरिराशिः केन गुणितोऽग्रान्तरेण द्विके~(२)~ न च युक्तः स्वाधोराशिना त्रिकेण शुध्यतीति लब्धा प्रत्यासन्ना मती रूपात्मिका~(१)~। अनया गुणितेऽधिकाग्रभागहारे दशकसंख्ये~(१०)~अधिकाग्रेण त्रिकेण~(३)~च युक्ते त्रयोदशकं~(१३)~जातम्~। अयमत्र प्रथमो भाज्यराशिः~। अयं दशभिस्त्र्यग्रः त्रिभिरेकाग्रः~। अथात्र रूपात्मिकायां मतावेकगुणोनाग्रभागहारे त्रिके दत्ते चतुष्कं~(४)~जातम्~। अस्मिन्नधिकाग्रभागहारेण दशकेन गुणितेऽधिकाग्रेण त्रिकेण च युक्ते त्रिचत्वारिंश~(४३)~ज्जाता~। अयमत्र द्वितीयो भाज्यराशिः~। एष च दशभिस्त्र्यग्रः~। त्रिभेरकाग्रः~। एवं भाज्यसहस्रमानेयन्~॥~ ११~॥\\

\indent
अथात्र प्रश्नविशेष उदाह्रियते~।\\
उद्देशकः \textendash
\begin{quote}
{\ku
ये राशयोऽष्टभक्ताश्चेच्चतुरग्रा भवन्त्यपि~।\\
अष्टाग्रा द्वादशाप्ताश्चेत्तेष्वाद्यांश्चतुरो वद~॥~५~॥~१२~॥}
\end{quote}

\indent
अत्राऽऽदावेव कल्पितायां रूपात्मिकायां मत्यामधिकाग्रभागहारेण द्वादशकेन गुणितायामधिकाग्रेणाष्टकेन युतायां विंशति~(२०)~ र्जाता~। अयं प्रथमो भाज्यः~। अथ द्वितीयया रूपत्रयात्मिकया मत्या पूर्ववदानीतो द्वितीयो

\footnotetext{
१~ख. त्रिकेणैकाग्रः~।}

\newpage
\thispagestyle{fancy}
\fancyhf{}
\chead{\textbf{कुट्टाकारशिरोमणिः~।}}
\rhead{\textbf{९}}
\noindent
भाज्यश्चतुश्चत्वारिंश~( ४४ )~त्संख्यः~। अथ तृतीयया रूपपञ्चकरूपया मत्या
पूर्ववदानीतस्तृतीयो भाज्यराशिरष्टषष्टि~( ६८ )~संख्यः~। अथ चतुर्थ्या रूपसप्तकरूपया मत्या पूर्ववदानीतश्चतुर्थो भाज्यराशिर्द्विनवति~( ९२ )~संख्यः~। एवमेते चत्वारो भाज्यराशयोऽष्टाभिश्चतुरग्रा द्वादशभिरष्टाग्राः~॥~१२~॥\\
\indent
अथ यत्र ''शेषपरस्परभक्तम् '' इत्यत्र फलपङ्क्ती द्वे भवतस्तदा शेषयोरप्येकः शुध्यति~। तत्रोदाह्रियते \textendash\\
उद्देशकः\textendash
\begin{quote}
{\ku
अष्टाभिश्चतुरग्रो यो द्व्यग्रोऽष्टादशभिर्हृतः~।\\
क्षिप्रमाख्याहि तं राशिं साग्रे यद्यस्ति संस्तवः~॥~१३~॥}
\end{quote}

\indent
अत्रोनाग्रच्छेदेनाष्टादशकेनाधिकाग्रच्छेदोऽष्टसंख्यश्छेत्तुं न शक्यते~।
तस्मादत्र यथास्थितयाेरेवानयाः परस्परभजनं कार्यमित्युपरिराशिनाऽष्टकनाधाेराशावष्टादशके विहृते द्वयं शिष्यते~। तेनाष्टसंख्यः स्वोपरिराशिः शुध्यति~। फलपङ्क्ती च द्वे भवतः~। तदत्राऽऽदावेव प्रत्यासन्नां मतिं
कल्पयित्वा पूर्ववदानीतः प्रथमो भाज्यराशिर्विंशति~(२०)~संख्यः~। अयमष्टभिश्चतुरग्रः~।
अष्टादशाभिर्द्व्यग्रः~। एवमेवंविधा भाज्याः सा\textsuperscript{१}\ ध्याः~॥~१३~॥\\

\indent
अथ द्विविधबहुभागहारवद्भाज्यराशिप्रश्नविषय उदाह्रियते \textendash\\
उद्देशकः \textendash
\begin{quote}
{\ku
प्रकृतेस्तिसृभि~(३)~र्भक्तो यो राशिः स्याद्विशे\textsuperscript{२}\ षतः~।\\
चतुर्दशाग्रः संस्कृत्या~(२४)~तिसृभिस्तं वदाऽऽशु मे~॥~७~॥~१४~॥}
\end{quote}

\indent
एषां द्विविधानां षण्णां हराणां सकाशादपवर्त्य हरौ द्वौ द्वौ~। तद्वधमपवर्तकेन संगुणयेदित्युक्तप्रक्रियया पूर्वमूनाग्रच्छेदोऽधिकाग्रछे\textsuperscript{३}\ दश्चाऽऽनीयते~। तद्यथा एकविंशतेर्द्वाविंशतेश्चापवर्तकाभावादनयोराहतिः पक्षरसार्णव
~(४६२)~संख्या~। अथास्यास्त्रयोविंशतेरप्यपवर्तकाभावादनयोः \textsuperscript{४}\ संवर्गः षडक्षितर्कदिक्~(१०६२६)~संख्यः~ अत्रायमूनाग्रच्छेदः~।  अथ चतुर्विशतेः पञ्चर्विंशतेश्चापवर्तका स्वादनयोराहतिः पू\textsuperscript{५}र्णखतर्कः~(६००~संख्या)~। अथास्याः षड्विंशतेश्चापवर्तको द्वि ~(२)~ संख्यः~।अनेनापवर्तितयोरेतयोर्वधः खाभ्राङ्कराम~(३९००)~संख्यः~। अयमपवर्तकेन द्विकेन गुणितः पूर्णाभ्रवसुमुनि~(७८००)~संख्यः~। अत्रायमधिकाग्रच्छेदः~। अथैवमाग \textendash

\footnotetext{
१~क. स्थाप्याः~।~~२~ख. षकः~।~~३~क. दस्याऽऽ~।~~ ४~क. संसर्गः~।~~५~ख. र्णाभ्रख~।}
       
\newpage
\thispagestyle{fancy}
\fancyhf{}
\chead{\textbf{महालक्ष्मीमुक्तावलीसहितः\textendash}}
\lhead{\textbf{१०}}
\noindent
ताभ्यामाभ्यामूनाग्रच्छेदाधिकाग्रच्छेदाभ्यां द्वादशसंख्याकेनाग्रान्तरेण च
पूर्ववदधिकाग्रभागहारमित्येवमादिनाऽऽनीतः पञ्चमो भाज्यराशिर्मनुवेदमुनिकृतरामरवि
~(१२३४७४१४)~संख्यः~। अयमेकविंशत्या द्वाविंशत्या त्रयोविंशत्या च विभक्तो द्व्यग्रः~। चतुर्विंशत्या पञ्चविंशत्या षड्विंशत्या च चतुर्दशाग्रः~। अत्राप्येवंविधा भाज्यराशयः पूर्ववदानेयाः~॥~१४~॥\\
\indent
अथ प्रश्नान्तरमुदाह्रियते\textendash \\
उद्देशकः\textendash 
\begin{quote}
{\ku
नवत्या सैकयैकाग्रो यो विंशत्या निरग्रकः~\\
द्रुतमाचक्ष्व दैवज्ञ तं राशिं मम पृच्छतः~॥~८~॥~१५~॥}
\end{quote}

\indent
अत्रैकनवतिसंख्येनाधिकाग्रच्छेदेन विंशतिसंख्येनोनाग्रच्छेदेन रूपात्मकेनाग्रान्तरेण च प्राग्वदानीत आद्यो भाज्यराशिर्नखाष्ट~( ८२० )~संख्यः~। अयमेकनवत्या रूपाग्रः, विंशत्या शून्याग्रः~। \textsuperscript{१}अतोऽन्येऽप्येवंविधा भाज्यराशयः पूर्ववदानेयाः~॥~१५~॥

\indent
अथ भिन्नबहुभागहारभिन्नबह्वग्रसाग्रकुट्टाकारविषयमार्याद्वयमाह \textendash \\
\begin{quote}
\qt
\hspace{1cm}
\textsuperscript{*} भिन्नबहुभागहारे कुट्टाकारे विभिन्नबह्वग्रे~।\\
\indent
\hspace{1cm}
हरयोर्द्वयोर्द्वयोर्यो भाज्यः सोऽग्रं भवेद्धरस्तस्य~॥\\
\indent
\hspace{1cm}
हरयोः संवर्गः स्यादेवं कार्यं पुनः पुनस्तावत्~।\\
\indent
\hspace{1cm}
यावद्विभाजकौ द्वौ भवतो भाज्योऽत्र पूर्ववत्साध्यः~॥\\
\end{quote}

\indent
अस्यार्थ उदाहरणेन व्यक्ती भविष्यति\textendash \\
उद्देशकः\textendash 
\begin{quote}
{\ku सप्तादिभिस्त्रिभिर्भक्ते यस्मिन्रूपशराब्धयः~।\\
शेषाः क्रमाद्भवेयुस्तं राशिमाचक्ष्व तान्त्रिक ~॥९~॥~१६~॥}
\end{quote}

\footnotetext{
\textsuperscript{*}लीलावतीग्रन्थे क्वचिदेते श्लोका दृश्यन्ते~। ते च प्रकृतोपयुक्तत्वादत्रोद्ध्रियन्ते\textendash\\
\begin{quote}
\qt हारे विभिन्ने गुणके च भिन्ने स्यादाद्यराशेर्गुणकस्तु साध्य~।\\
द्वितीयभाज्यघ्नतदाद्यजो गुणः क्षेपो भवेत्क्षेपयुतो द्वितीये~॥\\
द्वितीयभाज्यघ्नतदाद्यहारो भाज्यो भवेत्तत्र हरो हरः स्यात्~।\\
एवं प्रकल्प्यापि च कुट्टकेऽथ जातो गुणश्चाऽऽग्रहरेण निध्नः~।।\\
गुणो भवेदाद्यगुणेन युक्तो हरघ्नहारोऽत्र हरः प्रदिष्टः~।\\
अथ तृतीयेऽपि तथैव कुर्यादेवं बहूनामपि साधयेत्तु~॥\\
\end{quote}

\rule{0.5\linewidth}{0.7pt}\\

१~ख. अत्रा}


\newpage
\thispagestyle{fancy}
\fancyhf{}
\chead{\textbf{कुट्टाकारशिरोमणिः~।}}
\rhead{\textbf{११}}

\indent
अत्र क्रमादिमे हारोः~७,~८,~९~। अग्राणीमानि~१,~५ ,~४~एष्वाद्याभ्यां हराभ्यामग्नान्तरेण चोक्तवदानीतो भाज्यराशिरेकोनत्रिंशत्संख्यः~(२९)~। अयं सप्तभिर्भक्तो रूपाग्रः~(१)~। अष्टभिः पञ्चाग्रः~(५)~। अथायमेव भाज्यस्तृतीयेन सह कर्मणि कर्तव्ये 'हरयोर्द्वयोर्द्वयोर्यो भाज्यः सोऽग्रं भवेत्' इत्यग्रं भवति~। अस्याग्रस्य च्छेदयोर्वधः ' तस्य हरयोः संवर्गो हरः स्यात् ' इति च्छेदो भवति~। स चात्र षट्पञ्चाशत्संख्यः~(५६)~। अस्याग्रमेकोनत्रिंशत्संख्यम्~(२९)~। भाज्यराशिस्तृतीयच्छेदोनदश~(९)~संख्यः~। अस्याग्रं वेद~(४)~संख्यम्~। अग्रान्तरं पञ्चविंशति~(२५)~संख्यम्~। अथाऽऽद्याभ्यां छेदाभ्यामनेनाग्रान्तरेण चोक्तवदानीतस्त्रिच्छेदाग्रो भाज्यराशिः पञ्चाशीति~(८५)~संख्यः~। अयं सप्तभिर्भक्त एकाग्रः~(१)~। अष्टभिः पञ्चाग्रः~(५)~। नवभिश्चतुरग्रः~(४)~एवमेवंविधप्रश्नविषये भाज्यराशय आनेतव्याः~॥~१६~॥\\
\indent
अथात्रैव प्रश्नविशेष उदाह्रियते \textendash\\
उद्देशकः\textendash
\begin{quote}
{\ku को राशिः कैर्हरौर्भिन्नैर्भक्तो भिन्नावशेषकः~।\\
स एवाग्रैक्ययुक्तश्चेत्तैर्हरैरेव शुध्यति~॥~१०~॥~१७~॥ }
\end{quote}
\indent
अत्राऽऽदौ स्वेच्छयांऽग्राणि कल्पयित्वा तद्वशाद्वक्ष्यमाणप्रकारेण तेषां तेषां छेदानुत्पाद्यानन्तरप्रदर्शितभिन्नबहुभागहारभिन्नबह्वग्रभाज्यराश्यानयनप्रकारेण भाज्यराशिमानयेत्~।\\
\indent
अत्रेष्टाग्रवशात्तच्छेदानयनार्थमिदमभियुक्तप्रणीतं सूत्रम्\textendash\\
\hspace{1cm}\begin{quote}
{\qt
अग्राणीष्टानि स्युस्तद्योगोऽग्रान्वितः पृथक्छेदाः~।\\
अग्रैक्ययुते भाज्ये वियुते त्वग्रैक्यमग्रोनम्~॥}
\end{quote}
\indent
अस्यार्थः\textendash\ अग्राणि शेषाणीष्टानि स्युः~। अभिप्रेतानि कल्प्यानि भवेयुः~। तद्योगोऽग्रयोगः पृथगग्रान्वितः पृथक्पृथक्तदग्रयुतः सन् छेदास्तत्तदग्रवन्तश्छेदाः स्युः~। अग्रैक्ययुते भाज्ये भाज्यराशौ तत्तच्छेदानां निरग्रदानार्थंमग्रयोगेन युक्ते सत्येतद्वेदितव्यम्~। वियुते तु भाज्युराशौ तत्तच्छेदानां निरग्रदानार्थंमग्रयोगेन हीने सति पुनः 'अग्रैक्यमग्रोनम्' अग्रयोगः पृथक्पृथक् तत्तदग्रवियुतः संस्तत्तदग्रवन्तीश्छेदाः स्युः~।

\indent
उदाहरणम्\textendash
\indent\begin{quote}
{\ku
अत्र कल्पितान्यग्राणीमानि त्रीणि~(३)~। चत्वारि~(४)~। पञ्च }\
\end{quote}


\newpage
\thispagestyle{fancy}
\fancyhf{}
\chead{\textbf{महालक्ष्मीमुक्तावलीसहितः\textendash}}
\lhead{\textbf{१२}}
\noindent
(५)~। एषां योगो रवि~१२~संख्यः~। एष त्रेधा निहितः~१२,~१२,~१२~। एषु पूर्ववत्कल्पितेष्वग्रेषु~३\textendash४\textendash५~क्षिप्तेषु जाताश्छेदाः पञ्चदश~(१५)~। षाेडश~(१६)~सप्तदश~(१७)~। एषां क्रमेणाग्राणि पूर्वलिखितान्येव~३\textendash४\textendash५~।एभिश्छेदैरेतैरग्रैश्चानन्तरप्रदर्शितभिन्नबहुभागहारभिन्नबह्वग्रभाज्यराश्यानयनप्रकारेणाऽऽनीतो भाज्यराशिर्वसुरसंपूर्णाब्धिः~(४०६८)~। एष पञ्चदशभिर्भक्तस्त्र्यग्रः~(३)~ षोडशभिश्चतुरग्रः~(४)~सप्तदशभिः पञ्चाग्रः~(५)~। अथायमेव भाज्यराशिरग्रैक्येन द्वादशकेन~(१२)~युक्तश्चेदेभिश्छेदैर्निरग्रश्च भवति~। एवमेवंविधा भाज्यराशयः साध्याः~॥~१७~॥\\
\indent
अथा\textsuperscript{१}ग्रैक्यहीनं भाज्यानयनमुदाह्रियते\textendash\\
उद्देशकः\textendash
\begin{quote}
{\ku को राशिः कैर्हरैर्भिन्नैर्भक्तो भिन्नावशेषकः~।\\
स एवाग्रैक्यहीनश्चेरेत्तैर्हरैरेव शुध्यति~॥~११~॥~१८~॥}
\end{quote}

\indent
अत्र पूर्ववत्प्रथमं स्वेच्छयाऽग्राणि परिकल्प्य तद्वशात् 'वियुते त्वग्रैक्यमग्रोनम् ' इत्युक्तप्रकारेण तेषां छेदानुत्पाद्य भिन्नबहुहारभिन्नबह्वग्रभाज्यानयनप्रकारेण भाज्यमानयेत्~। अत्र पूर्ववत्कल्पितान्यग्राणीमानि पञ्च~(५)~सप्त~(७)~नव~(९)~। एषां योगश्चन्द्रकर~(२१)~संख्यः~। अयं त्रिधा निहितः~२१\textendash २१\textendash २१~। एभ्य एष्वग्रेषु~५\textendash\ ७\textendash९~त्यक्तेषु जाताश्छेदाः षोडश~(१६)~चतुर्दश~(१४)~द्वादश~(१२)~। एषां क्रमेणाग्राण्यनन्तरप्रदर्शितान्येव ५\textendash\ ७\textendash ९~। एभिरग्रैरेतैच्छेदैश्च भिन्नबहुभागहारभिन्नबह्वग्रभाज्यराश्यानयनप्रकारेणाऽऽनीतो भाज्यराशिः शररसरामेन्दु~(१३६५)~संख्यः~। अयं षोडशभिः पञ्चाग्रः~। चतुर्दशभिः सप्ताग्रः~। द्वादशभिर्नवाग्रः~५\textendash\ ७\textendash९~। अयमेव भाज्यराशिरग्रयोगेन कुपक्ष~(२१)~संख्येन हीनश्चेदेभिश्छेदैर्निरग्रो भवति~। एवमेवं प्रकारा भाज्यराशय आनेतव्याः~। नन्वन्तरयोः प्रश्नयोः ''स एवाग्रैक्ययुक्तश्चेत्तैर्हरैरेव शुध्यति' 'स एवाग्रैक्यहीनश्चेतैर्हरैरेव शुध्यति इति च निरग्रता प्रतीयते~। तदनुपपन्नमनयोः
साग्राधिकारे कथनम्~। सत्यम्~। 'को राशिः कैर्हरैर्भिन्नैर्भक्तो भिन्नावशेषकः' इति साग्रताऽपि दृश्यत इति युक्तमुक्तम् \textendash\ अथ कक्ष्याकुट्टाकारः प्रदर्श्यते ।
तत्राऽऽदौ कक्ष्याग्रहानयनं ज्ञातव्यम्~। अतस्तदुच्यते\textendash

\footnotetext{
१~ख.~ग्रैक्यादीनां भा~।}


\newpage
\thispagestyle{fancy}
\fancyhf{}
\chead{\textbf{कुट्टाकारशिरोमणिः~।}}
\rhead{\textbf{१३}}
\indent
\textsuperscript{*}खव्योमखत्रयखसागरषट्कनागव्योमाष्टशून्ययमरूपनगाष्टचन्द्राः~(१८७१२०८०\textendash\ ८६४००००००)~इति भगवता सूर्येणोपदिष्टयाऽऽकाशकक्ष्यया गुणिते स्वेष्टदिनगणे सहस्राहतेन ग्रहकक्ष्याभूदिनसंवर्गेण विभक्ते लब्धं भगणादि ग्रहो भवैत्~। शेषं कक्ष्याग्रम्~। इष्टदिनानां स्वकक्ष्यागुणाकारः सहस्रगुणो ग्रहकक्ष्याभूदिनसंवर्गो भागहारः~। एतावपवर्तिताविष्टदिनानां गुणहरौ भवतः~। एतस्मिन्पक्षेऽपवर्तितयाऽऽकाशकक्ष्यया निहते स्वाभिमतदिनगणेऽपवर्तितेन
सहस्रनिहतग्रहकक्ष्याभूदिनाभ्यासेन विहृते लब्धं भगणादिग्रहः~। शेषं कक्ष्याग्रम्~। अथवा
खकक्ष्याभूदिनयोरपवर्तितयोर्यत्खकक्ष्यातो लब्धं तदिष्टदिनानां गुणः~। यत्तु भूदिनतो लब्धं तेन सहस्रगुणितेन गुणिता ग्रहकक्ष्या तेषां भागहारः~। अस्मिन्पक्षे पुनरिष्टदिनगणेऽपवर्तितया खकक्ष्यया गुणिते सहस्रगुणितेनापवर्तितभूदिनेन गुणितया ग्रहकक्ष्यया विभक्ते लब्धं भगणादिर्मध्यग्रहः~। शेषं कक्ष्याग्रम्~। एवं त्रिभिः प्रकारैः कक्ष्याग्रहानयनं कक्ष्याग्रं चोक्तम~। एष्वन्तिमप्रकारे यत्कक्ष्याग्रं तदेवोद्दिश्यते~। अस्य भागहारः सहस्रघ्नेनापवर्तितभूदिनेन गुणिता ग्रहकक्ष्या~। \\
उद्देशकः \textendash
\begin{quote}
{\ku कक्ष्याग्रं सवितुः खखाक्षिशिखरिक्ष्माक्ष्यद्रिचन्दाकृति\\
द्व्यर्थाब्ध्यब्धिरसाभ्रतर्ककरभूसंख्यं~१२६०\textendash\\
६४४५२२२१७२१७२००~मृगाङ्कस्य तु~।\\
लक्षघ्नाभ्रकरार्थबाणनवकद्व्यष्टेशनागर्त्वगा\\
७६८११८२९५५२०००००~द्व्यग्रं भाज्यमहर्गणं\\
च भगणान्यातान्रवीन्द्वोर्वद~॥~१८~॥}
\end{quote}
\indent
न्यासः\textendash\ रवेः कक्ष्याग्रमिदम्~(१२६०६४४५२२२१७२१७२००)~ इदमत्राधिकाग्रम्~। अथ स्व (ख) कक्ष्याभूदिनयोः 'विभजेद्धारविभाज्यौ ' इति परस्परभक्तयोः शेषं वेद~(४)~संख्यम्~। एष स्व (ख) कक्ष्याभूदिनयोरपवर्तकः~। अनेनापवर्तिता खकक्ष्या प्रयुतघ्ननृपद्व्यभ्रद्विखाष्टाद्रिरसाब्धि~(४६७८०२०२१६००००००)~संख्या~। अपवर्तितं भूदिनमचलार्थकृताङ्काद्रिवेदाब्धिनवपावक~(३९४४७९४५७)~संख्यम्~।एतत्सहस्रगुणितम्~(३९४४७९४५७०००)~। अथानपवर्तितायां खकक्ष्यायां सहस्रघ्नैर्यु\textendash

\footnotetext{
\textsuperscript{*}(सूर्यसिद्धान्ते भूगोलाध्याये श्लोक~९०~)}

\newpage
\thispagestyle{fancy}
\fancyhf{}
\chead{\textbf{महालक्ष्मीमुक्तावलीसहितः\textendash}}
\lhead{\textbf{१४}}
\noindent
गग्रहभगणैर्विभक्तायां योजनात्मिका ग्रहकक्ष्या भवतीति लब्धा रविकक्ष्यां
द्वादशशयोजनषष्ट्यंशसहितखखतिथिरामाग्निवेद~(४३३१५०० $\dfrac{\hbox{१२}}{\hbox{६०}}$)~ संख्या~।
एषा पूर्वलिखितसहस्राहताऽपवर्तितभूदिनाभ्यस्ता व्योमाभ्रशक्रनवविश्वनवाष्टतर्क
वेदाष्टशैलगजषड्वसुखाद्रिचन्द्र~(१७०८६८७८४६८९१३९१४००)~संख्या~।अयमत्राधिकाग्रभागहारः~। चन्द्रस्य कक्ष्याग्रमिदम् ~(७६८११८२९५५२००००००)~।
एतदिहोनाग्रं प्रागुक्तक्रमलब्धमेव~। चन्द्रकक्ष्या खत्रयाब्धिद्विदहन~(३२४०००)~संख्या~। एषा सहस्रघ्नाऽपवर्तितभूदिनगुणिता प्रयुतघ्नाष्टषट्खाब्धिवेदाग्नीशाष्टाघ~(~गा~)~र्क 
~(१२७८११३४४०६८००००००)~संख्या~। अयमत्रोनाग्रभागहारः~। अग्रान्तरं खखाक्ष्यत्यष्टिद्वीषुरसषट्द्व्यङ्कर्तुद्विरामाष्टत्रिगजेश~(११८३८३२६९२६६५२१७२००) संख्यम्~। अत्रैतेऽधिकाग्रभागहारोनाग्रभागहाराग्रान्तरराशयोऽपवर्तनं प्रयच्छन्तीत्यपवर्त्य प्रदर्श्यन्ते~। अत्रापवर्तकः पूर्णाभ्रतर्काकृतिखाग्नितर्कव्योमाभ्रचन्द्रगिरि
~(७१००६३०२२६००)~संख्यः~। अनेनापवर्तितोऽधिकाग्रभागहारो नवाष्टत्रिषट्खसिद्ध~(२४०६३८९)~संख्यः~। अपवर्तितोनाग्रभागहारोऽप्ययुतगुणधृति~(१८००००) संख्यः~। अपवर्तितताग्रान्तरं द्व्याकृतिमुनिरसनृप~(१६६७२२२)~संख्यम्~। एभिरधिकाग्रच्छेदगुणमित्येतदन्तेन कर्मकदम्बेन जातो राशिर्वसुमुनिगिरिरविगजाब्धि~(४८१२७७८)~संख्यः~। अयमत्र केनापवर्तकेन गुणितोऽधिकाग्रयुक्तश्चार्बुदाहताष्टिद्विनखखाष्टाद्रिरससिन्धु
~(४६७८०२०२१६००००००००)~संख्यः~। अयं द्विच्छेदाग्रो भाज्यराशिः~।अस्मिन्नपवर्तितखकक्ष्यया गुणेन विभक्ते लब्धो राशिः सहस्र~( १००० )~संख्यः~। एष स्वेष्टदिनगणः~। अथ द्विच्छेदाग्रे भाज्यराशावनपवर्तितेनाधिकाग्रभागहारेण द्विच्छेदेन ह~(हृ)~ते लब्धौ गतौ रविभगणौ द्वौ~(२)~। शिष्टमुद्दिष्टरविकक्ष्याग्रतुल्यम् । अथ द्विच्छेदाग्रे भाज्यराशावेवानपवर्तितोनाग्रभागहारेण चन्द्रच्छेदेन विहृते लब्धा गताश्चन्द्रभगणाः षट्त्रिंशत्~(३६)~। शेषमुद्दिष्टरविकक्ष्याग्रसमम्~। एवं द्वयोः कक्ष्याग्रयोरुद्दिष्टयोः सतोरानयप्रकारः~। बहूनां कक्ष्याग्रेषूद्दिष्टेषु भिन्नबहुभागहारभिन्नबह्वग्रयोः कुट्टाकारानयनप्रकारेण भाज्यराश्यहर्गणगत\textendash

\newpage
\thispagestyle{fancy}
\fancyhf{}
\chead{\textbf{कुट्टाकारशिरोमणिः~।}}
\rhead{\textbf{१५}}
\noindent
भगणाः साध्याः~। एवं सूर्यसिद्धान्तमार्गमवलम्ब्य कक्ष्याकुट्टाकारानयनं प्रदर्शितम्~। आर्यभटीयानुसारेण कक्ष्याकुट्टाकारानयने तु विशेषः~। अत्र सूर्यसिद्धान्तानयनमार्गेणैतदानयने यत्र यत्र सहस्रगुणनमुक्तं तत्तदिह न कर्तंव्यम्~। अन्या सर्वाऽपि प्रक्रिया समाना~॥~१८~॥\\
\indent
अथ खिलविशेषानुपपन्नप्रश्न उदाह्रियते \textendash\\
उद्देशकः \textendash
\begin{quote}
{\ku योऽष्टाभिर्नवभिः षड्भिर्दशभिश्चतुरग्रकः~।\\
द्व्याद्यैश्चतुर्भिरेकाग्रस्तं राशिं व्याहराञ्जसा~॥~२३~॥~१९~॥}
\end{quote}

\indent
अत्र षड्भिरष्टाभिर्वा यो राशिश्चतुरग्रो भवति स कथं चतुर्भिर्द्वाभ्यां वा रूपाग्रो भवतीत्यालोच्य प्रश्नोऽयमनुपपन्न इति ब्रूयात्~। एवमन्येऽप्यनुपपन्नाः प्रश्ना बोधव्याः~॥~१९~॥\\
\indent
अथ खिलविषय उदाह्रियते \textendash\\
उद्देशकः\textendash
\begin{quote}
{\ku द्विचत्वारिंशता द्व्यग्रः षड्विंशत्यैकशेषकः~।\\
यो राशिस्तं द्रुतं ब्रूहि साग्रं वेद भवान्यदि~॥~२४~॥~२०~॥}
\end{quote}

\indent
अत्र पूर्ववदागतानि प्रत्येकं रूपात्मकान्युपर्यधोभावेन निहितानि फलानि~१,~१,~१,~१~। शेषयोरुपरिराशिर्लोचन~(२)~संख्यः~। अधोराशिर्वेद~(४)~संख्यः~। अत्र मतिः कल्पयितुं न शक्यते~। तस्मादिह भाज्यराशिः खिल इति वक्तव्यम्~। यत्र च्छेदयोरपवर्तकोऽस्ति सोऽग्रान्तरस्य तु नास्ति तत्र भाज्यराशेः खिलत्वं द्रष्टव्यम्~। यत्रैषां त्रयाणामपि सद्दशोऽपवर्तकोऽस्ति तदभावो वाऽत्राखिलत्वं द्रष्टव्यम्~। एवं साग्रविषयेऽधिकाग्रादिसूत्रयुगं प्रतिपत्तॄणां बुद्धिव्यामोहो मा भुदित्येकयैव च रीत्या व्याख्यातम्~। अथैतदेवात्र किंचिद्विकल्प्य व्याक्रियते\textendash\ 'पूर्वं शेषपरस्परभक्तं मतिगुणमग्रान्तरे क्षिप्तम्' इत्यत्र फलपङ्क्त्यां समायां सत्यामग्रान्तरं समं कृत्वा मतिं कल्पयित्वा तथाऽल्पमुपरिराशिं गुणयित्वा तच्चाग्रान्तरे क्षिप्त्वा तदिधोराशिना हत्वा लब्धं समपङ्क्तीनां फलानामधो निहिताया मतेरधस्तन्निदध्यादित्युक्तम्~। इदानीं तत्र विषमायां फलपङ्क्त्यामग्रान्तरमृणं कृत्वा मतिं परिकल्प्य तयाऽल्पमधोराशिं संगुण्य तस्मादग्रान्तरं विशोध्य तदुपरिराशिना विभज्य लब्धं विषमपङ्क्तीनां फलानां मध्ये दत्ताया मतेरधः स्थापयेदित्युच्यते \textendash\ अन्यत्सर्वं पूर्ववत्~। एवं कृतेऽपि पूर्वप्रदर्शितमखिलमुदाहरणजातं प्रायशः सेत्स्यत्येव~।

\newpage
\thispagestyle{fancy}
\fancyhf{}
\chead{\textbf{महालक्ष्मीमुक्तावलीसहितः\textendash}}
\lhead{\textbf{१६}}
\noindent
अस्मिन्पक्षे 'अग्रान्तरे' इत्यन्तं विभक्तिविपरिणामो द्रष्टव्यः~। अग्रान्तरेणेति~। छन्दोवत्सूत्राणि भवन्तीत्युक्तत्वाच्छन्दसि हि विभक्तिविपरिणामः प्रसिद्धः~। क्षिप्तमित्येतदवहीनमित्यस्योपलक्षणम्~। तस्मादग्रान्तरेणोनमिति सिद्धं भवति~। तथैव वाऽस्तु पाठः~।
\indent
\hspace{1cm}{\begin{quote}
\qt
'ऊनाग्रभागहारं छिन्द्यादधिकाग्रभागहारेण~।\\
शेषपरस्परभक्तं मतिगुणमग्रान्तरे क्षिप्तम्~॥\\
अधउपरिगुणितम\textsuperscript{१} त्ययुगधिकाग्रच्छेदभाजिते शेषम्~।\\
ऊनाग्रच्छेदगुणं द्विच्छेदाग्रं विहीनशेषयुतम्~॥~इति~।
\end{quote}}
\indent
अस्मिन्पाठे यदा समायां फलपङ्क्त्या मतिकल्पनेष्टा तदाऽग्रान्तरमृणं कृत्वा मतिं कल्पयेत्~। यदा विषमायां तस्यां मतिकल्पनाऽभिमता तदाऽग्रान्तरं धनं कृत्वा मतिं कल्पयेत्~। एवमधिकाग्रादिसूत्रयुगलं साग्रमधिकृत्य विस्तरेण व्याख्यातम्~॥~२०~॥\\
\indent
अथैतत्सूत्रयुगलस्य विषयमार्यया दर्शयति \textendash
\begin{quote}
{\ks अवशेषविषमभावे शेषस्यैकस्य शून्यतायां च~।\\
अवतरति सूत्रयुगलकमिदमन्यत्र तु निगद्यते कर्म~॥~९~॥~२१~॥}
\end{quote}
\indent
इदं सूत्रयुगलकमवशेषविषमभावकस्य शेषस्य शून्यतायां चावतरति~। अन्यत्र तु कर्म निगद्यते~। यत्राग्रयोरुभयोर्बहूनां वाऽभावः समत्वं वा तत्र कर्मोच्यत इत्यर्थः\textendash\ सूत्रयुगलकमित्यत्र स्वार्थे कप्रत्ययः~॥~२१~॥\\
\indent
तदिदं कर्माऽऽर्ययाऽऽह \textendash\\
\begin{quote}
{\ks शेषाभावे हारावपवर्त्यामू परस्परं हत्वा~।\\
अपवर्तकेन गुणयेच्छेषसमत्वे तु शेषमपि युञ्ज्यात्~॥~१०~॥~२२~॥}
\end{quote}
अस्यास्तूदाहरणमेव व्याख्यानम्~॥~२२~॥\\

\indent
\centering
\textbf{इति श्रीदेवराजविरचिते कुट्टाकारशिरोमणौ साग्रपरिच्छेदः प्रथमः~॥~१~॥\\}
\rule{0.2\linewidth}{1.0pt}\\

\vspace{1cm}
\textbf{अथ द्वितीयः परिच्छेदः~।\\}

\justifying
\indent
उद्देशकः \textendash\begin{quote}
{\ks द्व्याद्यैर्दशकपर्यन्तैर्यो राशिः शून्यशेषकः~।\\
तं राशिमभिधेहि द्राक् तान्त्रिकग्रामणीर्मम~॥~२५~॥~२३~॥}
\end{quote}

\footnotetext{
१~क. मन्स्ययु....ग्रः~।}

\newpage
\thispagestyle{fancy}
\fancyhf{}
\chead{\textbf{कुट्टाकारशिरोमणिः~।}}
\rhead{\textbf{१७}}
\indent
अत्र द्रव्यादिषु नवसु च्छेदेषु द्विसंख्याकस्यास्य त्रिसंख्यस्य द्वितीयस्य चापवर्तकाभावादनयोः संवर्गः षट्~(६)~संख्यः~। अयमादितो द्वयोच्छेदयोर्निरग्रो भाज्यराशिः~। अथास्य च चतुःसंख्याकस्य तृतीयस्य चापवर्तको द्वि~(२)~संख्यः~। अनेनापवर्तितयोः क्रमात् त्रिद्विसंख्याकयो~(३, २)~ रेतयोर्वधोऽपि षट्~(६)~संख्यः~। अयमपवर्तकेन द्विकेन गुणितः सूर्य~(१२)~ संख्यः~। अयमादितस्त्रयाणां निरग्राे भाज्यः~। अथास्य च पञ्चसंख्यस्य चतुर्थस्य चापवर्तकाभावादनयोर्घातः षष्टि~(६०)~संख्यः~। अयं चतुर्णां निरग्रो भाज्यः~। अयास्य च षट्संख्यस्य पञ्चमस्य चापवर्तकः षट् (६) संख्यः~। अनेनापवर्तितयोः क्रमाद्दशैक~(१०\textendash\ १)~संख्ययोरेतयोरभ्यासो दशक~(१०)~संख्यः~। अयमपवर्तकेन षट्केन गुणितः षष्टि~(६०)~संख्यः~। अयमादितः पञ्चानां छेदानां निरग्रो भाज्यराशिः~। अथास्य च सप्तसंख्यस्य षष्ठस्य चापवर्तकाभावादनयोर्वधो नखवेद~(४२०)~संख्यः~। अयं षण्णां निरग्रो भाज्यः~। अथास्य चाष्टसंख्यस्य सप्तमस्य चापवर्तकश्चतुः~(४)~संख्यः ~। अनेनापवर्तितयोरेतयोरभ्यासो दिङ्नेत्र~(२१०)~संख्यः~। अयमपवर्तकेन चतुष्केण गुणितः खाब्धिगज~(८४०)~संख्यः~। अयं सप्तानां भेदानां निरग्रो भाज्यः~। अथास्य च नवसंख्याकस्य चाष्टमस्य चापवर्तकस्त्रि~(३)~संख्यः~। अनेनापवर्तितयोरेतयोः संवर्गश्च खाब्धिगजसंख्यः~(८४०)~। अयमपवर्तकेन त्रिकेण गुणितो नखतत्त्व~(२५२०)~संख्यः~। अयमष्टानां हाराणां निरग्रो भाज्यराशिः~। अयास्य च दशसंख्यस्य नवमस्य भागहारस्य चापवर्तको दशक~(१०)~संख्यः~। अनेनापवर्तितयोेरेतयोरभ्यासः पक्षतत्त्व ~(२५२)~संख्यः~। अयमपवर्तकेन दशकेन निहतः पूर्ववन्नखतत्त्व~(२५२०)~संख्यः~। अयं द्विपूर्वाणां दशपर्यन्तानां नवानां भागहाराणां निरग्रः प्रत्यासान्नो भाज्यराशिः~॥~ २३~॥\\
\indent
अथ शेषसाम्यविषय उद्देशकः \textendash
\begin{quote}
{\ku द्व्याद्यैर्दशकपर्यन्तैर्नवाभिर्भाजकैर्हृतः~।\\
यो भवेद्राशिरेकाग्रस्तमाचक्ष्वाविलम्बितम्~॥~२४~॥~}
\end{quote}

\indent
अत्र पूर्ववदानीतो नखतत्त्व~(२५२०)~संख्यो भाज्यराशिरेकेनाग्रेण युक्तः~(२५२१)~प्रत्यासन्नो भाज्यराशिः~। एवमनेन न्यायेन सूर्येन्दुजीवभौमशनैश्चराणां बुधभृगुशीघ्रोच्चयोश्चन्द्रोच्चपातयोश्च द्वियुगाद्यानयनं कर्त \textendash 

\newpage
\thispagestyle{fancy}
\fancyhf{}
\chead{\textbf{महालक्ष्मीमुक्तावलीसहितः\textendash }}
\lhead{\textbf{१८}}
\noindent
व्यम्~। उदाहरणं तु ग्रन्थविस्तरभयान्न प्रदर्श्यते~। अत्र च \textendash\ पूर्वेण पूर्वेण गतेन युक्तं स्थानं विनाऽन्त्यं प्रवदन्ति संख्याः~। इच्छाविकल्पैः क्रमशोऽभिनीय नीचे निवृत्तिः पुनरन्यनीतिः~(२३)~।। बृहत्संहितायामध्याये~(७६)~श्लोकः~(२२)~वराहमिहिरः । इत्याचार्योक्तप्रकारेणैषामेकयुगविषयाः प्रश्ना नव~(९)~। द्वियुगविषयाः षट्त्रिंत~(३६)~। त्रियुगविषयाश्चचुरशीतिः~(८४)~। चतुर्युगविषया रसनेत्रचन्द्र~(१२६)~संख्याः~। पञ्चयुगवियाश्च तावन्तः~(१२६)~। षड्युगविषयाश्चतुरशीतिः~(८४)~। सप्तयुगविषयाः षट्त्रिंशत्~(३६)~। अष्टयुगविषया नव~(९)~। नवयुगविषयः प्रश्न एकः~(१)~। एवं संभूय रुद्रबाण~(५११)~संख्याः प्रश्नाः संभवन्ति~। अत्र रविचन्द्रभौमजीवशनैश्चराणां बुधकाव्यशीघ्रोच्चयोश्चन्द्रोच्चपातयोश्च युगैः शून्याग्रो राशिर्वसुद्व्यष्टादिरूपाङ्कसप्ताद्रितिथि~(१५७७९१७८२८)~संख्याः~। आर्यभटीये तु व्योमशून्यशराद्रीन्दुरन्धाद्र्यद्रिशरेन्दु~(१५७७९१७५००)~संख्यः~। एवं साग्रकुट्टाकारो दर्शितः~॥२४॥\\

\centering
\textbf{इत्यत्रिकुलाभरणस्य स्कन्धत्रयवेदिनः सिद्धान्तवल्लभ इति प्रसिद्धापरनाम्नः श्रीवरदराजाचार्यस्य तनयेन देवराजेन विरचितायां कुट्टाकारशिरोमणिटीकायां महालक्ष्मीमुक्तावल्यां साग्रपरिच्छेद प्रथमः~॥~१~॥\\}
\rule{0.2\linewidth}{1.0pt}\\

\vspace{1cm}
\justifying
\indent
\textbf{अथ निरग्रपरिच्छेदो द्वितीयो व्याख्यायते~। \\}
तत्राऽऽदौ निरग्रविषयं प्रश्नमार्यया प्रकटयति\textendash

\begin{quote}
{\ks हारकभाज्यक्षेपैरुद्दिष्टैर्गुणफले य आख्याति~।\\
स निरग्रविदां श्रेष्ठो दैवज्ञानां महीतले भवति~॥~१~॥}
\end{quote}

\indent
य उद्दिष्टैर्भागहारकभाज्यक्षेपैर्गुणफले आख्याति स महीतले निरग्रविदां दैवज्ञानां मध्ये श्रेष्ठो भवति~। तत्रेदं प्रश्नशरीरम् \textendash\ उद्दिष्टोऽयं भाज्यराशिः केन गु\textsuperscript {१}णकारेण गुणितः, उद्दिष्टेनानेन संख्यान्तरेण वियुक्तः संयुक्तो वा उद्दिष्टेनानेन भागहारेण विभक्तः शुध्येत्तादृशः को वाऽत्र गु\textsuperscript {२}णकारः किंवा फलमिति । एतच्चोत्तरत्रोदाहरणेन व्यक्ती भविष्यति~॥~१~॥\\
\indent
अथ तत्प्रश्नभङ्गायाऽऽर्यामाह\textendash
\begin{quote}
{\ks कुट्टकसूत्रद्वितयं संयोज्य निरग्रकुट्टपद्धत्याम्~।\\
तत्रत्यैः कर्मचयैर्विद्वान् विदधीत गुणकफले~॥~२~॥}
\end{quote}

\footnotetext{
१~ख.गुणाकां~~ २~ ख. गुणांका}

\newpage
\thispagestyle{fancy}
\fancyhf{}
\chead{\textbf{कुट्टाकाराशिरोमणिः~।}}
\rhead{\textbf{१९}}
\indent
अत्र विद्वच्छेब्देन तान्त्रिक उच्यते~। कुट्टकसूत्रद्वितयं पूर्वलिखितं कुट्टाकारयुग्मविषयमार्यभटीयस्थमधिकाग्राद्यार्यांसूत्रयुगलं निरग्रकुट्टपद्धत्यां निरग्रकुट्टाकारमार्गे संयोज्य सम्पग्योजपित्वा तत्रत्यैर्निरग्रकुट्टाकारगणितसंबन्धिभिः कर्मचयैः कर्मसमूहैर्गुणकफले गुणकश्च फलं च गुणकफले ते अत्र विदधीत कुर्यात्~। अत्र यद्यपि सर्वं सूत्रद्वयं नोपयुज्यते तथाऽपि प्राचुर्यात्तदनादृत्य कुट्टकसूत्रद्वितीयमित्युक्तमित्यवगन्तव्यम्~॥~२~॥\\

\indent
अथाधिकाग्रादिसूत्रं प्रस्तुतप्रश्नभङ्गार्थं निरग्रमधिकृत्य व्याक्रियते~। अधिकाग्रभागहारम्~। अत्राग्रराब्देन संख्याऽभिधीयते~। भागहारशब्देन चात्र भागहारभाज्यराश्योः परस्परं भाजकत्वात्तौ द्वावपि गृह्येते~। अत्र ग्रहैकत्ववदेकत्वमविवक्षितम्~। अतोऽयमर्थः संपन्नः~। अधिकसंख्याकौ हरभाज्यराशी इति~। ऊनाग्रभागहारेण च्छिन्द्यात्~। अत्राप्यग्रशब्दः संख्यावचनः~। ऊनतंख्येन भागहारेण विभजेत्~। अपवर्तकेन संभवे सत्यपवर्तयेदित्यर्थः~। अथायं कथमपवर्तको ज्ञायते तदानयनमन्यत्रोक्तम्\textendash
\hspace{1cm}
\begin{quote}
{\qt
विभजेत हरविभाज्यौ परस्परं यावदेति संशुद्धिम्~।\\
एकस्थयोस्तदपरश्लेदो हरभाज्ययोर्भवति~॥~इति~।}
\end{quote}
\indent
अत्र च्छेदशब्दोऽपवर्तकपर्यायः~। इह येनापवर्तकेन हरभाज्याावपवर्त्येते तेन ऋणात्मकस्य वा धनात्मकस्य वा क्षेपस्यापवर्तनप्यर्थसिद्धमिति न कण्ठोक्तम्~। यत्र हरभाज्ययोरपवर्तनमप्यस्ति क्षेपस्य तु तन्नास्ति तदुदाहरणं खिलं विद्यात्~। उक्तं च \textendash
{\begin{quote}
\qt
"भाज्यहरप्रक्षेपान् सदृशच्छेदेन संभवे छिन्द्यात्~।\\
स चेद्विभाज्यहरयोश्छेदो न क्षेपकस्य च खिलं तत्~॥~इति~।
\end{quote}}
\indent
एवमुदाहरणस्याऽऽदौ सम्यक्खिलत्वं विज्ञायापवर्तनस्य संभवे सति हरभाज्यक्षेपानपवर्तयेत्~। अत्रेदमनुसंधेयम्~। अपवर्तनेऽप्यनपवर्तनेऽपि भाज्ये भाजकादूनेऽधिकेऽप्यविशेषेण सर्वदा भाज्यराशेरधो भाजकराशिं विन्यसेदिति~। शेषपरस्परभक्तं मतिगुणमित्येतद्द्वयमपि भाज्ये भागहारादूने सति पूर्ववत्~। अधिके तु विषमायां फलपङ्क्तौ सत्यामस्य मतिः कल्प्या~। अत्रापि मतिकल्पनायामेवमालापः~। अयमुपरिराशिः केन गुणित उद्दिष्टेन संख्याविशेषेण युक्तो वियुक्तो वाऽनेन स्वाधोराशिना शुध्यतीति~। अग्रान्तरे क्षिप्तम्~। अत्राप्यग्रशब्दः संख्यावचनः~। अन्तरशब्दो विशेषवचनः~।~(न)~विवरवचनः~।

\newpage
\thispagestyle{fancy}
\fancyhf{}
\chead{\textbf{महालक्ष्मीमुक्तावलीसहितः\textendash}}
\lhead{\textbf{२०}}
\noindent
अत्र मतिगुणितं प्रकरणं तदुद्दिष्टे धनात्मके रूपात्मके संख्याविशेषे क्षिप्तं कार्यम्~। एतदुपलक्षणम्~। तदृणक्षेपेण शोध्यमित्यस्य(र्थः)~। अधउपरिगुणितमन्त्ययुगिति वल्ल्युपसंहारः पूर्ववत् । ऊनाग्रच्छेदभाजिते शेषम्~। अत्राप्यग्रशब्देन संख्योच्यते~। छेदशब्देन पूर्ववद् हरभाज्यावभिधीयेते~। शेषस्य चैकत्वमविवक्षितम्~। संभवे सत्यूनसंख्याभ्यामपवर्तिताभ्यां हरभाज्याभ्यां क्रमादुपर्यधोराश्योर्विभक्तयोः शेषे कुट्टाकारफले भवतः~। अन्विषयमाणः प्रत्यासन्नो गुणकारस्तत्र फलं च लब्धं भवतीत्यर्थः~। अत्र भाज्ये भागहारादधिके सति त्वपवर्तिताभ्यां हरभाज्याभ्यां क्रमादधउपरिराश्योर्विभक्तयोः शेषे गुणफले भवतः~।अधिकाग्रच्छेदगुणमित्याद्यत्र नोपयुज्यते~।।~२~।।\\
\indent
अत्रायं संग्रहश्लोकः\textendash
\begin{quote}
{\ks कुट्टाकारे निरग्रे समधिकवपुषि च्छेदतो भाज्यराशौ\\
लब्धे युग्मे मतिः स्यादुपरि फलमधो व\textsuperscript{१}र्धकः स्याद्विहीने~।\\
युग्मे लब्धे मतिः स्यादुपरि तु गुणकोऽधः फलं स्याद्दृढेन\\
च्छेदेनाऽऽप्ते गुणेऽस्मिन्गुणक इह फले भाज्यभक्ते फलं स्यात्~।।~३~।।}
\end{quote}

\indent
अस्यार्थः\textendash\ निरग्रकुट्टाकारे भाज्यराशौ भागहारादधिके सति तत्र शेषपरस्परभक्तमिति लब्धानां फलानां पङ्क्तौ विषमायां सत्यां मतिः कल्प्या~। तदा तत्र वल्ल्युपसंहारं कृतेऽधोराशिर्गुणः~। उपरिराशिः फलम्~। अथ भाज्ये छेदादूने सति तत्र फलपङ्क्त्यां समायां मतिः कल्प्या । तदोपरिराशिर्गुणः~। अधोराशिः फलम्~। अत्रोभयत्र सति संभवे गुणेऽपवर्तितेन हारेण विभक्ते शेषो गुणः फले चापवर्तितेन भाज्येन विभक्ते शेषं फलं भवतीति~। इदं सर्वमुदाहरणेन व्यक्ती भविष्यति~। अत्रेदमनुसंधेयम्~। भाज्ये भागहारादूनेऽधिके च यत्र शेषपरस्परभक्तमित्यत्र फलपङ्क्तिरेका भवति तदा शेषयोश्चैकः शुध्यति~। भाज्ये भागहारादूने सत्येव शेषपरस्परभक्तमित्यत्र फलपङ्क्ती द्वे भवतः~। तदा शेषयोरप्येकः शुध्यति~। तत्रोभयत्राऽऽदावेव प्रत्यासन्नां मतिं कल्पयित्वा तया भाज्यं गुणयित्वा स्वाधःस्थेन च्छेदेन विभजेत्~। लब्धं फलं भवति~। मतिरेव गुणः~।।~३~।।

\footnotetext{
१~ख. बद्धकः~।}

\newpage
\thispagestyle{fancy}
\fancyhf{}
\chead{\textbf{कुट्टाकारशिरोमणिः~।}}
\rhead{\textbf{२१}}
\indent
\textbf{अथ खिलविषय उदाह्रियते \textendash}\\
उद्देशकः \textendash
\begin{quote}
{\ku खाब्ध्यग्नयो हताः केन युताः पञ्चभिराहृताः~।\\
पूर्णार्थमार्गणैः शुद्धाः शीघ्रं गुणफले वद~॥~४~॥}
\end{quote}
\indent
अत्र भाज्यः रववेदराम~(३४०)~संख्य~। छेदोऽत्र खभूतरशर~(५५०)~संख्यः~। धनक्षेपः शर~(५)~संख्यः~। एषु भाज्यभाजकयोः परस्परभक्तयोः शेषो दिक्~(१०)~संख्य~। अयमनयोर्हरभाज्ययोरपवर्तकः क्षेपस्त्वनेन भक्तो न शुध्यति~। तस्मादुद्दिष्टोदाहरणं खिलमिति वक्तव्यम्~॥~४~॥\\
अत्र भाज्ये भागहारादूने तत्रैवानपवर्तनविषय उदाह्रियते \textendash\\
	उद्देशकः \textendash 
\begin{quote}
{\ku चन्द्रेशाः~(१११)~केन गुणिता रूप~(१)~युक्त द्विखाग्निभिः~(३०२)~।\\
भक्ताः शुद्धा गुणफले ब्रूहि तूर्णं निरग्रवित्~॥~५~॥}
\end{quote}
\indent
अत्रापवर्तनं नास्ति~। भाज्योऽपि भाजकाद्धीनः~। अतस्तत्रोक्तप्रक्रिययाऽऽनीतः प्रत्यासन्नो गुणः शरधृति~(१८५)~संख्यः~। तत्राऽऽगतं फलं गजतर्क~(६८)~संख्यम्~।अथैवमागतयोरनयोर्गुणफलयोः क्रमादेकद्वित्र्यादिसंख्यागुणितौ हरभाज्यौ क्षिप्त्वा गुणफलसहस्रमानयेत्~। तद्यथा-पूर्वमागते शरधृति~(१८५)~संख्ये गुणके रूपगुणे द्विख्याग्नि~(३०३)~संख्ये भाजके क्षिप्ते मुनिवसुकृत~(४८७)~संख्याको गुणो भवित~। एवं पूर्वमागते~।गजतर्क~(६८)~संख्ये फले रूपगुणे चन्द्ररुद्र~(१११)~संख्ये भाज्ये दत्ते नवात्यष्टि~(१७९)~संख्यमन्यत्फलं भवति~। एवमन्यान्यपि गुणफलान्यानयेत्~॥~५~॥\\
अथात्रैव क्षेपमृणं परिकल्प्योदाह्रियते\textendash\\ 
उद्देशकः \textendash 
\begin{quote}
{\ku चन्द्रेशाः केन गुणिता रूपहीना द्विखाग्निभिः~।\\
भक्ताः शुद्धा गुणफले ब्रूहि तूर्णं निरग्रवित्~॥~६~॥}
\end{quote}
अत्रोक्तवदानीतो गुणो मुनिरुद्र(११७ )संख्यः~। फलं रामवेद(४३)संख्यम्~।अथानयोर्गुणफलयोः क्रमादेकद्वित्र्यादीष्टसंख्यागुणितौ हरभाज्यौ क्षिप्त्वा गुणफलसहस्रमानयेत्~॥~६~॥\\
\indent
\textbf{अथात्रापवर्तनविषय उदाह्रियते\textendash}\\
उद्देशकः\textendash 
\begin{quote}
{\ku खाब्ध्यग्नयो~(३४०)~हता केन युता दशाभि~(१०)~राहृताः~।\\
पूर्णार्थमार्गणैः~(५५०)~शुद्धा शीघ्रं गुणफले वद~॥~७~॥}\\
\end{quote}
अत्र भाज्यभाजकयोः परस्परभक्तयोः शेषो दिक्~(१०)~संख्यः~।

\newpage
\thispagestyle{fancy}
\fancyhf{}
\chead{\textbf{महालक्ष्मीमुक्तावलीसहितः\textendash}}
\lhead{\textbf{२२}}
\noindent
अपमत्रापवर्तकः~। अनेन हरभाज्यक्षेपानपवर्त्योक्तवदानीतो गुणः प्रकृति~(२१)~संख्यः~। फलं त्रयोदशसंख्यम्~। अथानयोः क्रमादिष्टसंख्यागुणितावपवर्तितौ हरभाज्यौ क्षिप्त्वा गुणफलशतमानयेत्~॥~७~॥\\
\indent
\textbf{अथात्र क्षेपमृणं कल्पयित्वोदाह्रियते\textendash}\\
उद्देशकः\textendash
\begin{quote}
{\ku खाबध्यग्नयो हताः केन हीना दशभिराहृताः~।\\
पूर्णार्थभार्गणैः शुद्धाः शीघ्रं गुणफले वद~॥~८~॥}
\end{quote}
\indent
अत्रोक्तवदानीतो गुणो वेदाग्नि~(३४)~संख्यः~। फलं प्रकृतिसंख्यम्~(२१)~। अथानयोः क्रमादिष्टगुणावपवर्तितौ हरभाज्यौ दत्त्वा गुणफलशतमानयेत्~॥~८~॥\\
\indent
अथ भाज्ये भाजकादधिके तत्राप्यनपवर्तनविषय उदाह्रियेत \textendash\\ 
उद्देशकः\textendash 
\begin{quote}
{\ku द्विखाग्नयो~३०२~हताः केन रूपयुक्ता हरेन्दुभिः~(१११)~।\\
भक्ताः शुद्धा गुणफले ब्रूहि तूर्णं निरग्रवित्~॥~९~॥}
\end{quote}

\indent
अत्रोक्तप्रकिययाऽऽनीतो गुणो रामवेदसंख्यः~(४३)~। फलं मुनिरुद्रः~(११७)~संख्यम्~। अथैतयोः क्रमादिष्टगुणौ हरभाज्यौ क्षिप्ता गुणफलशतमानषेत्~॥~९~॥\\
\indent
\textbf{अथात्र क्षेषमृणं परिकल्प्योदाह्रियते\textendash} \\
उद्देशकः\textendash 
\begin{quote}
{\ku द्विखाग्नयो हताः केन रूपहीना हरेन्दुभिः~।\\
भक्ताः शुद्धा गुणफले ब्रूहि तूणे निरग्रवित्~॥~१०~॥}
\end{quote}
\indent
अत्राऽऽगतो गुणो गजतर्क~(६८)~संख्यः~। फलं शरधृति~(१८५)~संख्यम्~। अथानयोः क्रमादिष्टगुणौ हरभाज्यौ क्षिप्त्वा गुणफलसहस्रमानयेत्~॥~१०~॥\\
\indent
अथात्रापवर्तनविषय उदाह्रियते\textendash 
उद्देशकः\textendash 

\begin{quote}
{\ku खार्थेषवो~(५५०)~हताः केन युता दशभि~(१०)~राहृताः~।\\
खवेदपुष्करैः~(३४०)~शुद्धाः शीघ्रं गुणफले वद~॥~११~॥}
\end{quote}

\indent
अत्राऽऽगतो गुण प्रकृति~(२१)~संख्यः~। फलं कृताग्नि~(३४)~संख्यम्~। अथानयोः क्रमादिष्टगुणावपवर्तितौ हरभाज्यौ क्षिप्त्वा गुणफलसहस्रमानयेत्~॥~१०~॥\\
\indent
\textbf{अथात्र क्षेपमृणं कल्पयित्वोदह्रियते\textendash}\\

\newpage
\thispagestyle{fancy}
\fancyhf{}
\chead{\textbf{कुट्टाकारशिरोमणिः~।}}
\rhead{\textbf{२३}}
उद्देशकः\textendash
\begin{quote}
{\ku खार्थेषवो हताः केन हीना दशभिराहृताः~।\\
खवेदपुष्करैः शुद्धाः शीघ्रं गुणफले वद~॥~१२~॥}
\end{quote}

\indent
अत्राऽऽगतो गुणो रामभू~(१३)~संख्यः~। फलं प्रकृति~(२१)~संख्यम्~। अथानयोः क्रमादिष्टगुणापवर्तितौ हरभाज्यौ दत्त्वा गुणफलसहस्रमानयेत्~॥~१२~॥\\
\indent
\textbf{अथात्र त्रैराशिकमार्गः प्रदर्श्यते\textendash}\\
उद्देशकः\textendash
\begin{quote}
{\ku खार्थेषवो हताः केन हीना दशभिराहृताः~।\\
खवेदपुष्करैः शुद्धाः शीघ्रं गुणफले वद~॥~१३~॥}
\end{quote}
\indent
अत्रापवर्तको दशक~(१०)~संख्यः~। अनेनापवर्तितो भाज्यराशिः पञ्चपञ्चाश(५५)~त्संख्यः~।भाजकश्चतुस्त्रिंश~(३४)~त्संख्यः~। ऋणक्षेपः पञ्ञ~(५)~संख्यः~।एभिरधिकाग्रभागहारमित्येवमादिनाऽऽनीतो गुण एकत्रिंश~(३१)~त्संख्यः~। फलं पञ्चाश~(५०)~त्संख्यम्~। एते गुणफले त्रैराशिकेनापि सिध्यतः~। तद्यथा\textendash खार्थेषवो हताः केन हीना दशभिराहिताः~। स्ववेदपुष्करैः शुद्धा इत्यत्रोदाहरणेऽनन्तरोक्तोदाहरणे चापवर्त्य दृढीकृतौ हरभाज्यौ समानौ ऋणरूपावपवर्त्य दृढीकृतौ क्षेपौ तु क्रमादनयो रूप~१~भूत~५~संख्यौ~। अत्रोदाहरणद्वयेऽपि दृढयोर्हरभाज्ययोर्वैषम्याभावात्रैराशिकमवतरति~। यदि रूप~(१)~संख्येन दृढेन क्षयक्षेपेण क्रमा\textsuperscript{१}द् द्विख~(कु)~१२~प्रकृति~२१~संख्ये गुणफले लब्धे पञ्चसंख्येन~५~दृढेनानेन क्षयक्षेपेण कियतीत्येवं त्रैराशिकेन लब्धो गुणः पञ्चषष्टि~(६५)~संख्यः~। फलं पञ्चोत्तरशत~(१०५)~संख्यम्~। एतयोः क्रमाद्दृढाभ्यां चतुस्त्रिंशत्~(३४)~पञ्चपञ्चाश~(५५)~त्संख्याभ्यां विभक्तयोः शिष्टे गुणफले एकत्रिंशत्~(३१)~पञ्चपञ्चाश~(५०)~त्संख्ये~। एवमत्र त्रैराशिकेनाऽऽनीतयोर्गुणफलयोरधिकाग्रादिसूत्रानीतयोश्च वैषम्यं नास्ति~। एवं धनक्षेपविषयेऽनपवर्तनविषये च त्रैराशिकसरणिरुन्नेया~। अत्रदेमनुसंधेयम्~। अयं त्रैराशिकमार्गो भाज्ये भागहारादूनेऽधिके वाऽपवर्तनविषयेऽनपवर्तनविषये वा सर्वदा समान इति~। एवं निरग्रविषये तावदिदमधिकाग्रादिसूत्रयुगं प्रतिपत्तॄणां बुद्धिक्षोभो मा भूदित्येकयैव रीत्या व्याकृतम्~। अथैतदेवात्र किंचिद्विकल्प्य व्याख्यायते~। पूर्वं

\footnotetext{
१~ ख. माल्लोक~(१३)~प्र.}

\newpage
\thispagestyle{fancy}
\fancyhf{}
\chead{\textbf{महालक्ष्मीमुक्तावलीसहितः\textendash}}
\lhead{\textbf{२४}}
\noindent
यत्रोदाहरणे भाज्यो भागहारादूनसंख्यस्तत्र 'शेषपरस्परभक्तं मतिगुणमग्रान्तर क्षिप्तम्' इत्यत्र फलपङ्क्यां समायां सत्यां मतिः कल्प्या~। मतिगुणितोपरिराशिरुद्दिष्ठक्षेपेण च यथास्वसंस्कृतः स्वाधोराशिना विभजनीय इत्युक्तम्~। इदानीं तत्फलपङ्क्त्यां विषमायां मतिः कल्पनीया~। मतिगुणितोऽल्पोऽधोराशिरुद्दिष्टेन
क्षेपेणापि वामं स्वसंस्कृतस्वोपरिराशिना विभजनीय इत्युच्यते~। यत्र चोदाहरणे भाज्यो भागहारादधिकस्तत्र फलपङ्क्त्यां विषमायां मतिः कल्पयितव्या~। मतिनिहतोऽल्प उपरिराशिरुद्दिष्टेन क्षेपेण च यथास्वसंस्कृतः स्वाधोराशिना विभाज्य इति पूर्वमुक्तम्~। इदानीं त्वत्र फलपङ्क्तौ समायां मतिः कल्प्या~।
मतिगुणितोऽल्पोऽधोराशिरुद्दिष्टेन क्षेपेण च वामं संस्कृतः स्वोर्ध्वराशिना हरणीय इतुच्यते~। अन्यत्सर्वं पूर्ववत्~। एवं कृतेऽपि पूर्वप्रदर्शितमखिलमुदाहरणजातं प्रायशः सिध्यत्येव~। एवमधिकाग्रादिसूत्रद्वयं निरग्रमधिकृत्य समासतो व्याख्यातम्~॥~१३~॥\\
\indent
\textbf{अथ संश्लिष्टकुट्टाकारास्रयः~।} तेषु यत्रैको भागहारस्तदग्राणि च भिन्नरूपाणि भिन्नरूपा भाज्याश्चोद्दिश्यन्ते स प्रथमः \textendash\\
अत्रोद्देशकः\textendash\begin{quote}
{\ku रुद्रा~११~विश्वे~१३~सायकाः~५~शैलवेदाः~४७\\
केनाभ्यस्तावर्जितास्ते क्रमेण~।\\
रूपै~१~रीशैः~११~सप्तभिः~७~शीतभासाः ~१\\
धृत्या~१८~भक्ताः शुद्धिमाप्ता वदेस्तम्~॥~१४~॥}
\end{quote}
\indent
अत्र पूर्ववदानीतो गुणः पञ्चसंख्यः~।।~१४~।।\\
\indent
अथ यत्रैको भाज्यो भिन्नरूपा भागहारास्तदग्राणि च तादृशान्युद्दिश्यन्ते स द्वितीयः\textendash\\
अत्रोद्देशकः\textendash
\begin{quote}
{\ku केनाभ्यस्ता रामबाणाश्चतुर्धा~५३\textendash५३\textendash५३\textendash५३\\
वेदै~४~रर्कै~१२~वासवै~१४~रक्षि~१०२~दिग्भिः~।\\
हीना भूपैः~१६~खार्णवैः~४०~षट्कतर्कैः~६६\\
खेशैः~११०~र्भक्ताः शुद्धिमाप्ता गुणः कः~॥~१५~॥}
\end{quote}
अत्र पूर्ववदानीतो गुणो वेद~(४)~संख्यः~॥~१५~॥\\
अथ यत्र भाज्या भागहाराश्च भिन्नात्मकास्ताद्दृशानि तदग्राणिचोदि.... स तृतीय\textendash\\
अत्रोद्देशकः\textendash

\newpage
\thispagestyle{fancy}
\fancyhf{}
\chead{\textbf{कुट्टाकारशिराेमणिः~।}}
\rhead{\textbf{२५}}
\begin{quote}
{\ku चन्द्रामराः~३३१~शरशिवा~११५~मुनिरामसूर्याः~१२३७\\
केनाऽऽहताः शिखिसुरै~३३३~र्नव~९~भिस्रिवेदैः~४३~।\\
हीना रसाग्निमनुभी~१४३६~रविभि~१२~र्गजशेः~११८\\
शुद्धाः क्रमेणं वदं तं गुणमाशु विद्वन्~॥~१६~॥}
\end{quote}
\indent
अत्रत्यो गुणस्तूत्तरत्र प्रदर्शयिष्यते~॥~१६~॥\\
\indent
एषु प्रथमद्वितीययोः संश्लिष्टकुट्टाकारयाेरेकस्य योऽधिकाग्रादिसूत्रलब्धो गुणः स सर्वेषां समान इति न तयोः किंचिदपि वक्तव्यमस्ति~। अतस्तौ विहाय तृतीयविषय आर्ये आह\textendash
\begin{quote}
{\ks भिन्नहरभाज्यकुट्टे सर्वेषां पूर्ववदगुणाः साध्याः~।\\
भाज्योऽल्पगुणहरः स्याद्धारोऽन्यहरो गुणान्तरमृणाग्रम्~॥१७॥\\
तज्जगुणभाज्यघातः साल्पगुणः स्फुटगुणोऽस्य हरघातः~।\\
हर इह ताभ्यां परगुणहाराभ्यां गुणक उक्तवत्साध्यः~॥~१८~॥}
\end{quote}
\indent
अनयोरर्थस्तूदाहरणमुखेन व्यक्ति भविष्यति~। अत्रत्यः प्रश्नश्लोकः पूर्वमेव लिखितः~। एतच्छ्लोके त्रयो भाज्याः~। तेषां क्रमाद् भागहाराश्च त्रयः ~। ऋणरूपाण्यग्राणि च क्रमेण त्रीणि~। एषु प्रथमाे भाज्यश्चन्द्रामर~(३३१)~संख्यः~। अस्य च्छेदो रसाग्निमनु~(१४३६)~संख्यः~। अत्र ऋणाग्रं शिखिसुर~(३३३)~संख्यम्~। एभिः पूर्ववदानीतः प्रथमो गुणस्र्यष्टाक्षि~(२८३)~संख्यः~। अथ द्वितीयो भाज्यः शरशिव~(११५)~संख्यः~। अस्य च्छेदो रवि~(१२)~संख्यः~। अत्र ऋणाग्रं नव~(९)~संख्यम्~। एभिः पूर्ववदानीतो गुणस्त्रि~(३)~संख्यः~। अथ तृतीयो भाज्यो मुनिरामसूर्य~(१२३७)~संख्यः~। अस्य हारो गजेश~(११८)~संख्यः~। इह ऋणाग्रं त्रिवेद ~(४३)~संख्यम्~। एभिः पूर्ववदानीतो गुणः सप्तरस~(६७)~संख्यः~। अथ प्रथमद्वितीययोर्भाज्यराश्योः  साधारणाे गुण आनीयते~। प्रथमगुणस्र्यष्टाक्षि~(२८३)~संख्यः~। द्वितीयगुणस्त्रि~(३)~संख्यः~। अत्र द्वितीयगुणस्याल्पत्वाद्भाज्येऽल्पगुणहरत्वं स्यादित्यल्पगुण उत्पाद्यमाने यश्छेदः सोऽत्र भाज्यो भवेदिति~। द्वितीयगुण उत्पाद्यमाने यश्छेदः सोऽत्र भाज्यो भवेत्~। स च रवि~(१२)~संख्यः~। 'हारोऽन्यहरः' इत्यधिकगुण उत्पाद्यमाने यश्छेदः स भाजको भवेदित्यत्राधिके प्रथमगुण उत्पाद्यमाने यश्छेदः स भाजको भवेत्~। स च रसाग्निमनु~(१४३६)~संख्यः~। गुणान्तरमृणाग्रमिति प्रथमद्वितीययो\textendash

\newpage
\thispagestyle{fancy}
\fancyhf{}
\chead{\textbf{महालक्ष्मीमुक्तावलीसहितः\textendash}}
\lhead{\textbf{२६}}
\noindent
गुणयोर्विवरमृणक्षेपो भवेत्~। स च खाष्टाक्षि~(२८०)~संख्यः~ एवमागतैर्भाज्यभाजकक्षेपैर्निरग्रकुट्टाकारप्रक्रियया यो गुण उत्पादितो भवति स तज्जगुणः~। स चात्र त्रिमनु~(१४३)~संख्यः~। तस्य पूर्वप्रदर्शितभाज्यस्य च रवि (१२)~संख्यस्य संवर्गो नृपात्यष्टि~(१७१६)~ संख्यः~। एषोऽत्र तज्जगुणभाज्यघातः~। असौ 'साल्पगुणः' इत्यत्रत्येन त्रिसंख्येनाल्पेन गुणेन युक्तश्छिद्ररूपमुनीन्दु~(१७१९)~संख्यः~। एष प्रथमद्वितीययोर्भाज्यराश्योः साधारणो गुणः~। अयमेवात्र स्फुटगुण उच्यते~। अथ तृतीयभाज्यस्य पूर्वभाज्ययोश्चानन्तरोक्तप्रकारेण साधारणगुणे साध्ये स्फुटगुण एव गुणः~। अस्य हर इति स्फुटगुणस्य प्रथमच्छेदयोः संवर्गच्छेदो भवेदित्यत्र प्रथमच्छेदस्य रवि~(१२)~संख्यस्य रसाग्निमनु~(१४३६)~संख्यस्य द्वितीयच्छेदस्य च घातो रदद्व्यत्यष्टि~(१७२३२)~संख्यः~। असौ छेदः~। अथाभ्यां गुणच्छेदाभ्यां चेह ताभ्यां परगुणहाराभ्यां गुणक उक्तवत्साध्य इति त्रयाणां भाज्यानां साधारणो गुणः साध्यते~। अत्र स्फुटगुणात्तृतीयो गुणोऽल्पः~। अतोऽत्र गजेश~(११८)~संख्यः~। तृतीयच्छेदो भाज्यः~। स्फुटगुणस्य रदद्व्यत्यष्टि~(१७२३२)~संख्यश्छेदो भाजकः~। स्फुटगुणस्य च्छिद्ररूपमुनीन्दु~(१७१९)~
संख्यस्य सप्तरस~(६७)~संख्यस्य तृतीयगुणस्य च विशेषो द्रव्यर्थनृप~(१६५२)~संख्यः~। एष क्षयक्षेपः~। एभिर्भाज्यभाजकक्षयक्षेपैर्निरग्रकप्रकारेणाऽऽनीतस्तज्जगुणः शक्र~(१४)~संख्यः~। अस्य च गजेश~(११८)~ संख्यस्य भाजकस्य च वधो द्रव्यर्थाष्टि~(१६५२)~संख्यः~। अयमत्राल्पेन सप्तरस~(६७)~संख्येन तृतीयगुणेन युतः पूर्ववच्छिद्ररूपमुनीन्दु~(१७१९)~संख्यः~। अयमत्र स्फुटगुणः~। अयमेवात्र त्रयाणां भाज्यराशीनां साधारणो गुणः । एवमेवंविधप्रश्नेषु गुणाः साध्याः~॥~१७~॥~१८~॥\\
\indent
अथ पूर्वोक्तनिरग्रन्यायेन चतुर्युगसंबन्धिनोऽधिमासावमदिवसग्रहभगणादीन्भाज्यराशीन्कल्पयित्वा चतुर्युगसंबन्धिनः सूर्यमासचान्द्रवासरभूमिसावनादिवसादीन्भागहारांश्च परिकल्प्याधिकावमभागादिशेषानपि क्षयक्षेपान्परिकल्प्य तत्र कुट्टाकारफलमन्यान्काश्चन विशेषानपि दर्शयिष्यन्नादावधिमासशेषावलम्बनेन प्रश्नत्रयमार्यया प्रकटयति\textendash
\begin{quote}
{\ks अधिकाग्रा\textsuperscript{१}नधिमासान्रविमासान्दिनगणांश्च यो ब्रूते~।\\
सलिलनिधिमेखलायां स भवति सांवत्सराग्रणीर्भूमा~॥~१९~॥}
\end{quote}

\footnotetext{
१~ख. ग्रादधि ।}

\newpage
\thispagestyle{fancy}
\fancyhf{}
\chead{\textbf{कुट्टाकारशिरोमणिः~।}}
\rhead{\textbf{२७}}
\indent
स्पष्टार्थेयमार्या~।।~१९~।।\\
\indent
अथैतत्प्रस्तुतप्रश्नत्रयभङ्गार्थमार्यात्रयमाह\textendash
\begin{quote}
{\ks सिद्धहृतमधिकशेषं द्वेधा विन्यस्य खाद्रिषट्कनखैः~।\\
नवखेषुद्रव्यद्रिरसैर्हत्वा नन्दाष्टवह्नितर्करसैः~॥~२०~॥\\
खचतुष्काङ्गकुपक्षैः क्रमेण विभजेत्फले पृथग्भवतः~।\\
शेषे गताधिमासा रविमासाश्च करमादिह भवन्ति~॥~२१~॥\\
तद्योगं युगदिवसैर्हत्वा विभजेत शीतकरमासैः~।\\
लब्धं फलं सरूपं शेषे सति तद्भवेद्दिवसवृन्दम्~॥~२२~॥}
\end{quote}

\indent
चतुर्विंशति~(२४)~भक्तमुद्दिष्टमधिकमासशेषमुपर्यधोभावेन द्विधा स्थापयित्वा तदुपर्यधोराशी खाद्रिषट्कनखैः~(२०६७०)~नवखेषुद्रव्यद्रिरसैश्च~(६७२५०९)~क्रमाद्धत्वा नन्दाष्टवह्नितर्करसैः~(६६३८९)~खचतुष्काङ्गकुपक्षैश्च~(२१६००००)~सति संभवे क्रमेण विभजेत्~। अत्र लब्धाभ्यां\textsuperscript{१}\ प्रयोजनं नास्ति~। इह शेषौ क्रमाद्गताधिमासा गतरविमासाश्च भवन्ति~। अत्र गताधिमासवृन्दं फले सति रविमासचयः कुट्टाकाराख्यो गुणः~। अथ गताधिमासगतरविमासयोगं वसुद्रव्यष्टाद्रिरूपाङ्कन्सप्ताद्रितिथि~(१५७७९१७८२८)~संख्यैश्चतुर्युगदिनैर्हत्वा रसत्रिगुणरामाग्निवेदपुष्करसायकैः~(५३४३३३३६)~ चतुर्युगचन्द्रमासैर्विभजेत्~। अत्रफलं लब्धं फलं चान्द्रमासशेषे विद्यमाने तस्य भागहारार्धादूनाधिकमनपेक्ष्य सर्वदा रूप~(१)~युक्तं कार्यम्~। तद्दर्शान्तावधिकानां गतानां चान्द्राणां मासानामनुपातसिद्धः सावनदिनसमूहः स्यात्~॥~२०~॥~२१~॥~२२~॥\\
उद्देशकः\textendash
\begin{quote}
{\ku सूर्याद्रिनन्दविधुराममहीध्रसंख्यो\\
दृष्टो मयाऽत्र मतिमन्नधिमासशेषः~।\\
अस्मादिमानधिकमासचयान्विवस्व-\\
न्मासांश्च तद्युतिभवानि वदेरहानि~॥~२३~॥}
\end{quote}

\indent
अत्राधिमासशेषोऽयं~(७३१९७१२)~। अयं सिद्ध~(२४)~हृतः सर्पाष्टतानवेदाभ्राग्नि~(३०४९८८)~संख्यः~। अयमुपर्यधोभावेन~(३०४९८८ । ३०४९८८)~द्विधा विन्यस्तः~।अत्रोपरिराशिः खाद्रिषट्कनखैः~(२०६७०)~ गुणितः खरसातिधृतिदिक्कृतखाग्निषट्~(६३०४१०१९६०)~

\footnotetext{
१~ख. भ्यां फलाभ्यां प्र.।}

\newpage
\thispagestyle{fancy}
\fancyhf{}
\chead{\textbf{महालक्ष्मीमुक्तावलीसहितः\textendash}}
\lhead{\textbf{२८}}
\noindent
संख्यः~। अस्मिन्नन्दाष्टवह्नितर्करसैः~(६६३८९)~विभक्ते शेषः सप्ताष्टाष्टि~(१६८७)~संख्यः~। अयं गताधिमासचयः। अथाधोराशिर्नवखेषुद्रव्यद्रिरसैः (६७२५०९) गुणितः पक्षाङ्काष्टाब्धिमुनिभूगिरिदिगिषुनख ~(२०५१०७१७४८९२)~संख्यः~। अस्मिन्खचतुष्काङ्क~(ङ्ग)~कुपक्षेः~(२१६००००)~विभक्ते शेषो द्रव्यङ्काष्टाब्धीषु~(५४८९२)~संख्यः~। अयं गतरविमासनिवहः~। अथ गताधिमासगतरविमासयोगो नवाद्रीषुरसेषु~(५६५७९)~संख्यः~। अयं गतचान्द्रमाससमूहः~। एष चतुर्युगभूमिसावनदिनैर्गुणितः सूर्याब्धिखाङ्काद्रिरविखाद्रिताराङ्काष्ट~(८९२७७०१२७९०४१२)~संख्यः~। अस्मिन्चतुर्युगचान्द्रमासौर्विभक्ते लब्धं रुद्राष्टाभ्राद्र्यष्टि~१६७०८११~संख्यम्~। अत्र शेषोऽष्टितानवेदाग्निद्रव्यद्रि
~७२३४९१६~संख्यः~। अत्र शेषोऽस्तीति लब्धं रूपयुक्तं सूर्याष्टखाद्र्यष्टि
~१६७०८१२~संख्यम्~। एतद्गतानां दर्शान्ताबधिकानां चान्द्राणां मासानां त्रैराशिकसिद्धं भुसावनदिनवृन्दं भवति~॥~२३~॥\\
\indent
अथाऽऽर्यया क्षयदिनशेषावलम्बनेन कांश्चित्प्रश्नान् स्फुटयति\textendash
\begin{quote}
{\ks अवमाग्रादिष्टदिनं विनेष्टदिवसैर्ग्रहांस्तदग्राणि~।\\
अधिमासानपि यातान्यो वक्ति म एव दैवज्ञः~॥~२४~॥}
\end{quote}
\indent
अत्र चत्वारः प्रश्नाः~। एतेष्ववमाग्रादिष्टदिनं यो वक्ति स एव दैवज्ञ इति प्रथमः~। अत्रेष्टदिनमिति जात्येकवचनम्~। अवमदिनशेषादानितैरिष्टदिवसैर्ग्रहास्तच्छेषगताधिकमासाश्च विज्ञेयाः~। अतोऽवमाग्राज्ज्ञातैरिष्टदिवसैर्विनाऽवमाग्रादेव ग्रहान् यो वदति स एव दैवज्ञ इति द्वितीयः~। अत्र ग्रहशब्देन भगणाद्या वाञ्छितान्ताः सचन्द्रोच्चपाता बुधशुक्रशीघ्रोच्चसहिता मध्यग्रहा उच्यन्ते\textendash\ अथावमाग्रादागतैरिष्टदिवसैर्विनाऽवमाग्रादेव ग्रहशेषान् यो वक्ति स एव दैवज्ञ इति तृतीयः~। अथावमाग्रावगतैरिष्टदिवसैर्विनाऽवमाग्रादेव यातानधिमासान् यो वक्ति स एव दैवज्ञ इति चतुर्थः~। अत्र प्रश्नचतुष्टयेऽपि दैवज्ञशब्देन कुट्टाकारज्ञ उच्यते\textendash\\एवमुत्तरत्रापि द्रष्टव्यम्~॥~२४~॥ \\
\indent
अथैषां प्रश्नानां भङ्गार्थमार्यापञ्चकमाह\textendash
\begin{quote}
{\ks अवमाग्रं वेदहृतं चन्द्राग्निगुणेषुनन्दनागघ्नम्~।\\
रामरसेषुखपर्वतनेत्राङ्गहृतं गतावमं शेषम्~॥~२५~॥}
\end{quote}

\newpage
\thispagestyle{fancy}
\fancyhf{}
\chead{\textbf{कुट्टाकारशिरोमणिः~।}}
\rhead{\textbf{२९}}
\begin{quote}
{\ks वेदहृतं क्षयशेषं मुनिषड्गुणकृतिविलोचनाद्रिशरैः~।\\
निहतं नखाभ्रखेषुस्वराम्बराकाशवेदसंभक्तम्~॥~२६~॥\\
शेषं गतावमोनं तद्विद्यादिष्टदिवससंघातम्~।\\
गतमवमं कुदिनहृतं क्षयाग्रयुक्तं भचक्रसंगुणितम्~॥~२७~॥\\
कुदिनहतावमभक्तं फलं भचक्रादिको भवति खेटः~।\\
शेषं युगावमहृतं तच्छेषं संभवेत्फलं लब्धम्~॥~२८~॥\\
वर्णसुरेभाङ्कगुणैर्यातान्यवमानि ताडयित्वाऽथ~।\\
गुणषट्पञ्चखशैलद्विरसैर्लब्धा गताधिमासाः स्युः~॥~२९~॥}
\end{quote}

\indent
उद्दिष्टमवमशेषं वेद~(४)~हृतं ततश्चन्द्राग्निगुणेषुनन्दनागैः~(८९५३३१)~हतं रामरसेषुखपर्वतनेत्राङ्गैः~(६२७०५६३)~हृतं कार्यम्~। अत्र शेषं गतावमदिनं भवति~। एतत्फलं भवत~। अथोद्दिष्टमवमाग्रं वेद~(४)~ हृतं ततो मुनिषड्गुणकृतिविलोचनाद्रिशरैः~(५७२२०३६७)~निहतं नखाभ्रखेषु स्वराम्बराकाशवेदैः~(४००७५००२०)~संभक्तं कार्यम्~। अत्र शेषं गतचान्द्रदिनं भवति~। कुट्टाकाराख्यो गुणोऽप्येतदेव~। एतद्गतावमदिनोनं कार्यम्~। अत्र यच्छेषं तदिष्टदिवससंचयं जानीयात्~। अथातीतमवमादिनं युगभूमिसावनैर्गुणितं तत उद्दिष्टोनावमदिनशेषेण युक्तं जिज्ञासितस्य ग्रहस्य युगभगणैः
संगुणितं युगभूसावनगणितैर्युगावमदिनैर्भक्तं कार्यम्~। अत्र लब्धं फलं भगणादिर्जिज्ञासितो मध्यग्रहो भवति~। अव यच्छेषं तद्युगावमदिनहृतं कार्यम्~। अत्र लब्धं फलं तच्छेषं भवति~। एतदुक्तं भवति~। मध्यग्रहो भगणरूपश्चेत्तदा लब्धं भगणशेषं भवति~। सराश्यन्तश्चेत्तद्राशिशेषम्~। सभागान्तश्चेत् तद्भागशेषम्~। सलिप्तान्तश्चेत्कलाशेषम्~। सविकलान्तश्चेत्तद्विकलाशेषं भवतीति~। अथ गतावमदिनानि वर्णसुरेभाङ्कगुणैः (३९८३३४) हृत्वा गुणषट्पञ्चखशैलद्विरसैः (६२७०५६३) हरेत्~। अत्र लब्धा गताधिमासा
भवन्ति~। अत्र शेषे भागहारार्धादाधिकेऽप्यर्थोत्तरत्वेन रूपक्षेपो न कार्यः~। नन्वत्रावमाग्रेणैव गताधिमासानयने कर्तव्ये गतावमदिनैस्तदानयनं प्रदर्शितम्~।
कथमेतदत्रोच्यते\textendash\ अवमाग्रादेवाऽऽनीतैर्गतैरवमैर्गताधिमासेष्वानीतेष्वपि पारम्पर्येणावमाग्रादेव गताधिमासानयनमित्यदोषः~॥~२५~॥~२६~॥~२७~॥~२८~॥~२९~॥

\newpage
\thispagestyle{fancy}
\fancyhf{}
\chead{\textbf{महालक्ष्मीमुक्तावलीसहित\textendash}}
\lhead{\textbf{३०}}
\indent
उद्देशकः\textendash
\begin{quote}
{\ku पूर्णाभ्रतर्कनयनेषुनवार्थनाग\textendash\\
रामेन्दुसंख्यमवमाग्रमतो दिनानि~।\\
एतैर्विनाऽम्बरचरान्भगणादिशेषान्\\
याताधिमासानिवहं तु वदाऽऽशु विद्वन्~।।~३०~।।}
\end{quote}
\indent
अत्रायमवमदिनशेषः~( १३८५२६०० )~। अस्मादुक्तवदानी तान्यभीष्टदिनानि सूर्याष्टखाद्र्यष्टि~(१६७०८१२)~संख्यानि एभिरिष्टदिनैर्विनोद्दिष्टेनैव क्षयाग्रेणोक्तवदानीतो भगणात्मको रविमध्योवर्णाद्रीषुकृत~(४५७४)~संख्यः~। भगणेषो वसुद्रव्यङ्काब्ध्यङ्कषडीशार्थ~(५११६९४६२८)~संख्यः~। एवं राश्यान्तादयो रविमध्यास्तत्तच्छेषाश्चन्द्रादयस्तत्तच्छेषा अप्यानेयाः~। अथात्रोक्तवदानीता गताधिमासाः सप्ताष्टाष्टि~(१६८७)~संख्याः~।\\
\indent
अत्रोदितो गणनकर्मलघुत्वहेतोर्भूवासराहतयुगक्षयवासरौघः~।
षट्पञ्चतर्कवसुनागकुशैलनन्दतत्त्वाग्न्यगाद्रिगिरिबाणनवाग्नि
~(३९५७७३२५९७१८८६५६)~संख्यः~॥~३०~॥\\
\indent
अथाऽऽर्यया कौचित्प्रश्नौ दर्शयति\textendash
\begin{quote}
{\ks ग्रहमध्यैक्याद्भादेर्वितत्परान्ताद्दिनानि यो वदति~।\\
तन्मध्यानि च तस्मात्पृथकपृथग्दिनचयं विना स सुधीः~॥~३१~॥}
\end{quote}
\indent
भादेराश्यादेर्वितत्परान्तात्तत्पराणां षष्ट्यंशो वितत्परा~। तदन्ताद्
ग्रहमध्यैक्यादुद्दिष्टग्रहमध्यमसंयोगाद्दिनानीष्टभूवासराणि यो वदति स सुधीरित्येकः प्रश्नः~। दिनचर्यं विना ग्रहमध्यैक्यादानीतमिष्टदिनसमूहं विना~। तस्माद् ग्रहमध्यैक्यादेव~। तन्मध्यानि च~। अत्र द्वित्वमविवक्षितम्~। तेन संयुक्तयोर्ग्रहयोर्मध्यमौ संयुक्तानां ग्रहाणां मध्यमानपि पृथक्पृथग्यो वदति स सुधीरित्युपरः प्रश्नः~। अत्र ग्रहमध्यैक्यानयनं प्रदर्श्यते~। इष्टग्रहयोरिष्टग्रहाणां वा युगभगणान्योजयित्वा तैरिष्टदिनानि हत्वा युगभूदिनैरेव विभजेत्~। अत्र लब्धा भगणास्तानपास्य शेषं द्वादशादिभिर्हत्वा युगभूदिनैरेव विभजेत्~। अत्र लब्धा राश्यादयो भवन्ति~। एवं वितत्परान्तं कुर्यात्~। अत्र वितत्पराशेषे युगभूदिनार्धादधिके सत्यर्धोत्तरत्वेन वितत्परास्वेकां वितत्परां निक्षिपेत्~। एवं कृते ग्रहमध्यैक्यं भवति~। एवमागतग्रहमध्यैक्यावम्बलनेनैवात्र प्रश्नद्वयमिति मन्तव्यम्~॥~३१~॥

\newpage
\thispagestyle{fancy}
\fancyhf{}
\chead{\textbf{कुट्टाकारशिरोमणिः~।}}
\rhead{\textbf{३१}}
\indent
अथैतत्प्रश्नद्वयभर्ङ्गाथमार्यात्रयमाह\textendash
\begin{quote}
{\ks ग्रहमध्यमसंयोगं वितत्परीकृत्य दृढदिनैर्हत्वा~।\\
नियुतहतषडिषुरसरसवेदैर्विभजेत्फलं क्षयक्षेपः~॥~३२~॥\\
तद्दृढभगणैक्यदिनैर्निरग्रमार्गेण गुणफले कुर्यात्~।\\
तत्त्रिगुणो दिननिकरः फलं ग्रहाणां गतानि चक्राणि~॥~३३~॥\\
मध्यैक्यं चक्रयुतं तत्तद्भगणैः पृथक् पृथग्हत्वा~।\\
तद्भगणैक्येन भजेत्तत्तन्मध्यं फलं सपर्यायम्~॥~३४~॥}
\end{quote}
\indent
अत्राऽऽदौ दृढदिनदृढभगणैक्ययोः स्वरूपं दर्शयति~। उद्दिष्टग्रहयोरुद्दिष्टग्रहाणां वा युगभगणान् संयोज्य तेषां युगभूदिनानामप्यपवर्तकोऽस्ति चेत्तेन संयुक्तानुद्दिष्टग्रहभगणान्युगभूदिनानि चापवर्तयेत्~। एवमपवर्तितानि युगदिनान्यत्र
दृढदिनानि~। एवमपवर्तितमिष्टग्रहयोरिष्टग्रहाणां वा युगभगणैक्यं दृढभगणैक्यम्~। अपवर्तको नास्ति चेद्युगदिनान्येव दृढदिनानि~। उद्दिष्टग्रहयुगभगणयोग
एव दृढभगणैक्यमिति~। अथोद्दिष्टग्रहभगणैक्यं वितत्परीकृत्य तान् दृढदिनैर्हत्वा चक्रवितत्पराभिः~(४६६५६०००००)~विभजेत्~। अत्र शेषे चक्रवितत्परार्धादधिके सत्यर्धोत्तरत्वेन रूपं फले क्षिपेत्~। एवमानीतं फलमत्र ऋणक्षेपः~। दृढभगणयोगो भाज्यराशिः~। दृढदिनचयो भाजकः~। एभिरुक्तवद्गुणफले आनयेत्~। अत्र गुण इष्टदिनसमूहः~। फलं तु संयुक्तग्रहगतभगणाः~। अथ गतभगणयुक्तं ग्रहमध्यैक्यं संयुक्तेषु ग्रहेष्वेकस्य युगभगणैः पृथक् पृथग्गुणयित्वा संयुक्तग्रहयुगभगणैक्येन विभजेत्~। लब्धं सपर्यायं तन्मध्यं भवति~। संयुक्तग्रहेषु यस्य यस्य युगभगणैर्गुणितं तस्य तस्य
भगणादिमध्यं भवतीत्यर्थः~। अत्र प्रकिया सम्यक् प्रदर्शितेति नोद्दिश्यते~।
एवमुत्तरत्रापि यत्र यत्र प्रक्रिया सम्यक्प्रदर्शयिष्यते तत्र तत्र नोद्दिश्यत इति मन्तव्यम्~॥~
३२~॥~३३~॥~३४~॥\\
\indent
अथाऽऽर्यया ग्रहविकलाशेषावलम्बनेन कांश्चनानुयोगानाह\textendash
\begin{quote}
{\ks विकलाशेषाद्दिवसान् मध्यं दिवसैर्विनाऽन्यतच्छेषात्~।\\
अवमदिनैस्तच्छेषावधिमासांश्च ब्रवीति यः स कृती~॥~३५~॥}
\end{quote}

\indent
अत्र सप्त प्रश्नाः~। एषु विकलाशेषाद्दिवसान्यो ब्रवीति स कृतीति प्रथमः~।
विकलाशेषादेव विकलाशेषवन्मध्यं यो वक्तीति द्वितीयः~। विकलाग्रादानीतै\textendash

\newpage
\thispagestyle{fancy}
\fancyhf{}
\chead{\textbf{महालक्ष्मीमुक्तावलीसहितः\textendash}}
\lhead{\textbf{३२}}
\noindent
रिष्टदिवसैर्विना विकलाग्रादेवान्यग्रहमध्यान्यो वदतीति तृतीयः~। विकलाग्रेणाऽऽनीतैरिष्टदिनैर्विना विकलाग्रेणैवान्यग्रहविकलाशेषान्यो वदतीति चतुर्थः~।
विकलाग्रादागतैरभीष्टदिनैर्विना विकलाग्रादेव गतावमदिनानि यः कथयतीति
पञ्चमः~। विलिप्ताशेषादवगतानि दिनानि विना विलिप्ताशेषादेवावमदिनशेषं
यो वदतीति षष्ठः~। विलिप्ताग्राज्जातेभ्योऽभीष्टदिनेभ्यो विना विलिप्तिकाग्रादेव
गताधिमासान् यो ब्रूते स कृतीति सप्तमः~॥~३५~॥\\

\indent
अथैतेष्वाद्ययोः प्रश्नयोर्भङ्गार्थं गुणफले प्रदर्शयिष्यन्नादौ तदुक्तक्रममार्ययाऽऽह\textendash
\begin{quote}
{\ks अथ गुणफले क्रमेण प्रतिखेटमिहैकयाऽऽर्ययोच्येते~।\\
दिनकरबुधकाव्यानां साधारणमेव ते निगद्येते~॥~३६~॥}
\end{quote}
\indent
एतदुक्तं भवति~। चन्द्राङ्गारकबुधबृहस्पति शनैश्चराणां बुधकाव्यशीघ्रोच्चयोश्चन्द्रोच्चपातयोच्च प्रत्येकमेकार्यया गुणफले क्रमेण उच्येते~।
रविबुधभार्गवाणां तु साधारणप्रेवैकयाऽऽर्यया गुणफले क्रमेणोच्येते इति~॥~३६~॥
\begin{quote}
{\ks रविभृगुविदां नवाम्बरशैलाक्ष्यत्यष्टिनयनशरचन्द्राः~।\\
गिरिपूर्णवेदवह्न्यगशरकृतरामाङ्कनन्दरामशराः~॥~३७~॥}
\end{quote}
\indent
सूर्यकाव्यबुधानां गुणः~(१५२१७२७०९)~। एषां फलं~(५३९९३४५७३४०७)~॥~३७~॥
\begin{quote}
{\ks चन्द्रस्य दन्तिसिन्धुरनन्दाकाशाङ्गवह्निभूमिरसाः~।\\
भूयुगनवककृतानलखशरर्तुवियच्छशाङ्कनन्दकराः~॥~३८~॥}
\end{quote}
\indent
चन्द्रस्य गुणः~(६१३६०९८८)~। अस्य फलं~(२९१०६५०३४९४१)~॥~३८~॥
\begin{quote}
{\ks धरणीसुतस्य नखधृतिचन्द्रार्थशरद्वयक्षपानाथाः~।\\
स्वरवियदुदधिगुणार्थद्विकृताद्रिक्ष्माधराङ्गवह्लियमाः~॥~३९~॥}
\end{quote}
\indent
भौमस्य गुणः~(१२५५११८२०)~फलं~(२३६७७४२५३४०७)~॥~३९~॥
\begin{quote}
{\ks विच्छीघ्रस्य वियद्रसगुणनन्दकृताभ्रचन्द्रशरपक्षाः~।\\
ऋषिखाब्धिगुणहयाचलभूताब्ध्यर्थाष्टरन्ध्रतर्कगुणाः~॥~४०~॥}
\end{quote}
\indent
बुधशीघ्रोच्चस्य गुणः~(२५१०४९३६०)~। अस्य फलं~(३६९८५४५७७३४०७)~॥~४०~॥

\newpage
\thispagestyle{fancy}
\fancyhf{}
\chead{\textbf{कुट्टाकारशिरोमणिः~।}}
\rhead{\textbf{३३}}
\begin{quote}
{\ks सुरपूजितस्य गुणरसबाणाक्षिनवाग्निबाहुरजनीशाः~।\\
तुरगवियदब्धिपुष्कररन्ध्राकाशाष्टगगनमुनिरामाः~॥~४१~॥}
\end{quote}
\indent
गुरोर्गुणः~(१२३९२५६३)~। फलं~(३७०८०९३४०७)~॥~४१~॥
\begin{quote}
{\ks सितशीघ्रस्य युगाष्टकमुनिवसुरसनयनकुञ्जरेशानाः~।\\
हयशून्यमनुशराभ्रद्वयसिन्धुमृगाङ्कदस्रगजतर्काः~॥~४२~॥}
\end{quote}
\indent
शुक्रशीघ्रोच्चस्य गुणः~(११८२६८७८४)~। अस्य फलं~(६८२१४२०५१४०७)~॥~४२~॥
\begin{quote}
{\ks सौरस्य रन्ध्रसप्तकवर्णाष्टकवह्निखाभ्रवायुसखाः~।\\
शैलाभ्रसिन्धुभूधरमुनिखरसेन्दुरसकृष्णवर्त्मानः~॥~४६~॥}
\end{quote}
\indent
शनैश्चरस्य गुणः~(३००३८४७९)~। फलं~(३६१६०७७४०७)~॥~४३~॥
\begin{quote}
{\ks चन्द्रोच्चस्य रसग्रहरसवसुशरशिखिमहीधराङ्कभुवः~।\\
गिरिगगनसागरग्रहगजमार्गणतर्कलोकरन्ध्रनगाः~॥~४४~॥}
\end{quote}
\indent
चन्द्रोच्चस्याग्निशून्याश्वीति~(सूर्य सि. अ.~१~श्लोक~३३)~पाठानुसारेण गुणः~(१९७३५८६९६)~। अस्य फलं~(७९१४६५८९४०७)~॥~४४~॥
\begin{quote}
{\ks राहोर्नवतिथिभूभृद्बाणाकाशाद्रिविष्णुपदरामाः~।\\
अचलाम्बरार्णवस्वररूपाद्रिनवर्तुभूतसर्पार्थाः~॥~४५~॥}
\end{quote}
\indent
वामं पातस्य वस्वग्नीति~।~(सूर्यसि. अ.~१~श्लोक~३३)~पाठानुरोधेन राहोर्गुणः~(३०७०५७१५९)~। अस्य फलं~(५८५६९११७४०७)~॥~४५~॥
\indent
अथैवं दर्शितैगुर्णफलौरिष्टदिनविकलाग्रहानयनमार्याभ्यामाह\textendash
\begin{quote}
{\ks गुणनिहतं विकलाग्रं कुदिनैर्विभजेद्यदुत्र परिशिष्टम्~।\\
पारावारैर्विभजेत्तत्फलमिष्टं भवेद्दिवसवृन्दम्~॥~४६~॥\\
विकलाशेषं स्वफलै\textsuperscript{१}र्हत्वा विकलीकृतैर्हरेच्चक्रैः~।\\
अवशेषमपि समुद्रैर्विभजेल्लब्धं विलिप्तिकाखेटः~॥~४७~॥}
\end{quote}
\indent
गुणनिहतं विकलाशेषं भूदिनैर्विभजेत्~। अत्र लब्धेन नास्ति प्रयोजनम्~।अत्र परिशिष्टं यत्तत्पारावारैर्विभजेत्~। अथ विकलाशेषमेव स्वफलैर्हत्वा विकलीकृतैर्विकलाग्रवद्ग्रहभगणैर्विभजेत्~। अत्रापि~(लब्धेन)~प्रयोजनं
\footnotetext{
१~क. लैर्हृत्वा~।}

\newpage
\thispagestyle{fancy}
\fancyhf{}
\chead{\textbf{महालक्ष्मीमुक्तावलीसहितः\textendash}}
\lhead{\textbf{३४}}
\noindent
नास्ति~। अत्रावशेषमपि समुद्रैर्विभजेत्~। अत्र लब्धं विलिप्तारूपो मध्यग्रहो भवेत्~॥~४६~॥~४७~॥\\

\indent
अथैवमागताभिर्ग्रहविकलाभिस्तद्विकलाशेषेण चान्यग्रहमध्यतच्छेषक्षयदिनतच्छे
षाधिमासानामानयनप्रकारमार्याचतुष्टयेनाऽऽह\textendash\begin{quote}
{\ks ग्रहविकला(युग)दिनहता विकलाग्रयुता ध्रुवं भवेत्तत्तु~।\\
जिज्ञासितस्य भगणैः संताड्य ज्ञातमण्डलैर्हत्वा~॥~४८~॥\\
लब्धं कुदिनैर्विभजेत्फलमिह जिज्ञासितो ग्रहो भवति~।\\
शेषं तद्विकलाग्रं भवत्यथ प्राक्प्रदर्शितध्रुवकम्~॥~४९~॥\\
हत्वा युगावमदिनैर्विभजेद्विकलीकृतैः खचरचक्रैः~।\\
लब्धे भूदिवसाप्ते फलशेषौ स्तः क्षयाहतच्छेषौ~॥~५०~॥\\
ध्रुवमधिमासैर्हत्वा विभजेत्खचरस्य भगणविकलाभिः~।\\
लब्धं फलं कुदिवसैर्विभजेल्लब्धं भवेयुरधिमासाः~॥~५१~॥}
\end{quote}
\indent
ग्रहविकला युगदिनाहता उद्दिष्टविकलाशेषयुक्ताश्च कार्याः~। एवं कृते तद्ध्रुवं भवति~। तत्तु ज्ञातुमिष्टस्य ग्रहस्य युगभगणैः संगुण्योद्दिष्टविकलाग्रवद्ग्रहयुगभगणैर्ह(र्ह)त्वा तत्र लब्धं च भूदिनैर्विभजेत्~। अत्राऽऽगतं फलं जिज्ञासितो मध्यग्रहो भवति~। शेषं तद्विकलाशेषं भवति~। अथ पूर्वोक्तध्रुवं युगावमदिनैर्हत्वा विलिप्तीकृतैरुद्दिष्टशेषवद्ग्रहयुगभगणैर्विभजेत्~। अत्र लब्धे भूदिवसाप्ते सति फलशेषौ क्रमाद्गतावमदिनतच्छेषौ भवतः~। अथ ध्रुवं युगाधिमासैर्हत्वोद्दिष्टविकलाग्रतो ग्रहस्य युगभगणविकलार्भिहरेत्~। अत्र\textsuperscript{१} लब्धं फलं च युगभूदिनैर्हरेत्~। अत्राऽऽगतं फलं गताधिमासाः स्युः~॥~४८~॥~४९~॥~५०~॥~५१~॥\\
\indent
उद्देशकः\textendash
\begin{quote}
 {\ku विलिप्ताग्रात्पूर्ष्णः षडिषुनवखाष्टीभगगन\textendash\\
क्षमासंख्यादस्मात्त्वरितमभिधत्स्वेष्टदिवसान्~।\\
विनैभिर्भास्वन्तं विधुमपि तदग्रं क्षयदिनं\\
ततश्चैतच्छेषं वद तदधिमासानपि गतान्~॥~५२~॥}
\end{quote}
\indent
रवेर्विकलाशेषमिदं~(१०८१६०९५६)~। अस्मादुक्तवदानीता दिवसाः सूर्याष्टखाद्र्यष्टि~(१६७०८१२)~संख्या~। एभिर्विना रविविकलाग्राद्देवोक्तप्रकारेणाऽऽनीतो विकलारविस्त्रिभाब्धिद्वित्र्यष्टद्रव्यङ्केषु~(५९२८३२४\textendash

\newpage
\thispagestyle{fancy}
\fancyhf{}
\chead{\textbf{कुट्टाकारशिरोमणिः~।}}
\rhead{\textbf{३५}}
\noindent
२७३)~संख्यः~। विकलाचन्द्र सप्तेन्दुद्विरसाब्ध्यगाब्धितत्त्वाङ्काद्रि~(७९२५४७४६२१७)~
संख्यः~। चन्द्रविकलशेषः संकृत्यग्नितिथिकृतषड्वेदाद्रि~(७४६४१५३२४)~संख्यः~। क्षयदिवसा गजार्थेषुषडक्षि~(२६५५८)~संख्याः~। अवमाग्रं खाभ्राङ्गाब्धीषुनन्दार्थाष्टलोक~(१४८५९५४६००)~संख्यम्~। अधिमासाः सप्ताष्टाष्टि~(१६८७)~संख्या~। एतत्सर्वं रविविकलाग्रादेवाक्तवदानीतम्~॥~५२~॥\\
\indent
अथ भगणादिशेषैरिष्टदिनग्रहोपायमार्ययाऽऽह \textendash
\begin{quote}
{\ks भगणग्रहभागलिप्ताशेषं भगणाग्रमेव रचयित्वा~।\\
तेन दिनग्रहविकलाः कुर्याद्भगणादिखेचरांश्चाऽऽभिः~॥~५३~॥}
\end{quote}
\indent
उद्दिष्टभगणादिशेषं विकलाशेषं कृत्वा तेनोक्तमार्गेणेष्टदिनानि ग्रहविकलाश्चाऽऽनयेत्~। अथाऽऽभिर्ग्रहविकलाभिर्युक्त्या भगणग्रहादींश्च कुर्षात्~॥~५३~॥\\
\indent
अथ स्वस्वविकलाग्राद्ग्रहाणां साधारणं विकलमध्यानयनं सार्धयाऽऽर्ययाऽऽह\textendash
\begin{quote}
{\ks सिन्धुहृते विकलाग्रे खत्रयरसरन्ध्रभास्करैर्भक्ते~।\\
यच्छेषं तद्धत्वा शैलाकाशाम्बुधिस्वराग्निकरैः~॥~५४~॥\\
खखखर्तुनन्दसूर्यैर्विभजेच्छेषं ग्रहो विलिप्तात्मा~।}
\end{quote}
\indent
सिन्धु~(४)~हृते विकलाग्रे खत्रयरसरन्ध्रभास्करै~(१२९६०००)~र्भक्ते सति तच्छेषं शैलाकाशाम्बुविस्वराग्निकरै~(२३७४०७)~र्हत्वा खखखर्तुनन्दसूर्यै~(१२९६०००)~र्विभजेत्~। लब्धेन प्रयोजनं नास्ति~। शेषं विकलात्मको मध्यग्रहो भवति~। अत्रेदमनुसंधेयम्~। यस्य विकलाग्राद्यदेवमानीतं मध्यं तत्तस्यैवेति~॥~५४~॥
\indent
अथाऽऽर्यया धनक्षेपविषयप्रश्नमाह\textendash
\begin{quote}
{\ks यो वेददिवसखेटौ विकलाशेषोनयुगदिनौघेन~।\\
भारतवर्षे सोऽस्मिन्भजति धनक्षेपमार्गवेदित्वम्~॥~५५~॥}
\end{quote}
\indent
विकलाशेषोनयुगदिनौघेन दिवसखेटौ यो वेद सोऽस्मिन्भारतवर्षे धनक्षेपमार्गवेदित्वं भजति~॥~५५~॥\\
\indent
अथैतत्प्रश्नभङ्गार्थमार्याद्वयं सार्धमाह\textendash
\begin{quote}
{\ks निजनिजगुणफलराशी निजनिजदृढदिवसभाज्यराशिभ्याम्~॥~५६~॥}
\end{quote}

\newpage
\thispagestyle{fancy}
\fancyhf{}
\chead{\textbf{महालक्ष्मीमुक्तावलीसहितः\textendash}}
\lhead{\textbf{३६}}
\begin{quote}
{\ks जह्यादिहावशेषौ खचराणां गुणफले क्रमाद्भवतः~।\\
ताभ्यां जलनिधिभक्तौ विमौर्विकाग्रोनयुगदिनस्तोमौ~॥~५७~॥\\
हत्वा विभजेद्दृढदिनभाज्याभ्यामत्र ये च परिशिष्टे~।\\
तत्राऽऽदिमं दिनं स्यादपरं खेटो विमौर्विकारूपः~॥~५८~॥}\\
\end{quote}
\indent
अत्र युगदिनचतुर्थांशः सर्वेषां दृढदिवसः~। स्वस्वविकलीकृतयुगभगमचतुर्थांशः स्वस्वभाज्यराशिः~। एवंविधाभ्यां निजनिजदृढदिवसभाज्यराशिभ्यां क्रमात्पूर्वोक्तौ निजनिजगुणफलराशी जह्यात्~। इहावशेषौ क्रमात्खचराणां निजनिजगुणफले भवतः~। ताभ्पां क्रमाज्जलनिधि~(४)~भक्तौ द्वावुद्दिष्टौ विमौर्विकाग्रोनयुगदिनस्तोमौ हत्वा निजनिजदृढभाज्याभ्यां क्रमाद्विभजेत्~। अत्र लब्धाभ्यां प्रयोजनं नास्ति~। अत्र ये च परिशिष्टे स्तस्तत्राऽऽदिमं दिनं स्यात्~। अपरं विमौर्विकारूपः खेटः स्यात्~। विमौर्विका विकला~॥~५६~॥~५७~॥~५८~॥\\
\indent
अथ स्वस्वविकलाशेषोनयुगदिनैः सर्वेषां साधारणं विकलाग्रहानयनप्रकारं सार्धयाऽऽर्ययाऽऽह\textendash
\begin{quote}
{\ks अथवा वारिधि~(४)~भक्तान्स्वस्वविलिप्ताग्रहोनयुगदिवसान्~।\\
रामाङ्कभूतवसुशरदिग्भिः~(१०५८५९३)~संताड्य चक्रविकलाभिः~(१२९६००)~॥~५९~॥\\
विभजेत तत्र शेषा भवन्ति खचरा विलिप्तिकात्मानः~।}
\end{quote}
\indent
स्पष्टोऽर्थः~। अत्रानन्तरोक्ते प्रकारद्वयेऽपि धनक्षेपत्वाद्विकलाग्रह एका विकलाऽधिका भवेत्~॥~५९~॥\\
\indent
अथाऽऽर्यया विकलाग्रहाश्रयणे दिनविकलाशेषविषयौ प्रश्नामाह\textendash
\begin{quote}
{\ks विकलाग्रहाद्दिनौघं विकलाशेषं च यो भणति शीघ्रम्~॥~६०~॥\\
चतुरुदधिमेखलायां चतुरमनीषो भवत्यसौ भूमौ~।}
\end{quote}
\indent
स्पष्टार्थेयमार्या~॥~६०~॥\\
\indent
अथैतत्प्रश्नभङ्गार्थमध्यर्धामार्यामाह\textendash
\begin{quote}
{\ks ग्रहविकलाः कुदिनहता भगणविलिप्ताहृता दिनं लब्धम्~॥~६१~॥}
\end{quote}

\newpage
\thispagestyle{fancy}
\fancyhf{}
\chead{\textbf{कुट्टाकारशिरोमणिः~।}}
\rhead{\textbf{३७}}
\begin{quote}
{\ks शेषेऽत्र सति तु लब्धं सैकं कार्यं ततो दिवसवृन्दम्~।\\
शेषोनभागहारो विकलाशेषो भवेद्ग्र\textsuperscript{१}हेऽप्यस्य~॥~६२~॥}
\end{quote}
\indent
भूदिनहता ग्रहविकला विकलीकुतग्रहयुगभगणहृताः कार्याः~। अत्र
शेषाभावे लब्धमेवेष्टं दिनं भवति~। शेषे सति तु लब्धं सैकं कार्यम्~। ततः
स्वेष्टदिवसवृन्दं भवति~। अत्र शेषाभावे विकलाशेषो नास्ति~। शेषे सति तु
शेषोनं विकलीकृतग्रहयुगपर्यायवृन्दं विकलाग्रं भवेत्~। अत्र शेषाभावः कादाचित्कः~॥~६१~॥~६२~॥\\
\indent
अथाऽऽर्ययैकस्यैव ग्रहस्यैकदिनजातभगणराशिभागकलाविकलाशेषयोगादिष्टदिनानि यो वेत्ति स एव कुट्टाकारं वेदेत्याह\textendash
\begin{quote}
{\ks चक्रसदनांशलिप्ताविलिप्तिकाग्रैक्यतोऽभिमतदिवसान्~।\\
यो जानाति स भूम्यां कुट्टाकारं स एव जानीते~॥~६३~॥}
\end{quote}
\indent
अवतारिकया गतार्थेयम्~॥~६३~॥\\
\indent
अथैतत्प्रश्नमङ्गाय सार्धामार्यामाह\textendash
\begin{quote}
{\ks अत्राग्रैक्ये कुदिनैर्विहृते शेषं भवेत्क्षयक्षेपः~।\\
त्र्यद्र्यङ्काद्रिकुविश्वैर्गुणितं ग्रहचक्रमस्य भाज्यं स्यात्~॥~६४~॥\\
कुदिनच्छेदस्त्वेतैर्गुणकं कुर्यादसावभीष्टदिनम्~।}
\end{quote}
\indent
अत्रोद्दिष्टे भागादिशेषयोगे सति संभवे क्षितिदिनैर्विभक्ते सति यच्छिष्यते तदत्र ऋणक्षेपः स्यात्~। नो चेद्भगणादिशेषयोग एवात्र ऋणक्षेपः स्यात्~। त्र्यद्र्यङ्काद्रिकुविश्वै~(१३१७९७३)~ र्गुणितानि ग्रहयुगचक्राण्यत्र भाज्यराशिर्भवेत्~। क्षितिदिनानीह भाजकः स्यात्~। एतैस्तु पूर्वोक्तमार्गेण गुणमानयेत्~। एवमागतोऽयं गुण एवेष्टदिनानि स्युः~। अत्र वारकुट्टाकारो द्विविधः~। अत्रोद्दिष्टभगणादिशेषावलम्बनेनैकः~।उद्दिष्टग्रहावलम्बनेनापरः~। एवं वेलाकुट्टाकारोऽपि वेदितव्यः~। अत्र यावुद्दिष्टग्रहाश्रयणेन
वारवेलाकुट्टाकारौ तावुत्तरत्र प्रदर्शयिष्येते~। अत्र तूद्दिष्टभागाद्यग्राश्रयेणाऽऽनीयमानौ
यौ वारवेलाकुट्टाकारौ तौ क्रमेण प्रदर्श्येते । वारकुट्टाकारगतभगणशेषावलम्बेनोद्देशकः\textendash
\begin{quote}
{\ks देवेन्द्रार्चितवारपूर्तिसमये राजीवबन्धोरभू\textendash\
च्चक्राग्रं नियुताहताष्टककरक्ष्माभृन्मही~(१७२८०००००)~संमितम्~।\\
भूयोऽप्येतदभूत्सितेन्दुकुजविद्वारावसाने समं\\
तत्कालान्वद वारकुट्टकविधौ यद्यस्ति ते संस्तवः~॥~६४~॥}
\end{quote}
\footnotetext{
१~ख. दग्रहेन्द्रस्य~।}

\newpage
\thispagestyle{fancy}
\fancyhf{}
\chead{\textbf{महालक्ष्मीमुक्तावलीसहितः\textendash}}
\lhead{\textbf{३८}}
\indent
अथैतदादिप्रश्नभङ्गायाऽऽर्यासप्तकमाह\textendash\
\begin{quote}
{\ks खचराणां युगभगणा भाज्यो राशिर्भवेत्तदानीं तु~॥~६५~॥\\
भगणाग्रे ह्युद्दिष्टे तद्भगणाग्रं भवेदृणक्षेपः~।\\
भवनीभूता भगणा भाज्यो राशिर्भवेत्तदानीं तु~॥~६६~॥\\
भवनाग्रे तूद्दिष्टे तद्भवनाग्रं भवेदृणक्षेपः~।\\
भागाग्रे तूद्दिष्टे तद्भागाग्रं भवेद्दृणक्षेपः~॥~६७~॥\\
भागीभूता भगणा भाज्यो राशिर्भवेत्तदानीं तु~।\\
लिप्ताग्रे तूद्दिष्टे तल्लिप्ताग्रं भवेदृणक्षेपः~॥~६८~॥\\
लिप्तीभूता भगणा भाज्यो राशिर्भवेत्तदानीं तु~।\\
विकलाग्रे तूद्दिष्टे तद्विकलाग्रं भवेदृणक्षेपः~॥~६९~॥\\
विकलीभूता भगणा भाज्यो राशिर्भवेत्तदानीं तु~।\\
युगभूसावनदिवसान्येवं स्याद्भाजकोऽत्र सर्वत्र~॥~७०~॥\\
एभिर्भाजकभाज्यक्षेपैर्यो भवति वर्धकः सोऽत्र~।\\
आदिमकालस्तस्मिन्दृढदिवसान्बुद्धिसंगुणान्युक्त्वा~॥~७१~॥\\
अभिहितवारांस्तांस्तान्कालानन्यांश्च साधयेत्प्राज्ञः~।}
\end{quote}

\indent
भगणशेष उद्दिष्टे सति तदुद्दिष्टं भगणशेषं क्षयक्षेपः स्यात्~। तदा पुनर्ग्रहाणां चतुर्युगभगाणा भाज्यराशिः स्यात्~। अत्र हिशब्दो हेतौ~। यस्मादुद्दिष्टभगणशेषमृणक्षेपो भवति तस्माद्ग्रहयुगभगणा भाज्यराशिः स्यादिति~। राशिशेष उद्दिष्टे सति तदुद्दिष्टं राशिशेषमृणक्षेषः स्यात्~। तदानीं तु द्वादश(१२)गुणितयुगभगणा भाज्यः स्यात्। भागशेष उद्दिष्टे तु तदुद्दिष्टं भागशेषमृणक्षेपः स्यात् । तदानीं तु षष्ट्युत्तरशतत्रया (३६०) हता युगभगणा भाज्यः स्यात् । कलाशेष उद्दिष्टे तु तदुद्दिष्टं कलाशेषमृणक्षेपः स्यात् । तदानीं तु खखषड्धन (२१६००)निहता युगभगणा भाज्यः स्यात्। विकलाशेष उद्दिष्टे सति तदुद्दिष्टं विकलाशेषमृणक्षेपः स्यात् । तदानीं तु खत्रयर्त्वङ्कार्का (१२९६००)भ्यस्ता ग्रहचतुर्युगभगणा  भाज्यराशिः स्यात् । अत्र सर्वत्र चतुर्युगभुदिनान्येव भागहारो भवेत् । एवं प्रदर्शितैरेतैर्भाज्यभाजकक्षेपैः पूर्वोक्तप्रकारेण यो गुणो भवति स इहोद्देशकश्लोकप्रथमकथितवारान्तान्तः प्रथमकालः स्यात् । अथ तस्मिन्नाद्यकाले मतिगुणितान्दृढ\textendash

\newpage
\thispagestyle{fancy}
\fancyhf{}
\chead{\textbf{कुट्टाकारशिरोमणिः।}}
\rhead{\textbf{३९}}
\noindent
दिवसान् निक्षिप्योद्देशकश्लोकान्तरोक्तवारान्तान्तानन्यान् कालानपि
कुट्टकगणितज्ञः साधयेत्~। एतदुक्तं भवति~। दृढदिनेषु सप्तहतेषु यच्छिष्यते तदवलम्बनेन दृढदिवसानां कियतीभिरावृत्तिभिरुद्दिष्टवाराः समायान्तीत्यालोच्य
तावतीभिरावृत्तिभिर्गुणितान्दृढदिवसान् गुणयित्वोद्देशकश्लोकान्तरोक्तवारावसानानन्यान्कालानपि साधयेत्~। 
अत्र फलं तदुद्दिष्टभगणादिशेषिवशाद्यथाक्रमं ग्रहभुक्तभगणादिभिर्भवति~।  न्यासः\textendash\ अत्रोद्देशकश्लोकस्थो रविभगणशेषोऽयम्~(१७२८०००००)~। अयमत्र ऋणक्षेपः~। रवियुगभगणाः खचतुष्करदार्णवाः~(४३२००००)~भाज्यराशिः~। युगभूदिवसाः वसुद्व्यष्टाद्रिरूपाङ्कसप्ताद्रितिथयः~(१५७७९१७८२८)~भाज्यराशिः~।  एभिर्भाज्यभाजकक्षेपैः पूर्ववदानीतो गुणः खाब्धि~(४०)~संख्यः~। अयमेवोद्देशकश्लोकप्रथमोक्तगुरुवारान्तान्तः सृष्ट्यादिर्दिनगणः~। एष प्रथमकालः~।अस्मिन्नेकगुणे सप्तेष्वब्ध्यङ्काद्रिकृताब्ध्यङ्काग्नि~(३९४४७९४५७)~संख्ये दृढदिने दत्ते सप्ताङ्काब्ध्यङ्काद्रिवेदाब्धिग्रहाग्नि~(३९४४७९४९७)~संख्यः~। शुक्रवासरावसानः सृष्ट्यादिर्दिनगणः~। एष द्वितीयकालः~। अथ पूर्वमागते गुणे~४०~चतुर्गुणदृढदिने क्षिप्ते वसुरसाष्टाद्रीन्द्वङ्काद्रिमुनितिथि~(१५७७९१७८६८)~संख्यश्चन्द्रवारान्तान्तः सृष्ट्यादिर्दिनगणः~।अयं तृतीयकालः~। अथ गुणे~४०~पञ्चगुणदृढदिने क्षिप्ते
तत्त्वत्र्यद्यङ्काग्निद्रव्यद्यतिधृति~(१९७२३९७३२५)~संख्य कुजवारान्तान्तः सृष्ट्यादिर्दिनव्रणः~। एष चतुर्थकालः~। अथ गुणे षड्गुणद्दृढदिने क्षिप्ते द्विवसुशैलरसाद्रिवसुरसर्तुत्रिद्वि~(२३६६८७६७८२)~संख्यः~। बुधवारान्तान्तः सृष्ट्यादिर्दिनव्रजः~। एष पञ्चमकालः~। एवमेतेषु पञ्चकालेषु यथास्वभगणेति~(सूर्यसि. अ.~१~श्लो.~५३)~सूर्यसिद्धान्तोक्तप्रकारेणाऽऽनीतं रविभगणशेषं नियुताहताष्टकरक्ष्माभृन्मही~(१७२८०००००)~संमितं भवति~। एवं सूर्यस्य राश्यादिशेषैरन्येषां भगणादिशेषैर्वारकुट्टाकारः कर्तव्यः~। अथ वारकुट्टाकारेवेलाकुट्टाकारयोः को वा विशेषः~। उच्यते\textendash\ वारान्तकालजानितं ग्रहं वा ग्रहभगणादिशेषं वाऽवलम्ब्य वारकुट्टाकारः प्रवर्तते~। वारान्तकालादन्यकालजनितं ग्रहं वा भगणादिशेषं वाऽवलम्ब्य वेलाकुट्टाकारः प्रवर्तत इति~। अथ वेलाकुट्टाकारः प्रदर्श्यते~। तत्राऽऽदौ वेलायां ग्रहभगणादिशेषाणि ज्ञातव्यानीति तदानयनमुच्यते\textendash\ भागानुबन्धजातौ रूपगुणच्छेदगुणस्वांशयुतेष्टदिनवृन्दं तत्रत्येन

\newpage
\thispagestyle{fancy}
\fancyhf{}
\chead{\textbf{महालक्ष्मीमुक्तावलीसहितः\textendash}}
\lhead{\textbf{४०}}
\noindent
च्छेदेन गुणयित्वा तदंशयुक्तं कृत्वा तच्च ग्रहयुगभगणैः संगुण्य तत्रत्यच्छेदनिहतैर्युगभूदिनैर्विभजेत्~। लब्धं भगणा भवन्ति~। तस्मिन्द्वादशादिगुणिते
पूर्ववद्विभक्ते च लब्धं राश्यादयः~। शेषं राश्यादिशेषम्~। उदाहरणम्~। अत्र
कल्पितस्वच्छेदांशसहितोऽभीष्टो दिनगणः~१३५२।~१~।~२~। अत्रोपरिपङ्क्तिस्थो रूपगुणः स चात्र द्वीषुविश्व~(१३५२)~संख्यः~। मध्यमपङ्क्तिस्थोंऽशः~। स पुनरिह रूप~(१)~मितः~।अधःपङ्क्तिस्थच्छेदः~। स पुनरत्र द्विक~२~मितः~। अथ रूपगुणे~(१३५२)~ तत्रत्येन च्छेदेन~(२)~गुणिते~( २७०४)~तत्रत्येनांशेन च~(१)~युक्ते जातो राशिः पञ्चाभ्रम~(२७०५)~संमितः~। अस्मिन्बुधयुगभगणैः खचतुष्करदार्णवै~(४३२००००)~ र्गुणिते~(१११६८५६०००००)~तत्रत्यच्छेदेन द्विकेन~(२)~गुणितैर्युगभूदिनैः षड्बाणरसार्थाग्न्यष्टेषुतिथ्यग्नि~(३१५५८३५६५६)~संख्यैर्विभक्ते लब्धा भगणास्त्रयः~। शेषं दन्तखाग्न्यङ्कखाष्टेन्दुद्विद्वि~(२२१८०९३०३२)~संख्यम्~। तदेव बुधस्य भगणाग्रम्~। एतस्मिन्द्वादशादिगुणिते पूर्ववद्विभक्ते च लब्धं राश्यादयो भवन्ति~। शेषं राश्यादिशेषं भवति~॥~६५~॥~६६~॥~६७~॥~६८~॥
~६९~॥~७०~॥~७१~॥\\
\indent
अत्र यद्भगणशेषं तदधुनोद्दिश्यते\textendash
\begin{quote}
{\ks लङ्कामध्याह्नकाले हिमकिरणतनूजन्मनाे मण्डलाग्रं\\
दन्ताभ्राग्न्यङ्कखाष्टक्षितिनयनभुजासंख्यमासीदमुष्मात्~।\\
पर्यायानस्य यातानभिमतदिवसव्यूहमंशेन युक्तं\\
वेलाकुट्टोपदेशो बलवदवगतश्चेत्त्वया ब्रूहि शीघ्रम्~॥~७२~॥}
\end{quote}
\indent
अथैतत्प्रश्नभङ्गाय सार्धमार्याद्वयमाह\textendash
\begin{quote}
{\ks वेलाकुट्टाकारे छेदाभ्यस्तः कुवासरश्छेदः~।\\
प्रक्षेपभाज्यराशी पुनरस्मिन्वारकुट्टवद्भवतः~॥~७३~॥\\
एभिः संपाद्येते यौ फलगुणकौ तयोः फलं त्वत्र~।\\
उद्दिष्टशेषवशतो गतभगणादिर्भवेदिह गुणस्तु~।।~७४~।।\\
छेदहतो दिवसगणः शिष्टं तच्छेदयोगिनोंऽशाः स्युः~।}
\end{quote}
\indent
अस्मिन् वेलाकुट्टाकारे मण्डलादिशेषानयने\\
\indent
भागानुबन्धजातौ रूपगुणच्छेदसंगुणः सांशः\textendash

\newpage
\thispagestyle{fancy}
\fancyhf{}
\chead{\textbf{कुट्टाकारशिरोमणिः~।}}
\rhead{\textbf{४१}}
\noindent
इत्यत्र यश्छेदस्तेन गुणिता युगभूदिवसा भाजको भवति~। वारकुट्टाकारे भगणादिशेषेषु क्षिप्तेषु सत्सु तत्र यौ क्षेपभाज्यावुक्तौ तावत्रापि समानौ~। एतैर्भाजकक्षेपभाज्यैर्यौ फलगुणावुत्पद्येते तयोः फलं त्विह भगणशेष उद्दिष्टे सति ग्रहभुक्तभगणा भवन्ति~। राशिशेषे तूद्दिष्टे तत्फलं ग्रहभुक्तराशयः~। भागशेषे पुनरुद्दिष्टे तत्पुनर्ग्रहभुक्तभागाः~। लिप्ताग्रे पुनरुद्दिष्टे तत्तु ग्रहभुक्त~(क्ति)~लिप्ताः~। विलिप्ताग्रे तूद्दिष्टे सति तत्कलं ग्रहभुक्त~(क्ति)~विलिप्ता भवन्ति~। अत्र यो गणस्तस्मिंस्तु भगणाद्यग्रानयने भागानुबन्धजातावित्यत्र यश्छेदस्तेन विहृते सति लब्धमिष्टदिनानि भवन्ति~। शिष्टं तच्छेदसंबन्धिनोंऽशा भवेयुः~। उदाहरणम्\textendash\ बुधस्योद्दिष्टं भगणशेषमिदम्~(२२१८०९३०३२)~। अयमृणक्षेपः~। अत्र भगणशेषोद्दिष्टत्वाद्बुधस्य युगभगणा इमे~(४३२००००)~भाज्यराशिः~। इह चक्रशेषानयेन भागानुबन्धजातावित्यत्र च्छेदौ द्वि~( २)~संमितः~। अतो द्विगुणिता भूदिवसा एते~(३१५५८३५६५६)~भाजकराशिः~। एवमेतैः क्षेपभाज्यभाजकैः पूर्ववदानीतं फलं त्रि~(३)~संमितम्~। एतद्बुधभुक्तभगणाः~। गुणः पञ्चाभ्रभ~(२७०५)~ संमितः~। एतस्मिंस्त्विह चक्रशेषानयने भागानुबन्धजातावित्यत्र द्वि~(२)~ संमितेन च्छेदेन विभक्ते लब्धं द्वीषुविश्व~(१३५२)~संख्यं जातम्~। एतदभीष्टदिनगणः~। शिष्टं तच्छेद~(२)~संबन्ध्यंशो रूप ~(१)~मितो जातः~। एवं च्छेदांशसहितो दिनगणोःऽयं~१३५२~।~१~।~२~।~इत्थं वारद्व्यंशे कुट्टाकारः प्रदर्शितः~। एवमनेन न्यायेन वारत्र्यंशवारचतुरंशवारपञ्चांशवारषडंशादिषु कुट्टाकाराे योज्यः~॥~७३~॥~७४~॥\\
\indent
अथ वारषष्ट्यंशे लिप्ताशेषाश्रयणेनोद्देशकः\textendash
\begin{quote}
{\ku लिप्तायामर्धरात्रात्प्रभृति विघटिका सप्तशत्यां गतायां\\
सैकायां खाभ्रतर्कत्रिगुणकृतशराष्टाभ्रशैलैकनागाः~।\\
लिप्ताग्रं सूर्यसूनाेः समभवदमुना वासरानंशयुक्तान्\\
लिप्तान्तं सूर्यसूनुं कथय विघटिका कुट्टवेदिन्ममाऽऽशु~॥~७५~॥}
\end{quote}
\indent
अत्रेदं शनैश्चरस्य कलाशेषम्~(८१७०८५४३३६००)~। अयमृणक्षेपः~। तस्य युगभगणा भुजङ्गषट्पञ्चरसवेदनिशाकराः~(१४६५६८)~(सूर्यसि.~अ~१ श्लो.~३२)~ भाज्यराशिः । वारस्य खखड्रामांशो~(३६०० )~विनाडीतिखखषड्रामगुणिता युगदिवसा इमे~(५६८०५०४१८०८००)~भाजकराशिः~। एभिः क्षेपभाज्यभाजकैः पूर्ववदानीतो गुणो भूखाङ्काद्रिखे\textendash\\
\indent
६

\newpage
\thispagestyle{fancy}
\fancyhf{}
\chead{\textbf{महालक्ष्मीमुक्तावलीसहितः\textendash}}
\lhead{\textbf{४२}}
\noindent
ष्वाकृति~(२२५०७९०१)~संख्यः~। अस्मिन्खखषड्रामैर्विभक्ते लब्धमभीष्टदिनगणो द्वीषुद्विषट्~(६२५२)~संख्यः~। शिष्टा विनाडिकाः सैकसप्तशत~७०१~।~३६००~संख्यः~। एवं सच्छेदांशसहितोऽभीष्टदिनगुणः~६२५२~।~७०१~।~३६००~। फलं कृतवेदांर्थार्क~(१२५४४)~संख्यम्~। एतत्कालात्मकं सौरमध्यम्~। अथानुकलासु क्रमादनुपातादारोपितासु षड्राशयः~(६)~। एकोनत्रिंशद्भागा~(२९ )~श्चतस्रः~(४)~कलाश्च~। एवमिदं राश्यादिकलान्तं सौरमध्यम्~॥~७५~॥\\

\indent
अथ ग्रहाश्रयणेन वारवेलाकुट्टाकारानयने प्रथमतो ग्रहस्य भगणादिशेषं
ज्ञातव्यम्~। अतस्तदुद्दिष्टग्रहानुगुण्येन धीमताऽप्यूह्यमित्यार्ययाऽऽह\textendash
\begin{quote}
{\ks खचरावलम्बकुट्टे परिवर्ताद्यग्रमादितो ज्ञेयम्~।\\
तत्परिवर्ताद्यग्रं सुधियोह्यं कथित\textsuperscript{१}खेचरेन्द्रवशात्~॥~७६~॥}
\end{quote}
\indent
अत्र परिवर्ताद्यग्रमूह्यमित्येतत्परिवर्ताद्यग्रेणानन्तरं तद्वशाद्वेलावारकुट्टाकारावप्यूह्यावित्यस्योपलक्षणम्~। अत्रायमूहेन सिद्धोऽर्थः\textendash\ वारकुट्टाकारे भगणात्मके ग्रह उद्दिष्टे सति ग्रहस्योद्दिष्टान्
भगणांश्चतुर्युगदिनैर्गुणयित्वा तस्य चतुर्युगभगणैर्विभज्य लब्धं कुत्रचिन्निधाय शेषं तस्य
चतुर्युगभगणेभ्यस्त्यजेत्~। ततस्तच्छेषं भवति~। अत्रैवमागतं भगणः शेषं क्षयक्षेपः~। ग्रहस्य
चतुर्युगभगणा भाज्यराशिः~। चतुर्युगदिनानि भाजकराशिः~। एभिः पूर्ववद्गुणफले कुर्यात्~। तयोर्गुणोऽभीष्टदिनगणो भवति~। फलं ग्रहभुक्तभगणा भवन्ति~। अथ परिवर्तादौ राश्यन्ते ग्रह उद्दिष्टे तं ग्रहं राशीकृत्य भूदिनैः संगुण्य राशीकृतैस्तस्य युगपरिवर्तैश्छित्त्वा फलं क्वचित्स्यापयित्वाऽवशेषं स्वभागहाराज्जह्यात्~। तच्छेपं तु राशिशेषम्~। अत्र राशिशेषमृणक्षेपः~। राशीकृता युगपरिवर्ता भाज्यः~। भूदिनानि भाजकः~। एतैः प्रागुक्तवद्गुणफले कुर्यात्~। तत्र गुणोऽभीष्टदिनगणः~। फलं तु गतराशिगणः~। अथ चक्रादौ भागान्ते ग्रह उद्दिष्टे तं भागीकृत्य क्षितिवासरैस्ताडयित्वा भागीकृतैस्तस्य युगचक्रैर्व्यवच्छिद्याऽऽप्तं कुत्रचिद्विन्यस्य शिष्ट स्वहारकाच्छोधयेत्~। तच्छेषं तु भागशेषम्~। अत्र भागशेषमृणक्षेपः~। भागीकृतग्रहस्य युगभगणा भाज्यः~। क्षितिवासरा भाजकः~। एभिरुक्तवद्गुणफले कुर्यात्~। तयोर्गुणोऽभीष्टदिवसचयः~। इह फलं तु गतभागगणः~। अथ पर्यायादौ कलान्ते ग्रह उद्दिष्टे तं कलीकृत्य भूदिवसैः संताड्य कलीकृतैस्तस्य युगपर्यायैर्भक्त्वा

\footnotetext{
१ क. तकेन्द्र... ।}

\newpage
\thispagestyle{fancy}
\fancyhf{}
\chead{\textbf{कुट्टाकारशिरोमणिः~|}}
\lhead{\textbf{४३}}
\noindent
लब्धं क्वचिद्दत्त्वाऽवशिष्टं स्वच्छेदाच्छोधयेत्~। तच्छेषं तु कलाशेषम्~। अत्र कलाशेषमृणक्षेपः~। कलीकृतास्तस्य युगपर्याया भाज्यः~। भूदिवसा भाजकः~। एतैः प्रागुक्तवद्गुणफले विदध्यात्~। तयोर्गुणोऽभीष्टदिवससमूहः अत्र फलं तु गतकलासमूहः~। अथ पर्यायादौ विलिप्तान्ते ग्रह उद्दिष्टे तं विलिप्तीकृत्य भूसावनैर्वर्धयित्वा विलिप्तीकृतैस्तस्य युगपर्यायः प्रविभज्य लब्धं फलमन्यत्र कुत्रचित्पालयित्वा शिष्टं स्वभाजकात्संशोधयेत्~। तच्छेषं तु विकलाशेषम्~। इह विकलाशेषमृणक्षेपः~। विलिप्तीकृतग्रहस्य युगपर्याया भाज्यः~। भूसावनदिनानि भाजकः~। एभिः पूर्वोक्तमार्गेण गुणफले कुर्यात्~। तत्र गुणोऽभिमतवासरनिवहः~। फलं तु ग्रहभुक्तविकलावृन्दम्~। अत्र सर्वत्र यद्यत्फलं तत्तत्तत्र तत्र शेषाभावेऽभीष्टदिनानि भवन्ति~। तत्र तत्र शेषे सति तु सैकं तत्तत्फलमभीष्टदिनानि भवन्ति~। अथ ग्रहस्य राशिस्थानमात्र उद्दिष्टे पुनरुद्दिष्टांस्तस्य राशीन दृढदिनैः संवर्ध्य द्वादशभि~(१२)~र्विभजेत्~। अत्र लब्धं भगणशेषम्~। अथ राश्यादौ भागान्ते ग्रह उद्दिष्टे तं भागीकृत्य दृढदिनैर्हत्वा खाङ्गाग्निभिः~(३६०)~विभजेत्~। लब्धं भागशेषम्~। अथ राश्यादौ कलान्ते ग्रह उद्दिष्टे तं कलीकृत्य दृढदिनैर्हत्वा खखनृपनेत्रैः~(२१६००)~हरेत्~। लब्धं भगणशेषम्~। अथ राश्यादौ विलिप्तान्ते ग्रह उद्दिष्टे तं विलिप्तीकृत्य दृढदिनैर्हत्वा खाभ्राङ्गाङ्कार्कैः~(१३९६००)~विभजेत्~। इहापि लब्धं भगणशेषम्~। अत्र सर्वत्र भगणशेषमृणक्षेपः~। ग्रहस्य दृढभगणा भाज्यः~। दृढदिवसा भाजकः~। एभिर्गुणफले कुर्यात्~। तयोर्गुणोऽभिप्रेतदिनव्रजः~। फलं गतभगणाः~। अत्र सर्वत्र दिवसाश्चतुर्युगभगणाश्चापवर्तिताः क्रमाद् दृढदिवसा दृढभगणाश्च भवन्ति~। अथ ग्रहस्य भागस्थानमात्र उद्दिष्टे तानुद्दिष्टान् भागान् दृढदिनैर्हत्वा त्रिंशता विभजेत्~। लब्धं राशिशेषम्~। अथ भागादौ लिप्तास्थानान्ते ग्रह उद्दिष्टे तं लिप्तीकृत्य दृढदिनैर्हत्वा खाभ्राष्टभूभि~(१८००)~र्विभजेत्~। लब्धं राशिशेषम्~। अथ भागादौ विलिप्तान्ते ग्रह उद्दिष्टे तं विलिप्तीकृत्य दृढवासरैर्हत्वा खखाभ्राष्टदिग्भि~(१०८०००)~र्विभजत्~। लब्धं राशिशेषम्~। अत्र सर्वत्र राशिशेषमृणक्षेपः~। राशि\textsuperscript{१}शेषदृढमण्डला भाज्यः~। दृढवासरा भाजकः~।

\footnotetext{
१~ख.~ शिभूतादृ~।}

\newpage
\thispagestyle{fancy}
\fancyhf{}
\chead{\textbf{महालक्ष्मीमुक्तावलीसहितः\textendash}}
\lhead{\textbf{४४}}
\noindent
तैर्गुणफले कुर्यात्~। तयोर्गुणो वाञ्छितदिनौघः~। फलं गतराशिगणः~। तत्र सर्वत्र चतुर्युगदिवसा राशीकृताश्चतुर्युगभगणाश्चापवर्तिताः क्रमाद् दृढदिवसा राशीकृता दृढमण्डलाश्च भवन्ति~॥~७६~॥\\
अथ राश्यादिलिप्तान्तग्रहाश्रयवारकुट्टाकारावलम्बनेनोद्देशकः\textendash
\begin{quote}
{\ku चापे भागा विंशतिर्लिप्तिके द्वे आसीत्सूरेर्मध्यमं सौम्यवारे~।\\
भूवोऽप्येतज्जीववारे कदा स्याद्ब्रूयाः कालौ तौ गतान् पर्ययांश्च~॥~७७~॥}
\end{quote}
अत्रेदं राश्यादिकलान्तं गुरोर्मध्यं~(८२०~।~२)~। एतत्कलीकृतं द्रव्यभ्रषट्कतिथि~(१५६०२)~संख्यं जातम्~। एतदत्रत्यैः सप्तेष्वब्ध्यङ्काद्रिकृताब्ध्यङ्काग्निभि~(३९४४७९४५७)~र्दृढदिनैर्हिहतं चक्रकलाभि~(२१६००)~र्विभक्त चार्धाेत्तरयाेगेन सह षडर्थाग्न्यष्टाग्न्यङ्काब्ध्यष्टद्वि~(२८४९३८३५६)~संख्यं जातम्~। अयमिह ऋणक्षेपः~। गुराेश्चतुर्युगभगणचतुर्थांशमिता अत्रत्या एते~९१०५५~दृढभगणा भाज्यराशिः~। इह पूर्वोक्तान्येतानि~(३९४४७९४५७)~दृढदिनानि भाजकराशिः~। एतैः पूर्ववदानीतो
गुणस्त्रिद्विखाष्टाग्नीन्द्रतत्त्व~(२५१४३८०२३)~संख्यः~। अयं सृष्ट्यादिरिष्टदिनगणः~। अयमेव बुधवारान्तान्तः प्रथमकालः अत्र फलं सप्तत्रिखाष्टार्थ~(५८०३७)~संख्यम्~। एतदेवात्र गुरुभक्तभगणाः~। अथैकदिनदृढयुक्तः प्रथमकाल एव गुरुवारान्तावसानः सृष्ट्यादिर्द्वितीयः कालः~। स च खाष्टाब्ध्यद्रीन्द्वङ्कार्थाब्धिषट्~(६४५९१७४८०)~संख्यः~।
अस्मिन्काले गुरोर्गतभगणाद्व्यङ्काखाङ्केन्द्राः~(१४९०९२)~। एवं ग्रहाश्रयवारकुट्टाकारः प्रदर्शितः~। अनया~(रीत्या)~ग्रहाश्रयवेलाकुट्टाकारोऽप्युन्नेयः~। अत्र यौ ग्रहाश्रयेण वारवेलाकुट्टाकारौ तौ कदाचिद्भवतः~। कदाचिदव्यभिचरतः~। तस्मात्तावसमीचीनौ~॥~७७~॥\\
\indent
अथ सूर्याब्दावसानचन्द्रादिष्टग्रहगराश्यादिमध्याद्गुतार्कवत्सरांश्चन्द्रादीष्टग्रहग
तभगणादींश्च यो वदति स निरग्रकुट्टाकारे प्रवीण इत्येकं प्रश्नमार्यया स्पष्टयति\textendash
\begin{quote}
{\ks सूर्याब्दपूर्तिकाले चन्द्रादीष्टग्रहं समालोक्य~।\\
अर्केष्टयातभगणानाचष्टे यः स दक्षिणः कुट्टटे~॥~७८~॥}
\end{quote}
अवतारिकया गतार्थेयम्~।।~७८~।।\\
\indent
अथैतत्प्रश्नभञ्जनाय सार्धामार्यामाह\textendash
\begin{quote}
{\ks दिनकरभगणाश्छेदश्चन्द्रादीष्टग्रहस्य चक्राणि~।}
\end{quote}

\newpage
\thispagestyle{fancy}
\fancyhf{}
\chead{\textbf{कुट्टाकारशिरोमणिः~।}}
\rhead{\textbf{४५}}
\begin{quote}
{\ks भाज्यक्षेपस्तूह्यो ग्रहमध्यमसिद्धिवासनां दृष्ट्वा~॥~७९~॥\\
तैर्जनिते गुणकाप्तीये चार्काब्दा गतेष्टभगणाश्च~।}
\end{quote}
\indent
अत्र क्षेपोत्पादनं विनाऽन्योऽर्थः सुबोधः~। अतस्तदेव प्रदर्श्यते~। ग्रहस्य राशिस्थानमात्र उद्दिष्टे सत्युद्दिष्टान् राशीन् युगरविभगणैः खचतुष्करदार्णवैः~(४३२००००)~संगुण्य द्वादशभिर्विभज्य लब्धं भगणशेषात्मकः क्षयक्षेपः~(१२९६०००)~। अथ राश्यादौ भागान्ते ग्रह उद्दिष्टे तं भागीकृत्य पूर्ववत्संगुण्य चक्रभागै~(३६०)~र्विभज्य लब्धमृणक्षेपः~। अथ राश्यादौ विकलान्ते ग्रह उद्दिष्टे तं विकलीकृत्य चक्र\textsuperscript{१}~(वि)~कलाभि~(१२९६०००)~र्विभज्य लब्धमृणक्षेपः~॥~७९~॥\\
\indent
अत्र रात्रिस्थानमात्रचन्द्रमध्यावलम्बनेनोद्देशकः\textendash
\begin{quote}
{\ku सूर्याब्दपूर्तौ मृगलाञ्छनस्य भपञ्चकं दृष्टमतोऽरुणाब्दान्~।\\
यातानथेन्दोर्भगणांश्च तूर्णमाख्यातु कुट्टे कुशलो भवांश्चेत्~।।~८०~।।}
\end{quote}
\indent
अत्र भाज्यश्चतुर्युगभगणाः~। भाजिको रविभगणाः~। अत्रोक्तप्रक्रियानीतो नियुताहतधृति~(१८०००००)~संख्योऽयमृणक्षेपः एभिरानीताः सूर्याब्दाः खत्रयतिथि~(१५०००)~संख्याः~। चन्द्रगतभगणा रदेष्वभ्रनख~(२००५३२)~संख्याः~॥~८०~॥\\
\indent
अथ फलज्ञातगुणानयनमार्ययाऽऽह\textendash
\begin{quote}
{\ks लब्धे भाजकगुणिते वारक्षेपेण संस्कृते भक्ते~।\\
भाज्येन लभ्यते यत्तद्गुणकारो भवेदिति ज्ञेयम् ~॥~८१~॥}
\end{quote}
\indent
अस्यास्त्वर्थ उदाहरणेन व्यज्यते~। निरग्रविषये पूर्वलिखितोद्देशकश्लोकोऽयम्\textendash
\begin{quote}
{\qt चन्द्रेशाः~(१११)~केन गुणिता रूप~(१)~युक्ता द्विखाग्निभिः~(३०२)~।\\
भक्ताः शुद्धा गुणफले ब्रूहि तृर्णं निरग्रवितु~॥}
\end{quote}

\indent
अत्र कुट्टाकारगणितसिद्धे गुणफले शरधृति~(१८५)~गजतर्क~(६८)~संख्ये~। एवमागताल्लब्धाद्गुणानयनार्थमियमार्या~। लब्धमिद~(६८)~मेतदनेन भाजकेन~(३०२)~ गुणितं~(२०५३६)~इदमनेन धनक्षेपेण~१~धनर्णवैपरीत्येन संस्कृतम्~(२०५३५)~। एतदनेन ~(१११)~भाजकेन विभज्य लब्धम्~(१८५)~। इदं कुट्टाकारगणितसिद्धगुणकारतुल्यमेव~॥~८१~॥
\footnotetext{
१~क.~क्रकलाभि~(२१६००)~वि.}

\newpage
\thispagestyle{fancy}
\fancyhf{}
\chead{\textbf{महालक्ष्मीमुक्तावलीसहितः\textendash}}
\lhead{\textbf{४६}}
\indent
अथाऽऽर्यया विकलाशेषाद्विकलाद्यानयनमुच्यत इति प्रतिजानीते\textendash
\begin{quote}
{\ks विकलाशेषाद्विकलाः कलाः कलाग्रं लवाग्रं च~।\\
भानि तदग्रं भगणाग्रमपि यथा स्युस्तदुच्यते कर्म~॥~८२~॥}
\end{quote}
\indent
अथ प्रतिज्ञातमर्थमार्याचतुष्टयेन सार्धेनाऽऽह\textendash
\begin{quote}
{\ks विश्वघ्नाद्विकलाग्रात्खसिद्धभक्तावशेषवेदांशे~।\\
षष्टेस्त्यक्ते विकलास्तद्युगदिविसाहृ~(ह)~तिः सविकलाग्रा~॥~८३~॥\\
खाङ्गैर्हृतात्कलाग्रं तद्वेदांशात्त्रिचन्द्रसंगुणितात्~।\\
खरसविहृतावशेषे खरसेभ्यः शोधिते कलाशेषम्~॥~८४~॥\\
तद्युगदिवसाभ्यासः सकलाग्रः षष्टिभाजितोंऽशाग्रम्~।\\
विश्वहततद्युगांशात्खगुणाप्ताच्छेषहीनहारोंऽशाः~॥~८५~॥\\
तद्भूदिनसंवर्गः सांशाग्रः खाग्निभिर्हृतो भाग्रम्~।\\
तत्तुर्यांशेऽर्काप्ते शेषे हारोच्चशोधिते भानि~॥~८६~॥\\
तद्भूमिसावनवधः सह शेषोऽर्कैर्हृतो भगणशेषः~।}
\end{quote}
\indent
त्रयोदश~(१३)~गुणिताद्विकलाशेषाच्चत्वारिंशदधिकशतद्वयेन~(२४०)~ भुक्ताद्योऽवशेषस्तस्य यश्चतुर्थांशस्तस्मिन्षष्टे~(६०)~स्त्यक्ते यच्छिष्यते तद्विकला भवन्ति~। अथ तासां गतभूदिनानां च योऽभ्यासः स विकलाशेषयुक्तः षष्ट्या (६०) भक्तश्च कार्यः~। अत्र यल्लभ्यते तद्विकलाशेषं भवति~। अथ तस्य चतुर्थांशात्त्रयोदशगुणितात्षष्ट्या भक्ताद्यदवशिष्यते तस्मिन् षष्ट्यास्त्यक्ते योऽवशेषः स लिप्ता भवन्ति~। अथ तासां युगदिनानामपि यो वधः स लिप्ताशेषयुक्तषष्ट्या भक्तश्च कार्यः~। लब्धं भागशेषं भवति~।
अथ तस्य चतुर्थांशात्त्रयोदशगुणात्त्रिंशा~(३०)~द्भक्ताद्यच्छिष्टं तस्मिंस्त्रिंशतस्त्यक्ते शेषो भागा भवन्ति~। अथ शेषाणां युगभूदिनानां चाभ्यासः सभागशेषस्त्रिंशद्भक्तश्च कार्यः~। लब्धं राशिशेषं भवति~। अथ राशिशेषचतुर्थांशे द्वादशभक्ते यः शेषस्तस्मिन् द्वादशभ्यस्त्यक्ते यच्छिष्यते तद्राशयो भवन्ति~। अथावशेषाणां भूदिनानां च यः संवर्गः स राशिशेषयुक्तो द्वादशभिर्विभक्तश्च कार्यः~। लब्धं भगणशेषं भवति~। ननु विकलाशेषाद्दिवसानित्यादिना पूर्वोक्तेन सर्वमेतत्सिद्धं तत्किमनेन सिद्धसाधनेन~। सत्यम्~। तत्र कुट्टाकारगणितमत्र विपरीतगणितमित्यदोषः~॥~८२~॥~८३~॥~८४~॥~८५~॥~८६~॥

\newpage
\thispagestyle{fancy}
\fancyhf{}
\chead{\textbf{कुट्टाकारशिराेमणिः~।}}
\rhead{\textbf{४७}}
\indent
पूर्वं विकलाशेषाद्दिवसानित्येवमादिना ज्ञातग्रहविकलाग्राज्जिज्ञासितग्रहभगणादिशेषानयनमुक्तम्~। इदानीं प्रकारान्तरेण तदेव विवक्षुस्तद्विषयमनुयोगमार्थया स्पष्टयति\textendash
\begin{quote}
{\ks \textsuperscript{१}पूर्वं विकलाशेषाग्रादन्यस्येष्टभगणशेषादीन्~।\\
अविलम्बितं भणति यः कुट्टट\textsuperscript{२}कविदां सिन्धुरः स भवेत्~॥~८७~॥}
\end{quote}
\indent
स्पष्टोऽर्थः~॥~८७~॥\\
\indent
अथैतत्प्रश्नभञ्जनायाऽऽर्याद्वयमाह\textendash
\begin{quote}
{\ks कथितविकलावशेषस्थिरगुणगुणितेष्वभीष्टचक्रेषु~।\\
भक्तेषु भूमिदिवसैर्यः शेषस्तद्गुणोक्तविकलाग्रे~॥~८८~॥\\
क्षितिदिनभक्ते शेषे वेदहृतेऽभष्टिखचरभगणाग्रम्~।\\
राश्यादिशेषगणने चैवं रव्यादि (१२३६०२१६००\textendash
१२९६०००) गणितपरि\textsuperscript{३}वर्तैः~॥~८९~॥}
\end{quote}
\centering
\textbf{इति श्रीदेवराजविरचिते कुट्टाकारशिरोमणौ निरग्रपरिच्छेदो द्वितीयः~॥~२~॥\\}
\rule{0.2\linewidth}{1.0pt}\\

\vspace{1cm}
\justifying

\indent
उद्दिष्टस्य ग्रहविकलाशेषस्य यः स्थिरगुणः पूर्वं रविभृगुविदां नवाम्वरेत्येवमादिना पठितस्तेन गुणितेष्वभीष्टग्रहयुगभगणेषु यः शेषस्तद्गुणितोद्दिष्टविकलावशेषस्थिरगुणगुणगुणितेषु जिज्ञासितस्य द्वादशगुणितयुगचक्रेषु भूदिनभक्तेषु यः शिष्यते तेन गुणिते तूद्दिष्टविकलाशेषे भूदिनभक्ते यच्छिष्टं तद्वेदविहतं जिज्ञासितस्य राशिशेषं भवति~। अनया दिशा भागादिशेषाश्चाऽऽनेयाः~॥~८८~॥~८९~॥\\
\indent
अथ संप्रदायाविच्छेदार्थं भास्कराचार्यकृतलीलावतीगतानि सोदाहरणानि
निरग्रकुट्टकसूत्राणि कतिचिल्लिख्यन्ते\textendash
\hspace{1cm}
\begin{quote}
{\qt 
भाज्यो हारः क्षेपकश्चापवर्त्यः केनाप्यादौ संभवे कुट्टकार्थम्~।\\
येन च्छिन्नौ भाज्यहारौ न तेन क्षेपश्चैतद्दुष्टमुद्दिष्टमेव~॥\\
परस्परं भाजितयोर्ययोर्यः शेषस्तयोः स्यादपवर्तनं सः~।\\
स्वेनापवर्तेन विभाजितौ यौ तौ भाज्यहारौ दृढसंज्ञकौ स्तः~॥}\\
\end{quote}
\footnotetext{
१~ख. एवास्य त्रिलिप्ताग्राद्~।~~२~ ख. कवेदी स सिन्धुरशमाश्चग्र~।~~३~
ख. रिवेत्रैः }

\newpage
\thispagestyle{fancy}
\fancyhf{}
\chead{\textbf{महालक्ष्मीमुक्तावलीसहितः\textendash}}
\lhead{\textbf{४८}}

\begin{quote}
{\qt 
मिथो भजेत्तौ दृढभाज्यहारौ यावद्विभाज्ये भवतीह रूपम्~।\\
फलान्यधोऽधस्तदधो निवेश्यः क्षेपस्तदन्ते खमुपान्तिमेन~॥\\
स्वोर्ध्वे हतेऽन्त्येन युते तदन्त्यं त्यजेन्मुहुः स्यादिति राशियुग्मम्~।\\
ऊर्ध्वो विभाज्येन दृढेन तष्टः फलं गुणः स्यादधरो हरेण~॥\\
एवं तदेवात्रं यदा समास्ताः स्युर्लब्धयश्चेद्विषमास्तदानीम्~।\\
यदागतौ लब्धिगुणौ विशोध्यौ स्वतक्षणाच्छेषमितौ तु तौ स्तः~।}\\
\end{quote}
\centering
\textbf{श्रीभास्करीयलीलावत्यां कुट्टकाध्याये श्लो.~१\textendash\ ५~॥~इति करणसूत्राणि पञ्च~॥\\}

\rule{0.2\linewidth}{1.0pt}\\

\vspace{1cm}
\justifying
\indent
एषामर्थः\textendash\ संभवे सति कुट्टकार्थमादौ केनाप्यपवर्तितेन भाज्यो भाजकः क्षेपकश्चापवर्तनीयः~। येनापवर्तकेन भाज्यभाजकौ विभक्तौ तन विभक्तः क्षेपो न शुध्येच्चेत्तद्दुष्टमुद्दिष्टमेव\textendash\ खिलमित्यर्थः\textendash\ ययोः परस्परभक्तयोर्यच्छेषं तत्तयोरपवर्तकं स्यात्~। यौ भाज्यभाजकौ स्वेनापवर्तकेन भक्तौ तौ दृढसंज्ञौ भवेताम्~। एतदुपलक्षणम्~। भाज्यभाजकयोर्योऽपवर्तकस्तद्विभक्तः क्षेपोऽपि दृढसंज्ञ इति\textendash\ तौ दृढभाज्यभाजकौ परस्परं तावद्विभजेद्यावद्विभाज्ये रूपं भवति~। अत्र विभाज्यशब्देन भाज्यभाजकयोः परस्परभजनात्तावुभावप्यभिधीयेते~। परस्परभजने क्रियमाणे यानि फलानि लभ्यन्ते तान्यधोऽधो निधेयानि~। तेषामधो दृढक्षेपस्तथा निधेयः~। अन्ते खं निधेयम्~। क्षेपस्याधः शून्यं निधेयमित्यर्थः\textendash\ उपान्तिमेन 'स्वोर्ध्वे हतेऽन्त्येन युते तदन्त्यं त्यजेन्मुहुः स्यादिति राशियुग्ममिति ' वल्ल्युपसंहार उच्यते\textendash\ प्राक्स्पष्टार्थमभिहितः~। ऊर्ध्वराशिर्दृढेन भाज्येन तनूकृतः फलं भवेत्~। अधोराशिर्दृढभाजकेन गुणः स्यात्~। एवं पङ्क्तौ समायां वेदितव्यम्~। विषमायां तु तस्यामेवमागतौ लब्धिगुणौ क्रमाद्दृढभाज्यभाजकाभ्यां विशो\textsuperscript{१}ध्य सिद्धेष्टौ लब्धिगुणौ भवत इति ज्ञेयम् \textendash\ अत्र कुट्टाकारे निरग्रे समधिकवपुषीत्यादिनियमो नास्ति ॥~१~॥~२~॥~३~॥~४~॥~५~॥\\
उद्देशकः\textendash
\begin{quote}
{\ku एकविंशतियुतं शतद्वयं यद्गुणं गणक पञ्चषष्टियुक~।\\
पञ्चवर्जितशतद्वयोद्धृतं शुद्धिमेति गुणकं वदाऽऽशु तम्~॥~९०~॥}
\end{quote}

\textbf{श्रीभास्करीयलीलावत्यां कुट्टकाध्याये\textendash }

\footnotetext{
१~क. ष्ट्य शिष्टौ ल.~।}

\newpage
\thispagestyle{fancy}
\fancyhf{}
\chead{\textbf{कुट्टाकारशिरोमणिः~।}}
\rhead{\textbf{४९}}
\indent
\textbf{न्यासः\textendash\ }भाज्यः~२२१~। हारः~१९५~। क्षेपः~६५~। भाज्यभाजकयोः
परस्परभक्तयोः शेषं १३~। अनेनापवर्तिता दृढसंज्ञा भाज्यभाजकक्षेपाः क्रमेणेमे~१७\textendash१५\textendash५~। परस्परभजनार्थमुपर्यधोभावेन विन्यस्तौ भाज्यभाजकौ~१७~।~१५~। अनयोः परस्परभक्तयोर्लब्धे क्रमेणोपर्यधोभावेन निहिते~१~।~७~। अनयोरधः क्षेपो दत्तः~१~।~७~।~५~। एषामधः शून्यं निक्षिप्तम्~१~।~७~।~५~।~०~। अथ वल्ल्युपसंहारे कृते जातौ राशी~४०~।~३५~। इमौ दृढभाज्यभाजकाभ्यां~१७~।~१५~तष्टौ जातौ लब्धिगुणौ~६~।~५~। अत्रापि त्रैराशिकमार्गः पूर्ववद्योज्यः~। नन्वेवं मतिकल्पनायासमन्तरेण कुट्टकगणिते विद्यमाने किमर्थमार्यभटाचार्यादिभिर्मतिकल्पनाऽङ्गाकीकृता~। अत्रोच्यते\textendash\ साग्रे तावदवश्यं मतिकल्पनया भवितव्यम्~। अतः साग्रनिरग्रयोर्द्वयोरपि साऽस्त्वित्यङ्गीकृतेत्यदोषः~॥~९०~॥\\

\indent
\centering
\textbf{इत्यत्रिकुलाभरणस्य स्कन्धत्रयवेदिनः सिद्धान्तवल्लभ इति प्रसिद्धापरनाम्नः
श्रीवरदराजाचार्यस्य तनयेन देवराजेन विरचितायां कुट्टाकारशिरोमणिटीकायां महालक्ष्मीमुक्तावल्यां निरग्रपरिच्छदो द्वितीयः~।।}\\

\vspace{1cm}
\indent
\textbf{अथ मिश्रश्रेढीमिश्रकुट्टाकाकारपरिच्छेदस्तृतीयः~।}\\

\vspace{2mm}
\justifying
\indent
\textbf{अथ मिश्रश्रेढीमिश्रकुट्टाकारपरिच्छेदो व्याख्यायते~। }\\
अस्मिन्परिच्छेदे प्रश्नतद्भङ्गेषु यथासंभवमुपकरणानां वर्गवर्गमूलघनघनमूलसंकलितचितिधनवर्गचितिघनघनचितिघनश्रेढीफलानामानयनविधिरार्यभटीये गणितसारे द्रष्टव्यः~। इह तु ग्रन्थविस्तरभयान्न प्रदर्श्यते~।\\
\indent
अत्राऽऽदावार्यया प्रश्नत्रयं व्यञ्जयति\textendash
\begin{quote}
{\ks अथ संकलितघनैक्याद्घनपदयोगाच्च वर्गघनयोगात्~।\\
विकलावशेषजाताद्दिनमभ्यूह्यं विवेकिना पुंसा~॥~१~॥}
\end{quote}

\indent
अत्र पदशब्देन विकलाशेषो~(शेषरूपो)~गच्छ उच्यते\textendash\ विकलावशेषजाताद्वर्गघनयोगाच्च विवेकिना पुंसा दिनमप्यूह्यम्\textendash\ अत्र यद्यपि संकलितादियोगो मुख्यवृत्त्या विकलावशेषजातो न भवति तथाऽपि योगिनां तज्जातत्वमवलम्ब्य तथोक्तम्~॥~१~॥\\
\indent
एषां भङ्गार्थमार्यामाह\textendash
\begin{quote}
{\ks कथितमिह यद्यदैक्यं तस्मिंस्तस्मिंश्च विहितघनमूले~।\\
पूर्णतया घनमूलं यद्यत्तत्तद्विंलिप्तिकाशेषम्~॥~२~॥}
\end{quote}

\newpage
\thispagestyle{fancy}
\fancyhf{}
\chead{\textbf{महालक्ष्मीमुक्तावलीसहितः\textendash}}
\lhead{\textbf{५०}}
\indent
इह यद्यदैक्यं कथितं तस्मिंस्तस्मिंश्च विहितघनमूले सति पूर्णतया यद्यद्घनमूलं तत्तदव विलिप्तिकाशेषं भवेत्~। विलिप्तिकाशेषाज्जातादिष्टदिनानयनं प्रागेवोक्तम्~। अतस्तदत्र न वक्तव्यम्~॥~२~॥\\
\indent
अथ प्रश्नान्तरमार्ययाऽऽह\textendash
\begin{quote}
{\ks घनचितिघनवर्गैक्याद्विपूर्वलिप्ताग्रसंभवाद्दिवसान्~।\\
सद्यः समीक्षते यः स वेद~(एव)~गणितानि वेद वसुमत्याम्~॥~३~॥}
\end{quote}

\indent
यो विपूर्वलिप्ताग्रसंभवाद्घनचितिघनवर्गयोरैक्याद्दिवसान् सद्यः समीक्षते स एव वसुमत्यां गणितानि वेद~। विपूर्वं लिप्ता विलिप्ताः~॥~३~॥\\
\indent
एतत्प्रश्नभङ्गार्थमार्यामाह\textendash
\begin{quote}
{\ks घनचितिघनवर्गैकये कृतवर्गपदेऽत्र यो भवेच्छेषः~।\\
एतस्य वर्गमूलं यत्तद्विकलावशेषः स्यात्~॥~४~॥}
\end{quote}

\indent
स्पष्टोऽथः~॥~४~॥\\
\indent
अथाऽऽर्यया प्रश्नान्तरमाह\textendash\

\begin{quote}
{\ks लिप्ताषष्ट्यंशाग्रे जातात्संकलितघनचितिघनैक्यात्~।\\
यो भवति चितिघनोऽस्माद्यो दिननिवहं वदत्यसौ धीमान्~॥~५~॥}
\end{quote}

\indent
अथैतत्प्रश्नभङ्गार्थमार्यामाह\textendash
\begin{quote}
{\ks षड्गुणितादुद्दिष्टात्पूर्णं घनमूलमिह च यत्तस्मात्~।\\
वर्गभवं मूलं यत्तस्माद्द्विघ्नात्पदं विलिप्ताग्रम्~॥~६~॥}
\end{quote}

\indent
इह षड्गुणितादुद्दिष्टाद् यत्पूर्णं घनमूलं तस्माद्वर्गभवं मूलं विलिप्ताग्रं भवति~॥~६~॥\\
\indent
अथाऽऽर्यया प्रश्नान्तरं व्यनक्ति\textendash
\begin{quote}
{\ks पदकृतिसंकलितैक्याल्लिप्ताषष्ट्यंशशेषसंजातात्~।\\
इष्टदिनानां निवहः शक्यं सुधिया सुखेन विज्ञातुम~॥~७~॥}
\end{quote}

\indent
अत्र पदशब्दो गच्छवचनः~। लिप्ताषष्ट्यंशशेषजातात्पदकृतिसंकलितानां समासादिष्टदिननिवहः सुधिया सुखेन विज्ञातुं शक्यम्~। शक्यमित्यव्ययम्~। शक्यमरविन्दसुराभिरिति कालिदासः~॥~७~॥\\
\indent
अथैतत्प्रश्नभङ्गार्थमार्यामाह\textendash
\begin{quote}
{\ks कृतिपदसंकलितैक्ये संकलितं स्यात्कृशानुभिर्भक्ते~।\\
संकलिते नयनह ते विहितपदे शेषमात्रविकलाग्रम्~॥~८~॥}
\end{quote}

\newpage
\thispagestyle{fancy}
\fancyhf{}
\chead{\textbf{कुट्टाकारशिरोमणिः~।}}
\rhead{\textbf{५१}}
\indent
कृतिर्वर्गः~। पदं गच्छः~। संकलितं प्रसिद्धम्~। ऐक्यं योगः~। विकलाशेषभवानां कृतिपदसंकलितानामैक्ये कृशानुभि~(३)~र्भक्ते सति लब्धं संकलितं स्यात् । अथैवमागते संकलिते नयनहते विहितवर्गमूले च सत्यत्र शेषं विकलाग्रं भवति~॥~८~॥\\
\indent
अथाऽऽर्यया प्रश्नान्तरमाह\textendash
\begin{quote}
{\ks योऽम्बरचरेन्द्रलिप्तापूर्णरसांशावशेषजातेन~।\\
वर्गचितिघनेन दिनं विद्याद्विद्याविशेषवानेषः~॥~९~॥}
\end{quote}
\indent
लिप्तापूर्णरसांशा विलिप्ताः~। शिष्टं स्पष्टम्~।।~९~।।\\
\indent
अथैतत्प्रश्नभङ्गार्थमार्यामाह\textendash
\begin{quote}
{\ks त्रिगुणाद्वर्गचितिघनाद्यत्पूर्णत्वेन घनपदं भवति~।
तद्ग्रहविकलाशेषं भवतीति विपश्चिता विनिर्देश्यम्~॥~१०~॥}
\end{quote}
\indent
त्रिगुणाद्विकलाग्रभावाद्वर्गचितिघनाद्यत् पूर्णत्वेन घनमूलं भवति तद्ग्रहविकलाशेषं भवतीति विपश्चिता विनिर्देश्यम्~॥~१०~॥\\
\indent
अथाऽऽर्यया प्रश्नान्तरमाह\textendash
\begin{quote}
{\ks घनवर्गवर्गचितिघनसंकलितसमासतः समुद्दिष्टात्~।\\
खचरविलिप्ताग्रभवात्कथमपि गदितुं दिनं बुधैः शक्यम्~॥~११~॥}
\end{quote}

\indent
समासो योगः~। खचरविलिप्ताग्रवशात् समुद्दिष्टाद् घनवर्गवर्गचितिघनसंकलितसमासतो बुधैर्दिनं कथमपि कथयितुं शक्यम्~॥~११~॥\\
अथैतत्प्रश्नभङ्गार्थमार्यामाह\textendash
\begin{quote}
{\ks अत्राेद्दिष्टे राशौ सागरभक्ते फलं भवति यत्तत्~।\\
वर्गचितिघनस्त्वमुना विकलाग्रं साधयेद्बुधः प्राग्वत्~॥~१२~॥}
\end{quote}
\indent
अत्रोद्दिष्टे राशौ सागरैश्चतुर्भिर्भक्ते यत्फलं भवति तद्विकलाग्रजातो वर्गचितिघनः~। अमुना बुधः प्राग्वद्विकलाग्रं साधयेत्~॥~१२~॥\\
\indent
अथाऽऽर्यया प्रश्नान्तरमाह\textendash
\begin{quote}
{\ks कृतिघनसंकलितैक्यात्खचरविलिप्तावशेषसंभूतात्~।\\
वासरचयं वदेद्यो वागीशो भवति भूतले स पुमान्~॥~१३~॥}
\end{quote}
\indent
स्पष्टोऽर्थः~॥~१३~॥\\
\indent
अथैतत्प्रश्नभङ्गार्थमार्यामाह\textendash
\begin{quote}
{\ks कृतिघनसंकलितैक्ये वह्निहृते भवति वर्गसंकलितम्~।\\
तस्माद्गगनचराणां प्राग्वद्विकलावशेषमानेयम्~॥~१४~॥}\\
\end{quote}

\newpage
\thispagestyle{fancy}
\fancyhf{}
\chead{\textbf{महालक्ष्मीमुक्तावलीसहितः\textendash}}
\lhead{\textbf{५२}}
\indent
वर्गसंकलितं वर्गचितिघनम्~। शिष्टं स्पष्टम्~॥~१४~॥\\
\indent
अथाऽऽर्यया प्रश्नान्तरमाह\textendash
\begin{quote}
{\ks विकलाशेषेष्टांशश्रेढ्यां विकलावशेषभक्तायाम~।\\
यत्तेनांशमुखाभ्यामुत्तरतश्चेष्टदिनचयो वाच्यः~॥~१५~॥}
\end{quote}
\indent
\textsuperscript{*} 'इष्टं व्येकं दलितं सपूर्वमुत्तरगुणं समुखमध्यम्~।इष्टगुणितमिष्टधनम्' इत्यार्यभटीयसूत्रानीतमिष्टधनमत्र श्रेढीशब्देनोच्यते\textendash\  विकलाशेषस्य रूपांशद्व्यंशत्र्यंशादीष्टांशजाता श्रेढी विकलाशेषेष्टांशश्रेढी~। तस्यां
विकलावशेषभक्तायां शुद्धायां सत्यां यत्फलं भवति तेनांशादिमोत्तरैश्च
कथितैरिष्टदिननिचयो वाच्यः~॥~१५~॥\\
\indent
अथैतत्प्रश्नभञ्जनायाऽऽर्यामाह\textendash
\begin{quote}
{\ks अंशहते फलराशौ विमुखे चोत्तरहृते हते द्वाभ्याम्~।\\
रूपयुर्तेऽशकगुणिते यत्स्याद्विकलाग्रमेव तद्भवति~॥~१६~॥}
\end{quote}
\indent
उद्दिष्टे फलराशावुद्दिष्टेनांशेन गुणितेन तदुद्दिष्टेन मुखेन हीने तदुद्दिष्टेनोत्तरेण भक्ते ततो द्वाभ्यां~(२)~ गुणिते ततो रूप~(१)~सहिते पुनश्चांश~(शे)~गुणिते यत्स्यात्तद्विकलाग्रं भवति~। व्यक्त्यर्थमेतदुदाहरणेन प्रदर्श्यते~। अत्र ग्रहस्य कल्पितो विकलाशेषः शत~(१००)~संख्यः अस्य चतुर्थांशः पञ्चविंशति~(२५)~संख्यः~। इदमिष्टम्~। अत्र प्रथमं श्रेढीफलमानीयते~। इष्टम्~(२५)~। एतद्व्येकम्~। चतुर्विंशति~(२४)~संख्यम्~। एतद्दलितं द्वादश~(१२)~ संख्यम्~। इदमत्र कल्पितेनोत्तरेण त्रिकेण गुणितं षट्त्रिंश~(३६)~त्संख्यम्~। एतत्कल्पितेन मुखेन चतुष्टयेन~(४)~युक्तं चत्वारिंश~(४०)~त्संख्यम्~। इदं मध्यधनम्~। एतदुद्दिष्टेन पञ्चविंशति~(२५)~संख्येनांशेन गुणितं सहस्र~(१०००)~संख्यं जातम्~। इदमत्र विकलाशेषेष्टांशश्रेढी~। इयं विकलाशेषेण शत~(१००)~संख्येन भक्ता शुध्यति~। अत्र फलं दश~(१०)~संख्यम्~। अस्मिंश्चांशे मुखे चोत्तरे च कथिते सति विकलाग्रमानेतुं शक्यम्~। तस्मात्कथितैरेभिर्श्चतुर्भिर्भिर्विकलाग्रमनीयते~। इह फलराशिर्दश~(१०)~संख्यः~। अयमंशेन चतुष्टयेन~(४)~गुणितश्चतवारिंश~(४०)~त्संख्यः~। अयं मुखेन चतुष्टयेन (४)~हीनः षट्त्रिंश~(३६)~त्संख्यः~। अयमुत्तरेण त्रिकेण~(३)~ भक्तो द्वादश (१२)~संख्यः~। एष द्वाभ्यां गुणितश्चतुर्विंशति~(२४)~संख्यः~। अयं रूपेण~(१)~ सहितः पञ्चविंशति~(२५)~संख्यः~। अयं पुनरप्यंशेन चतुष्टयेन~(४)~गुणितः शत~(१००)~संख्यः~। एष ग्रहस्य विकलाशेषः~।।~१६~।।

\footnotetext{
\textsuperscript{*} आर्यभट्टीये गणितपादे श्लोक १९}

\newpage
\thispagestyle{fancy}
\fancyhf{}
\chead{\textbf{कुट्टाकारशिरोमणिः~।}}
\rhead{\textbf{५३}}
\indent
\textbf{अथाऽऽर्ययोपसंहरति\textendash}
\begin{quote}
{\ks दिङ्मात्रेण मयाऽस्मिन्कुट्टाकारः प्रदर्शितो ग्रन्थे~।\\
शेषोऽखिलो विशेषो भाष्यादौ दृश्यतां विपश्चिद्भिः~॥~१७~॥}
\end{quote}

\indent
भाष्यमार्यभटीयविषयं भास्कराचार्यविरचितम्~। आदिशब्देन भट्टब्रह्मगुप्तविरचितब्रह्मसिद्धान्तादिरभिधीयते~। शिष्टं सुबोधम्~॥~१७~॥\\
\indent
अथैष ग्रन्थो न स्वतन्त्रः, किंत्वार्यभटाचार्यप्रणीतकुट्टाकारविषयसूत्रयुग्मस्य कुट्टाकारशिरोमणिनामधेयो व्याख्यानविशेष एवेत्यार्ययाऽऽह\textendash
\begin{quote}
{\ks आचार्यार्यभटोदितकुट्टाकारार्थसूत्रयुगलस्य~।\\
कुट्टाकारशिरोमणिनामा व्याख्याविशेष एवैषः~॥~१८~॥}
\end{quote}
\indent
इयमवतारिकया विवृतार्था~॥~१८~॥\\

\textbf{अथेमं ग्रन्थमभ्यस्यतां फलं दर्शयितुमार्यामाह\textendash}
\begin{quote}
{\ks कुट्टटाकारशिरोमणिमिममभ्यस्यति दृढेन यो मनसा~।\\
तरणेः प्रसादतोऽयं सहसा तान्त्रिकशिरोमणिर्भवति~॥~१९~॥}
\end{quote}
\centering
\textbf{इति श्रीदेवराजविरचिते कुट्टाकारशिरोमणौ मिश्रश्रेढीमिश्रकुट्टाकारपरिच्छेदस्तृतीयः~॥~३~॥\\}
\rule{0.2\linewidth}{1.0pt}\\

\vspace{1cm}
\justifying
\indent
इमं कुट्टकारशिरोमाणिं दृढेनाव्यापृतेन मनसा योऽभ्यस्यति संशीलयति स
तरणेरादित्यस्य प्रसादतः सहसा तान्त्रिकशिरोमणिर्भवति~। तान्त्रिको ज्ञातसिद्धान्तः~॥~१९~॥\\

\centering
\indent
 \textbf{सद्वर्णसूत्रकलिता समवृत्ताऽतुला नवा~।\\
  समर्पिता महालक्ष्म्यै टीका मुक्तावली मया~॥}\\

\indent
\textbf{इत्यत्रिकुलाभरणस्य स्कन्धत्रयवेदिनः सिद्धान्तवल्लभ इति प्रसिद्धापरनाम्नः
श्रीवरदराजाचार्यस्य तनयेन देवराजेन विरचितायां कुट्टाकारशिरोमणि
टीकायां महालक्ष्मीमुक्तावल्यां मिश्रश्रेढीमिश्रकुट्टाकारपरिच्छेदस्तृतीयः~॥~३~॥\\
}
\indent
\textbf{समाप्ता चेयं महालक्ष्मीमुक्तावली नाम कुट्टाकारशिरोमणिटीका~।\\
}
\centering
\rule{0.2\linewidth}{1.0pt}


 \end{document}