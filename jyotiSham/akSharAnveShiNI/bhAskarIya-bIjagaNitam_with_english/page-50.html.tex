\documentclass[]{article}
\usepackage{lmodern}
\usepackage{amssymb,amsmath}
\usepackage{ifxetex,ifluatex}
\usepackage{fixltx2e} % provides \textsubscript
\ifnum 0\ifxetex 1\fi\ifluatex 1\fi=0 % if pdftex
  \usepackage[T1]{fontenc}
  \usepackage[utf8]{inputenc}
\else % if luatex or xelatex
  \ifxetex
    \usepackage{mathspec}
  \else
    \usepackage{fontspec}
  \fi
  \defaultfontfeatures{Ligatures=TeX,Scale=MatchLowercase}
\fi
% use upquote if available, for straight quotes in verbatim environments
\IfFileExists{upquote.sty}{\usepackage{upquote}}{}
% use microtype if available
\IfFileExists{microtype.sty}{%
\usepackage[]{microtype}
\UseMicrotypeSet[protrusion]{basicmath} % disable protrusion for tt fonts
}{}
\PassOptionsToPackage{hyphens}{url} % url is loaded by hyperref
\usepackage[unicode=true]{hyperref}
\hypersetup{
            pdfborder={0 0 0},
            breaklinks=true}
\urlstyle{same}  % don't use monospace font for urls
\IfFileExists{parskip.sty}{%
\usepackage{parskip}
}{% else
\setlength{\parindent}{0pt}
\setlength{\parskip}{6pt plus 2pt minus 1pt}
}
\setlength{\emergencystretch}{3em}  % prevent overfull lines
\providecommand{\tightlist}{%
  \setlength{\itemsep}{0pt}\setlength{\parskip}{0pt}}
\setcounter{secnumdepth}{0}
% Redefines (sub)paragraphs to behave more like sections
\ifx\paragraph\undefined\else
\let\oldparagraph\paragraph
\renewcommand{\paragraph}[1]{\oldparagraph{#1}\mbox{}}
\fi
\ifx\subparagraph\undefined\else
\let\oldsubparagraph\subparagraph
\renewcommand{\subparagraph}[1]{\oldsubparagraph{#1}\mbox{}}
\fi

% set default figure placement to htbp
\makeatletter
\def\fps@figure{htbp}
\makeatother


\date{}

\begin{document}

{\ldots{}.48\ldots{}.}

{from the original equation we may get the value of x.}

{यस्त्रिपञ्चगुणो राशिः पृथक्सैकः कृतिर्भवेत् । }

{वदं तं बीजमध्येऽसि मध्यमाहरणे पटुः ।। १७० ।। }

{Find the number which multiplied by 3 and 5 separately and increased by
1 is a square.}

{को राशिस्त्रिभिरभ्यस्तः सरूपो जायते घनः । }

{घनमूलं कृतीभूतं त्र्यभ्यस्तं कृतिरेकयुक् ।। १७१ ।। }

{What is that number which multiplied by 3 and increased by 1 is a cube
and thrice the square of the cube root increased by 1 is square?}

{वर्गान्तरं कयो राश्योः पृथग्द्वित्रिगुणं त्रियुक् । }

{वर्गौ स्यातां वद क्षिप्रं षट्कपञ्चकयोरिव ।। १७२ ।। }

{Find two numbers such that when the difference of their squares is
separately multiplied by 2 and 3 and increased by 3, we get square
numbers. The numbers required are like 6 and 5 but different from them.}

{क्वचिदादेः क्वचिन्मध्यात्क्ववचिदन्त्यात्क्रिया बुधैः । }

{आरभ्यते यथा लघ्वी निर्वहेच्च यथा तथा ।। १७३ ।। }

{Some times we should start from beginning, sometimes from the middle
and sometimes from the end. The process leading to the solution should
be short.}

{वर्गादेर्यो हरस्तेन गुणितं यदि जायते । }

{अव्यक्तं तत्र तन्मानमभिन्नं स्याद्यथा तथा }

{कल्प्योऽन्यवर्णवर्गादिस्तुल्यं शेषं यथोक्तवत् ।। १७४ ।। }

{When we multiply by the coefficient of the square we should assume
square of some unknown and the remaining process should be as given
before so that the value of x will be integral.\\
}

\end{document}
