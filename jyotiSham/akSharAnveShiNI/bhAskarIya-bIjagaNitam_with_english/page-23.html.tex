\documentclass[]{article}
\usepackage{lmodern}
\usepackage{amssymb,amsmath}
\usepackage{ifxetex,ifluatex}
\usepackage{fixltx2e} % provides \textsubscript
\ifnum 0\ifxetex 1\fi\ifluatex 1\fi=0 % if pdftex
  \usepackage[T1]{fontenc}
  \usepackage[utf8]{inputenc}
\else % if luatex or xelatex
  \ifxetex
    \usepackage{mathspec}
  \else
    \usepackage{fontspec}
  \fi
  \defaultfontfeatures{Ligatures=TeX,Scale=MatchLowercase}
\fi
% use upquote if available, for straight quotes in verbatim environments
\IfFileExists{upquote.sty}{\usepackage{upquote}}{}
% use microtype if available
\IfFileExists{microtype.sty}{%
\usepackage[]{microtype}
\UseMicrotypeSet[protrusion]{basicmath} % disable protrusion for tt fonts
}{}
\PassOptionsToPackage{hyphens}{url} % url is loaded by hyperref
\usepackage[unicode=true]{hyperref}
\hypersetup{
            pdfborder={0 0 0},
            breaklinks=true}
\urlstyle{same}  % don't use monospace font for urls
\IfFileExists{parskip.sty}{%
\usepackage{parskip}
}{% else
\setlength{\parindent}{0pt}
\setlength{\parskip}{6pt plus 2pt minus 1pt}
}
\setlength{\emergencystretch}{3em}  % prevent overfull lines
\providecommand{\tightlist}{%
  \setlength{\itemsep}{0pt}\setlength{\parskip}{0pt}}
\setcounter{secnumdepth}{0}
% Redefines (sub)paragraphs to behave more like sections
\ifx\paragraph\undefined\else
\let\oldparagraph\paragraph
\renewcommand{\paragraph}[1]{\oldparagraph{#1}\mbox{}}
\fi
\ifx\subparagraph\undefined\else
\let\oldsubparagraph\subparagraph
\renewcommand{\subparagraph}[1]{\oldsubparagraph{#1}\mbox{}}
\fi

% set default figure placement to htbp
\makeatletter
\def\fps@figure{htbp}
\makeatother


\date{}

\begin{document}

{\ldots{}.21\ldots{}.}

{येन संगुणिताः पञ्च त्रयोविंशतिसंयुताः । }

{वर्जिता वा त्रिभिर्भक्ता निरग्राः स्युः स को गुणः ।। ६४ ।। }

{What is that number which multiplied by 5 and increased or decreased by
23, is divisible by 3 without a remainder?}

{येन पञ्च गुणिताः खसंयुताः पञ्चषष्टिसहिताश्च तेऽथवा । }

{स्युस्त्रयोदश हृता निरग्रकास्तं गुणं गणक कीर्तयाऽऽशु मे ।। ६५ ।। }

{What is that number which multiplied by 5 and increased by zero is
divisible by 13 without a remainder and which multiplied by 5 and
increased by 65 is divisible by 13 without a remainder?}

{क्षेपं विशुद्धिं परिकल्प्य रूपं पृथक्तयोर्ये गुणकारलब्धी । }

{अभीप्सितक्षेपविशुद्धिनिघ्ने स्वहारतष्टे भवतस्तयोस्ते ।। ६६ ।। }

{In the given कुट्टक taking c = ± find x and y. By the desired value of
c (क्षेप) multiply the x and y obtained. Divide the results by a and b
respectively. The remainders are the true values of x and y.}

{कल्प्याऽथ शुद्धिर्विकलावशेषं षष्टिश्च भाज्यः कुदिनानि हारः । }

{तज्जं फलं स्युर्विकलागुणस्तु लिप्ताग्रमस्माच्च कलालवाग्रम् । }

{एवं तदूर्ध्वं च तथाऽधिमासावमाग्रकाभ्यो दिवसा रवीन्द्वोः ।। ६७ ।। }

{We take 60 as dividend, c, a negative number indicating residual
vikala, and divisor indicating the days in a युग. The x that we get is
residual kala and y is vikala for the planet. From this we find kala,
residual degree and so on.}

{In the same manner from अधिमास and अवमाग्रक we get total lunar and
solar days that have elapsed.}

{एको हरश्चेद् गुणकौ विभिन्नौ तदा गुणैक्यं परिकल्प्य भाज्यम् । }

{अग्रैक्यमग्रं कृत उक्तवद्यः संश्लिष्टसंज्ञः स्फुटकुट्टकोऽसौ ।। ६८ ।। }

{In two pulverisers if the divisor is common, we should add the two
values of a and take the\\
}

\end{document}
