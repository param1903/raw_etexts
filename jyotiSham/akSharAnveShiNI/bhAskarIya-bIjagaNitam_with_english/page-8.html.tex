\documentclass[]{article}
\usepackage{lmodern}
\usepackage{amssymb,amsmath}
\usepackage{ifxetex,ifluatex}
\usepackage{fixltx2e} % provides \textsubscript
\ifnum 0\ifxetex 1\fi\ifluatex 1\fi=0 % if pdftex
  \usepackage[T1]{fontenc}
  \usepackage[utf8]{inputenc}
\else % if luatex or xelatex
  \ifxetex
    \usepackage{mathspec}
  \else
    \usepackage{fontspec}
  \fi
  \defaultfontfeatures{Ligatures=TeX,Scale=MatchLowercase}
\fi
% use upquote if available, for straight quotes in verbatim environments
\IfFileExists{upquote.sty}{\usepackage{upquote}}{}
% use microtype if available
\IfFileExists{microtype.sty}{%
\usepackage[]{microtype}
\UseMicrotypeSet[protrusion]{basicmath} % disable protrusion for tt fonts
}{}
\PassOptionsToPackage{hyphens}{url} % url is loaded by hyperref
\usepackage[unicode=true]{hyperref}
\hypersetup{
            pdfborder={0 0 0},
            breaklinks=true}
\urlstyle{same}  % don't use monospace font for urls
\IfFileExists{parskip.sty}{%
\usepackage{parskip}
}{% else
\setlength{\parindent}{0pt}
\setlength{\parskip}{6pt plus 2pt minus 1pt}
}
\setlength{\emergencystretch}{3em}  % prevent overfull lines
\providecommand{\tightlist}{%
  \setlength{\itemsep}{0pt}\setlength{\parskip}{0pt}}
\setcounter{secnumdepth}{0}
% Redefines (sub)paragraphs to behave more like sections
\ifx\paragraph\undefined\else
\let\oldparagraph\paragraph
\renewcommand{\paragraph}[1]{\oldparagraph{#1}\mbox{}}
\fi
\ifx\subparagraph\undefined\else
\let\oldsubparagraph\subparagraph
\renewcommand{\subparagraph}[1]{\oldsubparagraph{#1}\mbox{}}
\fi

% set default figure placement to htbp
\makeatletter
\def\fps@figure{htbp}
\makeatother


\date{}

\begin{document}

{CONTENTS}

{Page}

{ INVOCATION AND INTRODUCTION }{7}

{1 धनर्णषड्वधम् । }

{ The SIX RULES for positive and negative numbers 7 }

{2 शून्यषड्विधम् । }

{ SIX RULES for ZERO 9 }

{3 वर्णषड्विधम् । }

{ SIX RULES for ALGEBRAIC NUMBERS 10}

{4 करणीषड्विधम् । }

{ SIX LAWS for SURDS 13}

{5 कुट्टकविवरणम् । }

{ EQUATION of the form ax + c = by 17}

{6 वर्गप्रकृतिः । }

{ EQUATION of the form ax2 + b = y2 22}

{7 एकवर्णसमीकरणम् । }

{ EQUATION with one unknown 26 }

{8 मध्यमाहरणम् । }

{ A device to solve a QUADRATIC EQUATION 33}

{9 अनेकवर्णसमीकरणम् । }

{ EQUATIONS involving more than one unknown 38}

{10 अनेक वर्णसमीकरणान्तर्गतं मध्यमाहरणम् । }

{ Device for solving EQUATIONS with more than one unknown 42}

{11 भा}{क्ति}{म् । }

{ EQUATIONS involving product of unknowns 50}

{ ग्रंथसमाप्तिः । }

{ EPILOGUE 52}

{ APPENDIX some words denoting numbers 54}

{ GLOSSARY of TECHNICAL TERMS 55}

{\\
}

\end{document}
