\documentclass[]{article}
\usepackage{lmodern}
\usepackage{amssymb,amsmath}
\usepackage{ifxetex,ifluatex}
\usepackage{fixltx2e} % provides \textsubscript
\ifnum 0\ifxetex 1\fi\ifluatex 1\fi=0 % if pdftex
  \usepackage[T1]{fontenc}
  \usepackage[utf8]{inputenc}
\else % if luatex or xelatex
  \ifxetex
    \usepackage{mathspec}
  \else
    \usepackage{fontspec}
  \fi
  \defaultfontfeatures{Ligatures=TeX,Scale=MatchLowercase}
\fi
% use upquote if available, for straight quotes in verbatim environments
\IfFileExists{upquote.sty}{\usepackage{upquote}}{}
% use microtype if available
\IfFileExists{microtype.sty}{%
\usepackage[]{microtype}
\UseMicrotypeSet[protrusion]{basicmath} % disable protrusion for tt fonts
}{}
\PassOptionsToPackage{hyphens}{url} % url is loaded by hyperref
\usepackage[unicode=true]{hyperref}
\hypersetup{
            pdfborder={0 0 0},
            breaklinks=true}
\urlstyle{same}  % don't use monospace font for urls
\IfFileExists{parskip.sty}{%
\usepackage{parskip}
}{% else
\setlength{\parindent}{0pt}
\setlength{\parskip}{6pt plus 2pt minus 1pt}
}
\setlength{\emergencystretch}{3em}  % prevent overfull lines
\providecommand{\tightlist}{%
  \setlength{\itemsep}{0pt}\setlength{\parskip}{0pt}}
\setcounter{secnumdepth}{0}
% Redefines (sub)paragraphs to behave more like sections
\ifx\paragraph\undefined\else
\let\oldparagraph\paragraph
\renewcommand{\paragraph}[1]{\oldparagraph{#1}\mbox{}}
\fi
\ifx\subparagraph\undefined\else
\let\oldsubparagraph\subparagraph
\renewcommand{\subparagraph}[1]{\oldsubparagraph{#1}\mbox{}}
\fi

% set default figure placement to htbp
\makeatletter
\def\fps@figure{htbp}
\makeatother


\date{}

\begin{document}

{\ldots{}.11\ldots{}. }

{Addition and subtraction are carried out with like terms and unlike
terms are kept separately,}

{स्वमव्यक्तमेकं सखे सैकरूपं धनाव्यक्तयुग्मं विरूपाष्टकं च । }

{युतौ पक्षयोरेतयोः किं धनर्णे विपर्यस्य चैक्ये भवे}{त्}{ किं }

{वदाऽऽशु ।। २३ ।। }

{What is the result when 1x + 1 and 2x - 8 are added together ? If in
these the positive signs are changed to negatives what will be the sum
then?}

{धनाव्यक्तवर्गत्रयं सत्रिरूपं क्षयायुक्तयुग्मेन युक्तं च किं स्यात्
।।२४।।}

{What will be the sum of 3x3 + 3 and -2x ? }

{धनाव्यक्तयुग्माद् ऋणाव्यक्तषट्कं स्वरूपाष्टकं प्रोज्झ्य शेषं }

{वदाऽऽशु ।। २५ ।। }

{If from 2x we subtract -6x + 8 what will be left?}

{स्याद् रूपर्णाभिहतौ तु वर्णौ व्दित्र्यादिकानां समजातिकानाम् ।। }

{वधे तु तद्वर्गघनादयः स्युस् तद् भावितं चासमजातिघाते । }

{भागादिकं रूपवदेव शेषं व्यक्ते यदुक्तं गणिते तदत्र ।। २६ ।। }

{The product of an arithmetical number with an unknown is an unknown
number. The product of two like numbers is its square, the product of
three like numbers is its cube and so on. The product iof unlike
variables is called भावित.}

{The rule for division in algebra is same as given in arthmetic.}

{गुण्यः पृथग् गुणकखण्डसमो निवेश्यस् }

{तैः खण्डकः क्रमहतः सहितो यथोक्त्या । }

{अव्यक्तवर्गकरणीगुणनासु चिन्त्यो }

{व्यक्तोक्तखण्डगुणनाविधिरेवमत्र ।। २७ ।। }

{Having separated the terms of the multiplier the multiplicand is to be
placed with each term. Each term of the multiplier multiplies the
multi-}

\end{document}
