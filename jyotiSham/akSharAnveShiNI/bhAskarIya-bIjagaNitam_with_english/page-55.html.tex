\documentclass[]{article}
\usepackage{lmodern}
\usepackage{amssymb,amsmath}
\usepackage{ifxetex,ifluatex}
\usepackage{fixltx2e} % provides \textsubscript
\ifnum 0\ifxetex 1\fi\ifluatex 1\fi=0 % if pdftex
  \usepackage[T1]{fontenc}
  \usepackage[utf8]{inputenc}
\else % if luatex or xelatex
  \ifxetex
    \usepackage{mathspec}
  \else
    \usepackage{fontspec}
  \fi
  \defaultfontfeatures{Ligatures=TeX,Scale=MatchLowercase}
\fi
% use upquote if available, for straight quotes in verbatim environments
\IfFileExists{upquote.sty}{\usepackage{upquote}}{}
% use microtype if available
\IfFileExists{microtype.sty}{%
\usepackage[]{microtype}
\UseMicrotypeSet[protrusion]{basicmath} % disable protrusion for tt fonts
}{}
\PassOptionsToPackage{hyphens}{url} % url is loaded by hyperref
\usepackage[unicode=true]{hyperref}
\hypersetup{
            pdfborder={0 0 0},
            breaklinks=true}
\urlstyle{same}  % don't use monospace font for urls
\IfFileExists{parskip.sty}{%
\usepackage{parskip}
}{% else
\setlength{\parindent}{0pt}
\setlength{\parskip}{6pt plus 2pt minus 1pt}
}
\setlength{\emergencystretch}{3em}  % prevent overfull lines
\providecommand{\tightlist}{%
  \setlength{\itemsep}{0pt}\setlength{\parskip}{0pt}}
\setcounter{secnumdepth}{0}
% Redefines (sub)paragraphs to behave more like sections
\ifx\paragraph\undefined\else
\let\oldparagraph\paragraph
\renewcommand{\paragraph}[1]{\oldparagraph{#1}\mbox{}}
\fi
\ifx\subparagraph\undefined\else
\let\oldsubparagraph\subparagraph
\renewcommand{\subparagraph}[1]{\oldsubparagraph{#1}\mbox{}}
\fi

% set default figure placement to htbp
\makeatletter
\def\fps@figure{htbp}
\makeatother


\date{}

\begin{document}

{\ldots{}.53\ldots{}.}

{दुस्तरः स्तोकबुद्धीनां शास्त्रविस्तारवारिधिः । }

{अथवा शास्त्रविस्तृत्या किं कार्यं सुधियामपि ।। ६ ।। }

{The expanse in science is like the vast ocean and difficult to cross
for ordinary intellect. On the other hand what is the necessity of
details for an intelligent person?}

{उपदेशलवं शास्त्रं कुरुते धीमतो यतः । }

{तत्तु प्राप्यैव विस्तारं स्वयमेवोपगच्छति ।। ७ ।। }

{Whatever particle an intelligent man receives from his teacher, that
well received knowledge spreads itself extensively.}

{जले तैलं खले गुह्यं पात्रे दानंमनागपि । }

{प्राज्ञे शास्त्रं स्वयं याति विस्तारं वस्तुशक्तितः ।। ८ ।। }

{A drop of oil put in water, a secret deposited in the ears of a villain
or a gift bestowed on a deserving person spreads. In like manner
knowledge spreads in an intelligent mind by the force of its merits.}

{गणक भणिति रम्यं बाललीलावगम्यं । }

{सकलगणितसारं सोपपत्तिप्रकारं । । }

{इति बहुगुणयुक्तं सर्वदोषैर्विमुक्तं । }

{पठ पठ मतिवृद्ध्यै लध्विदं प्रौढसिद्ध्यै ।। ९ ।। }

{Oh pupil of mathematics, this is pleasing, easy for beginners. It is
the essence of all mathematics and deals with basic laws. It has many
merits and is free from faults. I say, read this small book to sharpen
your intellect and you are sure to rise.}

{Thus ends Bhaskaras bijaganit and its version in English.}

\end{document}
