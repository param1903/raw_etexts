\documentclass[]{article}
\usepackage{lmodern}
\usepackage{amssymb,amsmath}
\usepackage{ifxetex,ifluatex}
\usepackage{fixltx2e} % provides \textsubscript
\ifnum 0\ifxetex 1\fi\ifluatex 1\fi=0 % if pdftex
  \usepackage[T1]{fontenc}
  \usepackage[utf8]{inputenc}
\else % if luatex or xelatex
  \ifxetex
    \usepackage{mathspec}
  \else
    \usepackage{fontspec}
  \fi
  \defaultfontfeatures{Ligatures=TeX,Scale=MatchLowercase}
\fi
% use upquote if available, for straight quotes in verbatim environments
\IfFileExists{upquote.sty}{\usepackage{upquote}}{}
% use microtype if available
\IfFileExists{microtype.sty}{%
\usepackage[]{microtype}
\UseMicrotypeSet[protrusion]{basicmath} % disable protrusion for tt fonts
}{}
\PassOptionsToPackage{hyphens}{url} % url is loaded by hyperref
\usepackage[unicode=true]{hyperref}
\hypersetup{
            pdfborder={0 0 0},
            breaklinks=true}
\urlstyle{same}  % don't use monospace font for urls
\IfFileExists{parskip.sty}{%
\usepackage{parskip}
}{% else
\setlength{\parindent}{0pt}
\setlength{\parskip}{6pt plus 2pt minus 1pt}
}
\setlength{\emergencystretch}{3em}  % prevent overfull lines
\providecommand{\tightlist}{%
  \setlength{\itemsep}{0pt}\setlength{\parskip}{0pt}}
\setcounter{secnumdepth}{0}
% Redefines (sub)paragraphs to behave more like sections
\ifx\paragraph\undefined\else
\let\oldparagraph\paragraph
\renewcommand{\paragraph}[1]{\oldparagraph{#1}\mbox{}}
\fi
\ifx\subparagraph\undefined\else
\let\oldsubparagraph\subparagraph
\renewcommand{\subparagraph}[1]{\oldsubparagraph{#1}\mbox{}}
\fi

% set default figure placement to htbp
\makeatletter
\def\fps@figure{htbp}
\makeatother


\date{}

\begin{document}

{\ldots{}.27\ldots{}.}

{value of the unknown becomes known.}

{If the number of unknowns is two or more assume one unknown to be x.
Multiplying x by 2 or any other number or dividing it by something or
adding to that or subtracting from that some number one should assume
suitable values for the other unknowns.}

{एकस्य रुपत्रिशती षड्श्वाअश्वा दशान्यस्य तु तुल्यमौल्याः । }

{ऋणं तथा रूपशतं च यस्य तौ तुल्यवित्तौ च किमश्वमौल्यम् ।। ९० ।।}

{ One man has 300 rupees and 6 horses and another man has 10 horses and
a debt of rupees 100. If they are equally rich and the price of each
horse be the same, tell me the price of one horse.}

{यदाद्यवित्तस्य दलं दियुक्तं तत्तुल्यवित्तो यदि वा द्वितीयः । }

{आद्यो धनेन त्रिगुणोऽन्यतो वा पृथक् पृथङ् मे वद वाजिमौल्यम् ।। ९१ ।। }

{(1) If two rupees are added to half the wealth of the first man, he is
equal in wealth with the second man. (2) Three times the wealth of the
second man is equal to the wealth of the first man. Tell me the price of
one horse in each case separately.}

{माणिक्यामलनीलमौक्तिकमितिः पञ्चाष्ट सप्त क्रमा- }

{देकस्यान्यतरस्य सप्त नव षट् तद्रत्नसंख्या सखे । }

{रूपाणां नवतिर्द्विषष्टिरनयोस्तौ तुल्यवित्तौ त}{त्तया}{ }

{बीजज्ञ प्रतिरत्नजानि सुमते मौल्यानि शीघ्रं वद ।। ९२ ।। }

{One man has 5 rubies, 8 saphires, 7 pcarls and 90 rupees. Second man
has 7 rubies, 9 saphires, 6 pearls and 62 rupees. They are equally rich.
Find the price of each jewel.}

{एको ब्रवीति मम देहि शतं धनेन त्वत्तो भवामि हि सखे द्विगुणस्तततोन्यः । }

{ब्रूते दशार्पयसि चेन्मम षड्गुणोऽहं त्वत्तस्तयोर्वद धने मम किं प्रमाणे
।।९३ ।। }

{One man says to the other, ``If you will give me 100 rupees, I shall be
twice yourself in wealth.''\\
}

\end{document}
