\documentclass[]{article}
\usepackage{lmodern}
\usepackage{amssymb,amsmath}
\usepackage{ifxetex,ifluatex}
\usepackage{fixltx2e} % provides \textsubscript
\ifnum 0\ifxetex 1\fi\ifluatex 1\fi=0 % if pdftex
  \usepackage[T1]{fontenc}
  \usepackage[utf8]{inputenc}
\else % if luatex or xelatex
  \ifxetex
    \usepackage{mathspec}
  \else
    \usepackage{fontspec}
  \fi
  \defaultfontfeatures{Ligatures=TeX,Scale=MatchLowercase}
\fi
% use upquote if available, for straight quotes in verbatim environments
\IfFileExists{upquote.sty}{\usepackage{upquote}}{}
% use microtype if available
\IfFileExists{microtype.sty}{%
\usepackage[]{microtype}
\UseMicrotypeSet[protrusion]{basicmath} % disable protrusion for tt fonts
}{}
\PassOptionsToPackage{hyphens}{url} % url is loaded by hyperref
\usepackage[unicode=true]{hyperref}
\hypersetup{
            pdfborder={0 0 0},
            breaklinks=true}
\urlstyle{same}  % don't use monospace font for urls
\IfFileExists{parskip.sty}{%
\usepackage{parskip}
}{% else
\setlength{\parindent}{0pt}
\setlength{\parskip}{6pt plus 2pt minus 1pt}
}
\setlength{\emergencystretch}{3em}  % prevent overfull lines
\providecommand{\tightlist}{%
  \setlength{\itemsep}{0pt}\setlength{\parskip}{0pt}}
\setcounter{secnumdepth}{0}
% Redefines (sub)paragraphs to behave more like sections
\ifx\paragraph\undefined\else
\let\oldparagraph\paragraph
\renewcommand{\paragraph}[1]{\oldparagraph{#1}\mbox{}}
\fi
\ifx\subparagraph\undefined\else
\let\oldsubparagraph\subparagraph
\renewcommand{\subparagraph}[1]{\oldsubparagraph{#1}\mbox{}}
\fi

% set default figure placement to htbp
\makeatletter
\def\fps@figure{htbp}
\makeatother


\date{}

\begin{document}

{\ldots{}.35\ldots{}.}

{कः खेन विहृतो राशिः कोट्या युक्तोऽथवोनितः । }

{वर्गितः स्वपदेनाऽऽढ्यः खगुणो नवतिर्भवेत् ।। १२० ।। }

{What is that number which when divided by zero and then increased or
decreased by ten millions, then squared and increased by its square root
and multiplied by zero becomes 90? {[}Commentators say this gives the
equation x}{2}{ + x =90.{]}}

{कः स्वार्धसहितो राशिः खगुणो वर्गितो युतः । }

{स्वपदाभ्यां खभक्तश्च जातः पञ्चदशोच्यताम् ।। १२१ ।। }

{What is that number to which its half is added, then multiplied by
zero, squared and then united with double the square root and divided by
zero gives 15?}

{{[} They say this gives 9x}{2}{ + 12x = 60}

{राशिर्द्धादशनिघ्नो राशिघनाढ्यश्च कः समो यस्य । }

{राशिकृतिः षड्गुणिता पञ्चत्रिंशद्युता विद्वन् ।। १२२ ।। }

{Solve the equation 12x + x}{3}{ = 6x}{2}{ + 35}

{को राशिर्द्विशतीक्षण्णो राशिवर्गयुतो हतः । }

{द्वाभ्यां तेनोनितो राशिवर्गवर्गोऽयुतं भवेत् । }

{रूपोनं वद तं राशिं वेत्सि बीजक्रियां यदि ।। १२३ ।। }

{Solve x}{4}{ - 2 (200x + x}{2}{) = 10000 -1}

{वनान्तराले ष्लवगाष्टभागः संवर्गितो वल्गति जातरागः । }

{ब्रूत्कारनादप्रतिनादहृष्टा दृष्टा गिरौ द्वादश ते कियन्तः ।। १२४ ।। }

{Out of a herd of monkeys in a forest, a number equal to the sqaure root
of 1/8th of them were sportively galloping. Cheered by the reverberation
of their sounds remaining 12 went to a hillock. How many were they? }

{यूथात् पञ्चांशकस्त्र्यूनो वर्गितो गह्वरं गतः । }

{दृष्टः शाखामृगः शाखामारूढो वद ते कति ।। }

{कर्णस्य त्रिलवेनोना द्वादशाङ्गुलशंकुभा । }

{चतुर्दशाङ्गुला जाता गणक ब्रूहि तां द्रुतम् ।। १२५ ।। }

{There was a troupe of monkeys, out of that\\
}

\end{document}
