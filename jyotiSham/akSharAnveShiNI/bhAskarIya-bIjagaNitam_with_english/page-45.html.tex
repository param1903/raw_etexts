\documentclass[]{article}
\usepackage{lmodern}
\usepackage{amssymb,amsmath}
\usepackage{ifxetex,ifluatex}
\usepackage{fixltx2e} % provides \textsubscript
\ifnum 0\ifxetex 1\fi\ifluatex 1\fi=0 % if pdftex
  \usepackage[T1]{fontenc}
  \usepackage[utf8]{inputenc}
\else % if luatex or xelatex
  \ifxetex
    \usepackage{mathspec}
  \else
    \usepackage{fontspec}
  \fi
  \defaultfontfeatures{Ligatures=TeX,Scale=MatchLowercase}
\fi
% use upquote if available, for straight quotes in verbatim environments
\IfFileExists{upquote.sty}{\usepackage{upquote}}{}
% use microtype if available
\IfFileExists{microtype.sty}{%
\usepackage[]{microtype}
\UseMicrotypeSet[protrusion]{basicmath} % disable protrusion for tt fonts
}{}
\PassOptionsToPackage{hyphens}{url} % url is loaded by hyperref
\usepackage[unicode=true]{hyperref}
\hypersetup{
            pdfborder={0 0 0},
            breaklinks=true}
\urlstyle{same}  % don't use monospace font for urls
\IfFileExists{parskip.sty}{%
\usepackage{parskip}
}{% else
\setlength{\parindent}{0pt}
\setlength{\parskip}{6pt plus 2pt minus 1pt}
}
\setlength{\emergencystretch}{3em}  % prevent overfull lines
\providecommand{\tightlist}{%
  \setlength{\itemsep}{0pt}\setlength{\parskip}{0pt}}
\setcounter{secnumdepth}{0}
% Redefines (sub)paragraphs to behave more like sections
\ifx\paragraph\undefined\else
\let\oldparagraph\paragraph
\renewcommand{\paragraph}[1]{\oldparagraph{#1}\mbox{}}
\fi
\ifx\subparagraph\undefined\else
\let\oldsubparagraph\subparagraph
\renewcommand{\subparagraph}[1]{\oldsubparagraph{#1}\mbox{}}
\fi

% set default figure placement to htbp
\makeatletter
\def\fps@figure{htbp}
\makeatother


\date{}

\begin{document}

{\ldots{}.43\ldots{}.}

{order that common man should acquire knowledge. That thought got the
name bijaganita.}

{एकस्य पक्षस्य पदे गृहीते द्वितीयपक्षे यदि रूपयुक्तः । }

{अव्यक्तवर्गोऽत्र कृतिप्रकृत्या साध्ये तदा ज्येष्ठकनिष्ठमूले ।। }

{ज्येष्ठं तयोः प्रथमपक्षपदेन तुल्यं कृत्वोक्तवत्प्रथमवर्णमिति प्रसाध्या
। }

{ह्रस्वं भवेत्प्रकृतिवर्णमितिः सुधीभिरेवं कृतिप्रकृतिरत्र नियोजनीया ।।
१५१ ।।}

{If it is possible to have the square root of one side and the other
side has an absolute term and the square of an unknown, by the method of
वर्गप्रकृति we should have ज्येष्ठ and कनिष्ठ (stanza 70). Equating the
square root of the first side with ज्येष्ठ we can get the value of x.
And the ह्रस्व should be taken as the value of the coefficient of
प्रकृति. This is how stanza 70 can be used.}

{को राशिर्द्विगुणो राशिवर्गैः षड्भिः समन्वितः । }

{मूलदो जायते बीजगीणतज्ञ वदाऽऽशु तम् ।। १५२ ।। }

{What is that number whose double added to six times its square is an
exact square?}

{राशियोगकृतिर्मिश्रा राश्योर्योगघनेन च । }

{द्विघ्नस्य घनयोगस्य सा तुल्या गणकोच्यताम् ।। १५३ ।। }

{Two numbers are such that the square of their sum added to the cube of
their sum is equal to twice the sum of their cubes. Give those two
numbers.}

{द्वितीयपक्षे सति संभवे तु कृत्याऽपवर्त्यात्र पदे प्रसाध्ये । }

{ज्येष्ठं कनिष्ठेन तथा निहन्याच्चेद्वर्गवर्गेण कृतोऽपवर्तः । । }

{कनिष्ठवर्गेण तदा निहन्याज्ज्येष्ठं ततः पूर्ववदेव शेषम् ।। १५४ ।। }

{It is possible to find the square root of the second side, we may
divide by the square of x By वर्गप्रकृति we can get ज्येष्ठ and कनिष्ठ.
Product of these two can be called new ज्येष्ठ. If we divide by x}{4}{
and then get ज्येष्ठ, कनिष्ठ we may assume the product of\\
}

\end{document}
