\documentclass[]{article}
\usepackage{lmodern}
\usepackage{amssymb,amsmath}
\usepackage{ifxetex,ifluatex}
\usepackage{fixltx2e} % provides \textsubscript
\ifnum 0\ifxetex 1\fi\ifluatex 1\fi=0 % if pdftex
  \usepackage[T1]{fontenc}
  \usepackage[utf8]{inputenc}
\else % if luatex or xelatex
  \ifxetex
    \usepackage{mathspec}
  \else
    \usepackage{fontspec}
  \fi
  \defaultfontfeatures{Ligatures=TeX,Scale=MatchLowercase}
\fi
% use upquote if available, for straight quotes in verbatim environments
\IfFileExists{upquote.sty}{\usepackage{upquote}}{}
% use microtype if available
\IfFileExists{microtype.sty}{%
\usepackage[]{microtype}
\UseMicrotypeSet[protrusion]{basicmath} % disable protrusion for tt fonts
}{}
\PassOptionsToPackage{hyphens}{url} % url is loaded by hyperref
\usepackage[unicode=true]{hyperref}
\hypersetup{
            pdfborder={0 0 0},
            breaklinks=true}
\urlstyle{same}  % don't use monospace font for urls
\IfFileExists{parskip.sty}{%
\usepackage{parskip}
}{% else
\setlength{\parindent}{0pt}
\setlength{\parskip}{6pt plus 2pt minus 1pt}
}
\setlength{\emergencystretch}{3em}  % prevent overfull lines
\providecommand{\tightlist}{%
  \setlength{\itemsep}{0pt}\setlength{\parskip}{0pt}}
\setcounter{secnumdepth}{0}
% Redefines (sub)paragraphs to behave more like sections
\ifx\paragraph\undefined\else
\let\oldparagraph\paragraph
\renewcommand{\paragraph}[1]{\oldparagraph{#1}\mbox{}}
\fi
\ifx\subparagraph\undefined\else
\let\oldsubparagraph\subparagraph
\renewcommand{\subparagraph}[1]{\oldsubparagraph{#1}\mbox{}}
\fi

% set default figure placement to htbp
\makeatletter
\def\fps@figure{htbp}
\makeatother


\date{}

\begin{document}

{\ldots{}.57\ldots{}..}

{Page}

{मध्यमाहरण a device to remove middle term in quadratic 33}

{मान value 38}

{मूल square root 9}

{युक्त accompanied 49}

{युगुल, युग्म pair 11, 32}

{युत united 50}

{युति, योग addition, sum 7}

{रहित decreased 47}

{राशि quantity 51}

{रूप number, one 13}

{लघु twice the product of two surds; small 13}

{लघुघ्नम् multiplied by laghu 13}

{लब्धि quotient 19}

{लम्ब, लम्बक perpendicular 31}

{वज्र diamond 29}

{वज्राभ्यास cross product 22}

{वध multiplication, product 11}

{वर्ग square 11}

{वर्गांतर difference of squares वर्जित subtracted 19}

{वर्ण algebraic number 10}

{विकला second of arc 21}

{विकलावशेष residue of vikala 21}

{वितस्ति a measure for length equal to half kar, twelve angulas 32}

{विनिघ्न multiplied विपर्यय not definite 50}

{विपर्यास change of sign 9}

{विभज्य after dividing 50}

{विभिन्न जाति dissimilar, unlike 10}

{वियोग, विवर difference subtraction 23}

{वियोगमूल square root of the difference of two unknowns }

{increased by an augment 46}

{विलोम reverse process 38}

{विवर्जित decreased, left 20}

{विशोध्य fit for subtraction 18}

{विश्लेषसूत्र rule to analyse 14}

{विषम odd 18}

{विहृत divided 15}

{व्यक्त, व्यक्तगणित arithmetic 7}

{व्यस्त opposite, 8}

{शंकु gnomon 35}

{शेष, शेषक remainder, residue 18}

{शुध्यति divided without a remainder 20}

{शोध्य see विशोध्य 18}

{श्रुति hypotenuse 36}

{सकलान्तर amount 29}

{संकलन addition 8}

{संक्रमण addition, subtraction and division by two 46}

{सम even, similar 18}

{समजातिक, समानजाति like 11}

{समास, संयुति sum 45}

{सहस्रधा in several ways 46}

{सहित increased सांख्य relating to number 7}

{सूत्र rule 14}

{सैक increased by one 13}

\end{document}
