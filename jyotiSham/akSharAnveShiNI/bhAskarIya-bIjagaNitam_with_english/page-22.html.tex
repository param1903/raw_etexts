\documentclass[]{article}
\usepackage{lmodern}
\usepackage{amssymb,amsmath}
\usepackage{ifxetex,ifluatex}
\usepackage{fixltx2e} % provides \textsubscript
\ifnum 0\ifxetex 1\fi\ifluatex 1\fi=0 % if pdftex
  \usepackage[T1]{fontenc}
  \usepackage[utf8]{inputenc}
\else % if luatex or xelatex
  \ifxetex
    \usepackage{mathspec}
  \else
    \usepackage{fontspec}
  \fi
  \defaultfontfeatures{Ligatures=TeX,Scale=MatchLowercase}
\fi
% use upquote if available, for straight quotes in verbatim environments
\IfFileExists{upquote.sty}{\usepackage{upquote}}{}
% use microtype if available
\IfFileExists{microtype.sty}{%
\usepackage[]{microtype}
\UseMicrotypeSet[protrusion]{basicmath} % disable protrusion for tt fonts
}{}
\PassOptionsToPackage{hyphens}{url} % url is loaded by hyperref
\usepackage[unicode=true]{hyperref}
\hypersetup{
            pdfborder={0 0 0},
            breaklinks=true}
\urlstyle{same}  % don't use monospace font for urls
\IfFileExists{parskip.sty}{%
\usepackage{parskip}
}{% else
\setlength{\parindent}{0pt}
\setlength{\parskip}{6pt plus 2pt minus 1pt}
}
\setlength{\emergencystretch}{3em}  % prevent overfull lines
\providecommand{\tightlist}{%
  \setlength{\itemsep}{0pt}\setlength{\parskip}{0pt}}
\setcounter{secnumdepth}{0}
% Redefines (sub)paragraphs to behave more like sections
\ifx\paragraph\undefined\else
\let\oldparagraph\paragraph
\renewcommand{\paragraph}[1]{\oldparagraph{#1}\mbox{}}
\fi
\ifx\subparagraph\undefined\else
\let\oldsubparagraph\subparagraph
\renewcommand{\subparagraph}[1]{\oldsubparagraph{#1}\mbox{}}
\fi

% set default figure placement to htbp
\makeatletter
\def\fps@figure{htbp}
\makeatother


\date{}

\begin{document}

{\ldots{}.20\ldots{}.}

{will be correct. Putting this value in the original equation we get the
value for लब्धि i. e. y.}

{क्षेपाभावोऽथवा यत्र क्षेपः शुध्येद्धरोद्धृतः । }

{ज्ञेयः शून्यं गुणस्तत्र क्षेपो हरहृतः फलम् ।। ५८ ।। }

{When the remainder is zero or where it is divisble by the divisor, x
will be zero. And the quotient obtained by dividing the remainder by the
divisor will be the value of y. }

{इष्टाहतस्वस्वहरेण युक्ते ते वा भवेतां बहुधा गुणाप्ती ।। ५९ ।। }

{If we multiply the divisor and the dividend by any number and add the
respective products to the values of y and x already obtained we get
infinite values for (y, x).}

{एकविंशतियुतं शतद्वयं यद्गुणं गणक पञ्चषष्टियुक्। }

{पञ्चवर्जितशतद्वयोद्धृतं शुद्धिमेति गुणकं वदाऽऽशु तम् ।। ६० ।। }

{Find an integral number such that when it is multiplied by 221 and
increased by 65 the result is divisible by 195 without a remainder.}

{शतं हतं येन युतं नवत्या विवर्जितं वा विहृतं त्रिशष्ट्या । }

{निरग्रकं स्याद् वद मे गुणं तं स्पष्टं पटीयान् यदि कुट्टकेऽसि ।। ६१ ।। }

{Find that number which multiplied by 100 and increased by 90 is
divisible by 63 without a remainder.}

{यद् गुणाक्षयगषष्टिरन्विता वर्जिता च यदि वा त्रिभिस्ततः । }

{स्यात् त्रयोदशहृता निरग्रका तं गुणं गणक मे पृथग् वद ।। ६२ ।। }

{What is that number which multiplied by -60 and increased by 3 or
decreased by 3 divisible by 13 without a remainder?}

{अष्टादश गुणाः केन दशाढ्या वा दशोनिताः । }

{शुद्धं भागं प्रयच्छन्ति क्षयगैकादशोद्धृताः ।। ६३ ।। }

{What is that number which multiplied by 18 and increased by 10 or
decreased by 10 is divisible by -11 without a remainder?}

\end{document}
