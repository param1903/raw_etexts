\documentclass[]{article}
\usepackage{lmodern}
\usepackage{amssymb,amsmath}
\usepackage{ifxetex,ifluatex}
\usepackage{fixltx2e} % provides \textsubscript
\ifnum 0\ifxetex 1\fi\ifluatex 1\fi=0 % if pdftex
  \usepackage[T1]{fontenc}
  \usepackage[utf8]{inputenc}
\else % if luatex or xelatex
  \ifxetex
    \usepackage{mathspec}
  \else
    \usepackage{fontspec}
  \fi
  \defaultfontfeatures{Ligatures=TeX,Scale=MatchLowercase}
\fi
% use upquote if available, for straight quotes in verbatim environments
\IfFileExists{upquote.sty}{\usepackage{upquote}}{}
% use microtype if available
\IfFileExists{microtype.sty}{%
\usepackage[]{microtype}
\UseMicrotypeSet[protrusion]{basicmath} % disable protrusion for tt fonts
}{}
\PassOptionsToPackage{hyphens}{url} % url is loaded by hyperref
\usepackage[unicode=true]{hyperref}
\hypersetup{
            pdfborder={0 0 0},
            breaklinks=true}
\urlstyle{same}  % don't use monospace font for urls
\IfFileExists{parskip.sty}{%
\usepackage{parskip}
}{% else
\setlength{\parindent}{0pt}
\setlength{\parskip}{6pt plus 2pt minus 1pt}
}
\setlength{\emergencystretch}{3em}  % prevent overfull lines
\providecommand{\tightlist}{%
  \setlength{\itemsep}{0pt}\setlength{\parskip}{0pt}}
\setcounter{secnumdepth}{0}
% Redefines (sub)paragraphs to behave more like sections
\ifx\paragraph\undefined\else
\let\oldparagraph\paragraph
\renewcommand{\paragraph}[1]{\oldparagraph{#1}\mbox{}}
\fi
\ifx\subparagraph\undefined\else
\let\oldsubparagraph\subparagraph
\renewcommand{\subparagraph}[1]{\oldsubparagraph{#1}\mbox{}}
\fi

% set default figure placement to htbp
\makeatletter
\def\fps@figure{htbp}
\makeatother


\date{}

\begin{document}

{\ldots{}.19\ldots{}. }

{भवति कुट्टविधेर्युतिभाज्ययोः समपवर्तितयोरपि वा गुणः । }

{भवति यो युति भाजकयोः पुनः स च भवेद् अपवर्तनसंगुणः ।। ५३ ।। }

{If we divide c and a by a common factor and then adopt the कुट्टक
process we get correct value for x but not for y . To get the value for
y for the original equation, the value got by the process should be
multiplied by the common factor.}

{योगजे तक्षणाच्छच्छद्धे गुणाप्ती स्तो वियोगजे । }

{धनभाज्योद्भवे तद्वद् भवेताम् ऋणभाज्यजे ।। ५४ ।। }

{The values of x and y obtained when c is positive must be subtracted
from b and a respectively for the case when c is negative. In the same
way the values of x and y when a is positive must be subtracted from b
and a for the case when a is negative. }

{गुणलब्ध्योः समं ग्राह्यं धीमता तक्षणे फलम् ।। ५५ ।। }

{In the making selection of proper pair (x, y) the intelligent person
will take care to see that the values correspond with each other.}

{हरतष्टे धनक्षेपे गुणलब्धी तु पूर्ववत् । }

{क्षेपतक्षणलाभाढ्या लब्धिः शुद्धौ तु वर्जिता ।। ५६ ।। }

{When c\textgreater{}b divide c by b and take the remainder as new c and
calculate (x, y) as before. Value of x will be correct. To get the
correct value of y, to its calculated value add the quotient obtained
when b divides c. If c is negative this quotient should be subtracted
from the calculated value.}

{अथवा भागहारेण तष्टयोः क्षेपभाज्ययोः । }

{गुणः प्राग्वत् ततो लब्धिर् भाज्याद्धत युतोद् धृतात् ।। ५७ ।। }

{Alternative method : Divide a by b and c by b and take the respective
remainders for new a and new c. Calculate गुण and लब्धि by the process.
गुण}

\end{document}
