\documentclass[]{article}
\usepackage{lmodern}
\usepackage{amssymb,amsmath}
\usepackage{ifxetex,ifluatex}
\usepackage{fixltx2e} % provides \textsubscript
\ifnum 0\ifxetex 1\fi\ifluatex 1\fi=0 % if pdftex
  \usepackage[T1]{fontenc}
  \usepackage[utf8]{inputenc}
\else % if luatex or xelatex
  \ifxetex
    \usepackage{mathspec}
  \else
    \usepackage{fontspec}
  \fi
  \defaultfontfeatures{Ligatures=TeX,Scale=MatchLowercase}
\fi
% use upquote if available, for straight quotes in verbatim environments
\IfFileExists{upquote.sty}{\usepackage{upquote}}{}
% use microtype if available
\IfFileExists{microtype.sty}{%
\usepackage[]{microtype}
\UseMicrotypeSet[protrusion]{basicmath} % disable protrusion for tt fonts
}{}
\PassOptionsToPackage{hyphens}{url} % url is loaded by hyperref
\usepackage[unicode=true]{hyperref}
\hypersetup{
            pdfborder={0 0 0},
            breaklinks=true}
\urlstyle{same}  % don't use monospace font for urls
\IfFileExists{parskip.sty}{%
\usepackage{parskip}
}{% else
\setlength{\parindent}{0pt}
\setlength{\parskip}{6pt plus 2pt minus 1pt}
}
\setlength{\emergencystretch}{3em}  % prevent overfull lines
\providecommand{\tightlist}{%
  \setlength{\itemsep}{0pt}\setlength{\parskip}{0pt}}
\setcounter{secnumdepth}{0}
% Redefines (sub)paragraphs to behave more like sections
\ifx\paragraph\undefined\else
\let\oldparagraph\paragraph
\renewcommand{\paragraph}[1]{\oldparagraph{#1}\mbox{}}
\fi
\ifx\subparagraph\undefined\else
\let\oldsubparagraph\subparagraph
\renewcommand{\subparagraph}[1]{\oldsubparagraph{#1}\mbox{}}
\fi

% set default figure placement to htbp
\makeatletter
\def\fps@figure{htbp}
\makeatother


\date{}

\begin{document}

\ldots{}.56\ldots{}.

Page Page

खण्डगुणन multiplication by धन, धनात्मक positive 9\\
distribution 11 धात्री base of a triangle 31\\
खहर, खहार with zero divisor { } 10 निघ्न multiplied by { } 21\\
गच्छ number of terms { }34 निरग्र, निरग्रक without\\
गणित mathematics, remainder 21\\
arithmetic 11 निरेक decreased by one 13\\
गुण, गुणक, गुणकार multiplier 18 निहति product 50\\
गुणनजफल product 13 निहत्य after multiplying 33\\
गुण्य multiplicand { } { } 13 पक्ष side पक्षद्वय both sides 33\\
घन cube { }{ }पञ्चांशक one fifth 35\\
घनपद घनमूल cube root { }11 { }पद square root, root\\
घात product 31 पदप्रद perfect square 33\\
चक्रवाल circle, cyclic 23 प्रकृति coefficient of {x}{2 }22\\
चतुर्गुण, चतुराहत four times 33 प्रच्युत, प्रोज्झ्य sabtracted 12\\
चय common difference 34 फल area, interest, quotient,\\
च्छिन्न divided च्युत removed 17 result of calculation 18\\
छेद denominator, divisor { } 12 बीज basis, origin 7\\
ज्येष्ठ, ज्येष्ठमूल second बीजक्रिया process of algebra 50\\
variable 22 बीजगणित algebra 42\\
तक्षण cutting, 18 भक्त divided 49\\
तष्ट divided तुल्य equal 18,50 भागहार division, divisor 8\\
त्र्यस्र right angled triangle { }31 भाजक divisor 19\\
त्र्यून decreased by three { }35 भाज्य dividend 12\\
दुष्ट defective, improper, 17 भावना generator 22\\
wrong 18 भावित product of dissimilar\\
दृढ firm, puccā unknowns 11\\
दृढ भाज्य dividend in its 18 भावितज्ञ expert in solving\\
lowest terms bhāvita 50\\
दृढहार divisor in its lowest\\
terms 18 {भिन्न} fraction 8\\
दोः arm, horizontal side in भुज see दो; 37\\
a right angled triangle { } 37 भू base 30\\
द्रम्म a coin in use 30 महती sum of two numbers\\
द्विविध of two kinds under radical sign 13

\end{document}
