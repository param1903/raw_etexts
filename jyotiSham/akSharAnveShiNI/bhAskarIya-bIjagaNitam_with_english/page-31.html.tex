\documentclass[]{article}
\usepackage{lmodern}
\usepackage{amssymb,amsmath}
\usepackage{ifxetex,ifluatex}
\usepackage{fixltx2e} % provides \textsubscript
\ifnum 0\ifxetex 1\fi\ifluatex 1\fi=0 % if pdftex
  \usepackage[T1]{fontenc}
  \usepackage[utf8]{inputenc}
\else % if luatex or xelatex
  \ifxetex
    \usepackage{mathspec}
  \else
    \usepackage{fontspec}
  \fi
  \defaultfontfeatures{Ligatures=TeX,Scale=MatchLowercase}
\fi
% use upquote if available, for straight quotes in verbatim environments
\IfFileExists{upquote.sty}{\usepackage{upquote}}{}
% use microtype if available
\IfFileExists{microtype.sty}{%
\usepackage[]{microtype}
\UseMicrotypeSet[protrusion]{basicmath} % disable protrusion for tt fonts
}{}
\PassOptionsToPackage{hyphens}{url} % url is loaded by hyperref
\usepackage[unicode=true]{hyperref}
\hypersetup{
            pdfborder={0 0 0},
            breaklinks=true}
\urlstyle{same}  % don't use monospace font for urls
\IfFileExists{parskip.sty}{%
\usepackage{parskip}
}{% else
\setlength{\parindent}{0pt}
\setlength{\parskip}{6pt plus 2pt minus 1pt}
}
\setlength{\emergencystretch}{3em}  % prevent overfull lines
\providecommand{\tightlist}{%
  \setlength{\itemsep}{0pt}\setlength{\parskip}{0pt}}
\setcounter{secnumdepth}{0}
% Redefines (sub)paragraphs to behave more like sections
\ifx\paragraph\undefined\else
\let\oldparagraph\paragraph
\renewcommand{\paragraph}[1]{\oldparagraph{#1}\mbox{}}
\fi
\ifx\subparagraph\undefined\else
\let\oldsubparagraph\subparagraph
\renewcommand{\subparagraph}[1]{\oldsubparagraph{#1}\mbox{}}
\fi

% set default figure placement to htbp
\makeatletter
\def\fps@figure{htbp}
\makeatother


\date{}

\begin{document}

{\ldots{}.29\ldots{}.}

{sums.}

{एकशतदत्तधनात्फलस्य वर्गं विशोध्य परिशिष्टम् । }

{पञ्चकशतेन दत्तं तुल्यः कालः फलं च तयोः ।। ९७ ।। }

{A sum was lent out at 1 p. c. p. m. subtracting the square of the
interest for some period from the sum the remainder was lent out at 5 p.
c. p. m. To produce the same interest as before, time was the same. Find
the sum.}

{माणिक्याष्टकमिन्द्रनीलदशकं मुक्ताफलानां शतं }

{सद्वज्राणि च पञ्च रत्नवणिजां येषां चतुर्णां धनम् । }

{संगस्नेहवशेन ते निजधनाद् दत्त्वैकमेकं मिथो }

{जातास्तुल्यधनाः पृथग्वद सखे तद्रत्नमौल्यानि मे ।। ९८ ।। }

{There were four jewel mechants. First had 8 rubies, second had 10
saphires, third had 100 pearls and fourth had 5 diamonds. Because of
friendly love each gave one jewel to others and they became equal in
wealth. Tell me the price of each jewel separately.}

{पञ्जकशतेन दत्तं मलं सकलान्तरं गते वर्षे । }

{द्विगुणं षोडशहीनं लब्धं किं मूलमाचक्ष्व ।। ९९ ।। }

{A sum was lent out at 5 p. c. p. m. for a year. The sum added to the
interest is less than twice the sum by 16. Tell me the sum.}

{यत्पञ्जकद्विकचतुष्कशतेन दत्तं खण्डैस्त्रिभिर्नवतियुक् त्रिशती धनं तत् ।
}

{मासेषु सप्तदशपञ्चसु तुल्यमाप्तं खण्डत्रयेऽपि सकलं वद }

{खण्डसंख्याम् ।। १०० ।। }

{Rupees 390 were divided into three parts and they were lent out at 5, 2
and 4 p. c. p. m. After 7, 10 and 5 months respectively the amounts were
equal. Tell me the sums in three parts.}

{पुरप्रवेशे दशदो द्विसंगुणं विधाय शेषं दशभुक् च निर्गमे । }

{ददौ दशैवं नगरत्रयेऽभवत्त्रिनिघ्नमाद्यं वद तत्कियद्धनम् ।। १०१ ।। }

{A merchant started with a sum. Entering a }

\end{document}
