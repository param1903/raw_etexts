\documentclass[]{article}
\usepackage{lmodern}
\usepackage{amssymb,amsmath}
\usepackage{ifxetex,ifluatex}
\usepackage{fixltx2e} % provides \textsubscript
\ifnum 0\ifxetex 1\fi\ifluatex 1\fi=0 % if pdftex
  \usepackage[T1]{fontenc}
  \usepackage[utf8]{inputenc}
\else % if luatex or xelatex
  \ifxetex
    \usepackage{mathspec}
  \else
    \usepackage{fontspec}
  \fi
  \defaultfontfeatures{Ligatures=TeX,Scale=MatchLowercase}
\fi
% use upquote if available, for straight quotes in verbatim environments
\IfFileExists{upquote.sty}{\usepackage{upquote}}{}
% use microtype if available
\IfFileExists{microtype.sty}{%
\usepackage[]{microtype}
\UseMicrotypeSet[protrusion]{basicmath} % disable protrusion for tt fonts
}{}
\PassOptionsToPackage{hyphens}{url} % url is loaded by hyperref
\usepackage[unicode=true]{hyperref}
\hypersetup{
            pdfborder={0 0 0},
            breaklinks=true}
\urlstyle{same}  % don't use monospace font for urls
\IfFileExists{parskip.sty}{%
\usepackage{parskip}
}{% else
\setlength{\parindent}{0pt}
\setlength{\parskip}{6pt plus 2pt minus 1pt}
}
\setlength{\emergencystretch}{3em}  % prevent overfull lines
\providecommand{\tightlist}{%
  \setlength{\itemsep}{0pt}\setlength{\parskip}{0pt}}
\setcounter{secnumdepth}{0}
% Redefines (sub)paragraphs to behave more like sections
\ifx\paragraph\undefined\else
\let\oldparagraph\paragraph
\renewcommand{\paragraph}[1]{\oldparagraph{#1}\mbox{}}
\fi
\ifx\subparagraph\undefined\else
\let\oldsubparagraph\subparagraph
\renewcommand{\subparagraph}[1]{\oldsubparagraph{#1}\mbox{}}
\fi

% set default figure placement to htbp
\makeatletter
\def\fps@figure{htbp}
\makeatother


\date{}

\begin{document}

{\ldots{}.37\ldots{}.}

{famous theorem. Give the proof of this.}

{दोः कोट्यन्तरवर्गेण द्विघ्नो घातः समन्वितः । }

{वर्गयोगसमः स स्याद्द्वयोरव्यक्तयोर्यथा ।। १२९ ।। }

{(दोः - कोटी)}{2}{ + २ दोः x कोटी = दोः}{2}{ + कोटी}{2}

{is like the theorem for two unknowns.}

{भुजात्त्र्यूनात्पदं व्येकं कोटिकर्णान्तरं सखे । }

{यत्र तत्र वद क्षेत्रे दोःकोटिश्रवणान्मम ।। १३० ।।}

{Find the three sides of a right angled triangle, given that}

{hypotenuse - vertical side = }{√}{base - 3-1}

{वर्गयोगस्य यद्राश्योर्युतिवर्गस्य चान्तरम् । }

{द्वि}{ग्न}{घातसमानं स्याद्वयोरव्यक्तयोर्यथा ।। }

{चतुर्गुणस्य घातस्य युतिवर्गस्य चान्तरम् । }

{राश्यन्तरकृतेस्तुल्यं द्वयोरव्यक्तयोर्यथा ।। १३१ ।। }

{( x + y )}{2}{ - (x}{2}{ +y}{2}{ ) = 2xy}

{and ( x + y )}{2}{ - 4xy = (x - y)}{2}

{These rules for two unknowns are applicable for two numbers.}

{चत्वारिंशद्युतिर्येषां दोः कोटिश्रवसां वद}

{भुजकोटिवधो येषु शतं विंशतिसंयुतम् ।। १३२ ।। }

{Sum of three sides of a right angled triangle is 40 and product of two
sides which contain the right angle is 120. Please give all the three
sides.}

{योगो दोःकोटिकर्णानां षट्पञ्चाशद्वधस्तथा । }

{षट्शती सप्तभिः क्षुण्णा येषां तान्मेपृथग्वद ।। १३३ ।। }

{Sum of three sides of a right angled triangle is 56, product of the
three sides is 7 x 600. Find all the three sides separately.\\
}

\end{document}
