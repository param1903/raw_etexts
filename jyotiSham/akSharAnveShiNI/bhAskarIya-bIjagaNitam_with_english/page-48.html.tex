\documentclass[]{article}
\usepackage{lmodern}
\usepackage{amssymb,amsmath}
\usepackage{ifxetex,ifluatex}
\usepackage{fixltx2e} % provides \textsubscript
\ifnum 0\ifxetex 1\fi\ifluatex 1\fi=0 % if pdftex
  \usepackage[T1]{fontenc}
  \usepackage[utf8]{inputenc}
\else % if luatex or xelatex
  \ifxetex
    \usepackage{mathspec}
  \else
    \usepackage{fontspec}
  \fi
  \defaultfontfeatures{Ligatures=TeX,Scale=MatchLowercase}
\fi
% use upquote if available, for straight quotes in verbatim environments
\IfFileExists{upquote.sty}{\usepackage{upquote}}{}
% use microtype if available
\IfFileExists{microtype.sty}{%
\usepackage[]{microtype}
\UseMicrotypeSet[protrusion]{basicmath} % disable protrusion for tt fonts
}{}
\PassOptionsToPackage{hyphens}{url} % url is loaded by hyperref
\usepackage[unicode=true]{hyperref}
\hypersetup{
            pdfborder={0 0 0},
            breaklinks=true}
\urlstyle{same}  % don't use monospace font for urls
\IfFileExists{parskip.sty}{%
\usepackage{parskip}
}{% else
\setlength{\parindent}{0pt}
\setlength{\parskip}{6pt plus 2pt minus 1pt}
}
\setlength{\emergencystretch}{3em}  % prevent overfull lines
\providecommand{\tightlist}{%
  \setlength{\itemsep}{0pt}\setlength{\parskip}{0pt}}
\setcounter{secnumdepth}{0}
% Redefines (sub)paragraphs to behave more like sections
\ifx\paragraph\undefined\else
\let\oldparagraph\paragraph
\renewcommand{\paragraph}[1]{\oldparagraph{#1}\mbox{}}
\fi
\ifx\subparagraph\undefined\else
\let\oldsubparagraph\subparagraph
\renewcommand{\subparagraph}[1]{\oldsubparagraph{#1}\mbox{}}
\fi

% set default figure placement to htbp
\makeatletter
\def\fps@figure{htbp}
\makeatother


\date{}

\begin{document}

{\ldots{}.46\ldots{}.}

{यच्चैतत्पदपञ्चकं च मिलितं स्याद्वर्गमूलप्रदं }

{तौ राशी कथयाऽऽशु निश्चलमते षट्काष्टकाभ्यां विना ।। १६४ ।। }

{We have two different numbers. To their product we add the smaller
number and find the cube root of half the sum. We find the square root
of the sum of their squares. We find the square root of their sum
increased by 2. Then we find the square root of the difference of the
squares of those numbers increased by 8. It is observed that the sum of
these 5 roots is a perfect square. Find the two numbers other than 6 and
8.}

{एवं सहस्रधा गूढा मूढानां कल्पना यतः । }

{क्रियया कल्पनोपायस्तदेर्थमथ कथ्यते ।। १६५ । । }

{In this way we can form an idea in several ways. But for common man
that is difficult. Therefore, the device to the thought is being given
here,}

{सरूपमव्यक्तमरूपकं वा वियोगमूलं प्रथमं प्रकल्प्यम् । }

{योगान्तरक्षेपकभाजिताद्यद्वर्गान्तरक्षेपकतः पदं स्यात् । । }

{तेनाधिकं तत्तु वियोगमूलं स्याद्योगमूलं तु तयोस्तु वर्गौ । }

{स्वक्षेपकोनौ हि वियोगयोगौ स्यातां ततः संक्रमणेन राशी ।। १६६ ।। }

{First of all we can think of a new unknown with a number or without a
number as the square root of the difference of two unknowns increased by
an augment (कल्पित वियोगमूल). After that we divide the augment
corresponding to the difference of squares of the given unknowns by the
augment of difference of these unknowns and find the square root of the
quotient. When that root is added to the root of वियोगमूल as assumed
before we get the योगमूल. After that the योगमूल and वियोगमूल are squared
and from them the respective augments are\\
}

\end{document}
