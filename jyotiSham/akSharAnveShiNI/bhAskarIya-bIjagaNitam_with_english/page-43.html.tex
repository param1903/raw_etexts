\documentclass[]{article}
\usepackage{lmodern}
\usepackage{amssymb,amsmath}
\usepackage{ifxetex,ifluatex}
\usepackage{fixltx2e} % provides \textsubscript
\ifnum 0\ifxetex 1\fi\ifluatex 1\fi=0 % if pdftex
  \usepackage[T1]{fontenc}
  \usepackage[utf8]{inputenc}
\else % if luatex or xelatex
  \ifxetex
    \usepackage{mathspec}
  \else
    \usepackage{fontspec}
  \fi
  \defaultfontfeatures{Ligatures=TeX,Scale=MatchLowercase}
\fi
% use upquote if available, for straight quotes in verbatim environments
\IfFileExists{upquote.sty}{\usepackage{upquote}}{}
% use microtype if available
\IfFileExists{microtype.sty}{%
\usepackage[]{microtype}
\UseMicrotypeSet[protrusion]{basicmath} % disable protrusion for tt fonts
}{}
\PassOptionsToPackage{hyphens}{url} % url is loaded by hyperref
\usepackage[unicode=true]{hyperref}
\hypersetup{
            pdfborder={0 0 0},
            breaklinks=true}
\urlstyle{same}  % don't use monospace font for urls
\IfFileExists{parskip.sty}{%
\usepackage{parskip}
}{% else
\setlength{\parindent}{0pt}
\setlength{\parskip}{6pt plus 2pt minus 1pt}
}
\setlength{\emergencystretch}{3em}  % prevent overfull lines
\providecommand{\tightlist}{%
  \setlength{\itemsep}{0pt}\setlength{\parskip}{0pt}}
\setcounter{secnumdepth}{0}
% Redefines (sub)paragraphs to behave more like sections
\ifx\paragraph\undefined\else
\let\oldparagraph\paragraph
\renewcommand{\paragraph}[1]{\oldparagraph{#1}\mbox{}}
\fi
\ifx\subparagraph\undefined\else
\let\oldsubparagraph\subparagraph
\renewcommand{\subparagraph}[1]{\oldsubparagraph{#1}\mbox{}}
\fi

% set default figure placement to htbp
\makeatletter
\def\fps@figure{htbp}
\makeatother


\date{}

\begin{document}

{\ldots{}.41\ldots{}.}

{नवभिः सप्तभिः क्षुण्णः को राशिस्त्रिंशता हृतः । }

{यदग्रैक्यं फलैक्याढ्यं भवेत्षड्विंशतेर्मितम् ।। १४३ ।। }

{What is that number which multiplied by 9 and 7 separately and divided
by 30 give such quotients and remainders that the sum of these four is
26?}

{कस्त्रिसप्तनवक्षुण्णो राशिस्त्रिंशद्विभाजितः । }

{यदग्रैक्यमपि त्रिंशद्धृतमेकादशाग्रकम् ।। १४४ ।। }

{What is that number which multiplied by 3, 7 and 9 separately and
divided by 30 give such remainders that the sum of the three remainders
on being divided by 30 leave 11 as the remainder?}

{कस्त्रयोविंशतिक्षुण्णः षष्ट्याऽशीत्या हृतः पृथक् }

{यदग्रैवयं शतं दृष्टं कुट्टकज्ञ वदाऽऽशु तम् ।। १४५।। }

{What is that number which multiplied by 23 divided by 60 and 80
separately give remainders whose sum is 100?}

{अत्राधिकस्य वर्णस्य भाज्यस्थस्येप्सिता मितिः । }

{भागलब्धस्य नो कल्प्या क्रिया व्यभिचरेत्तथा ।। १४६ ।। }

{The variable multiplying dividend and that multiplying the divisor in a
kuttak is known as guna and labdhi. These x and y are expressed in terms
of additional variable say t. This t cannot be given values at random. }

{कः पञ्चगुणितो राशिस्त्रयोदशविभाजितः । }

{यल्लब्धं राशिना युक्तं त्रिंशज्जातं वदाऽऽशु तम् ।। १४७ । । }

{Find a number such that when it is multiplied by 5 and then divided by
13, the quotient added to the number gives 30 as the sum.}

{षडष्टशतकाः क्रीत्वा समार्घेन फलानि ये । }

{विक्रीय च पुनः शेषमेकैकं पञ्चभिः पणैः ।। }

{जाताः समपणास्तेषां कः क्रयो विक्रयश्च कः ।। १४८।।}

{\ldots{}.4\ldots{}.\\
}

\end{document}
