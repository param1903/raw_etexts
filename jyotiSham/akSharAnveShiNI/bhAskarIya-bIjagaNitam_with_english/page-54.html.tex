\documentclass[]{article}
\usepackage{lmodern}
\usepackage{amssymb,amsmath}
\usepackage{ifxetex,ifluatex}
\usepackage{fixltx2e} % provides \textsubscript
\ifnum 0\ifxetex 1\fi\ifluatex 1\fi=0 % if pdftex
  \usepackage[T1]{fontenc}
  \usepackage[utf8]{inputenc}
\else % if luatex or xelatex
  \ifxetex
    \usepackage{mathspec}
  \else
    \usepackage{fontspec}
  \fi
  \defaultfontfeatures{Ligatures=TeX,Scale=MatchLowercase}
\fi
% use upquote if available, for straight quotes in verbatim environments
\IfFileExists{upquote.sty}{\usepackage{upquote}}{}
% use microtype if available
\IfFileExists{microtype.sty}{%
\usepackage[]{microtype}
\UseMicrotypeSet[protrusion]{basicmath} % disable protrusion for tt fonts
}{}
\PassOptionsToPackage{hyphens}{url} % url is loaded by hyperref
\usepackage[unicode=true]{hyperref}
\hypersetup{
            pdfborder={0 0 0},
            breaklinks=true}
\urlstyle{same}  % don't use monospace font for urls
\IfFileExists{parskip.sty}{%
\usepackage{parskip}
}{% else
\setlength{\parindent}{0pt}
\setlength{\parskip}{6pt plus 2pt minus 1pt}
}
\setlength{\emergencystretch}{3em}  % prevent overfull lines
\providecommand{\tightlist}{%
  \setlength{\itemsep}{0pt}\setlength{\parskip}{0pt}}
\setcounter{secnumdepth}{0}
% Redefines (sub)paragraphs to behave more like sections
\ifx\paragraph\undefined\else
\let\oldparagraph\paragraph
\renewcommand{\paragraph}[1]{\oldparagraph{#1}\mbox{}}
\fi
\ifx\subparagraph\undefined\else
\let\oldsubparagraph\subparagraph
\renewcommand{\subparagraph}[1]{\oldsubparagraph{#1}\mbox{}}
\fi

% set default figure placement to htbp
\makeatletter
\def\fps@figure{htbp}
\makeatother


\date{}

\begin{document}

{ग्रंथसमाप्तिः । }

{EPILOGUE}

{आसीन्महेश्वर इति प्रथितः पृथिव्या- }

{माचार्यवर्यपदवीं विदुषां प्रयातः । }

{लब्ध्वाऽवबोध कलिकां तत एव चक्रे }

{तज्जेन बीजगणितं लघ भास्करेण ।। १ ।। }

{There was a renowned professor Maheshwar by name. He held the highest
honour in the academic field. His son Bhaskar who got the spark of
knowledge from his father, is said to have compiled this concise
bijaganita.}

{ब्रह्माह्वयश्रीधरपद्मनाभबीजानि यस्मादतिविस्तृतानि । }

{आदाय तत्सारमकारि नूनं सद्युक्तियुक्तं लघु शिष्यतुष्ट्यै ।। २ ।।}{ }

{Before him were books on bijaganita by Brahmagupta, Shridhar and
Padmanabh. They are stupendous. So for the satisfaction of pupils, by
taking good points from those, this concise book has been compiled
containing some nice devices. }

{अत्रानुष्टुप्सहस्रं हि ससूत्रोद्देशके मितिः ।। ३ ।।}{ }

{Here are one thousand of anushtup measure. They include formulas and
examples.}

{क्वचित्सूत्रार्थविषयं व्याप्तिं दर्शयितुं क्वचित् । }

{क्वचिच्च कल्पनाभेदं क्वचिद्युक्तिमुदाहृतम् ।। }

{क्वचित्सूत्रार्थविषयं दर्शयितुमुदाहृतम् ।। ४ ।।}{ }

{Here and there we find subject and scope the formula; sometimes we get
variety in thought and often devices.}

{न हि उदाहरणान्तोऽस्ति स्तोकमुक्तमिदं यतः ।। ५ ।। }

{Examples have no end, hence this is given in few words.\\
}

\end{document}
