\documentclass[]{article}
\usepackage{lmodern}
\usepackage{amssymb,amsmath}
\usepackage{ifxetex,ifluatex}
\usepackage{fixltx2e} % provides \textsubscript
\ifnum 0\ifxetex 1\fi\ifluatex 1\fi=0 % if pdftex
  \usepackage[T1]{fontenc}
  \usepackage[utf8]{inputenc}
\else % if luatex or xelatex
  \ifxetex
    \usepackage{mathspec}
  \else
    \usepackage{fontspec}
  \fi
  \defaultfontfeatures{Ligatures=TeX,Scale=MatchLowercase}
\fi
% use upquote if available, for straight quotes in verbatim environments
\IfFileExists{upquote.sty}{\usepackage{upquote}}{}
% use microtype if available
\IfFileExists{microtype.sty}{%
\usepackage[]{microtype}
\UseMicrotypeSet[protrusion]{basicmath} % disable protrusion for tt fonts
}{}
\PassOptionsToPackage{hyphens}{url} % url is loaded by hyperref
\usepackage[unicode=true]{hyperref}
\hypersetup{
            pdfborder={0 0 0},
            breaklinks=true}
\urlstyle{same}  % don't use monospace font for urls
\IfFileExists{parskip.sty}{%
\usepackage{parskip}
}{% else
\setlength{\parindent}{0pt}
\setlength{\parskip}{6pt plus 2pt minus 1pt}
}
\setlength{\emergencystretch}{3em}  % prevent overfull lines
\providecommand{\tightlist}{%
  \setlength{\itemsep}{0pt}\setlength{\parskip}{0pt}}
\setcounter{secnumdepth}{0}
% Redefines (sub)paragraphs to behave more like sections
\ifx\paragraph\undefined\else
\let\oldparagraph\paragraph
\renewcommand{\paragraph}[1]{\oldparagraph{#1}\mbox{}}
\fi
\ifx\subparagraph\undefined\else
\let\oldsubparagraph\subparagraph
\renewcommand{\subparagraph}[1]{\oldsubparagraph{#1}\mbox{}}
\fi

% set default figure placement to htbp
\makeatletter
\def\fps@figure{htbp}
\makeatother


\date{}

\begin{document}

{\ldots{}.13\ldots{}.}

{यावत्तावत्त्रयमृणमृणं कालकौ नीलकः स्वं }

{रूपेणाऽऽढ्या द्विगुणितमितैस्तैस्तु तैरेव निघ्नाः । }

{किं स्यात्तेषां गुणनजफलं गुण्यभक्तं च किं स्याद् }

{गुण्यस्याथ प्रकथय कृतिं मूलमस्याः कृतेश्च ।। ३३ ।। }

{If the expression -3x-2y+z+1 is multiplied by its double, what is the
product? If this product is divided by the origival expression what will
be the quotient? Also find the square of the original expression and
work out the process of extracting the root of the square.}

{ }

{४ करणीषड्विधम् । }

{Six laws for surds.}

{योगं करण्योर्महतीं प्रकल्प्य घातस्य मूलं द्विगुणं लघुं च । }

{योगान्तरे रूपवदेतयोस्ते वर्गेण वर्गं गुणयेद् भजेच्च । । }

{लघ्व्या हतायास्तु पदं महत्या सैकं निरेकं स्वहतं लघुघ्नम् । }

{योगान्तरे स्तः क्रमशस्तयोर्वा पृथक् स्थितिः स्याद् यदि नास्ति मूलम ।।
३४ ।। }

{The sum of two numbers under the root sign denoted by M. Twice the
square root of their product is denoted ny L. The sum and difference of
the two surds are respectively}{√}{M+L }{and}{√}{M-L.}

{If a surd is to be multiplied or divided by a given number, multiply or
divide the number under the radical sign by the square of the given
number.}

{Another method for adding and subtracting two given surds: Let the
greater surd be }{√}{G and the smaller surd be }{√}{S. Extract the
square root of G/S. To the square root add +1 and -1 separately.
Squaring the two results and multiplying them by S we get the sum and
difference.\\
}

\end{document}
