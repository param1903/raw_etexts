\documentclass[]{article}
\usepackage{lmodern}
\usepackage{amssymb,amsmath}
\usepackage{ifxetex,ifluatex}
\usepackage{fixltx2e} % provides \textsubscript
\ifnum 0\ifxetex 1\fi\ifluatex 1\fi=0 % if pdftex
  \usepackage[T1]{fontenc}
  \usepackage[utf8]{inputenc}
\else % if luatex or xelatex
  \ifxetex
    \usepackage{mathspec}
  \else
    \usepackage{fontspec}
  \fi
  \defaultfontfeatures{Ligatures=TeX,Scale=MatchLowercase}
\fi
% use upquote if available, for straight quotes in verbatim environments
\IfFileExists{upquote.sty}{\usepackage{upquote}}{}
% use microtype if available
\IfFileExists{microtype.sty}{%
\usepackage[]{microtype}
\UseMicrotypeSet[protrusion]{basicmath} % disable protrusion for tt fonts
}{}
\PassOptionsToPackage{hyphens}{url} % url is loaded by hyperref
\usepackage[unicode=true]{hyperref}
\hypersetup{
            pdfborder={0 0 0},
            breaklinks=true}
\urlstyle{same}  % don't use monospace font for urls
\IfFileExists{parskip.sty}{%
\usepackage{parskip}
}{% else
\setlength{\parindent}{0pt}
\setlength{\parskip}{6pt plus 2pt minus 1pt}
}
\setlength{\emergencystretch}{3em}  % prevent overfull lines
\providecommand{\tightlist}{%
  \setlength{\itemsep}{0pt}\setlength{\parskip}{0pt}}
\setcounter{secnumdepth}{0}
% Redefines (sub)paragraphs to behave more like sections
\ifx\paragraph\undefined\else
\let\oldparagraph\paragraph
\renewcommand{\paragraph}[1]{\oldparagraph{#1}\mbox{}}
\fi
\ifx\subparagraph\undefined\else
\let\oldsubparagraph\subparagraph
\renewcommand{\subparagraph}[1]{\oldsubparagraph{#1}\mbox{}}
\fi

% set default figure placement to htbp
\makeatletter
\def\fps@figure{htbp}
\makeatother


\date{}

\begin{document}

{\ldots{}.49\ldots{}.}

{को वर्गश्चतुरूनः सन्सप्तभक्तो विशुध्यति । }

{त्रिंशदूनोऽथवा कस्तं यदि वेत्सि वद द्रुतं ।। १७५ ।। }

{(1) Find a square which when decreased by 4 will be a multiple of 7.
(2) Find a square from which when 30 is subtracted it will be divisible
by 7 without a remainder.}

{हरभक्ता यस्य कृतिः शुध्यति सोऽपि द्विरूपपदगुणितः । }

{तेनाऽऽहतोऽन्यवर्णो रुपपदेनान्वितः कल्प्यः ।। १७६ ।। }

{न यदि पदं रूपाणां क्षिपेद्धरं तेषु हारतष्टेषु । }

{तावद्यावद्वर्गो भवति न चेदेवमपि खिलं तर्हि ।। १७७ ।। }

{हत्वा क्षिप्त्वा च पदं यत्राऽऽद्यस्येह भवति तत्रापि । }

{आलापित एव हरो रूपाणि तु शोधनानि सिद्धानि ।। १७८ ।। }

{{[}We are discussing square kuttak like x}{3}{ = by + c.{]}}

{If c is a perfect square, we assume y = bz}{2}{ + 2}{√}{c.z and derive
x = bz + }{√}{c .Obviously b2 is divisible by b and 2b}{√}{c is }{also
divisible by b.}

{If c is not a perfect square we should divide c by b and get the
remainder. To this remainder we should add some multiple of b to make it
a perfect square. If this is not possible then the example is improper.}

{Where by multiplying adding etc. we are able to find the square root of
the first side we should take b as given, the remainder should be
decided by the process.}

{षड्भिरूनो घनः कस्य पञ्चभक्तो विशुध्यति । }

{तं वदास्ति तवालं चेद् अभ्यासो घनकुट्टके ।। १७९ ।। }

{x}{3}{ - 6 is divisible by 5 without a remainder. If you know how to
solve a cubic kuttak, find the value of x.}

{यद्वर्गः पञ्चभिः क्षुण्णस्त्रियुक्तः षोडशोद्धृतः । }

{शुद्धिमेति समाचक्ष्व दक्षोऽसि गणिते यदि ।। १८० ।।}{\\
}

\end{document}
