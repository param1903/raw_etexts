\documentclass[]{article}
\usepackage{lmodern}
\usepackage{amssymb,amsmath}
\usepackage{ifxetex,ifluatex}
\usepackage{fixltx2e} % provides \textsubscript
\ifnum 0\ifxetex 1\fi\ifluatex 1\fi=0 % if pdftex
  \usepackage[T1]{fontenc}
  \usepackage[utf8]{inputenc}
\else % if luatex or xelatex
  \ifxetex
    \usepackage{mathspec}
  \else
    \usepackage{fontspec}
  \fi
  \defaultfontfeatures{Ligatures=TeX,Scale=MatchLowercase}
\fi
% use upquote if available, for straight quotes in verbatim environments
\IfFileExists{upquote.sty}{\usepackage{upquote}}{}
% use microtype if available
\IfFileExists{microtype.sty}{%
\usepackage[]{microtype}
\UseMicrotypeSet[protrusion]{basicmath} % disable protrusion for tt fonts
}{}
\PassOptionsToPackage{hyphens}{url} % url is loaded by hyperref
\usepackage[unicode=true]{hyperref}
\hypersetup{
            pdfborder={0 0 0},
            breaklinks=true}
\urlstyle{same}  % don't use monospace font for urls
\IfFileExists{parskip.sty}{%
\usepackage{parskip}
}{% else
\setlength{\parindent}{0pt}
\setlength{\parskip}{6pt plus 2pt minus 1pt}
}
\setlength{\emergencystretch}{3em}  % prevent overfull lines
\providecommand{\tightlist}{%
  \setlength{\itemsep}{0pt}\setlength{\parskip}{0pt}}
\setcounter{secnumdepth}{0}
% Redefines (sub)paragraphs to behave more like sections
\ifx\paragraph\undefined\else
\let\oldparagraph\paragraph
\renewcommand{\paragraph}[1]{\oldparagraph{#1}\mbox{}}
\fi
\ifx\subparagraph\undefined\else
\let\oldsubparagraph\subparagraph
\renewcommand{\subparagraph}[1]{\oldsubparagraph{#1}\mbox{}}
\fi

% set default figure placement to htbp
\makeatletter
\def\fps@figure{htbp}
\makeatother


\date{}

\begin{document}

{\ldots{}.33\ldots{}.}

{where the two strings meet.}

{८ मध्यमाहरणम् । }

{A device to solve a quadratic equations. }

{अव्यक्तवर्गादि यदावशेषं पक्षौ तदेष्टेन निहत्य किंचित् । }

{क्षेप्यं तयोर्येन पदप्रदः स्यादव्यक्तपक्षस्य पदेन भूयः ।। }

{व्यक्तस्य पक्षस्य समक्रियैवमव्यक्तमानं खलु लभ्यते तत् । }

{न निर्वहश्चेदघनवर्गवर्गेष्वेवं तदा ज्ञेयमिदं स्वबुद्ध्या ।। }

{अव्यक्तमूलर्णगरूपतोऽल्पं व्यक्तस्य पक्षस्य पदं यदि स्यात् । }

{ऋणं धनं तच्च विधाय साध्यमव्यक्तमानं द्विविधं क्वचित्तत् ।। ११५ ।। }

{Taking the quadratic and first degree term on one side we are to
multiply both sides by some number and add some number in order to
complete the sqaure. After that square roots are equated and the value
of the unknown is obtained.}

{If the third power and the fourth power of the unknown is present this
device does not work. We have in that case, to adopt some artifice of
our own.}

{In a quadratic equation if the number in the square root of the unknown
side be negative and smaller than the number in the square root of the
other side, then we should assume plus and minus for that and get two
values for the unknown. In some questions both values are admissible.}

{'चतुराहतवर्गसमै रूपैः पक्षद्वयं गुणयेत् । }

{पूर्वाव्यक्तस्य कृतेः समरूपाणि क्षिपेत्तयोरेव' इति ।। ११६ ।। }

{Multiply both sides by four times the coefficient of the square of the
unknown. Add to both sides the square of the coefficient of the
unknown.\\
}

\end{document}
