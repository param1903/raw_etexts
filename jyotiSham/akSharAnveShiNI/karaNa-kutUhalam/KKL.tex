\documentclass[11pt, openany]{book}
\usepackage[text={4.65in,7.45in}, centering, includefoot]{geometry}

\usepackage[table, x11names]{xcolor}
\usepackage{fontspec,realscripts}
\usepackage{polyglossia}

\usepackage{enumerate}
\pagestyle{plain}
\usepackage{fancyhdr}
\pagestyle{fancy}
\renewcommand{\headrulewidth}{0pt}
\usepackage{afterpage}
\usepackage{multirow}
\usepackage{amsmath}
\usepackage{amssymb}
\usepackage{graphicx}
\usepackage{longtable}
\usepackage{footnote}
\usepackage{perpage}
\MakePerPage{footnote}
%\usepackage{bigfoot}
%\DeclareNewFootnote[para]{default}
%\usepackage{dblfnote}
\usepackage{xspace}
%\newcommand\nd{\textsuperscript{nd}\xspace}
\usepackage{array}
\usepackage{emptypage}

\usepackage{hyperref}% Package for hyperlinks
\hypersetup{
colorlinks,
citecolor=black,
filecolor=black,
linkcolor=blue,
urlcolor=black
}

\usepackage[Devanagari, Latin]{ucharclasses}

\setdefaultlanguage{english}
\setotherlanguage{sanskrit}
\setmainfont[Scale=1]{Times New Roman}

\newfontfamily\s[Script=Devanagari, Scale=1.1]{Shobhika}
\newfontfamily\regular{Times New Roman}

\newcommand{\devanagarinumeral}[1]{%
	\devanagaridigits{\number \csname c@#1\endcsname}} % for devanagari page numbers

\setTransitionTo{Devanagari}{\s}
\setTransitionFrom{Devanagari}{\regular}

\XeTeXgenerateactualtext=1 % for searchable pdf

\begin{document}
\thispagestyle{empty}

\begin{center}
\textbf{श्रीगणेशाय नमः}~।
\vspace{3mm}

{\Large\textbf{श्रीमद्भास्कराचार्यप्रणीतं}}
\vspace{5mm}

{\Huge\textbf{करणकुतूहलम्}}
\vspace{5mm}

{\large\textbf{सटीकं प्रारभ्यते}}
\end{center}
\vspace{3mm}

\begin{quote}
{\color{violet}श्च्योतच्छैलशिलोच्छलज्जलकणैराकामबाणाम्बुधौ\\
मग्नस्तन्मुखपङ्कजं कमलवद्रेजे सरोजश्रिया~।\\
तत्राप्यद्भुतधैर्ययुक्सुरवरैर्यः स्तूयते कोटिशो\\
मित्रामित्रसमस्वभावविभवः श्रीपार्श्वराट् सश्रिया~॥}
\end{quote}

{\small \textbf{अथारब्धग्रन्थनिर्विघ्नपरिसमाप्तयेऽभीष्टदेवतानमस्काररूपं ~मङ्गलाचरणमभि-धेयं चाह \textendash }}

\phantomsection \label{1.1}
\begin{quote}
{\large \textbf{{\color{purple}गणेशं गिरं पद्मजन्माच्युतेशान्\\ 
ग्रहान्भास्करो भास्करादींश्च नत्वा~।\\ 
लघुप्रक्रियं प्रस्फुटं खेटकर्म \\
प्रवक्ष्याम्यहं ब्रह्मसिद्धान्ततुल्यम्~॥~१~॥}}}
\end{quote}

इतः षट् भुजङ्गप्रयाताः~। अथ सिद्धान्तपाटीबीजगणितकरणानन्तरम्, उक्तं ~~च ~~"{\color{violet}रसगुणपूर्णमहीसम\textendash \,(१०३६)\,\textendash \,शकनृपसमयेऽभवन्ममोत्पत्तिः~। ~रसगुणवर्षेण मया सिद्धान्तशिरोमणी रचितः~॥}" भास्करनामाहं ग्रन्थकारो गणपतिं सरस्वतीं ब्रह्मविष्णुशिवान् सूर्यादीन् ग्रहांश्च प्रणम्य \hyperref[1.1]{खेटा}नां ग्रहाणां \hyperref[1.1]{कर्म} मध्यादिविधानं कथयिष्यामीति सटंकघटना~। अप्रामाण्यशंकानिरासाय विशेषणं ~~\hyperref[1.1]{ब्रह्मसिद्धान्ततुल्यम्}~। ~ब्रह्मसिद्धान्तेनावगतार्थत्वमाशंक्याह~। लघ्वी प्रक्रिया कर्त-

\afterpage{\fancyhead[RE,LO]{{\small{अ.\,१}}}}
\afterpage{\fancyhead[CE]{{\small{करणकुतूहलम्~।}}}}
\afterpage{\fancyhead[CO]{{\small{गणककुमुदकौमुदीटीकासमेतम्~।}}}}
\afterpage{\fancyhead[LE,RO]{({\small{\thepage}})}}
\cfoot{}
\newpage
\renewcommand{\thepage}{\devanagarinumeral{page}}
\setcounter{page}{2}

\noindent व्यता यस्मिंस्तत् प्रकियालाघवेन प्रस्फुटं सुगममित्यर्थवानेव ( अतः ) अस्या-रम्भः~। ननु सिद्धान्तादपि खेटकर्मैव मुख्यत्वेन साध्यग्रहगणितमिति नाम्ना प्रसिद्धत्वात्तदस्य शास्त्रस्य किं नामोच्यते~। तत्र गणितशास्त्रं त्रिधा सिद्धान्त-तन्त्रकरणत्वेन ~तल्लक्षणम्~। यत्र कल्पादेरारभ्य ~गताब्दमासदिनादेः सौर-सावनचान्द्रमानान्यवगम्य ~सौरसावनगताहर्गणान्मध्यमादीनां ~~कर्मोच्यते तत्सिद्धान्तलक्षणम्~। वर्तमानयुगादेर्वर्षाण्येव ज्ञात्वोच्यते तत्तन्त्रलक्षणम्~। वर्तमानशकमध्येऽभीष्टदिनादारभ्यैवं ~ज्ञात्वोच्यते ~~तत्करणलक्षणम्~। ननु नमस्कारस्य विघ्नविघाते कथं सामर्थ्यम्~। उच्यते~। नमस्कारपुण्येन विघ्नाः प्रहन्यन्त ~इति~। यत्रापि ~च ~नमस्कारमन्तरेणापि ~निर्विघ्नशास्त्रपरिसमाप्ति-र्दृश्यते तत्रापि मानसिकप्रणिधानरूपोऽयं घटत एवेति सफलो नमस्कार-व्यापारः किं च मङ्गलाभिधेयमिति तत्र मङ्गलमभिहितं नमो वचनेनाभिधेयं चात्र \,शास्त्रे \,ग्रहाणां \,मध्यमस्पष्टादिस्वरूपं \,वाच्यं \,प्रयोजनं\, च\, शिष्यानुग्रहः परम् इदम् ऐहिकमुक्तम्~। पारत्रिकन्तु सम्यक् ज्ञानप्रकाशत्वेन उभयोः अपि निःश्रेयसावाप्तिरिति ~~वासना ~~भाष्यतोऽवसेयः~। ~~काचिच्चतुरचित्तचम-त्कारकारिणी ~~युक्तिरन्तरा ~~दरीदृश्यते~। ~अथोदाहरणोपयोगित्वाद्ग्रंथारम्भे कल्पादितो \,गताब्दादि \,लिख्यते \,तत्र \,कल्पगताब्दाः \,१९७२९४८२८४~। अधि-मासाः ७२७६६१६८९~। अवमगणः १४५५२२५४५४~। उदये सावनोऽहर्गणः ७२०६३३६००७४५२~॥~१~॥

\newpage

{\small \textbf{अथ वर्तमानशकादारभ्या{ह}र्गणादिसाधनमाह~।}}

\phantomsection \label{1.2}
\begin{quote}
{\large \textbf{{\color{purple}शकः पञ्चदिक्चन्द्रहीनोऽर्कनिघ्नो\\
मधोर्यातमासान्वितोऽधो द्विनिघ्नात्~।\\
रसाङ्गान्वितात्स्वाभ्रखाङ्कांशहीना-\\
च्छराङ्गैरवाप्ताधिमासैर्युगूर्ध्वः~॥~२~॥}}
\vspace{1mm}

\phantomsection \label{1.3}
\textbf{{\color{purple}खरामाहतो याततिथ्यन्वितोऽध-\\
स्त्रियुक्तात्स्वरामाभ्रशैलांशयुक्तात्~।\\
युगाङ्कैरवाप्तावमोनस्तदूर्ध्वो\\
भवेज्जीववारादिकोऽहर्गणोऽयम्~॥~३~॥}}}
\end{quote}

यस्मिन्नभीष्टाब्दमासदिने \,ग्रहाणां \,मध्यमाद्यं \,साधयितुमिष्यते \,तत्तत्सम-यकः ~\hyperref[1.2]{शकः}~। शकनृपगताब्दपिण्डः ~\hyperref[1.2]{पञ्चदिक्चन्द्रैः} ~पञ्चोत्तरैरेकादशशतै\textendash \,११०५\textendash \,र्हीनः कार्यः यच्छेषं ते ग्रन्थारम्भस्य गताब्दाः सौरा भवन्ति स गता-ब्दगणो\hyperref[1.2]{ऽर्कै}र्द्वादशभि\hyperref[1.2]{र्निघ्नो} गुणितः सौरमासगणो जातः स \hyperref[1.2]{मधो}श्चैत्रादिगत-मासैर्युक्तः \;स \;ए\hyperref[1.2]{वाधो} \;द्वितीयस्थाने धार्यस्ततोऽधःस्थितो \;द्वाभ्यां \;गुणितो \hyperref[1.2]{रसाङ्गान्वितः} षट्षष्टिभि\textendash \,६६\textendash \,र्युतः स्वकीये\hyperref[1.2]{नाभ्रखांकांशे}न नवशतांशेन ९०० हीनः कार्यस्तस्मा\hyperref[1.2]{च्छरांकैः} पञ्चषष्टिभि\textendash \,६५\textendash \,र्भागेऽपहृते यदाप्तं तेऽधिमासाः अनेन ~प्रकारेण ~कदाचित्पतितोऽधिमासो \;न \;लभ्यते ~कदाचिदपतितोऽपि \;लभ्यते \;तत्र \;स्पष्टोऽधिमासोऽधिमासः \;स्फुटः \;स्यादित्यनेन \;तन्निश्चयं \;कृत्वा तथा गताधिमासेषु सैकता निरेकता वा कार्या~। उक्तञ्च {\color{violet}सिद्धान्त-}

\newpage

\noindent {\color{violet}शिरोमणौ\textendash "स्पष्टोऽधिमासः पतितोऽप्यलब्धो यदा यदा वापतितोऽपि लब्धः\,। सैकैर्निरेकैः क्रमशोऽधिमासैस्तदा दिनौघः सुधिया विधेयः~॥}" तैरेवाधिमासै-रुपरिस्थितोऽङ्को \hyperref[1.2]{युक्} युतः कार्यश्चान्द्रमासो जातोऽसौ \hyperref[1.3]{खरामै}स्त्रिंशद्भि\textendash \,३०\textendash \,र्गुणितः सन् शुक्लपक्षमादीकृत्य मासस्य गतदिनैर्युतश्चान्द्रोऽहर्गणो भवति स द्विःस्थाप्यस्ततस्त्रिभि\textendash \,३\textendash \,र्युतः स्वकीयेन \hyperref[1.3]{रामाभ्रशैलांशे}न त्र्युत्तरसप्तशतां-शेन ७०३ युतस्तस्मा\hyperref[1.3]{द्युगाङ्गै}श्चतुःषष्टिभि\textendash \,६४\textendash \,राप्तावमानि अवमेऽप्यधिमा-सवत्सैकता ~निरेकता \;वा \;कार्या~। उक्तञ्च\textendash ~"{\color{violet}अभीष्टवारार्थमहर्गणश्चेत्सैको निरेकस्तिथयोऽपि तद्वत्}" इति तैरवमैरुपरिस्थितश्चान्द्रोऽहर्गणो हीनः कार्यः स च गुरुवारादिकोऽहर्गणो भवति ग्रन्थारम्भादारभ्यैते भूदिना अर्कसावना इति ~एतावन्तः ~सूर्योदया ~जाताः~। उक्तञ्च\textendash ~"{\color{violet}इनोदयद्वयान्तरं ~तदेव ~सावनं दिनम्~। तदेव मेदिनीदिनं भवासरस्तु {भ}भ्रम्}~॥" इत्यहर्गणसिद्धिः~। अथोदा-हरणम्~। संवत् १६७६ चैत्रादिवर्षे शाके १५४१ चन्द्रे ज्येष्ठकृष्ण\textendash \,१४\textendash \,रवौ घट्यादिः ५४।२० अश्विनीनक्षत्रं घट्यादिः २७।२६ सौभाग्यो योगो घट्यादिः ४४।१२ अत्र दिने गतघटी १।० समये ग्रहाणां साधनं तत्र प्रथमाहर्गणार्थे यथा शकः १५४१ पञ्चदिक्चन्द्रै\textendash \,११०५\textendash \,र्हीनः गताब्दपिण्डेऽ\textendash \,४३६\textendash \,यमर्कै\textendash \,१२\textendash \,र्गुणितः ५२३२ चैत्रतो गतचान्द्रमासेन १ ~युतो\textendash {५२३३}\textendash {ऽधो}\textendash {५२३३}\textendash {ऽस्माद्द्वि}-गुणितात् \;१०४६६ \;रसाङ्गान्वितात् \;१०५३२ \;स्वशब्देनाधःस्थिता\textendash \;१०५३२\textendash \,दभ्रखाङ्कै\textendash \,९००\textendash \,र्भक्तो

\newpage

\noindent लब्धेनो\textendash \,११\textendash \,परिस्थाङ्कः ~१०५३२ ऊनः १०५२१ शराङ्गैः ६५ भक्ताल्लब्धाधि-मासै\textendash {१६१}\textendash {रुपरिष्ठो} ५२३३ युतश्चान्द्रमासगणो जातो\textendash \,५३९४\textendash \,ऽयं खरामै\textendash \,३०\textendash \,र्गुणितः ~~१६१८२० ~~शुक्लप्रतिपदादिगणनया ~~गततिथिभि\textendash \,२८\textendash \,र्युत\textendash \,१६१८४८\textendash{श्चान्द्रगणोऽयमध}\textendash{१६१८४८}\textendash{स्त्रिभि}\textendash{३}\textendash{र्युतः} १६१८५१ स्वशब्दे-नाधो १६१८५१ रामाभ्रशैले\textendash \,७०३\textendash \,र्भक्तो लब्धेनो\textendash \,२३०\textendash \,परिष्ठो १६१८५१ युतो ~१६२०८१ ~युगाङ्गै\textendash \,६४\textendash \,र्भक्तो ~लब्धावमै\textendash \,२५३२\textendash \,रुपरिष्टो ~१६१८४८ हीनो जातोऽहर्गणः १५९३१६ सप्तभक्ते शेषं ३ बृहस्पतितो गणनाकृते शनिर्गत उदये रविः~॥~३~॥\\

{\small \textbf{अथ ग्रन्थारम्भे चैत्रादिशुक्लप्रतिपदि सूर्योदयिका मध्यमास्तेषां ग्रहाणां ध्रुवका अब्दबीजयुताः क्षेपकत्वेन कृतास्तानाह\textendash }}

\phantomsection \label{1.4}
\begin{quote}
{\large \textbf{{\color{purple}दिशो गोयमा विश्वतुल्याः खमर्के,\\
विधौ खेन्दवोऽङ्काश्विनः पञ्चखाक्षाः~।\\
विधूच्चेऽब्धयोऽक्षेन्दवोऽर्कानवाक्षा,\\
नवात्यष्टितत्वा ग्रहाश्चन्द्रपाते~॥~४~॥}}
\vspace{1mm}

\phantomsection \label{1.5}
\textbf{{\color{purple}कुजेऽश्वाः कुदस्रा जिनाः क्वक्षितुल्या\\
बुधे द्वौ कुनेत्राणि शक्राः खरामाः~।\\
गुरौ क्षेपको द्वौ कृताः खङ्कुबाणाः \\
सितेऽष्टौ धृतिर्मार्गणाः पञ्चबाणाः~॥~५~॥}}
\vspace{1mm}

\phantomsection \label{1.6}
\textbf{{\color{purple}युगान्यग्नयस्त्र्यब्धयः शैलचन्द्राः \\
शनौ चेति राश्यादिना क्षेपकेण~।\\
द्युपिण्डोत्थखेटो युतः स्वेन मध्यो \\
भवेदुद्गमेऽर्कस्य लङ्कानगर्य्याम्~॥~६~॥}}}
\end{quote}

\newpage

\begin{center}
{\large \textbf{सूर्यादीनां राश्यादिक्षेपकाः~।}}
\end{center}

\begin{small}
\begin{tabular}{ccccccccc}
\hspace{-4mm} सूर्यः & चन्द्रः & विधूच्चम् & चन्द्रपातः & भौमः & बुधः & गुरुः & शुक्रः & शनिः \\
\hspace{-4mm} १० & १० & ४ & ९ & ७ & २ & २ & ८ & ४ \\
\hspace{-4mm} २९ & २९ & १५ & १७ & २१ & २१ & ४ & १८ & ३ \\
\hspace{-4mm} १३ & ५ & १२ & २५ & २४ & १४ & ० & ५ & ४३ \\
\hspace{-4mm} ०० & ५० & ५९ & ९ & २१ & ३० & ५१ & ५५ & १७\\
\end{tabular}
\end{small}
\vspace{4mm}

\hyperref[1.6]{द्युपिण्डोत्थखेट} \;इति \;वक्ष्यमाणप्रकारेणाहर्गणादुत्पन्नो \;ग्रहो \;राश्यादिः~। राश्यादिना ~स्वक्षेपकेण ~युतो ~मध्यमसूर्योदयकालिकक्षितिजासन्नलंकादे-शीयो ~मध्यमो ~ग्रहः ~स्यादित्यर्थः~। उक्तञ्च~। "{\color{violet}दशशिरःपुरि ~मध्यमभास्करे क्षितिजसन्निधिगे सति मध्यमः}" इति~॥~६~॥\\

अथ मध्यमग्रहानयनम्~। तत्र तावन्मध्यमत्वं किमुच्यते ग्रहस्य क्षेत्रात्मक-नियतपूर्वगत्या द्वादशराशिभोगो भगणसंज्ञा इत्युच्यते~। एवं कल्पे यावत् कृत्या द्वादशराशिभोगास्तावन्तस्तपनस्य भगणाः सम्भवन्ति~। तत्र वर्तमान-भगणस्य \,यावान् \,भागो \,राश्याद्यो \,भुक्तः \,स \,मध्यमो \,ग्रहसंज्ञ \,इत्युच्यते \,अथ सूर्यानयनमाह\textendash \,इतः श्लोकत्रयमुपजातिकम्~।

\phantomsection \label{1.7}
\begin{quote}
{\large \textbf{{\color{purple}अहर्गणो विश्वगुणस्त्रिखाङ्कै\textendash \,९०३\textendash \\
र्भक्तः फलोनो द्युगणो लवाद्याः~।\\
रविज्ञशुक्राः स्युरथाब्दवृन्दा-\\
द्वेदाङ्ग\textendash \,६४\textendash \,लब्धेन कलादिनोनाः~॥~७~॥}}}
\end{quote}

अहर्गणो द्विः स्थाप्यस्तत्रैकस्थो \hyperref[1.7]{विश्वै}स्त्रैयोदशभि\textendash \,१३\textendash \,र्गुणित\hyperref[1.7]{स्त्रिखाङ्कै}\textendash \,९०३\textendash \,स्त्र्युत्तरनवशतैर्भक्त आप्ते-

\newpage

\noindent नांशादिनोपरिस्थितोऽहर्गणो ~~हीनोंऽशादयो ~~रविबुधशुक्रा ~~भवन्ति ~~अत्र भाज्यभाजकयोर्ग्रहानयनेऽपवर्त्यमध्ये बीजान्यन्तर्भूतान्युक्तानि तानि सान्त-रितानि ~~तन्निरासार्थङ्करणगताब्दपिण्डा\hyperref[1.7]{द्वेदाङ्गै}\textendash \,६४\textendash \,श्चतुःषष्टिभिर्भागे ~~हृते यदाप्तं कलादिकं तेन हीनाः कार्याः~। यथाहर्गणः १५९३१६ अयं द्वितीयस्थाने स्थितः १५९३१६ एकत्रस्थो विश्वै\textendash \,१३\textendash \,र्गुणित\textendash \,२०७११०८\textendash \,स्त्रिखाङ्कै\textendash {९०३}-र्भक्तो \;लब्धमंशाः \;२२९३ \;शेषं \;५२९ \;षष्टिगुणः \;३१७४० \;भाजकेन \,९०३ \,भक्ते लब्धाः कलाः ३५ शेषं १३५ षष्टिगुणं ८१०० भाजकेन भक्ते लब्धाः कलाः ८ कलाद्यानयनमेव परिपाटी सर्वत्र ज्ञेया~। अंशाद्येना\textendash \,२२९३।३५।८\textendash \,हर्गणो १५९३१६ हीनः १५७०२३ अत एकमंशं गृहीत्वा षष्टिकलाः कृृत्वा ३५ पातिते शेषं २५ अस्यैकं गृहीत्वा षष्टिविकलाः कृृत्वा विकलाः ८ शुद्धे शेषम् ५२ एव-मंशादिः १५७०२२।२४।५२~॥\\

अथाब्दबीजसंस्कारः\textendash ~गताब्दा \,४३६ \,वेदाङ्गै\textendash \,६४\textendash \,र्भक्ता \,लब्धेन \,कला-दिना ६।४८ पूर्वागतकलासु हीनाः १५७०२२।१८।४ त्रिंशद्भक्तं लब्धं ५२३४ शेषमंशाः २ राशीनां ५२३४ द्वादशभक्ते लब्धं भगणः ४३६ शेषं द्वौ राशी २ एवं भगणादिः भगणः ४३६ राश्यादिः २।२।१८।४ स्वक्षेपेण राश्यादिना १०।२९। १३।० \;युतः \;१२।३१।३१।४ \;भागानां \;त्रिंशद्भक्ते \;लब्धेनो\textendash \,१\textendash \,परिराशिस्थाने युतः १३ द्वादशभक्तः

\newpage

\noindent लब्धेन भगणस्थाने युतः\textendash \,४३७ एवं राश्यादिः १।१।३१।४ लङ्कायां सूर्योदये मध्यमो रविरयमेव बुधः शुक्रश्च ज्ञेयः एषा परिपाटी सर्वत्र ग्रहानयने ज्ञेया~। अथ गणितोपयुक्तमङ्कशोधनं यथोक्तं {\color{violet}बीजदत्तैः\textendash \,"गुण्ये गुणे नवहृते परिशेष-घाते नन्दै\textendash \,९\textendash \,र्हृते भवति यः परिशेषराशिः~। घातेन गुण्यगुणयोर्नवशेषितेन साम्येन तस्य निगदेद्गणितस्य शुद्धिम्"~॥~१~॥} यथा गुण्योऽहर्गणः १५९३१६ नवभक्ते शेषं ७ गुणकः १३ नवभक्ते शेषम् ४ उभयोः शेषयोः ७।४ हतिः नव-हृतः शेषम् १ अथ गुणिताङ्कः २०७११०८ नवभक्ते शेषम् १ उभयोः साम्ये गुणि-तोऽङ्कः २०७११०८ शुद्ध एवं सर्वत्र अथ भागहारशोधनं {\color{violet}पाटीगणिते} प्रोक्तं\textendash \,{\color{violet}"हाराप्त्या नवशेषस्तद्यत्तेनाढ्यशेषनवशेषः~। भाज्याङ्को नवशेषस्तुल्यः स्यात् तदा \;शुद्धः"} \;इति~। भाज्ये \;नवभिर्हृते \;यच्छिष्टं \;तदेवाप्तिः \;भाजकयोर्नवभिः शेषितयोर्मिथो वधेन युक्तस्य शेषस्य नवभागे शेषे तत्तुल्यं स्यात्तदावाप्तञ्च शेषशुद्धम्~। यथा भाज्ये २०७११०८ नवभक्ते शेषम् १ भाजकेऽपि ९०३ नव-भक्ते शेषम् ३ लब्धमपि २२९३ नवभक्ते शेषम् ७ भाजकशेषयोर्घार्ते २१ शेषम् ५२९ युतम् ५५० नवभक्तं शेषं, पूर्वशेषेण १ तुल्यं ततो लब्धम् २२९३ शेषञ्च ५२९ शुद्धमेव सर्वत्रापि प्रायशः सूर्यभगणानां गताब्दानां च साम्यं कदाचिद-नन्तरम्~॥~७~॥

\newpage

{\small \textbf{अथ मध्यमचन्द्रानयनमाह\textendash }}

\phantomsection \label{1.8}
\begin{quote}
{\large \textbf{{\color{purple}अह्नां गणः शक्रगुणो विहीनः \\
स्वात्यष्टिभागेन लवादिरिन्दुः~।\\ 
अहर्गणात्खाभ्ररसाष्ट\textendash \,८६००\textendash \,भक्ता-\\
दाप्तेन भागादिफलेन हीनः~॥~८~॥}}}
\end{quote}

अह्नामिति~। अस्योदाहरणार्थः\textendash \,यथाहर्गणः १५९३१६ \hyperref[1.8]{शक्रै}श्चतुर्दशभिर्गु-णितः २२३०४२४ \hyperref[1.8]{स्वात्यष्टि}\textendash {१७}\textendash {भागेन} शक्रगुणोऽहर्गणः, हीनः, \hyperref[1.8]{अत्यष्टिभिः} १७ सप्तदशभिर्भक्तो लब्धेन पूर्ववदंशादिना १३१२०१।२४।४२ शक्रगुणोऽह-र्गणः \;२२३०४२४ \;हीनः \;२०९९२२२।३५।१८ \;अथ \;संस्कारो \;यथाहर्गणात् १५९३१६ \hyperref[1.8]{खाभ्ररसाष्ट}षडशीतिशतैः ८६०० भक्तादाप्तेनांशादिना १८।३१।३० पूर्वागतो २०९९२२२ हीनः २०९९२०४।३।४८ पूर्ववद्भगणादिः ५८३१।१।१४। ३।४८ स्वक्षेपेण १०।२९।५।५० युतश्चन्द्रो मध्यमः स्यात्~। भगणः, ५८३२~। राश्यादिः ०।१३।९।३८~॥~८~॥\\

{\small \textbf{अथोच्चानयनमाह\textendash }}

\phantomsection \label{1.9.1}
\begin{quote}
{\large \textbf{{\color{purple}गणो द्विधा गोभि\textendash \,९\textendash \,रिनाभ्रवेदै\textendash \,४०१२\textendash \\
र्लब्धैक्यमंशादि भवेद्विधूच्चम्~।}}}
\end{quote}

\hyperref[1.9.1]{गणो}ऽहर्गणः \;१५९३१६ \;\hyperref[1.9.1]{द्विधा} \;स्थानद्वये \;स्थाप्य \,एकत्र \,\hyperref[1.9.1]{गोभि}\textendash \,९\textendash \,र्भक्तो लब्धमंशादिः १७७०१।४६।४० अपरत्र गण १५९३१६ \hyperref[1.9.1]{इनाभ्रवेदै}र्द्वाशाधि-

\newpage

\noindent कचतुःसहस्रैः ४०१२ भक्ताल्लब्धांशादिना ३९।४२।३५ युतम् १७७४१।२९।१५ पूर्ववद्भगणादिः ४९।३।११।२९।१५ स्वक्षेपेण ४।१५।१२।५९ युतं जातं चन्द्रो-च्चम् ४९।७।२६।४२।१४~॥\\

{\small \textbf{अथ पातानयनमाह\textendash }}

\phantomsection \label{1.9}
\begin{quote}
{\large \textbf{{\color{purple}द्विधांकचन्द्रैः १९ खखभै\textendash \,२७००\textendash \,र्दिनौघा-\\
दाप्तांशयोगो भवतीन्दुपातः~॥~९~॥}}}
\end{quote}

द्विःस्थितादहर्गणा-१५९३१६-देक\hyperref[1.9]{त्रांकचन्द्रै}रेकोनविंशतिभिर्लब्धमंशादिः ८३८५।३।९ \;अपरत्र \;\hyperref[1.9]{खखभैः} \;सप्तविंशतिशतै\textendash \,२७००\textendash \,र्भक्तो \;लब्धमंशादिः ५९।०।२१ अनयोरंशादिफलयोर्योगः ८४४४।३।३० पूर्ववद्भगणादिः २३।५। १४।३।३० \;स्वक्षेपेण \;९।१७।२५।९ \;युतो \,जातः \;मध्यमः \;पातः \,२४।३।१।२८। ३९~॥~९~॥\\

{\small \textbf{अथ भौमबुधशीघ्रोच्चानयनं शार्दूलविक्रीडितेनाह\textendash }}

\phantomsection \label{1.10}
\begin{quote}
{\large \textbf{{\color{purple}रुद्रघ्नो ११ द्युचयो द्विधा शशियमै\textendash \,२१\textendash \,र्वेदाब्धिसिद्धेषुभि\textendash 
५२४४४\textendash \,र्भक्तोंऽशादिफलद्वयं तु सहितं स्यान्मेदिनीनन्दनः~।\\
वेद\textendash \,४\textendash \,घ्नो द्युचयः स्वकीयदहनाब्ध्यंशेन ४३ युक्तो भवे-\\
द्भागादिक्ष्वचलं गणात्क्षितियमेन्द्रा\textendash \,१४२१\textendash \,प्तांशकै-\\
{\color{white}अ} \hfill र्वर्जितम्~॥~१०~॥}}}
\end{quote}

द्युगणोऽहर्गणः १५९३१६ \hyperref[1.10]{रुद्रै}रेकादशभि\textendash \,११\textendash \,र्गुणितः १७५२४७६ एकत्र \hyperref[1.10]{शशियमै}रेकविंशतिभिर्भक्तोंऽशादिः ८३

\newpage

\noindent ४५१।१४।१७ ~अपरत्र ~\hyperref[1.10]{वेदाब्धिसिद्धेषुभिः} चतुश्चत्वारिंशच्चतुःशताधिकद्वि-पञ्चाशत्सहस्रैः \;५२४४४ \;भक्तोंऽशादिः \;३३।२४।५८ \;अनयोः \;फलयोः \;योगः ८३४८४।३९।१५ \;पूर्ववद्भगणादिः \;२३१।१०।२४।३९।१५ \;स्वक्षेपेण \;७।२१। २४।२१ युतो जातो \hyperref[1.10]{मेदिनीनन्दनो} भौमः २३२।६।१६।३।३६। \hyperref[1.10]{द्युचयो}ऽहर्गणो १५९३१६ \;\hyperref[1.10]{वेदै}श्चतुर्भि\textendash \,४\textendash \,र्गुणितः \,६३७२६४ \,स्वकीयेन \,\hyperref[1.10]{दहनाब्ध्यंशेन} \,त्रिच-त्वारिंशांशेन ४३ \;वेदघ्नोऽहर्गणोऽधःस्थाप्यः \;६३७२६४ \,त्रिचत्वारिंशता \;४३ \;भक्तो \;लब्धमंशादि\textendash \,१४८२०।५।३४\textendash \,रुपरिष्ठोऽङ्को \;६३७२६४ \;युतो \;जातम् अंशादिबुधशीघ्रोच्चम् ६५२०८४।५।३४ एतदहर्गणात् १५९३१६ \hyperref[1.10]{क्षितियमेन्द्रा-प्तांशकै}रेकविंशत्युत्तरचतुर्दशशतैः \;१४२१ \;भक्ताल्लब्धमंशादिना \;११२।६।५५ हीनम् ६५१९७१। ५८।३९ जातो भगणादिरयम् १८११।०।११।५८।३९ स्वक्षे-पेण २।२१।१४।३० युतं जातं बुधोच्चम् १८११।३।३।१३~॥~१०~॥\\

{\small \textbf{अथ गुर्व्वानयनमुपजातिकपूर्वार्द्धेनाह\textendash }}

\phantomsection \label{1.11.1}
\begin{quote}
{\large \textbf{{\color{purple}गणो द्विधार्कैर्भयमाब्धिभिश्च \\
भक्तः फलांशान्तरमिन्द्रमन्त्री~। }}}
\end{quote}

अहर्गणो \;१५९३१६ \;द्विधैक\hyperref[1.11.1]{त्रार्कैः} \;द्वादशभि\textendash {१२}\textendash {र्भक्तो}\textendash \,१३२७६।२०। ०।\textendash \,ऽन्यत्र \;\hyperref[1.11.1]{भयमाब्धिभिः} \;सप्तविंशत्युत्तरद्विचत्वारिंशच्छतैः \;४२२७ \;भक्तः \;३७।४१।२४ अनयोः फलांशयोरन्तरम् १३२३८।३८।३६ भगणादिः ३६।९।८। ३८।३६

\newpage

\noindent स्वक्षेपेण २।४।०।५१ युतो जातो ३६।११।१२।३९।२७ मध्यम् \hyperref[1.11.1]{इन्द्रमन्त्री} गुरुः\,॥\\

{\small \textbf{अथ शुक्रशीघ्रोच्चानयनमुपजात्युत्तरार्द्धेणेन्द्रवज्रापूर्वार्धेनाह\textendash }}

\phantomsection \label{1.11}
\begin{quote}
{\large \textbf{{\color{purple}नृपाहतोऽह्नां निचयो द्विधासौ \\
भूबाणवेदाद्रिभि\textendash \,७४५१\textendash \,रभ्रचन्द्रैः~॥~११~॥\\
भक्तो लवाद्यं फलयोर्यदैक्यं \\
तज्जायते दैत्यगुरोश्चलोच्चम्~।}}}
\end{quote}

\hyperref[1.11]{अह्नां} दिनानां \hyperref[1.11]{निचयो} गणो \hyperref[1.11]{नृपैः} षोडशभिः १६ \hyperref[1.11]{आहतो} गुणितोऽयं पुन\hyperref[1.11]{र्भू-बाणवेदाद्रिभि}रेकपञ्चाशदुत्तरचतुःसप्ततिशतैः ७४५१ भक्तस्तथा चाभ्रचन्द्रैः १० दशभिश्चेति \hyperref[1.11]{द्विधा} प्रकारद्वयेन भक्त एवं लब्धस्य फलद्वयस्य य\hyperref[1.11]{दैक्यं} योग-स्त\hyperref[1.11]{द्दैत्यगुरोः} शुक्रस्यांशादिकं चलोच्चं जायते~। यथाहर्गणः १५९३१६ नृपैः १६ हतः २५४९०५६ एकत्र भूबाणवेदाद्रिभिः ७४५१ भक्तः ३४२।६।३३ अपरत्र दशभक्तोंऽशादिः २५४९०५।३६।० उभयोरैक्यम् २५५२४७।४२।३३ पूर्ववद्भ-गणादिः ७०९।०।७।४२।३३ स्वक्षेपेण ८।१८।५।५५ युतो जातं शुक्रशीघ्रो-च्चम् ७०९।८।२५।४८।२८~॥~११~॥\\

{\small \textbf{अथेन्द्रवज्रोत्तरार्द्धेन शनिमाह\textendash }}

\phantomsection \label{1.12}
\begin{quote}
{\large \textbf{{\color{purple}भक्तोऽभ्ररामै\textendash \,३०\textendash \,स्तुरगाङ्गराम-\\
नन्दै\textendash \,९३६७\textendash \,र्द्विधांशादिफलैक्यमार्किः~॥~१२~॥}}}
\end{quote}

यथाहर्गणः १५९३१६ द्विधैक\hyperref[1.12]{त्राभ्ररामै}स्त्रिंशद्भिः ३० भक्तो लब्धम् ५३१०। ३२।० अपरत्र \hyperref[1.12]{तुरगाङ्गरामनन्दैः}

\newpage

\noindent सप्तषष्ट्युत्तरत्रिनवतिशतैः ९३६७ भक्तो लब्धम् १७।०।२९ उभयोरैक्यमंशादिः ५३२७।३२।२९ प्राग्वद्भगणादिः १४।९।३२।२९ स्वक्षेपेण ४।३।४३।१७ युतो जातो १५।१।२१।१५।४६ मध्यमार्किः शनिर्मध्यमः~॥~१२~॥\\

{\small \textbf{अथ सूर्यादीनां मध्यगतिं शार्दूलविक्रीडितेनाह\textendash }}

\phantomsection \label{1.13}
\begin{quote}
{\large \textbf{{\color{purple}नन्दाक्षा भुजगा रवेः शशिगतिः खाङ्काद्रयोऽक्षाग्नय-\\
स्तुङ्गस्याङ्गकलाः कुवेदविकलाः पातस्य रामा भुवाः~। \\
माहेयस्य महीगुणा रसयमाक्षस्येपुसिद्धा रदाः \\
पञ्चेज्यस्य सितस्य षण्णवमिताश्चाष्टौ, शनेर्द्वे कले~॥~१३~॥}}}
\end{quote}

\hyperref[1.13]{रवेः} \,सूर्यस्य \,\hyperref[1.13]{नन्दाक्षा} \,एकोनषष्टिः \,कलाः \,\hyperref[1.13]{भुजगा} \,अष्टौ \,विकलाः \,५९।८ गतिः~। \hyperref[1.13]{शशि}नश्चन्द्रस्य गतिः \hyperref[1.13]{खाङ्काद्रयो} नवत्युत्तरसप्तशतकलाः \hyperref[1.13]{अक्षाग्नयः} पञ्चत्रिंशद्विकलाः ७९०।३५ \hyperref[1.13]{अङ्गाः} षट् कलाः \hyperref[1.13]{कुवेदा} एकचत्वारिंशद्विकलाः ६।४१ चन्द्रोच्चस्य~। \hyperref[1.13]{रामा}स्त्रयो \hyperref[1.13]{भवा} एकादश चन्द्रपातस्य गतिः ३।११~। \hyperref[1.13]{मही-गुणा} एकत्रिंशत्कला \hyperref[1.13]{रसयमाः} षड्विंशतिर्विकला भौमस्य ३१।६~। \hyperref[1.13]{इषुसिद्धाः} पञ्चचत्वारिंशदधिकद्विशतीकला \hyperref[1.13]{रदाः} द्वात्रिंशद्विकला बुधशीघ्रस्य २४५। ३२~। \;पञ्चकला \;\hyperref[1.13]{ईज्यस्य} \;गुरोः \;५।०~। \;षण्णवतिकला \;अष्टौ \;विकलाः \;शुक्र-शीघ्रोच्चस्य ९६।८~। कलाद्वयं शनेः २।०~। अनया भुक्त्या युतोऽग्रिमदिनस्य मध्यमो भवति~। इयं भुक्तिरल्पोत्तरत्वात्सुखार्थं प्रति विकलां

\newpage

\noindent विहायाचार्येणोक्ता विशेषश्चात्रैकमहर्गणं प्रकल्प्याहर्गणो विश्वगुण इत्यादिना स्वस्वकरणविधिना कृते स्वस्वगतयो भवन्ति~। उक्तञ्च\textendash {\color{violet}"महीमिता\textendash {१}\textendash {दह}-र्गणात्फलानि यानि तत्कलाः~। भवन्ति मध्यमाः क्रमान्नभः सदां द्युभुक्तयः"} इति \;परमियं \;गतिः \;कक्षाया \;लघुमहत्त्वेन \;कलादिकाग्रहणाद्भिन्ना \;भवति~। योजनात्मिका तु दिनगतिः सर्वदा सर्वेषाम् ११८५८~। ४५ समानैव ज्ञेया~। कल्पे \;१८७२०६९२०००००००० \;एतावन्ति \,योजनानि \,सर्वे \,समाना \,भ्रमन्ती-त्यूह्यं \;चद्रोच्चं \;विनान्येषां \;मन्दोच्चानां \;गतयो \;लिख्यन्ते \;ग्रन्थान्तरात्~। \;वर्षैः सप्ततिभिर्विकलैका रवेर्मन्दोच्चस्य गतिः~। द्वादशभिर्वर्षर्विकलैका भौमस्य~। बुधस्य वर्षैर्द्वादशभिः~। बृहस्पतेश्चतुर्भिः~। शुक्रस्य पञ्चभिः~। शनेरेकादशभि-र्वर्षैरेका \;विकला~। पुनरुक्तं \;संवत्सरायुतैः \;१०००० \;तेषां \;गतयः \,स्युः \,कला-दिकाः \;प्रायशस्त्रयोदशभिर्वर्षैरेका \;विकला \;भौमपातस्य \;गतिः~। \,साधिकैः षड्भिर्वर्षैरेका \,विकला \,बुधपातस्य \,गतिः~। \,किञ्चिन्न्यूनैश्चतुःपञ्चषड्भिर्वर्षैरेका विकला गुरुपातस्य~। किश्चिन्न्यूनैश्चतुर्भिर्वर्षैरेका विकला भृगुपातस्य~। किञ्चि-न्न्यूनैः षड्भिर्वर्षैरेका विकला शनिपातस्य~॥~१३~॥\\

{\small \textbf{अथ देशान्तरोपयायिनीं भूमध्यरेखां भुजङ्गप्रयातेनाह \textendash }}

\phantomsection \label{1.14}
\begin{quote}
{\large \textbf{{\color{purple}पुरी रक्षसां देवकन्याथ कान्ती \\
सितः पर्वतः पर्य्यलीवत्सगुल्मम्~। \\
पुरी चोज्जयिन्याह्वया गर्गराटं \\
कुरुक्षेत्रमेरू भुवो मध्यरेखा~॥~१४~॥}}}
\end{quote}

\newpage

लङ्कापुर्य्यां सूत्रस्यैकमग्रं बद्ध्वान्यदग्रं मेरुशिरसि धार्यमियं मध्यरेखातः सूत्राधो यानि नगराणि तानि मध्यरेखानगराणीत्यर्थः~॥~१४~॥\\

{\small \textbf{अथ ग्रहाणां स्वदेशीयकरणार्थं देशान्तरकर्मोपजात्याह \textendash }}

\phantomsection \label{1.15}
\begin{quote}
{\large \textbf{{\color{purple}रेखा स्वदेशान्तरयोजनघ्नी \\
गतिर्ग्रहस्याभ्रगजैर्विभक्ता~। \\
लब्धा विलिप्ता खचरे विधेया \\
प्राच्यामृणं पश्चिमतो धनं ताः~॥~१५~॥}}}
\end{quote}

प्राचीप्रतीचीसूत्रं स्वदेशस्थं भूमध्यरेखान्तर्गतं यत्स्थानं तस्मान्मध्यरेखा-स्थानात्स्वदेशस्यान्तरे यावन्ति योजनानि तैर्ग्रहस्य भुक्तिर्गुणि\hyperref[1.15]{ताभ्रगजै}रशी-तिभिः ८० भक्ता लब्धा विकला मध्यरेखातः पूर्वदेशे ग्रहे हीनः पश्चिमे धनं विधेयं यथा पारंपर्यत्वाद्गर्गराट् मध्यरेखावशात्पश्चिमदेशे ३० योजने शिवपुरी अतो योजनैः ३० सूर्यमध्यगतिः ५९।८ गुणिता १७७४ अभ्रगजैः ८० विभक्ता लब्धं विकला २२ रवेर्देशान्तरं पश्चिमत्वाद्धनमेवं चन्द्रादीनामपि मध्यगत्या कृत्वा पत्रे लिखितम्~। {\color{violet}रेखा\textendash {"}पलश्रुतिघ्ना रविभाजिता च विलिप्तिकाः प्राच्य-परेऽस्तमाद्यम्"~।} पाठोऽसङ्गतो यथा मध्यदेशे सूर्यः १।१।३१।४ विकलाः २२ धनं देशान्तरशुद्धः १।१।३१।२६ एवं सर्वे ज्ञेयाः~॥~१५~॥\\

{\small \textbf{अथ \,ग्रहानयने \,कृतापवर्तशेषन्यायेनान्तरविनाशार्थमिन्द्रवंशस्थाभ्यां \,कृत्वोप-जात्या बीजकर्माह \textendash }}

\newpage

\phantomsection \label{1.16}
\begin{quote}
{\large \textbf{{\color{purple}अब्दा गजाश्वै\textendash \,७८\textendash \,स्त्रिरसै\textendash \,३६\textendash \,र्विभाजिता \\
ऋणं विलिप्तासु शशीज्ययोः क्रमात्~। \\
विश्वैः १३ खरामै\textendash \,३०\textendash \,र्द्वियमै\textendash \,२२\textendash \,श्च खेचरैः ९ \\
पातोच्चसौम्यास्फुजितां धनं तथा~॥~१६~॥}}}
\end{quote}

\begin{center}
{\large \textbf{इतीह भास्करोदिते ग्रहागमे कुतूहले \\
विदग्धबुद्धिवल्लभे नभोगमध्यसाधनम्~॥~१~॥ }}
\end{center}

\hyperref[1.16]{अब्दाः} \;करणगताब्दाः \;\hyperref[1.16]{गजाश्वै}रष्टसप्ततिभिः \;७८ \;भक्ता \;लब्धं \;\hyperref[1.16]{शशिन}-श्चन्द्रस्य विकलास्वर्णं स्यात्~। एवं \hyperref[1.16]{त्रिरसै}स्त्रिषष्टिभि\textendash \,६३\textendash \,र्लब्धं गुरोर्विकला-स्वर्णम्~। अथ \hyperref[1.16]{विश्वै}स्त्रयोदशभिः १३ \hyperref[1.16]{खरामैः} त्रिंशद्भिः ३० \hyperref[1.16]{द्वियमै}र्द्वाविंशद्भिः २२ \;\hyperref[1.16]{खेचरै}र्नवभिः \;९ \;लब्धं \;क्रमेण \;पातचन्द्रोच्चबुधशुक्राणां \;विकलासु \;धनं भवेत्~। करणगताब्दाः ४३६ गजाश्वैः ७८ भक्ता लब्धं विकलाश्चन्द्रस्यर्णम् ५ पुनरब्दाः ४३६ त्रिरसैः ६३ भक्ता लब्धं विकलाः ६ गुरोर्ऋणम्~। अब्दाः ४३६ विश्वैर्लब्धं विकलाः ३३ पातस्य धनम्~। अब्दाः ४३६ त्रिंशद्भक्तं लब्धं विकलाः १४ चन्द्रोच्चस्य धनम् अब्दा ४३६ द्वियमैः २२ भक्ता लब्धं विकलाः १९ बुधशी-घ्रोच्चस्य धनम्~। अब्दाः ४३६ खेचरैः ९ भक्ता लब्धं विकलाः ४८ शुक्रशीघ्रस्य धनम्~। रविभौमशनीनां नास्तीदङ्कर्म्म~। लोकैरब्दबीजत्वेन व्यवह्रियते~। षट् कर्मणां नामान्युच्यन्ते देशान्तरम् १ अब्दबीजम् २ रामबीजम् ३ भांशफलम् ४ उदयान्तरम् ५ चरकर्म ६ तत्र देशान्तरमुक्तमब्दबीजं तु ग्रन्थकृता क्षेपेष्वेव दत्तम्~। अथ ग्रन्थारम्भतो यावत्प्रमाणं बीजं तत्पत्रे लिखितं परं स्वल्पान्तर-त्वादुपेक्षितं राम-

\newpage

\noindent बीजम् {\color{violet}आधुनिकगणकै}रुक्तं तल्लिख्यते~। {\color{violet}"कलाद्वयं धनं सूर्ये चन्द्रे तिथिकला ऋणम्~। भौमे स्वं कलिकाशीतिर्बुधे सप्तशती धनम्~॥ गुरावृणं खनन्दैका तथा शुक्रे खभानि च~। शनौ धनं खनन्दाश्च त्रिंशत्स्वर्णोच्चपातयोः~॥ एवं कृतेऽधुना \,खेटा \,जायन्ते \,च \,तदा \,ध्रुवम्~। नलिकायन्त्रयोग्याश्च \;ग्रहणादिषु सर्वदा"~॥} ~एतान्यपि ~सान्तराणि ~ज्ञात्वा ~{\color{violet}रामचन्द्राचार्यैः} ~कृतास्तान्यपि लिख्यन्ते\textendash \,{\color{violet}"कलाद्वयं चाथ रवौ धनं स्यादृणं च चन्द्रे कृतराम\textendash \,३४\textendash \,लिप्ताः~। भौमेऽभ्रविश्वप्रमिताः १३० कलाः स्वं बुधेऽस्य शीघ्रे धनमभ्रषट् च~॥~१~॥ गुरावृणं खाङ्कशशिप्रमाणाः १९० सितस्य शीघ्रे त्रिशती ऋणं च ३००~। मन्दे च खाष्टाश्वि\textendash \,२८०\textendash \,धनप्रमाणास्त्रिंशत्स्वमुच्चे ऋणमत्र पाते~॥~२~॥"} इत्युभयं यन्त्रतो ज्ञेयम्~। भांशफलं चन्द्रस्यैव~। उदयान्तरं रविचन्द्रयोरेव~। चरकर्म सर्वेषाम्~। उक्तं च {\color{violet}करणप्रदीपे\textendash \,"यातं च देशान्तरमाब्दकं च भुजान्तरं केऽपि वदन्ति रामम्~। प्रमाणमत्रागम एव खेटाः स्युः संस्कृतास्तैरिह कर्मयोग्याः"} इति~। कानिचित्कर्माणि मध्यमेषु दीयन्ते कानिचित्स्फुटेषु चरदलसंस्कार-विधिः स्फुटक्रियानन्तरं सद्भिः~। अत्र देशान्तराब्दबीजरामबीजानि मध्यमेषु देयानि भांशफलं मध्यमचन्द्र एव ग्रन्थकृतोदयान्तरचरकर्मणि स्पष्टतामननू-ह्योक्ते तेन स्पष्टेषु दीयत इति स्वयमूह्यं किं बहुना~। अथ प्रकृतमुच्यते\textendash \,\hyperref[1.16]{'अब्दा गजाश्वैः'} इत्यादिकर्म देशान्तरं रामबीजं च ग्रहेषु दत्तं यन्त्रतो ज्ञेया औदायिका मध्यमाः~॥~१६~॥

\newpage

\begin{center}
{\large \textbf{लंकायामुदयकालिका ग्रहाः~।}}
\vspace{4mm}

\begin{tabular}{ccccccccc}
सू. & चं. & उ. & पा. & मं. & बु. & बृ. & शु. & श. \\
१ & ० & ७ & ३ & ६ & ३ & ११ & ८ & १\\
१ & १३ & २६ & १ & १६ & ३ & १० & २५ & २१\\
३१ & ९ & ४२ & २८ & ३ & १३ & ३९ & ४८ & १५\\
४ & ३८ & १४ & ३९ & ३६ & १० & २७ & २८ & ४६\\
५९ & ७९० & ६ & ३ & ३१ & २४५ & ५ & ९६ & २\\
८ & ३५ & ४१ & ११ & २६ & ३२ & ० & ८ & १०
\end{tabular}
\vspace{6mm}

{\large \textbf{कलादि रामबीजम्~।}}
\vspace{4mm}

\begin{tabular}{ccccccccc}
सू. & चं. & उ. & पा. & मं. & बु. & बृ. & शु. & श. \\
२ & १५ & ३० & ३० & ८० & ७०० & १९० & २७० & ९०\\
धनम् & ऋणम् & ध. & ऋ. & ध. & ध. & ऋ. & ऋ. & ध.
\end{tabular}
\vspace{6mm}

{\large \textbf{देशान्तरं कलादि~।}}
\vspace{4mm}

\begin{tabular}{ccccccccc}
सू. & चं. & उ. & पा. & मं. & बु. & बृ. & शु. & श. \\
० & ४ & ० & ० & ० & १ & ० & ० & ०\\
२२ & ५६ & २ & १ & १२ & ३२ & २ & ३६ & १\\
ध. & ध. & ध. & ध. & ध. & ध. & ध. & ध. & ध.
\end{tabular}
\vspace{6mm}

{\large \textbf{अब्दबीजं विकलादि~।}} 
\vspace{4mm}

\begin{tabular}{ccccccccc}
सू. & चं. & उ. & पा. & मं. & बु. & बृ. & शु. & श. \\
० & ५ & १४ & ३३ & ० & १९ & ६ & ४८ & ०\\
 & ऋ. & ध. & ध. & & ध. & ऋ. & ध. & 
\end{tabular}
\vspace{6mm}

{\large \textbf{अंशादि रामबीजम्~। }}
\vspace{4mm}

\begin{tabular}{ccccccccc}
सू. & चं. & उ. & पा. & मं. & बु. & बृ. & शु. & श. \\
० & ० & ० & ० & २ & १ & ० & ५ & ४\\
२ & ३४ & ३० & ३० & १० & ० & १० & ० & ४०\\
ध. & ऋ. & ध. & ऋ. & ध. & ध. & ऋ. & ऋ. & ध.
\end{tabular}
\end{center}

\afterpage{\fancyhead[RE,LO]{{\small{अ.\,२}}}}
\newpage

\begin{center}
{\large \textbf{त्रिकर्मसंस्कृताः सूर्यादिग्रहाः~। }}
\vspace{4mm}

\begin{tabular}{ccccccccc}
सू. & चं. & उ. & पा. & मं. & बु. & बृ. & शु. & श. \\
१ & ० & ७ & ३ & ६ & ३ & ११ & ८ & १\\
१ & १२ & २७ & ० & १८ & ४ & ९ & २० & २५\\
३३ & ४० & १२ & ५९ & १३ & १५ & २९ & ४९ & ५५\\
२६ & २९ & ३० & १३ & ४८ & १ & २३ & ५२ & ४१
\end{tabular}
\vspace{6mm}

{\large \textbf{गतिः~।}}
\vspace{4mm}

\begin{tabular}{ccccccccc}
५९ & ७९० & ६ & ३ & ३१ & २४५ & ५ & ९६ & २\\
८ & ३५ & ४१ & ११ & २६ & ३२ & ० & ८ & १०
\end{tabular}
\vspace{6mm}

{\large \textbf{इति करणकुतूहलवृत्तावेतस्यां सुमतिहर्षरचितायां गणककुमुद-\\
कौमुद्यां विवृतग्रहमध्यमानयनं प्रथमोऽध्यायः~॥~१~॥}}\\
\vspace{2mm}

\noindent\rule{7cm}{1pt}
\end{center}
\vspace{4mm}

{\small \textbf{अथ द्वितीयः स्पष्टाधिकारो व्याख्यायते~। तत्रादौ मन्दकेन्द्रोपयुक्तानि मन्दोच्चा-नीन्द्रवज्रयाह\textendash }}

\phantomsection \label{2.1}
\begin{quote}
{\large \textbf{{\color{purple}मन्दोच्चमर्कस्य गजाद्रि\textendash \,७८\textendash \,भागा \\
भौमादिकानां सदलाष्टसूर्याः~। \\
तत्त्वाश्विनः २२५ सार्द्धयमाद्रिचन्द्राः\\
१७२।३० क्वष्टौ शशाङ्काङ्गयमाः क्रमेण~॥~१~॥}}}
\end{quote}

अष्टसप्तत्यंशा राशिद्वयमष्टादशांशाः २।१८ सूर्यस्य मन्दोच्चम्~। चन्द्रोच्चं तु गणो द्विधेत्यादिनोक्तम्~। भौमस्य \hyperref[2.1]{सदलाष्टसूर्या}स्त्रिंशत्तलाधिकाष्टाविंशत्यु-त्तरशतांशाः राशिचतुष्टयम् ४ अष्टांशाः ८ त्रिंशत्कलाः भौमस्य मन्दोच्चम्~। \hyperref[2.1]{तत्त्वाश्विनः} \;पञ्चविंशत्युत्तरद्विशतांशाः \;पञ्चदशांशाधिकसप्तराशयो \;बुधस्य मन्दोच्चम् ७।१५~। \hyperref[2.1]{सार्द्धा}स्त्रिंशत्कलाधिका \hyperref[2.1]{यमाद्रिचन्द्रा} द्विसप्त-

\newpage

\noindent त्युत्तरशतांशा द्वाविंशत्यंशाधिकराशिपञ्चकं गुरोः ~५।२२ \hyperref[2.1]{क्वष्टौ} ८१ एकाशी-त्यंशा एकविंशत्यंशाधिकराशिद्वयं शुक्रस्य २।२१, शनेः पाठद्वयं \hyperref[2.1]{शशाङ्काङ्ग-यमाः} २६१ एकषष्ट्युत्तरद्विशतांशाः राशयोऽष्टौ ८ एकविंशत्यंशाः २१ शनेः~। वा मतङ्गाग्नियमा इति पाठः~। अष्टविंशत्यंशाधिकराशिसप्तकम् ७।२ अत्र हेतु-रुच्यते {\color{violet}सिद्धान्तशिरोमणौ}\textendash \,शनिमन्दोच्चस्य कल्पे युगेषवो ५४ भगणास्त-त्पक्षे मतङ्गाग्नियमा इति पाठः~। मन्दोच्चस्य कुसागराः ४१ भगणा इति पाठ-स्तत्पक्षे \,शशाङ्काङ्गयमा \,२६१ \,उत्पद्यन्ते~। यत्पक्षे \,युगेषवो \,भगणा \,इ{\color{violet}त्यार्यसि-द्धान्त}मतं तथा चोक्तम्\textendash \,सवितुरमीषाञ्च तथा धात्रादिसदग्निसिच्च मन्दोच्च-मिति~। तथा च {\color{violet}सूर्यसिद्धान्ते}\textendash \,गोऽग्नयः शनिमन्दस्य तत्पक्षे शनेर्मन्दोच्चम् ७।२८।३७ एतदेवार्यभटसम्मतम्~। अथ मृगांककरणप्रदीपे करणभाष्यादिषु मतङ्गाग्नियमाः २३८ इति मन्दोच्चं गृहीतं क्रमेण तद्वाक्यमिति महीधरा हस्ति-करा नृपाः खम् ७।२८।१६।० मन्दोच्चकानि कुञ्जरानलकराः २३८ शनेर्लवाः कीर्तिता इति निजमृदूच्चसम्भवा निजमृदूच्चं सम्भवति मन्दोच्चं तु मतङ्गाग्नियमा इति \,शनेर्विशेष \,इत्याचार्यस्यैव \,पक्षेऽङ्गीकृतम्~। \,उक्तं \,च\textendash \,{\color{violet}"ज्योतिषमागम-शास्त्रं विप्रतिपत्तौ न योग्यमस्माकम्~। स्वयमेव विकल्पयितुं किन्तु बहूनां मतं वक्ष्ये"} इति वचनम्~॥~१~॥

\newpage

\begin{center}
{\large \textbf{मन्दोच्चानि~।}} 
\vspace{4mm}

\begin{tabular}{cccccc}
सू. & मं. & बु. & बृ. & शु. & श. \\
२ & ४ & ७ & ५ & २ & ७\\
१८ & ८ & १५ & २२ & २१ & २८\\
० & ३० & ० & ३० & ० & ०\\
० & ० & ० & ० & ० & ० 
\end{tabular}
\end{center}
\vspace{4mm}

{\small \textbf{अथ शीघ्रफलोपयुक्तान्पराख्यान् शीघ्रोच्चानुपजात्याह\textendash }}

\phantomsection \label{2.2}
\begin{quote}
{\large \textbf{{\color{purple}कुकुञ्जरा वेदकृतास्त्रिदस्राः \\
सप्ताहयो विश्वमिताः पराख्याः~। \\
भौमादिकानामथ मध्यमोऽर्कः \\
शीघ्रोच्चमिज्यारशनैश्चराणाम्~॥~२~॥}}}
\end{quote}

एकाशीतिः \,८१ \,चतुश्वत्वारिंशत् \,४४ \,त्रयोविंशतिः \,२३ \,सप्ताशीतिः \,८७ त्रयोदश १३ परमाः क्रमेण \hyperref[2.2]{भौमादीनां पराख्याः} एते भौमादीनां परमफलानि जीवारूपाणि एतेषां धनूंषि परमशीघ्रफलानि तद्यथा भौमस्य ४२।४० बुधस्य ११।३४।४० गुरोः ११।० शुक्रस्य ४।६।५० शनेः ६।११।५ अथ ग्रहाणां शीघ्रोच्चं कथयति\textendash \,बुधशीघ्रोच्चं शुक्रस्य च मध्यमाधिकारे प्रोक्तम्, अन्येषां गुरुभौम-शनीनां मध्यमोऽर्कः शीघ्रोच्चं ज्ञेयम्~॥~२~॥\\

{\small \textbf{अथ मन्दकेन्द्रशीघ्रकेन्द्रपदधनर्णसंज्ञामुपजात्याह\textendash }}

\phantomsection \label{2.3}
\begin{quote}
{\large \textbf{{\color{purple}ग्रहोनमुच्चं मृदु चञ्चलं च केन्द्रे भवेतां मृदुचञ्चलाख्ये~। \\
त्रिभिस्त्रिभिर्भैः पदमत्र कल्प्यं स्वर्णं फलं मेषतुलादिकेन्द्रे~॥~३~॥}}}
\end{quote}

\newpage

देशान्तराब्दबीजविशुद्धेन \,ग्रहेण \,हीनं \,\hyperref[2.3]{मृदूच्चं} \,मन्दोच्चं \,\hyperref[2.3]{चलमुच्चं} \,शीघ्रोच्चं क्रमान्मन्दकेन्द्रं शीघ्रकेन्द्रं स्यात्, तथा च द्वादशराशीनां पदचतुष्टयं भवति तेषां प्रथमतृतीययोर्विषमसंज्ञा द्वितीयचतुर्थयोः समसंज्ञा, अथ मेषादिषट्क-गते केन्द्रे फलं मध्यमग्रहे मन्दस्पष्टे च धनं तुलादिकेन्द्रे ऋणम्~॥~३~॥\\

{\small \textbf{अथ भुजकोटीनुपजात्याह\textendash }}

\phantomsection \label{2.4}
\begin{quote}
{\large \textbf{{\color{purple}त्र्यूनं भुजः स्यात्त्र्यधिकेन हीनं \\
भार्द्धं च भार्द्धादधिकं विभार्द्धम्~। \\
नवाधिकेनोनितमर्कभं च \\
भवेच्च कोटिस्त्रिगृहं भुजोनम्~॥~४~॥}}}
\end{quote}

राशित्रयादूनं यदि केन्द्रं तदा स एव भुजः स्यात् यदि राशित्रयादूर्ध्वं केन्द्रं तर्हि तेन हीनं राशिषट्कं भुजः राशिषट्कादधिकं केन्द्रं विभार्द्धं रशिषट्कहीनं भुजः स्यात् नवराश्यधिकेन केन्द्रेण हीना द्वादशराशयः भुजः अथ भुजेन हीनं राशित्रयं कोटिः स्यात्~॥~४~॥\\

{\small \textbf{अथ भौममन्दोच्चपराख्ययोः स्फुटीकरणं वज्रयाह\textendash }}

\phantomsection \label{2.5}
\begin{quote}
{\large \textbf{{\color{purple}भौमाशुकेन्द्रे पदयातगम्य-\\
स्वल्पस्य लिप्ताः खखवेद\textendash \,४००\textendash \,भक्ताः~।\\
लब्धांशकैः कर्कमृगादिकेन्द्रे \\
हीनान्वितं स्पष्टमसृग्मृदूच्चम्~॥~५~॥\\
लब्धांशकानां त्रिलवेन हीनः \\
स्पष्टः परः स्यात् क्षितिनन्दनस्य~। }}}
\end{quote}

भौमस्य शीघ्रकेन्द्रं कृत्वेति मन्दस्पष्टात्तु यत् शीघ्रकेन्द्रं कृत्वेति तद्ग्राह्यं यदुक्तं {\color{violet}सिद्धान्तशिरोमणौ\textendash \,मन्दस्फुटोऽस्मा-}

\newpage

\begin{sloppypar}
\noindent {\color{violet}च्चलकेन्द्रपूर्वम्}" इति, {\color{violet}नरपता}वपि\textendash \,{\color{violet}"कार्यं चोच्चफलं प्राग्वद्द्वितीये~मन्द-कर्मणि~। तेन संस्कृतमन्दोच्चं संस्कृतं स्यात्परिस्फुटम्}" इति~॥ {\color{violet}लक्ष्मी-दासमृगाङ्ककारि}भिरित्थमेव प्रतिपादितम्, {\color{violet}"अतो मध्यममन्दोच्चेन प्रथम-कलानयनं तत स्पष्टान्मन्दोच्चादानेय}मिति पूर्वाचार्यमतमस्माकमप्येतदे-वाभिमतम्~। अथ शीघ्रकेन्द्रस्य त्रिभिस्त्रिभिर्भैः पदमिति पदे कल्पिते तस्य यातं तच्च राशित्रयात् पतितं तद्गम्यं तयोर्गतगम्ययोर्यदल्पं तस्य कलाः कार्यास्ताश्चतुःशतैः ४०० भक्ता लब्धमंशादिकं फलं तेन कर्कादिराशिषट्के गते भौमशीघ्रकेन्द्रे \hyperref[2.5]{असृग्मृदूच्चं} भौमस्य मन्दोच्चं हीनं कार्यं मकरादिषट्कगते तु युतः सन् स्फुटं भौममन्दोच्चं स्यात्~॥~५~॥ अथ स्वल्पस्य लिप्ताभ्यः खखवेदैर्लब्धफलस्यांशादिफलस्य \hyperref[2.5]{त्रिलवेन} तृतीयभागेन भौमस्य पराख्यो भौमपरो हीनः सन् भौमस्य परः स्फुटः स्यात्~॥\\

{\small \textbf{अथ ज्यासाधनं सार्धेन्द्रवज्रामाह\textendash }}

\phantomsection \label{2.6}
\begin{quote}
{\large \textbf{{\color{purple}रूपाश्विनौ २१ विंशति\textendash \,२०\textendash \,रङ्कचन्द्रा १९\\
अत्यष्टि\textendash \,११\textendash \,तिथ्यर्कनवेषुदस्राः~॥~६~॥\\
ज्याखण्डकान्यंशमितेर्दशाप्तं \\
स्युर्भुक्तखण्डान्यथ भोग्यनिघ्नाः~। \\
शेषांशकाः खेन्दु\textendash \,१०\textendash \,हृता यदाप्तं \\
तद्भुक्तखण्डैक्ययुतं भवेज्ज्या~॥~७~॥}}}
\end{quote}

एकविंशतिः २१ विंशतिः २० एकोनविंशतिः १९ सप्तदश
\end{sloppypar}

\newpage

\begin{sloppypar}
\noindent १७ पञ्चदश १५ द्वादश १२ नव ९ पञ्च ५ द्वि २ मितानि नव खण्डानि\textendash 

\begin{center}
{\large \textbf{ज्याखण्डानि}} 
\vspace{4mm}

\begin{tabular}{ccccccccc}
१ & २ & ३ & ४ & ५ & ६ & ७ & ८ & ९\\
२१ & २० & १९ & १७ & १५ & १२ & ९ & ५ & २\\
२१ & ४१ & ६० & ७७ & ९२ & १०४ & ११३ & ११८ & १२०
\end{tabular}
\end{center}
\vspace{1mm}

एतानि नव ज्याखण्डानि, यस्य ज्या साध्या तस्यांशाः कार्यास्तेभ्यो १० दशभिर्यावल्लब्धं तावन्ति भुक्तखण्डानि शेषमंशादि \hyperref[2.6]{भोग्ये}नाग्रिमखण्डेन गुणितं १० दशभिर्भक्तमाप्तं भुक्तखण्डानां पूर्वलब्धानामैक्यं युतं ज्या भवेत्~। यत्र १० दशभिर्भागो न लभ्यते तत्र प्रथमखण्डो भोग्यः~॥~७~॥\\

{\small \textbf{अथ धनुःकरणमुपजात्याह\textendash }}

\phantomsection \label{2.8}
\begin{quote}
{\large \textbf{{\color{purple}विशोध्य खण्डानि दशघ्नशेषात् \\
अशुद्धलब्धं धनुरंशकाद्यम्~। \\
विशुद्धसङ्ख्याहतदिग्\textendash \,१०\textendash \,युतं स्यात् \\
व्यस्तैर्दलैर्व्यस्तधनुर्ज्यके स्तः~॥~८~॥}}}
\end{quote}

यस्य ज्यायां धनुः साध्यते तस्या आद्यखण्डादारभ्य यावन्ति खण्डानि शुध्यन्ति तावन्ति विशोध्य शेषाद्दश\textendash \,१०\textendash \,गुणादशुद्धखण्डेन भक्ताल्लब्धं तद्विशुद्धसङ्ख्यया गुणितैर्दशभि\textendash \,१०\textendash \,र्युतमंशाद्यं धनुः स्यात्, अथ \hyperref[2.8]{व्यस्तै}र्वैपरीत्येन स्थितैर्दलैरेवं खण्डाः २।५।९।१२।१५।१७।१९।२०।२१~। यथोक्तविधिनोत्क्रमधनुरुक्तं ज्या च स्यात्, परमत्र प्रयोजनमनयोर्नास्ति, प्रसङ्गादुक्तम्~। अथाल्पान्तरत्वाल्लाघवायाचार्येणोक्तमपि भोग्यखण्डस्पष्टी-करणं {\color{violet}सिद्धान्तशिरोमणौ\textendash \,"यातैष्ययोः खण्डकयोर्विशेषः शेषांश-}
\end{sloppypar}

\newpage

\noindent {\color{violet}निघ्नो नख\textendash \,२०\textendash \,हृत्तदूनम्~। युक्तं गतैष्यैक्यदलं स्फुटं स्यात् क्रमोत्क्रमज्याक-रणेऽत्र भोग्यम्"~॥} यथा चतुर्विंशत्यंशानां २४ ज्यायां साध्यमानायां दशाप्ता लब्धं २ शेषं ४ गतखण्डं २० भोग्यखण्डं १९ अनयोरैक्यात् ३९ अर्द्धं १९।३० क्रमज्यात्वादूनितं १९।१८ स्फुटं भोग्यखण्डं जातं ततो भोग्यनिघ्नाः शेषांशका इत्यादि कर्म स्फुटं भोग्यखण्डेन कार्यमिति कृते जाता ज्या ४८।४३ इयं परम-क्रान्तिज्या ज्ञेया, अथ धनुःकरणे भोग्यखण्डस्फुटीकरणं {\color{violet}सिद्धान्ता}द्यथा\textendash {\color{violet}"विशोध्य खण्डान्यथ शेषकार्द्धनिघ्नं गतैष्यान्तरमेष्यभक्तम्~। फलोनयुग्भो-ग्यगतैष्यखण्डं चापार्थमेवं स्फुटभोग्यखण्डम्"} इत्यादिना गतैष्ययोरन्तरम् १ शेषांशैः ४ गुणितम् ४ नखैः २० भक्तम् ०।१२ अनेन भक्तं {\color{violet}'फलोनयुग्भोग्यगतै-ष्यखण्डं चापार्थमेवं स्फुटभोग्यखण्डम्'~॥} यथा जिनांशज्यायां ४८।४३ धनुः-करणे विशोध्य खण्डानि २१~।२० शेषम् ७।४३ अस्यार्द्धेन ३।५२ गतैष्यान्तरम् १ \;गुणितम् \;३।५२ \;गम्येन \;१९ \;भाजितम् \,०।१२ \;लब्धेन \,गतैष्यार्द्धम् \,१९।३० क्रमधनुःकरणत्वादूनम् १९।१८ स्फुटभोग्यखण्डं जातम्, अशुद्धलब्धमित्ययं विधिरनेन कार्यः जातं धनुः २४ इत्यन्यच्च सूक्ष्मतामिच्छता विधेयम्~॥~८~॥\\

{\small \textbf{अथ सूर्यादिनां मन्दफलानयनमिन्द्रवज्रोपजातिभ्यामाह\textendash }}

\phantomsection \label{2.9}
\begin{quote}
{\large \textbf{{\color{purple}सूर्यादिकानां मृदुकेन्द्रदोर्ज्यां दिग्घ्नी विभाज्याथ खपञ्चबाणैः~। \\
नागाग्निदस्रैर्गिरिपूर्णचन्द्रैर्वस्वङ्कभूभि-}}}
\end{quote}

\newpage

\phantomsection \label{2.10}
\begin{quote}
{\large \textbf{{\color{purple}र्वसुनेत्रनेत्रैः~॥~९~॥}}
\vspace{1mm}

\textbf{{\color{purple}युगाष्टशैलैर्मुनिपञ्चचन्द्रैः \\
फलं लवाः केन्द्रवशाद्धनर्णम्~। \\
कार्यं ग्रहे सूर्यविधू स्फुटौ स्तो \\
मन्दस्फुटाख्या इतरे स्युरेवम्~॥~१०~॥}}}
\end{quote}

सूर्येति~। सूर्यादीनां मन्दकेन्द्रस्य भुजः कार्यः स्वस्यांशमितेर्दशाप्तमित्य-नेन ज्या कार्या सा \hyperref[2.9]{दिग्घ्नी} दशगुणा सा सूर्यस्य सम्बन्धिनी चेत्तदा \hyperref[2.9]{खपञ्चबाणैः} ५५० \;सार्द्धपञ्चशतैर्भाज्या \;लब्धमंशादिफलम् \;१।३४।३२ \;अनेन \;संस्कृतो मध्यमोऽर्कः \,१।१।३३।२६ \,जातो \,मन्दफलसंस्कृतः \,सूर्यः \,१।३।७।५८ \,अस्मा-ज्जातं चरमृणम् ८६ अनेन संस्कृतः स्पष्टोऽर्कः\textendash \,१।३।६।३२ सा ज्या चेच्चन्द्रस्य तर्हि \hyperref[2.9]{नागाग्निदस्रै}रष्टत्रिंशदुत्तरद्विशत्या २३८ भाज्या लब्धमंशादिफलं स्यात्, भौमस्य \hyperref[2.9]{गिरिपूर्णचन्द्रैः} सप्तोत्तरशतेन १०७ भाज्या, बुधस्य \hyperref[2.9]{वस्वङ्कभूभि}र्द्व्यू-नद्विशत्या १९८, गुरो\hyperref[2.10]{र्वसुनेत्रनेत्रै}रष्टाविंशत्युत्तरद्विशत्या २२८, शुक्रस्य \hyperref[2.10]{युगा-ष्टशैलै}श्चतुरशीत्युत्तरसप्तशत्या ७८४, शने\hyperref[2.10]{र्मुनिपञ्चचन्द्रैः} सप्तपञ्चाशदुत्तरश-तेन १५७, एतदंशादि\hyperref[2.10]{फलं केन्द्रवशाद्धनर्णं कार्यं} मेषतुलादिकेन्द्रे क्रमाद्धनर्णं मध्यग्रहे कार्यम्, एवं कृते रविचन्द्रौ स्फुटौ भवतः, \hyperref[2.10]{इतरे} भौमादयो मन्दस्फुटा भवन्तीत्यर्थः~। अथैषां परमं मन्दफलं कलादि रवेः १३०।५०, चन्द्रस्य ३०२। ३१, भौमस्य ६७२।५४, बुधस्य ३६२।१०, गुरोः ३१५।४३, शुक्रस्य ११०।०, शनेः ४५८।३३ इति~॥~१०~॥

\newpage

{\small \textbf{अथ मध्यमार्कोदयिकास्तेषां स्फुटोदयिककरणाय कर्मोपजात्यर्द्धेनाह\textendash }}

\phantomsection \label{2.11}
\begin{quote}
{\large \textbf{{\color{purple}भानोः फलं भै\textendash {२७}\textendash {र्विहृतं} च चन्द्रे मध्ये विधेयं रविवद्धनर्णम्~।}}}
\end{quote}

भानोरिति~। ज्यायाः खपञ्चबाणैः ५५० लब्धं भानोः फलं \hyperref[2.11]{भैः} सप्तविंशति-भिर्भक्तं फलमंशादि मध्यमचन्द्रे धनर्णं कार्यम्, रविफलं रवौ यदि धनं स्यात् तदा \;मध्यमचन्द्रे \;धनं \;कार्यम् \;ऋणं \;चेत्तदा \;ऋणमेव \;कार्यं \;ततश्चन्द्रः \;स्पष्टो विधेयः अन्येषां स्वल्पान्तरत्वान्न कृतम्~॥\\

{\small \textbf{अथ सूर्यादीनां गतेर्मन्दफलमुपजात्यर्द्धेनेन्द्रवज्रया चाह\textendash }}

\begin{quote}
{\large \textbf{{\color{purple}स्वभोग्यखण्डं नव\textendash \,९\textendash \,हृत्खरांशो-\\
र्विश्वा\textendash \,१३\textendash \,हतं वेद\textendash \,४\textendash \,हृतं हिमांशोः~॥~११~॥}}
\vspace{1mm}

\phantomsection \label{2.12}
\textbf{{\color{purple}द्विघ्नं नगा\textendash \,७\textendash \,प्तं कुजसौम्ययोश्च \\
खाक्षै\textendash \,५०\textendash \,रिनैः १२ खार्क\textendash \,१२०\textendash \,मितैश्च भक्तम्~। \\
जीवादिकानां च गतेः फलं त-\\
त्स्वर्णं क्रमात्कर्कमृगादिकेन्द्रे~॥~१२~॥}}}
\end{quote}

यस्य ग्रहस्य मुक्तिफलं साध्यं तस्य भुजज्याकरणे यद्भोग्यखण्डं तत्सूर्य-सम्बन्धि \,चेत्तदा \,नव\textendash \,९\textendash \,भिर्भाज्यं \,लब्धं \,कलादिकं \,ग्राह्यं \,तत्सूर्यगतिफलं स्यात्, \;एवं \;चन्द्रस्य \;स्वभोग्यखण्डं \;विश्वै\textendash \,१३\textendash \,र्गुणितं \;वेदै\textendash \,४\textendash \,र्भक्तं \;लब्धं चन्द्रस्य गतिफलं स्यात्, अथ द्वाभ्यां २ गुणितं सप्तभि\textendash {७}\textendash {र्भक्तं} कुजस्य गति-फलं स्यात्, एवं बुधस्य च केवलम्, गुरोः स्वभोग्यखण्डं \hyperref[2.12]{खाक्षैः} पञ्चाशद्भिः ५० भक्तं लब्धं गुरोर्गतिफलं स्यात्, एवं शुक्रस्य स्वभोग्यखण्डमितैर्द्वादशभि\textendash \,१२\textendash \,र्भक्तं भृगुगतिफलं स्यात्,

\newpage

\noindent शनेः स्वभोग्यखण्डं \hyperref[2.12]{खार्कमितै}\textendash \,१२०\textendash \,र्भक्तं शनिगतिफलं स्यात् एवं स्वस्व-गतिफलं स्वस्वमध्यगतौ कर्कादिमन्दकेन्द्रे सति धनं कार्यम्, मकरादिकेन्द्रे त्वृणं कार्यम्, एवं कृते रविचन्द्रयोः स्फुटा गतिर्भवति, भौमादिकानां तु मन्द-स्फुटाख्या स्यात्~॥~१२~॥\\

{\small \textbf{अथ भौमादीनां शीघ्रफलानयनं शार्दूलविक्रीडितेनाह\textendash }}

\phantomsection \label{2.13}
\begin{quote}
{\large \textbf{{\color{purple}कोटिज्या चलकेन्द्रजा परगुणा द्विघ्नी तयोनान्विता\\
केन्द्रे कर्कमृगादिके परकृतिः खाभ्राब्धिशक्रै\textendash \,१४४००\textendash र्युता~।\\
तन्मूलं श्रवणः परेण गुणिता दोर्ज्याथ कर्णोद्धृता \\
तच्चायं चपलं फलं धनमृणं मन्दस्फुटे स्यात्स्फुटः~॥~१३~॥}}}
\end{quote}

मन्दस्फुटं \,शीघ्रोच्चाद्विशोध्य \,यच्छेषं \,तच्छीघ्रकेन्द्रं \,भवति \,तस्य \,कोटिः साध्या तस्याश्च ज्या साध्या सा स्वकीयेन पराख्येण गुण्या सैव पुनर्द्वाभ्यां २ गुणिताधः स्थाप्या षष्टिभक्ता फलेनान्विता कार्या, अथ परस्य कृतिस्तेनैवा-ङ्केन तस्यैवाङ्कस्य गुणनं कृतिः सा परस्य कृतिः \hyperref[2.13]{खाभ्राब्धिशक्रै}श्चतुःशताधिक-चतुर्दशसहस्रै\textendash \,१४४००\textendash \,र्युताः कार्या एतादृशा परकृतिः कर्कादिशीघ्रकेन्द्रे सति परगुणा द्विस्था कोटिज्या हीना कार्या तस्येदं मूलम् {\color{violet}"त्यक्त्वान्त्याद्विष-मात्"} इति {\color{violet}लीलावत्यु}क्तविधिना कृतश्रवणः कर्णसंज्ञकोऽङ्को बोद्धव्यः, अथ शीघ्रकेन्द्रस्य दोर्ज्या पराख्येण गुणिता कर्णसंज्ञकेन

\newpage

\noindent भक्ता लब्धस्य चापं धनुः विशोध्य खण्डानीत्यनेन कार्यं तदेव धनुः चञ्चलं शीघ्रफलं तत्फलं मन्दस्फुटे ग्रहे धनमृणं कार्यं मेषतुलादिकेन्द्रवशात्, मेषा-दिकेन्द्रे धनं तुलादिकेन्द्रे ऋणं तदा ग्रहः स्फुटः स्यात्~॥~१३~॥\\

{\small \textbf{अथास्याप्यतिस्फुटत्वज्ञानं भौमस्य विशेषं चोपजात्याह\textendash }}

\phantomsection \label{2.14}
\begin{quote}
{\large \textbf{{\color{purple}तदुत्थमाद्येन चलेन मध्य-\\
श्चेत्संस्कृतः स्पष्टतरस्तदा स्यात्~। \\
दलीकृताभ्यां प्रथमं फलाभ्यां\\
ततोऽखिलाभ्यामसकृत्कुजस्तु~॥~१४~॥}}}
\end{quote}

तदुत्थेति~। यत्स्फुटो जातस्तस्मादुत्थं \hyperref[2.14]{तदुत्थं} तेन तदुत्थेन मान्देन पुनस्त-दुत्थचलेन चेन्मध्यमः संस्क्रियते तदासौ स्पष्टतरः स्यात् तदुक्तं भवति यः स्पष्टः \;कृतः \;स \;एव \;मध्यमः \;परिकल्प्यः \;तस्मात्प्रागुक्तप्रकारेण \;मन्दफलं संसाध्य मध्यमे संस्कार्यमित्यनयैव रीत्या स्पष्टतरः स्यात्, एवमसकृद्यावत् स्थिरः \,स्यात्, \,भौमस्य \,तु \,विशेषः \,प्रथमं \,मन्दफलेन \,दलीकृतेन \,शीघ्रफलेन दलीकृतेन चैवं \hyperref[2.14]{दलीकृताभ्यां} प्रथमं फलाभ्यां ततो\hyperref[2.14]{ऽखिलाभ्यां} सम्पूर्णाभ्या-मिति ~यस्तु ~दलीकृतमन्दफलाभ्यां ~संस्कृतस्तस्मान्मान्दं ~सकलं ~मध्यमे संस्कार्यं तस्माच्छीघ्रं संसाध्य तस्मिन्नेव सकलं शीघ्रं संस्कुर्यात् इति पुनः पुनर्दलीकृताभ्यां ~ततोऽखिलाभ्यां ~सम्पूर्णाभ्यामिति, ~यस्तु ~दलीकृताभ्यां संस्कृतस्तन्मान्दं \;सकलं \;मध्यमे \;संस्कार्यं \;तस्माच्छैघ्र्यं \;संसाध्यं \;तस्मिन्नेव सकलं शैघ्र्यं संस्कुर्यादिति पुनः

\newpage

\noindent पुनर्दलीकृताभ्यां ततोऽखिलाभ्यामेवमसकृद्यावदविशेषः स्यात्तावदिति, एवं भौमः स्पष्टतरः स्यात्, यैर्बुधादीनामपि दलीकृताभ्यां प्रथमं फलाभ्यामिति कृतं तदसत् लक्ष्मीदासगणकवचनम्, अथ भौमगतौ विशेषः दलीकृताभ्या-मिति तथैवोक्तं {\color{violet}सिद्धान्तशिरोमणौ\textendash \,"स्यात्संस्कृतो मन्दफलेन मध्यो मन्द-स्फुटोऽस्माच्चलकेन्द्रपूर्वम्~। विधाय शैघ्र्येण फलेन चैवं खेटः स्फुटः स्याद-सकृत्फलाभ्याम्~॥ दलीकृताभ्यां प्रथमं फलाभ्यां ततोऽखिलाभ्यामसकृत्कु-जस्तु~॥ स्फुटौ रवीन्दू मृदुनैव वेद्यौ शीघ्राख्यतुङ्गस्य तयोरभावात्"} इत्यत्र दलीकृताभ्यामित्यादि यदुक्तं तत्कुजस्यैव न त्वन्येषामिति भौमः प्रथमाभ्यां द्वाभ्यां \;फलाभ्यामर्द्धीकृताभ्यां \;संस्कार्य \;ततो \;द्वाभ्यां \;समग्राभ्यां \;संस्कृत्य स्पष्टो भवति किन्त्वसकृत् कर्मणा यावत्स्थिरो भवत्यपरे ग्रहाः समग्राभ्यां फलाभ्यां प्रथमाभ्यां च सकृदेव स्पष्टाः भवन्ति~॥~१४~॥\\

{\small \textbf{अथ गतिस्पष्टीकरणं सार्धेन्द्रवज्रयोपजात्यर्द्धेनाह\textendash }}

\phantomsection \label{2.15}
\begin{quote}
{\large \textbf{{\color{purple}गतेः फलेनैव तु संस्कृता या \\
मध्या गतिर्मन्दगतिर्भवेत् सा~।\\
ग्रहस्य मन्दस्फुटभुक्तिवर्जिता \\
स्वाशीघ्रकेन्द्रस्य गतिर्भवेत् सा~॥~१५~॥}}
\vspace{1mm}

\phantomsection \label{2.16}
\textbf{{\color{purple}द्राक्केन्द्रभुक्तिर्गुणिताशुचाप-\\
भोगज्यया खाब्धि\textendash \,४०\textendash \,गुणा च भक्ता~। \\
सप्तघ्नकरणेन चलोच्चभुक्तेः \\
शोध्या विशेषा स्फुटखेटभुक्तिः~॥~१६~॥}}}
\end{quote}

\newpage

द्राक्केन्द्रेति~। मन्दस्फुटा गतिः शीघ्रोच्चभुक्तेः शोधयेत् शेषं द्राक्केन्द्रभुक्ति-र्भवति सा शीघ्रचापभोग्यज्यया गुणिता शीघ्रमूलानि परमाणि करणशीघ्र-फलार्थं \,चापकरणे \,यदशुद्धं \,खण्डं \,तेन \,गुण्या \,पुनः \,\hyperref[2.16]{खाब्धि}भिश्चत्वारिंशद्भि-र्गुणिता \hyperref[2.16]{सप्तघ्नकरणेन} सप्तगुणेन कर्णसंज्ञकेन भक्ता लब्धं कलाद्यं शीघ्रोच्च-भुक्तेः शोध्यं शेषं स्फुटा भुक्तिर्भवति यदा सप्तघ्नकरणाप्तफलस्याधिक्याच्छी-घ्रोच्चभुक्तिस्तेन हीना न स्यात्, तदा विपरीतशुद्ध्या फलं शीघ्रोच्चभुक्त्या हीनं शेषं स्फुटभुक्तिस्तत्र वक्रगतिरिति ज्ञेयम्, तदा ग्रहो वक्रीत्यर्थः~। कैश्चिद्गतीना-मप्यसकृत्कर्म कृतं तदसत् यतोऽन्त्यकर्मण्येव गतः साधनं प्राक्तनगतेरनुप-योगित्वात्, अथ विशेषश्चात्र यदा ग्रहाणां परमशीघ्रफलं तदा मध्यमैव गतिः स्पष्टा \,ज्ञेया \,यदुक्तं \,{\color{violet}सिद्धान्तशिरोमणौ\textendash \;"कक्षामध्यात्तिर्यग्रेखा \,प्रतिवृत्तस-म्पाते~। मध्यैव गतिः स्पष्टा परं फलं तत्र खेटस्य"~॥} स्वशीघ्रोच्चसमे ग्रहे स्पष्टा शीघ्रगतिः परमोच्यते~॥~१६~॥\\

{\small \textbf{अथोदयान्तरचरादिषूपयुक्तायनांशानुपजात्यर्द्धेनाह\textendash }}

\phantomsection \label{2.17}
\begin{quote}
{\large \textbf{{\color{purple}अथायनांशाः करणाब्दलिप्ता \\
युक्ता भवास्तद्युतमध्यभानोः~॥~१७~॥}}}
\end{quote}

\hyperref[2.17]{अथा}नतये मङ्गलार्थं वा करणगताब्दतुल्याभिः कलाभिर्युक्ता \hyperref[2.17]{भवा} एका-दशांशा \,अयनांशा \,भवन्ति \,विशेषश्चात्र \,यथाब्दादावयनांशाः \,११।२४ \,ग्रन्थ-कृता चतुर्विंशतिविकलान्विहायांशा एव \hyperref[2.17]{भवा} एकादशमिता गृहीताः अथास्य साध-

\newpage

\noindent नम्, एषामयनांशानां प्रतिवर्षमेकैककलाश्चाधिका उत्पद्यन्ते तत्पक्षे परमाय-नांशास्त्रिंशदंशाः ३० भवन्ति यत्पक्षे प्रतिवर्षं चतुःपञ्चाशद्विकलाः ५४~ उत्प-द्यन्ते तत्पक्षे सप्तविंशत्यंशाः परमायनांशा उत्पद्यन्ते, उक्तञ्च {\color{violet}सूर्यसिद्धान्त-टीकायां\textendash ~~प्रतिवर्षमेकलिप्ता ~~षष्टिवर्षैरंशमेकमष्टादशवर्षशतैरेको ~~राशिः षट्-शताधिकैरेकविंशतिसहस्रैर्वर्षैर्भगणमेकं ~~२१६०० ~~चतुर्युगाब्दैरेभिर्वर्षैः ४३२०००० भगणशतद्वयम् २०० एवोत्पद्यते}, अतः {\color{violet}सूर्यसिद्धान्ते\textendash \,त्रिंशत्-अंशावधिचलनांशा जायन्ते प्राक्चलनेऽयनांशाः गणकैर्योज्यन्ते न तु कदापि पात्यन्ते}, साम्प्रतं विषमराशित्वात्प्राक् चलनं प्राक् चलिते ये भक्तभागास्ते वृद्धिमन्तोऽयनांशा \;ज्ञेयाः \;एवं \;प्रत्यक् \;चलनमग्रतः \;समराशौ \;भविष्यति तत्र ये भुक्तभागास्ते त्रिंशद्भिः ३० शोधिता अत एवापवृत्त्या शेषं ज्ञेयास्तथाय-नांशा भविष्यन्ति तद्युक्तादर्कतः क्रान्त्यादिकं साध्यम्, {\color{violet}'अयनांशाः प्रदातव्या लग्ने कान्तौ चरागमे~। सत्रिभे वित्रिभे लग्ने दृक्कर्मणि सदा बुधैः~॥ कृते कर्मणि ते त्याज्या न त्याज्याः सूर्यपर्वणि'} पुनः {\color{violet}'युक्तायनांशादवमः प्रसाध्यः कालौ च खेटात्खलु भुक्तभागौ'} इति, अतो यावन्तो गताब्दास्तावन्त्यः कला अयनांशा इत्युपपन्नं सविस्तरं ग्रन्थान्तरात्, यथात्र करणे गताब्दाः ४३६ कलाः षष्टि-भक्ताः ७।१६ अंशादि अनेनांशादिना \hyperref[2.17]{भवा} एकादशांशाः ११ अन्विताः १८। १६।० चैत्रात्प्रतिमासं पञ्च विकला

\newpage

\noindent वर्द्धन्ते ततो मासद्वयं गतं तेन दश १० विकलाः युक्ता एवमयनांशाः १८।१६। १०~॥~१७~॥\\

{\small \textbf{अथ पूर्वं ये ग्रहाः साधितास्ते मध्यमसावनोदयिकास्तान् स्फुटसावनोदयिक-करणायोदयान्तरं कर्मोपजात्याह\textendash }}

\phantomsection \label{2.18}
\begin{quote}
{\large \textbf{{\color{purple}द्विघ्नस्य दोर्ज्या शरहृद्विलिप्ता \\
भानोर्विधोः क्वक्षि\textendash \,२१\textendash \,हृताः कलास्ताः~। \\
स्वर्णं च युग्मौ जयदस्थितेऽर्के \\
क्रमेण कर्मेत्युदयान्तराख्यम्~॥~१८~॥}}}
\end{quote}

द्विघ्नस्येति\textendash \,तैरयनांशैर्युक्तस्य मध्यमार्कस्य द्विगुणस्य भुजज्या पञ्चभिः ५ भक्ता लब्धा विकला समविषमपदस्थिते सायनेऽर्के क्रमेण धनमृणं वा रवौ कार्यं समपदस्थे सायनेऽर्के धनं कार्यं विषमपदस्थे सायनेऽर्के ऋणं सैव भुज-ज्यैकविंशतिभिः २१ भक्ता लब्धं कलादि सूर्यवच्चन्द्रे धनमृणं वा कार्यम् एतदु-दयान्तराख्यं कर्मापरेषामन्येषां स्वल्पान्तरत्वान्न कृतं यथारयनांशैः १८।१६। १० मध्यमोऽर्कः १।१।३३।२६ युतः १।१९।४९।३६ द्विगुणः ३।९।३९।१२ अस्य भुजः २।२०।२०।४८ अस्य ज्या ११८।४ पञ्च\textendash \,५\textendash \,भक्ता लब्धा विकलाः २३ रवेरुदयान्तरं सायनोऽर्को विषमपदे तेन स्पष्टे खावृणं कार्यमेवमेषैव भुजज्या ११८।४ एकविंशतिभिः २१ भक्ता लब्धं कलादि ५।३७ सूर्यवत् स्पष्टे चन्द्रे ऋणं कार्यम्~॥~१८~॥

\newpage

{\small \textbf{अथोन्मण्डलस्थानां, ग्रहाणां स्वक्षितिजस्यासन्नकरणाय चरकर्म प्रोक्तं तदक्ष-प्रभाक्षसाध्यं तेन तद्द्वयमपि मालिनीद्वयेनाह\textendash }}

\phantomsection \label{2.19}
\begin{quote}
{\large \textbf{{\color{purple}अयनलवदिनैः प्राङ्मेषसङ्क्रान्तिकाला-\\
द्भवति दिवसमध्ये याक्षभाक्षप्रभा सा~। \\
दश\textendash \,१०\textendash \,गज\textendash \,८\textendash \,दश\textendash \,१०\textendash \,निघ्नी साक्षभान्त्या त्रिभक्ता \\
प्रतिगृहचरखण्डान्यायनांशाढ्यभानोः~॥~१९~॥}}
\vspace{1mm}

\phantomsection \label{2.20}
\textbf{{\color{purple}भुजगृहमितयोगो भोग्यखण्डांशघातात् \\
खगुणलवयुगस्वं स्वं चरं गोलयोः स्यात्~। \\
चरपलगतिघातः षष्टिभक्तो विलिप्ताः \\
स्वमृणमुदयकाले व्यस्तमस्तग्रहेषु~॥~२०~॥}}}
\end{quote}

अयनेति~। अयनांशोत्पन्नदिनैर्मेषसङ्क्रमपूर्वतो यद्दिनं तस्य मध्याह्ने द्वाद-शाङ्गुलशङ्कोर्या \,छाया \,साक्षप्रभा \,स्यात्, \,अथायनांशोत्थदिनानयनम्, \,अय-नांशाः \;स्वद्विचत्वारिंशदंशेन \;हीनाः \;क्रियन्ते \;अयनदिनानि \;भवन्ति, \;अथ सूर्यस्य मेषादौ गतिः तस्मिन् ५९८ अयनांशाः सङ्गुण्य षष्ट्या विभजेल्लब्धम-यनदिनानि \;मध्यन्दिने \;मध्याह्नसमये \;द्वादशाङ्गुलशङ्कोर्या \;छाया \;साक्षप्रभा भवति \;साक्षप्रभा \;स्थानत्रयस्थिता \;दशभिरष्टभिर्दशभिश्च \;क्रमेण \;गुणिता ततोऽन्त्या या दशगुणा सा त्रिभिर्भक्ता त्रीणि चरखण्डानि भवन्ति एतावन्ति चरखण्डानि \,यावदष्टाङ्गुलाक्षभा \,तावत्समीचीनानि \,ततः \,परं \,सान्तराणीति विचार्य ततोऽयनांशैर्युक्तस्य सूर्यस्य स्फुटार्कस्य भुजः कार्यस्तद्भुजराशिसम-सङ्ख्यचरखण्डयोगो विधेयः एवं जातं चरं सौम्यगोले

\newpage

\noindent मेषादिराशिषट्कस्थे सायनेऽर्के ऋणमृणरूपं भवति चरं याम्यगोले तुलादि-षट्के स्वं धनरूपम्, अथ पूर्वागतैश्चरपलैर्ग्रहस्य गतिर्गुण्या षष्ट्या भाज्या लब्धं विकला \;औदयिके \;स्वमृणमिति \;पूर्वोक्तवद्धनरूपत्वेन \;धनमृणरूपत्वेनर्णं विधेयं ग्रहे, अस्तकाले तु व्यस्तं धनत्वमृणमृणत्वं धनमेवं चरकर्मसंस्कृतो ग्रहः \;स्यात्, \;अस्य \;किमपि \;विशेषान्तरमुच्यते; \;औदयिके \;धनर्णं \;यथागतं कुर्यात्, अस्ते व्यस्तं कुर्यात्, मध्याह्ने मध्यरात्रे च चरपलसंस्काराभावः दक्षि-णोत्तरवृत्तस्यैकत्वादित्येवावगन्तव्यं \;परन्त्वेषां \;चतुर्णां \;स्थानानां \;व्यवधान-स्थिते \,ग्रहे \;किं \,कार्यमित्युच्यते, \;उदयकालाद्घटीभिश्चाल्यते \,तस्य \;मध्याह्ने चास्तेऽप्यर्द्धरात्रे सर्वत्र तद्व्यवधानस्थे चौदयिकवज्ज्ञेयं गत्यर्द्धं कृत्वास्तका-लिकः कृतस्तस्माद्घटीभिश्चालितस्य व्यस्तमेव सर्वत्र ज्ञेयं गतिचतुर्थांशं दत्त्वा पादोनां च दत्त्वा दक्षिणोत्तरवृत्तस्थं च कृत्वा तस्माद्यावद्दिनावधि चाल्यते तत्र तस्यैव \,तत्फलाभाव \,इति \,ज्ञेयम्~। अयनांशाः \,१८।१६।१० \,गोमूत्रिकया सूर्य-गत्या ५८।३५ गुणिता १०७०।१७।५ षष्ट्या भक्ता लब्धानि दिनानि १७।५०। १७ अथवा स्वद्विचत्वारिंशदंशेन ०।२६।५ गुणितोऽयनांशैः १८।१६।१० हीनः कार्यः \;१७।५०।५ \;एभिर्दिनैर्मेषसङ्क्रमतः \;पूर्वदिनैर्मध्याह्ने \;द्वादशाङ्गुलशङ्को-श्छाया शिवपुर्यामक्षभाङ्गुलाद्या ५।३०

\newpage

\noindent त्रिस्था दशाष्टदशगुणा ५५।४४।५५ अन्त्या त्रिभक्ता चरखण्डानि ५५।४४। १८ \;उदयान्तरसंस्कृतः \;स्पष्टोऽर्कः \;१।३।७।३४ \;अयनांशैः \;१८।१६।१० \;युतः १।२१।२३।४४ अस्य भुजोऽयमेव भुजगृहमितयोगः भुजे १ राशिस्तेनैकचर-खण्डस्य योगः ५५ भुक्तं शेषांशाः २१।२३।४४ भोग्यखण्डेन ४४ गुणितं ९४१। २४।१६ त्रिंशद्भक्तं लब्धेन ३१।२२ भुक्तचरखण्डयोगो ५५ युक्तः ८६।२२ सुगम-त्वात् २२ त्यक्तमुत्तरगोले सायनार्कस्तेनर्णरूपाणि, अथ चरकर्मचरपलैः ८६ सूर्यस्पष्टा गतिः ५७।९ गुणिता षष्ट्या भक्ता लब्धं विकला ८२ एवं चन्द्रस्य विकला १२०३ भौमस्य ७ बुधस्य १४८ गुरोः १६ शुकस्य ८७ शनेः १० राहोः ४ राहुं विना सर्वेषामृणम्, वक्रिणि विपरीतमिति वचनाद्राहोर्वक्रत्वाद्धनमेवं सर्वकर्म स्पष्टा औदयिका यन्त्रे लिखिताः~॥~२०~॥\\

{\small \textbf{अथ तिथ्यादिसाधनं द्रुतविलम्बितद्वयेनाह\textendash }}

\phantomsection \label{2.21}
\begin{quote}
{\large \textbf{{\color{purple}विरविचन्द्रलवारवि\textendash \,१२\textendash \,षड्\textendash \,६\textendash \,हृताः \\
पृथगितास्तिथयः करणानि च~। \\
कुरहितानि बवाच्छकुनिप्रभृ-\\
त्यसितभूतदलादि चतुष्टयम्~॥~२१~॥}}

\phantomsection \label{2.22}
\textbf{{\color{purple}ग्रहकलाः सरवीन्दुकलाहृताः \\
खखगमैश्च ८०० भयोगमिती क्रमात्~। \\
अथ हृताः स्वगतैष्यविलिप्तिकाः \\
स्वगतिभिश्च गतागतनाडिकाः~॥~२२~॥}}}
\end{quote}

\newpage

\begin{center}
{\large \textbf{इतीह भास्करोदिते ग्रहागमे कुतूहले\\ विदग्धबुद्धिवल्लभे नभःसदां स्फुटक्रिया~॥~२~॥}}
\end{center}

अर्कोनचन्द्रस्यांशा द्विस्था एकत्र रवि\textendash \,१२\textendash \,हृता अपरत्र षड्\textendash \,६\textendash \,हृता द्वादशाप्ता ~~\hyperref[2.21]{इता} ~~गताः ~~शुक्लपक्षप्रतिपदादित\hyperref[2.21]{स्तिथयः} ~~द्वादशभिरंशैरेका तिथिः~। उक्तं च\textendash {\color{violet}"अर्काद्विनिःसृतः प्राचीं यद्यात्यपहरञ्छशी~। तच्चन्द्रमानमं-शैस्तु ज्ञेया द्वादशभिः स्थितिः"} इति~। षडाप्ता एकरहिता निगदितकरणानि तानि बवाद्बवनामकरणात् शुक्लपक्षप्रतिपदुत्तरार्द्धाद्भवति कृष्णचतुर्दश्युत्त-रार्द्धाच्छकुनादि चतुष्टयं भवति, उक्तं च चतुर्दश्यन्त्यार्द्धादि शकुनादि चतु-ष्टयं शकुनिचतुष्पदनागकिंस्तुघ्नं प्रथमे प्रतिपत्तिथ्यर्द्धाद्बवादिति, \hyperref[2.22]{ग्रहकला} इति यस्य ग्रहस्य नक्षत्रं जिज्ञासितं भवति तस्य कला अष्टशतैः ८०० भाज्याः लब्धसमान्यश्विनीतो ~गतनक्षत्राणि, ~चन्द्रार्कयोगस्य ~कला ~अष्टशतैः ~८०० भक्ता लब्धसमा योगा गता विष्कुम्भादयः, अथ सर्वत्र भक्तादवशिष्टं तद्गतं तदेव गतं स्वहरात्पतितं गम्यं भवति गतगम्ययोः विकलाः स्वभुक्त्या भक्ता गता गम्याश्च घटिका भवन्ति, यदि कलीकृतया भाज्यन्ते गम्यया प्रतिविक-लया भाज्या लब्धा मता गम्याश्च घटिका भवन्तीति विशेषः, यथा सूर्यभुक्त्यून-याप्तास्तिथिकरणघट्यस्तथा तस्यैव गत्याप्ता नक्षत्रघटिका रविचन्द्रयोगाप्ता योगघटिकाः, यथा रविणा १।३।६।१२ चन्द्रः ०।८।४७।२१ ऊनः ११।५।४१।९ अस्यांशाः ३३५।४१।९ द्वादश-

\newpage

\noindent भक्ता लब्धं २७ गतास्तिथयः कृष्णद्वादशीगता शेषं ११।४१।९ त्रयोदश्यागतं स्वहरात् पतितं गम्यम् ०।१८।५१ अस्य प्रतिविकलाः ६७८६० गतस्य प्रतिवि-कलाः २५२४१४० सूर्यचन्द्रगत्यन्तरेण ७८१।५२ विकलीकृतेन ४६९१२ भक्ता लब्धं गतघटिकाः ५३।४८ अस्य गम्यस्य प्रतिविकलाः ६६८६० पूर्वभाजकेन ४६९१२ भक्ता लब्धा घटिका १।२६ तिथिपत्रे ५४।२० तद्दिनोच्चौदयिकौ रवि-चन्द्रौ \,अथ \,त्रयोदश्यागता \,तत्र \,को \,हेतुरुच्यते\textendash \,मासान्तत्वाच्चतुस्त्रिंशत्कला रामबीजस्यर्णत्वात्तिथिः पञ्चदशकलाः रामबीजशुद्धाच्चन्द्रात् साधिता घटि-काद्वयेनाधिका यत्र तिथ्यादिसाधनं भवति तद्यथा अत्रैव दिने रविः १।३।६। १२ पञ्चदशकला रामबीजशुद्धाः स्पष्टचन्द्रः ९।७।२८ विरविचन्द्र इत्यादिना चतुर्दशी, एषा तिथिघट्यादि ५५।९ अथामावास्यामौदयिकश्चतुस्त्रिंशत्कला रामबीजासक्तौ \,स्पष्टो \,रविः \,१।४।५।२९ \,गतिः \,५७।२८ \,चन्द्रः \,०।३।१७।१६ गतिः ८४०।५० कर्मशुद्धौ, आभ्यां तिथिरमावास्यैव ५०।५१।९ तिथिपत्रेऽमा-वास्या ४८।४५ यदत्रान्तरं \;तद्रामबीजभक्तं नान्यदिति ज्ञेयम्, मया बालाव-बोधाय \,विस्तीर्योक्तं \,धीमन्तः \,स्वयमेव \,वदन्ति, \,अथ \,करणसाधनं \,विरवि-चन्द्रांशाः ३३५२ षड्-भक्ता लब्धं करणानि ५५ एकोनानि ५४ सप्तभक्ते शेषं ५ बवादिगणने गतानि पञ्चकरणानि षष्ठं करणं भोग्यं वणिजमागतमस्य गतम् अंशादि ५।४१।९ स्वहरात्पतितं गम्यम् ०।१८।५१ गतस्य प्रतिविकला

\newpage

\noindent १२२८१४० विकलीकृतेन गत्यन्तरेण ४६९१२ भक्ता लब्धं गतघटिकाः २६।१० गम्यस्य प्रतिविकलाः ६७८६० विकलीकृतेन गत्यन्तरेण ४६९१२ भक्तं लब्धं गम्या घटिकाः १।२६ अथ चन्द्रनक्षत्रार्थं चन्द्रः ०।८।४७।२१ कलाः ५२७।२१ खखगजैः ८०० भक्तं लब्धं ० अश्विन्या एव गतम् ५२७।२१ अस्य प्रतिविकलाः १८९८४६० \;चन्द्रगत्या \;विकलारूपया \;५०३६० \;भक्ता \;लब्धं \;गतघटिकाः \;३७। ४१ गतं स्वहरात्पतितं गम्यं २७२।३९ प्रतिविकलाः ९५१५४० चन्द्रगत्या ५०३६० भक्ता लब्धं गम्या घटिकाः १९।२९ अथ सूर्यः १।३।६।१२ अस्य कलाः १९८६ अश्विनीतो गतनक्षत्रम् २ कृत्तिकाभोग्यमिदं शेषम् ३८६।१२ कृत्तिकाया गतम् अस्माद्यदि गतघटिका आनीयन्ते तदा भुक्तिसमग्रकृत्तिका याति तद-पेक्षया \;नक्षत्रचरणकलारूपया \;३४४८ \;भक्ता \;लब्धं \;कृत्तिकाद्वितीयपादस्य दिनानि ३।१४।२४ गतं यदि गम्याः साध्यन्ते तदा शेषम् १८६।१२ स्वहरात् २०० \,पतितम् \,१३।४८ \,विकला \,८२८ \,सूर्यगत्या \,३४४८ \,भक्ते \,लब्धं \,गम्या \,दिन-घटिकाः ०।१४।२४ एवं सर्वेषां ग्रहाणां गतं गम्यं दिनानि घटिका वा साध्याः स्वबुद्ध्या सर्वमूह्यम्~। अथ योगार्थं रविचन्द्रयोगः १।११।५३।३३ अस्य कला अष्टशतैः ८०० भक्ता लब्धं ३ भोग्यः सौभाग्ययोगः शेषं ११३।३३ गम्यं ६८६।२७ गतियोगेन ८९८६

\afterpage{\fancyhead[RE,LO]{{\small{अ.\,३}}}}
\newpage

\noindent ५।४८ ~~प्राग्वत् ~~भक्तं ~~लब्धं ~~गम्यं ~~घटिकादि ~~४५।५५~॥ ~अथ ~~ग्रहाणां भूमेरुपरिस्थितियोजनानि ~~सूर्यस्य ~~६८९३७७ ~~चन्द्रस्य ~~५१५६६१ ~~भौमस्य १२९६६१८ ~~बुधस्य ~~१६६०३१ ~~गुरोः ~~८१७६५३७ ~~शुकस्य ~~४२०४०८ ~~शनेः २०३१९०६८ नक्षत्रस्य ४१३६२६६७ ध्रुवस्य ८२७२५३१५ करणविवृतौ स्फुटतादिखेटानाम्~॥~२१~॥~२२~॥
\vspace{2mm}

\begin{center}
{\large \textbf{इति श्रीकरणकुतूहलवृत्तौ स्पष्टाधिकारो द्वितीयः समाप्तः~॥~२~॥}}\\
\vspace{2mm}

\noindent\rule{7cm}{1pt}
\end{center}
\vspace{4mm}

{\small \textbf{अथ त्रिप्रश्नाधिकारो व्याख्यायते, त्रिभिः प्रकारैः प्रश्नकथनं यस्मिन्स त्रिप्रश्नः के ते त्रिप्रकाराः दिग्देशकालस्त्रयः प्रश्नाश्चक्र इष्टच्छायाद्यास्तेषां परिज्ञानमिति तत्रादौ लङ्कादेशीयस्वदेशीयमेषादिलग्नपरिमाणमुपजातिजातकचतुष्केणाह\textendash }}

\phantomsection \label{3.1}
\begin{quote}
{\large \textbf{{\color{purple}लङ्कोदया नागतुरङ्गदस्रा\\ 
गोऽङ्काश्विनो रामरदा विनाड्यः~। \\
क्रमोत्क्रमस्थाश्चरखण्डकैः स्वैः \\
क्रमोत्क्रमस्थैश्च विहीनयुक्ताः~॥~१~॥}}
\vspace{1mm}

\phantomsection \label{3.2}
\textbf{{\color{purple}मेषादिषण्णामुदयाः स्वदेशे \\
तुलादितोऽमी च षडुत्क्रमस्थाः~। \\
तात्कालिकोऽर्कोऽयनभागयुक्त-\\
स्तद्भोग्यभागैरुदयो हतः स्वः~॥~२~॥}}
\vspace{1mm}

\phantomsection \label{3.3}
\textbf{{\color{purple}खाग्न्युद्धृतस्तं रविभोग्यकालं\\ 
विशोधयेदिष्टघटीपलेभ्यः~। \\
तदग्रतो राश्युदयांश्च शेषम् \\
अशुद्धहृत्खाग्निगुणं लवाद्यम्~॥~३~॥}}}
\end{quote}

\newpage

\phantomsection \label{3.4}
\begin{quote}
{\large \textbf{{\color{purple}अशुद्धपूर्वैर्भवनैरजाद्यै-\\
र्युक्तं तनुः स्यादयनांशहीनम्~।\\
भोग्याल्पकाले खगुणाहतोऽर्कः \\
स्वीयोदयाप्तांशयुतो विलग्नम्~॥~४~॥}}}
\end{quote}

\hyperref[3.1]{नागकुरङ्गदस्रा} \,अष्टसप्ताधिकद्विशतानि \,२७८ \,पलानि \,मीनमेषयोर्लङ्को-दयप्रमाणम्, \hyperref[3.1]{गोऽङ्काश्विनः} २९९ वृषकुम्भयोः, \hyperref[3.1]{रामरदाः} ३२३ मकरमिथुनयो-\hyperref[3.1]{र्विनाड्यः} पलानि, एवं क्रमेण कर्कादीनाम्, एते षडुत्क्रमेण तुलाद्याः~। अथ स्वदेशीयार्थमेते लङ्कोदयाः \hyperref[3.1]{क्रमोत्क्रमस्थाः} स्वैः प्रागुक्तप्रकारेण स्वदेशसाधि-तैश्चरखण्डैः क्रमोत्क्रमस्थैर्विहीनयुक्ता इत्यर्थः~। एवं कृते \hyperref[3.2]{मेषादिषण्णां} राशी-नामुदया \;भवन्ति, \;स्वदेशीया \;विलोमसंस्था \;\hyperref[3.2]{अमी} \;मेषोदयास्तुलादिषण्णां भवन्ति~॥ अथ श्लोकत्रयव्याख्या~। यातैष्यनाडीगुणितेऽत्यग्रे वक्ष्यमाणप्रका-रेण \,तात्कालिकं \,सूर्यं \,सायनांशं \,कृत्वा \,तस्य \,भागाद्यं \,यद्गतांशाद्यं \,त्रिंशद्भिः शोधितं \;भोग्यांशादिकं \;कृत्वा \;तेन \;स्वोदयो \;यस्य \;राशौ \;सूर्यस्तस्य \;राशेः स्वोदयो गुणनीयस्ततः खाग्न्युद्धृतो यल्लभ्यते तद्रवेर्भोग्यकालमुच्यते, तदिष्ट-घटीनां \;पलेभ्यो \;विशोधयेत्, \;तच्छेषादग्रस्थराश्युदया \;यावन्तः \;शुध्यन्ति तावतो विशोधयेत्, तच्छेषं खाग्निगुणं कृत्वा शुद्धोदयेन भजेत् फलमंशादि-ग्राह्यं \;ततोऽशुद्धराशेः \;पूर्वं \;यावन्तो \;मेषाद्या \;राशयस्तत्सङ्ख्यायां \;मूर्ध्नि \;युक्तं सद्राश्याद्यं लग्नं स्यात्, ततोऽयनांशहीनं कार्यम्, तत्स्पष्टलग्नं स्यात्, ग्रहाणां तात्कालिकीकरणे विशेष उच्यते\textendash \,षष्ट्या हता

\newpage

\noindent इत्यनेनोदयाद्गता गम्या वा घट्यः सावना ग्राह्याः, यदा ग्राह्यास्तदोदयासुभिः सूर्यभुक्तिर्गुण्या राशिकलाभि\textendash \,१८००\textendash \,र्भाजिता यदा स्वरूपे लब्धं तत्सहिता षष्टिर्हर \;इत्यर्थादवगन्तव्यम्, \;अथ \;लग्नार्थं \;तात्कालिकसूर्यकरणे \;विशेषः यदीष्टाः सावना घटिकास्तदा तात्कालिकत्वं कार्यं यदार्क्ष्यं तदा नेत्यर्थः~। उक्तं च {\color{violet}गोलाध्याये\textendash "लग्नार्थमिष्टघटिका यदि सावनास्तास्तात्कालिकार्ककरणेन भवेयुरार्क्ष्याः~। आर्क्ष्योदया ~हि ~सदृशीभ्य ~इहापनेयास्तात्कालिकत्वमथ ~न क्रियते यदार्क्ष्यः~॥~१~॥"}\\

अथ \;प्रकृतं \;भोग्याल्पकाल \;इति \;यद्यल्पेष्टकालस्तदैतदुक्तम् \;भवन्ति\textendash \;यदीष्टकालपलेभ्योऽर्कस्य \;भोग्यं \;न \;शुद्ध्यति \;तस्मिन्निष्टकाले \;खगुणाहते त्रिंशद्भिर्गुणिते ~सति ~तत्स्वोदयेनाप्तमंशाद्यं ~तद्युतोऽयनांशहीनोऽर्को ~लग्नं भवति, यथेष्टघट्यः ११ आभिः सूर्यस्पष्टा गतिः ५७।२८ गुणिता ६३२।८ षष्ट्या ६० भक्तं लब्धं कलादि १०।३२ अनेनौदयिकोऽर्कः १।३।६।१२ युक्तः १।३।१६। ४४ \;इष्टकाले \;जातस्तात्कालिकोऽर्कः \;१।३।१६।४४ \;अयनांशैः \;१८।१६।१० युक्तः १।२१।३२।५४ अस्य भागास्त्रिंशद्भ्यः ३० शुद्धा जाता भोग्यांशाः ८।२७। ६ स्वोदयो वृषस्तस्य पलानि २५५ भोग्यांशैर्गुणितानि २१।५५।१०।३० खाग्नि-भक्ते लब्धं रविभोग्यकालः

\newpage

\noindent ७१ इष्टघटी ११ पलेषु ६६० शुद्धः शेषम् ५८९ तदग्रिमराशेर्मिथुनस्य पलानि ३०५ शुद्धानि शेषम् २८४ त्रिंशद्गुणम् ८५२० अशुद्धकर्कराशिपलैः ३४१ भक्तं लब्धमंशादि २४।५९ अशुद्धः कर्कस्तस्य पूर्वं मिथुनस्तेन युतं राश्यादि ३।२४। ५९।७ \;अयनांशैर्हीनं \;३।६।४२।५७ \;जातं \;लग्नम्, \;अथाल्पेष्टकाले \;कल्पितम् इष्टम् १।० एतत्समयकः सूर्यः १।२१।२३।२२ भोग्यकालः ७३ इष्टकालपलेभ्यो ६० न शुध्यति तदा भोग्याल्पकालः ६० खगुणा ३० हतः १८०० स्वीयोदयपलै-र्वृषपलैः २२५ भक्तः लब्धमंशादि ७।३।३१। सूर्यः १।२१।२३।२२ लब्धभागैर्युतः १।२८।२६।५३ अयनांशैः १८।१६।१० हीनं जातमल्पेष्टकाललग्नम् १।१०।१०। ४३~॥~४~॥\\

{\small \textbf{अथ लग्नादिष्टकालानयनं जातकद्वयेनाह\textendash }}

\phantomsection \label{3.5}
\begin{quote}
{\large \textbf{{\color{purple}अर्कस्य भोग्यस्तनुभुक्तयुक्तो \\
मध्योदयाढ्यः समयो विलग्नात्~। \\
यदैकभे लग्नरवी तदैत-\\
द्भागान्तरघ्नोदयखाग्निभक्तः~॥~५~॥}} 
\vspace{1mm}

\phantomsection \label{3.6}
\textbf{{\color{purple}स्यादिष्टकालो यदि लग्नमूनं\\
शोध्यो द्युरात्रादथवा रजन्याः~। \\
रात्रीष्टकाले तु सषड्भसूर्या-\\
ल्लग्नं ततोऽयुक्तवदिष्टकालः~॥~६~॥}}}
\end{quote}

अर्कस्येति\textendash \,तद्भोग्यभागैरुदयो हतः स्व इत्यादिनानीतो रविभोग्यकाल-स्तथा सायनलग्नभुक्तांशैर्लग्नोदयो हतः खा-

\newpage

\noindent ग्न्युद्धृतो लग्नभुक्तकालः अनयोर्योगः अर्कयुतराशेर्लग्नपर्यन्तं ये मध्यस्था उद-यास्तैर्युतः एवं कृते लग्नादिष्टकालः स्यात् यथा लग्नं ३।६।४२।५७ सायन\textendash \,३।२४।५९।७\textendash \,मस्य \;भुक्तभागैः \;२४।५९।७ \;स्वोदयकर्कपलानि \;३४१ \;गुणि-तानि ८५१९।५८।४७ त्रिंशद्भक्ते लब्धं तनुभुक्तकालः २८३ अर्कभोग्य\textendash \,७१\textendash \,तनुभुक्तयोर्योगः \;३५४ \,रविलग्नमध्ये \;मिथुनं \,तस्य \,पलैर्युतः \;६५९ \,षष्ट्याभक्ते लब्धघटिकाः १०।५९ शेषत्यागात् पलोन एवमिष्टकालो जातः, \hyperref[3.5]{यदैकभ} इति यदा लग्नरवी एकराशिस्थितौ तदा तयोरंशविवरेणोदयो गुणितः खाग्निभक्तः आप्तोऽभीष्टकालः, यदैकराशौ सूर्याल्लग्नमूनं भवति तदा \hyperref[3.6]{द्युरात्रादि}ति षष्टिघ-टिकाभ्यः शोध्यः सूर्योदयात्कालः स्यादिति घटिका भवन्ति, अथवा रजन्याः शोध्यः \;इति \;मानाच्छोध्यः \;स \;सूर्यास्तादिष्टघटिका \;इति \;रात्रिगतघटिका भवन्ति, यथा सायनं लग्नम् १।२८।२६।५३ सायनोऽर्कः १।११।२३।२२ अनयो-रन्तरमधिकादूनः \;पात्य \;इत्यर्थः~। अन्तरम् \;१७।३।३१ \;उदयस्पष्टवृषपलै\textendash \,२५५\textendash \,र्गुणितम् १७९९।५६ त्रिंशद्भक्तं लब्धमिष्टकालपलानि ५९ अत्र लग्नम् अधिकं \;तेनायमेवेष्टकालः, \;यदा \;प्रष्टुः \;सावनघटिका \;इष्टास्तदा \;तात्कालि-कोऽर्कस्तदा तात्कालिकोऽर्कोऽसकृत्कर्तव्यः, यदार्क्ष्या इष्टास्तदौदयिकः सूर्यः सकृत् कालेन साध्यः, उक्तं

\newpage

\noindent चासकृत्कालश्चेत्सावनाः \;प्रष्टुरभीष्टनाड्यस्तदैव \;तात्कालिकलग्नम्, \;आर्क्ष्या यदेष्टघटिकाविलग्नमर्काच्च ~तत्रौदयिकात्सकृच्च~। अथानुक्तमपि ~ऋणलग्न-साधनमुच्यते\textendash ~~तात्कालिकसायनार्कभुक्तभागैः ~~स्वोदयो ~~हतस्त्रिंशद्भक्तो लब्धं ~~भुक्तकालस्तमिष्टकालपलेभ्यः ~~शोधयेच्छेषात्तत्पश्चादुदयाः ~~शोध्याः शेषं ~~त्रिंशद्गुणमशुद्धभक्तं ~~लब्धमंशादिकमशुद्धराशेः ~~शोध्यमयनांशहीनं राश्यादिलग्नं ~स्यात्, ~यथा ~रात्रिशेषघटी ~१० ~समयको ~रविः ~१।२।५६।३८ सायनः ~१।२१।१२।४८ ~अस्य ~भुक्तभागैः ~२१।१२।४८ ~स्वोदयपलानि ~२५५ गुणितानि ५४०९ त्रिंशद्भक्ते लब्धं रविभुक्तकालः १८० इष्टघटी १० पलेभ्यः ६०० विशोध्य शेषं ४२० शेषात् पश्चादुदयराशिर्मेषस्तत्पलानि २२३ शुद्धानि शेषं \;१९७ \;त्रिंशद्गुणम् \;५९१० \;अशुद्धैर्मीनपलैः \;२२३ \;लब्धमंशादि \;२६।३०।८ एतदशुद्धराशेर्मीनतः \;शुद्धम् \;११।३।२९।५२ \;अयनांशहीनं \;१०।१५।१३।४२ जातमृणं लग्नम्, अथ भुक्तकालादिष्टेऽल्पे प्रकारः, भुक्ताल्पकाले खगुणाह-तेऽर्कः स्वीयोदयाप्तांशहीनं लग्नं स्यात्, अयनांशहीनञ्चेति\textendash \,अस्मादिष्टकाला-नयनम्, तत्रार्कस्य भुक्तम् १८० उदयस्पष्टवृषपलैः २५५ गुणितम् १७९९।५६ त्रिंशद्भक्तं ~लब्धमिष्टकालपलानि ~५९~। अत्र ~लग्नमधिकं ~तेनायमेवेष्टकालः यदा प्रष्टुर्भुक्तं तनुभोग्यार्थं सायनलग्नस्य ११
 
\newpage

\noindent ।३।२९।५२ \,भोग्यांशाः \,२६।३।८ \,स्वोदयमीनपलैः \,२२३ \,गुणिताः \,५९०९।५९ त्रिंशद्भक्ते लब्धं १९७ तनुभोग्योऽर्कभुक्तः १८० अनयोर्योगः ३७७ लग्नात् सूर्य-पर्यन्तमन्तरे राशिर्मेषस्तस्य पलैः २२३ युतः ६०० षष्टिभक्ते लब्धा घट्यः १० एवं शेषरात्रिगतघट्यः १० इष्टागताः, अथेष्टकालः षष्टिभ्यः शोध्यः प्राग्दिन-स्योदयाद्गतघट्यः ५० जाताः, अथ यदा रात्रेरिष्टकालाल्लग्नं साध्यते तदा रविः सषड्भः \,कार्यस्तस्माल्लग्नं \,प्राग्वद्रात्रीष्टकाले \,साध्यं \,रात्रीष्टकालेऽपि \,सषड्भा-र्काच्च साधनीयम्~॥~६~॥\\

{\small \textbf{अथ दिनरात्रिप्रमाणं ततो नतोन्नतानि मालिन्याह\textendash }}

\phantomsection \label{3.7}
\begin{quote}
{\large \textbf{{\color{purple}चरपलयुतहीना नाडिकाः पञ्चचन्द्रा \\
द्युदलमथ निशार्धं याम्यगोले विलोमम्~। \\
द्युदलगतघटीनामन्तरं तन्नतं स्या-\\
न्नतरहितदिनार्द्धञ्चोन्नतञ्जायतेऽत्र~॥~७~॥}}}
\end{quote}

पञ्चचन्द्रनाडिकाः १५ द्विधा \hyperref[3.7]{चरपलैर्हीनयुताः} कार्या एकत्र हीनाः परत्र युताः सत्यो द्युदलं निशार्द्धञ्च क्रमेण भवति, यत्र युतास्तत्र द्युदलं यत्र हीना-स्तत्र रात्रिदलं तदुत्तरगोले मेषादिषड्राशिस्थितेऽर्क इत्यर्थः~। \hyperref[3.7]{याम्यगोले} तुला-दिषड्राशिस्थितेऽर्के विलोमं यत्र हीनास्तत्र दिनार्द्धं यत्र युक्तास्तत्र रात्र्यर्द्धं यथा\textendash \,चरपल\textendash \,८६\textendash \,घटीभिः १।२६ उत्तरगोलत्वात् पञ्चचन्द्राः १५।० घट्यो युता जातं दिनार्द्धं १६।२६ हीना जातं रात्र्यर्द्धम् १३।३४ एतद्रविसावनद्युदलं जातम्

\newpage

\noindent एवमन्यग्रहं सूर्यं प्रकल्प्य चरखण्डैश्चरं प्रसाध्य दिनमानं कार्यं तत् स्वसावन-दिनं भवति, सूर्यस्य नक्षत्रसावनं, दिनं विरूपयोगित्वाल्लिख्यते~। तथा च {\color{violet}सूर्य-सिद्धान्ते\textendash \,"ग्रहोदयप्राणहता \,खखाष्टैकोद्धृता \,गतिः~। चक्रासवो \;लब्धयुता स्वाहोरात्रासवः स्मृताः~॥"} इति षड्गुणितानि पलान्यसवो भवन्ति, वृषराश्यु-दयासुभिः १५३० रविगतिः ५७।२८ गुणिता ८७९२४ राशिकलाभि\textendash \,१८००\textendash \,र्भक्ता लब्धा असवः ४८ एतच्चतुर्थभागेन १२ षड्भक्ते पलरूपेण २ पञ्चदश\textendash \,१५\textendash \,युताः \;१५।२ \;पूर्वागतचरपलैः \;८६ \;उत्तरगोलत्वाद्युताः \;१६।२८ \;जातं नाक्षत्रसावनदिनार्द्धमेतत्त्रिंशद्\textendash \,३०\textendash \,घटिकाभ्यश्च्युतं \;जातं \;नाक्षत्रसावन-रात्र्यर्द्धं \;१३।२२ \;दिनार्द्धस्य \;दिनगतघटीनां \;च \;यदन्तरं \;तन्नतं \;स्यात्, \;गत-घटिका दिनार्द्धमध्येन पतन्ति तदा प्राङ्नतं, गतघटीषु दिनार्द्धञ्चेत्पतति तदा परनतमित्यर्थः~। तेन नतेन हीनं दिनार्द्धमुन्नतं भवति, अत्रास्मिन्कालसाधना-दिष्वेतन्नतं \;ज्ञेयम्~। अन्यत्र \;जातके \;त्वन्यथैव \;यथा \;दिनार्द्धम् \;१६।२६ \;इष्ट-घटीभिः ११ हीनञ्जातं प्राङ्नतं ५।२६ तेन नतेन ५।२६ हीनं दिनार्द्धं १६।२६ जात-मुन्नतम् ११।०~॥\\

{\small \textbf{अथेष्टकालाच्छायां छायाया इष्टकालं च द्रुतविलम्बितपञ्चकेनाह\textendash }}

\phantomsection \label{3.8}
\begin{quote}
{\large \textbf{{\color{purple}दिनदलं विशरं खहरो भवे-\\
न्नतकृतिः पृथगभ्रशराहता~। \\
खखनवाढ्यपृथक्स्थितया हृता \\
खहरतः पतितोऽभिमतो हरः~॥~८~॥}}
\vspace{1mm}

\textbf{{\color{purple}अथ नतं यदि पञ्चद-}}}
\end{quote}

\newpage

\phantomsection \label{3.9}
\begin{quote}
{\large \textbf{{\color{purple}शाधिकं \\
दिनदलात् पतितं स हरस्तदा~। \\
प्रथमखण्डहृतं दलितं चरं \\
स्वगुणितं स्वषडंशविवर्जितम्~॥~९~॥}}
\vspace{1mm}

\phantomsection \label{3.10}
\textbf{{\color{purple}दशयुतं पलकर्णहतं हृति-\\
र्हरहता श्रवणोऽङ्गुलपूर्वकः~। \\
रवियुतोनितकर्णहतेः पदं \\
द्युतिरिनद्युतिवर्गयुतेः श्रुतिः~॥~१०~॥}}
\vspace{1mm}

\phantomsection \label{3.11}
\textbf{{\color{purple}श्रुतिविभक्तहृतिस्तु हरो भवेत् \\
स पतितः खहरादवशेषकम्~। \\
पृथगिदं खखनन्दहतं हरात् \\
खविषयैरवशेषविवर्जितैः~॥~११~॥}}
\vspace{1mm}

\phantomsection \label{3.12}
\textbf{{\color{purple}फलपदं हि नतं यदि शेषकं \\
दिगधिकं हर एव तदोन्नतम्~। \\
इति कृतं लघु कार्मुकशिञ्जिनी-\\
ग्रहणकर्म विना द्युतिसाधनम्~॥~१२~॥}}}
\end{quote}

दिनार्द्धं \hyperref[3.8]{विशरं} पञ्चरहितं खहरो मध्याह्नकालीनो हरः स्यात्~। अथ \hyperref[3.8]{नत-कृतिः} नतेनैव गुणितन्नतं नतकृतिर्भवति सा पृथक् द्विधा स्थाप्या, एक\hyperref[3.8]{त्राभ्र-शरैराहता} पञ्चाशद्भिर्गुणनीया ततः \hyperref[3.8]{खखनवभि}र्नवशतैर्युक्तया द्वितीयस्थान-स्थितया \;नतकृत्या \;भाज्या \;लब्धं \;मध्याह्नकालीनहरात् \;पातयेत् \;शेषम् \;इष्ट-कालिको हरः स्यात्, यथा दिनार्द्धं १६।२६ पञ्च\textendash \,५\textendash \,हीनं ११२६ खहरोऽयम्, अथ नतस्य ५।२६ गोमूत्रिकया नतेनैव गुणनं नतकृतिः २९।३१ पृथक् २९।३१ एकत्राभ्रशरैः ५० गुणिता १४७५।५० अपरत्र खनवत्या ९०० युक्तया नतकृत्या ९२९।३१ भक्ता सवर्णितया ५५७७१ सवर्णितौ ८८५६ लब्धं घट्यादि १।३५ खहरतः ११।

\newpage

\noindent २६ पतितः ९।५१ अयमिष्टहरः~। अथेष्टच्छायाकर्णमाह\textendash \,अथ नतं यदि पञ्च-दशघटीभ्योऽधिकं स्यात्तदा नतेन हीनं दिनार्द्धमेवेष्टहरः स्यात्, यथा नतं १६ दिनार्द्धात् पतितम् ०।२६ अयमेवेष्टहरः प्राय उदय एव भवति, प्रथमखण्ड इति चरपलानि प्रथमखण्डेन भाज्यानि लब्धस्यार्द्धं वा अथवा चरपलानाम् अर्द्धं प्रथमचरखण्डेन भक्तं लब्धं तुल्यमेव, ततः स्वगुणितेनैव तदेव गुणितं स्वकीयेन \;षष्ठांशेन \;विवर्जितं \;दशभिर्युतं \;तत्पलकर्णेन \;गुणितं \;हृतिः \;स्यात्, अथाक्षकर्णानयनम्, अक्षप्रभा भुजः शङ्कुः कोटिः तद्वर्गपदं कर्णः स्यादितीष्ट-हरेण भक्तः लब्धमङ्गुलादिरिष्टकर्णो भवति, यथा चरपलानि ८६ प्रथमचर-खण्डेन ५५ भजेल्लब्धम् १।३३ अर्द्धम् ४६ अथवा चरपलानामर्द्धं ४३ प्रथम-चरखण्डेन हतं ४६ तुल्यमेव, अस्य स्वगुणितं वर्गः ३५ अयं स्वकीयेन षडंशेन ६ रहितं २९ दश\textendash \,१०\textendash \,युतं १०।२९ पलकर्णेन हतमक्षकर्णार्थं पलभा ५।३० भुजः शङ्कुः १२ कोटिः भुजवर्गः ३०।१५ कोटिवर्गः १४४ अनयोर्योगः १७४।१५ अस्य मूलम् १३।१३ अक्षकर्णः अनेन दशयुतं १०।२९ गुणितम् १३८।३३ अयं हृतिरिष्टहरेण \;भक्तेष्टकर्णोङ्गुलादि\textendash \,१४।५\textendash \,रवियुतोनितकर्णहतेः \;पदमिति कर्णो द्विस्थ एकत्र द्वादशयुतः २६।५ अन्यत्रोनः २।५३ उभयोर्घातस्य ५४।२० मूलम् ७।२३ इष्टघटीसमये द्वादशाङ्गुलशङ्कोर्द्युतिश्छाया जाता ७।२३ अथ

\newpage

\noindent छायात इष्टसमयज्ञानम्, इनद्युतिः श्रुतिविभक्त इति फलपदमिति इनो द्वादशं १२ द्युतिश्छाया तयोर्वर्गयोगस्तस्य मूलं कर्णः श्रुतिः शेषा व्याख्या सुगमा~॥ उदाहरणम्, इन\textendash \,१२\textendash \,वर्गः १४४ छायावर्गः ५४।३० अनयोर्योगस्य मूलं १४। ७ कर्णः, अनेन हृतिः पूर्वागता १३८।३२ भक्ता ९।४९ इष्टहरः ९।४९ खहरात् पतितः १।३७ इदमवशेषसंज्ञकं पृथक् खखनन्दैः ९०० गुणितं १४५५।० इदं द्वितीयस्थानस्थितेनावशेषेण १।३७ हीनैः खविषयैः ५० जातैः ४८।२३ भक्तं लब्धस्य ३०।४ मूलं ५।२६ नतं जातं पूर्वागतसमानं ५।२६ चेत् स्वल्पान्तरं तदा स्वल्पान्तरत्वान्न दोष एवं नतं ५।२६ दिनार्द्धात् १६।२६ शुद्धं जातम् इष्टकालः ११।१० अथेष्टहारेण हीनस्य खहरस्य यदि शेषकं दिगधिकं दश-भ्योऽधिकं स्यात्तदा हर एव नतं स्यात्, यथेष्टहरेण ०।२६ खहरो ११।२६ हीनः ११।० शेषकस्य दशाधिकत्वादिष्टहर एव ०।२६ उन्नतम्, अनेन हीनं दिनार्द्धं १६ नततामितिकृतः लघु इत्यमुना प्रकारेण कार्मुकधनुः शिञ्जिनी तयोर्ग्रह-कर्मसाधनप्रकारस्तेन विना लघुशीघ्रमिति स्वल्पकर्मणा छायासाधनं कृतम् इत्यर्थः~॥~१२~॥\\

{\small \textbf{अथ \,विषुवद्वृत्ताद्दक्षिणतः \,उत्तरतश्च \,यैरंशैः \,राशिप्रचारमार्गास्तेषाम् \,अंशानां क्रान्त्यंशसंज्ञां तत्परिज्ञानमिन्द्रवज्राद्वयेनाह\textendash }}

\newpage

\phantomsection \label{3.13}
\begin{quote}
{\large \textbf{{\color{purple}स्युः क्रान्तिखण्डानि यमाङ्गरामाः\\ 
क्वब्ध्यग्नयो गोनवबाहवश्च~। \\
षडश्विनः खेषुभुवो द्विबाणा \\
युक्तायनांशग्रहबाहुभागाः~॥~१३~॥}}
\vspace{1mm}

\phantomsection \label{3.14}
\textbf{{\color{purple}तिथ्युद्धृता लब्धमितानि तानि\\
योज्यानि भोग्याहृतशेषकस्य~। \\
तिथ्यंशकैः क्रान्तिकला भवन्ति\\
युक्तायनांशग्रहगोलदिक्काः~॥~१४~॥}}}
\end{quote}

स्युः क्रान्तीति\textendash \,\hyperref[3.13]{यमाङ्गरामा} द्विषष्ट्यधिकं शतत्रयं ३६२ \hyperref[3.13]{क्वब्ध्यग्नय} एक-चत्वारिंशदधिकं \;शतत्रयं \;३४१ \;\hyperref[3.13]{गोनवबाहवो} \;नवनवत्यधिकशतद्वयं \;२९९ \hyperref[3.13]{षडश्विनः} \;षट्-त्रिंशदधिकशतद्वयं \;२३६ \,\hyperref[3.13]{खेषुभुवः} \;सार्द्धशतं \;१५० \,\hyperref[3.13]{द्विबाणा} द्विपञ्चाशत् ५२ एतानि षट् खण्डकानि भवन्ति, अथ यस्य ग्रहस्य क्रान्तिश्चि-कीर्षिता सोऽयनांशैर्युक्तः कार्यस्तदीयभुजस्य येंऽशास्ते \hyperref[3.14]{तिथ्युद्धृताः} पञ्चदश-भिर्भक्ता \;लब्धसंज्ञकानि \;भुक्तखण्डानि \;योज्यानि \;तेषां \;भुक्तखण्डानां \;योग इत्यर्थः, ~ततो ~भोग्यखण्डगुणितशेषांशादेस्तिथ्यंशकेन ~योगो ~योज्यस्ताः क्रान्तिकलाः स्युः, युक्तायनांशो ग्रहो यादृशे गोले दक्षिणोत्तरौ तद्वशाद्या दिक् सा भवति सायनांशग्रहे दक्षिणगोले दक्षिणा क्रान्तिः, उत्तरगोलस्य उत्तरा क्रान्तिरित्यर्थः, उदाहरणम्\textendash \,यथेष्टकालिकः सूर्यः १।२१।३२।५४ अस्य भुजः १।२१।३२।५४ \;अस्यांशाः \;५१।३२।५४ \;पञ्चदशभक्ता \;लब्धखण्डानि \;३ \;एषां त्रयाणां

\newpage

\noindent खण्डानां योगः १००२ भोग्यखण्डेन २३६ शेषांशाः ६।३२।५४ गुणिताः १५४५। २४ तिथि\textendash \,१५\textendash \,भक्तं लब्धेन १०३।१ पूर्वखण्डयोगः १००२ युक्तः ११०५।५१ जाताः क्रान्तिकलाः षष्टि\textendash \,६०\textendash \,भक्ता अंशादिः १८।२५।१ सायनो रविरुत्तर-गोले तेनोत्तराः~॥~१४~॥\\

{\small \textbf{अथ खण्डकैर्विना क्रान्तिसाधनं भुजङ्गप्रयातेनाह \textendash }}

\phantomsection \label{3.15}
\begin{quote}
{\large \textbf{{\color{purple}भुजांशोननिघ्नाः खनागेन्दवस्त-\\
न्नगाश्वांशहीनैस्रिवेदाब्धिभिस्ते~। \\
कलाष्टादशोनैर्विभक्ता लवादि-\\
र्भवेत् क्रान्तिरेवं विना खण्डकैर्वा~॥~१५~॥}}}
\end{quote}

\hyperref[3.15]{खनागेन्दवो}ऽशीत्युत्तरशतं १८० सायनांशग्रहस्य \hyperref[3.15]{भुजांशोननिघ्नाः} कार्याः ग्रहस्य भुजांशैरूनाः कार्याः भुजांशैरेव गुणिताः कार्यास्ते पृथगनष्टाः स्थाप्या एकत्र तेभ्यः \hyperref[3.15]{नगाश्वांशैः} सप्तसप्ततिभिः ७७ भक्ते लब्धेन रहितैरष्टादशकलोनैः १८ त्रिवेदाब्धिभिरंशैः ४४३ त्रिचत्वारिंशदधिकैश्चतुःशतैः ४४२।४२ द्वितीय-स्थानस्थिताः \;भुजांशोननिघ्नाः \;खनागेन्दवो \;भाज्या \;लब्धमंशादिः \;क्रान्तिः स्यात्, पूर्वोक्तैः क्रान्तिखण्डैर्विना क्रान्तिर्भवतीत्यर्थः, उदाहरणं यथा\textendash \,सायन-सूर्यस्य \,१।२१।३२।५४ \,भुजांशैः \,५१।३२।५४ \,खनागेन्दवः \,१८० \,ऊनाः \,१२८। २७।६ एते भुजांशैरेव ५१।३२।५४ गुणिताः ६६२१।२८।९ द्विधा एकत्र सप्त-सप्ततिभिः ७७ भक्ता लब्धेन ८५।५९।३५ अष्टादशकलोनास्त्रिवेदाब्धयः ४४२। ४२ हीनाः

\newpage

\noindent ३५६।४२।२५ एभिरनष्टाः स्थापिताः ६६२१।२८।९ भक्ता लब्धमंशादिक्रान्ति-रुत्तरा १८।३३।४५ प्रकारान्तरत्वात् स्वरूपान्तरम्~॥~१४~॥\\

{\small \textbf{अथाक्षांशसाधनं भुजङ्गप्रयातेनाह\textendash }}

\phantomsection \label{3.16}
\begin{quote}
{\large \textbf{{\color{purple}दशाब्ध्यन्विताक्षप्रभाषष्टिभागोऽ-\\
क्षकर्णान्वितस्ते न भक्ता प्रभा सा~। \\
खनन्दाहता दक्षिणाः स्युः पलांशाः \\
पलः संस्कृतः क्रान्तिभागैर्नतांशाः~॥~१६~॥}}}
\end{quote}

\begin{center}
{\large \textbf{इतीह भास्करोदिते ग्रहागमे कुतूहले\\
विदग्धबुद्धिवल्लभे त्रिप्रश्नता स्फुटक्रिया~॥~३~॥}}
\end{center}

स्वीयदेशीयाक्षभा \hyperref[3.16]{दशाब्धि}भिर्दशाधिकचतुःशत्या ४१० युक्ता कार्या ततः षष्टिभिर्भाज्या लब्धेन स्वदेशीयोऽक्षकर्णो युक्तो भाजकः स्यात्, \hyperref[3.16]{खनन्दै}र्नव-तिभिर्गुणिताक्षभा भाजकेन भक्ता लब्धेन \hyperref[3.16]{पलांशाः} अक्षांशाः स्युस्ते लङ्काया उत्तरतः सदा दक्षिणा एव ते पलांशाः क्रान्तिभागैः संस्कृता भिन्नदिक्त्वेऽन्तरं समदिक्त्वे योगस्ते नतांशाः स्युः, यथाक्षभा ५।३० दशाब्ध्यन्विता ४१५।३० षष्टिभक्ता लब्धम् ६।५५ अक्षकर्णेन १३।१३ युक्तं २०।८ हरो जातः, खनन्दैः ९० गुणिता पलभा ४९५ हरेण २०।८ भक्ता लब्धमक्षांशाः २४।३५।९ दक्षिणा एभिरंशैर्ध्रुवः क्षितिजादुच्चः क्रान्तिभागा उत्तरा १८।३३।४५ भिन्नदिक्त्वाद-न्तरं जाताः नतांशाः ६।१।२४ अक्षांशाधिकत्वाद्दक्षिणाः~॥~१६~॥
\vspace{2mm}

\begin{center}
{\large \textbf{इति करणकुतूहलवृत्तौ त्रिप्रश्नाध्यायः समाप्तः~॥~३~॥}}
\end{center}

\afterpage{\fancyhead[RE,LO]{{\small{अ.\,४}}}}
\newpage

{\small \textbf{अथ चन्द्रग्रहणाधिकारो व्याख्यायते तत्रादौ परिपाटी लिख्यते\textendash}} \,संवत् १६७७ आषाढादिवर्षे शकः १५४२ मार्गशीर्षशुक्ला १५ पूर्णिमा बुधवासरे घट्यादिः ३८।२४ चन्द्रपर्वविलोकनार्थं श्रीब्रह्मतुल्योपरि गताब्दा ४३७ अधिमासा १६२ मासगणः ५४१४ अवमदिनानि २५४२ उदयेऽहर्गणः १५९८९३ अर्द्धगतियुक्ता औदयिकास्तेन तात्कालिका भवन्ति, अयनांशाः १८।१७४२ रामबीजकलाः ३४ सचन्द्रेषु सर्वेषु प्रसिद्धत्वात् कृतं रवेर्मन्दफलमृणम् ०। ३८।४ गतिफलं धनं \;२।१३ \,चन्द्रमन्दफलमृणं \;४।१९।५२ \,चन्द्रगतिफलं \;धनं \,३९ \;चरपलानि याम्यगोलत्वाद्धनानि \;चरपलान्यस्तकालत्वादृणम् \;६ \;अस्तात् \;पूर्णिमोत्थ-घट्यः ११।५२ दिनार्द्धं १३।११ दिनमानं २६।२२ रात्र्यर्द्धं १६।४९ रात्रिः ३३।३८ अथेदं \;दृष्टमात्रायां \,तिथौ \;ग्रहणस्य \;सम्भवासम्भवज्ञानमुच्यते \,{\color{violet}पर्वमालिनः\textendash \,"दर्शान्तमेकनाड्यूनं गतञ्चार्काह्निशेषकं~। पञ्चभ्यः पूर्णिमान्ते चेदधिकं तत्र पर्वणि"}~॥~१~॥\\

{\small \textbf{अथ युक्तमारभ्यते तमादौ नतकर्म द्रुतविलम्बितत्रयेणाह\textendash }}

\phantomsection \label{4.1}
\begin{quote}
{\large \textbf{{\color{purple}नतविहीनहतैः खगुणैर्हृताः\\
खशरभानुभुवो दशवर्जिताः~। \\
रविहरः सविधोर्विदशांशको \\
निजफलं निजहारहृतं क्रमात्~॥~१~॥}}
\vspace{1mm}

\phantomsection \label{4.2.1}
\textbf{{\color{purple}धनमृणं परपूर्वनते रवौ\\
शशिनि पूर्वनते स्वमृणे फले~। \\
इतरथोभयतोऽपि}}}
\end{quote}

\newpage

\phantomsection \label{4.2}
\begin{quote}
{\large \textbf{{\color{purple}फलक्षयः \\
स्फुटतरौ ग्रहणेऽथ ततस्तिथिः~॥~२~॥}}
\vspace{1mm}

\phantomsection \label{4.3}
\textbf{{\color{purple}इति नतं क्रममौर्विकयोदितं \\
क्रमजमेव हि जिष्णुजसम्मतम्~। \\
यदपरैः कृतमुत्क्रमजीवया \\
वलनदृङ्नतकर्म न तन्मतम्~॥~३~॥}}}
\end{quote}

द्युदलगतघटीनाम् \;इत्यादिनोक्तेनेति\textendash \;नतेन \;हीनैः \;पुनर्नतेनैव \;गुणितैः \hyperref[4.1]{खगुणै}स्त्रिंशद्भिः \hyperref[4.1]{खशरभानुभुवः} सार्द्धशतद्वयैकादशसहस्राणि भक्ते लब्धम् अंशादि दशभी रहितं रविहरः स्यात्, स एव रविहरः स्वदशांशेन रहितश्चन्द्र-हरः स्यात्, अथार्कमन्दफलं रविहरेण चेद्भक्तं तदा रविफलं कलादिकं रवि-नतफलं ५ स्यात् चन्द्रमन्दफलं चन्द्रहरेण भक्तं कलादिकं चन्द्रनतफलं स्यात्, रविनतफलं पश्चिमकपालस्थे रवौ धनं पूर्वकपालस्थेऽर्के ऋणम् अर्द्धरात्रा-न्मध्याह्नपर्यन्तं पश्चिमकपालमित्यर्थः, अथ चन्द्रग्रहणे तु रात्रिरेव दिनत्वेन व्यवह्रियत इति चन्द्रस्य वैपरीत्येन कपालव्यवस्था, मध्याह्नादर्धरात्रपर्यन्तं पूर्वकपालः, अर्द्धरात्रान्मध्याह्नपर्यन्तं पश्चिमकपाल इत्यर्थः, अथ चन्द्रे पूर्व-कपालस्थे चन्द्रस्य मन्दफले ऋणे सति नतफलं चन्द्रे धनमर्थात् पश्चिमनते चन्द्रे फले ऋणे सत्यृणम्\textendash \,इतरथा फले धने स\hyperref[4.2.1]{त्युभयतः} प्राक्कपालस्थे पर-कपालस्थे वा चन्द्रे नतफलमृणम्, एवमेतौ ग्रहणे स्फुटतरौ विधाय ताभ्याम् एवात्र तिथिः साध्या~। कैश्विदित्यत्रासकृदिदं कर्म कृतं परमनुक्तत्वान्नासकृत् क्रियते, तथा कैश्चिदपि सूर्याचन्द्रमसोर्भुक्तिरपि नतेन संस्कृता तदैवं

\newpage

\noindent रविगतिफलं रविहरेण भाज्यं लब्धेन रविचन्द्रविगतिरपि संस्कार्या, चन्द्र-गतिफलं \;चन्द्रहरेण \;भाज्यं \;लब्धं \;प्राक्कपालस्थे \;चन्द्रे \;गतिफले \;ऋणे \;भुक्तौ धनम्, अन्यथा ऋणमित्यर्थः~। एतन्नतकर्म सूक्ष्ममिच्छतान्यत्रापि तिथ्यानयने सूर्याचन्द्रमसोः पृथङ्नतं विधाय कर्तव्यम्, उक्तं च, {\color{violet}इदं ग्रहाणां नतकर्म युक्तं स्वल्पान्तरत्वान्न \;कृतं \;तदाद्यै}रिति \;ग्रहणदृग्गणितयोरेकत्वप्रयोजनायावश्यं कर्तव्यमिति भावः~। इति नतं क्रमज्यानतक्रमं सिद्धान्ते प्रोक्तम्, तदेवानेन प्रकारेण मयोक्तम्, खशरभानुभुव इत्यङ्कानयनं क्रमज्ययोत्पन्नमित्यर्थः~। ब्रह्म-गुप्ताचार्यस्य मते तदेव सम्मतम्, यत्कैश्चिद्वलननतदृक्कर्म उत्क्रमज्यया कथितं तदस्माकं न मतम्, चन्द्रमसो दिनं रात्रिरिति वचनात् सूर्यस्य रात्रिदलं चन्द्र-दिनार्द्धमिति, रात्रिदलम् १६।४९ इष्टषटी पूर्णिमाघटी ११।५१ उभयोरन्तरं ४।७७ \;प्राङ्नतमुन्नतं \;२५।३ \;सूर्यस्य \;नतार्थं \;रात्रिशेषे \;गते \;वा \;भवति \;समये चेज्जन्म तत्तद्घटीभिः संयुक्ते वासरार्द्धे खलु नतघटिकाः प्राक्-प्रतीच्यो भवेयु-रिति सूर्यस्य २५।३ पश्चिमेऽर्थाच्चन्द्रोन्नतमेव सूर्यस्य नतम्, नतेन ४।५७ हीनाः खगुणाः २५।३ नतेनैव ४।५७ गुणिताः १२४।० सूर्यनतेन २५।३ त्रिंशद्धीनाः ४।५७ सूर्यनतेनैव गुणिता १२४।० एवमेभिः खशरभानुभुवः ११२५० भक्ता लब्धमंशादि ९०।४३।३२ दशवर्जितं ८०।४३।३२

\newpage

\noindent रविहरोऽयं स्वदशांशेन ८।४।२१ ऊनो जातश्चन्द्रहरः ७२।२९।१२ रविमन्द-फलं ०।३८।४ रविहरेणाप्तमंशादि ०।०।२८ चन्द्रमन्दफलं ४।१९।५२ चन्द्र-हरेण लब्धं ०।३।३४ प्राक्कपालत्वात् फलस्यर्णत्वाद्धनम्, गतेः स्वल्पान्तर-त्वान्नतफलमुपेक्षितम्, नतफलसंस्कृतोऽर्कः ८।०।४।१७ चन्द्रः १।२७।३५। १८ आभ्यां तिथिरेष्या ११।३८~॥\\

{\small \textbf{अथ ग्रहाणां तात्कालिकत्वमिन्द्रवज्रयाह\textendash }}

\phantomsection \label{4.4}
\begin{quote}
{\large \textbf{{\color{purple}यातैष्यनाडीगुणिता द्युभुक्तिः \\
षष्ट्याहृता तद्रहितो युतश्च~। \\
तात्कालिकः स्यात्स्वचरः शशीनौ\\
पर्वान्त एवं समलिप्तकौ स्तः~॥~४~॥}}}
\end{quote}

यातैष्येति\textendash \,गतगम्येष्टघटीभिर्ग्रहस्य दिनभुक्तिर्गुण्या षष्ट्या भाज्या लब्धे-नौदयिको ग्रहस्तात्कालिकोऽन्यो वा गतफलेन रहितो गम्येन युतो वक्रिणि विपरीतमेवं तात्कालिकः स्पष्टौ चरफलेन संस्कृतौ, एवं तात्कालिकौ रवि-चन्द्रौ कृतौ समलिप्तकौ स्तः~। अत्र पर्वान्तघटीकरणे शेषत्यागो भवति, तेन चन्द्रो \;गतिबाहुल्यादल्पान्तरं \;भवति, \;अतो \;विभान्विन्दोरंशा \;द्वादशभि-र्भाज्याः शेषांशा हरात्त्याज्यास्तेषां कलास्ताभिश्चन्द्रसूर्ययोर्गती पृथक् भुक्त्य-न्तरेणाप्तेन कलाफलेन युतौ समन्वितौ समलिप्तिकौ स्तः~। यथैष्यघटीभिः ११।३८ रविगति\textendash \,६१।२१\textendash \,र्गुणिता ७१३।४२ षष्ट्या भक्ता लब्धेन

\newpage

\noindent कलादिना ११।५३ युतः, एवं गम्यघटीभिश्चन्द्रगतिः ८२९।३५ यातघटी ३।११ गुणिते षष्ट्या \;भक्ते \;कलादिफलेन \;युतौ \;गम्यत्वात् \;स्वल्पान्तरत्वाद्विकला-भेदेऽपि न दोषः~। अमावास्यायां रविचन्द्रौ राश्यादिसमौ पूर्णिमायां च षड्रा-श्यन्तरे लवादिसमौ तात्कालिकश्चन्द्रः २।०।१६।८ सूर्यः ८।०।१६।१० पातः ४।१।३६।१५~॥~४~॥\\

{\small \textbf{अथ शरसाधनमुपजात्याह\textendash }}

\phantomsection \label{4.5}
\begin{quote}
{\large \textbf{{\color{purple}सपाततात्कालिकचन्द्रदोर्ज्या \\
त्रिघ्नी कृताप्ता च शरोऽङ्गुलादिः~। \\
सपातशीतद्युतिगोलदिक्स्या-\\
न्मेषादिषड्भं खलु सौम्यगोलः~॥~५~॥}}}
\end{quote}

सपातेति\textendash \;तात्कालिकपातेन ~सहितस्य ~तात्कालिकचन्द्रस्य \;भुजज्या कार्या सा त्रिभिर्गुणिता \hyperref[4.5]{कृतै}श्चतुर्भिर्भक्ता लब्धमङ्गुलादिः शरः स्यात्~। सपात-चन्द्रस्य \;गोलवशाद्दिग्यस्य \;तादृशः \;सपातचन्द्रे \;सौम्यगोलस्थे \;सौम्यशरः याम्यगोलस्थे याम्यशरः सगोलः कथं दिगित्याह\textendash \,मेषादीति\textendash \,मेषादिराशि-षड्भं सौम्यगोलः, अपरं तुलादिषड्भं याम्यगोलः यथा पातः ४।१।३६।१५ चन्द्रः २।०।१६।८ संयुतः सपातचन्द्रः ६।१।५२।२३ भुजः ०।१।५२।२३ ज्या ३।५६ त्रिघ्नी \;११।४८ \;कृताप्ता \;२।५७ \;शरोऽङ्गुलादिः \;सपातचन्द्रो \;दक्षिणगोले \;तेन दक्षिणः~॥~५~॥\\

{\small \textbf{अथायनज्ञानं प्रकारान्तरेण शरानयनं सार्द्धेन्द्रवज्रयाह\textendash }}

\newpage

\phantomsection \label{4.6}
\begin{quote}
{\large \textbf{{\color{purple}याम्योऽपरं कर्कमृगादिषट्के \\
ते चायने दक्षिणसाम्यके स्तः~। \\
खाश्वाः ७० शराङ्गानि ६५ रसेषवो\textendash \,५६\textendash \,ऽग्नि-\\
वेदाश्च ४३ धिष्ण्यानि २७ खगाः ९ शरस्य~॥~६~॥\\
खण्डानि तैः क्रान्तिवदत्र साध्यो \\
बाणः कलादिस्त्रिहृतोऽङ्गुलादिः~।}}}
\end{quote}

मकरादिषट्कमुत्तरायणं कर्कादिषट्कं दक्षिणायनं भवति~। खाश्वाः ७० \hyperref[4.6]{शरा-ङ्गानि} \;पञ्चषष्टिः \;६५ \;\hyperref[4.6]{रसेषवः} \;षट्पञ्चाशत् \;५६ \;\hyperref[4.6]{अग्निवेदा}स्त्रिचत्वारिंशत् \;४३ \hyperref[4.6]{धिष्ण्यानि} \;सप्तविंशतिः \;२७ \;\hyperref[4.6]{खगा} \;नव \;९ \;एतानि \;शरस्य \;खण्डानि \;षट्, तैः ~\hyperref[4.6]{क्रान्तिवत्} ~कान्तिसाधनोक्तविधिना ~~बाहुभागास्तिथ्युद्धृता ~~इत्यादिना कलादिः शरो ~भवति त्रिभिर्विभक्तेऽङ्गुलादिः स्यात्~। अथ सपातचन्द्रः ६।१। ५२।२३ भुजः ०।१।५२।२३ अस्यांशाः १।५२।२३ पञ्चदशभिर्भागो न पतति तेन भुक्तशरखण्डाभावः, भोग्यप्रथमखण्डेन ७० भुजांशाः १।५२।२३ गुणिताः १३१।६ पञ्चदशभिर्भक्तं लब्धं कलादिः ८।४४ त्रिहृतोऽङ्गुलादिः २।५५ विधि-भेदादल्पान्तरः अङ्गुलादिशरस्यैवात्रोपयोगस्तस्माद्भुक्तमग्रे सर्वत्र कलादेरु-पयोगः~॥~६~॥\\

{\small \textbf{अथ चन्द्रादिबिम्बानामङ्गुलमानमिन्द्रवज्रार्द्धेनोपजात्या चाह\textendash }}

\newpage

\phantomsection \label{4.8}
\begin{quote}
{\large \textbf{{\color{purple}बिम्बं विधोः स्यात् स्वगतिर्युगाद्रि-\\
भक्ता रवेर्दस्रहता शिवाप्ता~॥~७~॥ \\
त्रिघ्नीन्दुभुक्तिस्तुरगाङ्गभक्ता \\
भूभार्कभुक्त्यद्रिलवेन हीना~। \\
राहुः कुभामण्डलगः शशाङ्कं \\
शशाङ्कगश्छादयतीनबिम्बम्~॥~८~॥}}}
\end{quote}

चन्द्रस्य ~स्फुटभुक्ति\hyperref[4.8]{र्युगाद्रि}भिश्चतुःसप्ततिभि\textendash \,७४\textendash \,र्भक्ता ~लब्धं ~चन्द्र-बिम्बाङ्गुलानि स्युः, रविस्पष्टा गति\hyperref[4.8]{र्दस्रा}भ्यां २ गुणिता \hyperref[4.8]{शिवै}रेकादशभि\textendash \,११\textendash \,र्भक्ता लब्धं रविबिम्बं स्यात्~। अथ चन्द्रभुक्तिस्त्रिगुणा \hyperref[4.8]{तुरगाङ्गैः} सप्तषष्टिभि\textendash \,६७\textendash \,र्भक्ता सूर्यभुक्तेः सप्तांशेन ७ हीना \hyperref[4.8]{भूभा} छायाबिम्बं स्यात्~। भूछायावद्वि-धुकक्षा \,तावद्वर्तते, \,यथा \,चन्द्रगतिः \,८।२९।३५ युगाद्रि\textendash \,७४\textendash \,भक्ता \,लब्धं चन्द्रबिम्बम् ११।१२ अङ्गुलादिः~। रविगतिः ६१।२१ दस्रहता १२२।४२ शिव\textendash \,११\textendash \,भक्ता \;लब्धं \;रविबिम्बं \;११।९ \;चन्द्रगतिः \;८२९।३ \;त्रिगुणा \;२४८८।४५ तुरगाङ्ग\textendash \,६७\textendash \,भक्ता लब्धं ३७।३२ रविभुक्तेः ६१।२१ सप्तमांशेन ८।४५ हीनं भूछायाबिम्बम् ~२८।२२~। ~स्वस्वयोजनबिम्बानयनम्~। ~स्वस्वपातकलाभिः कक्षा गुणिता चक्रकला २१६०० भक्ता लब्धं योजनात्मकं बिम्बं भवति~। एवं कृतरवेर्बिम्बं योजनात्मकं ६५२२ चन्द्रस्य ४८० एवं सर्वेषां यथास्थानं प्रदर्श-यिष्यामः~। अथ छादकमाह राहुर्भूभामण्डलगश्चन्द्रं छादयति, चन्द्रमण्डलगः सूर्यबिम्बमाच्छादयति~। अतश्चन्द्र-

\newpage

\noindent ग्रहे \;चन्द्रबिम्बं \;छाद्यं \;भूभा \;छादिका~। सूर्यग्रहणे \;सूर्यबिम्बं \;छाद्यं \;चन्द्रबिम्बं छादकः~॥~८~॥\\

{\small \textbf{अथ ग्रासप्रमाणमुपजात्याह\textendash }}

\phantomsection \label{4.9}
\begin{quote}
{\large \textbf{{\color{purple}यच्छाद्यसंछादकमण्डलैक्य-\\
खण्डं शरोनं स्थगितं तदाहुः~। \\छन्नं पुनश्च्छाद्यविवर्जितं तत् \\
खच्छन्नमेतन्निखिलग्रहे स्यात्~॥~९~॥}}}
\end{quote}

छाद्यच्छादकबिम्बमानयोर्योगस्यार्द्धं \;शरेण \;हीनं \;स्थगितं \;ग्रासप्रमाणं छन्नमित्यर्थः~। तच्चेच्छरोनं भवति तदा ग्रहणाभावः छन्नं छाद्यमानेन हीनं सद-ङ्गुलादिखच्छन्नं स्यात्, एतत्तुल्यमाकाशं बिम्बादुपरि छादयति, परमखच्छन्नं नवाङ्गुलासन्नम्, छन्नं विंशत्यङ्गुलासन्नम् एतत्सर्वग्रहणे सम्भवति~। यथा छाद्यं चन्द्रबिम्बं ११।१२ छादकं भूभाबिम्बम् २८।२२ उभयोरैक्यार्द्धं १९।४७ यदा शराभावस्तदा \;परममानैक्यखण्डमङ्गुलविंशत्यासन्नं \;१९।४७ \;शरेण \;२।५७ हीनं स्थगितं १६।५० छन्नं चैतच्छाद्यमानेन ११।१२ विवर्जितं जातं खच्छन्नम् ५।३८ अथ विंशोपकार्थं क्षेपकश्लोकः\textendash \,{\color{violet}"छन्नं नख\textendash \,२०\textendash \,गुणं कृत्वा छाद्यमा-नेन भाजितम्~।"} छन्नं १६।५० विंशति\textendash \,२०\textendash \,गुणं ३३६।४० छाद्येन ११।१२ भक्तं लब्धं विंशोपका भवन्तीति व्यवहारः ३०।३ परमविंशोपकाश्चत्वारिंश-दासन्ना भवन्ति~॥~९~॥

\newpage

{\small \textbf{अथ स्थितिविमर्दे भिन्नजातेरुपजात्याह\textendash }}

\phantomsection \label{4.10}
\begin{quote}
{\large \textbf{{\color{purple}द्विघ्नाच्छराच्छन्नयुताहतात् पदं \\
खाष्टेन्दुनिघ्नं विवरेण गत्योः~। \\
भक्तं स्थितिः स्याद्घटिकादिरेवं \\
खच्छन्नतो मर्दमपि प्रजायते~॥~१०~॥}}}
\end{quote}

पूर्वानीतः \,शरो \,द्वाभ्यां \,गुणनीयश्छन्नेन \,युतः \,पुनस्तच्छन्नेन \,गुणितस्त-स्मात् \,\hyperref[4.10]{पदं} \,मूलं \,ग्राह्यं \,तन्मूलं \,\hyperref[4.10]{खाष्टेन्दु}भिरशीत्युत्तरशतेन \,गुणितं \,चन्द्रसूर्य-स्फुटभुक्त्यन्तरेण \,भक्तं \,लब्धं \,घटिकादिस्थितिर्भवति, \,छन्नवत् \,खच्छन्नेनैव विमर्दं भवति, यतो भूभा सूर्यगत्या पूर्वतो यात्यतो रविगतिर्गृहीता~। यथा शरः २।५७ द्विघ्नः ५।५४ छन्नेन १६।५२ युतः २२।४४ छन्नेनैव गुणितः ३८२।४० मूलं १९।३४ खाष्टेन्दुभिः १।८० गुणितं ३५२२।० चन्द्रसूर्ययोर्गत्यन्तरेण ७६८।१४ भक्तं लब्धं ४।३५ मध्यस्थितिर्घटिकादिका, परमा स्थितिः पञ्चघटिकासन्ना, एवं खच्छन्नतो मर्दः, द्विगुणशरः ५।५४ खच्छन्नेन ५।३८ युतः ११।३२ खच्छ-न्नेन गुणितः ६४।५८ पदं ८ खाष्टेन्दुगुणं १४४० भुक्त्यन्तरेण ७६८।१४ भक्तं लब्धं मर्दघटिका १।५५ परमविमर्दः~। घटिकाद्वयासन्नः~॥~१०~॥\\

{\small \textbf{अथ ग्रहणमोक्षयोः स्थितिविमर्दानयनं पञ्चकालसाधनमिन्द्रवज्रात्रयेणाह\textendash }}

\phantomsection \label{4.11.1}
\begin{quote}
{\large \textbf{{\color{purple}विक्षेपतो नागयुगैर्विभक्ता \\
नाड्यादिकं यत् फलमत्र}}}
\end{quote}

\newpage

\phantomsection \label{4.11}
\begin{quote}
{\large \textbf{{\color{purple}लब्धम्~। \\
द्विष्ठा स्थितिस्तेन युता विहीना \\
स्यातां क्रमात् स्पार्शिकमोक्षके ते~॥~११~॥}}
\vspace{1mm}

\phantomsection \label{4.12}
\textbf{{\color{purple}ओजे पदे पातयुतो विधुश्चेत् \\
युग्मेऽन्यथैवं स्थितिवद्विमर्दे~। \\
सूर्योदयादस्तमयाच्च गम्यो \\
मध्यो ग्रहः पर्वविरामकाले~॥~१२~॥}}
\vspace{1mm}

\phantomsection \label{4.13}
\textbf{{\color{purple}स्थित्या विमर्देन च वर्जितेऽस्मिन् \\
स्तः स्पर्शसम्मीलनके क्रमेण~। \\
युक्तेऽथ तस्मिन् स्थितिमर्दकाभ्यां \\
मुक्तिस्तथोन्मीलनकं निजाभ्याम्~॥~१३~॥}}}
\end{quote}

\hyperref[4.11.1]{विक्षेपत} इति शरात् किंलक्षणान्नागयुगै\textendash \,४८\textendash \,र्विभक्ता नाड्यादिकं फलं लब्धं तेन फलेन द्विष्ठा स्थितिरेकत्र युतान्यत्र हीना सती क्रमेण स्पर्शमोक्षयोः स्थिती भवतः, अत्रैतस्मिन्नेव चन्द्रग्रहणे न तु सूर्यग्रहणे चेद्यदि पातयुतो विधु-रोजपदे स्यात्तदैवम्, अथ यदि सपातचन्द्रो युग्मे पदे स्यात्तदा फलयुता मोक्ष-स्थितिः \,फलहीना \,स्पर्शस्थितिर्भवति \,एवं \,स्थितिवद्विमर्देऽपि \,साध्ये, \,अत्र भास्थे \,स्पर्शमोक्षशरादेतत् \,कर्म \,साधितम्, \,तद्वचनं \,च \,विक्षेपत \,इति मध्य-विक्षेपान्न \,भवति, \,स्वस्वविक्षेपादिति \,ज्ञेयम्, \,मध्यविक्षेपादेतत् \,कर्म \,कर्तुं \,न युज्यते यतो मध्यस्थितयोः सममन्तरं न भवति~। शरस्यान्यदिक्त्वात्, तस्मात् स्वमौक्षिके ते~॥~११~॥ स्वविक्षेपादिति सम्भवतीति मया तु वृद्धसम्प्रदायमनु-सृत्योदाह्रियते~। यथा मध्य-

\newpage

\noindent शरः २।५७ नागयुगै\textendash \,४८\textendash \,र्भक्तः लब्धेन ०।३ मध्यस्थितिः ४।३५ द्विष्ठा ४।३५ सपातचन्द्रो विषमे पदे तेन युता सती स्पर्शस्थितिः ४।३८ हीना सती मोक्ष-स्थितिः ४।३२ एवं लब्धफलेन ०।३ मध्यविमर्दः १।५५ पुनः सन् १।५८ स्पर्श-विमर्दः, हीनः सन् १।५२ मोक्षमर्दः~। सूर्योदयादिति\textendash {सूर्यग्रहणे} सूर्योदयाच्चन्द्र-ग्रहणे सूर्यास्तमयाद्भावि, ऐष्ये पर्वावसानकाले दर्शपूर्णमास्यन्ते स्फुटमध्य-ग्रहा \,स्यात्, \,चन्द्रग्रहणे \,स \,मध्यग्रहः \,पूर्णमास्यन्तः \,सूर्यग्रहणे \,स \,एव \,लम्बन-संस्कृतः \,स्फुटदर्शान्तः, \,अथ \,स्पार्शिकस्थित्या \,स्पार्शिकविमर्देन \,च \,वर्जिते हीने पर्वान्ते तिथ्यन्ते यथाक्रमं स्पर्शं सम्मीलनकं च स्यात्, अथ निजाभ्यां स्वीयाभ्यां मौक्षिकाभ्यां युक्तेऽस्मिन् पर्वान्तकाले क्रमेण मोक्षसमुन्मीलनकं च स्यात्, उभयोश्छाद्यच्छादकयोर्मण्डले सम्पर्कः सम्मीलनमिति, छायावृत्ता-खिलग्रसनं छाद्यबिम्बादर्शनमित्यर्थः~। मध्यग्रहमिति यावत्, शराङ्गुलमानेन छादनच्छादनं छन्नमित्यर्थः~। तन्मध्यग्रहणमेतस्मादधिकं न छाद्यते इत्युन्मी-लनमिति, \;मध्यग्रहणान्मुच्यमाने \;छाद्यबिम्बस्पर्शनमात्रं \;मोक्ष \;इत्युभयोर्बि-म्बयोः पृथग्भावः छाद्यबिम्बाधिको यावद्ग्रासः, स्पर्शस्थित्या ४।३८ पर्वान्त-कालं ११।३८ हीनो जातः स्पर्शकालः ७।० स्पर्शमर्देन १।५८ रहितो जातः सम्मीलनकालः ९।४० तिथ्यन्तः ११।३८ मोक्ष-

\newpage

\noindent स्थित्या ४।३२ युक्तो जातो मोक्षकालः १६।१० स एव पर्वान्तकालः ११।३८ मोक्षमर्देन १।५२ युक्तो जात उन्मीलनकालः १३।३० स्पर्शकालः ७।० मोक्ष-कालः ~~१६।१० ~~अनयोर्द्वयोरन्तरं ~~९।१० ~~जातस्पर्शमोक्षयोरन्तरकालः घट्यादिः पुण्यकालग्रहस्थितिः~॥~१३~॥\\

{\small \textbf{अथ वलनानयनमुपजातीन्द्रवज्रोपजात्याह\textendash }}

\phantomsection \label{4.14}
\begin{quote}
{\large \textbf{{\color{purple}खाङ्का\textendash \,९०\textendash \,हतं स्वद्युदलेन भक्तं \\
स्पर्शे विमुक्तौ च नतं लवाः स्युः~।\\ 
तज्ज्याहताश्चाक्षलवा विभक्ता-\\
स्त्रिभज्यया प्रागपरे नते स्यात्~॥~१४~॥}}
\vspace{1mm}

\phantomsection \label{4.15}
\textbf{{\color{purple}सौम्यान्तकाशा वलनं ग्रहस्य \\
युक्तायनांशस्य तु कोटिजीवा~। \\
बाणैर्विभक्तायनदिक्तथान्य-\\
द्भागाद्यमेकान्यदिशोस्तयोस्तु~॥~१५~॥}}
\vspace{1mm}

\phantomsection \label{4.16}
\textbf{{\color{purple}योगान्तरज्याहतमानयोग-\\
खण्डं त्रिभज्याहृतमङ्गुलाद्यम्~। \\
स्फुटं भवेत्तद्वलनं रवीन्द्वोः \\
प्राग्ग्रासमोक्षे विपरीतदिक्के~॥~१६~॥}}}
\end{quote}

स्पर्शविमुक्ते \,मोक्षे \,च \,यन्नतं \,तत् \,\hyperref[4.14]{खाङ्कै}र्नवतिभिः \,९० \,स्वदिनार्द्धेन \,चन्द्र-ग्रहणे रात्र्यर्द्धेन रविग्रहणे दिनार्द्धेन भक्तं नतांशाः स्युः, तेषां नतांशानां ज्यया गुणिताः \;स्वदेशाक्षांशास्त्रिज्यया \;विंशत्युत्तरशतेन \;१२० \;भक्ता \;लब्धं \;स्पर्श-मोक्षवलनमंशादिः स्यात्, पूर्वनते प्राक्कपाले सौम्यं वलनम्, अपरनते प्रत्यक्क-पालेऽन्तकाशं ~याम्यं ~वलनम्~। अयनांशयुक्तस्य ~तात्कालिकग्रहस्य ~रवेश्च-न्द्रस्य वा कोटिज्या \hyperref[4.15]{बाणैः} पञ्चभिर्भक्ता

\newpage

\noindent लब्धमंशाद्यमायनं वलनं स्यात्, तच्चायनदिक्, यदि सायनो ग्रहः सौम्येऽयने तदा \,सौम्यम्, \,यदि \,याम्येऽयने \,तदा \,याम्यम्, \,तयोरक्षवलनायनवलनयोरेक-दिशि योगो भिन्नदिशि वियोगः कार्यः तस्य योगस्यान्तरस्य च या ज्या तया गुणितं छाद्यछादकमानैक्यार्द्धं त्रिभज्यया भक्तं स्फुटं वलनं भवति, योगपक्षे सैव \;दिक्, \;अन्तरपक्षेऽधिकवलनसम्बधिनी \;भवति~। तच्च \;\hyperref[4.16]{वलनं \;रवीन्द्वोः प्राग्ग्रासमोक्षे विपरीतदिक्कम्} इति, प्राग्ग्रासं स्पर्शवलनं रवेर्ग्रहणे विपरीतं देयं याम्यं चेत् सौम्यं ज्ञेयम्, सौम्यं चेद्याम्यमिति, इन्दोर्ग्रहणे मोक्षवलनं विपरीतं याम्यश्चेत्तदा \;सौम्यम्, \;सौम्यं \;चेत्तदा \;याम्यम्~। विशेषश्चात्र \;खमध्ये \;पाताले चाक्षजवलनाभावः, वलनं चतुर्विंशत्यंशप्रमाणम्, कर्कादौ मकरादौ च वल-नाभावः सदोह्यम्, परमं स्पष्टवलनमङ्गुलपञ्चदशासन्नम्~। यत्र षट्षष्टिपलां-शास्तत्र परमं स्पष्टवलनं विंशत्यङ्गुलमुक्तप्रकारेण भवतीति ज्ञेयम्~। इह सम-मण्डलं द्रष्टुः प्राची, सममण्डलादिष्टे नते काले विषुवन्मण्डलप्राची यावतायन-श्चलति तावत्तद्दिक्कपालोद्भवं ज्ञेयम्~। अथ विषुवन्मण्डलात्क्रान्तिमण्डलप्राचीं यावतायनश्चलति तदायनं तद्दिग्ज्ञेयम्~। तयोर्योगवियोगात् स्फुटमिति~। सम-मण्डलात् क्रान्तिमण्डलं यावतायनश्चलति तत्स्फुटं वलनमिति~। अथ ग्रस्तो-दये ग्रस्तास्ते स्पर्शमोक्षवलनानयनमुच्यते रात्रेः शेषे गते वा इत्यादिना जात-कवचनान्नतमानीय खाङ्कहतं स्वदिनार्द्धेन भक्तं लम्बनन-

\newpage

\noindent तांशास्तेषां भुजं कृत्वा ज्या साध्या ततः प्राग्वदक्षवलनमानेयम्~। उक्तं च {\color{violet}पर्वमालायाम्\textendash \,स्वरात्रौ \;शेषजा \;नाड्यो \;द्युदलं \;स्पर्शमोक्षयोः~। नतं \;स्यात् स्वदिनं \,प्राग्वत् \,तन्नाड्यः \,षडुणा \,लवाः~॥~१~॥ तद्भुजज्या \,पलांशघ्नी \,त्रिज्या-भक्ता लवादिकम्~। वलनं स्यादुदग्याम्यं ग्रहणं प्राक्परस्थिते~॥~२~॥} पुनरुक्तम्, {\color{violet}रात्रेः शेषघटीयुक्तं दिनार्द्धं प्राङ्नतं रवेः~। रात्रेर्गतघटीयुक्तं दिनार्द्धं प्रत्यङ्नतं रवेः~॥~१~॥ दिनशेषघटीयुक्तं निशार्द्धं प्राङ्नतं भवेत्~। सूर्योदयाद्युङ्निशार्द्धं प्रत्यगिन्दोर्नतं मतम्~॥~२~॥} ततः खाङ्काहतमित्यादिकार्यम्~। ब्रह्मतुल्यभाष्ये तु ग्रस्तोदये स्पार्शिकनतं ग्रस्तास्ते मौक्षिकं नतम्~। स्वदिनार्द्धाद्यावदधिकं भवति ~तस्य ~स्वदिनार्द्धवशादंशान्प्रसाध्य ~नवते\textendash \,९०\textendash \,र्विशोध्य ~तज्ज्यां कृत्वाक्षांशैः सङ्गुण्य त्रिज्यया भाज्यं तदक्षजं वलनं स्यात्~। ग्रस्तोदये सौम्यं ग्रस्तास्ते ~दक्षिणमिति~। एतदुदाहरणं ~यथास्थानं ~दर्शयिष्यामः~। अस्तात् स्पर्शघटी ७ चन्द्रदिनार्द्धम् १६।४९ अनयोरन्तरं स्पर्शनतं ९।४९ प्राक् खाङ्का-हतं ८८३।३० स्वदिनार्द्धेन १६।४९ भक्तं लब्धं नतांशाः ५२।३२।१३ एषां ज्या ९५।२ अनयाक्षांशाः २४।३५।९ गुणिताः २३३६।२८।५८ त्रिज्यया १२० भक्ते लब्धमाक्षजवलनमंशादिः १९।२८।१४ प्राङ्नतत्वादुत्तरम्, अथायनांशाः स्पर्श-कालः ७ यात्येष्यनाडीत्यादिना स्पार्शिकश्चन्द्रः १।२९।११।३८ सायनः २।१७। १९।२० कोटिः ०।१२।४।४० ज्या २६।१ बाणैर्भक्ता लब्धमायनं वलनम्

\newpage

\noindent ५।१३ \,सौम्यायनत्वात् \,सौम्यम्, \,आसन्नाक्षयोरेकदिक्त्वाद्योगः \,२४।४०।४० अस्य ज्या ४९।५२ अनया चन्द्रभूभामानयोगखण्डं १९।४७ हतं ९८६।३१।४४ त्रिज्यया १२० भक्तं लब्धं स्पष्टवलनमङ्गुलादिः ८।१३।९ मोक्षकालघटी १६। १० चन्द्रदिनार्द्धयोरन्तरं प्राङ्नतं ०। ३९ खाङ्कै\textendash \,९०\textendash \,र्गुणितं ५८।३० चन्द्रदि-नार्द्धेन १६।४९ भक्तं लब्धं नतांशाः ३।२८।४३ एषां ज्या ७।१८ अनयाक्षांशाः २४।३५।९ गुणिताः १७९।२८ त्रिभज्यया भक्ते लब्धं १।२९।४४ प्राङ्नतत्वादुत्त-रम्~। अथायनम्, मोक्षकालः १६।१० तात्कालिकश्चन्द्रः २।१।१८।४४ सायनः २।१९।२६।२६ कोटिः ०।१०।३३।३४ ज्या २१।४७ बाणै\textendash \,५\textendash \,र्भक्ते लब्धम् ४। २१।२४ उत्तरायणत्वादुत्तरम्~। अक्षजायनयोरेकदिक्त्वाद्योगः ५।५१।८ अस्य ज्या \,१२।१७ \,अनया \,मानयोगखण्डं \,१९।४७ \,गुणितं \,२५६।११ \,त्रिज्यया \,१२० भक्ते लब्धं स्फुटवलनं २।८।५ सौम्यम्~॥~१६~॥\\

{\small \textbf{अथ स्पार्शिकमौक्षिकशराविन्द्रवज्रयाह\textendash }}

\phantomsection \label{4.17}
\begin{quote}
{\large \textbf{{\color{purple}माध्यः शरस्त्वोजपदोद्भवश्चेत् \\
स्थित्यग्नि\textendash \,३\textendash \,भागोनयुतो युतोनः~। \\
युग्मे विधोर्वा प्रथमान्त्यबाणौ \\
चन्दग्रहे व्यस्तदिशः शराः स्युः~॥~१७~॥}}}
\end{quote}

\hyperref[4.17]{माध्य} इति मध्यग्रहणकालिकः शरो यदि विषमपदस्थसपातचन्द्रादुत्प-न्नस्तदा मध्यस्थितितृतीयभागे\hyperref[4.17]{नोनो} रहितः

\newpage
\noindent सन् \;स्पर्शकालिकः \;शरो \;भवति~। स्थिते \;तृतीयभागेन \;युतो \;मध्यकालीनः शरोऽन्त्यकालीनो मोक्षकालीनः शरो भवति~। युग्मपदोद्भवश्चेत्तदा स्थिति-तृतीयभागेन युतः सन् स्पार्शिकः, हीनः सन् मौक्षिकः शरः स्यात्~। एवं विधो-श्चन्द्रग्रहणे न तु सूर्यग्रहणे~। \hyperref[4.17]{वा} अथवा विधोस्तात्कालिकात् सपातचन्द्रात् प्रथमान्त्यबाणौ ~साध्यौ, ते ~स्पर्शमध्यमोक्षशराश्चन्द्रग्रहणे ~परिलेखकर्मणि विपरीतदिशो ज्ञेया नान्यत्र~। उक्तं च, {\color{violet}"नित्यशोऽर्कस्य विक्षेपा परिलेखे यथा-क्रमम्~। विपरीतं शशाङ्ककस्य"} इति~। यथा मध्यस्थितिः ४।३५ तृतीयभागेन १।३१ ओजपदत्वान्माध्यः शरः २।५७ हीनो जातः स्पर्शशरः १।२६ मध्यशरः २।५७ स्थितितृतीयभागेन १।३१ युतो जातो मोक्षशरः ४।२८ इदं कर्म माध्य-शरस्यैव~। अथ प्रकारान्तरेण स्पार्शिकश्चन्द्रः १।२९।११।३८ तात्कालिकपा-तेन \,४।१।३५।५३ \,युतः \,६।०।४७।३१ \,भुजः \,०।०।४७।३१ \,ज्या \,१।३९ \,त्रिघ्नी ४।५७ कृताप्ता शरो जातो याम्यः १।१४ स्पर्शकालीनः मोक्षकालिकः पातः ४।१।३६।२१ चन्द्रेण २।१।१८।४४ युक्तः ६।२।५५।५ भुजः ०।२।५५।५ ज्या ६।७ त्रिघ्नी १८।२१ कृताप्ता मोक्षशरो दक्षिणः ४।३५।० प्रकारान्तरभेदादल्पा-न्तरम्~॥~१७~॥ \\

{\small \textbf{अथ परिलेखमिन्द्रवज्राद्वयेनाह\textendash }}

\phantomsection \label{4.18}
\begin{quote}
{\large \textbf{{\color{purple}ग्राह्यार्द्धसूत्रेण विधाय वृत्तं \\
मानैक्यखण्डेन च साधिताशम्~।}}}
\end{quote}

\newpage

\phantomsection \label{4.19}
\begin{quote}
{\large \textbf{{\color{purple}बाह्येऽत्र वृत्ते वलनं यथाशं \\
प्राक्स्पार्शिकं पश्चिमतश्च मोक्षम्~॥~१८~॥}}
\vspace{1mm}

\textbf{{\color{purple}देयं रवेः पश्चिमपूर्वतस्ते \\
ज्यावच्च बाणौ वलनाग्रकाभ्याम्~। \\
उत्पाद्य मत्स्यं वलनाग्रकाभ्यां \\
माध्यः शरस्तन्मुखपुच्छसूत्रे~॥~१९~॥}}}
\end{quote}

समे भूतले पदादौ वा \hyperref[4.18]{ग्राह्य}स्य चन्द्रग्रहणे चन्द्रस्य रविग्रहणे रवेः बिम्ब-मानाङ्गुलस्यार्द्धेनाभीष्टस्थानकल्पितबिन्दोर्मण्डलं ~कृत्वा ~ग्राह्यग्राहकमान-योगार्द्धप्रमाणेन कर्काटकेन सूत्रेण वान्यद्वृत्तं तस्मादेव बिन्दोः कृत्वा तस्मा-द्बिदोरुपरि पूर्वापरं तथा दक्षिणोत्तरं रेखाद्वयं कर्तव्यम्~। एवं साधितदिक्के बाह्येऽत्र वृत्ते वलनं देयं तच्च \hyperref[4.19]{यथाशं} दक्षिणस्यां याम्यं सौम्यमुत्तरस्याम्, प्राग्ग्रा-समोक्षे विपरीतदिक्त्वम् इति पूर्वं सम्प्रसार्य्य तत्रापि चन्द्रग्रहणे स्पार्शिकं वलनं प्राचीचिन्हात्, मौक्षिकं पश्चिमचिह्नान्नेयम्, रविग्रहणे तु स्पार्शिकं पश्चि-मतो मौक्षिकं पूर्वतः, ततश्चन्द्रग्रहे व्यस्तदिशः शराः स्युरिति पूर्वं सम्प्रसार्य्य स्पार्शिकमौक्षिकवलनाग्रचिह्नाभ्यां स्पार्शिकशराङ्गुलपरिमिते शलाके क्रमेण स्वदिगभिमुखे ज्यारूपेण देयः~। स्पर्शवलनाग्रात् स्पर्शशरो देयः~। मौक्षिक-वलनाग्रान्मौक्षिकशरो देयो ज्यावत्~। एवं धनुराकारे बाह्यवृत्ते रेखाप्रदेशे शर-द्वयाग्रं चिह्नं विधाय मध्यशरार्थमाह\textendash \,उत्पाद्येति\textendash \,वलनद्विकान्तरमितादिकं सूत्रस्यैकाग्रं स्पर्शवलनस्योपरि धृत्वा तेन बिम्बार्द्धं कुर्यात्,

\newpage

\noindent अथ मौक्षिकवलनाग्रात्तेनैव सूत्रेण बिम्बार्द्धं कार्यं तयोर्बिम्बार्द्धयोर्यत्र सङ्गम-स्तत्र मुखपुच्छे प्रकल्प्ये~। अथ वलनाग्रकाभ्यां समतुल्यप्रमाणेन कर्काटकेन वृत्तद्वये कृते वृत्तयोः समासे मत्स्याकार उत्पद्यते तन्मध्यसूत्रे तन्मुखपुच्छ-सूत्रमिति, तस्य मुखात् केन्द्रव्यापिनीं पुच्छपर्यन्तं रेखां कृत्वा~॥~१९~॥\\

{\small \textbf{अथ स्पर्शादिस्थानमुपजात्याह\textendash }}

\phantomsection \label{4.20}
\begin{quote}
{\large \textbf{{\color{purple}केन्द्राद्यथाशं स्वशराग्रकेभ्यो \\
वृत्तैः कृतेर्ग्राहकखण्डकेन~। \\
स्युः स्पर्शमध्यग्रहमोक्षसंस्था \\
अथाङ्कयेन्मध्यशराग्रचिन्हात्~॥~२०~॥ }}}
\end{quote}

केन्द्रादिति\textendash \;तस्मात् ~\hyperref[4.20]{केन्द्रान्म}ध्यस्थितबिन्दोर्मध्यशरः ~स्वदिगभिमुखो देयः, एवं शरत्रयाग्रं चिह्नयित्वा रविग्रहे ग्राहकस्य चन्द्रस्य खण्डकेन चन्द्रग्र-हणे भूभायाः खण्डकेन मानार्द्धेन तत्केन्द्रादिति\textendash \,कर्काटकेन स्पर्शशरचिह्ना-द्वृत्तं कुर्यात्~। तदभ्यन्तरे ग्राह्यवृत्ते यत्र स्पृशति तत्र स्पर्शो ज्ञेयः, एवं मौक्षिक-शराग्रकृतवृत्तसम्पर्कान्मोक्षस्थानंं ~ज्ञेयम्, ~मध्यशराग्रकृतवशान्मध्यग्रहण-संस्थानं ज्ञेयम्, तत्र यदा मध्यग्रासो ग्राह्यबिम्बमुल्लङ्घ्य यावद्बहिः पतति ताव-दाकाशं गृह्यते तत्र सर्वग्रहणं ज्ञेयम्, यदा तु ग्राह्यबिम्बैकदेशे गृह्णाति तदा तावदेव खण्डग्रहणम्, यदा ग्राह्यं न स्पृशति तदा ग्रहणाभावः~॥~२०~॥ \\

{\small \textbf{अथ सर्वग्रहणोपयोगमिष्टग्रासं चेन्द्रवज्रात्रयेणाह\textendash }}

\newpage

\phantomsection \label{4.21}
\begin{quote}
{\large \textbf{{\color{purple}आद्यन्त्यबाणाग्रगते च रेखे \\
ज्ञेयाविमौ प्रग्रहमुक्तिमार्गौ~। \\
मानान्तरार्द्धेन विलिख्य वृत्तं \\
केन्द्रेऽथ तन्मार्गयुतद्वयेऽपि~॥~२१~॥}}
\vspace{1mm}

\phantomsection \label{4.22}
\textbf{{\color{purple}भूभार्द्धसूत्रेण विधाय वृत्ते \\
सम्मीलनोन्मीलनके च वेद्ये~। \\
मार्गप्रमाणे विगणय्य पूर्वं \\
मार्गाङ्गुलघ्नं स्थितिभक्तमिष्टम्~॥~२२~॥}}
\vspace{1mm}

\phantomsection \label{4.23}
\textbf{{\color{purple}इष्टाङ्गुलानि स्युरथ स्वमार्गे \\
दद्यादमृनिष्टवशात्तदग्रे~।\\
वृत्ते कृते ग्राहकखण्डकेन \\
स्यादिष्टकाले ग्रहणस्य संस्था~॥~२३~॥}}}
\end{quote}

\begin{center}
{\large \textbf{इतीह भास्करोदिते ग्रहागमे कुतूहले \\
विदग्धबुद्धिवल्लभे शशाङ्कपर्वसाधनम्~॥~४~॥}}
\end{center}

अथ मध्यशराग्रबिन्दोः स्पर्शाग्रपर्यन्तं रेखां लिखेत् स ग्राहकस्य ग्रहण-मार्गः, यतो मध्यशराग्रचिह्नादेव मोक्षशराग्रपर्यन्तं रेखां कुर्यात् स मोक्षमार्गः, शरत्रयं \;स्पर्शिनी \;सा \;रेखा \;धनुराकारा \;भवति, \;\hyperref[4.21]{मानान्तरार्द्धेने}ति\textendash \;ग्राह्य-ग्राहकमानयोरन्तरस्य यदर्द्धं तत्प्रमितेन कर्काटकेन केन्द्रे वृत्तं कुर्यात्, तस्य \;वृत्तस्य \;पूर्वविहितग्रहमार्गरेखायाश्च \;यत्र \;योगस्तत्र \;भुच्छायामानार्द्धमिति कर्काटकेन वृत्तं कुर्यात्, तद्बाह्यवृत्तं यत्र स्पृशति तत्र सम्मीलनमोक्षासनम्, \hyperref[4.22]{सम्मीलन}मिति वृत्तमोक्षमार्गयोर्यत्र योगस्तत्र भूच्छायामानार्द्धमिति कर्काट-केन वृत्ते कृते तेन ग्राह्यात् संस्पर्शात् सम्मीलनं ज्ञेयम्, स्पर्शासन्नम् \hyperref[4.22]{उन्मीलन}-मिति विशेषा-

\newpage

\noindent श्चात्र च्छादकार्द्धसूत्रेणेति नोक्तं भूभयैवोक्तं तस्मात् सूर्यग्रहणे सम्मीलनो-न्मीलनयोरभावो ज्ञेयः~। कदाचित् स्वल्पान्तरमुच्यते तथापि बिम्बयोग एव भवति \;तत्तु \;खच्छन्नम्, \;उभयोश्छाद्यच्छादकयोर्बिम्बसाम्यात्~। अथेष्टग्रास-मोक्षग्रासः~॥ मार्गेति~। कश्चित् पृच्छति स्पर्शकालादनन्तरमभीष्टकालगते मोक्षकालात् \;पूर्वं \;वाभीष्टकाले \;कियान्ग्रासस्तदा \;ग्रासमार्गरेखा \;मोक्षमार्ग-रेखाङ्गुलैः परिमीय तैर्मार्गाङ्गुलैरिष्टकालं गुणयेत्, ततः क्रमेण स्पर्शस्थिति-मोक्षस्थितिघटिकाभिर्गुणयेत् ~~तन्मितानीष्टाङ्गुलानि ~तान्यङ्गुलानि ~यथेष्टं \,स्वमार्गे \,दद्यात्, \,यथेष्टग्रास \,इष्टस्पर्शशरात् \,स्पर्शे \,मार्गे \,मोक्षशरान्मोक्षमार्गे इष्टाङ्गुलैश्चिह्नं कृत्वा तदग्र इष्टाङ्गुलाग्रचिह्ने ग्राहकमानार्द्धेन वृत्ते यावद्ग्राह्यमा-च्छाद्यते \,तावदिष्टकाले \,ग्रहणस्य \,संस्थानं \,ज्ञेयम्~। यथा \,स्पर्शादिष्टघटी \,१ स्पर्शमानाङ्गुलैः १९ गुणितैः १९ स्पर्शस्थित्या ४।३८ भक्ताल्लब्धम् ४।६ इष्टाङ्गु-लानि~। अथ \;परिलेखं \;विना \;स्पर्शमध्यमोक्षज्ञानमुच्यते\textendash \;{\color{violet}"सौम्यांशे \;यदि \;शायकोऽनलदिशि प्राच्यां शशाङ्कग्रहश्छन्नं पूर्वमथान्तकस्य दिशि तन्मुक्तिः क्रमाद्रक्षसाम्~। याम्यश्चेद्विशिखस्तदेन्द्रककुभः ~स्पर्शः ~पुरारेर्दिशि ~च्छन्नं सौम्यदिशीदमुक्तमनला मुक्तीरिमाः स्युः क्रमात्"} इति~॥~१~॥ यया ग्राह्यार्द्धम् ५।३६ मानैक्यार्द्धं १९।४६ स्फुटं स्पर्शवलनं सौम्यं ८।२१।३ स्पष्टं मोक्षवलनं सौम्यं २।८ परे

\newpage

\noindent दक्षिणे \,स्पर्शशरः \,१।१६ \,मध्यशरः \,२।२७ \,मोक्षशरः \,४।४८ \,परः \,उत्तराभू-च्छायामानार्द्धं १४।११ मानार्द्धम् ८।३५ एवमत्र निपुणेन विचार्य परिलेखो विधेयः~। अथ ग्रस्तोदये चन्द्रस्पर्शनताया उदाहरणम्, शके १५२८ भाद्रपद-पूर्णिमायनांशौ २६।३३ अब्दाः ४२३ अधिमासाः १५७~। अवमानि २४५९ अह-र्गणः १५४६९५० औदयिको मध्योऽर्कः ५।७।५९ चन्द्रः १०२५।१३।३४ उच्चं २।२२।३६।४४ पातः ६।२६।३।५४ औदयिकाः स्वदेशीयाः, चन्द्रस्य रामबीजं ०।१५।० रवेर्मन्दफलमृणं २।८।१२ चन्द्रमन्दफलं धनं ४।२७।३४ स्पष्टोऽर्कः ५।४।५७।० चन्द्रः १०।२९।२७।५ अयनांशाः १८।३।२७ चरपलान्यृणानि १२ रविगतिः ५८।३५ चन्द्रगतिः ८१९~। चरपलसंस्कृतदिनार्द्धं १५।१२ रात्र्यर्द्धम् १४।४८ एष्या तिथिः २६।२२ दिने पूर्णिमायां यातत्वान्नतफलार्थम्, नतानयनम्, दिनशेषघटीयुक्तमिति दिनशेषः ४।२२ रात्र्यर्द्धम् १४।४८ अनयोर्योगः १९।० नतं प्राक्, एतावता मध्याह्नांशुमतार्थ एव विवरेत्यादिप्रकारेण सूर्योन्नतमेव चन्द्रनतम्, ततो नतविहीनहतैरित्यादिना नतफलं सूर्यस्य धनं ०।२।५५ चन्द्र-नतफलमृणं ०।६।४६ नतफलसंस्कृतो मध्यग्रहणकालः समकलः सपातात् कालिकेत्यादिना शरोऽङ्गुलादिः २।२४ ऋणे चन्द्रबिम्बं ११।४ भूभा २८।० छन्नं १७।१८ स्थितिः ४।३७ ऋणम्, अथ स्पर्शनतार्थमुदयाद्गतघट्यादि-

\afterpage{\fancyhead[RE,LO]{{\small{अ.\,५}}}}
\newpage

\noindent २२।२१\textendash \,समये स्पर्शस्तेन दिनशेषः ८।३० चन्द्रस्य रात्रिशेष एवेत्यादिवत् २१ रात्रिशेषघट्यः \,८।३ \,चन्द्रदिनार्द्धम् \,१४।४८ \,अनयोर्योगः \,२२।५१ \,स्पर्शनतं प्राक् \;खाङ्कैर्गुणितं \;२००६।३० \;द्युदलेन \;१४।४८ \;भक्तं \;नतांशाः \;१३८।५७।९ राश्यादि\textendash \,४।१८।५७।९\textendash \,भुजः १।११।२।५१ ज्या ७८।३४ अक्षांशाः २४।३५। ९ गुणिता १९३१।१०।३७ त्रिज्यया १२० भक्तं लब्धमाक्षवलनमुत्तरम् १६।५ भाष्यप्रकारे तु नतात् २२।५१ दिनार्द्धं १४।४८ शुद्धं ८।३ खाङ्कैर्गुणितं ७२४। ३० \,भक्तं \,चन्द्रदिनार्द्धेन \,लब्धं \,नतांशाः \,४८।५७ \,नवतेः \,शुद्धाः \,४१।३० \,पूर्व-भुजांशसदृशमतः \;प्राग्वद्वलनमिति, \;अन्यत्यागवत्कार्यमिति \;तद्विशेषत्वात् वलनबोधाय \,दर्शितम्, \,सुबुद्धीनामनवगतं \,किञ्चिन्नास्ति, \,एवं \,ग्रस्तास्तेऽपि नतदिक्स्वयमूह्यम्, सूर्यग्रहणेऽपि ग्रस्तोदये ग्रस्तास्ते वक्तव्यं तद्विक्साधनम्~। 
\vspace{2mm}

\begin{center}
{\large \textbf{इति श्रीकरणकुतूहलवृत्तौ चन्द्रग्रहणाध्यायश्चतुर्थः~॥~४~॥ }}
\end{center}
\vspace{2mm}

{\small \textbf{अथ सूर्यग्रहणाधिकारो व्याख्यायते तत्रादौ परिपाटी लिख्यते\textendash }}\\

संवत् \,१६५७ \,आषाढकृष्णे \,वर्षे \,शके \,१५२२ \,प्रवर्तमाने \,लौकिकश्रावण-वद्यामायां चन्द्रे २८।४६ अत्र दिने सूर्यपर्वविलोकनार्थं गताब्दाः ४१७ अधि-मासाः १५४ अवमानि ९४९३ औदयिकोऽहर्गणः १५२४३६ रामबीजादिबीज-देशान्तरशुद्धा मध्यमाः, सूर्यः ३।०।३६।३२ चन्द्रः २।२९

\newpage

\noindent ।४०।२३ उच्चम् ६।११।२।५६ पातः २।२६।२० चन्द्रे रामबीजं कलाद्यर्णं १५ रविमन्दफलमृणं ०।२८।३६ स्पष्टोऽर्कः ३।०।६।३४ चन्द्रमन्दफलं धनं ४।४१। ५० स्पष्टचन्द्रः २।२४।१।५७ पातः २।२६।५५ अयनांशाः १७।५७।२० चरम-लानि १०३ ऋणमतस्तिथिघटी एष्या २८।४६ दिनार्द्धं १६।४३ दिनं ३३।२६ रात्र्यर्द्धं १३।१७ रात्रिः २६।३४ द्युदलगतघटीनामिति, दिनार्द्धं १६।४३ गत-घटी २८।४६ अनयोरन्तरं पश्चिमनतं १२।३ चन्द्रस्य सहचरत्वं तुल्यमेव यदा रात्रावमावास्यान्तो \;भवति \;तदा \;रात्रिशेषघटीयुक्तमित्युक्तवन्नतं \;साध्यम्~। अथ नतफलार्थं नतविहीनहतैरित्यादिना नतेन १२।३ खगुणा हीनाः १७।५७ नतेनैव १२।३ गुणाः २१६ एभिः खशरभानुभुवः ११२५० भक्ता लब्धं ५२।५ दश १० रहितं जातो रविहरः ४२।५ असौ ४२।५ स्वदशांशेन ४।१२ हीनो जातोंऽशादिश्चन्द्रहरः ~~३७।५२ ~~निजफलं ~~निजहारहृतमिति ~~सूर्यनतफलं कलादि ०।४० चन्द्रनतफलं कलादि ७।२२ रवौ पश्चिमनतत्वाद्धनं नतफल-संस्कृतो रविः ३।०।७।१४ चन्द्रस्य पश्चिमनतत्वाच्चन्द्रे २।२४।१।७ ऋणं नत-फलसंस्कृतश्चन्द्रः \;२।२३।५३।४५~। अथ \;भुक्तेर्नतफलानयनमाह \;रविगति-फलम् २।५५ हारेण ४२।५ हृतं फलं विकला ३ रविवद्भुक्तौ संस्कृते जाता रविगतिः ५६।५८ चन्द्रस्य गतिकलम् २९।१५ धनं हारेण भक्तं लब्धं कलादि ०।४६ अपरकपालत्वाद्भुक्तौ ८१९।५०

\newpage

\noindent ऋणं \;नतफलसंस्कृता \;चन्द्रस्य \;गतिः \;८१८।४ \;अतः \;स्पष्टातिथिरमावास्या एष्या \,घट्यः \,२९।२४ \,एतत्कालिकाः \,यातैष्यनाडीगुणिता \,द्युभुक्तिरित्यादिना जातौ समकलौ, एवं सर्वत्र ज्ञेयं धीमता~। \\

{\small \textbf{अथ प्रस्तुतमारभ्यते नतोन्नतभागान् त्रिभोनलग्नस्येन्द्रवज्रयाह\textendash }}

\phantomsection \label{5.1}
\begin{quote}
{\large \textbf{{\color{purple}दर्शान्तकाले त्रिभहीनलग्नं \\
कार्यं च तत्क्रान्तिपलान्तरैक्यम्~। \\
भिन्नैकदिक्त्वे नतभागकाः स्युः \\
खाङ्कच्युतास्ते पुनरुन्नतांशाः~॥~१~॥}}}
\end{quote}

दर्शान्तेति~। तात्कालिकोऽर्क इत्यादिनामावास्यान्तकालीनं लग्नं संसाध्य राशित्रयेण हीनं कार्यं तेन वित्रिभलग्नेन समश्चेद्दर्शान्तकालिकः सूर्यो भवति तदा \;लम्बनाभावः, \;वित्रिभलग्नादूनेऽधिके \;वा \;रवौ \;लम्बनं \;स्यादिति \;ज्ञेयम्, विशेषोऽत्र वित्रिभलग्नशब्देन दशमं भलग्नं ततश्चानीतं सम्यग्वलनं भवति, सूर्यसिद्धान्तादौ दशमादेव साधितं यतो मध्याह्नसमान्ते दर्शान्ते नताभावात् सूर्य एव दशमो भावस्तदा लम्बनस्याभावः, वित्रिभादानीतं कदाचिन्मध्याह्ना-सन्नकाले गहणादिकं न मिलति तदा दशमादानीयते तदेव वास्तवं परं ग्रन्थ-कृता कदाचिदपेक्षया स्थूलपक्षोऽप्यङ्गीकृतः, यदा लग्ने राशित्रयं न शुध्यति तदा चक्रं दत्वा विशोधयेत्~। उक्तं च भत्रयं चेन्न शुध्येत चक्रं दत्वा विशो-

\newpage

\noindent धयेत्~। सूर्यादूनस्तदा वाच्यो वित्रिभस्त्वधिकोऽपि सन् १ अथ वित्रिभलग्नस्य सायनांशस्य \,क्रान्तिः \,कार्यांशादिस्तस्याः \,क्रान्तेः \,स्वदेशीयांशानां \,भिन्नदि-क्त्वेऽन्तरमेकदिक्त्वे योगः कार्यस्तेषां दिगन्तरेऽधिकस्यैव दिक्, योगे सैव दिङ्नतांशाः स्युस्ते नतांशाः खाङ्केभ्यो नवतिभ्य\textendash \,९०\textendash \,श्च्युताः शेषमुन्नतांशा स्युः~। तथा तात्कालिको रविः ३।०।३५।८ दर्शान्तघटी २९।२४ अत्र सुखादु-त्क्रमलग्नं सायनम् ८।२५।१४।५८ वित्रिभम् ५।२५।१४।५८ अस्मात् प्राग्वत् क्रान्तिरंशादिः \;१।५४।५८ \;उत्तराः, \;सीरोह्यामक्षांशाः \;दक्षिणाः \;२४।३५।९ भिन्नदिक्त्वादनयोरन्तरं जाता नतांशाः २२।४०।३२ अक्षांशशेषत्वाद्दक्षिणाः, यतोऽक्षांशाः सदैव दक्षिणा भवन्ति नवति\textendash \,९०\textendash \,भ्यः शुद्धा जाता उन्नतांशाः ६७।१९।२८ ज्या ११०।३५ नतांशज्या ४६।५~॥~१~॥ \\

{\small \textbf{अथ लम्बननत्योरानयनं वंशस्थद्वयेनाह\textendash }}

\phantomsection \label{5.2}
\begin{quote}
{\large \textbf{{\color{purple}त्रिभोनलग्नार्कविशेषशिञ्जिनी \\
खरामभक्ता घटिकादिलम्बनम्~। \\
तदुन्नतज्या निहतं नखेन्दुभि-\\
र्हृतं स्फुटं स्यात्खमृणं तिथौ क्रमात्~॥~२~॥}}
\vspace{1mm}

\phantomsection \label{5.3}
\textbf{{\color{purple}त्रिभोनलग्नाधिकहीनके रवे-\\
स्ततोऽसकृल्लग्नविलम्बनादिकम्~। \\
नतांशजीवार्कलवान्विताष्टहृत् \\
नतांशदिक् चाङ्गुलपूर्वका नतिः~॥~३~॥}}}
\end{quote}

\newpage

वित्रिभलग्नदर्शान्तकालीनसूर्ययोर्विवरस्य या ज्या सा \hyperref[5.2]{खरामै}\textendash \,३०\textendash \,\hyperref[5.2]{र्हृता} भक्ता लम्बनं घट्यादि ~मध्यलम्बनं स्यात्तदुन्नतांशज्यया गुणितं \hyperref[5.2]{नखेन्दुभि}\textendash \,१२०\textendash \,र्भजेल्लब्धं स्फुटं लम्बनं भवति, तत्स्पष्टलम्बनं वित्रिभे रवेरधिके तिथौ दर्शान्ते धनम्, रवेर्हीने वित्रिभलग्ने तथर्णमेतावता पूर्वकपाले खावृणं पश्चिम-कपाले \,धनमिति \,निर्णीतिर्ज्ञेया, \,त्रिभोनलग्नेऽधिक \,इति \,लक्षणे \,कदाचित् पूर्वकपालेऽपि \;धनं \;पश्चिमकपाले \;ऋणमिति \;सम्भवति \;तस्माद्युक्तिसहाय-पक्षोऽसौ न भवति~। मध्यलग्नसमे भानौ हरिजस्य न सम्भव इत्यादिपक्षो युक्तिमान् दृश्यते~। परन्तु यद्यत्पद्यैर्महद्भिरङ्गीकृतं तदस्मदाद्यैरप्येवं व्याख्येय-मेवमसकृत्ततो लम्बनसंस्कृततिथिर्लग्नं साध्यमेवं पुनरपि यावत्स्थिरं लम्बनं स्याद्यथा सायनं वित्रिभम् ५।२५।१४।५८ सायनोऽर्कः ३।१८।३२।२८ अनयो-रन्तरम् २।६।४२।३० अस्य भुजज्या ११०।२ खरामभक्ता लब्धं घटिकादि ३। ४० एतन्मध्यलम्बनं तदुन्नतज्यया ११०।३५ गुणितं ४०५।२८ नखेन्दुभिर्भक्तं लब्धं लम्बनं ३।२ रवेः सकाशत्त्रिभोनलग्नमधिकं पश्चिमकपालत्वाच्च दर्शान्ते २९।२४ धनं जातम् ३२।४६ एवमसकृत् ३२।४६ यातैष्यनाडीत्यादिना रवि-फलम् ३१।६ औदयिके सूर्ये धनं कृतमथवा लम्बनेन संगुण्य षष्ट्या विभज्य लब्धं कलादि समकलसूर्यमध्ये लब्धं धनं कार्यमृणे लम्बने हीनं कार्यमिति कृते

\newpage

\noindent तात्कालिकः \;सूर्यः \;३।०।३८।२० \;सायनः \;३।१८।३५।४० \;दर्शान्तः \;३२।४६ अस्माल्लग्नं कार्यं ९।१४।३९।३७ वित्रिभम् ६।१४।३९।३७ अस्मात् क्रान्तिः ५। ५३।४८ याम्या नतांशाः ३०।२८।५७ उन्नतांशाः ५९।३१।३ एषां ज्या १०३।२५ सायनयोर्वित्रिभसूर्ययोरन्तरम् २।२६।३।५७ अस्य ज्या ११९।२२ खराम\textendash \,३०-भक्ता घटिकादिलम्बनं मध्यमं ३।५९ प्राग्वत्स्पष्टं ३।२५ पश्चिमकपालत्वात् तिथौ २९।२४ धनम् ३२।४९ उत्तरे तात्कालीनो रविः ३।०।३८।२२ सायनः ३। १८।३५।४२ सायनलग्नम् ९।१४।५७।२१ वित्रिभं ६।१४।५७।२१ क्रान्तिः ६। ०।५६ दक्षिणा नतांशाः ३०।३६।५ याम्या उन्नतांशाः ५९।२३।५५ नतज्या ६१। १ उन्नतज्या १०३।१६ वित्रिभसूर्ययोरन्तरम् २।२६।२१।३९ अस्य ज्या ११९।१६ खराम\textendash \,३०\textendash \,भक्ता \;घटिकादिलम्बनं \;३।५८ \;स्पष्टलम्बनं \;३।२५ \;जातं \;स्थिरं लम्बनं ३।२५ दर्शान्ततिथौ २९।२४ धनं जातः स्थिरो दर्शान्तः ३२।४९ मध्य-ग्रहणकालोऽयम्~। अथ नत्यानयनम्~। नतांशजीवेति नतांशानां ज्या स्वदशां-शेनान्विताष्टभक्ता \;लब्धनतांशानां \;दिगेव \;दिग्यस्याः \;सा \;नतांशदिक्, \;यथा स्थिरलम्बनानयनमङ्गुलाद्या नतिः स्यात् नतांशदिक् नतांशाः ३०।३६।५ एषां ज्या ६१।१ इयं स्वदशांशेन ५।५ युक्ता ६६।६ अष्टभिर्भक्ता लब्धम् ८।१५ इयम् अङ्गुलाद्या नतिः सदैव

\newpage

\noindent दक्षिणा, उदयास्ते परमलम्बनं घटिकाचतुष्टयम् ४ परमनतिरङ्गुलाद्या १६। ५५~॥~२~॥~३~॥ \\

{\small \textbf{अथ सकृत्प्रकारेण लम्बनानयनं वसन्ततिलकाद्वयेनाह\textendash }}

\phantomsection \label{5.4}
\begin{quote}
{\large \textbf{{\color{purple}सप्ताद्रयः कुमनवोऽष्टधृती नवेन्दु-\\
दस्राः शरत्रियमलाः खजिनाश्च पिण्डाः~। \\
षट्त्र्यश्विनो जिनयमा द्विशती त्रिभोन-\\
लग्नार्कयोर्विवरभागमितेर्भवाप्ताः~॥~४~॥}}
\vspace{1mm}

\phantomsection \label{5.5}
\textbf{{\color{purple}पिण्डो गतस्त्वगतयातवियोगनिघ्न-\\
शेषेशभागरहितः सहितश्च भोग्ये~। \\
ऊनाधिके खरसहृत्खलु लम्बनं वा \\
प्राग्वत् स्फुटः सकृदतो नतिरन्यलग्नात्~॥~५~॥}}}
\end{quote}

सप्तेति~। ७७।१४१।१८८।२१९।२३५।२४०।२३६।२२४।२०० इत्यादयो नव लम्बनपिण्डा एकादशैकादशान्तरितभागानामसकृत्साधितलम्बनस्य पानी-यपलानीत्यर्थः~। यदा त्रिभोनलग्नार्कयोरन्तरं षट्षष्टिभागा भवन्ति तदा परमं लम्बनमसकृत्कर्मणा \;साधितं \;घटीचतुष्टयात्मकमुपचयात्मकमुत्पद्यते \;ततः परमोपचयात्मकमेकादशभिरंशैरष्टौ खण्डकानि भवन्ति तेन नवमः पिण्डो नवत्यंशा द्वितीयमुक्तम्, अथ गणितगतपर्वान्तकालीनत्रिभोनलग्नार्कयोर्वि-वरस्यान्तस्य भुजभागेभ्यो \hyperref[5.4]{भवै}रेकादशभक्तेभ्यो यल्लब्धं सत्संख्यः पिण्डो गतः उक्तं \;च \;{\color{violet}करणप्रकाशे\textendash ~वित्रिभलग्नार्कान्तरभुजभागोनघ्नरदभुवो \;भक्ता~। भाष्टककुभिः सकृद्वा लम्बननाड्यः स्फुटाः प्राग्वत्~॥}

\newpage

\noindent अथ \,गतगम्यपिण्डान्तरेण \,गुणिताच्छेषादेकादशभिर्भक्तेन \,गतपिण्डो \,युता गम्ये पिण्डेऽधिके सति गम्ये पिण्डे हीने रहितः एवं संस्कृते गते पिण्डे खरसै-र्भक्ते लब्धं मध्यमलम्बनं प्राग्वत्, तदुन्नतज्यानिहतं नखेन्दुभिर्भक्तमित्यादिना कार्यमेवं \,सकृदेवैकवारमपि \,भवति, \,अतो \,लम्बनसंस्कृततिथ्यन्तकालीन-लग्नान्नतांशादिक्रमेण नतिः साध्या यथा दर्शान्तकालीनवित्रिभलग्नार्कयोर-न्तरं २।६।४२।३० भुजभागाः ६६ एकादशभक्ता लब्धं ६ षष्ठो गतः २४० गम्यः २३६ पिण्डयोरन्तरेण ४ शेषांशादि ०।४२।२७ गुणितम् २।५० एकादशभक्तं लब्धेन ०।१५।२७ गतपिण्डे गम्यपिण्डस्य हीनत्वाद्धीनं २३९।४४।३३ षष्टि-भक्ते लब्धं मध्यमलम्बनं ३।५९ वित्रिभलग्नोत्पन्नोन्नतज्यया ११०।३५ गुणितं ४४०।२९ नखेन्दुभक्तं जातं घट्यादिलम्बनं सकृत्स्थिरं रवेः सकाशादधिकं वित्रिभं तस्माद्धनं ३।४० दर्शान्ततिथौ २९।२४ युक्तं जातः स्थिरो ग्रहणस्य मध्यकालः ३३।४ नतिरन्यलग्नादिति मध्यग्रहणसमयिकः ३३।४ सूर्यः ३।०। ३८।३७ लग्नं ९।१६।२६।७ वित्रिभं ६।१६।२६।७ क्रान्तिः ६।४६।३७ नतांशा याम्याः ३१।९।४० ज्या ६१।५८ स्वद्वादशांशेन ५।९ युक्ता ६७।७ अष्टभक्ता ८।२३ इयं नतिः सकृत्प्रकारेण लम्बने कृत उपयोगिनी ज्ञेया~॥~५~॥

\newpage

{\small \textbf{अथ मध्यस्थित्यानयनमिन्द्रवज्राद्वयेनाह\textendash }}

\phantomsection \label{5.6}
\begin{quote}
{\large \textbf{{\color{purple}स्पष्टोऽत्र बाणो नतिसंस्कृतः स्यात्\\
छन्नं ततः प्राग्वदतः स्थितिश्च~। \\
स्थित्योनयुक्ताद्गणितागताच्च \\
तिथ्यन्ततो लम्बनकं पृथक्स्थम्~॥~६~॥}}
\vspace{1mm}

\phantomsection \label{5.7}
\textbf{{\color{purple}स्वर्णं च तस्मिन् प्रविधाय साध्यः\\
तात्कालिकः स्पष्टशरः स्थितिश्च~। \\
तयोनयुक्ते गणितागते तत् \\
स्वर्णं पृथक्स्थं मुहुरेवमेतौ~॥~७~॥}}}
\end{quote}

\hyperref[5.6]{स्पष्ट} ~इति ~स्थिरलम्बनसंस्कृततिथ्यन्तकालीनसपातचन्द्राज्जातः ~शरो नत्या संस्कृत एकदिक्त्वे युतिः भिन्नदिक्त्वेऽन्तरं स स्पष्टशरः स्यात्, स्पष्ट-शरेणैव प्राग्वच्छन्नं साध्यं स्थितिश्च साध्या यथा सकृत्प्रकारेण तिथ्यन्तः ३२। ४९ तत्समयिकश्चन्द्रः ३।१।२।४६~। पातः २।२६।२१।३८ अनयोर्योगः ५।२७। ४३।२४ \;भुजः \;०।२।१६।३६ \;ज्या \;४।४६ \;त्रिघ्नी \;१४।१८ \;कृता\textendash \,४\textendash \,प्ता \;३।३४ शरोऽङ्गुलादिरुत्तरः नतिर्याम्या ८।१५ भिन्नदिक्त्वात्तयोरन्तरं जातः स्पष्टशरो याम्यः ४।४१ अथ बिम्बानयनम्~। बिम्बं विधोरिति चन्द्रगतिः ८१९।४ युगाद्रि\textendash \,७४\textendash \,भक्ता जातं चन्द्रबिम्बम् ११।४ अङ्गुलादिरविगतिः ५६।५८ द्विघ्नी ११३। ५६ शिवा\textendash \,११\textendash \,प्ता रविबिम्बं १०।२१ सूर्यग्रहे छाद्यः सूर्यः १०।२२ छादकश्चन्द्रः ११।४ अनयोरैक्यार्द्धम् १०।४२ शरेण ४।४१ हीनम् ६।१ इदं स्थगितं छन्नं विश्वाकरणार्थं छन्नं ६।१ नख-

\newpage

\noindent २० \,गुणं \,१२०।२० \,छाद्येन \,१०।२१ \,भक्तं \,विश्वा \,११।३७ \,अथ \,स्थित्यानयनं द्विघ्नाच्छरात् \,९।२२ \,छन्नेन \,६।१ \,युता \,१५।२३ \,हतात् \,९२।३३ \,मूलम् \,९।३७ खाष्टेन्दु\textendash \,१८०।१७३१।०\textendash \,चन्द्रार्कयोर्गत्यन्तरेण \;७६२।६ \;भक्तं \;घटिकादिकं स्थित्यर्द्धं २।१६ प्रायशः सूर्यग्रहणे विमर्दाभावः~॥~७~॥ \\

{\small \textbf{अथ स्पर्शमोक्षमिन्द्रवज्रार्द्धेनोपजात्या चेन्द्रवज्रयोपदिश्योपजात्याह\textendash }}

\phantomsection \label{5.8}
\begin{quote}
{\large \textbf{{\color{purple}स्यातां स्फुटौ प्रग्रहमुक्तिकालौ \\
सकृत्कृते लम्बनके सकृत्स्नः~। \\
तन्मध्यकालान्तरगे स्थिती स्फुटे \\
शेषं शशाङ्कग्रहणोक्तमत्र हि~॥~८~॥}}}
\end{quote}

सकृत्स्थित्येत्यादिना गणितेन तिथ्यानयनप्रकारेणागतस्तिथ्यन्तः स्पर्श-काले साध्ये स्पर्शस्थित्योनः कार्यः मोक्षे साध्ये मोक्षस्थित्या युक्तः कार्यस्तादृ-शात्तिथ्यन्तात् पूर्वोक्तप्रकारेण लम्बनं संसाध्यं द्विः स्थाप्यः एकस्मिन् स्थिति-लम्बनं तस्मिन् तिथौ हीनयुक्ते तिथ्यन्ते स्वमृणं प्राग्वद्विधेयं तस्मात् तात्का-लिकः स्पष्टः शरः स्थितिश्च कार्या तया स्थितया स्थित्या स्पर्शमोक्षयोरून-युक्ते \;गणितागततिथ्यन्ते \;द्वितीयस्थानस्थितं \;लम्बनं \;स्वमृणं \;कार्यं \;तस्मात् पुनर्लम्बनं स्फुटशरः स्थितिश्चेत्येवमसकृद्यावदवशेषः स्यादेवं ग्रहमुक्तिकाले स्फुटौ स्तः, यदि सकृद्विधिना लम्बनं तदा सकृदेवायातौ प्रग्रहमुक्तिकालौ स्तः, तयोः स्पर्शमुक्तिकालयोः मध्यग्रहणकालयोर्येऽन्तरे तद्गते स्पर्शमोक्ष-

\newpage

\noindent स्थिती \,स्फुटौ \,स्तः, \,स्पर्शकालमध्यकालयोरन्तरं \,स्पर्शस्थितिः \,मोक्षकाल-मध्यकालयोरन्तरं \,मोक्षस्थितिरित्यर्थः~। \,यथा \,मध्यस्थित्या \,२।१६ गणिता-गततिथ्यन्तः \,२९।२४ \,स्पर्शत्वादूनः \,२७।८ \,एतत्कालीनः \,सूर्यः \,३।०।३२।१९ इष्टकाले २७।८ सायनः ३।१८।३०।१९ सुखादुत्क्रमलग्नं सायनं ८।१३।१७।५ वित्रिभं ५।१३।१७।५ क्रान्तिः ६।४०।५८ नतांशाः २७।५४।११ उन्नतांशाः ७२। ५।४९ \,ज्या \,११४।२ \,त्रिभोनलग्नार्कयोरन्तरम् \,१।२४।४६।४६ \,ज्या \,९७।४४ खरामभक्ता \;३।१५ \;मध्यमलम्बनं \;स्पष्टलम्बनं \;३।५ \;धनं \;पृथगेवं \;लम्बनं स्थित्यूनं गणितागतेन २७।८ धनं जातं ३०।१३ स्थित्यर्थं यथा ३०।१३ एतत्का-लीनसूर्यः ३।०।३५।५५ नत्यर्थं वित्रिभं सायनं ५।२९।३३।३७ नतांशाः २४। २४।३३ दशमनतज्या ४९।२२ नतिर्याम्या ६।४१ एतत्कालीन\textendash \,३०।१३\textendash \,श्चन्द्रः ३।०।४६।१४ पातः २।२६।२१।३१~। योगः ५।२७।७।४५ शरः ४।३१ सौम्यः नतिसंस्कृतः स्पष्टशरो याम्यः २।१० मानयोगार्द्धम् १०।४२ शरेण २।१० हीनं ८।३२ छन्नम्~। शरात् २।१० द्विघ्नात् ४।२० छन्नेन ८।३२ युतं १२।५२ हतात् १०९।४७ मूलम् १०।२९ स्थितिः २।२८ अनया गणितागततिथ्यन्तः २९।२४ स्पर्शत्वाद्धीनः २६।५६ अस्मिन् पृथक्स्थापितम् ३।५ धनं जातं स्थूलस्पर्श-कालः ३०।१२ एवम-

\newpage

\noindent सकृत्करणार्थं तात्कालीनसूर्यः ३।०।३५।४४ सायनः ३।१८।३३।४० सायन-वित्रिभम् ५।२८।३०।१६ क्रान्तिः ०।३६।५। सोम्या नतांशाः २३।५९।४ उन्न-तांश\textendash \,६६।०।५६\textendash \,ज्या १०९।२४ वित्रिभार्कयोरन्तर\textendash \,२।९।५७\textendash \,ज्या ११२। ५७ लम्बनम् ३।४५ इदं पृथक् मध्यमसायनस्थित्यूनस्तिथ्यन्तः २७।८ धनं जातम् ३०।३३ स्थित्यर्थम् ३०।३३ एतत्कालीनरविः ३।०।३६।१४ चन्द्रः ३।०। ५७।४७ पातः २।२६।२१।३२ सायनसूर्यः ३।१८।३३।३४ सायनवित्रिभं ६।१। ३२।३५ नतांशा याम्याः २५।१२।२३ उन्नतांशा ६४।४७।३७ नतिः ६।५३ सपा-तचन्द्रः ५।२७।१२।१९ शरः सौम्यः ४।२४ नतिसंस्कृतस्पष्टशरः २।२९ छन्नम् ८।२३ स्थितिः २।२८ अनयोनतिथ्यन्ते २६।५६ पूर्वागतं लम्बनं ३।२५ धनम् ३।२१ अथासकृत्कर्मणार्थम् ३०।२१ एतत्कालीनसूर्यः ३।०।२६।२ वित्रिभं ६। ०।२१।३४ नतांशा याम्याः २४।४३।४९ उन्नतांश\textendash \,६५।१६।११\textendash \,ज्या १०८।४४ वित्रिभार्कयोरन्तरं २।११।४८।१२ ज्या ११३।५४ लम्बनं ३।४७ स्पष्टं स्थिर-लम्बनम् ३।२५ पृथगिदं मध्यग्रहणस्थित्यूने दर्शान्ते २७।८ धनं जातम् ३०।

\newpage

\noindent ३३ स्थित्यर्थम् ३०।३३ एतत्कालीनवित्रिभं ६।१।३२।३५ नतिः ६।५ सपात-चन्द्रः ५।२७।१२।५९ शरः ४।२४ स्पष्टशरः २।१९ स्पार्शिको दक्षिणः स्थितिः २।२८ अनयोनतिथ्यन्ते २६।५६ पृथक्स्थं लम्बनं धनं २।२५ जातः स्थिरस्पर्श-कालः ३०।२१ स्थिरमध्यग्रहणकालः ३२।४९ अनयोरन्तरं स्पष्टा स्पर्शस्थितिः २।२८ अथ सकृल्लम्बनेन स्पर्शकालः साध्यते यथा मध्या मध्यस्थितिः २।१६ गणितागततिथ्यन्तः २९।२४ ऊनः २७।८ एतत्कालीनसायनसूर्यः ३।१८।३०। ९ सायनवित्रिभम् ५।१३।१७।५ अनयोरन्तरं १।२४।४६।५६ भागाः ५४।४६। ५६ भवाप्ता ४ गतपिण्डः २१९ गतगम्यपिण्डयोरन्तरेण १६ शेषांशादिः १०। ४६।५६ \;गुणितम् \;१७२।२८।१६ \;गतपिण्डे \;२१९ \;गम्यपिण्डस्याधिकत्वाद्युतः २३४।४० षष्टिभक्ता ३।५४ उन्नतज्यया ११४।२ गुणितं ४४४।४३ नखेन्दुभक्तं ३।४२ स्पष्टं सकृल्लम्बनम् ३।४२ अनेन स्थित्यूनतिथ्यन्तः २७।८ धनम् ३०।५० स्थिरः स्पर्शकालः~। एतत्कालीनलग्नान्नतिः साध्या, एतत्कालीनसपातचन्द्रा-च्छरः \;साध्यः~। अथ \;मोक्षकाला \;नयनम्\textendash \,मध्यस्थित्या \;२।१६ \;गणितागत-तिथ्यन्तो २९।२४ युक्तः ३१।४० एतत्कालीनोऽर्कः ३।०।३७।१८ सायनः ३। १८।३४।३८ सायनवित्रिभम् ६।

\newpage

\noindent ८।८।४ कान्तिर्याम्या ३।१६।४२ नतांशः २७।५१।५२ उन्नतांशाः ६२।८।८ ज्या १०५।५५ \;वित्रिभार्कयोरन्तरं \;२।१९।३४।२६ \;ज्या \;११७।४७ \;लम्बनं \;३।५५ स्पष्टलम्बनं ३।२७ पृथक् स्थितियुक्तगणितागततिथ्यन्ते ३१।४० युतं जातम् ३५।३५।७ एतत्कालीनरविचन्द्रपाताः सूर्यः ३।०।४।३४ चन्द्रः ३।१।५४।१३ पातः २।२६।२१।४६ सायनोऽर्कः ३।१८।३७।५४ रात्रिगतघटी १।४१ समयिकं लग्नं ९।२८।३३।५७ वित्रिभं ६।२८।३३।५७ क्रान्तिर्याम्या ११।८।४८ नतांशाः ३५।३३।५७ उन्नतांशाः ५४।१६।३ नतिर्दक्षिणा ९।३५ तिथिः २४ अथ शरार्थं सपातचन्द्रः ५।२८।१५।५९ शरः सौम्यः २।५४ नतिसंस्कृतस्पष्टशरो याम्यः ६।४१ \;छन्नं \;४।१० \;शरात् \;६।४१ \;द्विघ्नात् \;१३।२२ \;छन्नायुतहतान्मूलं \;८।२३ स्थितिः १।५८ अनया गणितागततिथ्यन्तः २९।२४ युतो जातः ३१।२२ अस्मिन् पृथक् स्थापितं लम्बनम् ३।२७ युतं जातम् ३४।४९ स्थूलो मोक्षकालः, अस-कृत्कर्मणार्थमेतत् ३४।४९ कालीनः सूर्यः ३।१८।३७।३७ रात्रिगतघटी १।२३ समयिकं सायनवित्रिभं ६।२६।४७।२७ क्रान्तिर्याम्या १०।३०।२०~। नतांशा ३५।५।११~। उन्नतांशा ५४।५४।४९ एषां ज्या ९७।५३

\newpage

\noindent वित्रिभार्कयोरन्तरम् ८।९।५० अत्र देशे रात्रौ मोक्षस्तेन परकीयो मोक्षदर्शन-कालः साध्यः, अथ लम्बनार्थं वित्रिभलग्नर्कयोरन्तरस्य ३।८।१९।५० ज्याक-रणार्थं भुजः २।२१।५०।१० अत्रानुदितोऽपि यतो जीवा भुजकोटी विना न भवति उक्तं च ग्रन्थान्तरेऽपि यत्र जीवा विहिता तत्रानुक्तमपि भुजं विधायैव जीवा कार्या १२८।२२ खरामैर्भक्ता लम्बनम् ३।५६ पूर्वलम्बनं स्पष्टम् ३।१३ रवेः सकाशात्त्रिभोनमधिकं तेनेदं धनमिदं पृथक् मध्यस्थितियुक् तिथ्यन्ते ३०।४० युते जातम् ३४।५३ अथ स्थित्यर्थम् ३४।५३ एतत्कालीनरविः ३।०। ४०।२१ चन्द्रः ३।१।४९।५६ पातः २।२६।२१।५७ शरोङ्गुलादिः २।५० नत्यर्थम् एतत्कालीनसायनवित्रिभं ~६।२७।११।७ ~कान्तिर्याम्या ~१०।३९।० ~नतांशा याम्याः ३५।१४।९ उन्नतांशाः ५४।४५।५१ नतज्या ६८।५४ नतिः ९।१९ नति-संस्कृतशरः ६।२९ छन्नम् ४।१३ द्विघ्नाच्छरात् १२।५८ छन्नयुताहतात् ५३८।२२ स्थितिः १।५८ अनया गणितागततिथ्यन्तः २९।२० युतः ३१।२२ अस्मिन्पृथक् स्थितं लम्बनं ३।१२ धनं जातं ३४।३५ असकृत्साध्यमतः कालाद्वित्रिभं कृत्वा पूर्ववल्लम्बनं साध्यं पृथक् स्थाप्यमेकत्रस्थलम्बनं मध्यस्थितियुक्ते गणितागत-तिथ्यन्ते विधेयम्, अथाधिके धनं हीने हीनमेवं लम्बनसंस्कृतकालात्पूर्ववत् स्थितिमानीय तथा गणितागततिथ्यन्ते युक्तं कार्यं तस्मिन्ननष्टलम्बनं पूर्वव-द्विधेयं संस्थितो मोक्षकाल-

\newpage

\begin{sloppypar}
\noindent स्थिरमध्यग्रहणकालयोरन्तरं स्पष्टमोक्षस्थितिः~। अथ सकृत्प्रकारार्थं मध्य-स्थितियुक्तगणितागततिथ्यन्तः ~३१।४० ~एतत्कालीनवित्रिभसूर्यान्तरम् ~२। १९।३४।२६ अंशाः ७९ भवाप्ता ७ पिण्डः २३६ शेषं २।३४।२६ लम्बनम् ३।५३ उन्नतज्यया १०५।५५ गुणितं नखेन्दु\textendash \,१२०\textendash \,भक्तं ३।२५ स्पष्टं सकृत्पूर्ववत् स्थितिः ३१।४० युक्तम् ३५।५ सकृत्प्रकारेण स्थिरो मोक्षकालः, अस्मान्नतिः शरश्च ~साध्यः~। अथ ~स्पर्शकालीनलम्बनं ~यथा ~दिनोदयाद्गतघटी ~३०।२१ समये \;स्पर्शः, \;अत्र \;द्युदलगतघटीनामित्यादिना \;नतं \;यथा \;दिनदलं \;१६।४३ गतघटी ३०।२१ अनयोरन्तरं नतं १३।३८ पश्चिमे खाङ्का\textendash \,९०\textendash \,हतम् १२२७~। दिनार्धेन भक्ता ७३।२३।५९ नतांशाः एषां ज्या ११४।४३ अनयाक्षांशाः २४। ३५।९ गुणिताः २८२०।१९।३६ त्रिज्यया लब्धं २३।३०।९ पश्चिमनतत्वाद्दक्षिणे इदमाक्षजं वलनम् २३।३०।९~॥ अथायनवलनम्~॥ स्पर्शकालीनसूर्यः ३।०। २६।२ सायनः ३।१८।३३।२२ भुजः २।११।२६।३८ कोटिः ०।१८।३३।२२ ज्या ३८।६ बाणै\textendash \,५\textendash \,र्भक्ता लब्धं ७।३७।१२ सायनसूर्यो दक्षिणायने तेन दक्षिणा-क्षजं २३।३०।९ दक्षिणायनम् ७।३७।१२ अनयोरेकदिक्त्वाद्योगः ३१।७। २१ याम्योऽस्य ज्या ६१।५४ अनया मानैक्यार्द्धं १०।४२ गुणितं ६६२।१९ त्रिज्यया १२० भक्तं लब्धं स्पष्टवलनं याम्यं ५।३२ प्राग्ग्रासमोक्षे विपरीतदिक्क इति रवेः
\end{sloppypar}

\newpage

\begin{sloppypar}
\noindent स्पर्शवलनमुत्तरे ग्रस्तान् मोक्षकालीननते तत्रादौ नतानयनम्\textendash \,रात्रिशेषगत इत्यादिना रात्रिगतघटी १।९ दिनदलयोर्योगः १७।५२ पश्चिमनतं खाङ्काहतं १६०८ दिनार्धेन १६।४३ भक्तं लब्धा नतांशाः ९६।५।३० भुजः २।२३।५४।३० ज्या ११९।४१ अक्षांशैः २४।३५।९ गुणिता २९।४२।३० त्रिज्यया १२० भक्ता लब्धम् २४।३१ आक्षजं वलनं याम्यम् २४।३१ उदयाद्गतघटी ३४।३५ समयिकः सूर्यः ३।०।४५।४१ सायनः ३।१८।३७।२४ भुजः २।११।२२।३६ कोटिः ०।१८।३७।२४ ज्या ३८।१४ पञ्चभक्ता ७।३९ जातमायनं वलनं याम्यं दक्षिणायनत्वाद्दक्षिणयोरायनाक्षजयोरेकदिक्त्वाद्योगोंऽशाः ३२।१० ज्या ६३।४१ मानैक्यार्द्धेन १०।४२ गुणिता ६८१।२४ त्रिज्यया १२० भक्तं लब्धमङ्गुलादिस्पष्टमोक्षवलनं ५।४० याम्यम्~॥~८~॥\\

{\small \textbf{अथ ग्रहणे ग्रहस्यानादेशतां वर्णज्ञानं चेन्द्रवज्रयाह\textendash }}

\phantomsection \label{5.9}
\begin{quote}
{\large \textbf{{\color{purple}अर्कांशकोऽर्कस्य विधोर्नृपांशो \\
नादेशनीयः खलु खण्डितोऽपि~। \\
अल्पार्द्धसर्वग्रहणे शशी स्यात् \\
धूम्रोऽसितो बभ्रुरिनस्तु कृष्णः~॥~९~॥}}}
\end{quote}

\begin{center}
{\large \textbf{इतीह भास्करोदिते ग्रहागमे कुतूहले \\
विदग्धबुद्धिवल्लभे रविग्रहस्य साधनम्~॥~५~॥ }}
\end{center}

अर्कबिम्बमानस्य द्वादशांशतुल्योऽर्कग्रासो नादेशनीयः, विधुबिम्बमान-षोडशांशतुल्यो विधुग्रासो न वक्तव्यः~। अल्प-
\end{sloppypar}

\afterpage{\fancyhead[RE,LO]{{\small{अ.\,६}}}}
\newpage

\begin{sloppypar}
\noindent ग्रहणे चन्द्रो धूम्रवर्णः स्यात्~। अर्द्धग्रहणे कृष्णः, सर्वग्रहणे पिशङ्गः, सूर्यस्तु सर्वदा सर्वग्रहणे कृष्ण एव, सूर्यग्रहणं विंशतिविश्वाधिकं न भवति~॥~९~॥ करणकुतूहलवृत्तावेतस्यामिष्टदेवताकृपया गणककुमुदकौमुद्यां व्याख्येयं रविग्रहे नियतम्~॥
\vspace{2mm}

\begin{center}
{\large \textbf{इतिकरणकुतूहलवृत्तौ सूर्यपर्वाधिकारः समाप्तः~॥~५~॥}}
\end{center}
\vspace{2mm}

{\small \textbf{अथोदयास्ताधिकारो व्याख्यायते, तत्रादौ जीवस्योदयास्तौ स्थूलं तथा शार्दूल-विक्रीडितद्वयेनाह\textendash }}

\phantomsection \label{6.1}
\begin{quote}
{\large \textbf{{\color{purple}इष्टोऽह्नां निचयोऽब्ददिग्लवयुतः पञ्चाभ्रभूवर्जितो \\
भक्तो नन्दनवाग्निभिस्तिथिमितैः शेषैर्गुरोरुद्गमः~। \\
अस्तो वेदगजाग्निभिस्तदधिकैरूनैर्गतैष्यैर्दिनै-\\
स्तात्कालार्कघटीफलं च तिथिवत् सूर्याहतं शेषकैः~॥~१~॥ }}
\vspace{1mm}

\phantomsection \label{6.2}
\textbf{{\color{purple}राशिभ्यामुदये युताद्दिनकरादस्ते त्रिभिः संयुतात् \\
वच्चोक्तार्कघटीफलं च खगुणैः सूक्ष्मं धनर्णं तथा~। \\
संक्रान्तेरुदयात् खखाग्निरहितात्तिथ्याहतात् स्वोदये- \\
नाप्तं तच्च गुरूदये धनमृणं चास्ते तु तत्सप्तमात्~॥~२~॥}}}
\end{quote}

इष्टोऽहर्गणः करणगताब्दानां दशमांशेन युतः \hyperref[6.1]{पञ्चाभ्रभूवर्जितः} पञ्चोत्तर-शतेन १०५ रहितः \hyperref[6.1]{नन्दनवाग्निभि}\textendash \,३९९\textendash \,र्भक्तो लब्धस्य प्रयोजनाभावा-ल्लब्धं त्याज्यं शेषै\hyperref[6.1]{स्तिथिमितैः} पञ्चदश\textendash \,१५\textendash \,मितैः गुरोरुदयो ज्ञेयः, \hyperref[6.1]{वेद-गजाग्निभि}श्च ३८४ अधिकैः शेषैर्गतावुदयास्तौ तदूनैः शेषै\hyperref[6.1]{रेष्यौ} भाविनौ~। अथ शेषस्पष्टीकरणं \hyperref[6.1]{तात्कालार्क} इति तस्मिन्काले यस्मिन् राश्यंशेऽर्को भवति तस्य घटीफलं \hyperref[6.1]{सूर्याहतं} द्वादशगुणं षष्ट्या विभज्य दिना-
\end{sloppypar}

\newpage

\begin{sloppypar}
\noindent दि कृत्वा शेषैस्तिथिवद्धनमृणं वा कार्यं संक्रान्तिषु पक्षयोरर्कघटीफलं यथा\textendash \,कर्के मृगे त्रयः षट्कं ३।६~। सिंहे कुम्भे गजा दश ८।१०~। कन्या मीने भवारुद्राः ११।११~। तुला मेषे शिवा दश ११।१०~। वृषेऽलौ च गजास्तर्काः ८।६~। धनुर्युग्मे त्रिशून्यकं ३।०~। रविनाड्यो मृगादौ स्वकर्कादौ च ऋणं क्रमात्~॥ इति~। अर्कस्य पञ्चदशभागाः कार्या मन्दफलस्यांशानां नाड्यः एताः सिद्धा घटिताः अत्र घटिका द्विगुणा नवभक्ता भागा भवन्ति, एते मध्यमरविगतिविवरेण भक्तास्ते दिनाधिकेनाधिकास्तदत्र लाघवार्थं द्वादश\textendash \,१२\textendash \,गुणिता नाड्यः षष्ट्या यदाप्यते ते दिनानि भवन्ति, अत्र सुबुधेरनुकम्पया सुखार्थं शेषार्थमत्र \hyperref[6.2]{वच्चोक्तार्कघटीफलम्} इति मुहुर्मुहुरुक्ताः कर्कमकरयोः प्रथमार्द्धे तिस्रो घटिकाः ३ षडुत्तरार्द्ध एवं सर्वत्र कर्कादावृणं मकरादौ धनमित्यर्थः~। यतः शेषस्फुटीकरणमुदये जाते \hyperref[6.2]{राशिभ्यां} राशिद्वयेन युतात् सूर्यादस्ते राशित्रयेण युताद्रवेर्बह्वाचार्योक्तकर्मघटीफलं ग्राह्यं तच्च \hyperref[6.2]{खगुणै}\textendash \,३०\textendash \,र्गुणितं तथा \hyperref[6.2]{धनर्णं} शेषे ऋणं याम्ये सौम्ये धनमित्यर्थः~। पुनः स्पष्टीकरणम्\textendash \,अथ \hyperref[6.2]{संक्रान्तेरुदयात्} सूर्याक्रान्तराश्युदयात् त्रिप्रश्नोक्तात् \hyperref[6.2]{खखाग्नि}भि\textendash \,३००\textendash \,स्त्रिशतैर्हीनात् \hyperref[6.2]{तिथि}भिः पञ्चदशभि\textendash \,१५\textendash \,र्गुणितात् \hyperref[6.2]{स्वोदयेन} संक्रान्त्युदयेन भक्ताद्यल्लब्धं तच्छेषे गुरोरुदये धनं कार्यम्~। अथास्तसाधनम्\textendash \,अस्ते तु संक्रान्तेः सप्तमराश्युदयात् खखाग्नि\textendash \,३००\textendash \,रहिताद्द्विप्रकारेणानीतं फलमृणं शेषे कार्यं ततः स्पष्टशे-
\end{sloppypar}

\newpage

\begin{sloppypar}
\noindent षस्योदये पञ्चदशभिः १५ सहान्तरमितैर्गम्योऽधिकैरिष्टदिनाद्गतोदयः, अस्तो वेदगजाग्निभिः सहान्तरं ततः प्राग्वज्ज्ञेयं यथा शाके १५४३ लौकिका-षाढकृष्णे ४ भौमे गताब्दाः ४३८ अधिमासाः १६२ अहर्गणः १६००७४ अयं गताब्दानां ४३८ दशांशेन ४३।४८ युतः १६०११७।४८ पञ्चाभ्रभूभि\textendash \,१०५\textendash \,र्हीनः १६००१२।४८ नन्दनवाग्निभि\textendash \,३९९\textendash \,र्भक्ता लब्धस्य न प्रयोजनं शेषम् १३।४८ पञ्चदशभ्यो हीनस्तेन गम्योदयः~। शेषस्पष्टीकरणम्\textendash \,अथ वृषस्य द्वितीयपक्षे रविस्तेनार्कघटीपलं ६ द्वादशगुणम् ७२ षष्टया लब्धं दिनादि १।१२ मकरादित्वाच्छेषम् १३।४८ धनं जातम् १५।० अथ रविः १।२९ उदयत्वाद्द्विराशि\textendash \,२\textendash \,युतं ३।२९ तस्यार्कघटीफलं ६ त्रिंशद्गुणितम् १८०० षष्ट्या भक्तम् ३।०।० लब्धं दिनादि कर्कादित्वाच्छेषे १५।० ऋणं जातम् १२।०।० पुनः सङ्क्रान्तेरुदयः २५६ खखाग्निभि\textendash \,३००\textendash \,र्हीनोऽत्र हीनो न भवति तेन शोध्यो न शुद्ध्येद्यदा तदा कार्यं व्यस्तविशोधनमिति तेनोदयः २५६ खखाग्नि\textendash \,३००\textendash \,शुद्धः ४४ तिथिगुण\textendash \,६६०\textendash \,वृषोदय\textendash \,२५६\textendash \,पलैः भक्तं लब्धं दिनादि ३।३४।४१ गुरूदयत्वाद्धनं शेषं परं व्यस्तं तदुक्तमिति शेषम् १५।०।० ऋणे कृते शेषम् ९।२५।१९ स्पष्टं जातं पञ्चदशभिः १५ सहान्तरम् ५।३४।४१ एभिर्दिनादिभिरिष्टदिना-
\end{sloppypar}

\newpage

\begin{sloppypar}
\noindent त्पञ्चदशभ्यः शेषस्योनत्वाद्गम्य एवमस्तसाधने श्लोकोक्तवत्क्रिया~॥~२~॥\\

{\small \textbf{अथ शुक्रस्योदयास्तौ शार्दूलविक्रीडितद्वयेनाह\textendash }}

\phantomsection \label{6.3}
\begin{quote}
{\large \textbf{{\color{purple}पञ्चेशोनगणोऽब्दभूपलवयुग्वेदाष्टबाणैर्हृतः \\
पश्चात्षट्कृतिभिर्नगाष्टयुगलैः शुक्रोदयास्तौ क्रमात्~। \\
शेषैः प्राग्नगगोयमैर्गजयुगप्राणैस्तदानीं च यः \\
तिग्मांशोरुदयः खखाग्निरहितः पश्चात्तु तत्सप्तमः~॥~३~॥ }}
\vspace{1mm}

\phantomsection \label{6.4}
\textbf{{\color{purple}क्षुण्णः पञ्चगुणैः शरैश्च विषयैरङ्गाग्निसंख्यैः क्रमात्\\
भक्तस्तेन च भोदयेन फलयुक् प्राक्छेषकैरुद्गमः~। \\
ज्ञेयश्चास्तमयः फलेन रहितैः शोध्यं न शुध्येद्यदा \\
कार्यं व्यस्तविशोधनं धनमृणं व्यस्तं तदुक्तं तदा~॥~४~॥ }}}
\end{quote}

अहर्गणः \hyperref[6.3]{पञ्चेशोनः} पञ्चदशाधिकशतेन ११५ हीनोऽब्दकरणगताब्दानां \hyperref[6.3]{भूपलवे}न षोडशांशेन १६ युक्तो \hyperref[6.3]{वेदाष्टबाणै}श्चतुरशीत्यधिकपञ्चशतेन ५८४ भक्तो लब्धस्य प्रयोजनाभावः शेषैः \hyperref[6.3]{षट्कृतिभिः} ३६ षड्-त्रिंशतासमैः पश्चादुदयः, \hyperref[6.3]{नगाष्टयुगलैः} २८७ पश्चादस्तमयः, \hyperref[6.3]{नगगोयमैः} २९७ शेषैः प्रागु-द्गमः, \hyperref[6.3]{गजयुगप्राणैः} ५४८ शेषैः प्रागस्तमय इति~। अथ शेषस्पष्टीकरणे तदानीं \hyperref[6.3]{यस्तिग्मांशोरुदयः} यस्याः संक्रान्तौ रविर्भवति तस्य पलानीत्यर्थः~। \hyperref[6.3]{खखाग्नि}भिः ३०० हीनाः कार्याश्चेद्यदि पश्चादुदयास्तौ तदा तत्समुदयः खखाग्निभिः ३०० हीनः कार्यस्ततः क्रमात् पश्चिमोदये 
\end{sloppypar}

\newpage

\begin{sloppypar}
\noindent साध्यमाने \hyperref[6.4]{पञ्चगुणैः} पञ्चत्रिंशद्भिः ३५ गुणनीयाः पश्चादस्ते \hyperref[6.4]{शरैः} ५ पञ्चभिः प्रागुदये \hyperref[6.4]{विषयैः} पञ्चभिः ५ प्रागस्तेऽ\hyperref[6.4]{ङ्गाग्निसंख्यैः} ३६ षट्त्रिंशद्भिर्गुणनीयास्ततः स्वोदयेन भजेत् \hyperref[6.4]{फलयुक् शेषकै}रिति फलेन युक्ताश्च ये प्राक्शेषाः~पूर्वा-गतशेषकैरुद्गमो वाच्यः फलेन रहितैः प्राक्शेषैरस्तमयो वाच्यः~। अथ विशे-षमाह\textendash \,शोध्यमङ्कं खखाग्नि\textendash \,३००\textendash \,लक्षणमुदये न शुध्येत्तदा व्यस्तशोधनं कार्यं खखाग्निभ्यः ३०० उदयः शोध्यस्तदुत्थं भाजकेन भक्तं यत्फलं लभ्यते तच्छेषे \hyperref[6.4]{व्यस्तं धनमृणं} कुर्याद्यत्र धनं तत्रर्णं यत्रर्णं तत्र धनं कुर्यादयं विधिरन्यत्रापि ज्ञेयो यथाहर्गणः १६००७४ पञ्चेशोनः १५९९५९ अब्दानां ४३८ भूपलवेन २७।२२ युक्तः १५९९८६।२२ वेदाष्टबाणैः ५८४ भक्तः लब्धस्य २७३ प्रयोजनाभावः शेषैः ५५४।२२ गजयुगबाणेभ्योऽधिकेन प्रागस्तो गतः~। शेषस्पष्टीकरणमत्र प्रागस्तत्वाद्वृषस्योदयः २५५ खखाग्निभिः ३०० रहितो न भवति तेनोदयः २५५ खखाग्नि\textendash \,३००\textendash \,मध्ये शुद्धः ४५ प्रागस्त-त्वादङ्गाग्निभि\textendash \,३६\textendash \,र्गुणितः १६२० स्वेन स्वोदयेन २५५ भक्तः लब्धेन ६।२१ प्रागागतं शेषम् ५५४।२२ अस्तत्वादूनं क्रियते परं व्यस्तं तदुत्थमिति युतम् ५६०।४३ प्रोक्ताङ्कैः ५४८ सहान्तरम् १२।४३ प्रोक्तादधिकशेषैरेभिर्दिनैरिष्टाहर्गणाद्गतोऽस्त इति, एवमन्यदपि यथास्थानं कार्यम्~॥~३, ४~॥
\end{sloppypar}

\newpage

{\small \textbf{अथ शीघ्रकेन्द्रांशेभ्यो वक्रादिसाधनमुपजात्याह\textendash }}

\phantomsection \label{6.5}
\begin{quote}
{\large \textbf{{\color{purple}द्राक्केन्द्रभागैस्त्रिनृपैः शरेन्द्रैः \\
तत्त्वेन्दुभिः सप्तनृपैस्त्रिरुद्रैः~। \\
स्याद्वक्रता भूमिसुतादिकानाम् \\
अवक्रता तद्रहितैश्च भांशैः~॥~५~॥}}}
\end{quote}

\begin{sloppypar}
असकृत्कर्मणा स्थैर्यागतं भौमकेन्द्रं तस्य \hyperref[6.5]{भागा} अंशा यदि \hyperref[6.5]{त्रिनृपा}स्त्रिषष्ट्युत्तरशतपरिमिता भवन्ति तदा भौमस्य वक्रत्वमाह~। एवं स्थिरशीघ्रकेन्द्रभागैः \hyperref[6.5]{शरेन्द्रैः} १४५ पञ्चचत्वारिंशदुत्तरशतभागमितैर्बुधस्य वक्रत्वमाह~। गुरो\hyperref[6.5]{स्तत्त्वेन्दुभि}रंशमितैः १२५~। शुक्रस्य \hyperref[6.5]{सप्तनृपै}रंशैः १६७~। शने\hyperref[6.5]{स्त्रिरुद्रै}रंशैः ११३~। अथोक्तांशैर्भांशेभ्य ३६० शोधितैस्तत्तद्ग्रहस्यावक्रता गतिः~। भौमस्य नगगोचन्द्रैरंशैः १९७ बुधस्य तिथिनेत्रैः २१५~। गुरोः पञ्चाग्निदस्रैः २३५~। शुक्रस्य त्रिनवेन्दुभिः १९३~। शनेर्नगसिद्धैः २४७ वक्रतात्यागो मार्गी स्यात्~। द्राक्केन्द्रभागकैर्विशेषश्चात्र यतो वक्रारम्भे वक्रत्यागे च गतिः पूर्णं यातो यावद्भिः शीघ्रकेन्द्रांशैर्गतिः पूर्णं स्यात्तावन्तः केन्द्रांशाः पाठे पठिताः यथा शनेस्त्रिरुद्रपरिमिते केन्द्रांशे ११३ भुजज्या ११०।१८ कोटिज्या ४६।४२ शीघ्रफलम् १।४।३ कर्णः ११५।३३ केन्द्रगतिः ५७।८ गतिफलम् ५९।८ शीघ्रभुक्तिः ३९।८ शोधितम् ०।६ स्पष्टागतिरेवम् एवं मुनिसिद्धमिते केन्द्रे\textendash \,२४७\textendash \,ऽपि भुक्तेरेवं सर्वेषां ज्ञेयम्~॥~५~॥
\end{sloppypar}

\newpage

\begin{sloppypar}
{\small \textbf{अथोदयास्तसम्भवज्ञानं भौमगुरुशनीनामुपजात्याह\textendash }}

\phantomsection \label{6.6}
\begin{quote}
{\large \textbf{{\color{purple}प्राच्यामुदेति क्षितिजोऽष्टदस्रैः\\
शक्रैर्गुरुः सप्तकुभिश्च मन्दः~। \\
स्वस्वोदयांशोनितचक्रभागै-\\
स्त्रयो व्रजन्त्यस्तमयं प्रतीच्याम्~॥~६~॥}}}
\end{quote}

प्राच्यामिति~। भौमोऽष्टाविंशत्या २८ स्थितैः केन्द्रांशकैः प्राच्यामुदेति तथा गुरुश्चतुर्दशभिः १४ शनिः सप्तदशभिः १७ अथोदयांशैश्चक्रांशेभ्यः ३६० शोधितैस्तत्तद्ग्रहस्य प्रतीच्यामस्तमयो भवति तद्यथा भौमो द्विदेवैः ३३२ जीवोऽङ्गवेदाग्निभिः ३४६ शनिस्त्रिवेददहनैः ३३४ प्रतीच्यामस्तमेति~। उक्तं च {\color{violet}"रवेरूनभुक्तिर्ग्रहः प्रागुदेति प्रतीच्यामसावस्तमेत्यन्यथान्यः"}~॥~६~॥\\

{\small \textbf{अथ बुधशुक्रयोरुदयास्तज्ञानं वसन्ततिलकेनाह\textendash }}

\phantomsection \label{6.7}
\begin{quote}
{\large \textbf{{\color{purple}खाक्षै\textendash \,५०\textendash \,र्जिनै\textendash \,२४\textendash \,र्ज्ञसितयोरुदयः प्रतीच्याम् \\
अस्तश्च पञ्चतिथिभि\textendash \,१५५\textendash \,र्मुनिसप्तभूभिः १७७~।\\
प्रागुद्गमः शरनखै\textendash \,२०५\textendash \,स्त्रिधृतिप्रमाणैः १८३ \\
अस्तश्च तत्र दशवह्निभि\textendash \,३१०\textendash \,रङ्गदेवैः ३३६~॥~७~॥}}}
\end{quote}

पञ्चाशद्भिः शीघ्रकेन्द्रांशैर्बुधस्य प्रतीच्याम् उदयः, चतुर्विंशद्भिः २४ शुक्रस्य, पञ्चपञ्चाशदुत्तरशतेन १५५ बुधस्य प्रतीच्यामस्तः, सप्तसप्तत्युत्तर-शतेन १७७ शुक्रस्य पश्चिमायामस्तः~। अथ बुधस्य पञ्चोत्तरद्विशत्या २०५ प्राच्यामुदयः, शुक्रस्य त्र्यशीत्युत्तरशतेन १८३ प्राच्यामुदयः, बुधस्य दशो-
\end{sloppypar}

\newpage

\begin{sloppypar}
\noindent त्तरत्रिशत्या ३१० प्राच्याम् अस्तः शुक्रस्य षट्त्रिंशदधिकशतत्रयेण ३३६ प्राच्यामस्तः~। ज्ञशुक्रावृजू प्रत्यगुद्गम्य वक्रां गतिं प्राप्य तत्रैव यातः प्रतिष्ठाम्~। ततः प्राक् समुद्गम्य वक्रावृजुत्वं समासाद्य तत्रैव चास्तं व्रजेतामिति विशेषः~। यदि स्पष्टार्कस्य स्पष्टग्रहयोरन्तरं वक्ष्यमाणकालांशतुल्यं स्यात्तदोदयोस्तो वा भविष्यतीति ज्ञेयमिदं तु मध्यमसूर्यस्फुटसूर्ययोरन्तरं कालं यदा भवति तदोदयास्तौ स्थूलत्वमेवाङ्गीकृतं तथाविधो ग्रहो यावति केन्द्रेण सता स्पष्टो भवति तावत्केन्द्रमुदयास्तसूचकमुक्तम्, अथवैभिरुक्तशीघ्रकेन्द्रभागैः शीघ्रफलं यावन्तोंऽशा उत्पद्यन्ते तदंशस्वकालांशयोर्योगः~। एतावदंशाः स्फुटास्ते सूर्यात् पूर्वतः परतश्च तावन्त एव, यथा सूर्यः ११।७।५६।१४ मन्दफलसंस्कृतो गुरुः १०।२३।५६।१४ शीघ्रकेन्द्रम् ०।४ भुजज्या २९ कोटिज्या ११६ कर्णः ४२।२१ शीघ्रफलम् २।१३।५२ स्पष्टो गुरुः १०।२६।१०।६ मध्यार्कः ११।७।५६।१४ अन्तरम् ०।११।४६ रुद्रमित-कालांशतुल्या भागा मध्यार्कग्रहयोरन्तरस्थिता अतोऽत्रोदयसम्भवः स्थूल- त्वादल्पान्तरमुपेक्षितम्~॥~७~॥\\

{\small \textbf{अथ वक्रादीनां दिनाद्यानयनमाह\textendash }}

\phantomsection \label{6.8}
\begin{quote}
{\large \textbf{{\color{purple}अवक्रवक्रास्तमयोदयोक्त-\\
भागाधिकोनाः कलिका विभक्ताः~। \\
द्राक्केन्द्रभुक्त्याप्तदिनैर्गतैष्यैः\\
अवक्रवक्रास्तमयोदयाः स्युः~॥~८~॥}}}
\end{quote}
\end{sloppypar}

\newpage

\begin{sloppypar}
अवक्रादीनां मार्गादीनां ये प्रोक्ता भागास्तेभ्योऽधिकभागानां कलाः शीघ्रकेन्द्रगत्या भक्ता लब्धैर्दिनादिभिर्गतैरवक्रादयो भवन्ति अथ यदा प्रोक्तभागेभ्यो द्राक्केन्द्रभागा ऊनास्तदा प्राग्वल्लब्धदिनादिभिर्गम्यैरवक्रादयः स्युः, अत्रासन्ने वक्री कोऽपि नास्ति तेन तदुदाहरणं न दर्शितम्~। मार्गस्योदाहरणं यथा भौमशीघ्रकेन्द्रम् ६।२२।३३।४७ गतिः ३१।७ ग्रन्थोक्तमार्गकेन्द्रगतिः ६।१७।०।० अनयोरन्तरांशाः ५।३३।४५ विकलाः २००२५ द्राक्केन्द्रगत्या विकलारूपया भक्ता लब्धैर्दिनादिभिः १०।४३।३२ प्रोक्तेभ्योऽधिककेन्द्रांशत्वादिष्टसमयाद्गतो मार्गी~। एवं वैशाखकृष्णे ३ बुधे उदयाद्गतघटी १६।२८ समये मार्गी भौम एवं सर्वेषां कार्यं बुधस्य पश्चिमोदयज्ञानाय शीघ्रकेन्द्रम् २।३।८।४३ प्रोक्तकेन्द्रम् १।२०।०।० अनयोरन्तरांशाः १३।८।४३ विकलाः ४७३२३ द्राक्केन्द्रगत्या १८०।२४ भक्ता लब्धं दिनादि ४।२२।१९ स्वल्पत्वात्पलानि त्यक्तानि प्रोक्तेभ्योऽधिकत्वाद्गत उदयः एवं वैशाखकृष्णे ८ भौम उदयाद्गतघटी ३७।४१ समये बुधस्य पश्चिमोदयो जात एवमन्येऽपि~॥~८~॥\\

{\small \textbf{अथ ग्रहाणां कालांशाः पातविक्षेपाः शार्दूलविक्रीडितेनाह\textendash }}

\phantomsection \label{6.9.1}
\begin{quote}
{\large \textbf{{\color{purple}सूर्याः १२ सप्तदश १७ त्रिभूपरिमिता १३ \\
\emph{\color{white}अ} \hfill रुद्रा ११ नवाक्षेन्दवः~१५९ \\
कालांशाः शशिनोऽनृजोः}}}
\end{quote}
\end{sloppypar}

\newpage

\phantomsection \label{6.9}
\begin{quote}
{\large \textbf{{\color{purple}कुरहिताः पाताः कुजाद्राशयः~।\\ 
रुद्रे\textendash \,११\textendash \,शोङ्क\textendash \,९\textendash \,दश\textendash \,१०\textendash \,द्विपा ८ अथ लवा \\
\emph{\color{white}अ} \hfill अष्टौ ८ ग्रहाः ९ कुञ्जराः ८ \\
शून्यं ० शैलभुवः १७ स्वचञ्चलफलैर्व्यस्तैरमी संस्कृताः~॥~९~॥}}}
\end{quote}

\begin{sloppypar}
\hyperref[6.9.1]{शशिन}श्चन्द्रादाराभ्य सूर्यादयः कालांशाश्चन्द्रस्य द्वादशकालांशाः भौमस्य सप्तदश बुधस्य त्रयोदश गुरोरेकादश शुक्रस्य नव शनेः पञ्चदश १५ \hyperref[6.9.1]{अनृजो}र्वक्रिणो ग्रहस्य \hyperref[6.9]{कुरहिता} एकेन हीना अत एव कालांशाः यथा वक्रिणो भौमस्य षोडश १६ बुधस्य द्वादश १२ गुरोर्दश १० भृगोरष्ट ८ शनेश्चतुर्दश १४ अथ भौमादारभ्य रुद्रादिराशयोऽष्टाद्यंशाधिकाः पाताः स्युः यथा भौमस्य पातो राश्यादिः ११।८ एवं बुधस्य ११।९ गुरोः ९।८ भृगोः १०।० शनेः ८।१७ \hyperref[6.9]{अमी} पाताः स्वस्वशीघ्रफलैर्व्यस्तैः संस्कृताः कार्याः धनैर्हीनाः हीनैर्युक्ताः स्पष्टाः स्युः बुधशुक्रयोः पाताः स्वमन्दाभ्यां फलाभ्यां ग्रहवदेव युक्तहीनौ कार्यौ तयोः पातौ स्पष्टौ स्तः केनचिदनयोर्मन्दफलं व्यस्तं कृतं तदसत्ते सम्यग्वासनां न जानन्ति~॥~९~॥\\

{\small \textbf{अथ भौमादारभ्य कलात्मकविक्षेपात्तथा साधनं च शार्दूलविक्रीडितेनाह\textendash }}

\phantomsection \label{6.10.1}
\begin{quote}
{\large \textbf{{\color{purple}मन्दाभ्यां बुधशुक्रयोरथ कुजाद्विक्षेपकाः खेश्वरा ११० \\
द्वीषुक्ष्माः १५२ षडगाः ७६ षडग्निश-}}}
\end{quote}
\end{sloppypar}

\newpage

\phantomsection \label{6.10}
\begin{quote}
{\large \textbf{{\color{purple}शिनः १३६ खत्रीन्दवो १३० लिप्तिकाः~।\\
खेटात् पातयुतात्तथा ज्ञसितयोः शीघ्रोच्चतो दोर्ज्यका \\
क्षेपघ्नी चलकर्णहृत्त्रिविहृता स्यादङ्गुलाद्यः शरः~॥~१०~॥}}}
\end{quote}

\begin{sloppypar}
भौमस्य विक्षेपः शरः कलात्मकदशोत्तरशतम् ११० एवं बुधस्य १५२ गुरोः ७६ भृगोः १३६ शनेः १३० अथ शरसाधनम्~। खेटादिति~। स्फुटग्रहः स्वीयस्फुटपातेन युक्तः कार्यः बुधशुक्रयोस्तु स्वशीघ्रोच्चं स्वीयस्फुट-पातेन युतं कार्यं ततः सपातस्य ग्रहस्य भुजज्या स्वस्वशरेण गुण्या स्वस्वशीघ्रकर्णेन भक्ता लब्धं कलात्मकं त्रिभिर्भक्तमङ्गुलात्मकं शरो भवति स च सपातग्रहदिक्~। यथा स्पष्टो बुधः १।७।२७।६ गतिः १०३।२८ मन्दफलं धनम् ०।१५।७ शीघ्रोच्चम् ३।४।५७।१६ शीघ्रकेन्द्रम् २।३।२८।४३ गतिः १८०।२४ कर्णः १४५।१५ अथ पातः ११।९।०।० मन्दफलेन धनरूपेण ०।१५।७ युतो जातः स्पष्टः पातः ११।९।१५।७ शीघ्रोच्चेन ३।४।५७।१६ युतस्य २।१४।१२।२३ भुजज्या ११५।६ शरेण १५२ गुणिता १७४८५।१२ कर्णेन १४५।१४ भक्ता लब्धं कलादिशरः १२०।२७ सपातशीघ्रोच्चमुत्तरगोले तेनोत्तरस्त्रिभक्तोऽङ्गुलादिः ४०।९ शरः~॥~१०~॥\\
\end{sloppypar}

{\small \textbf{अथायनाक्षजदृक्कर्मद्वयमपि लाघवार्थमैक्यत्वेन शार्दूलविक्रीडितत्रयेणाह\textendash }}

\phantomsection \label{6.11.1}
\begin{quote}
{\large \textbf{{\color{purple}प्राक्पश्चात्त्रिभहीनयुक्तखचरक्रान्त्यक्षतोंऽशा नताः}}}
\end{quote}

\newpage

\phantomsection \label{6.11}
\begin{quote}
{\large \textbf{{\color{purple}शुद्धास्ते नवतेः स्युरुन्नतलवाः साध्ये पृथक्तज्ज्यके~।\\
क्षेपघ्नी नतशिञ्जिनी गुण\textendash \,३\textendash \,गुणा भक्तोनतांशज्यया\\
स्वर्णे लब्धकला ग्रहे शरनतांशैकान्यदिक्त्वे क्रमात्~॥~११~॥}}
\vspace{1mm}

\phantomsection \label{6.12}
\textbf{{\color{purple}पश्चाद्व्यस्तमितीह दृष्टिखचरस्तत्सूर्ययोरल्पकः \\
कल्प्योऽर्कस्त्वपरस्तनुश्च घटिकाः प्राग्वत्तयोरन्तरे~। \\
पश्चात् षड्भयुतात्तु ता रसहताः कालांशकाः सन्ति तैः \\
प्रोक्तेभ्योऽभ्यधिकैर्गतः समुदयो न्यूनैस्तु गम्यस्ततः~॥~१२~॥}}
\vspace{1mm}

\phantomsection \label{6.13}
\textbf{{\color{purple}व्यस्तश्चास्तमयस्तदन्तरकलाः खाभ्राग्निभिः सङ्गुणा\\
भानो राश्युदयेन चेदपरतस्तत्सप्तमेनोद्धृताः~। \\
ताः स्युः क्षेत्रकला जवान्तरहृता वक्रे जवैक्योद्वृता \\
यातैष्योऽस्तमयोऽथवा समुदयो ज्ञेयोऽत्र लब्धैर्दिनैः~॥~१३~॥}}}
\end{quote}

\begin{sloppypar}
\hyperref[6.11.1]{प्राक्} इति प्राच्यामुदेति क्षितिजोऽष्टदस्त्रैरित्यादि पूर्वसम्बन्धिकेन्द्रांशैर्यदि ग्रहस्योदयास्तावायाति तस्मिन् दिने तदासन्नग्रहं स्फुटं विधाय यदि प्राच्याम् उदयास्तौ तदा राशित्रयेण हीनं कुर्याद्ग्रहं प्रतीच्यां चेत्तदा त्रिराशियुतं कुर्यादेवंभूतस्य ग्रहस्य क्रान्तिं कुर्यात् परं सा क्रान्तिः शरेण संस्कृता न कार्या केवलैव ग्राह्या, उक्तं च भाष्येऽत्र केऽपीयं क्रान्तिः शरेण संस्कार्येति क्रान्तिमुत्पादयन्ति सा वृथैव ज्ञेया कस्मात्त्रिभहीनयुक्तग्रहस्य विमण्डलाभावात्तस्मादेव केवलाः क्रान्त्यंशाः प्रसाध्याः इति ततः क्रान्त्य-क्षयोर्भिन्नदिक्त्वेऽन्तरमेकदिक्त्वे योग इत्यनेन
\end{sloppypar}

\newpage

\begin{sloppypar}
\noindent नतांशाः साध्याः नतांशैर्हीना नवति\textendash \,९०\textendash \,रुन्नतांशाः स्युः~। अथ नतांशा-नामुन्नतांशानां च पृथक् पृथक् ज्यां साधयेत्~। अथ नतांशज्यका~स्वकीयेन खेटात् पातयुतादित्यादिनागतेन स्फुटशरेणाङ्गुलादिना गुणिता~पुन\hyperref[6.11]{र्गुणैः} त्रिभिर्गुणितोन्नतांशज्यया भक्ता लब्धं कलादिकं फलं शरनतांशानाम् एकदिक्त्वे स्फुटग्रहे धनं कुर्याद्भिन्नदिक्त्वे हीनं कुर्यात् प्राच्यां दिश्यथ प्रतीच्यामुदयास्तौ तदा व्यस्तं शरनतांशयोरेकदिक्त्वे हीनं भिन्न-दिक्त्वे धनं कुर्यादेवं जातो दृष्ट\hyperref[6.12]{खचरो} ग्रहः स्यात्तस्य दृक्कर्मग्रहस्य, तात्कालीनस्फुटसूर्यस्य च मध्ये योऽल्पः स रविः प्रकल्प्यः, अथ यो ग्रहोऽधिकस्तत्तनुर्लग्नं प्रकल्पनीयं तयोः कल्पितयोरर्कलग्नयोरन्तरे घटिकाः प्राग्वत् कार्याः, अर्कस्य भोग्यस्तनुभुक्तयुक्त इत्यादिना साध्याः~। प्रतीच्यां~तु राशिषट्कयुक्तयो रविदृग्ग्रहयोर्घटिकाः साध्यास्ता घटिका रसैर्हृताः कालां-शका इष्टा भवन्ति त इष्टकालांशाः प्रोक्तेभ्यः पूर्वपाठपठितकालांशेभ्यो यद्यधिकास्तदोदयो गतः न्यूनाश्चेत्तदा गम्यः~। अस्तमयस्तु व्यस्तः यदीष्ट-कालांशा अधिकास्तदास्तो गम्यः न्यूनास्तदास्तो गतः अथ प्रोक्ता ये~कालां-शास्तेषामिष्टकालांशानां चान्तरकलाः \hyperref[6.13]{खाभ्राग्निभिः} शतत्रयेण गुणिताः सायनसूर्याधिष्ठितराशिस्वोदयपलैर्भक्ताः प्रतीच्यां तु सूर्याधिष्ठितराशेः सप्तमराश्युदयपलैर्भक्ता लब्धकला ग्रहस्य रवेश्च भुक्त्यन्तरेण भक्ताः यदि ग्रहो वक्री तदा भुक्तियोगेन भक्ता
\end{sloppypar}

\newpage

\begin{sloppypar}
\noindent यल्लब्धं तत्संख्यागता उदयास्तयोर्दिवसाः भवन्ति यथा पश्चिमोदयत्वात् त्रिभयुक्तग्रहः ४।१७।२७।६ सायनः ५।५।४१।६ क्रान्तिः सौम्याः ९।३२।५६ अक्षांशा याम्या २४।३५।९ भिन्नदिक्त्वादनयोरन्तरम् १५।२।३ नतांशा उन्नतांशाः ७४।५७।४७ नतज्या ३१।४ उन्नतज्या १२५।२८ नतज्या ३१। ४ शरेणाङ्गुलाद्येन ४०।९ गुणिता १२४७।१९ गुण\textendash \,३\textendash \,गुणा ३७४१। ५७ उन्नतज्यया भक्ते लब्धम् ३२।२४ कलादिदृक्फलं शरनतांशयोरन्य-दिक्त्वादृणं परं पश्चिमत्वाद्धनं ग्रहे १।१७।२७।६ दृक्कर्मसंस्कृतो ग्रहः १।१७।५९।३० सूर्यः १।३।६।१२ उदयग्रहसूर्ययोरल्पः सूर्य एव ग्रहो लग्नं पश्चिमत्वात् सषड्भार्कस्य भोग्यः ९८ लग्नभुक्तम् ७१ अनयोर्योगो १६९ घटी २।४९ षड्गुणा १६।५४ इष्टकालांशाः प्रोक्तकालांशेभ्योऽधिकास्तेनोदयो गतः, उभयोरन्तरांशाः ३।५४ कलाः २३४ खाभ्राग्निभिर्गुणिताः ७०२०० सूर्याक्रान्तसप्तमवृश्चिकोदयेन ३४३ भक्ता लब्धं क्षेत्रकलाः २०४।३९ सवर्णिता १२२८९ बुधगतिः १०३।३८ सूर्यगतिः ५७।२८ अनयोरन्तरेण ४७।१० सवर्णितेन २७७० लब्धं दिनगतम् ४।३५।५८ एवमिष्टदिनात्पूर्वं वैशाखकृष्ण\textendash \,८\textendash \,भौमे घट्यः ३४।२ समये उदयो बुधः पश्चिमे एवं सर्वेषां कर्तव्यम्~॥~१३~॥\\
\end{sloppypar}

{\small \textbf{अथ विशेषमिन्द्रवज्रयाह\textendash }}

\phantomsection \label{6.14.1}
\begin{quote}
{\large \textbf{{\color{purple}प्राग्दृग्ग्रहश्चेदधिको रवेः स्यात् \\ ऊनोऽथवा पश्चिमदृग्ग्र-}}}
\end{quote}

\newpage

\phantomsection \label{6.14}
\begin{quote}
{\large \textbf{{\color{purple}हश्च~। \\
प्रोक्तेष्टकालांशयुतेः कलाभिः \\
साध्यास्तदानीं दिवसा गतैष्याः~॥~१४~॥}}}
\end{quote}

\begin{sloppypar}
रवेः सकाशात् प्राग्दृग्ग्रहश्चेदधिकः प्रत्यग्दृग्ग्रह ऊनः स्यात् तदा प्रोक्तकालांशानामिष्टकालांशानां च योगः कार्यस्तस्य कलाभिर्गतगम्या दिवसाः साध्याः साध्यानन्तरकलाभिः किन्तु संयोगविषय इष्टकालां-शानामधिकत्वे ये गता दिवसाः प्राप्तास्ते गम्या ज्ञेया न तु गताः गम्याश्चेद्गता इत्यर्थः~। उक्तं च {\color{violet}सिद्धान्तशिरोमणौ\textendash \,"तथा यदीष्टकालांशाः प्रोक्तेभ्योऽभ्यधिकास्तदा~। व्यत्ययश्च गतैष्यत्वे ज्ञेयोऽह्नां सुधिया खलु"} इति~॥~१४~॥\\

{\small \textbf{अथ अगस्त्योदयास्तमुपजात्याह\textendash }}

\phantomsection \label{6.15}
\begin{quote}
{\large \textbf{{\color{purple}अक्षभाष्टहतियुक्तवर्जिताः \\ 
अष्टगोमितलवा गजाद्रयः~। \\
तत्समे दिनमणौ च कुम्भभूः \\
याति दर्शनमदर्शनं क्रमात्~॥~१५~॥}}}
\end{quote}

\begin{center}
{\large \textbf{इतीह भास्करोदिते ग्रहागमे कुतूहले \\
विदग्धबुद्धिवल्लभे ग्रहोदयास्तसाधनम्~॥~६~॥}}
\end{center}

स्वदेशाक्षभाया अष्टभि\textendash \,८\textendash \,र्या \hyperref[6.15]{हति}र्गुणना तया युक्ता अष्टाधिक-नवति\textendash \,९८\textendash \,मितांशास्तत्तुल्ये सूर्येऽगस्त्यस्योदयो भवति~। अष्टाहतिप्रभा-हीनाष्टसप्तत्यंशमिते रवावूनेऽगस्त्यस्यास्तो भवति~। यथा संरोह्यामक्षभा ५।३० अष्ट\textendash \,८\textendash \,गुणा ४४।० अनेनाष्टगोमितलवाः ९८ युक्ताः १४२।०
\end{sloppypar}

\newpage

\begin{sloppypar}
\noindent त्रिंशद्भक्ता राश्यादिः ४।२२ एतत्समो यदा सूर्यो भवति तदागस्त्य-स्योदयः~। अथ पूर्वागतेन ४४।० गजाद्रयो भागाः ७८ हीनाः ३४।० राश्यादिः १।४।०।० एतत्सदृशे रवावगस्त्यस्यास्तः~। यदागस्त्योदयकालोऽभीष्टदिनात् कियद्भि-र्दिनैरिति ज्ञातुमिष्यते तदेष्टदिनार्कस्योदयार्कस्य चान्तरकला रविभुक्त्या भाज्या लब्धदिनैरगस्त्योदय एष्यः यद्युदयसूर्यो महान् यदि न्यूनस्तदा गत एवमस्तसूर्यादस्तमयोऽपि~। अथ चन्द्रस्योदयास्तसाधनम्\textendash \,शके १५७१ फाल्गुनशुक्ल\textendash \,१०\textendash \,गुरौ चन्द्रोदयविलोकनार्थं गताब्दाः ४१२ अहर्गणः १५०८४३ अस्तकालिकाः स्वदेशीयाः ग्रहास्तत्र सूर्यः १०।२१।२।११ चन्द्रः ११।६।१९।४३ उच्चम् १।३।४२।३६ पातः ०।२।५५।३८ अयनांशाः १७।५२।५५ स्पष्टोर्कः १०।२२।५८।५३ गतिः ६०।८ चन्द्रः ११।१९।२६।१७ गतिः ७३५।४ चरमृणम् ३५ चरपलसंस्कृतसूर्यः १०।२२।५८।१८ चन्द्रः ११।९।१८।५५ पातः ०।२।५५।३३ अङ्गुलाद्यः शरो याम्यः २८।५२ अथ पश्चिमोदयत्वात् त्रिभयुक्तचन्द्रः २।९।१८।५९ क्रान्तिः सौम्या २३।५०।१७ अक्षांशा याम्याः २४।३५।९ नतांशा याम्याः ०।४२।५२ दृक्कर्मफलमृणम् १।७ दृक्कर्मसंस्कृतश्चन्द्रः
\end{sloppypar}

\newpage

\begin{sloppypar}
\noindent ११।९।१७।५२ शरनतांशैकान्या दिक्कर्म इत्युक्तत्वादत्र फलं धनमागतं~परं पश्चाद्व्यस्तमित्यत्रर्ण कार्यमिति~। अथेष्टकालांशानयनम्\textendash \,पश्चात्~षड्भयुता-दिति स्थापितो दृक्कर्मशुद्धः सायनः सषड्राशियुतश्चन्द्रः ५।२७।१०।४७ सषड्भसायनरविः ५।१०।५१।१३ अनयोरल्पोऽर्कः कल्प्योऽधिको~लग्नं पश्चात् अन्तरं यद्येकभे लग्नरवीत्यादिनाप्तघटी ३।१ रस\textendash \,६\textendash \,गुणा~१८।६ एत इष्टकालांशाः प्रोक्तकालांशेभ्योऽधिकास्तेन गतोदय इष्टकालांशप्रोक्त-कालांशयोरन्तरकलाः ३६६ खाभ्राग्निः ३०० गुणः १०९८०० पश्चिमा-यामुदयस्तेन सायनसूर्याक्रान्तराशयः सायनसप्तमराश्युदयेन कन्यालग्न-मानेन ३३३ भक्ताः ३२९।४३ सूर्यचन्द्रभुक्त्यन्तरेण ६७५।३२ भक्ते लब्धं दिनादि ०।२९।१७ एभिर्दिनैश्चन्द्रस्योदयो गतः~। अथ चन्द्रास्तसाधनम्~। शके १५२३ लौकिकज्येष्ठकृष्ण\textendash \,३०\textendash \,गुरावत्रदिने पूर्वस्यां चन्द्रास्तसाधने गताब्दाः ४१८ अधिमासाः १५५ अहर्गणः १५२७६२ स्वदेशीया मध्यमाः प्रातःकालिकाः ग्रहास्तत्र सूर्यः १।२०।५५।४० चन्द्रः १।११।५९।१८ उच्चम् ७।१७।१४।२८ पातः ३।१३।३३।३२ अयनांशाः १७।५८।१० स्पष्टोऽर्कः १।२१।५४।५९ चन्द्रः १।२१।२९।४६ चरमृणम् १०४ चरपलसंस्कृतोऽर्कः १।२१।५३।१७ चन्द्रः १।२१।४।५८ पातः ३।१३।३३।२७ शरोऽङ्गुलादिरुत्तरः ५१।४९ प्राक्स्थितत्वाद्वित्रिभचन्द्रः
\end{sloppypar}

\newpage

\begin{sloppypar}
\noindent १०।२१।४।५८ क्रान्तिः १२।५३।१० नतांशा याम्याः १२।३३।१६ उन्नतांशाः ७७।२६।४४ नतज्या २६।६ उन्नतज्या ११६।४३ ऋणं दृक्कर्मकलाः ३४।४५ दृक्कर्मसंस्कृतश्चन्द्रः १।१०।३०।१३ प्राग्वदिष्टकालांशाः ११।१८ पूर्वत्वात् सषड्भोनः कार्यः प्रोक्तेभ्यः १२ ऊनोऽस्तत्वाध्यस्तश्चास्तमय इतिवचनाद्गतः प्रोक्तेष्टकालांशान्तरकलाः ४२ खाभ्राग्निभि\textendash \,३००\textendash \,र्गुणाः १३६०० सायनार्क-राश्युदयमिथुनपलैः ३०५ भक्ताः क्षेत्रकलाः ४१।१८ गत्यन्तरेण ८०१।५८ भक्ते दिनादयः ०।३।५ एवं गतास्तं चतुर्दश्यां शेषरात्रीष्टसमये ३।५ चन्द्रास्तः सूर्यासन्नवशेन चन्द्रस्यास्तः पूर्वस्यां कृष्णचतुर्दश्यासन्ने कार्यं सूर्यासन्नवशेन चन्द्रोदयः पश्चिमायां द्वितीयासन्ने~। अथ ग्रहाणां प्रत्यहमुदयास्तज्ञानमुच्यते\textendash \,तत्र प्रत्यहमुदयः पूर्वस्यां सर्वेषां ग्रहाणां साध्यः प्रत्यहमस्तः प्रतीच्यां साध्यः यः प्राग्दृग्ग्रहः स ग्रहस्योदयलग्नं यः पश्चिमदृग्ग्रहः स ग्रहस्यास्तलग्नं ज्ञेयमुक्तं च {\color{violet}भास्कराचार्यैः\textendash \,प्राग्दृग्ग्रहः स्यादुदयाख्यलग्नमस्ताख्यकं पश्चिमदृग्ग्रहः स} इत्युदाहरणं शके १५१७ माघकृष्ण\textendash \,४\textendash \,शनौ चन्द्रोदयविलोकनार्थं गताब्दाः ४१२ अहर्गणः १५०८३१ स्वदेशीया मध्यमा अस्तकालिकाः ग्रहाः सूर्यः १०।९।१२।३६ चन्द्रः ५।२८।१२।४५ उच्चम् १।२।२२।१५ पातः ०।१।२७।२६ अयनांशाः १७।५२।५२ चरपल\textendash \,५७\textendash \,संस्कृतोऽर्कः
\end{sloppypar}

\afterpage{\fancyhead[RE,LO]{{\small{अ.\,७}}}}
\newpage

\noindent १०।१०।५३।५६ गतिः ६०।२८ चन्द्रः ५।२६।५०।५४ गतिः ८५५।३५ पातः ०।१।२७।२३ गतिः ३।११ अङ्गुलाद्यः शरः सौम्यः २।५५ प्रागुदयविलोकनाय वित्रिभचन्द्रः २।२६।५।५४ क्रान्तिरुत्तरा २३।८।५५ अक्षांशा याम्याः २४।३५। ९ नतांशा याम्याः १।२६।१४ उक्तवद्दृक्फलमृणम् ८।१३ दृक्कर्मसंस्कृतश्चन्द्रः ५।२६।५।४१ \;अयं \;प्राग्दृग्ग्रहः \;इदं \;चन्द्रस्योदयलग्नमीदृशे \;लग्ने \;क्षितिजस्थे चन्द्रस्योदयः~। उक्तं च\textendash \,{\color{violet}"निजनिजोदयलग्नसमुद्गमे समुदयेऽपि भवेद्भनभः सदाम्~। भवति चास्तविलग्नसमुद्गमे प्रतिदिनेऽस्तमयः प्रवहभ्रमात्"~॥} इत्यु-दयाद्गतनाडिकानयनम्~। अथास्तकाल इष्टलग्नम् ४।१०।५३।५६ स्पष्टसूर्य-मध्ये \;राशिषट्कयुक्तेऽस्तकाल \;इष्टलग्नं \;भवतीत्युदयलग्नं \;च \;५।२६।५०।४१ अनयोरन्तरकाल ऊनस्य भोग्योऽधिकभुक्तयुक्तो मध्योदयाढ्यः समयो विल-ग्नादिति प्रकारेण सायनोदयलग्नेष्टलग्नयोरन्तरकालघटी ८।३० समयेऽस्ता-द्गते पूर्वस्यां चन्द्रोदयो भविष्यतीत्युक्तं च\textendash \,{\color{violet}"प्राग्दृग्ग्रहोऽल्पोऽत्र यदीष्टलग्नात् गतो गमिष्यत्युदयं बहुश्चेत्~। ऊनाधिकः पश्चिमदृग्ग्रहश्चेदस्तं गतो यास्यति चेति ~वेद्यम्~॥ \;तदन्तरोत्था ~घटिका ~गतैष्यास्तच्चालितः ~स्यात्सनिजोद-योऽस्तः~।} इत्वेवं सर्वेषां प्रत्यहमुदयास्तौ साध्यौ~॥~१५~॥

\begin{center}
{\large \textbf{इति करणकुतूहलवृत्तौ ग्रहोदयास्ताधिकारः~॥~६~॥}}
\end{center}
\vspace{2mm}

{\small \textbf{अथ शृङ्गोन्नत्यधिकारो व्याख्यायते तत्रादौ वलनसाधनमाह\textendash }}

\phantomsection \label{7.1.1}
\begin{quote}
{\large \textbf{{\color{purple}क्रान्तेः कलाः सायकसंस्कृतेन्दोः}}}
\end{quote}

\newpage

\phantomsection \label{7.1}
\begin{quote}
{\large \textbf{{\color{purple}सषड्भसूर्यायमसंस्कृतास्ताः~।\\
व्यर्केन्दुदोर्ज्यागुणितैः पलांशैः \\
खार्कोद्धृतेरप्यथ संस्कृताश्च~॥~१~॥ \\
व्यर्केन्दुदोराशिभिरिन्द्रियघ्नैः \\
भक्ता भवेयुर्वलनाङ्गुलानि~।}}}
\end{quote}

\begin{sloppypar}
अत्र चन्द्रार्कयोर्दक्षिणोत्तरमन्तरं भुजः पूर्वापरमन्तरं कोटिरनयोरन्तरं तिर्यक्करण एवं त्र्यस्रक्षेत्रं शृङ्गोन्नतिसाधनहेतुकं कृष्णपक्ष औदयिकस्य शुक्लपक्षेऽस्तकालीनस्य चन्द्रस्य, उक्तं च\textendash \,{\color{violet}"मासान्तपादे प्रथमेऽथवेन्दोः शृङ्गोन्नतिर्यद्दिवसेऽवगम्या~। तदोदयेऽस्ते निशि वा प्रसाध्य"} इति~तस्य चन्द्रस्य कलादिका क्रान्तिः कलादिना शरेण संस्कृता भिन्नदिक्त्वे वियुक्तैकदिक्त्वे युक्ता तदौदयिकस्य तात्कालिकस्य वा राशिषट्कयुतस्य सूर्यस्य क्रान्त्या पूर्ववत्संस्कृता सषड्भः कस्मात्क्रियते तदुच्यते उभयोः क्रान्त्योरेकदिश्यन्तरमन्यदिशि योग एवं कृत उभयोर्दक्षिणान्तरमन्तरम् उत्पद्यतेऽत्रैकदिशि योगोऽन्यदिश्यन्तरमित्युक्तं तत्सार्थक्यकरणाय सषड्भः कृतः सषड्भकृते दिगन्यत्वं भवतीत्यतः सषड्भः कृतः अथार्केण हीनस्य चन्द्रस्य भुजज्ययाक्षांशा गुणिताः खार्कै\textendash \,१२०\textendash \,र्भक्ता लब्धेन पूर्ववत् संस्कृताक्षांशवशेन दिक्~। अथ रविहीनचन्द्रस्य भुजराशयः पञ्च ५ गुणास्तैः सर्वसंस्कारसंस्कृता चन्द्रक्रान्तिर्भाज्या लब्धमङ्गुलाद्यं वलनं भवति 
\end{sloppypar}

\newpage

\begin{sloppypar}
\noindent क्रान्तेर्या दिक् सा वलनस्यापि ज्ञेया~। यथा शके १५१७ फाल्गुनशुक्ल\textendash \,१\textendash \,गुरौ चन्द्रोदयस्तत्रापि शृङ्गोन्नतिर्विलोक्यते स्पष्टचन्द्रोऽस्तकालिकः ११।९।१८।५९ सायनः ११।२७।११।५४ भुजः ०।२।४८।६ क्रान्तिकला याम्या ६७।३६ शरकलाभिर्दक्षिणाभिः ८६।१६ एकदिक्त्वाद्युता १५३।५२ पुनः सषड्भसूर्यस्य क्रान्तिकलाः सौम्याः~४५६। १५ भिन्नदिक्त्वादन्तरं सौम्यम् ३०२।२३ षष्टिभक्तांशादि ५।२।२३ व्यर्केन्दु-दोरिति सूर्योनचन्द्रः ०।१६।२०।४१ भुजज्यया ३३।४१ अक्षांशा याम्या २४।३५।९ गुणिताः ८२८।७।५८ खार्को\textendash \,१२०\textendash \,द्धृता लब्धम् ६।५४।४ अक्षांशवशाद्याम्या अनेन ६।५४।४ सौम्यायाः~क्रान्ते\textendash \,५।२।२३\textendash \,रन्तरं याम्यम् १।५१।४१ इयं सर्वसंस्कारसंस्कृता क्रान्तिः व्यर्केन्दुभुजं राश्यादिकं पञ्चगुणं कृत्वा सर्वं राश्यादिकं तदेवांशादिकं प्रकल्प्य सवर्णितं कृत्वा तेन सवर्णितेन सवर्णिता सर्वसंस्कारसंस्कृता क्रान्तिर्भाज्या लब्धम् अङ्गुलादिवलनं सर्वसंस्कारसंस्कृतक्रान्तेर्या दिक् सा वलनस्य दिक् सर्वसंस्कारसंस्कृतक्रान्तेर्यदोत्तरा तदा वलनमत्युत्तरं यदा दक्षिणा तदा वलनमपि दक्षिणम्~। अथ व्यर्केन्दुभुजः ०।१६।२०।४१ पञ्च\textendash \,५\textendash \,गुणं राश्यादिजातम् २।२१।४३।२५ तदेव राश्यादिकमंशादिकं प्रकल्प्य २।२१।४३ षष्ट्या सवर्णितम् ८५०३ सवर्णिता सर्वसंस्कारसंस्कृता क्रान्तिः ६७०१ अनेन ८५०३ भक्ता लब्धमङ्गुलाद्यं वलनम् ०।४७ सर्वसंस्कारसंस्कृता क्रान्तिर्याम्या तस्माद्वलनमपि याम्यम्~॥~१~॥
\end{sloppypar}

\newpage

{\small \textbf{अथ सितासितभागकथनमिन्द्रवज्रयाह\textendash }}

\phantomsection \label{7.2}
\begin{quote}
{\large \textbf{{\color{purple}व्यर्केन्दुकोट्यंशशरेन्दुभागो \\
हारोऽमुना षट्कृतितो यदाप्तम्~॥~२~॥ \\
द्विष्ठं च हारोनयुतं तदर्धे \\
स्यातां क्रमादत्र विभास्वभाख्ये~।}}}
\end{quote}

\begin{sloppypar}
अर्कोनचन्द्रस्य कोटिः पञ्चदश\textendash \,१५\textendash \,भक्ता यदाप्तं यद्धारो भवत्यमुना हारेण \hyperref[7.2]{षट्कृतिः} षट्त्रिंशत् ३६ भक्ता कार्या यल्लब्धं तत्स्थानद्वयस्थितमेकत्र हारेणोनमपरत्र युतं तयोरर्धे \hyperref[7.2]{विभास्वभाख्ये स्यातां} सितासिताख्ये भवतः हारोनार्द्धं विभा हारयुतार्द्धं स्वभेत्यर्थः~। यथा व्यर्केन्दुकोट्यंशादिः ७३।३९। १९ शरेन्दु\textendash \,१५\textendash \,भक्ते लब्धम् ४।५४ अयं हारोऽनेन षट्त्रिंशद्भक्ताः लब्धम् ७।२० द्विष्ठः ७।२० एकत्र हारेणोना २।२६ अर्द्धम् १।१३ जाता विभा, अपरत्र हारेण युता १२।२४ अर्द्धं ६।७ जाता स्वभा~॥~२~॥\\
\end{sloppypar}

{\small \textbf{अथ परिलेखकथनमर्द्धोपजात्या चेन्द्रवज्रयाह\textendash }}

\phantomsection \label{7.3}
\begin{quote}
{\large \textbf{{\color{purple}विधाय सूत्रेण षडङ्गुलेन \\
वृत्तं दिगङ्कं वलनं च वृत्ते~॥~३~॥}}}
\end{quote}

\newpage

\phantomsection \label{7.4}
\begin{quote}
{\large \textbf{{\color{purple}प्राक्शुक्लपक्षे परतश्च कृष्णे \\ 
केन्द्राद्विभां तद्वलनाग्रसूत्रे~। \\
कृत्वा विभाग्रे स्वभया च वृत्तं \\
ज्ञेयेन्दुखण्डाकृतिरेवमत्र~॥~४~॥}}}
\end{quote}

\begin{center}
{\large \textbf{इतीह भास्करोदिते ग्रहागमे कुतूहले \\
विदग्धबुद्धिवल्लभे शृङ्गोन्नतिप्रसाधनम्~॥~७~॥}}
\end{center}

\begin{sloppypar}
समायां भूमौ कागदे पट्टे वा षडङ्गुलेन सूत्रेण कर्काटकेन वा चन्द्रबिम्बं कृत्वा तस्मिन् बिम्बे प्रागपरादिक्चिह्नं कृत्वा प्रागानीतं वलनं यथादिशं देयं तच्च शुक्लपक्षे प्राग्भागे देयं कृष्णपक्षे पश्चिमभागे ततो यत्र वलनाग्रं ततः केन्द्राभिमुखं सूत्रं प्रसार्य्य तस्मिन् सूत्रे विभासूत्रकं देयं तस्मिन् वलनाग्रसूत्रे केन्द्राद्वलनाग्रगा विभा देया, एवं कदाचिद्बिम्बमुल्लिख्य दूरेऽपि विभासूत्रं भवति तस्माद्विभाग्रस्थस्वभामिते कर्काटकेन वृत्तं कर्तव्यमस्माद्वृत्ताद्बहिर्भूतमिन्दुमण्डलमिन्दुखण्डाकृतिश्चेति ज्ञेयं यद्दिक्कं वलनं तद्दिक्स्थं शृङ्गं नीचं ज्ञेयमन्यदिक्स्थमुन्नतं विद्धि जानीहीत्यर्थः~।~अथ कश्चिद्विशेषः यदा षडङ्गुलाधिका विभा भवति तदा केन्द्रात् वलनाग्र-रेखाभिमुखं विभाप्रमाणेन भूमौ वा पट्टे विभाचिह्नं कार्यं ततस्तस्य प्रभाप्रमाणेन वृत्तं कार्यमित्येतत् प्रायः शुक्लाष्टमीदिने न भवति~। अमावा-स्यायां चन्द्रार्कयोरेकत्वाद्विभाया अभावः षडङ्गुला स्वभा भवति, अष्टम्याम् अष्टादशाङ्गुला विभा स्वभा च भवति यत्र तत्र कोटेरभावस्तदा हाराभावः प्रायस्तत्र याम्यं वलनं
\end{sloppypar}

\newpage

\begin{sloppypar}
\noindent ज्ञेयं शिवपुर्यां परमवलनमङ्गुलचतुष्टयासन्नं भवति ततो यथा यथाक्षांशा उपचीयन्ते तथा तथा वलनमुपचीयते तद्दक्षिणस्यामेव भवत्युत्तरवलनं त्वेकाङ्गुलमध्ये भवति~। अन्यदुदाहरणं शके १५३९ आश्विनशुक्ले ६ शुक्रेऽब्दाः ४३४ उदयेऽहर्गणः १५८७३३ मध्यमाः सायंकालिकः सूर्यः ५।२७।२६।३२ चन्द्रः ८।१७।४४।३४ उच्चम् ५।२२।२०।२८ पातः २।०।६।४६ अयनांशाः १८।१४।३१ रवेर्मन्दफलमृणम् २।८।५ चन्द्रमन्दफलमृणम् ५।०।९ चरफलं धनम् २६ विपरीतमृणं चरपलसंस्कृतो रविः ५।२५।१७।१७ गतिः ५९।२१ चन्द्रः ८।१२।३।० गतिः ७७४।५ चन्द्रस्य क्रान्तिकला दक्षिणा १४३७।११ योधपुरेऽक्षभा ५।५ शरकला याम्या १९७।५७ शरसंस्कृता क्रान्तिर्याम्या १६३५।८ अथ सषड्भसूर्यस्य ११।२५।१७।१७ क्रान्तिः ३१६।३१ सौम्यानया पूर्वागता भिन्नदिक्त्वर्द्धिता १३०८।३७ अंशादिः २१।४८।३७ व्यर्केन्दुः २।१७।१६।४३ अस्य भुजोऽयमेवास्य ज्या ११६।२१ योधपुरस्याक्षांशैः २५।५८।४३ गुणिता ३०३०।७।२३ खार्कै\textendash \,१२०\textendash \,र्भक्ता लब्धम् २५।१५।३ अक्षांशवशाद्दक्षिणानेन २५।१५।३ पूर्वागतांशादिः १२।४८।३७ युता ४७।३।४० अथ व्यर्केन्दुभुजः २।१७।
\end{sloppypar}

\newpage

\begin{sloppypar}
\noindent १६।४३ पञ्चभि\textendash \,५\textendash \,र्गुणिता राश्यादयो जातमंशादिः १२।२६।२३ विकली-कृतम् ४४७८३ अनेन सवर्णिता पूर्वागतसर्वसंस्कारसंस्कृताः क्रान्ति-विकलाः १६९४२७ भक्ता लब्धमङ्गुलादिवलनम् ३।४६।५९ दक्षिणं व्यर्केन्दुः २।१७।१६।४३ अस्य कोटिः ०।१२।४३।१७ अस्य शरेन्दुभागस्तेन चेत्कोटौ राशिर्भवति तदा राशेरंशाः कार्याः अधोंऽशाः ज्ञेयास्तत्र पञ्चदशभिर्भाज्या अत्र राशेरभावोंऽशाद्यम् १२।४३।१७ पञ्चदशभक्तं लब्धम् ०।५०।५३ अयं हारोंऽशादिहरेण सवर्णितेन ३०५३ विकलीकृताः षट्त्रिंशदंशाः १२९६०० भक्ता लब्धमंशादि ४२।२७ इदं लब्धं द्विष्ठम् ४२।२७ एकत्रांशादिहरेण ०।५०।५३ क्रमादूनम् ४१।३६।७ एकत्र युतम् ४३।१७।५३ अनयोरर्द्धे क्रमाद्विभास्वभाख्ये विभाङ्गुलादिः २०।४८।३३ स्वभा २१।३६।५३~। अथ मासान्तपादस्योदाहरणम् शके १५३९ लौकिककार्तिककृष्ण\textendash \,१३\textendash \,शुक्रे उदये शृङ्गोन्नत्यर्थमहर्गणः १५८७५४ औदयिका मध्यमाः सूर्यः ६।१८।३।३३ चन्द्रः ५।१७।४४।५३ उच्चम् ५।२४।३७।२५ पातः २।१।१२।० अयनांशाः १८।१४।३४ चरपलानि ६६ चरपलसंस्कृताः सूर्यः ६।१७।२०।३६ गतिः ६१।२८ चन्द्रः ५।१८।३४।३० गतिः ७२२।२०
\end{sloppypar}

\afterpage{\fancyhead[RE,LO]{{\small{अ.\,८}}}}
\newpage

\begin{sloppypar}
\noindent पातः २।१।१२।० गतिः ३।११ चन्द्रस्य कलादिका क्रान्तिः ६४।३४ दक्षिणा शरो दक्षिणः २४।५० एकदिक्त्वादुभयोर्योगः ३६९।२४ सषड्भसूर्यसायनस्य क्रान्तिः ८१४।२१ उत्तरानया भिन्नदिक्त्वादन्तरम् ४४४।५७ उत्तरा व्यर्केन्दुदोर्ज्या ५७।३९ पलांशैः २४।३५।९ गुणिता १४१७।२२।२३ खार्कै\textendash \,१२०\textendash \,र्भक्ता लब्धेन ११।४८।४१ दक्षिणेन भिन्नदिक्त्वात् पूर्वागतांशादि ७।२४।५७ उभयोरन्तरम् ४।२३।४४ याम्यं व्यर्केन्दुकोट्यंशाः ०।१।१३।५४ भुजः ०।२८।४६।६ राशिभिरिन्द्रियगुणितैः ४।२३।५० सवर्णितैः १५८३० पूर्वागतमन्तरम् ४।२३।४४ सवर्णितम् १५८२४ भक्तमङ्गुलादिवलनं याम्यम् ०।५९~। हारः ३।५० विभा २।४५ स्वभा ६।३६ करणकुतूहलवृत्तौ विहितं शृङ्गोन्नतेर्नयनम्~॥~४~॥
\vspace{2mm}

\begin{center}
{\large \textbf{इति ब्रह्मतुल्यवृत्तौ शृङ्गोन्नत्यधिकारः सप्तमः समाप्तः~॥~७~॥}}
\end{center}
\vspace{2mm}

{\small \textbf{अथ ग्रहयुत्यधिकारो व्याख्यायते\textendash }}\\

तत्रादौ भौमादीनां योजनमयानि बिम्बानि लिख्यन्ते\textendash \,भौमस्य १८८५ साधिकं बुधस्य २७९ गुरोः १६६४९ शुक्रस्य १११० शनेः २९५५ भूमेः सकाशाद्यो दूरस्थस्तस्य बिम्बं सूक्ष्मं दृश्यते यस्तु भूमेरासन्नः स स्थूल इति, उक्तं च {\color{violet}उच्चस्थितो व्योमचरः सुदूरे नीचः स्थितः स्यान्निकटे धरित्र्याः~। अतोऽणु बिम्बं पृथुलं च भाति भानोस्तथासन्नसुदूरवृत्तेः~॥} इति~।
\end{sloppypar}

\newpage

\begin{sloppypar}
{\small \textbf{अथ भौमादिकानां कलामयानि बिम्बानि तत्स्पष्टत्वं चेन्द्रवज्रोपजात्यर्द्धेनाह\textendash }}

\phantomsection \label{8.1}
\begin{quote}
{\large \textbf{{\color{purple}पञ्चाङ्गसप्ताङ्कशराः पृथक्स्थाः \\
त्रिज्याशुकर्णान्तरसङ्गुणास्ताः~। \\
त्रिघ्नैः पराख्यैर्विहृताः फलोन-\\
युक्ताः पृथक्स्थास्त्रिभमौर्विकायाः~॥~१~॥}}
\vspace{1mm}

\phantomsection \label{8.2.1}
\textbf{{\color{purple}कर्णेऽधिकोने त्रिहृता भवन्ति \\
बिम्बाङ्गुलानीति कुजादिकानाम्~।}}}
\end{quote}

पञ्चकलापरिमितं ५ भौमस्य मध्यबिम्बं बुधस्य षट्कलाः ६ सप्त गुरोः ७ नव शुक्रस्य ९ पञ्च शनेः ५ ताः पञ्च कलाः पृथक्स्थाप्यास्त्रिज्या-शीघ्रकर्णयोरन्तरेण गुणिताः स्वकीयैस्त्रिगुणैः पराख्यैर्भक्ता यल्लब्धं तेन पृथक्स्था ऊनयुक्ताः कार्याः यदि त्रिज्यातः १२० अधिकः कर्णस्तदा पृथक्स्था हीनाः कार्याः यद्यल्पस्तदा युक्ताः कार्याः ताः स्फुटा ग्रहाणां बिम्बकलाः स्युस्त्रिभि\textendash \,३\textendash \,र्भक्ता लब्धं भौमादिकानां बिम्बाङ्गुलानि भवन्ति~। अथ गुरोर्मध्यबिम्बकलाः ७ त्रिज्याशीघ्रकर्णयोरन्तरेण १५।९ गुणिताः १०६।३ गुरोः पराख्यै\textendash \,२३\textendash \,स्त्रिगुणैः ६९ भक्ता लब्धम् १।३१ अनेन त्रिज्यातः कर्णस्याधिकत्वात्पृथक्स्था गुरोर्मध्यबिम्बकलाः ७ ऊना जाता गुरोर्बिम्बकलाः स्पष्टाः ५।२९ त्रिविभक्ता जातानि स्पष्टानि गुरोर्बिम्बाङ्गुलानि १।४९~। अथ शुक्रस्य मध्या बिम्बकलाः ९ पृथस्थाः ९ त्रिज्या १२० शीघ्र-कर्णयो\textendash \,८९।२३\textendash \,रन्तरेण ३०।३७ गुणिताः २७५।३३ पराख्येण ८७ त्रिगुणेन २६१ भक्ता लब्धम् १।३ त्रिज्यातः
\end{sloppypar}

\newpage

\begin{sloppypar}
\noindent १२० कर्णस्योनत्वात् पृथक्स्थाः ९ युताः १०।३ त्रिभिर्विभक्ता जातं शुक्रस्य बिम्बं स्फुटमङ्गुलात्मकम् ३।२१ एवं सर्वेषां बिम्बाङ्गुलानि कार्याणि~॥~१~॥\\

{\small \textbf{अथ युतिकालज्ञानं सार्धोपजात्याह\textendash }}

\phantomsection \label{8.2}
\begin{quote}
{\large \textbf{{\color{purple}दिवौकसोरन्तरलिप्तिकौघात् \\
गत्योर्वियोगेन हृताद्यदैकः~॥~२~॥}}
\vspace{1mm}

\phantomsection \label{8.3}
\textbf{{\color{purple}वक्री जवैक्येन दिनैरवाप्तैः \\
याता तयोः संयुतिरल्पभुक्तौ~॥ \\
वक्रेऽथवा न्यूनतरेऽन्यथैष्या \\
द्वयोरनृज्वोर्विपरीतमस्मात्~॥~३~॥}}}
\end{quote}

ययोर्ग्रहयोर्युतिर्जिज्ञासिता ताविष्टदिने स्पष्टौ कृत्वा तयोरन्तरस्य कलाः भाष्येऽत्रायनदृक्कर्म कृत्वा ततो युतिः साध्येत्युक्तं तदपि समीचीनं यत उक्तम्\textendash \,{\color{violet}दृक्कर्म कृत्वायनमेव भूयः साध्येति तात्कालिकयोर्युतिर्यत्} इति~। परमिह ग्रन्थकृता कर्मद्वयं सहैवोक्तं तेन केवलयोरेव साध्या चेद्भिन्नं भवेत्तर्हि आयनं कर्म कृत्वैव युतिः साधयितुं योग्या तत्रोक्तं च दृक्कर्मणायनभवेन न संस्कृतौ चेत्सूत्रे तदा त्वपमवृत्तजयाम्यसौम्ये~। यद्यकृते दृक्कर्मणि युतिः साध्यते सापि भवति तदा सुस्थिरं तयोरन्तरकलाः भुक्त्यन्तरेण हृता लब्धं दिनादि, अथ यदि तयोर्मध्य एको वक्री तदा तद्गत्योरैक्येन भाज्या लब्धं दिनादि स्यात् तत्र योऽल्पभुक्तिर्ग्रहः सोऽधिकभुक्तिग्रहादूनः, अथ यो वक्री स न्यूनः स्यात्तदा लब्धदिनैर्याता गता युतिर्ज्ञेया, अतोऽन्यथाल्पभुक्तौ ग्रहे वा वक्रिणि ग्रहेऽधिके 
\end{sloppypar}

\newpage

\begin{sloppypar}
\noindent गम्या ज्ञेया, अथ यदि द्वावपि वक्रिणौ स्तस्तदा विपरीतमिति तदन्तर-कला~गत्योरन्तरेण भाज्या लब्धदिनैरल्पभुक्तौ ग्रहे न्यूने गम्या, अल्प-भुक्तावधिके गतेत्यर्थः~। यथा शके १५४१ चान्द्रवैशाखकृष्ण\textendash \,१४\textendash \,रवा-वुदयेऽहर्गणः १५९२८८ गुरोः शीघ्रकेन्द्रम् १।३।९।५ शीघ्रफलम् ७।४२।३१ स्पष्टो गुरुः ११।१६।३९।७ गतिः ११।३१ शुक्रस्य मन्दफलम् १।३१।९ धनं शीघ्रोच्चम् ८।२०।४९।५२ शीघ्रकर्णः ८९।२३ स्पष्टशुक्रः ११।१६।५१।२६ गतिः ६०।५७ स्पष्टोऽर्कः १।३।६।१२ अयनांशाः १८।१६।१० चरपलम् ८६ स्थापितो गुरुः ११।१६।३९।७ शुक्रः ११।१६।५१।२६ अनयोरन्तरम् ०।०।१२।१९ अस्य सवर्णिता विकलाः ७३९ गत्योरन्तरेण ४९।२६ सवर्णितेन २९६६ भक्ता लब्धं दिनादि ०।१४।५६ अत्राल्पभुक्तिर्ग्रहो गुरुरधिकभुक्तेः शुक्रादूनस्तेनाप्तदिनादिभिर्युतिर्गता ज्ञेया~॥~३~॥\\
\end{sloppypar}

{\small \textbf{अथ युतिसाधनोपयुक्तकर्तव्यता मन्दाक्रान्तात्रयेणाह\textendash }}

\phantomsection \label{8.4.1}
\begin{quote}
{\large \textbf{{\color{purple}एवं लब्धैग्रहर्युतिदिनैश्चालितौ तौ समौ स्तः \\
कार्यौ बाणाविह शशिशरः संस्कृतोऽसौ स्वनत्या~। \\
एकान्याशौ यदि खगशरावन्तरैक्यं तयोर्यत् \\
याम्योद-}}}
\end{quote}

\newpage

\phantomsection \label{8.4}
\begin{quote}
{\large \textbf{{\color{purple}क्स्थं खचरविवरं सिद्धभक्तं कराः स्युः~॥~४~॥}}
\vspace{1mm}

\phantomsection \label{8.5}
\textbf{{\color{purple}ज्ञेयौ खेटौ निजशरदिशावेकदिक्त्वेऽल्पबाणो
\\
व्यस्ताशः स्यादितरखचरादन्तरं स्यात् स्फुटेषुः~।\\
मानैक्यार्द्धाद्द्युचरविवरेऽल्पे भवेद्भेदयोगः \\
कार्यं सूर्यग्रहवदखिलं कर्म यल्लम्बनाद्यम्~॥~५~॥}}
\vspace{1mm}

\phantomsection \label{8.6}
\textbf{{\color{purple}मन्दाक्रान्तोऽनृजुरपि रविः शीघ्र इन्दुर्विकल्प्यो
\\
नृज्योर्व्यस्तं भवति च युतोऽर्काद्विधुः सा शरांशा~।\\
लग्नादल्पे निशि दिविचरे भार्द्धयुक्तादनल्पे \\
दृश्यो योगो निजदिनगते लग्नमर्कान्न खेटात्~॥~६~॥}}}
\end{quote}

\begin{center}
{\large \textbf{इतीह भास्करोदिते ग्रहागमे कुतूहले \\
विदग्धबुद्धिवल्लभे ग्रहोत्थयोगसाधनम्~॥~८~॥}}
\end{center}

\begin{sloppypar}
एवं प्रागवाप्तैर्ग्रहयुतिर्दिनादिभिश्चालितौ यातैष्यनाडीत्यादिना गतायां युतौ हीनौ युतौ वक्रिणि ग्रहे व्यस्तमुभयोर्वक्रिणोरपि व्यस्तं गतायां युतौ युतं गम्यायां हीनं कार्यमिति कृते \hyperref[8.4.1]{समौ} राश्यादिसदृशौ स्तस्ततस्तयोः प्राग्वत् \hyperref[8.4.1]{बाणौ} शरौ कार्यौ तयोर्मध्ये वक्ष्यमाणकल्पनया चन्द्रस्य वक्ष्यमाण-प्रकारेणानीतया नत्या चन्द्रशरः भिन्नैकदिक्त्व ऊनयुतः कार्य इति संस्कार्य तौ तयोर्ग्रहयोः शरौ यद्येकदिक्कौ स्तस्तदा तयोरन्तरं कार्यं भिन्नदिक्कौ चेत्तदा तयोः शरयोर्योगः कार्यस्तदेव \hyperref[8.4.1]{याम्योदक्स्थं} दक्षिणोत्तरं खचरविवरं \hyperref[8.4]{ग्रहान्तरं} ज्ञेयं
\end{sloppypar}

\newpage

\begin{sloppypar}
\noindent तदेवान्तरं चतुर्विंशतिभि\textendash \,२४\textendash \,र्भक्तं हस्ता भवन्ति यथा पूर्वागतदिनादिभिः ०।१४।५६ चालितौ जातौ समौ गुरुः ११।१६।३६।१५ शुक्रः ११।३६।३६।१५ जातावेतत्कालीनौ पुनः सष्टौ कृत्वा शरौ साध्यावत्र स्वल्पत्वात् पुरा-कृतशीघ्रफलं मन्दफलं ताभ्यां शरौ साध्येते स्वचञ्चलफलैरित्यादिना शरः साध्यते गुरुपातः ९।८।०।० शीघ्रफलेन ७।४२।३१ व्यस्तः संस्कृतः धनत्वात् पाते हीनः जातो गुरोः स्पष्टपातः ९।०।१७।३२ स्पष्टगुरुणा ११।१६।३६।१५ युतो जातः सपातः ८।१६।५३।४४ अस्य भुजज्या ११६।२६ क्षेपेण ७६ गुणा ८८४८।५६ कर्णेन १३५ भक्ता ६५।२६ त्रिभक्ताङ्गुलाद्यः शरः २१।४८ सपातो दक्षिणगोले तेन दक्षिणशरः शुक्रस्य पातः १०।०।०।० मन्दफलेन १।३१।११ युतः १०।१।३१।११ शीघ्रोच्चेन ८।२०।४९।५२ युतः ६।२२।२१।३ भुजज्या ४५।२८ क्षेपेण १३६ गुणिता ६१८३।२८ कर्णेन ८९।२३ भक्ता ६९।१० त्रिभक्ता २३।२३ अङ्गुलादिशरो याम्यः~। अथ लक्षणान्तरं ज्ञेयमिति यस्य ग्रहस्य दक्षिणशरः स दक्षिणस्थो ज्ञेयः यस्य सौम्यः स उदक्स्थो ज्ञेयः शरयोर्दिक्साम्ये यस्याल्पशरः स इतरग्रहाद्बृहच्छरग्रहादन्यदिक्स्थो ज्ञेयः~। अथ
\end{sloppypar}

\newpage

\begin{sloppypar}
\noindent याम्योत्तरस्थयोरन्तरे स्प\hyperref[8.5]{ष्टेषुः} स्पष्टो बाणो ज्ञेयः स स्फुटशरो द्वयोर्दृश्या युतिर्भवति तस्माल्लम्बनादि साध्यम्~। उक्तं च {\color{violet}"खेटौ तौ दृष्टियोग्यौ यदि युतिसमये कार्यमेवं तदेव"} इति तल्लक्षणमग्रे वक्ष्यति~। अथ लम्बनार्थं~चन्द्र-सूर्यकल्पना द्वयोर्मार्गग्रहयोर्मध्ये यो \hyperref[8.6]{मन्दाक्रान्तो} मन्दगतिको ग्रहः स रविः कल्प्यः यदि वा यो वक्री स शीघ्रो वा मन्दो वा रवेरन्यश्चन्द्रः कल्प्य उभयोर्वक्रिणोर्व्यस्तमिति यस्तु शीघ्रः स रविरन्यश्चन्द्रः प्रकल्प्य एवं कल्पयित्वा कल्पितार्कात्कल्पितो विधुर्यत्र यस्यां दिश्युत्तरतो दक्षिणतो वा व्यवस्थितः सा दिक् शरस्य ज्ञेया यस्तु सूर्यः स छाद्यश्चन्द्रश्छादक इति~। अथ लम्बनसाधनोपायः यो युतिसमयः समागतस्ता एव दर्शान्तघटिकाः कल्प्यास्ताभ्य इष्टघटीभ्यः सषड्भसूर्याल्लग्नं साध्यं सषड्भसूर्यः कथं कृतो यतः सूर्यस्य रात्रावेव युतिर्दृश्या भवत्यतः सषड्भः कृतः कदाचित्सूर्यस्य रात्रावपि भचक्रवशाद्ग्रहस्यास्तत्वाद्युतिर्न दृश्यते ग्रहस्य सूर्यदिने त्ववश्यं न दृश्यते तल्लक्षणं वक्ष्यति ततस्तल्लग्नं सायनं वित्रिभं कार्यं ततः समकला पूर्वविधिना सूर्यं प्रकल्प्य तत्त्रिभोनलग्नकल्पितसूर्यान्तरभागेभ्यः सप्ताद्रय इति सकृत्प्रकारेण मध्यमलम्बनं स्पष्टलम्बनं कृत्वा युतिसमयघटिकासु कल्पितरविग्रहस्य प्राग्भाग ऋणं विभोनलग्नेऽहीने सत्युभयथापि तुल्यं भवति पश्चिमत्रिभोनलग्नेऽधिके सति वा धनं कार्यं कल्पितरवेः सकाशात् अग्रेतनाः षड्राशयोऽधिकास्तत्पृष्ठ-
\end{sloppypar}

\newpage

\begin{sloppypar}
\noindent भागस्थाः षड्राशयो न्यूना इति अधिकोनता ज्ञेया न त्वङ्कानां संख्यायाः प्राङ्नतेस्त्रिभोनं न्यूनमेव भवति पश्चिमनतेर्वित्रिभमधिकमेव भवत्येतत् सर्वं सूर्यग्रहणे व्याख्यातं प्रायो यो यत्र राशौ भवति तस्मादग्रेतनाः षडधिका राशयः तत्पृष्ठषड्राशय ऊना एवेति भावः~। एवं लम्बनसंस्कृतः स्फुटो भवति स एव सायनः लम्बनसंस्कृतकालान्न्यूनं वित्रिभं लग्नं कृत्वा नतांशाः, कर्तव्यास्तन्नतांशान् शीघ्रग्रहस्य मध्यगतिपञ्चदशांशेन सङ्गुण्य त्रिज्यया विभजेत् सा कलादिका नतिः सा पुनः सार्द्धद्वयेन भक्ता सत्यङ्गु-लादिनतिर्भवति तया चन्द्रशरः संस्कार्यः~। उक्तं च {\color{violet}"दृक्क्षेप इन्दोर्निजमध्य-भुक्तिस्तिथ्यंशनिघ्नौ त्रिगुणोद्धृतौ तौ~। नती रवीन्द्वोरि}ति"~। यथा युति-समये रविरात्रिशेषघटी १४।५६ प्रमाणः लम्बनार्थमेतत्कालीनसूर्यः १।२। ५१।५४ इष्टकालः १४।५६ अयनलग्नं तच्छुद्धसूर्यात् साध्य न तु कल्पित-क्रान्तिस्तच्छुद्धसूर्यात् सुखार्थमुत्क्रमलग्नं सायनम् ९।२६।२।३९ अस्य क्रान्तिः १०।२०।१६ दक्षिणा नतांशाः ३४।५५।२५ उन्नतांशाः ५५।४।३५ उन्नतज्या ९८।५ मन्दो गुरुस्तेन सूर्यः कल्पितः शीघ्रः शुक्रः सचन्द्रः सायनो युतिसमयिको गुरुः ०।४।५२।२५ वित्रिभम् ६।२६।२१।३९ अनयोरन्तरम् ५।८।३०।४६ भुजः ०।२१।२९।१४ सप्ताद्रय इत्यनेन मध्यमलम्बनम्
\end{sloppypar}

\newpage

\begin{sloppypar}
\noindent २।१० स्पष्टलम्बनम् १।५२ इदं ग्रहस्य शेषराशित्वात् प्राङ्नतं प्राग्लक्षणेन वित्रिभमपि न्यूनं तेनर्णं युतिसमये गता युतिः शनिवार उदयाद्गतघट्यः ४५।४ मध्ये ऋणम् ४३।१२ स्पष्टो युतिसमयो नत्यर्थमेतत्कालीनः सूर्यः १।३।४३।१२ सायनार्कः १।२१।१६।१७ वित्रिभम् ७।११।१७।३९ पूर्ववन्नतांशाः ४०।२२ कल्पितचन्द्रमध्यगतिः ५९।८ तिथ्यंशेन ३।० गुणिताः १५८।८ त्रिज्यया भक्तं लब्धं फलम् १।१९ नतिः सार्द्धद्वयेन २।३० भक्ता अङ्गुलाद्या नतिर्याम्या ०।३१ अनया चन्द्रशरो याम्यः २३।३ संस्कृतः २३।३४ विशरयोः २३।४८ एकदिक्त्वादन्तरं याम्यम् १।४६ इदं याम्योत्तरं स्पष्टबाणश्चैकदिक्त्वादल्पशरो गुरुरुत्तरे शुक्रात् मानैक्यार्द्धात् २।३५ ऊनः स्पष्टबाणस्तेन भेदयोगः परं भचक्रवशाद्गुरोर्ग्रहस्य रात्रिस्तेनैतत्समये युतिर्न दृश्यते, निशीति सूर्यरात्रौ ग्रहयुतिकालीनलग्नाद्ग्रहेऽल्पे सति भार्द्धयुक्तात्सषड्भलग्नादनल्पे बहुतरे ग्रहे सति तत्रापि निजनिजगते ग्रहस्य दिने न तु ग्रहस्य रात्रौ योगो युतिर्दृश्या ज्ञेयान्यथा नेति भावः~। अत्र ग्रहः ०।४।५५।२५ इष्टलग्नात् ९।२६।२१।३९ अधिकः सषड्भात् ३।२६।२१।३९ न्यूनस्तेन युतिर्न दृश्यातोऽन्यत्कर्म न कृतं पुनरुदाहरणान्तरं शाके १५४१ फाल्गुनशुद्धे १३ सोम उदये गताब्दाः ४३६ अहर्गणः १५९६२५ मध्यमाः गुरुशीघ्रफलम् ४।३९।३१
\end{sloppypar}

\newpage

\begin{sloppypar}
\noindent ऋणं रविः ११।१६।६।२८ गुरुः ८।०।५९।५८ शुक्रः १।६।२३।४० शुक्रमन्द-फलम् १।२९।४५।४० कर्णः १८१।० अयनांशाः १८।१६।५७ चरपलम् ६४ दिनमानम् २९।४८ रात्रिमानम् ३०।१२ चरपलसंस्कृतः सूर्यः ११।८।१४।१९ गतिः ५९।४१ गुरुः ०।१।१९।१३ गतिः १३।२७ शुक्रः ०।१।५७।५७ गतिः ७३।४ उभयोः पूर्ववद्युतिदिनादि ०।३८।१४~। गतं तेन फाल्गुनशुद्धे~१२ रवावुदयाद्घटी २१।४६ समये युतिरत्र समयिका मध्यमाः स्पष्टाः कार्याः अत्र स्वल्पत्वाद्गत्या चालिता पलादि तदेव गृहीतं युतिसमयिकः सूर्यः ११।७।३६।१८ गुरुः ०।१।१०।५८ शुक्रः ०।१।१०।५५ पूर्ववच्छरौ गुरु-शरोऽङ्गुलादिः २०।५६ शुक्रशरोऽङ्गुलादिः ११।३४ गुरोर्बिम्बम् १।३८ शुक्र-बिम्बम् २।१८ लम्बनार्थं युतिसमयिकं सायनांशवित्रिभम् १।१२।५६।४१ पूर्ववदस्योन्नतांशाः ८१।२६।५२ ज्या ११८।१७ कल्पितसायनरविः ०।१९। २७।४८ वित्रिभयोरन्तरम् ०।२३।२८ खण्डकैर्मध्यमलम्बनम् २।२८ स्पष्टम् पश्चिमनतत्वाद्वित्रिभाधिक्याद्युतौ २१।४६ धनम् २३।४८ स्पष्टो युतिसमय एतत्कालीनार्कः ५।२५।५६ दिनशेषम् ०।६ लग्नवित्रिभम् २।२३।३६।५५ नतांशाः ५।३७।३७ चन्द्रमध्यमगतिस्तिथ्यंशेन ३।५७ गुणा २२।१३ त्रिज्या १२० लब्धम्
\end{sloppypar}

\newpage

\begin{sloppypar}
\noindent ०।१२ सार्द्धद्वयेन २।३० भक्तं लब्धम् ०।४ याम्यं चन्द्रशरः ११।३४ याम्यः संस्कृतः ११।३८ याम्योत्तरमन्तरम् ९।१८ स्पष्टबाणः सिद्ध\textendash \,२४\textendash \,भक्ते हस्तादि ०।९।१८ शुक्र उत्तरे मानैक्यार्द्धम् १।५८ शरादधिकं तेन भेद-योगो नास्ति परं नतादि उदाहरणार्थं कल्पितशरोऽङ्गुलादि १।० सूर्य-ग्रहणवत्साधनं मानैक्यार्द्धम् १।५८ शरेणोनः ०।५८ छन्नं द्विघ्नाच्छरात् इत्यादिना स्थितिघटिकाः ५।५६ अनया स्पष्टयुतिसमयः २३।४८ उभयोर्युतिः स्पर्शमोक्षकालौ, उदयाद्गतघट्यः १८।४२ स्पर्शः, उदयाद्गतघटीसमये २८।४४ मोक्षकालः~। अथ स्पष्टार्थं सूर्यगुरू स्पर्शकालीनौ सूर्यः ५।२५। ५०।१२ गुरुः ०।०।१९।२० वित्रिभलग्नम् ०।२३।१३।३३ उन्नतांशाः ७४।३३। ५१ ज्या ११५।१६ गुरुरविकल्पितवित्रिभलग्नान्तराखण्डकैः स्पष्टलम्बनम् ०।२४ पश्चिमनतत्वाद्धनं ग्रहस्य मध्याह्नासन्नत्वात्स्वल्पं स्पष्टः स्पर्शः १९।६ एतत्कालीनलग्नम् ३।२८।५६।४०~। अथ ग्रहस्य दिनमानार्थम् उदयलग्नं वित्रिभो ग्रहः सायनः १।१९।२७।४ कान्तिः २२।१९।५१ याम्या नतांशाः ४६।५५।० उन्नतांशाः ४३।५।० नतज्या ८७।२२ उन्नतज्या ८१।३७ क्षेपघ्नी नतशिञ्जिनीत्यादिना फलम् ६७।१३ कलाधनं दृक्कर्मसंस्कृतः ०।२।२८।४० उदयलग्नम्~।
\end{sloppypar}

\newpage

\begin{sloppypar}
\noindent अथास्तलग्नार्थं सत्रिभसायनो ग्रहः ३।१९।२७।४८ क्रान्तिरुत्तरा २२।१९। ५१ पूर्ववत् दृक्कर्मफलमृणम् २।२९ दृक्कर्मसंस्कृतः ०।१।८।२२ अस्तलग्नात् सषड्भः ६।१।८।२२ इदमस्तलग्नं सायनोदयलग्नम् ०।२०।३५।१ अस्य भोग्यमस्तलग्नस्य भुक्तम् २।२५ मध्योदयाः १५७७ एषां योगो गुरोर्ग्रहस्य दिनमानम् ३१।२ षष्टेः शुद्धम् २८।५८ रात्रिमानमथवा सायनग्रहं सूर्यं प्रकल्प्य चरखण्डकैः दशगजदशेत्यादिनोत्पन्नैश्चरखण्डकैः पलानि प्रसाध्य चरपलयुतोनेत्यादिना दिनमानं साध्यं यथा सायनो ग्रहः ०।१९।४७।४८ चरखण्डकैः ५५।४४।१८ चरपलैरुत्तरैः ३५ दिनमानम् ३१।१० अथ स्पर्शनतार्थं दिनगतघटिकानयनं स्पर्शकालीनेष्टलग्नम् ३।२८। ५६।४० उदयलग्नम् ०।२०।२५ अनयोरन्तरकाल ऊनस्य भोग्योऽधिक-भुक्तयुक्तो मध्योदयाढ्य इत्यादिना भोग्यमुदयलग्नस्य भुक्तम् ३।२९ मध्योदयानां ५६० योगाद्घटी १६।० गुरोर्दिनगतघटिका द्युदलगतघटीनाम् इत्यादिना दिनगतम् १६।० दिनार्द्धम् १५।३१ अनयोरन्तरं नतम् ०।२९ पश्चिमनतं खाङ्काहतमित्यादिना द्युदलं ग्रहदिनार्द्धम् १५।३१ ग्राह्यं जातं मोक्षवलनं दक्षिणम् १।१२ मध्याह्नासन्नत्वादल्पं स्पर्शकालीनग्रहः ०।१९। २७।१३ कोटिज्या ११३।६ आयनं सौम्यम् २२।३९ मानैक्यार्द्धम् १।५८ स्पष्टवलनमुत्तरम् ०।४३ अथ
\end{sloppypar}

\afterpage{\fancyhead[RE,LO]{{\small{अ.\,९}}}}
\newpage

\begin{sloppypar}
\noindent मोक्षकालीनाः २८।५४ सूर्यः ५।२६।०।२० गुरुः ०।१९।२९।२२ वित्रिभलग्नम् २।२१।८।२७ अस्य क्रान्तिः २३।१९।१७ पूर्ववदुन्नतज्या ११९।४६ पूर्ववत्स्पष्टं वलनं धनम् ३।५७ ग्रहस्य साध्यत्वात् परमं स्पष्टो मोक्षकालः ३२।५१ उदयाद्गतघटी ज्ञेया एतत्कालीनोऽर्कः ५।२६।४।१५ गुरुः ०।१९।३०।१४ रात्रिगतघटी ३०।३ क्रमलग्नम् ६।१२।३६।४५ उदयलग्नेऽ\textendash \,०।३।३५।१\textendash \,ष्ट\textendash \,६।१२।३६।४५\textendash \,पूर्ववदन्तरकालः २९।४६ ग्रहस्य दिनगतघट्यः प्राग्वन्नतम् १४।४५ पश्चिमं खाङ्काहतमित्यादिना द्युदलेन ग्रहस्य १५।३१ मोक्षवलनं याम्यम् २४।१९ संध्यत्वात्परमं मोक्षकालीनग्रहः ०।१९।३०।१४ कोटिज्या ११३।१४ आयनं सौम्यम् २२।३८ स्पष्टं मोक्षवलनं याम्यम् ०।३० एवं भेदयोगे कर्तव्यताप्रकारो ज्ञेयः~। अथ यो ग्रहश्चन्द्रः कल्पितः स चेदल्पभुक्तिर्वक्रो भवति तदा प्राच्यां दिशि स्पर्शः प्रतीच्यां मोक्षः, अधिकभुक्तिस्तथा मार्गी चेत्तदा प्रतीच्यां स्पर्शः प्राच्यां मोक्षः~। करणकुतूहलकुमुदकौमुद्यां समाप्ति-युतिसाधनं स्पष्टम्~।
\end{sloppypar}
\vspace{2mm}

\begin{center}
{\large \textbf{इति श्रीकरणकुतूहलवृत्तौ युत्यधिकारोऽष्टमः~॥~८~॥}}
\end{center}
\vspace{2mm}

{\small \textbf{अथ पाताधिकारो व्याख्यायते\textendash ~तत्रादौ पीठिका लेख्यते\textendash }}

\newpage

\begin{sloppypar}
शाके १५३९ लौकिककार्तिककृष्णे १० भौमे गताब्दाः ४३४ अहर्गणः १५८७५१ उदयकाले मध्यमा योधपुरे स्पष्टास्तत्र रविः ६।१२।४५।३२ चन्द्रः ४।१२।२।४७ अयनांशाः १८।१४।३४ पातः २।१।२।३१ सायनार्कः ७।१।०।६ चन्द्रः ५।०।१७।२१~॥\\

{\small \textbf{अथ प्रस्तुतमारभ्यते पातसम्भवं गतगम्यज्ञानं च वंशस्थेनेन्द्रवज्रयार्द्धेनाह\textendash }}

\phantomsection \label{9.1}
\begin{quote}
{\large \textbf{{\color{purple}विना सपातैन्दुमिहायनांशकैः \\
युतो रविः शीतरुचिश्च गृह्यते~। \\
समायनत्वे व्यतिपातवैधृता-\\
ह्वयस्तदैक्ये रसभेऽर्कभे क्रमात्~॥~१~॥}}
\vspace{1mm}

\phantomsection \label{9.2.1}
\textbf{{\color{purple}पातस्तदूनाधिकलिप्तिकाभ्यो \\
भुक्त्यैक्यलब्धैष्यगतैरहोभिः~।}}}
\end{quote}

इह पातसाधने सपातेन्दुं विना रविचन्द्रश्चायनांशैर्युत एव गृह्यते यत्र सपातचन्द्र इति नोक्तं तत्रायनांशयुक्तो रविश्चन्द्रश्च ग्राह्यस्तदैक्ये तयोः सायनांशयो रविचन्द्रयोर्योगे \hyperref[9.1]{रसभे} षड्राशितुल्येऽ\hyperref[9.1]{र्कभे} द्वादशराशितुल्ये \hyperref[9.1]{व्यतिपातवैधृताह्वयौ} क्रमात्स्यातां यत्र षड्राशितुल्यो योगस्तत्र व्यतिपातनामा पातः यत्र द्वादशराशितुल्यो योगस्तत्र वैधृतनामा पातः स्यात्, क्व सति \hyperref[9.1]{समायनत्वे} सति यदा सूर्यचन्द्रयोः समक्रान्ती भवतस्तदेत्यर्थः~। समक्रा-न्तित्वे पातसम्भवो ज्ञेय इत्यर्थः~। तदूनेति\textendash \,तयोर्योगे षड्राशिभ्यस्तथा द्वाद-शराशिभ्यश्चोना अधिका वा लिप्ताः
\end{sloppypar}

\newpage

\begin{sloppypar}
\noindent कलास्ता रविचन्द्रयोर्भुक्तियोगेन भाज्या लब्धं दिनादिकं ग्राह्यम् ऊनासु कलासु भोग्यं दिनादिकं ज्ञेयमधिकासु गतं दिनादिकं ज्ञेयम्~॥~१~॥\\

{\small \textbf{अथ सपातस्य गतैष्यज्ञानं सार्द्धेन्द्रवज्रयाह\textendash }}

\phantomsection \label{9.2}
\begin{quote}
{\large \textbf{{\color{purple}तात्कालिकौ तौ च तमश्च कृत्वा \\
प्राग्वत् प्रसाध्यो विशिखः कलादिः~॥~२~॥}}
\vspace{1mm}

\phantomsection \label{9.3}
\textbf{{\color{purple}ओजे पदे युग्मपदे विधुश्चेत् \\
एकान्यगोलश्च सपातचन्द्रात्~। \\
ज्ञेयस्तदानीं खलु यातपातो \\
गम्योऽन्यथात्वेन ततोऽपि कालात्~॥~३~॥}}}
\end{quote}

\hyperref[9.2]{तात्कालिका}विति~। तैः पूर्वागतैरेष्यगतदिनादिभि\hyperref[9.2]{स्तौ} चन्द्रार्कौ~\hyperref[9.2]{तमश्च} पातं च यातैष्यनाडीगुणितेत्यादिना तात्कालिकान् कृत्वा प्राग्वत् खण्डकेभ्यः \hyperref[9.2]{कलादिविशिखः} शरः साध्यः~। अथ गतगम्यलक्षणं चेद्यद्योजपदे सायनो विधुः स्थित्वा सपातचन्द्रादेकगोले भवति तथा युग्मपदे स्थित्वा,~सपात-चन्द्रादन्यगोले भवति तदा प्रागागतात् कालाद्यस्मिन् काले द्वादश~पद-राशयो जातास्तस्मादित्यर्थः वक्ष्यमाणकालेन गतपातो ज्ञेयः, अन्यथा तु चन्द्रो विषमपदे स्थित्वा सपातचन्द्रादन्यगोले तथा युग्मपदे स्थित्वै-कगोले तदा पूर्वागतकालादेवैष्यः पातो ज्ञेयः~। सायनांशो रविः ७।१।०।६ सपातचन्द्रः ५।०।१७।२१ अनयोर्योगः १२।१।१७।२७ अथ द्वादशराशि-तोऽधिकस्तेनाधिकमंशादि
\end{sloppypar}

\newpage

\begin{sloppypar}
\noindent १।१७।२७ कला ७७।२७ चन्द्रार्कभुक्तियोगेन ८०१।५८ भक्ते लब्धं दिनादि ०।५।४७ द्वादशभ्योऽधिकत्वाल्लब्धदिनादिभिः ०।५।४७ गतः पातः~। अथ तात्कालिककरणम्~। नवम्यां शेषरात्रिघटी ५।४७ यातैष्यनाडीत्यादिना तात्कालिकोऽर्कः ६।१२।३९।४५ चन्द्रः ४।१०।५१।१७ पातः २।१।२।१२ सपातचन्द्रात्खाश्वाः शराङ्गानि खण्डकेभ्यः सायनोऽर्कः ७।०।५४।९ चन्द्रः ४।२९।५।५१ योगः ०।०।०।० कलादिशरो याम्यः ५७।२९ सपातचन्द्रः ६।११।५३।३० याम्यगोले सायनचन्द्रोऽत्र ४।२९।५।२१ समपदे द्वितीयपद उत्तरगोले तेन समपदत्वात्सपातचन्द्रभिन्नगोलत्वाद्गतः पातः पूर्वागतादपि ०।५।४७ वक्ष्यमाणैः स्पष्टादिभिरिति~॥~३~॥\\

{\small \textbf{अथ पातस्य गतैष्यकालसाधनार्थं संज्ञामिन्द्रवज्रार्द्धेनाह\textendash }}

\phantomsection \label{9.4.1}
\begin{quote}
{\large \textbf{{\color{purple}क्रान्तीषुखण्डानि धनं क्रमेण \\
व्यस्तानि तानि स्वमृणं प्रकल्प्यम्~।}}}
\end{quote}

क्रान्तिखण्डानि चेषुखण्डानि तानि क्रमात् षट्क्रान्तिखण्डानि धनसंज्ञानि तान्येव षडुत्क्रमाणि ऋणसंज्ञानि, पुनरपि षट्क्रमाद्धनसंज्ञानि तान्युत्क्रमाणि पुनः षड्-ऋणसंज्ञानि कल्प्यानि, एवं चतुर्ष्वपि पदेषु गणनाधोऽधः संस्थाप्य
\end{sloppypar}

\newpage

\begin{sloppypar}
\noindent कार्या, एवं शरखण्डकानामपि स्थापना धनर्णसंज्ञा चतुर्ष्वपि पदेषु कार्या~॥\\

{\small \textbf{अथ साधनमिन्द्रवज्रार्द्धेनाह\textendash }}

\phantomsection \label{9.4}
\begin{quote}
{\large \textbf{{\color{purple}चन्द्रस्य पातेन्दुयुतस्य भागाः\\
तिथ्युद्धृताः स्युर्गतखण्डकानि~॥~४~॥}}}
\end{quote}

सायनचन्द्रस्य तथा च सपातचन्द्रस्य सर्वस्य च ये भागास्ते पृथक् पञ्चदशभिर्भाज्या लब्धमुभयत्र स्वस्वगतखण्डकानि ज्ञेयानि तेषां पूर्वोक्त-धनर्णसंज्ञितस्थापितखण्डकेभ्यः गणनात् क्रमादुत्क्रमात् क्रमादुत्क्रमाच्च कार्या शरस्य खण्डार्थं सपातचन्द्रः ६।११।५३।३० भागाः १९१।५३।३० पञ्चदश १५ भक्ता लब्धम् १२ शेषम् ११।५३।३१ यथा सायनचन्द्रस्य ४।२९।५।२१ भागाः १४९।५।२१ तिथिभक्ता लब्धम् ९ गतखण्डकानि शेषम् १४।५।२१ क्रान्तिखण्डतो गणना कृता दशमं खण्डं भोग्यं द्वितीयपदस्य चतुर्थखण्डम् २९९ ऋणसंज्ञकं शरखण्डकेषु तु त्रयोदशं शरखण्डकं तेषु शरखण्डकेषु गणना कृता तृतीयपदस्य प्रथमखण्डं ७० धनसंज्ञकमथ भक्तशेषमुभयत्र क्रान्तिशेषम् १४।५।२१ शरशेषम् ११।५३।३०~॥~४~॥\\
\end{sloppypar}

{\small \textbf{अथ साधनमुपजातीन्द्रवज्राद्वयेन चोपजातीन्द्रवज्रयाचष्टे\textendash }}

\phantomsection \label{9.5.1}
\begin{quote}
{\large \textbf{{\color{purple}क्रमोत्क्रमात्तद्गणना च कार्या \\
चापाह्वयाः शेषल-}}}
\end{quote}

\newpage

\phantomsection \label{9.5}
\begin{quote}
{\large \textbf{{\color{purple}वा व्यतीते~। \\
पातेऽथ गम्ये तिथितश्च्युतास्ते \\
द्विधा द्विधा भोग्यदलादिकानि~॥~५~॥}}
\vspace{1mm}

\phantomsection \label{9.6}
\textbf{{\color{purple}द्वित्रीणि विन्यस्य पृथग्दलानि \\
गम्यानि गम्येऽथ गते गतानि~। \\
एकस्थमेवास्य तु भोग्यखण्डं \\
यस्याल्पकाश्चायलवा भवन्ति~॥~६~॥}}
\vspace{1mm}

\phantomsection \label{9.7}
\textbf{{\color{purple}विश्वांशकेनापमभोग्यकस्य \\
भोग्यादितः क्रान्तिदलानि तानि~। \\
संस्कृत्य पूर्वं शरखण्डकैश्च \\
स्युः संस्कृतानि क्रमशः स्फुटानि~॥~७~॥}}}
\end{quote}

\begin{sloppypar}
गते पाते उभयत्र ये शेषभागास्ते \hyperref[9.5.1]{चापाह्वया}श्चापांशका ज्ञेयाः~। अथ गम्ये पाते पूर्वागताः शेषांशा पञ्चदशभ्यः १५ शुद्धाः शेषं चापांशसंज्ञका उभयत्र ज्ञेयम्~। एवं क्रान्तेः शरस्य च चापांशान् विधायैकान्ते स्थापयेद्यथात्र गतपातत्वाच्छेषांशका एव चापांशा एव चापांशाः क्रान्तेः १४।५।२१ शरस्य च ११।५३।३० ततः क्रान्तेः शरस्य च भोग्यखण्डमादीकृत्य द्वित्रीणि खण्डानि द्विधा पृथग्विन्यसेत्, यदि पातं गतलक्षणं भवति तदा गतखण्डानि एष्यलक्षणे पाते एष्यलक्षणानि खण्डानि विन्यसेत्, शरक्रान्त्योर्मध्ये यस्य चापांशाः स्वल्पास्तस्य भोग्यखण्डमेकस्थमेव स्याप्यमन्यानि द्विधा द्विधा यथागतानि तेषां स्थापितखण्डानां धनर्णसंज्ञा पूर्वं कृतैव स्वल्पशरे खण्डकं द्वयं महच्छरे त्रय एव मध्यमा भवन्ति~।
\end{sloppypar}

\newpage

\begin{sloppypar}
\noindent अनया रीत्या संस्थाप्य स्फुटानि कार्याणि अत्र गतपातत्वाद्भोग्यखण्डम् आदीकृत्य गतखण्डकानि द्विधा स्थाप्य तानि शरस्य स्वल्पचापांशत्वात् शरस्य भोग्यखण्डमेकत्र स्थापितं शेषाणि गतखण्डकानि द्विधा स्थापितानि धनर्णसंज्ञा पूर्वं कृतैव यन्त्रतो ज्ञेया~। अथ स्पष्टक्रिया~। विश्वांशकेनेति~। अपक्रमस्य क्रान्तेर्भोग्यखण्डस्य \hyperref[9.7]{विश्वांशकेन} १३ त्रयोदशांशकेन~भोग्या-दीनि क्रान्तिखण्डानि पूर्वं संस्कृत्यैकजात्योर्धनरूपयोः ऋणरूपयोः वान्तरमन्यजात्योर्धनर्णरूपयोर्योग एवमेकगोले वैपरीत्यमेकजात्योर्योगो भिन्नजात्योर्वियोगो ग्रन्थेऽनुक्तं भाष्य उक्तत्वात् कार्यम्, एवं~संस्कृतानि क्रान्तिखण्डानि स्फुटानि स्युः पुनः शरखण्डकैः संस्कृतानि~स्फुटतराणि भवन्ति, संस्कारास्त्वेकजात्योर्योग एवमन्यगोले त्वेकजात्योर्योगोऽन्य-जात्योरन्तरमिदमेव सिद्धं भाष्य उक्तत्वात्~। यथा क्रान्तिभोग्यखण्डस्य विश्वांशकेन १३ शरांशकेनर्णरूपेणान्यान्यृणरूपाण्यन्यगोलत्वाद्युक्तानि, यथा विश्वांशकेन युक्ता शरखण्डकैर्भिन्नगोलत्वादेकजात्योरन्तरं भिन्न-जात्योर्योगः~॥~७~॥\\
\end{sloppypar}

{\small \textbf{अथ पातमध्यानयनमुपजातीन्द्रवज्रयाचष्टे\textendash }}

\phantomsection \label{9.8}
\begin{quote}
{\large \textbf{{\color{purple}आद्येऽल्पचापांशमितो गुणः स्यात् \\ चापान्तरांशाः समखण्डकेषु~। \\
तिथिच्युतास्ते विषमेषु जह्यात् \\
स्वांशघ्नखण्डानि तिथिघ्नबाणात्~॥~८~॥}}
\vspace{1mm}

\phantomsection \label{9.9.1}
\textbf{{\color{purple}शेषं त्वशुद्धेन हृतं लवाद्यं \\
संशुद्धखण्डांशयुतं विभक्तम्~।}}}
\end{quote}

\newpage

\phantomsection \label{9.9}
\begin{quote}
{\large \textbf{{\color{purple}गत्याविधोः षष्टिगुणं गतैष्यैः \\
लब्धैर्दिनैः स्यात् खलु पातमध्यम्~॥~९~॥}}}
\end{quote}

\begin{sloppypar}
आद्येऽल्पचापांशमितो गुणः स्यादुभयोश्चापांशान्तरेण विचतुर्थे~सति सम्भवे षष्ठं खण्डं गुणयेत्ते चापांशसमखण्डगुणरूपाः पञ्चदशभ्यः संशोध्य शेषेण तृतीयपञ्चमखण्डं गुणयेदेवं सर्वाणि सङ्गुण्य स्फुटानि च स्वांशघ्न-संज्ञकानि भवन्ति~। एवं गुणकल्पनायां कृतायां किं कार्यमित्यत आह\textendash \,जह्यादिति\textendash \,तानि स्वांशघ्नखण्डानि पञ्चदशघ्नबाणात् पञ्चदशगुणितचन्द्र-शरकलामध्ये यावन्ति शुद्धयन्ति तावन्ति शोधयेत्, शेषमशुद्धेन स्फुट-खण्डकेन भजेन्नतु स्वांशघ्नेन फलं लवाद्यं ग्राह्यं तद्विशुद्धखण्डांशयुते यावन्ति खण्डानि शुद्धानि तेषां ये गुणाश्चापांशोद्भवास्तेषां योगं कृत्वा तैरंशैर्युतं कुर्यात् ततः षष्ट्या सङ्गुण्य चन्द्रगत्या भजेद्दिनादिकं ग्राह्यं पूर्वं चेत्पातस्य लक्षणं गतमागतं तदा यस्मिन् काले द्वादशराशयो जाताः तस्मात्कालादधुनागतैर्दिनादिभिर्गतैः पातमध्यं स्यादित्यर्थः~। अथ चेदेष्य-लक्षणं तदा पूर्वोक्तकालादेतावद्भिर्दिनैर्गम्यैः पातमध्यं स्यादित्यर्थः~। अथ तिथिघ्नबाणस्य ८३२।१५ स्वल्पत्वात् खण्डानि न शुद्ध्यन्ति तेन खण्डकानां गुणकान् कृत्वानेन शेषम् ८३२।१५ अशुद्धस्फुटखण्डकेन ३९२ भक्तं लब्धं लवादि ०।२७।२३ गुणकाभावाच्छुद्धखण्डांशानामिति ताः षष्टिगुणाः चन्द्रगत्या ७४१।५० भक्तं लब्धं दिनादि
\end{sloppypar}

\newpage

\begin{sloppypar}
\noindent ०।१०।१८ एभिः पूर्वकलादपि ५।४७ गतपातत्वात् पातमध्यं गतमेवं कार्तिककृष्णे नवम्यां शेषरात्रिघटी १६।५ समये पातमध्यं ज्ञेयं पात-मध्यसमयिकाद्रविचन्द्रपातान् तात्कालिकान् कृत्वा रवेः क्रान्तिः साध्यो-भयोः क्रान्तिसाम्यं तदा पातमध्यं शुद्धं नान्यथा यथा रविः ६।१२।२६।४२ चन्दः ४।८।१०।१० पातः २।१।१।३२ सूर्यक्रान्तिः १२।५४।३१ सूर्यक्रान्तिः १२।५४।३२ उच्चं शरः ४८।४८ कलादि दक्षिणं शरसंस्कृता क्रान्तिः १२।११।४३ सूर्यकान्तिः ११।५६।४२ उभयोः साम्यं स्वल्पत्वान्न दोषाय~॥~९~॥\\

{\small \textbf{अथ विशेषमाह प्रमाणिकया\textendash }}

\phantomsection \label{9.10}
\begin{quote}
{\large \textbf{{\color{purple}अपक्रमस्य भोग्यकं यदेषुखण्डतश्च्युतम्~। \\
गतैष्यतो विपर्ययात्तदात्र पातसाधने~॥~१०~॥}}}
\end{quote}

चन्द्रक्रान्तेर्यद्भोग्यखण्डं तद्यदा शरखण्डाद्भोग्याच्छुद्ध्यति तदा गतैष्यस्य व्यत्ययत्वं स्याद्गतः पात एष्यो ज्ञेय एष्यो गतो ज्ञेयः~॥~१०~॥\\

{\small \textbf{अथ स्थितिसाधनं प्रमाणिकयाह\textendash }}

\phantomsection \label{9.11}
\begin{quote}
{\large \textbf{{\color{purple}अशुद्धखण्डभाजितास्त्रिखाश्विदस्रनाडिकाः~। \\
स्थितिश्च मध्यपूर्वतोऽग्रतोऽपि तत्प्रमाणिका~॥~११~॥}}}
\end{quote}

त्रिखाश्वि\textendash \,२०३\textendash \,हताः नाडिकाः पूर्वागतेनाशुद्धखण्डेन भाज्या लब्धं घट्यादि मध्यकालात् पूर्वं स्थितिस्तत्प्रमा-
\end{sloppypar}

\newpage

\begin{sloppypar}
\noindent णिका तावदेव मध्यकालादग्रतोऽपि स्थितिर्भवति यथा २२०३ अशुद्धखण्डेन ३२९ भक्ता लब्धम् ५।३७ स्थितिः गतपातत्वान्मध्यकालमध्ये १६।५ हीने पातान्तः १०।२८ युते पातादिः १२।४२ उदयात् गतघटी ३८।१९~। समये स्पर्शः, उदयाद्गतघटी ४३।५५ समये पातमध्य उदयाद्गतघटी ४२।३२ समये पातमोक्षः~॥~११~॥\\

{\small \textbf{अथ प्रमाणिकाद्वयेन सकलशुद्धखण्डविवक्षामाह\textendash }}

\phantomsection \label{9.12}
\begin{quote}
{\large \textbf{{\color{purple}यदाखिलेषु खण्डकेष्विहाद्यखण्डजातिषु~। \\
च्युतेष्वपीह शेषकं खनागसागराधिकम्~॥~१२~॥}}
\vspace{1mm}

\phantomsection \label{9.13}
\textbf{{\color{purple}तदा न पातसम्भवो यदास्ति सम्भवस्तदा~। \\
विशुद्धखण्डभागतो गतैष्यकालताधनम्~॥~१३~॥}}}
\end{quote}

यदा पञ्चदशगुणितानां मध्ये स्वगुणकगुणितेष्वखिलेष्विह च्युतेषु शुद्धेषु सत्सु कथम्भूतेषु खण्डेष्वाद्यखण्डजातिष्वाद्यमिति क्रमेण धनरूपेषु षट्सु च्युतेषु खण्डेष्वथवा क्रमेण रूपेषु षट्सु च्युतेषु बाणशेषं खनागसागरेभ्यो ४८० यद्यधिकं भवति तदा पातसम्भवो नास्ति यदा सर्वेषु खण्डेष्वशुद्धेषु बाणशेषकं खनागसागरेभ्यो\textendash \,४८०\textendash \,ऽल्पं भवति तदा सम्भवोऽस्ति~। अथैवं विधिपातसम्भवे गतगम्यकालसाधनमाह\textendash \,\hyperref[9.13]{विशुद्धखण्डभागतः}~शरमध्ये यानि गुणकगुणितानि खण्डकानि शुद्धानि तेषां गुणकभागादि प्रतिखण्डा-नामेकीकृत्य संशुद्ध-
\end{sloppypar}

\newpage

\begin{sloppypar}
\noindent खण्डं शरयुतं विभक्तं गत्या विधोः षष्टिगुणमित्यनेन प्रकारेण गतैष्यसाधनं कार्यं गतैष्यतालक्षणं प्राग्वत्~॥~१३~॥\\

{\small \textbf{अथ सर्वखण्डकेषु शुद्धेषु पातसम्भवे स्थितिसाधनमाह\textendash }}

\phantomsection \label{9.14}
\begin{quote}
{\large \textbf{{\color{purple}तथा शरावशेषकं खनागवेदतश्च्युतम्~। \\
नवघ्नमन्त्यखण्डहृत् दलीकृतं स्थितिस्तदा~॥~१४~॥}}}
\end{quote}

येन शरेण पातसम्भावस्तच्छेषं खनागवेदतः ४८० \hyperref[9.14]{च्युतं} संशोध्य यच्छेषं त\hyperref[9.14]{न्नवघ्नं} नव\textendash \,९\textendash \,गुणं कार्यमन्त्यखण्डकेन भाज्यमाप्तस्य फलस्य घट्यादिकस्यार्द्धं \hyperref[9.14]{स्थितिः} स्याद्यथा मध्याह्नात् पूर्वतः परतश्च भवति पातमध्यात् पूर्वं स्पर्शः पातमध्यात् पश्चान्मोक्षः~॥~१४~॥\\

{\small \textbf{अथ प्रकारान्तरेण क्रान्तिसाम्ये स्थितिलक्षणमुष्णिहा छन्दसाह तल्लक्षणम् अलङ्कारचूडामणिविषयिके तथा वृन्दचूडामणौ उष्णिहा छन्दःसंसृतौ स्याद्रजौ गुरुरिति विवेकाद्वाच्यम्~।}}

\phantomsection \label{9.15}
\begin{quote}
{\large \textbf{{\color{purple}मानयोगखण्डतो यावदल्पमन्तरम्~।\\
क्रान्तिसाम्यमेव तत् तावदेव हि स्थितिः~॥~१५~॥}}}
\end{quote}

\begin{center}
{\large \textbf{इतीह भास्करोदिते ग्रहागमे कुतूहले \\
विदग्धबुद्धिवल्लभे सकृच्च पातसाधनम्~॥~९~॥}}
\end{center}

चन्द्रसूर्ययोर्मानयोगार्द्धाद्बिम्बैक्यार्द्धाद्यावत् क्रान्त्यन्तरं स्वल्पं भवति तावत् कान्तिसाम्यमेव ज्ञेयम्~। तावदेव तस्य पातस्य स्थितिरस्तीति ज्ञेया, अङ्गुलाद्यविवेकार्द्धं
\end{sloppypar}

\newpage

\begin{sloppypar}
\noindent त्रिगुणं सत्कलादिकं भवति क्रान्त्यन्तरस्य कलादिकत्वात्~। अथ पातस्य फलं {\color{violet}सिद्धान्ते\textendash \,"पातस्थितिकालान्तरमङ्गलकृत्यं न शस्यते तज्ज्ञैः~। स्नान-जपहोमदानादिकमत्रोपैति खलु वृद्धिम्"}~॥~१५~॥

\begin{center}
इतीह भास्करोदित इति तु स्पष्टार्थमेव~।
\end{center}

अथ पुनरुदाहरणं शाके १५४३ श्रावणशुक्ला ९ भौमे घट्यः १६।३३ विशाखा १२।५४ शुक्लः १५।२६ घट्यः ३८।४२ गताब्दाः ४३८ अहर्गणः १६१२३० मध्यमः सूर्यः ३।१६।५६।१६ चन्द्रः ६।२६।१८।३० पातः ४।१३।४५।३४ उदये स्वदेशीया अयनांशाः १८।१८।२१ चरपलमृणम् ९२ चरपलसंस्कृतः सूर्यः ३।१५।५१।५७ गतिः ५७।२ चन्द्रः ७।०।२०।४ गतिः ८२९।३५ पातः ४।१३।४५।३० सायनार्कः ४।४।११।२३ चन्द्रः ७।१८।५४।१५ योगः ११।२३।५।३८ द्वादशराशिभ्यः १२ शुद्धोंऽशादिः ६।५४।२२ अस्य कलाः ४१४।२२ गत्योरैक्येन ८८६।३७ भक्ता लब्धं दिनादि ०।२७।२ द्वादशभ्यः क्रमो योगस्तेन गम्यं ज्ञेयम् ०।२७।२ एतत्कालीनः सायनः सूर्यः ४।४।३८।२ चन्द्रः ७।२५।२१।५१ पातो निरयणः ४।१३।४६।५९ सपात-
\end{sloppypar}

\newpage

\begin{sloppypar}
\noindent चन्द्रः ११।२०।५८।२९ बाणः कलादिः ४२।४४ याम्यः, अथ पातस्य गतैष्यज्ञानं सायनचन्द्रो विषमे पदे सपातचन्द्रादेकगोले तेन गतपातो ज्ञेयः~। अथ साधनं सायनचन्द्रस्य सर्वभागाः २३५।२१।५१ तिथिभक्ता लब्धं क्रान्तेर्गतखण्डकानि १५ भोग्यखण्डम् २३६ शेषम् १०।२१।५१ अस्य सपातचन्द्रस्य भागाः ३५०।५८।२९ तिथि\textendash \,१५\textendash \,भक्ते लब्धम् २३ शरस्य गतखण्डकानि भोग्यखण्डकमृणम् ७० शेषम् ५।५८।२९ गणना पूर्वस्थापितखण्डकेभ्यो धनर्णसंज्ञा च कार्या गतपातत्वादुभयत्र ये शेषांशास्त एव चापांशा ज्ञेयाः शरस्याल्पचापांशत्वाद्भोग्यखण्डं शरस्यैकस्थं गतपातत्वात् क्रान्तिशरखण्डकानि गतानि द्विस्थापितानि द्वित्रीणीत्युक्तत्वात् खण्डद्वयं द्विधा स्थापितमल्पशरत्वात्~पूर्वमुदाहरणं भाष्याद्यपेक्षया कृतमिदानीं सर्वसम्मतमुच्यते शरसंस्कारविषये विश्वां-शकेनापमभोग्यस्यैतस्यापि संस्कारेऽपि गोलस्यापेक्षया कर्तव्या किन्त्वेक-जात्योर्योगो भिन्नजात्योरन्तरमित्यर्थः~। उभयोर्धनरूपयोस्तथार्णरूपयोः योग एव कार्यः भेदेऽन्तरं कार्यमित्यर्थ इति सर्वसम्मितसंस्कारः~।~अथ क्रान्तिभोग्यखण्डस्य २३६ विश्वांशकेन धनरूपेण धनसमरूपाणि क्रान्ति-खण्डानि पूर्वस्थापितान्येकजातित्वाद्युतानि तान्येव शरखण्डकैः पूर्णस्था-पितैर्ऋणरूपैर्भिन्नजातित्वाद्रहितानि गोलापेक्षया न कृता~। अथ शरम्
\end{sloppypar}

\newpage

\begin{sloppypar}
\noindent ४२।४४ पञ्चदशगुणितस्याल्पत्वा\textendash \,६४।१०\textendash \,दाद्ये चापांशमिति प्रकारो न कृतस्तेन शरशेषम् ६४१।० अशुद्धस्पष्टखण्डेन १८४।९ भक्तं लब्धं लवाद्यम् ३।२८।२५ षष्टिगुणम् २०८।५१ चन्द्रगत्या ८२९।३५ भक्ते~लब्धं दिनादि ०।१।१५ एभिर्गतः पातः पूर्वकालाद्यस्मिन्कालयोगे द्वादशजाताः तस्मादिति तेन श्रावणशुद्धे ९ भौम उदयाद्गतघटी २२ समये रवियोगे द्वादशजाता अत्र कालात् ०।१५।१३ एभिर्गतः पातोऽत्र दिन उदयाद्गतघट्यः १२।४९ समये पातमध्यं प्रतीत्यर्थमेतत्कालीनाः सायनौ रविचन्द्रौ पातश्च सूर्यः ४।४।२२।३४ चन्द्रः ७।२१।५१।२८ पातः ४।१३।४६।११ रवेः क्रान्तिः १९।१८।५३।३० अंशादि चन्द्रस्य १८।२९।५३।३० अंशादिचन्द्रस्य १८।२९।५३ याम्या याम्यशरेण ५९।१८ संस्कृता १९।२९।११ उभयोः क्रान्त्योरन्तरं विकला ०।१८ अन्तरस्य स्वल्पत्वान्न दोष उभयोः क्रान्तिसाम्यं जातं तेन शुद्धमिदं साधनं जातम्~। अथ स्थितिसाधनम्\textendash \,त्रिखाश्विदस्रनाडिकाः २२०३ अशुद्धखण्डेन १८४।९ भक्ता लब्धं घट्यः ११।५७ मध्यकालः १२।४९ अनयोरन्तरमुदयाद्गतनाङ्यः ०।५२ समये पातस्पर्शकालः युते मोक्षकालः २४।४६ समये मोक्षः रविबिम्बम् ११।१२ चन्द्रबिम्बम् १०।
\end{sloppypar}

\newpage

\begin{sloppypar}
\noindent १९ योगः २१।३१ अस्यार्द्धमङ्गुलादि १०।४६ त्रिगुणम् ३२।१८ स्पर्शकालीनः सूर्यः ४।४।१२।१२ चन्द्रः ७।१९।६।१५ पातः ४।१३।४५।३८~। रविक्रान्तिः १९।३१।५१ सौम्या चन्द्रक्रान्तिर्याम्या १७।४६।३४ याम्यशरः कलादि ६७।४५ स्पष्टा क्रान्तिः १८।५४।११ क्रान्त्योरन्तरम् ०।२२।३२ मानैक्यम् ३२।२१ अतः क्रान्तेः साम्यं स्यादिति~। अथ स्पर्शकालघटी २ द्वयसमयिकाः सूर्यः ४।४।१४।६ चन्द्रः ७।१९।३३।५४ पातः ४।१३।४५।३८ रविक्रान्तिः २८।५८।१९ याम्या चन्द्रक्रान्तिः १७।५३।४९ शरकला दक्षिणा ६९।३९ स्पष्टाक्रान्तिः १९।३।२८ कान्त्योरन्तरं ५।९ मानैक्यात् ३२।१८ ऊनमतः क्रान्तिसाम्यं स्पर्शकालः मोक्षकालीनः सूर्यः ४।४।३४।५५ चन्द्रः ७।२४। ३६।४१ पातः ४।१३।४६।४९ रविक्रान्तिः १९।२५।५४ उत्तरा चन्द्रक्रान्तिः दक्षिणा १९।१३।१३ शरकला दक्षिणा ४६।५१ स्पष्टा क्रान्तिः २०।०।४ रविक्रान्तिचन्द्रक्रान्त्योरन्तरम् ०।३४।७ मानैक्यादधिकं तेन पातनिर्गमः अथ पुनरुदाहरणं शाके १२९० कालीनसंवत्सरे कार्तिकशुक्ला ७ गुरौ तत्रोदयेऽहर्गणः ६७८२३ मध्यमः सूर्यः ७।५।४२।५८ चन्द्रः ३।२९।४।१७ पातः ९।१२।१०।१६ स्पष्टौ रविचन्द्रौ सूर्यः ७।४।१८।५३ गतिः
\end{sloppypar}

\newpage

\begin{sloppypar}
\noindent ६०।४० चन्द्रः ३।२०।४५।१७ गतिः ७२५।३५ चरपलम् ५८ अयनांशाः १४।५।४० सायनार्कोऽयम् ७।१८।२३।५३ सायनचन्द्रः ४।४।५४।१७ अनयोरैक्यम् ११।२३।१८।१० द्वादशभ्यः शुद्धेंऽशादि ६।४१।५० गतैक्येन ७८६।१५ भक्ते लब्धं दिनादि ०।३०।४० एवं कार्तिकशुद्धे ७ गुरावुदयाद्गतघटी ३०।४० समये पातो भविष्यतीति पातसम्भवघटी ३०।४० समयिकः सूर्यः ७।४।४९।५७ चन्द्रः ३।२७।०।८ पातः ९।१२।११।१५ सपातचन्द्रात् कलादिर्बाणः १६९।२१ उत्तरः सायनचन्द्रः ४।१२।५।५८ सपातचन्द्रा\textendash \,१।९।१२।१\textendash \,देकगोले वर्तते तस्मादेष्यः पातो ज्ञेयः यस्मिन् काले योगो द्वादश जातास्तस्मात् कालात् क्रान्तीषुखण्डानां धनर्णसंज्ञास्थापना च प्राग्वत् सायनचन्द्रस्य भागाः १३१।५।५८ तिथि\textendash \,१५\textendash \,भक्ता लब्धम् ८ शेषं भागादि ११।५।८ अष्टौ क्रान्तिखण्डानि गतानि क्रमोत्कमाद्गणनयापक्रमस्य तृतीयं भोग्यखण्डमृणसंज्ञम् २३६ सपातचन्द्रस्य भागाः ३९।१२।१ तिथि\textendash \,१५\textendash \,भक्ता लब्धम् २ शरस्य क्रमेण भुक्तखण्डद्वयं तृतीयं भोग्यम् ५६ धनसंज्ञे शेषे ९।२१।१ गम्यपातत्वादुभयोः शेषांशास्तिथितश्च्युताः उभयोश्चापांशाः क्रान्तिचापांशाः ३।५०।५२ शर-चापांशाः ५।४।४७ भोग्यखण्डमादीकृत्य गम्यपातत्वा-
\end{sloppypar}

\newpage

\begin{sloppypar}
\noindent द्भोग्यखण्डं कृतमुभयोर्द्विधा स्थापितानि स्वल्पक्रान्तेश्चापांशादिकस्थं भोग्यस्थं यथा द्वित्रीणीति वचनात् खण्डकद्वयं द्विधा स्थापितं~क्रान्ति-भाग्यखण्डस्य २३६ विश्वांशभागेनर्णरूपेणर्णरूपाणि क्रान्तिखण्डानि युक्तानि १३ यथा शरखण्डकैर्धनरूपाणि क्रान्तिखण्डानि योज्यान्यृण-रूपैर्ऋणरूपाणि योज्यानि एकत्रर्णरूपमन्यद्धनरूपं तदा भिन्नजातित्वात् अन्तरं कार्यमत्र भिन्नजातित्वादन्तरं जातं शरसंस्कृतानि स्फुटखण्डानि प्रथमखण्डम् १९७ स्वल्पचापांशैः ५४।१२ गुणितम् ७७५।३ पञ्चदशगुणात् पूर्वागतबाणात् २५।४०।१५ शोधितं शेषम् १७६५।१२ द्वितीयखण्डं ६१ चापांशाः ३।५४।१२ चापांशाः ५।४७।५७ अनयोरन्तरांशैः १।५३।४५ गुणितम् ४९२।३।२७ पूर्वशरशेषात् १७६५।१२ शुद्धम् १२७३।९ शेषं विषमं तृतीयखण्डम् २७४ अन्तरांशैः १।५३।७ तिथितः १५ शुद्धैः १३।६।५३ गुणितम् ३५९३ एतच्छेषात् १२७३।९ न शुद्ध्यति तेन शरशेषम् १२७३।९ अशुद्धेनास्फुटखण्डेन २७४ लब्धं लवाद्यम् ४।३८।४८ संशुद्धं खण्डांशम् ३।४।५२ उभाभ्यां युतम् १०।२६।४७ षष्टिगुणम् ६२७।४७ चन्द्रगत्या ७२५।३५ भक्तं दिनादि ०।५१।५० एभिर्दिनादिभिः पूर्वकालाद्गम्यं पातमध्यं पूर्वकालघट्यः ३०।४० मध्ये युक्तम्
\end{sloppypar}

\afterpage{\fancyhead[RE,LO]{{\small{अ.\,१०}}}}
\newpage

\begin{sloppypar}
\noindent १।२२।३० एवं कार्तिकवदि ८ भृगावुदयाद्घट्यः २२।३० पातमध्यमेतत्कालीनः सूर्यः ७।५।४१।५० चन्द्रः ४।७।२६।५७ पातः ९।१२।१४।३८ रविक्रान्तिः ८७१।४४ चन्द्रशरः २०५।५७ स्पष्टक्रान्तिः ७७७।४४ उभयोः क्रान्ति-साम्यत्वाच्छुद्धं साधनमेता घटिकाः २२।३ अशुद्धेन २७४ भक्ता घट्यः ८।२ स्थितिरियं पातमध्यात् २२।३० शुद्धाः १४।२८ उदयाद्गतघटीषु पातस्पर्शः पातमध्ये युता ३०।३२ एवमुदयाद्गतघटीषु पातनिर्गमः मानं सूर्यस्य ११।३ चन्द्रस्य ९।४८ अनयोरैक्यार्द्धम् १०।२५ अङ्गुलादि त्रिगुणितं कलात्मकम् ३१।१५ एवं स्पर्शकालीनमोक्षकालीनक्रान्त्यन्तरं प्रसाध्यं मानयोगखण्डतो यावदल्पिका स्थितिः प्रतीतः पश्चाद्यः करणपातमध्ययोर्हि निर्णीतः~॥
\end{sloppypar}
\vspace{2mm}

\begin{center}
{\large \textbf{इति श्रीब्रह्मतुल्यवृत्तौ पाताध्यायो नवमः~॥~९~॥}}
\end{center}
\vspace{2mm}

{\small \textbf{अथ चन्द्रसूर्ययोर्ग्रहसम्भवाध्यायो व्याख्यायते तत्रादौ
शरसाधनमाह\textendash }}

\phantomsection \label{10.1}
\begin{quote}
{\large \textbf{{\color{purple}द्विघ्नो मासगणस्त्रिहृद्द्विभयुतो वर्षाभ्रदस्रांशयुक् \\
वच्चोक्तार्कघटीफलं शरहृतं स्वर्णं तु तस्मिँल्लवाः~।\\
युक्ता मासमितैर्गृहैरथ रवे राश्यर्द्धयुक्ताश्च ते \\
तद्बाहू च लवा निजार्द्धसहिताः स्यादङ्गुलाद्यः शरः~॥~१~॥}}}
\end{quote}

\newpage

\begin{sloppypar}
इतश्चत्वारः श्लोकाः शार्दूलविक्रीडितेनाह\textendash \,शकः पञ्चदिक्चन्द्रहीन~इत्या-दिनाधिमासैर्युगूर्ध्व इत्यन्तेन साधितो मासगणो द्वाभ्यां गुणनीयस्त्रिभिः भाज्यो लब्धाङ्को भागादिर्द्विसप्तत्युत्तरद्विशत्या २७२ युतः कार्यः ततो \hyperref[10.1]{वर्षा}णां करणगताब्दानाम् \hyperref[10.1]{अभ्रदस्रांशे}न २० विंशतितमांशेन युक्तः कार्य एवंभूते तस्मिन्नंशाद्ये बह्वाचार्योक्तार्कघटीफलं पञ्चभि\textendash \,५\textendash \,र्भक्तं तद्यथासम्भवं धनमृणं कार्यमेवं तस्मिन्नंशाद्ये मासगणतुल्यै राशिभी राशिस्थाने युतं कार्यमथ रवेर्ग्रहणसम्भवे \hyperref[10.1]{राश्यर्द्धे}न पञ्चदशभिरंशैः पूर्वागतं राश्यादियुतं कार्यं ततस्तस्य भुजः कार्यस्तस्यांशाः स्वीयेनार्द्धेन सहिताः शरोऽङ्गुलादिः स्यात् सूर्यग्रहणे राश्यर्द्धयुक्तो यस्मिन् गोले तद्दिक् ज्ञेया~। यथा शके १५४२ मार्गशीर्षपूर्णिमा बुधे गताब्दाः ४३७ मासगणः ५४१४ द्विगुणः १०८२८ विभक्तो लवादि ३६०९।२० द्विभ\textendash \,२७२\textendash \,युतः ३८८१।२० गताब्दाः ४४७ एषामभ्रदस्रांशेन २१।५१ युक्तः ३९०३।११ धनुषः पूर्वपक्षार्कघटीफलम् ३ पञ्चभि\textendash \,५\textendash \,र्भक्तं लब्धमंशाद्यम् ०।३६ पूर्वस्मिन् ३९०३।११ कर्कादित्वादृणम् ३९०२।३५ अयमंशादी राश्यादिकृतस्त्रिंशद्भक्ते लब्धं राश्यादयः १३०।२।३५।० राशिस्थाने मासगणः ५४१४ युतः ५५४४।२।३५।० राशिस्थाने द्वादशभिर्भक्ते लब्धम् ४६२ लब्धस्य
\end{sloppypar}

\newpage

\begin{sloppypar}
\noindent प्रयोजनाभावात् त्यक्तं शेषं राश्यादि ०।२।३५।० अस्य भुजोऽयमेवांशाः २।३५।०~। स्वीयेनार्द्धेन १।१७ युतात् ३।५२ अयमङ्गुलादिशरः अथ~सूर्य-ग्रहणसम्भवार्थं शाके १५२२ लौकिकश्रावणवदि ३० तिथौ सोमे गताब्दाः ४१७ मासगणः ५१६१ सूर्यः ३।०।३५ दिनार्द्धम् १६।३ पूर्वघटी २८।५६ मासगणः ५१६१ द्विघ्नः १०३२२ त्रिभक्तोंऽशादि ३४४०।४।० द्विभयुतः ३७१२।४०।० वर्षाणां ४१७ विंशांशेन २०।५१ युतः ३७३३।३१।० कर्कपूर्वपक्षघटी ३ फलं शरभक्तेन ०।३६ हीनः ३७३२।५५।० त्रिंशद्भक्तं राश्यादि १२४।१२।५५।० राशिस्थानं मासगणैः ५१६१ युतम् ५२८५।१२। ५५।० द्वादशभक्तं शेषं राश्यादि ५।१२।५२।० सूर्यग्रहणत्वाद्राश्यर्द्धेन ०।१५।०।० युतं जातं राश्यादि ५।२७।५५~। अस्य भुजः ०।२।५।० अस्यांशाः १।५।० निजार्द्धेन १।२ युतः ३।७ शरोङ्गुलादिरुत्तरः राश्यर्द्धयुक्तस्य राश्यादिसौम्यगोले स्थितत्वात्~॥~१~॥\\
\end{sloppypar}

{\small \textbf{अथ नतसाधनमाह\textendash }}

\phantomsection \label{10.2.1}
\begin{quote}
{\large \textbf{{\color{purple}दर्शान्ते नतनाडिकाब्धिरहितो युक्तो गृहाद्यो रविः \\
प्राक्पश्चादयनांशकैश्च सहितस्तद्दोर्गृहोनाहताः~।\\
शैलास्ते द्विगुणा लवादिरयमस्तात्स्वाक्षतोंऽशा}}}
\end{quote}

\newpage

\phantomsection \label{10.2}
\begin{quote}
{\large \textbf{{\color{purple}नताः\\
तद्वेदांशमिता नतिश्च विशिखस्तत्संस्कृतोऽर्कग्रहे~॥~२~॥}}}
\end{quote}

\begin{sloppypar}
दर्शान्तकालीनं नतं कृत्वा तस्याङ्घ्रिश्चतुर्थांश इति नतघटिकानां चतुर्भि\textendash \,४\textendash \,र्भागे लब्धराशयः शेषं त्रिंशद्भिः सङ्गुण्य पुनश्चतुर्भिर्भागे हृते लब्धा भागाः शेषं षष्ट्या सङ्गुण्य चतुर्भक्ते लब्धं कला एवं राश्यादिफलं ग्राह्यं तेन दर्शान्तकालिको गृहाद्यो रविः प्राक्कपाले रहितः पश्चिमकपाले युक्त इति कृत्वा स एवायनांशैर्युक्तस्तस्य भुजं कृत्वा तेन भुजेनोना हताश्च \hyperref[10.2.1]{शैलाः} ७ सप्त कार्याः स लवादिक्रान्तिर्भवति ततः शरः स्वाक्षवशेन प्राग्वन्नतांशाः साध्यास्तेषां चतुर्थांशो नतिः स्यात् तया प्रागानीतः शरः संस्कृतः सन् स्फुटो भवति यथा दर्शान्तः २८।५६ दिनार्द्धम् १६।४३ अनयोरन्तरं घट्यादिनतम् १२।४३ पश्चिमं चतुर्भक्ते लब्धं राशयः ३ शेषम् ०।१३ त्रिंशद्गुणम् ६।३० चतुर्भक्तं लब्धमंशाः १ शेषम् २।३० षष्टिगुणम् १५० चतुर्भक्ते लब्धं कलाः ३७ शेषम् २ षष्टिगुणम् १२० चतुर्भक्तं लब्धं विकलाः ३० एवं राश्यादिना ३।१।३७।३० अमावास्यान्तकालीनः स्पष्टो वा गतेष्टनाडीत्यादिना स्थूलोऽपि रविः ३।०।३५।३४ पश्चिमनतत्वाद्युतः ६।२।१३।४ अयनांशैः १७।५७ युतः ६।२०।१०।२४ अस्य
\end{sloppypar}

\newpage

\begin{sloppypar}
\noindent भुजः ०।२०।१०।२४ अनेन सप्त ७ राशय ऊनाः ६।९।४९।३६ पुनर्भुजेनैव ०।२०।१०।२४ गुणिता गोमूत्रिकया २।४०।२० द्विगुणाः ५।२०।४० जातां-शादिः क्रान्तिः संस्काररहिता सायनोऽर्को याम्यगोलेऽस्माद्याम्या याम्या-क्षांशैः २४।३५।९ संस्कृता जाता नतांशा याम्याः २९।५५।४९ चतुर्भक्ता ७।२७ जाता नतिर्याम्यानया पूर्वानीतसौम्यशरः ३।७ संस्कृतो भिन्नदिक्त्वात् अन्तरम् ४।२० जातः स्पष्टशरः सौम्यः~॥~२~॥\\

{\small \textbf{अथ ग्रहणसम्भवासम्भवमाह\textendash }}

\phantomsection \label{10.3}
\begin{quote}
{\large \textbf{{\color{purple}गोचन्द्रा हिमगोर्भवाश्च तरणेर्मानैक्यखण्डं शरे \\
तन्न्यूने ग्रहणं भवेदिति बुधैश्चिन्त्यः पुरा सम्भवः~।\\
चक्राद्यः खलु मध्यमार्कतमसोर्योगो द्विनिघ्नो द्वियुक् \\
पर्वेशो मुनिभक्तशेषकमितो ज्ञेयो विरञ्च्यादिकः~॥~३~॥}}}
\end{quote}

गोचन्द्रा इति\textendash \,\hyperref[10.3]{गोचन्द्रा} एकोनविंशतिश्चन्द्रस्य मानैक्यार्द्धं \hyperref[10.3]{शरं} चन्द्रशरं मानैक्यार्द्धादूने सति ग्रहणं भवेदिति विद्वद्भिः पूर्वं सम्भवो ज्ञेयः यथा चन्द्रशरः ३।५२ मानैक्यार्द्धात् १९ ऊनस्तेन चन्द्रग्रहणसम्भवोऽस्ति~ततः चन्द्रग्रहणवत् सूर्यग्रहणसाधनं कर्तव्यं सूर्यस्पष्टशरः ४।२० सूर्यस्य मानै-क्यार्द्धात् ११ ऊनस्तेन सूर्यस्य ग्रहणसम्भवोऽस्ति तस्य साधनं पूर्ववत् चन्द्रग्रहणस्य साधनं चन्द्रग्रहणोक्तवत् सूर्यग्रहणस्य साधनं सूर्यग्रहणोक्त-
\end{sloppypar}

\newpage

\begin{sloppypar}
\noindent वत्~। अथ चक्राद्य इति क्षेपकरहितयो\hyperref[10.3]{र्मध्यमार्कतमसो}र्मध्यसूर्यरहितयोर्मध्य-मार्कतमसोर्मध्यसूर्यपातयोर्भगणाद्यो योगो \hyperref[10.3]{द्विघ्नः} द्विगुणीकृतः द्वियुक्तः कार्यः सप्तभिर्भाज्यः शेषं तेन गता वर्तमानस्य राश्याद्यं भुक्तं ब्रह्मादिकः \hyperref[10.3]{पर्वेशः} ब्रह्मशशीन्द्रकुबेरवरुणाग्नियमाश्च पर्वेशा इति यथा चन्द्रग्रहणे क्षेपरहितो भगणाद्यो रविः ४३७।६।०।५९।३८ पातः २३।६।१४।३८।२५ अनयोर्योगः ४६१।०।१५।३८।३ द्विगुणः ९२२।०।३१।१६।६ द्वियुक् ९२४ सप्त\textendash \,७\textendash \,भक्तं शेषम् ० ब्रह्मतो गणनया सप्त गताः ब्रह्मा पर्वेशः सूर्यग्रहणे भगणादिरविः ४१७।४।१।२१।१३ पातः २२।५।९।२४।१८ अनयोर्योगः ४३९।९।१०।४५।३१ द्विगुणः ८७९।६।२१।३१।२ द्वियुक् ८८१।०।४३।२।४ सप्त\textendash \,७\textendash \,भक्ते शेषम् ६ पर्वेशो यमो ज्ञेयः~॥~३~॥\\
\end{sloppypar}

{\small \textbf{अथ ग्रन्थकृत्स्वनामपूर्वकवर्णनमाह\textendash }}

\phantomsection \label{10.4}
\begin{quote}
{\large \textbf{{\color{purple}आसीत् सज्जनधाम्नि गेहविवरे शाण्डिल्यगोत्रो द्विजः \\
श्रौतस्मार्तविचारसारचतुरः सौजन्यरत्नाकरः~।\\
ज्योतिर्वित्तिलको महेश्वर इति ख्यातः क्षितौ स्वैर्गुणैः \\
तत्सूनुः करणं कुतूहलमिदं चक्रे कविर्भास्करः~॥~४~॥}}}
\end{quote}

\begin{center}
{\large \textbf{इतीह भास्करोदिते ग्रहागमे कुतूहले \\
विदग्धबुद्धिवल्लभे रवीन्दुपर्वसम्भवः~॥~१०~॥}}
\end{center}

\afterpage{\fancyhead[RE,LO]{{\small{अ.\,११}}}}
\newpage

आसीदिति~। स्पष्टार्थो ज्ञेयः~॥~४~॥
\vspace{2mm}

\begin{center}
{\large \textbf{इति श्रीकरणकुतूहले पर्वसम्भवासम्भवमाध्यायानः \\
दशमः समाप्तिमगमत्~॥~१०~॥}}
\end{center}
\vspace{2mm}

{\small \textbf{अथ ग्रहणोपयोगी नीरदाध्यायो व्याख्यायते\textendash }}\renewcommand{\thefootnote}{}\footnote{टि.\textendash \,१ अयं नीरदाध्यायः केनचित् प्रक्षिप्त इति प्रतिभाति~।}

\phantomsection \label{11.1}
\begin{quote}
{\large \textbf{{\color{purple}समलिप्तीकृते भानौ राश्येकं शोधयेद्बुधः~। \\
अंशका मनवश्चैव शेषं चक्राच्च पातयेत्~॥~१~॥}}
\vspace{1mm}

\phantomsection \label{11.2}
\textbf{{\color{purple}कलितं वर्गितं द्विघ्नं चक्रलिप्ताभिरुद्धरेत्~। \\
लब्धाढ्य इतरे सङ्गे तरोर्विश्वांशकैर्युतः~॥~२~॥}}
\vspace{1mm}

\phantomsection \label{11.3}
\textbf{{\color{purple}समलिप्तार्कसंयुक्तात् शोधयेदुदयभास्करात्~। \\
यच्छेषमाद्यसंयुक्तं नीरदार्को हि संस्फुटः~॥~३~॥}}}
\end{quote}

\begin{sloppypar}
समकलसूर्यमध्य एको राशिश्चतुर्दशांशाः १।१४।०।० शोध्याः~शेषं द्वादशराशिभ्यः १२ शोधयेत् तस्य कलाः कार्यास्तासां वर्गो विधेयः~स द्विगुणः कार्यस्तं चक्रकलाभिः २१६०० भजेत् लब्धस्य पृथक् स्थापितस्य आढ्य इति संज्ञा कर्तव्या य इति संज्ञः स त्रयोदशभि\textendash \,१३\textendash \,रंशैः युतः समकलसूर्ये योज्यस्तत औदयिकः सूर्यः शोध्यो यच्छेषं तत् पूर्व-कृताढ्यसंज्ञेन युतं सन्नीरदार्कः स्फुटो भवति~। यथा चन्द्रग्रहणे समकल-सूर्ये ८।०।१६।१० एको राशिरंशाश्चतुर्दश १।१४।०।० शुद्धाः शेषम् ६।१६। १६।१० चक्रात् १२ शुद्धः ५।१३।४३।५० अस्य कलाः ९८२३।५० आसां वर्गः ९६५०७७०१।२१ द्विघ्नः 
\end{sloppypar}

\newpage

\begin{sloppypar}
\noindent १९३०१५४०२।४२ चक्रकलाभिः २१६०० भक्ते लब्धं कलादि ८९३५।५३ षष्टिभक्तं लब्धमंशादि १४८।५५।५३ त्रिंशद्भक्तं जातं राश्यादिः ४।२८। ५५।५३ एतस्य राश्यादिकस्याढ्य इतरसंज्ञकस्त्रयोदशभिरंशैः १३ युतः ५।११।५५।५३ समकलसूर्ये ८।०।१६।१० युतः १।१२।१२।३ औदयिकसूर्यात् ७।२९।३५।२० शुद्धः शेषम् ६।१७।२३।१७ आढ्येन ४।२८।५५।५३ युक्तम् ११।१६।१९।१० अयं नीरदार्कोऽत्र स्पष्टो ज्ञेयः~॥~३~॥\\

{\small \textbf{अस्य प्रयोजनमाह\textendash }}

\phantomsection \label{11.4}
\begin{quote}
{\large \textbf{{\color{purple}रविभौमांशकं दृष्ट्वा निरभ्रं ग्रहमादिशेत्~। \\
शनिसौम्यनवांशे चेत् सलिलं क्षुद्रवर्षणम्~॥~४~॥}}
\vspace{1mm}

\phantomsection \label{11.5}
\textbf{{\color{purple}शशिशुक्रनवांशे च प्रावृट्काले महज्जलम्~।\\
गुरोरंशकमासाद्य दृश्यते सबलाहकः~॥~५~॥}}
\vspace{1mm}

\phantomsection \label{11.6}
\textbf{{\color{purple}ग्रहणे वा विलग्ने वा मेघच्छायां विजानतः~।\\ 
तस्याहं पादयुगलं कुसुमाञ्जलिनार्चये~॥~६~॥}}}
\end{quote}

\begin{center}
{\large \textbf{इति श्रीभास्कराचार्यविरचिते करणकुतूहले\\
नीरदार्कविचाराध्यायः समाप्तः~॥~११~॥}}
\end{center}

यदि नीरदार्को रविभौमनवांशके भवति तदा मेघो नास्ति शनिबुध-नवांशके क्षुद्रवर्षणं स्वल्पवर्षणं चन्द्रशुक्रनवांशके प्रभूतपर्जन्यो मेघः गुरुनवांशे यदा तदा भवति क्षुद्रवर्षणम्~॥~५~॥
\end{sloppypar}

\newpage

{\small \textbf{अथ स्वप्राशस्त्यमाह\textendash }}

\begin{quote}
{\large \textbf{{\color{purple}यथामति मया प्रोक्तं सम्प्रदायादथापि वा~।\\
उचितानुचितं यन्मे तद्वाक्यं क्षम्यतां विदः~॥~१~॥}}
\vspace{1mm}

\textbf{{\color{purple}विन्ध्याद्रिं निकषा पुरी सुविदिता सर्वर्द्धिवृद्धान्विता\\
तन्नेतास्ति भटः स्ववंशतिलकश्चौलुक्यवंशोद्भवः~।\\
सुश्रीवीरमुदे सुनीतिनिपुणो हेमाद्रिरेखापुरो\\
योऽभूद्यानभूपतीन्स्थिरतरान्प्रोन्मूल्य राजन्यके~॥~२~॥}}
\vspace{1mm}

\textbf{{\color{purple}वैशाख्ये खलु मन्त्रिणि प्रियवृषे दानप्रसूक्तौ सति\\
मङ्गल्यादिकलामिते गतवति श्रीविक्रमात्संवति~।\\
मासे प्रौष्ठपदे विनायकतिथौ दैत्येज्यवारे वरे\\
चक्रे श्रीगुरुभावतः सुमतियुग्घर्षेण चैपा मुदा~॥~३~॥}}
\vspace{1mm}

\textbf{{\color{purple}ग्रन्थाग्रन्थशतान्यस्य सार्द्धाष्टादशसंख्यया १८५०~।\\
ज्ञेयं चेदङ्कबाहुल्यात् न्यूनाधिक्यं न दोषकृत्~॥~४~॥}}
\vspace{1mm}

\textbf{{\color{purple}करणवृतवितस्यां सुमतिहर्षरचितायाम्~।\\
गणककुमुदकौमुद्यां निर्णीतः पर्वसम्भवः~॥~५~॥}}}
\end{quote}
\vspace{4mm}

\begin{center}
{\large \textbf{इति श्रीसुमतिहर्षविरचितायां ब्रह्मतुल्यवृत्तौ गणककुमुद-\\
कौमुद्यां ग्रहणसम्भवाधिकारोऽत्र सनीरदार्कम् \\
विचाराध्यायः समाप्तः~॥~१०~॥}}
\end{center}

\newpage
\thispagestyle{empty}

\begin{center}
{\large \textbf{विक्रय्यपुस्तकें\textendash \,ज्योतिषग्रन्थाः~।}}
\end{center}
\vspace{-2mm}

\begin{longtable}{lrr}
\textbf{~~~~नाम.} &  & \textbf{की. रु. आ.}\\
लीलावती सान्वय भाषाटीका अत्युत्तम & .... & १\textendash ८\\
बृहज्जातकसटीक भट्टोत्पलीटीकासमेतजिल्द & .... & १\textendash १२\\
बृहज्जातकमहीधरकृतभाषाटीका अत्युत्तम & .... & १\textendash ८\\
वर्षदीपकपत्रीमार्ग (वर्षजन्मपत्र बनानेका) & .... & ०\textendash ४\\
मुहूर्त्तचिंतामणि प्रमिताक्षरा रफ रु. १ ग्लेज & .... & १\textendash ८\\
मुहूर्त्तचिंतामणि पीयूषधारा टीका & .... & २\textendash ८ \\
ताजिकनीलकंठीसटीकतंत्रत्रयात्मक & .... & १\textendash ० \\
\multicolumn{3}{l}{ताजिकनीलकण्ठी तंत्रत्रयात्मक महीधरकृत भाषाटीका} \\
~~~~~~अत्युत्तम टैपकी छपी & .... & १\textendash ८\\
ज्योतिषसार भाषाटीकासहित & .... & १\textendash ०\\
मुहूर्त्तचिंतामणिभाषाटीका महीधरकृत & .... & १\textendash ०\\
मानसागरीपद्धति (जन्मपत्रबनानेमें परमोपयोगी) & ... & १\textendash ०\\
बालबोधज्योतिष & .... & ०\textendash २\\
ग्रहलाघव सान्वय सोदाहरण भाषाटीका समेत & .... & १\textendash ०\\
जातकसंग्रह (फलादेश परमोपयोगी) & .... & ०\textendash १२\\
चमत्कारचिंतामणि भाषाटीका & .... & ०\textendash ४\\
जातकालंकारभाषाटीका & .... & ०\textendash ६\\
\multicolumn{3}{l}{बृहत्पाराशरहोराशास्त्रम्\textendash \,पूर्वखण्ड सारांश मूल व उत्तर खण्ड} \\
~~~~~~संस्कृतटीका तथा भाषाटीका सहित & .... & ५\textendash ०\\
जातकालंकारसटीक & .... & ०\textendash ६\\
जातकाभरण & .... & ०\textendash १२\\
प्रश्नचंडेश्वर भाषाटीका & .... & ०\textendash १२\\
पंचपक्षी सटीक & .... & ०\textendash ४\\
पंचपक्षी सपरिहार भाषाटीका समेत & .... & ०\textendash ६\\
लघुपाराशरी भाषाटीका अन्वय सहित & .... & ०\textendash ३
\end{longtable}

\afterpage{\fancyhead[RE,LO]{}}
\afterpage{\fancyhead[CE]{{\small{जाहिरात~।}}}}
\newpage

\begin{longtable}{lrr}
\textbf{~~~~नाम.} &  & \textbf{की. रु. आ.}\\
मुहूर्त्तगणपति & .... & ०\textendash १२ \\
मुहर्त्तमार्तण्ड संस्कृत टीका व भाषाटीका सहित & .... & १\textendash ० \\
शीघ्रबोधभाषाटीका & .... & ०\textendash ६\\
षट्पंचाशिका भाषाटीका & .... & ०\textendash ३\\
भुवनदीपक सटीक ४ भा., तथा भाषाटीका & .... & ०\textendash ८\\
जैमिनिसूत्रसटीक चार अध्यायका & .... & ०\textendash ६\\
रमलनवरत्न मूल & .... & ०\textendash ८ \\
\multicolumn{3}{l}{केशवीजातक सउदाहरण भाषाटीका चक्रों समेत (अतिव-} \\
~~~~~~उपयोगी) & .... & १\textendash ८\\
सर्वार्थचिंतामणि & .... & ०\textendash १०\\
लघुजातकसटीक & .... & ०\textendash ५\\
लघुजातक भाषाटीका & .... & ०\textendash ८\\
सामुद्रिकभाषाटीका & .... & ०\textendash ४\\
सामुद्रिकशास्त्र बड़ा सान्वय भाषाटीका & .... & १\textendash ४\\
वृद्धयवनजातक भाषाटीकासह & .... & १\textendash ०\\
यवनजातक & .... & ०\textendash २ 
\end{longtable}
\vspace{2mm}

\begin{center}
संपूर्ण पुस्तकोंका "बडासूचीपत्र" अलग है,\\
आध आनेका टिकट भेजकर मँगालीजिये.\\
\rule{5cm}{0.4mm} \\
\end{center}
\vspace{4mm}

\begin{flushright}
पुस्तकोंके मिलनेका पता\textendash ~~~~~~~\\
\vspace{2mm}

{\large \textbf{खेमराज श्रीकृष्णदास,}}~~~~~~~~\\
\vspace{1mm}

\textbf{"श्रीवेङ्कटेश्वर" छापाखाना, खेतवाड़ी\textendash \,बंबई.}
\end{flushright}

\end{document}
 