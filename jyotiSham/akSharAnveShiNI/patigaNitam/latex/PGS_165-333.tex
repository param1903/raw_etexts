\documentclass[10pt, openany]{book}
\usepackage[text={4.65in,7.45in}, centering, includefoot]{geometry}
\usepackage{mathtools}

\usepackage{ragged2e}
\usepackage{setspace} 

\usepackage[table, x11names]{xcolor}

\usepackage{fontspec,realscripts}
\usepackage{polyglossia}


\setdefaultlanguage{sanskrit}
\setotherlanguage{english}
\defaultfontfeatures[Scale=MatchUppercase]{Ligatures=TeX}

\newfontfamily\sanskritfont[Script=Devanagari, Scale=0.9]{Shobhika}
\newfontfamily\englishfont[Language=English, Script=Latin]{Linux Libertine O}

\newfontfamily\en[Language=English, Script=Latin]{Linux Libertine O}
\newfontfamily\bs[Script=Devanagari, Color=purple]{Shobhika-Bold}
\newfontfamily\eg[Script=Devanagari, Color=red]{Shobhika-Bold}
\newfontfamily\qt[Script=Devanagari, Scale=0.9, Color=violet]{Shobhika-Regular}
\newfontfamily\eqt[Language=English, Script=Latin, Color=violet]{Linux Libertine O}
\newfontfamily\bqt[Script=Devanagari, Scale=1, Color=brown]{Shobhika-Regular}
\newfontfamily\s[Script=Devanagari, Scale=0.9]{Shobhika-Regular}
\newcommand{\devanagarinumeral}[1]{%
	\devanagaridigits{\number \csname c@#1\endcsname}} 

\usepackage{enumerate}
%\pagestyle{plain}
\usepackage{fancyhdr}
\pagestyle{fancy}
\renewcommand{\headrulewidth}{0pt}
\usepackage{afterpage}
\usepackage{amsmath}
\usepackage{amssymb}
\usepackage{graphicx}
\usepackage{longtable}
\usepackage{footnote}
\usepackage{perpage}
\MakePerPage{footnote}
\usepackage[para]{footmisc}
\makeatletter
\def\blfootnote{\gdef\@thefnmark{}\@footnotetext}
\makeatother
%\usepackage{dblfnote}
\usepackage{xspace}
%\newcommand\nd{\textsuperscript{nd}\xspace}
\usepackage{array}
\usepackage{emptypage}

\usepackage{hyperref}   % Package for hyperlinks
\hypersetup{
colorlinks,
citecolor=black,
filecolor=black,
linkcolor=blue,
urlcolor=black
}
\begin{document}

\fancyhead[CE] {श्रेढीव्यवहारः}
\fancyhead[CO]{श्रेढीव्यवहारः}
\fancyhead[LE,RO]{\thepage}
\cfoot{}
\renewcommand{\thepage}{\devanagarinumeral{page}}
\setcounter{page}{112}

\renewcommand*{\arraystretch}{0.7}
{अत्रैवं\renewcommand{\thefootnote}{१}\footnote{श्रत्रैव~।} क्षेत्रकरणम्\textendash \,\hyperref[81]{'पदमेकं तल्लम्ब'} इति लम्बः १, चयस्यास्य ३
दलेन\begin{tabular}{c}३\\ २\end{tabular}मुखं २}
{हीनमिति प्राग्वत् धरा\begin{tabular}{c}१\\ २\end{tabular}, एषैव भूमिः चयेन ३ सहिता\begin{tabular}{c}७ \\२ \end{tabular}एतत् वक्त्रं,
\hyperref[81]{'कुर्यात् सूत्रेण तच्चिह्नम्'} इत्यादि प्राग्वत्, धनात्मकत्वात् \hyperref[82]{'सूत्रप्रसृतिर्वज्रवत्'} इत्यादि नास्ति~। इत्थं हास्तिकश्रेढीक्षेत्रम्~।}
\vspace{3mm}

न्यासः\textendash 

\hspace{15mm} \includegraphics[width=9cm, height=4cm]{Images/page-0165as.jpeg} 
\vspace{3mm}

 {अथार्धिकक्षेत्ररचना\textendash \;धरोनमुखम् \,३, \,इष्टावलम्बेन\,\begin{tabular}{c}१\\ २\end{tabular}\,गुणितम्\,\begin{tabular}{c}३\\ २\end{tabular}, अवनियुक् \,२ \,एतत् वक्त्रम्~।}
\vspace{3mm}

न्यासः\textendash 
\vspace{-3mm}

\hspace{25mm} \includegraphics[width=5cm, height=3cm]{Images/page-0165bs.jpeg}
\vspace{3mm}

{अतः क्षेत्रफलं\renewcommand{\thefootnote}{२}\footnote{क्षत्र*~।}\textendash \,भूमुखयोगः \begin{tabular}{|c|}५\\ २\\\hline \end{tabular}\,, अस्यार्धं \begin{tabular}{|c|} ५\\ ४\\\hline \end{tabular}\,,
लम्बेन\renewcommand{\thefootnote}{३}\footnote{लवेण~।} \begin{tabular}{|c|} १\\ २\\\hline \end{tabular} गुणितं \begin{tabular}{|c|} ५ \\८\\\hline  \end{tabular}~।}
\vspace{3mm}

{अत्र ब्रुवते यद्यनयोरेवाद्युत्तरयोर्हास्तिकक्षेत्रफलं\renewcommand{\thefootnote}{४}\footnote{पद्यनयोरेखा*~।}  रूपद्वयं
तर्हि आर्धिके रूपेण भाव्यं कथं}
{पञ्चाष्टभागा इति, तत्रायं न्यायः\textendash \,यदीदं\renewcommand{\thefootnote}{५}\footnote{यदीयं~।} हास्तिकं क्षेत्रं
समचतुरश्रमुच्यते\renewcommand{\thefootnote}{६}\footnote{*चतुरश्राद्युच्यते~।} तदर्धे रूपे}
{आयतचतुरश्रे तदर्धं फलं, इदं च क्षेत्रं शरावाकृति सर्वत एव हास्तिकं
पृथक्पृथक्प्रमाणभूवदनभुजं, तस्मादस्य प्रथममर्धमल्पं द्वितीयं तु विपुलमिति
कुतोऽनयोः समांशता~।}
{तथा च प्रथमार्धस्य यद्वक्त्रं तदुत्तरार्धस्य भूमिः २, वक्त्रं \begin{tabular}{|c|} ७\\ २\\\hline \end{tabular}
लम्बः \begin{tabular}{|c|} १\\ २\\\hline \end{tabular}  इति}
{विस्तारस्तावदुपचित एव, क्षेत्रफलं च भूमुखयोगस्यास्य \begin{tabular}{|c|}११\\ २\\\hline \end{tabular} अर्धं \begin{tabular}{|c|}११\\ ४\\\hline \end{tabular} लम्बेन\begin{tabular}{c}१\\ २\end{tabular}गुणितं}
\begin{tabular}{|c|} ११\\ ८\\\hline \end{tabular} प्रथमार्धफलस्यास्य च\renewcommand{\thefootnote}{७}\footnote{?} योगो हास्तिकफलमेव रूपद्वयम्~।
\vspace{3mm}

 {पञ्चराशिकं चात्र प्रत्ययनिबन्धनमस्ति~। अर्धभुवोऽर्धचतुर्थवक्त्रस्य
रूपलम्बस्य क्षेत्रस्य यदि रूपद्वयं क्षेत्रफलं तदार्धभुवो द्विवक्त्रस्यार्धलम्बस्य किमिति,}

\newpage

\noindent{भूवदने समस्य न्यासः\textendash}

\renewcommand*{\arraystretch}{1}
\begin{center}
{\begin{tabular}{|cc|cc|}
\hline 
भू & ४ & $\begin{matrix}
\vspace{-1mm}
\mbox{{५}}\\
\vspace{-1mm}
\mbox{{२}}
\vspace{1mm}
\end{matrix}$ & भू \\ 
\hline 
ल  & १ & $\begin{matrix}
\vspace{-1mm}
\mbox{{१}}\\
\vspace{-1mm}
\mbox{{२}}
\vspace{1mm}
\end{matrix}$ & ल \\ 
\hline 
फ & २ & {} & {}\\\hline \end{tabular}}
\end{center}

\renewcommand*{\arraystretch}{0.7}
\noindent{\hyperref[45]{'नीते फलेऽन्यपक्षम्'} इत्यादिना लभ्यते\begin{tabular}{c}५ \\८\end{tabular}, इदं प्रथमार्धे क्षेत्रफलम्~।}
\vspace{3mm}

{अथापरं पञ्चराशिकम्\textendash \,अर्धभुवोऽर्धचतुर्थवक्त्रस्य\renewcommand{\thefootnote}{१}\footnote{वत्क्रस्य~।} रूपलम्बस्य फलं
यदि द्वौ त(दा)}
{द्विभुवोऽर्धचतुर्थवक्त्रस्यार्धहस्तलम्बस्य किमिति~। भूवदने समस्य}
{न्यासः\textemdash}

\renewcommand*{\arraystretch}{1}
\begin{center}
\begin{tabular}{|c|c|}
\hline
 ४ & $\begin{matrix}
\vspace{-1mm}
\mbox{{११}}\\
\vspace{-1mm}
\mbox{{२}}
\vspace{1mm}
\end{matrix}$\\ 
 \hline
 १ &  $\begin{matrix}
\vspace{-1mm}
\mbox{{१}}\\
\vspace{-1mm}
\mbox{{२}}
\vspace{1mm}
\end{matrix}$\\
 \hline
 २ & {} \\
 \hline
 \end{tabular}
\end{center}
 
\renewcommand*{\arraystretch}{0.7}
\noindent{\hyperref[45]{'नीते फलेऽन्यपक्षम्'} इत्यादिना लभ्यते द्वितीयार्धे क्षेत्रफलं \begin{tabular}{|c|} ११\\ ८\\\hline \end{tabular}~।}
\vspace{3mm}

{तदेवं क्षेत्रस्वरूपवशादर्धयोः फलवैषम्यम्\renewcommand{\thefootnote}{२}\footnote{*वशादर्धयो फलं वैष*~।}\,। अत एव राशिगते
गणितान्तरं वक्ष्यति}
{\hyperref[89]{'निर्विकलपदघ्नचय'} इत्यादि~।}
\vspace{3mm}

{न्यासः\textendash \,\hspace{4mm}आ २, उ ३ , ग \begin{tabular}{|c|} १\\ २\\\hline \end{tabular}~।}
\vspace{3mm}

{निर्विकलं पदं (०), अनेन चयः ३ गुणितः \hyperref[21]{'सङ्गुणने (खेन) च खमेवे'}ति भवति
०,}
{आदिना २ सहितः \hyperref[21]{'क्षेपसमं खं योगे'} इति भवति २,
एषोऽनष्टसञ्ज्ञोऽग्रेऽपेक्षमाणो\renewcommand{\thefootnote}{३}\footnote{उपेक्ष्यमाणो~।} द्वितीयस्थाने}
{स्थाप्यः, एष एव च मुखेन २ अन्वितः ४, चयेन ३ विहीनः\renewcommand{\thefootnote}{४}\footnote{विहीन~।} १, निर्विकलपदम्
(०) अस्यार्धं}
{शून्यमेव ततः \hyperref[21]{'सङ्गुणने खेन च खमेवे'}ति शून्यमेव, विकलं \begin{tabular}{|c|}१ \\ २\\\hline \end{tabular}
एतेन\renewcommand{\thefootnote}{५}\footnote{येतेन~।} गुणित उपरि}
{द्वितीयस्थानस्थापितोऽनष्टसञ्ज्ञको राशिः रूपद्वयं २ जातं रूपमेकं १,
एतत्\renewcommand{\thefootnote}{६}\footnote{यत*~।} गणितं समग्रप्रथमपदस्य~। रूपद्वयस्यार्धं रूपमिति~।}
\vspace{3mm}

{अथापरः प्रश्नः\textemdash}

\begin{quote}

{\eg पञ्चोत्तरद्विकादेः पञ्चमभागे पदे कथय~॥~१०३~॥}\end{quote}

{यस्याः प्रथमं पदं रूपद्वयप्रमाणं उत्तराणि पदानि यथोत्तरं पञ्चप्रचयानि
पञ्चमो}
{भागः पदं सा श्रेढी समग्रपदसमासेन किंफला\renewcommand{\thefootnote}{७}\footnote{किंफलं~।} भवतीति कथय~।}
\vspace{3mm}

{न्यासः\textendash \,\hspace{4mm}
\begin{tabular}{|c|}आ \\ २ \\  \\\hline \end{tabular}\begin{tabular}{c|}उ    \\ ५\\ \\\hline \end{tabular}\begin{tabular}{c|}    ग \\ १ \\ ५ \\\hline \end{tabular}

\newpage

{पदं \begin{tabular}{|c|} १ \\ ५\\\hline \end{tabular}\renewcommand{\thefootnote}{१}\footnote{\begin{tabular}{|c|}१ \\२\end{tabular}~।} व्येकमिति रूपं सवर्णितं
\begin{tabular} {|c|}५ \\ ५\\\hline \end{tabular} न पततीति विपरीतशुद्ध्या व्ययराशिशेष}
{ऋणा-त्मकोऽवतिष्ठते\renewcommand{\thefootnote}{२}\footnote{*कोतिष्ठिते~।} \begin{tabular}{|c|} $\begin{matrix}
\mbox{{४}}\\
\mbox{{५}}
\end{matrix}+$\\\hline \end{tabular} तस्यार्धं \begin{tabular}{|c|} $\begin{matrix}
\mbox{{२}}\\
\mbox{{५}}
\end{matrix}+$\\\hline \end{tabular}\,, चयेन ५ गुणितं
गुण्यगुणकयोश्छेदांशापवर्तनात्}
{२$+$, आदिना २ युतमिति समधनर्णयोगे शून्यं जायते\renewcommand{\thefootnote}{३}\footnote{जायते .~।} ०, एतत् पदेन
\begin{tabular}{|c|} १ \\५ \\\hline \end{tabular} सङ्गुणितं शून्यमेव,}
{नास्ति क्षेत्रफलमत्रेति~। तत्किमिति चेत् ननु गणि(त)मेवात्र प्रमाणम्~। 
एवंविधेष्वाद्युत्तरपदेषु}
{गणितं शून्यमेव भवति~। तथा शून्यादस्मादाद्यादयस्त\renewcommand{\thefootnote}{४}\footnote{*द्यार्दय*~।} एवायान्ति यथा\textendash}
\vspace{2mm}

{आदावज्ञाते तदानयनार्थो न्यासः\textendash \,\hspace{2mm} \begin{tabular}{|c}आ \\ \\  ०\\\hline \end{tabular}\begin{tabular}{|c|} उ \\ \\  ५  \\\hline \end{tabular}\begin{tabular}{c|}ग \\ १\\ ५\\\hline \end{tabular}\begin{tabular}{c|}फ \\  \\० \\\hline \end{tabular}
\vspace{3mm}

{\hyperref[86.1]{'आदिः पदहृतगणितम्'} इत्यादिना कर्म~। तत्र गणितं शू(न्यं) ०, पदेनानेन  \begin{tabular}{|c|}१\\ ५\\\hline \end{tabular}
हृतं\renewcommand{\thefootnote}{५}\footnote{स्वतं~।}
{शून्यम् एव, निरेकगच्छेन \begin{tabular}{|c|} $\begin{matrix}
\mbox{{४}}\\
\mbox{{५}}
\end{matrix}+$\\\hline \end{tabular} हतस्य चयस्यास्य \begin{tabular}{|c|} ४+\end{tabular} दलेन
\begin{tabular}{|c|} २+\end{tabular}  ऊनमिति वियोगे सति}
{संयोग\renewcommand{\thefootnote}{६}\footnote{सगम~।} एव शून्ये रूपद्वये क्षिप्ते रूपद्वयम् २~।}}
\vspace{3mm}

{उत्तरेऽज्ञाते तदानयनार्थो न्यासः\textendash \,\hspace{2mm}
\begin{tabular}{|c}आ \\ \\  २\\\hline \end{tabular}\begin{tabular}{|c|} उ \\ \\  ५  \\\hline \end{tabular}\begin{tabular}{c|}ग \\ १\\ ५\\\hline \end{tabular}\begin{tabular}{c|}फ \\  \\० \\\hline \end{tabular}}
\vspace{3mm}

{\hyperref[86]{'पदहृतफलं\renewcommand{\thefootnote}{७}\footnote{*दस्वत*~।} मुखोनम्'} इत्यादिना कर्म~। फलं ०, पदेन \begin{tabular}{|c|}१\\ ५\\\hline \end{tabular}
हृतं\renewcommand{\thefootnote}{८}\footnote{स्वतं शून्यमेव .~।} शून्यमेव ०, मुखेन २ ऊनं}
{{\qt 'स्याद्योगे\renewcommand{\thefootnote}{९}\footnote{अग्य~।} वियदूनेभ्यो वियोगे तद्विपर्यय'} इति जातं २$+$, निरेकस्य
पदस्य \begin{tabular}{|c|}$\begin{matrix}
\mbox{{४}}\\
\mbox{{५}}
\end{matrix}+$\\\hline \end{tabular} दलेन \begin{tabular}{|c|}$\begin{matrix}
\mbox{{२}}\\
\mbox{{५}}
\end{matrix}+$\\\hline \end{tabular}}
{हृतं$^{\scriptsize{\hbox{{\color{blue}८}}}}$ \hyperref[33]{'(छेदांशविपर्यासे हरस्य विहिते) विधिः पूर्वः'} इति
प्रत्युत्पन्नः \begin{tabular}{|c|} १०\\ २\\\hline \end{tabular} छेदांशयोरपवर्तने\renewcommand{\thefootnote}{१०}\footnote {च्छेदां*~।}}
{ऋणयोश्च धनं भवति लब्धमुत्तरप्रमाणम् ५~।}
\vspace{3mm}

{अथ पदेऽज्ञाते न्यासः\textendash \hspace{2mm} \begin{tabular}{|c|}आ \\२\\\hline \end{tabular}\begin{tabular}{c|} उ  \\ ५ \\\hline \end{tabular}\begin{tabular}{c|}   ग  \\ ० \\\hline \end{tabular}\begin{tabular}{c|}  फ \\  ० \\\hline \end{tabular}
\vspace{3mm}

{\hyperref[87]{'अष्टोत्तरहतफलत'} इत्यादिना कर्म~। फलम् ० अष्टहतं ० तथोत्तरेण ५ हतं ०,
आदिः}
{२ द्विघ्नः ४ अस्य च प्रचयस्य चास्य ५ विवरं १ अस्य कृतिः १, अनया
युक्तमिति जातं १}
{अस्मान्मूलमिति १ मुखेन २ स्वगुणेन ४ ऊनमिति यावत्सम्भवं शुद्धौ\renewcommand{\thefootnote}{११}\footnote{शुद्धे~।}
व्ययराशि(शेष)ऋणं}
{३$+$, चयेन ५ सहितं धनर्णगत्या जातं २, द्वाभ्यां भक्तं १ तथा चयेन ५ भक्तं\begin{tabular}{c}१\\ ५\end{tabular}, एष}
{गच्छः~।}
\vspace{3mm}

{अथाद्युत्तरयोरज्ञातयोस्तन्मिश्रे सप्तके ज्ञाते न्यासः\textemdash}
\vspace{3mm}

\hspace{2cm}{आद्युत्तरयुतिः ७, ग\renewcommand{\thefootnote}{१२}\footnote{ग\begin{tabular}{c} १\\२\end{tabular}।}\begin{tabular}{c}१\\ ५\end{tabular}, फलं शून्यम्~।}
\vspace{3mm}

{\hyperref[88]{'विपदपदवर्गे'}त्यादिना कर्म~। पदं \begin{tabular}{|c|} १ \\५\\\hline \end{tabular} वर्गः \begin{tabular}{|c|} १\\ २५\\\hline \end{tabular}~। पदेन
हीनं सवर्णनं कृत्वा}
{यावत्सम्भवं शुद्धौ व्ययराशिशेषऋणं \begin{tabular}{|c|} $\begin{matrix}
\mbox{{४}}\\
\mbox{{२५}}
\end{matrix}+$\\\hline \end{tabular} दलं \begin{tabular}{|c|} $\begin{matrix}
\mbox{{२}}\\
\mbox{{२५}}
\end{matrix}+$\\\hline \end{tabular} मिश्रेण (७) गुणितं \begin{tabular}{|c|}$\begin{matrix}
\mbox{{१४}}\\
\mbox{{२५}}
\end{matrix}+$\\\hline \end{tabular}}

\newpage

\noindent{फलेन शून्येन हीनं \begin{tabular}{|c|}$\begin{matrix}
\mbox{{१४}}\\
\mbox{{२५}}
\end{matrix}+$\\\hline \end{tabular}\,, पदं\begin{tabular}{c} १\\ ५\end{tabular}व्येकं \begin{tabular}{|c|}$\begin{matrix}
\mbox{{४}}\\
\mbox{{५}}
\end{matrix}+$\\\hline \end{tabular} दलं \begin{tabular}{|c|}$\begin{matrix}
\mbox{{२}}\\
\mbox{{५}}
\end{matrix}+$\\\hline \end{tabular}
व्येकं}
{धनर्णगत्या\renewcommand{\thefootnote}{१}\footnote{*गत्या \begin{tabular}{|c|}$\begin{matrix}
\mbox{{७}}\\
\mbox{{५}}
\end{matrix}$ \end{tabular} धनत्रामापन्नं~।} \begin{tabular}{|c|}$\begin{matrix}
\mbox{{७}}\\
\mbox{{५}}
\end{matrix}+$\\\hline \end{tabular} ऋणत्वमापन्नं पदेना\textendash \begin{tabular}{|c|}१\\ ५\\\hline \end{tabular}\textendash हतं \begin{tabular}{|c|}$\begin{matrix}
\mbox{{७}}\\
\mbox{{२५}}
\end{matrix}+$\\\hline \end{tabular}\,, अनेन
परिशिष्टं \begin{tabular}{|c|}$\begin{matrix}
\mbox{{१४}}\\
\mbox{{२५}}
\end{matrix}+$\\\hline \end{tabular}}
{\hyperref[33]{'छेदांशविपर्यासे'} इति भक्तं जातम् (२) एष आदिः, एतेनैव मिश्रे (७)
विशुद्धे शेषं\renewcommand{\thefootnote}{२}\footnote{शेषे  \begin{tabular}{|c|}७ \\ २५ \end{tabular}~।}}
{५, एष उत्तरः~।}
\vspace{3mm}

{तस्मात्कात्रानुपपत्तिर्यदि शून्यमेव फलं स्यात्~। सत्यमेतत्, किन्तु
यत्र रूपादिकमादिरस्ति उत्तरं चयनाभावस्वभावं पदं च वस्तुसत्तात्मकं तत्र धनं न
किञ्चिदिति}
{विचित्रमिव प्रतिभाति~। यदि हि रूपन्यूनत्वात्पदस्योत्तरसंयोगो\renewcommand{\thefootnote}{३}\footnote{*संयोगे~।} 
नास्ति नापि प्रथमपदयोग}
{इति प्रमाणता तत्प्रथमपदधनांशोऽपि केन भक्षितः, प्रथमपदं राशित्वादेव
गच्छः}
{राशिगतकृतं चैवंविधे\renewcommand{\thefootnote}{४}\footnote{चैव विधे~।} विषये पञ्चांशपदस्य\renewcommand{\thefootnote}{५}\footnote{पंचाश*~।} फलमायात्येव द्वौ भागौ,
क्षेत्रगतयुतौ तु}
{युक्त्युपन्यासं विना कथं वासनान्वेषिणो
वस्तुगतिसंवादास्वादमलब्धचमत्कारमासादयन्ति\renewcommand{\thefootnote}{६}\footnote{*मासाचयंति~।}\,?}
{उच्यते~। प्रथमपदे क्षेत्रतो द्विहस्ते द्वितीयपदे सप्तप्रमाणं न
सम्भवतीति यावता न्यूनस्य सम्भवस्थानत्वम् ऋणत्वमायाति यथा त्रिभ्यो दशसु शोध्यमानेषु त्रयाणां
शोध्यसम्भवात् सप्त ऋणं}
{भवन्ति~। इह च यदि भूमिर्धनं\renewcommand{\thefootnote}{७}\footnote{*र्धन~।}  स्यात् तदा भूमुखयोगः षट् तस्यार्धं
त्रयः ततो लम्बेन रूपेण}
{गुणितं त्रयो हस्ताः स्युः, न चैतदादिपदे फलमिष्यते~। यदा तु ऋणगता
भूमिस्तदाधस्त्र्यश्रक्षेत्रफलशुद्धमुपरितनत्र्यश्रक्षेत्रफलमादिधनमुपपद्यते~। यथाधस्त्र्यश्रे
क्षेत्रफलं भूमुखयोगः $\begin{matrix}
\mbox{{१}}\\
\mbox{{२}}
\end{matrix}+$ अतोऽर्धं $\begin{matrix}
\mbox{{१}}\\
\mbox{{४}}
\end{matrix}+$ लम्बेन\renewcommand{\thefootnote}{८}\footnote{लबेन~।}\begin{tabular}{c}१\\ १०\end{tabular}गुणितं\renewcommand{\thefootnote}{९}\footnote{गुणितं\begin{tabular}{c}$\begin{matrix}
\mbox{{१}}\\
\mbox{{४}}
\end{matrix}+$\end{tabular}।}} $\begin{matrix}
\mbox{{१}}\\
\mbox{{४०}}
\end{matrix}+$, उपरि त्र्यश्रे
भूमुखयोगः \begin{tabular}{|c|}९\\ २\\\hline \end{tabular} अर्धं \begin{tabular}{|c|}९\\ ४\\\hline \end{tabular} लम्बेन\begin{tabular}{c}९\\ १०\end{tabular}हतं\renewcommand{\thefootnote}{१०}\footnote{हत~।}
\begin{tabular}{|c|} ८१ \\४०\\\hline  \end{tabular}~। अत्र एकश्चत्वारिंशद्भागः प्रथमधनेऽधिको वर्तते
सोऽधस्त्र्यश्रफलेन\renewcommand{\thefootnote}{११}\footnote{*धंरत्र्यश्र*~।} शुद्धस्तात्विकं\renewcommand{\thefootnote}{१२}\footnote{शुद्धेस्त्वात्विकं~।}}
{फलं न विघटयति~। एकश्चायं चत्वारिंशद्भागोऽधस्तनत्र्यश्रे
हास्तिकलम्बदशभागसम्मिते एव}
{क्षेत्रफलमायातीति~। लम्बदशभागेन लम्बस्य दशभागः युक्तो\renewcommand{\thefootnote}{१३}\footnote{शून्यं~।}
{लम्बपञ्चभागः करोति\renewcommand{\thefootnote}{१४}\footnote{गो शून्यीकरोति~।}, दशभागद्वयस्य पञ्चभागात्मकत्वात्तदूर्ध्वं तात्विकस्थितेः~। तथाहि तस्मिन्
प्रदेशे, हास्तिकक्षेत्रमुखं \begin{tabular}{|c|} ९ \\२\\\hline \end{tabular}}
{धरया  \begin{tabular}{|c|}$\begin{matrix}
\mbox{{१}}\\
\mbox{{२}}
\end{matrix}+$\\\hline \end{tabular} ऊनमिति योगः ५ इष्टावलम्बेन \begin{tabular}{|c|}१\\ ५\\\hline \end{tabular} गुणितः\renewcommand{\thefootnote}{१५}\footnote{गुणितं~।} १ भूम्या $\begin{matrix}
\mbox{{१}}\\
\mbox{{२}}
\end{matrix}+$
युतमिति वियुतं\begin{tabular}{c}१\\ २\end{tabular},}
{एषैव भूः अनन्तरस्य पञ्चभागचतुष्टयात्मकलम्बस्य\renewcommand{\thefootnote}{१६}\footnote{*लंबनस्य~।} नवद्विभागवदनस्य
क्षेत्रस्य, तत्र च फलं,}
{भूमुखयोगः ५ अर्धं\begin{tabular}{c}५\\ २\end{tabular}लम्बेन\begin{tabular}{c}४\\ ५\end{tabular}हतमिति फलं २ एतन्न्याय्यम्~। अत
ऊर्ध्वं तत्र फलहानिः,}
{न च वदनमुपचयितुं शक्यं, उत्तरोत्तरपदधनविघटनात्, इति नास्ति
न्यायगणितयोरुभयोर्भेदः\renewcommand{\thefootnote}{१७}\footnote{*रूदयभेदः~।}\,।}
\vspace{3mm}

{अथात्रैवं क्षेत्ररचना~। पदमेकं तल्लम्बः १, चयस्यास्य ५ दलं\renewcommand{\thefootnote}{१८}\footnote{दले~।}\begin{tabular}{c}५\\ २\end{tabular}एतेन मुखं २ हीनं}
{व्ययाधिक्यादृणं $\begin{matrix}
\mbox{{१}}\\
\mbox{{२}}
\end{matrix}+$, एषा धरा एषैव सचया \begin{tabular}{|c|} ९\\ २\\\hline \end{tabular}\,, एतद्वक्त्रम्~।}

\begin{quote}
\hspace{3cm} \hyperref[81]{'कुर्यात्सूत्रेण तच्चिह्नम्~॥}\\
\hyperref[82]{भूमुखरेखाग्रस्पृक् प्रसारयेत् सूत्रमुभ(य)तो बाहू~।'} 
\end{quote}

\newpage

\noindent{इति कर्म नास्ति, \hyperref[82]{'सूत्रप्रसृतिर्वज्रवत्'} इत्यादि\renewcommand{\thefootnote}{१}\footnote{*तिवज्र*~।} कार्यम्~। एवं त्र्यश्रद्वये जाते उपरि त्र्यश्रे}
{लम्बः \begin{tabular} {|c|}९\\ १०\\\hline \end{tabular}	अधस्त्र्यश्रे लम्बः \begin{tabular}{|c|} १\\ १०\\\hline \end{tabular}}~।
\vspace{4mm}

{न्यासः\textendash \,}
\vspace{-2mm}

\hspace{10mm} \includegraphics[width=10cm, height=4cm]{Images/page-0169as.jpeg}
\vspace{4mm}

{एतत् प्रथमपदक्षेत्रम्~। धरोनमुखं ५, इष्टावलम्बेन \begin{tabular}{|c|} १\\ ५\\\hline \end{tabular}  गुणितं १,
अवनियुतं \begin{tabular}{|c|}१\\ २\\\hline \end{tabular}\,, एतत् वदनं इष्टावलम्बे क्षेत्रे~।}
\vspace{3mm}

{न्यासः\textendash \,}
\vspace{-2mm}

\hspace{15mm} \includegraphics[width=6cm, height=3cm]{Images/page-0169bs.jpeg}
\vspace{5mm}

{भूमुखयोगः शून्यं ०, अर्धं ०, लम्बेन\renewcommand{\thefootnote}{२}\footnote{शून्यं . अर्धं लवेण~।} \begin{tabular}{|c|} १ \\५\\\hline \end{tabular}  हतमिति (फलं) शून्यम्~। यदा उपरि त्र्यश्रे}
{भूमुखयोगार्धम्\renewcommand{\thefootnote}{३}\footnote{*योगार्धं लंवेन \begin{tabular}{|c|} १ \\२\end{tabular}~।} \bigg(\begin{tabular}{c}१\\ ४\end{tabular}, लम्बेन\begin{tabular}{c}१\\ १०\end{tabular}गुणितं\begin{tabular}{c}१\\ ४०\end{tabular}\bigg), अधस्त्र्यश्रे
भूमुखयोगार्धम् \begin{tabular}{|c|}$\begin{matrix}
\mbox{{१}}\\
\mbox{{४}}
\end{matrix}+$\\\hline \end{tabular}~\bigg(लम्बेन\begin{tabular}{c}१\\ १०\end{tabular}}
{गुणितं\begin{tabular}{c}१\\ ४०+\end{tabular}\bigg), ऋणधनसमत्वात्फलं शून्यम्~।}
\vspace{4mm}

{अपरः प्रश्नः\textendash}

\phantomsection \label{e104}
\begin{quote}
    
{\eg कुतपे तैलसम्पूर्णे सूक्ष्मछिद्रमधोऽभवत्\renewcommand{\thefootnote}{४}\footnote{*मधो भवेत्~।}\,। \\
 तेन क्षरति तैलं सत् कुतपो\renewcommand{\thefootnote}{५}\footnote{सक्त तपो~।} योजनत्रयम्~॥~१०४~॥ \\
 नेतव्यो भाटके तस्य प्रथमे योजने दश~।\\
 क्रमशः परयोर्द्व्यूनाः पणाः किं क्रोशभाटके~॥~१०५~॥}\end{quote}

\newpage

{स्नेहपात्रे तैलेन पूर्णे तदधस्तात्सूक्ष्मं सुषिरं जातं, तेन तैलं शनैः
स्रवति, स च कुतपो}
{योजनत्रयं प्रापयितव्यः~। तस्य च तत्प्रापकस्य प्रथमे योजने दश पणाः
भाटकं, द्वितीयेऽष्टौ,}
{तृतीये षट् इति तस्य क्रोशभाटकं\renewcommand{\thefootnote}{१}\footnote{त्क्रोश*~।} किं ? अयं भावः\textendash \,यस्याः प्रथमपदं दश
द्वितीयतृतीये}
{यथोत्तरं न्यूने चतुर्थो भागः पदं सा श्रेढी समग्रपदसमासेन किंफला
स्यादिति~।}
\vspace{3mm}

{न्यासः\textendash \hspace{2mm} \begin{tabular}{|c|}आ \\ १०\\ \\\end{tabular}
\begin{tabular}{c|} उ \\ २$+$ \\ \\\end{tabular}
\begin{tabular}{c|}  प \\  १ \\४\end{tabular}~। 
\vspace{3mm}

{पदं\begin{tabular}{c}१\\ ४\end{tabular}व्येकमिति रूपं सवर्णितं \begin{tabular}{|c|} ४ \\४ \end{tabular} न पततीति\renewcommand{\thefootnote}{२}\footnote{यतीति~।} विपरीतशुद्ध्या
व्ययराशि(शेष)ऋणं}
{\bigg(\,\begin{tabular}{|c|}$\begin{matrix}
\mbox{{३}}\\
\mbox{{४}}
\end{matrix}+$\\\hline \end{tabular} अस्य दलं\bigg) \begin{tabular}{|c|}$\begin{matrix}
\mbox{{३}}\\
\mbox{{८}}
\end{matrix}+$\\\hline \end{tabular} चयः २$+$ गुणितः {\qt 'ऋणमृणधनयोर्घातो धनमृणयो'}रिति
ऋणात्मकं}
{सद्धनात्मकं जायते \begin{tabular}{|c|}३\\ ४\end{tabular} , आदिना १० युतं \begin{tabular}{|c|} ४३\\ ४\end{tabular} , एतत्\renewcommand{\thefootnote}{३}\footnote{यतत्~।} पदेन \begin{tabular}{|c|} १\\ ४\end{tabular}
सङ्गुणितं\renewcommand{\thefootnote}{४}\footnote{*तं \begin{tabular}{|c|} ४३ \\१९\end{tabular}~।} \begin{tabular}{|c|} ४३\\ १६\end{tabular}\,, अतो}
{लब्धपणौ द्वौ २ शेषं\renewcommand{\thefootnote}{५}\footnote{शेषं\begin{tabular}{c|}११ \\१९\end{tabular}~।} \begin{tabular}{|c|}११ \\१६\end{tabular}  पणाभावात्काकिणीलाभार्थं
चतुर्गुणितांशं\renewcommand{\thefootnote}{६}\footnote{*शं \begin{tabular}{|c|}४४ \\१९\end{tabular}} \begin{tabular}{|c|} ४४\\ १६\end{tabular} अतो लब्धं
{काकिन्यौ २ काकिनीभागाश्च \begin{tabular}{|c|} ३ \\४\end{tabular}~।}
\vspace{3mm}

{पूर्व(व)दत्रापि क्षेत्ररचना\textendash \,पदमेकं तल्लम्बः १, चयस्यास्य\renewcommand{\thefootnote}{७}\footnote{चयास्या*~।} २$+$
दलेन १$+$ हीनं मुखं}
{१० {\qt 'शोध्यं यदा धनमृणादृणं\renewcommand{\thefootnote}{८}\footnote{धनमृणामृणं~।} धनाद्वा तदा क्षेप्यम्'} इति जातं ११ एषा
धरा, एषैव चयेन २$+$}
{सहिता {\qt 'धनयोर्धनमृणमृणयोर्धनर्णयोरन्तरम्'} इति जातं ९, एतद्वक्त्रम्~। रूपलम्बे इदं क्षेत्ररचनम्~।}
\vspace{3mm}

{न्यासः\textendash \,}
\vspace{-2mm}

\hspace{15mm} \includegraphics[width=8cm, height=2cm]{Images/page-0170as.jpeg} 
\vspace{3mm}

{इष्टलम्बके क्षेत्रदर्शनं यथा\textendash \,धरया ११ ऊनं मुखं ९ {\qt 'अधिकमूना'}दिति
वचनाद्व्ययराशिः ऋणं २$+$, इष्टलम्बकेन\renewcommand{\thefootnote}{९}\footnote{*लवकेन~।} \begin{tabular}{|c|} १\\ ४\end{tabular}  गुणितं \begin{tabular}{|c|}$\begin{matrix}
\mbox{{१}}\\
\mbox{{२}}
\end{matrix}+$ \end{tabular} अवनि ११ युतमिति
{\qt 'धनर्णयोरन्तरम्'} इति}
{जात(म्) \begin{tabular}{|c|} २१\\ २
\end{tabular}\,, एतत् वक्त्रम्\renewcommand{\thefootnote}{१०}\footnote{वत्क्रं~।}\,।}
\vspace{3mm}

{न्यासः\textendash \,}

\hspace{2mm} \includegraphics[width=11cm, height=2cm]{Images/page-0170bs.jpeg} 

{अत्र भूवदनसमासः  \begin{tabular}{|c|} ४३\\ २\end{tabular}\,, अर्धं \begin{tabular}{|c|} ४३ \\४ \end{tabular}\,, लम्बेन\renewcommand{\thefootnote}{११}\footnote{लवेन~।} \begin{tabular}{|c|}१ \\४\end{tabular}
गुणितं\renewcommand{\thefootnote}{१२}\footnote{*तं \begin{tabular}{|c|} ४३\\ १९\end{tabular}~।} \begin{tabular}{|c|} ४३\\ १६\end{tabular} , अतो लब्धपणौ}
{२ काकिन्यौ २ काकिनीभागाश्च\renewcommand{\thefootnote}{१३}\footnote{*गाश्च ३~।} \begin{tabular}{c}३\\ ४\end{tabular}\,, इदं च \hyperref[e104]{'कुतपे तैलसम्पूर्णे'}
इत्यादिकस्य फलम्\renewcommand{\thefootnote}{१४}\footnote{*कस्य...~।}\,।}

\newpage

{करणसूत्रमार्यापूर्वार्धम्\textendash}

\phantomsection \label{86.1}
\begin{quote}
{\bs आदिः पदहृतगणितं निरेकगच्छघ्नचयदलेनोनम्~।}\end{quote}

{गणितं पूर्वसूत्रोक्तमाद्युत्तर(पद)प्रकृतिकं श्रेढीसम्बद्धं
सङ्कलिताख्यं\renewcommand{\thefootnote}{१}\footnote{*ताध्यं~।}, तत् पदेन गच्छेन}
{हृतं\renewcommand{\thefootnote}{२}\footnote{स्वतं~।} भक्तं, निरेकेन गच्छेन पदेन हतस्य चयस्य दलेनार्धेनोनं
हीनमवशिष्टमादिः प्रथमधनं\renewcommand{\thefootnote}{३}\footnote{*धन~।} भवति~।}
\vspace{3mm}

{इदं \,च \,पूर्वसूत्रोक्तकर्मविपरीतात्मकं, \,तथाहि\textendash \;\hyperref[85]{'व्येकपदार्धघ्नचयः \,सादिः \,पदसङ्गुणो \,भवेत् गणितम्'} इति गणितं यत्पदसङ्गुणं तदिह पदहृतमुक्तं, आदेरेव
ज्ञेयत्वात्सादिरित्येतस्य}
{विपरीतं कर्म नास्ति, स चायमादिस्तत्र व्येकपदार्धघ्नचयः\renewcommand{\thefootnote}{४}\footnote{*परार्थघ्ने चयः~।} प्रक्षिप्तः
इह तु \hyperref[86.1]{'निरेकगच्छघ्नचयदलेनो(न)म्'} इति विपरीतकर्मोक्त्या मिश्रात्पृथक्पृथक्पारिशेष्यादानीतः\renewcommand{\thefootnote}{५}\footnote{*पारिशेषादा*~।}\,। ननु च तत्र पदार्धेन}
{हते\renewcommand{\thefootnote}{६}\footnote{पदार्थेन हक्ते~।} चये प्रक्षेप उक्त इह तु पद(घ्न)चयार्धेन विशुद्धिरुक्तेति कथं
युज्यते\,? उच्यते~। नैष दोषः~।}
{तदर्धेन वा घातः तद्घातस्य वार्धीकरणमित्युपायभेदमात्रमेतत्\renewcommand{\thefootnote}{७}\footnote{वाधींकरणमित्युपायभेदे*~।} न फलभेदः~। अथेदं व्यक्तम्}
{इदं यदि पूर्वसूत्रोक्तकर्मवैपरीत्याल्लब्धं पूर्वसूत्रोक्तं कर्म
त्वयुक्तिकमिति उच्यते~। इह तु सङ्कलिते}
{प्रथमपदघनं तावत्प्रत्येकं समस्तेषु पदेषु परिसमाप्यते
ततस्तत्पदसङ्ख्यागुणं भवति, उत्तरं तु}
{द्वितीयपदादिष्वेकादिगुणं\renewcommand{\thefootnote}{८}\footnote{*यमदा*~।} भवति \,यावत् \,व्येकपदगुणितम् \,अन्त्यस्य\renewcommand{\thefootnote}{९}\footnote{*मर्त्यस्य~।}
जायते, \,प्रथमपदधनसम्मितं\renewcommand{\thefootnote}{१०}\footnote{*धनं संमितं~।}}
{च \,धनं \,तत्रौपनीतिकं तत्सहितमन्त्यधनं भवति यदाहुः\textendash \,{\qt 'पदमेकहीनमुत्तरगुणितं संयुक्तमादिनान्त्यधनम्'} इति, मध्यपदस्य
त्वन्त्यपदापद्यमानोत्तरधनार्धमादियुतं धनं भवति}
{तथा च {\qt 'आदियुतान्त्यधनार्धं मध्यधनम्'} आहुः, भवता
त्वन्त्यपदापद्यमानमुत्तरधनार्धमादियुक्तमुक्तं}
{यदादिधनविहीनस्य दलेनादिधनयुतेन साम्यमेति, न तु
संवादवादवाक्यार्धेन\renewcommand{\thefootnote}{११}\footnote{*क्यार्थेन~।} कश्चित् समानः~।}
{पन्था गन्तव्यं\renewcommand{\thefootnote}{१२}\footnote{गंतव्य~।} तु समानमेव~। कथम्\renewcommand{\thefootnote}{१३}\footnote{कथ~।}? आदिधनवियुतान्त्यधनार्धे
यत्पुनरादिधनसंयोजनं क्रियते}
{तत्खलु समग्रान्त्यधनेऽर्धीक्रियमाणे तदन्तर्गतमादिधनमप्यर्धीकारितं\renewcommand{\thefootnote}{१४}\footnote{*प्यधी*~।}
तद्वियुक्तान्त्यधनमन्त्यपदापद्यमानोत्तरधनमेवान्वितं स्यान्न त्वादियुक्तं मध्यधनं भवतीति~। 
दलनात्प्रागादिः पृथक् क्रियते}
{अनन्तरं च प्रक्षिप्यते तान्येतानि त्रीणि कर्माणि वियोजनं दलनं
संयोजनमिति, संवादवाक्यं च}
{सूत्रं, सूत्रकाराश्च सूत्रतः कर्मतश्च लाघवमर्थयन्तीति कर्मद्वयेन
संवादवाक्यसूत्रमर्थमासादयेत्,}
{तथाहि अन्त्यधने त्वनर्धितेऽपि भूय आदिधनं प्रक्षिप्य यदर्धीकरणं
तेनादिधनं दलनाद्रक्षितं भवति}
{द्विगुणं हि तत्र जातमन्तःस्थितिपारतन्त्र्यादर्धीभवेत्
एकगुणमर्धीकृततुल्यमेवास्ति, तस्मादनेनापि}
{प्रकारेण मध्यधनमानीयते इति~। ज्ञाते मध्यधने तत् पदसङ्ख्यागुणितं
सङ्कलितं भवति~। कथमिति चेत् उच्यते, मध्यधनात्पूर्वोत्तरपदधनादिकक्ष्याक्रमेण द्वन्द्वशः
क्रमेण मिश्रीक्रियमाणानि च}
{मध्यधनात् द्विगुणानि भवन्ति, तानि लघूकरणार्थं प्रत्येकं समतया
परिकल्प्यमा(ना)नि मध्यधनतुल्यान्येव भवन्ति\renewcommand{\thefootnote}{१५}\footnote{भवति~।}, ततश्च तानि पदसङ्ख्यया गुणितानि
सङ्कलितस्वरूपमासादयन्तीति,}

\newpage

\noindent{तदाहुः\textendash \,{\qt 'मध्यधनं पदगुणं गणितम्'} इति~। इह चाचार्यो
\hyperref[85]{'व्येकपदेने'}त्यनेनान्त्यधने\renewcommand{\thefootnote}{१}\footnote{चाचर्यो व्येकपदेन्त्यधने मध्यपदे~।}}
{चयस्यापत्तिं साधयति नान्त्यधनं सङ्कलितानयनात्, अन्यथा तु चय आदिधनं
प्रक्षिप्येत~।}
{किन्तु तस्मिन्नर्धीकृते आदिप्रक्षेपे मध्यधनतुल्यो राशिर्जायते,
तत्कृतम् \hyperref[85]{'अर्धघ्नचयः सादिरि'}ति~।}
{यद्यप्यर्धीकृत्य चयेनात्र गुण(न)मुक्तं तथापि
च(य)गुणितदलीकृततुल्यमेतदिति नास्ति}
{कश्चिद्दोषः\renewcommand{\thefootnote}{२}\footnote{कश्चिदोषः~।}\,।{\qt 'पदगुणितम्'} इति च स्पष्टमेवेति~। 
सङ्कलितानयनसूत्रोक्तकर्मणीयं युक्तिः~।}
{तद्विपरीतकर्मणा चेहाद्यानयनमुक्तमेव~।}
\vspace{2mm}

{प्रथमोदाहरणे\renewcommand{\thefootnote}{३}\footnote{अथोदाहरणं~।} न्यासः\textendash}
\vspace{2mm}

\hspace{20mm} {आदिर्न ज्ञायते ~\begin{tabular}{|c|}उ \\ ३ \end{tabular}\begin{tabular}{c|} पदं \\ ५\end{tabular}\begin{tabular}{c|}धनं  \\ ४० \end{tabular}}
\vspace{2mm}

{धनं\renewcommand{\thefootnote}{४}\footnote{ध ४~।} ४० पदेन पञ्चभिः ५ अमीभिर्हृतं\renewcommand{\thefootnote}{५}\footnote{*भिः स्वतं~।}  जातम् ८ , एतत् गच्छस्यास्य
५ निरेकस्य}
{४ चयेनानेन ३ गुणितस्य १२ दलेनानेन\renewcommand{\thefootnote}{६}\footnote{*न ९~।} ६ ऊनमिति लब्धमादिः २~।}
\vspace{2mm}

{द्वितीयोदाहरणे न्यासः\textendash}
\vspace{2mm}

\hspace{20mm} {आदिर्न ज्ञायते, चयः ३, गच्छः \begin{tabular}{|c|} १\\ २\end{tabular} , सङ्कलितं \begin{tabular}{|c|} ५\\ ८\end{tabular}}~। 
 \vspace{2mm}

{गणितं \begin{tabular}{|c|} ५\\ ८\end{tabular} पदेनानेन \begin{tabular}{|c|} १\\ २\end{tabular} हृतं\renewcommand{\thefootnote}{७}\footnote{स्वतं} \begin{tabular}{|c|} ५\\
४\end{tabular}\,, एतत् गच्छेन \begin{tabular}{|c|}१\\ २\end{tabular} निरेकेन \begin{tabular}{|c|}$\begin{matrix}
\mbox{{१}}\\
\mbox{{२}}
\end{matrix}+$\end{tabular} हतस्य चयस्यास्य ३ जातस्य \begin{tabular}{|c|}$\begin{matrix}
\mbox{{३}}\\
\mbox{{२}}
\end{matrix}+$\end{tabular} दलेनानेन \begin{tabular}{|c|} $\begin{matrix}
\mbox{{३}}\\
\mbox{{४}}
\end{matrix}+$\end{tabular}
ऊनमिति धनर्णवृत्तेन युक्तं जातं २, अयमादिः~।}
\vspace{1mm}

{तृतीयोदाहरणे\renewcommand{\thefootnote}{८}\footnote{*हरणो~।} न्यासः\textendash}
\vspace{2mm}

\hspace{20mm} {आदिर्न ज्ञायते, उ\renewcommand{\thefootnote}{९}\footnote{८ गच्छः \begin{tabular}{|c|} १ \\५ \end{tabular}\,, सङ्कलितम् .~।} ५, गच्छः \begin{tabular}{|c|} १ \\५ \end{tabular}\,, सङ्कलितम् ०~।}
\vspace{2mm}

{कर्म\textendash \,गणितं\renewcommand{\thefootnote}{१०}\footnote{गणितं~।} ० पदेनानेन \begin{tabular}{|c|}१\\ ५ \end{tabular} हृतमिति\renewcommand{\thefootnote}{११}\footnote{स्वतमिति शून्यमेव .~।} शून्यमेव ०, एतत् गच्छेन \begin{tabular}{|c|} १\\ ५ \end{tabular} निरेकेन \begin{tabular}{|c|}$\begin{matrix}
\mbox{{४}}\\
\mbox{{५}}
\end{matrix}+$\end{tabular} हतस्य चयस्यास्य ५ जातस्य ४$+$ दलेन २$+$ ऊनमिति
युक्तं धनं जातं लब्धमादिः २~।}
\vspace{2mm}

{चतुर्थोदाहरणे न्यासः\textendash}
\vspace{2mm}

\hspace{20mm} {आदिर्न ज्ञायते, उ २$+$, प \begin{tabular}{|c|}१\\ ४\end{tabular} गणितं\renewcommand{\thefootnote}{१२}\footnote{गणितं \begin{tabular}
{|c|}४३\\१९\end{tabular}~।} \begin{tabular}{|c|} ४३\\ 
१६ \end{tabular}~।}
\vspace{2mm}

\hspace{0.01mm} \bigg(गणितं \;\begin{tabular}{c}४३\\ १६\end{tabular}\bigg)\,, \;एतत् \;पदेनानेन \;\begin{tabular}{|c|}१\\ ४\end{tabular} \;हृतं$^{\scriptsize{\hbox{{\color{blue}७}}}}$ \;\begin{tabular}{|c|}४३ \\४\end{tabular}\,, \;गच्छेनानेन \;\begin{tabular}{|c|}१\\ ४\end{tabular} \,निरेकेन \begin{tabular}{|c|}$\begin{matrix}
\mbox{{३}}\\
\mbox{{४}}
\end{matrix}+$\end{tabular} हतस्य
{चयस्यास्य २$+$ जातस्य {\qt 'धनमृणयोर्घात'} इति धनात्मकस्य \begin{tabular}{|c|}३\\ २\end{tabular}  
 दलेन \begin{tabular}{|c|}३\\ ४\end{tabular} ऊनं\renewcommand{\thefootnote}{१३}\footnote{ऊन~।} {जातं १०}, 
\noindent{अयमादिः~।}
\vspace{2mm}

{अथादिपदधनेषु ज्ञातेषु प्रचयाज्ञाने\renewcommand{\thefootnote}{१४}\footnote{*ज्ञान~।} तदानयनार्थं
करणसूत्रमार्यापरार्धमाह\textemdash}

\phantomsection \label{86}
\begin{quote}
    
{\bs पदहृतफलं\renewcommand{\thefootnote}{१५}\footnote{पदंस्वतफलं~।} मुखोनं निरेकददलहृतं\renewcommand{\thefootnote}{१६}\footnote{*लस्वतं} प्रचयः~॥~८६~॥}\end{quote}

{फलं श्रेढीक्षेत्रोद्भवं गणितं पदेन गच्छेन हृतं$^{\scriptsize{\hbox{{\color{blue}७}}}}$ भक्तं मुखेनादिनोनं
रहितं स(त्) पदस्य}
{निरेकस्य रूपोनकस्य दलेनार्धेन हृतं$^{\scriptsize{\hbox{{\color{blue}७}}}}$ भक्तं प्रचय उत्तरो भवति~।}

\newpage

{अथ चेयं युक्तिः~। \hyperref[86]{'पदहृतफलम्'}\renewcommand{\thefootnote}{१}\footnote{पदस्वत*~।} इति मध्यधनान्वेषणं, \hyperref[86]{'मुखोनम्'} इति तत्र
चयधनापत्तिपर्येषणा, \hyperref[86]{'निरेकपददलम्'} इति मध्यसङ्ख्या विरूपा जनिता उक्तं हि
प्रतिपदमेकोनतत्सङ्ख्यागुणः}
{प्रचयो वर्तते इति ततस्तया विरूपया मध्यपदसङ्ख्यया
मध्यपदापद्यमानप्रचयसङ्ख्या भक्ता प्रचयो भवतीति~।}
\vspace{2mm}

{अथ पूर्वोक्तप्रथमोदाहरणे उत्तरेऽज्ञाते न्यासः\textendash}
\vspace{1mm}

\hspace{2cm}{आ २, उत्तरो न ज्ञायते, पदं ५, सङ्कलितम् ४०~।}
\vspace{2mm}

{कर्म\textendash \,फलं ४० पदेन ५ हृतं\renewcommand{\thefootnote}{२}\footnote{फलं ४० पदेन ५ स्वतं~।} ८, एतन्मुखेनानेन २ ऊनं\renewcommand{\thefootnote}{३}\footnote{ऊनं ६~।} ६, एतत्
पदस्यास्य ५}
{निरेकस्य ४ दलेन २ हृतमिति\renewcommand{\thefootnote}{४}\footnote{स्वत*~।} लब्धमुत्त(र)प्रमाणम् ३~।}
\vspace{2mm}

{अथ द्वितीयोदाहरणे उत्तरेऽज्ञाते न्यासः\textendash}
\vspace{1mm}

\hspace{20mm} {आदिः २, उत्तरो न ज्ञायते, गच्छः \begin{tabular}{|c|} १\\ २\end{tabular} , सङ्कलितम् \begin{tabular}{|c|} ५\\ ८\end{tabular}~।}
\vspace{2mm}

\hspace{0.01mm} {\bigg(फलं \begin{tabular}{|c|}५\\ ८\end{tabular}\bigg) एतत् पदहृतं\renewcommand{\thefootnote}{५}\footnote{पदस्वतं~।} \begin{tabular}{|c|} ५\\ ४\end{tabular} 
मुखेन ऊनं \begin{tabular}{|c|}$\begin{matrix}
\mbox{{३}}\\
\mbox{{४}}
\end{matrix}+$\end{tabular}\,, निरेकगच्छस्य\renewcommand{\thefootnote}{६}\footnote{निरेगच्छेस्य~।} \begin{tabular}{|c|}$\begin{matrix}
\mbox{{१}}\\
\mbox{{२}}
\end{matrix}+$\end{tabular} दलेन \begin{tabular}{|c|}$\begin{matrix}
\mbox{{१}}\\
\mbox{{४}}
\end{matrix}+$\end{tabular} हृतं {\qt 'ऋणहृतमृणं\renewcommand{\thefootnote}{७}\footnote{स्वतं ऋणस्यत*~।} धनम्'} इति च लब्धमुत्तरप्रमाणम् ३~।}
\vspace{2mm}

{अथ तृतीयोदाहरणे उत्तरेऽज्ञाते न्यासः\textemdash}
\vspace{1mm}

\hspace{2cm}{आदिः २, चयो न ज्ञायते, गच्छः \begin{tabular}{|c|} १\\ ५\end{tabular}\,, सङ्कलितम्\renewcommand{\thefootnote}{८}\footnote{सङ्कलितम् .~।}  ०~।}
\vspace{2mm}

{पदहृतं\renewcommand{\thefootnote}{९}\footnote{पदस्यतं~।}  फलं शून्यमेव, मुखेन २ ऊनमिति २$+$, एतत् पदस्य \begin{tabular}{|c|} १\\ ५\end{tabular}
 निरेकस्य \begin{tabular}{|c|}$\begin{matrix}
\mbox{{४}}\\
\mbox{{५}}
\end{matrix}+$\end{tabular}\,, दलेन \begin{tabular}{|c|} $\begin{matrix}
\mbox{{२}}\\
\mbox{{५}}
\end{matrix}+$\end{tabular}  हृतमिति लब्धमुत्तरप्रमाणम् ५~।}
\vspace{2mm}

{अथ चतुर्थोदाहरणे उत्तरेऽज्ञाते न्यासः\textendash}
\vspace{1mm}

\hspace{2cm}{आदिः १०, उत्तरो न ज्ञायते, गच्छः \begin{tabular}{|c|}१ \\४ \end{tabular}, फलम् \begin{tabular}{|c|} ४३
\\१६ \end{tabular}~।}
\vspace{2mm}

{पदहृतफलं\renewcommand{\thefootnote}{१०}\footnote{पदस्यत*~।}  \begin{tabular}{|c|}४३\\ ४\end{tabular}\,, मुखेन १० ऊनं \begin{tabular}{|c|} ३\\ ४ \end{tabular}\,,
निरेकपदस्य \begin{tabular}{|c|}$\begin{matrix}
\mbox{{३}}\\
\mbox{{४}}
\end{matrix}+$\end{tabular} दलेन \begin{tabular}{|c|}$\begin{matrix}
\mbox{{३}}\\
\mbox{{८}}
\end{matrix}+$\end{tabular} हृतं {\qt 'भक्तमृणेन धनमृणम्'} इति ऋणात्मकं लब्धमुत्तरप्रमाणम् (२$+$)~।}
\vspace{2mm}

{अथादिप्रचयसङ्कलितेषु ज्ञातेषु पदे चाज्ञाते तदानयनार्थं
करणसूत्रमार्यामाह\textemdash}

\phantomsection \label{87}
\begin{quote}
    
{\bs अष्टोत्तरहतफलतो द्विगुणादिप्रचयविवरकृतियुक्तात्~।\\
 मूलं द्विगुणमुखोनं सचयं द्विचयोद्धृतं गच्छः~॥~८७~॥}\end{quote}

{फलाच्छ्रेढीसङ्कलितात्\renewcommand{\thefootnote}{११}\footnote{फलच्छ्रे*~।}  अष्टाभिस्तथोत्तरेण प्रचयेन तद्घातेन वा
गुणितात्, द्विगुणितस्यादेः प्रचयस्य वा गुणितस्यैव यद्विवरमन्तरं तस्य या
कृतिर्वर्गस्तद्युक्तात्, यन्मूलं वर्गमूलं,}
{त(त्) द्विगुणेन मुखेनादिना ऊनं, चयेन सहितं, द्विगुणचयेनोद्धृतं\renewcommand{\thefootnote}{१२}\footnote{*नोद्धातं~।} 
भक्तं गच्छो भवतीति~।}
\vspace{2mm}

{कात्रोपपत्तिरिति चेत् उच्यते~। सर्वाणीमानि करणसूत्राणि
युक्तिबीजमूलानि\renewcommand{\thefootnote}{१३}\footnote{युक्तबी*~।}\,।}
{युक्तिमूलं\renewcommand{\thefootnote}{१४}\footnote{युक्तमूलं~।}  यथा सङ्कलितं दर्शितं, वर्गादिमूलानि बीजमूलानि तथा
इदमेव सङ्कलितमूलानयनम्~।}

\newpage

\noindent{इह हि पदं न ज्ञायते अतस्तदव्यक्तमिति सञ्ज्ञायते~। एतदीयप्रमाणस्य
चानियतत्वाद्यावत्तावदिति}
{व्यवहारः~। गणितकर्मण्यपि यावत्तावच्छब्दाद्यक्षरेण 'या'
इत्यनेनास्योपलक्षणं भवति~। अज्ञातप्रमाणस्यापि चास्य समुदायरूपत्वात् समुदायगतेनैव सङ्ख्याविशेषेण योगो
भवति, यथा एको}
{यावत्तावदिति\renewcommand{\thefootnote}{१}\footnote{यावतावानिति~।} न्यासः या (१), द्वाविति न्यासः (या २), त्रय इति
न्यासः (या ३), एवमन्यत्}
{यदा च समुदायान्तरं प्रथमोद्दिष्टावुक्तराशिविलक्षणावयवसङ्ख्या जायते तदा
तस्य\renewcommand{\thefootnote}{२}\footnote{त्तस्य~।} यावत्तावत्प्रमाणत्वेऽपि पूर्वस्मादव्यक्तराशेर्वैलक्षण्यसूचनार्थं पूर्वाचार्यैः
कालकनीलकपीतश्वेतहरितादिविशेषवाचिशब्दान्यतमसञ्ज्ञा\renewcommand{\thefootnote}{३}\footnote{कालकंनी*~।} प्रवर्तिता~। गणितकर्मण्यपि\renewcommand{\thefootnote}{४}\footnote{गणितं*~।}
सञ्ज्ञाशब्दाद्यक्षरेणोपलक्षणं\renewcommand{\thefootnote}{५}\footnote{*रेणाप*~।} कृतमज्ञातप्रमाणस्यापि 'का' इत्यादि\renewcommand{\thefootnote}{६}\footnote{चेत्यादि~।} प्राग्वत् न्यासः\textendash \,का १, का २, का ३~। 
अत्रोपर्यप्यनन्ताव्यक्तराशिसमुपजन्मनि शेषसञ्ज्ञान्यतमयोजना, शेषं प्राग्वत्~। पृथक् पृथक् सञ्ज्ञाकरणं 
च भिन्नाव्यक्तानां}
{योगवियोगोक्तकर्मपरिव्यवहारात्~। यदा हि यावत्तावति कालको नीलकोऽन्यो
वाव्यक्तो वा कैश्चि-द्रूपराशिः प्रक्षिप्यते तदा समीप एव
स्वोपलक्षणवाचिवर्णोऽवस्थाप्यते, ततः}
{परस्परमङ्कयोगः क्रियते तौ च राशी बहवो वा एकराशिवत् भवन्ति~। स च
राशि(र)व्यक्तराशीनां भेदे\renewcommand{\thefootnote}{७}\footnote{भेद~।} राशिवधे भावितैकसञ्ज्ञः कार्यः~। तदर्थमपि
सञ्ज्ञाभेदप्रवर्तनम्~। अथान्यान्यपि सन्ति}
{प्रयोजनानि (अ)प्रासङ्गिकत्वान्नेहोच्यन्ते~।}
\vspace{3mm}

{एवंगते पदमिह या १~। अतः सङ्कलितमानीयते, यथा न्यासः\textendash}
\vspace{3mm}
 
\hspace{20mm} {आ २, उ ३, ग या १~।}
\vspace{3mm}

{कर्म\textendash \,पदं\renewcommand{\thefootnote}{८}\footnote{पद~।} या १, \hyperref[85]{'व्येकम्'} इति व्यक्ताव्यक्तयोर्भेदत्वात्
अव्ययः\renewcommand{\thefootnote}{९}\footnote{व्यक्ताव्यक्तयोरद्रत्काद्र
अव्ययः~।} राशिश्चोपलक्षणार्थमृणपदम् इति न्यासः\renewcommand{\thefootnote}{१०}\footnote{न्यासः या १$+$ ऊ १$+$~।}\textendash \,या १ रू १$+$, एतौ च राशी एकराशिवत्, तेनार्धं
क्रियमाणं प्रत्येकं}
{ततो जायते (या) \begin{tabular}{|c|} १\\ २\end{tabular}  रू\renewcommand{\thefootnote}{११}\footnote{ऊ~।} \begin{tabular}{|c|}$\begin{matrix}
\mbox{{१}}\\
\mbox{{२}}
\end{matrix}+$\end{tabular}\,, चयेन\renewcommand{\thefootnote}{१२}\footnote{चयेन ३$+$~।} ३ द्वयोरपि घातात् या\begin{tabular}{c}३\\ २\end{tabular}रू\renewcommand{\thefootnote}{१३}\footnote{\begin{tabular}{|c|} ३\\ २\end{tabular}~।} \begin{tabular}{|c|} $\begin{matrix}
\mbox{{३}}\\
\mbox{{२}}
\end{matrix}+$\end{tabular}\,, आदिना रू\renewcommand{\thefootnote}{१४}\footnote{ऊ ३~।} २}
{योगे रूपेणैव रूपाणि युज्यन्त इति धनराशिशेषे या\renewcommand{\thefootnote}{१५}\footnote{या ऊ २~।}\begin{tabular}{c}३\\ २\end{tabular}रू\begin{tabular}{c} १\\ २\end{tabular}, पदेन
यावत्तावता द्वयोरपि गुणने}
{प्रथमस्थाने सदृग्वधाद्वर्गव्यपदेशः तथा चोक्तं\textendash \,{\qt 'सदृशा(द्)
द्विगतादिवत्'} इति, द्वितीयस्थाने}
{द्वयोरपि गुणने\renewcommand{\thefootnote}{१६}\footnote{गुणेन प्रथमस्थाने सदृग्वधात् वर्गव्यपदेशः~।} अव्यक्तेन व्यक्तवधादव्यक्तभावः\renewcommand{\thefootnote}{१७}\footnote{*वदाद*~।} तदप्युक्तं
{\qt 'ताभ्यां व्यक्तगुणाहत'} इति,}
{लब्धं सबीजकं सङ्कलितं वर्गः \begin{tabular}{|c|} ३\\ २\end{tabular} या \begin{tabular}{|c|} १\\ २ \end{tabular}~। एतच्चत्वारिंशता
सममिति पक्षयोर्न्यासः\textemdash}
\vspace{3mm}

 प्रथमपक्षः\textendash \,व ३ छे २ या १ छे २ रू ०, अस्वमर्थादस्ववर्णो द्वितीयः पक्षः\textendash \,व ० छे ० या ० रू ४० छे १~। द्वाभ्यां गुणिते जातौ पक्षौ, प्रथमपक्षः\textendash \,व ३ या १ रू ०, द्वितीयपक्षः\textendash \,व ० या ० रू ८०~। एवं जाते\renewcommand{\thefootnote}{१८}\footnote{प्रक्षमपक्षः व २ च्छे 
२ या २ च्छेद ऊनः प्रथमपक्षः व २ या २ ऊ . अस्वमर्थात्स्ववर्णे द्वितीयः
पक्षः व . च्छे . यां . ३५ च्छे १ वर्गाव्यक्तविशोधने द्वितीयः पक्षः वर्ग . या . उ~।} 

\newpage

\begin{quote}
    {\qt रूपाणां च कृते कार्ये पक्षयोश्चापवर्तने\renewcommand{\thefootnote}{१}\footnote{*पवर्तते~।}\,।\\
अस्वपक्षे चतुर्वर्गहतेऽव्यक्तकृतौ युते~॥\\
मूलं व्यव्यक्तमर्धोनं\renewcommand{\thefootnote}{२}\footnote{व्यक्तमथोर्धोनं~।} प्रमाणं वर्गभाजितम्~।}
\end{quote}

 {तत्र प्रथमं तावत्पक्षशोधनं\renewcommand{\thefootnote}{३}\footnote{*क्षघोनं~।} क्रियते~। कथम्\,? यतो वर्गाव्यक्तानि
शोध्यन्ते तत्पक्षस्थानि}
{रूपाणीतरपक्षवर्तिभ्यो\renewcommand{\thefootnote}{४}\footnote{*वर्तेभ्यो~।} रूपेभ्यः शोध्यन्ते इति~। तेनेह
द्वितीयपक्षगतो वर्गराशिः शून्यात्मकः}
{अव्यक्तराशिश्च \,प्रथमपक्षस्थाद्वर्गराशेरव्यक्तराशेश्च\renewcommand{\thefootnote}{५}\footnote{*राशेश्च व्यक्तराशिश्च~।} \,विशोधितः \,प्रथमपक्षे \,शेषः \,वर्गः ३ \,या १,}
{प्रथमपक्षस्थापितरूपराशिः शून्यात्मकः द्वितीयपक्षस्थाद्रूपराशेर्विशोधितः
द्वितीयपक्षस्य व्यक्तशेषः}
{रू\renewcommand{\thefootnote}{६}\footnote{ऊ ८ . चतुर्हते ३२ वर्गसंख्याया द्धृते ९६ . अव्यस्य~।} ८०, \,चतुर्हते ३२०, \,वर्गसङ्ख्यया \,हते ९६०, \,अव्यक्तस्य १ \,कृतौ १ \,युते ९६१, \,अतो \,मूलं}
{३१, अव्यक्तराशि\textendash \,१\textendash \,विरहितं ३०, अर्धोनं १५, वर्गसङ्ख्यया ३ भाजितम् ५,
एतदव्यक्तबीजस्य प्रमाणमिति लब्धं गच्छप्रमाणं रू ५~।}
\vspace{3mm}

{एतदेव \,कर्म \,सूत्रेण \,कृतं, \,तथा \,हि \,प्रथमपक्षे \,यो \,द्वयात्मकश्छेदः\renewcommand{\thefootnote}{७}\footnote{*पक्षयोद्वर्या*~।}
स्थितः \,सच्छेदसाध्ये तन्नाशाय\renewcommand{\thefootnote}{८}\footnote{तत्पाशाय~।} चत्वारिंशता गुणनं च\renewcommand{\thefootnote}{९}\footnote{गुणाचेतः~।} ततोऽनन्तरं\renewcommand{\thefootnote}{१०}\footnote{अन्तरं~।}
चतुर्हतमिति चत्वारोऽपि वर्गहता}
{इति वर्गसङ्ख्या त्रयोऽपि, तत्र द्विकचतुष्कयोर्गुणयोर्घात एको गुणः
कृतोऽष्टौ, वर्गसङ्ख्यापि}
{त्रिकमुत्तरसममेवेह सङ्कलिता भवतीति तदाख्ययैव गुणक उक्तः~। यत् तु\renewcommand{\thefootnote}{११}\footnote{त्व~।}
{\qt 'अव्यक्तकृतौ युत'} इति तत् \hyperref[87]{'द्विगुणादिप्रचयविवरकृतियुक्तात्'} इति कृतम्~। कथम्\,?
द्विघ्नः आदिः ४, प्रचयः}
{३, अनयोर्विवरं रूपं १ तदेवेहाव्यक्तप्रमाणं\renewcommand{\thefootnote}{१२}\footnote{रूपं १~। १~।तदेवेहावक्त*~।} भवति, कृतिश्च
तयोस्तुल्यैव, मूलं चोभयोरपि}
{तुल्यमेव~। (यत्तु) \hyperref[87]{'द्विगुणमुखोनं सचयम्'} इति कृतं तदिदं
{\qt 'व्यव्यक्तम्'} इति\renewcommand{\thefootnote}{१३}\footnote{यदिदं व्यक्त*~।}\,। तत्कथम्\,?}
{द्विगुणं मुखम् ४, अनेन\renewcommand{\thefootnote}{१४}\footnote{चयेन~।}  यदूनं कृतं पश्चात्प्रचयेन योजितं
तत्परमार्थतो रूपेण वियोजितं}
{भवति, अव्यक्तं चेह रूपप्रमाणमेव~। {\qt 'अर्धोनं प्रमाणं वर्गभाजितम्'} इति
यत्तदिदं कृतं}
{\hyperref[87]{'द्विचयोद्धृतं गच्छः'}~। द्वाभ्यामुद्धरणमर्धीकरणं च तुल्यार्थं\renewcommand{\thefootnote}{१५}\footnote{तुलार्थं~।},
चयेनोद्धरणं वर्गेण भजनं चेति~।}
{एतस्माद्बीजानयनकर्मक्रमप्राप्तगुणभागशोध्यक्षेपा\renewcommand{\thefootnote}{१६}\footnote{*भागगोध्यक्षेपा~।}  एव
सङ्कलितनियतभाविना रूपेण लाघवार्थं}
{निबद्धा गणितपाटीकृद्भिरित्युपपत्तिः~।}
\vspace{3mm}

{अथ प्रथमोदाहरणे पदेऽज्ञाते न्यासः\textendash}
\vspace{2mm}

\hspace{2cm}{आ २, उ ३, गच्छो न ज्ञायते, सङ्कलितम् ४०~।}
\vspace{2mm}

{कर्म\textendash \,फलतः ४० अष्टहतात्\renewcommand{\thefootnote}{१७}\footnote{अष्टहतात्
३२ .~।}  ३२०, उत्तरेण (३) हतात्\renewcommand{\thefootnote}{१८}\footnote{हतात् ८९६~।}  ९६०,
आदे\textendash \,२\textendash \,र्द्विघ्नस्य ४}
{प्रचयस्य चास्य ३ विवरं १ अस्य कृतिः १ अनया युक्तात् ९६१, मूलं ३१, मुखेन
२ द्विगुणेन}
{४ ऊनं २७, चयेन ३ सहितं ३०, द्व्युद्धृतं १५, चयेन ३ उद्धृतम् ५, एष गच्छः~।}

\newpage

{अथ द्वितीयोदाहरणेऽज्ञाते पदे न्यासः\textemdash}
\vspace{2mm}

\hspace{2cm}{आ २, उ ३, गच्छो न ज्ञायते, सङ्कलितम् \begin{tabular}{|c|} ५\\ ८\end{tabular}~।}
\vspace{3mm}

{कर्म\textendash \,फलतः \begin{tabular}{|c|}५\\ ८\end{tabular} अष्टहतात् ५, उत्तरेण~। ३~। हतात् १५,
आदे\textendash \,२\textendash \,र्द्विघ्नस्य ४ प्रचयस्य}
{चास्य ३ विवरं १ अस्य कृत्या\renewcommand{\thefootnote}{१}\footnote{कृत्वा~।}\,। १~। युक्तात् १६, मूलं ४, मुखं २
द्विगुणं ४ अनेनोनं\renewcommand{\thefootnote}{२}\footnote{ऊन~।} शून्यं ०,}
{चयेन ३ सहितं ३, द्व्युद्धृतं \begin{tabular}{|c|} ३ \\२ \end{tabular}\,, चयेन ३ उद्धृतं\begin{tabular}{c}१\\ २\end{tabular},
पदप्रमाणम्\renewcommand{\thefootnote}{३}\footnote{उद्धतं पदे प्र*~।} \begin{tabular}{|c|} १\\ २\end{tabular}~।}
\vspace{3mm}

{अथ तृतीयोदाहरणे पदेऽज्ञाते न्यासः\textemdash}
\vspace{2mm}

\hspace{2cm}{आ २, उ ५, गच्छो\renewcommand{\thefootnote}{४}\footnote{गच्छ~।} न ज्ञायते, सङ्कलितम्\renewcommand{\thefootnote}{५}\footnote{*तम् .~।} ०~।}
\vspace{3mm}

{कर्म\textendash \,अष्टोत्तरहतफलात्\renewcommand{\thefootnote}{६}\footnote{अष्टोत्तरफलहतफलात् .~।} ०, आदे\textendash \,२\textendash \,र्द्विघ्नस्य ४ प्रचयस्य ५
विवरं १ अस्य कृत्या १}
{युक्तात् १, मूलं १, मुखेन २ द्विगुणेन ४ ऊनं ३$+$, चयेन (५) सहितं २,
द्वाभ्यामुद्धृतं १, चयेन}
{५ उद्धृतम् \begin{tabular}{|c|}१ \\५\end{tabular} , एष गच्छः~।}
\vspace{3mm}

{अथ चतुर्थोदाहरणे पदेऽज्ञाते न्यासः\textendash}
\vspace{2mm}

\hspace{2cm}{आ १०, उ २$+$, गच्छो न ज्ञायते, सङ्कलितम् \begin{tabular}{|c|} ४३\\ १६\end{tabular}~।} 
\vspace{3mm}

{कर्म\renewcommand{\thefootnote}{७}\footnote{कर्मे~।}\textendash \,फलात् \begin{tabular}{|c|}४३\\ १६\end{tabular}  अष्टहतात् \begin{tabular}{|c|} ४३ \\२\end{tabular} , उत्तरेण च २$+$ हतात्
४३$+$, आदे\textendash \,१०\textendash \,र्द्विघ्नस्य २० प्रच(य)स्य चास्य २$+$ विवरम् २२ अस्य कृत्या ४८४ युक्तात्
४४१, मूलं २१,}
{मुखेन १० द्विगुणेन २० ऊनं १, सचयं १$+$, द्व्युद्धृतं\renewcommand{\thefootnote}{८}\footnote{द्वयूद्धतं\begin{tabular}{|c|} १\\ २\end{tabular}~।} \begin{tabular}{|c|}$\begin{matrix}
\mbox{{१}}\\
\mbox{{२}}
\end{matrix}+$\end{tabular} , चयेन २$+$
उद्धृतं\renewcommand{\thefootnote}{९}\footnote{चयेन २ उद्धतं~।} \begin{tabular}{|c|} १\\ ४\end{tabular} , एष}
{गच्छः~।}
\vspace{3mm}

{अस्य चतुर्थोदाहरणस्य \hyperref[86.1]{'आदिः पदहृतगणि(त)म्'} इत्यतः प्रभृत्युदाहरणेषु टीकाकृता त्रयाणामुदाहरणानामाद्याद्यानयनप्रकारे दर्शिते भूय
आद्याद्यानयनप्रकारो\renewcommand{\thefootnote}{१०}\footnote{भू
द्याद्या*~।} न दर्शित इति}
{वृत्तिग्रन्थे मूलग्रन्थे चास्योदाहरणस्य
दृष्टत्वादस्माभिर्लेखनावसर\renewcommand{\thefootnote}{११}\footnote{*र्लिखना*~।} एव तदानयनप्रकारो दर्शित इत्यलम्~।}
\vspace{3mm}

{अथास्यां श्रेढीगणितपरिपाट्यामुपयोगिधनर्णपरिकर्म श्रीश्रीधराचार्य
आर्याभिर्निबध्नाति\textendash }

\begin{quote}
    
 {\qt धनयोर्धनमृणमृणयोर्धनर्णयोरन्तरं समैक्यं खम्~।\\
 खर्णैक्यमृणं धनशून्ययोर्धनं\renewcommand{\thefootnote}{१२}\footnote{धनं शू*~।}  शून्ययोः शून्यम्~॥}\end{quote}

{द्वयोर्धनराश्योर्योगो\renewcommand{\thefootnote}{१३}\footnote{*र्योगं~।} धनं भवति~। ऋणराश्योर्योगः ऋणं भवति~। 
धनराशिऋणराश्योर्योगे कर्तव्ये अन्तरं कार्यम्~। {समैक्यमिति}
समयोर्धनराशिऋणराश्योरैक्ये कर्तव्ये {खं}
{शून्यं जायते~। तथा शून्यस्य ऋणस्य च योगे ऋणं स्यात्~। धनस्य शून्यस्य च
योगो}
{धनं, द्वयोः शून्ययोर्योगः शून्यम्~। इति सङ्कलितम्~।} 

\newpage

{व्यवकलिते आर्याद्वयम्\textendash}

\begin{quote}
    
{\qt ऊनमधिकाद्विशोध्यं धनं धनादृणमृणादधिकमूनात्~।\\
 व्यस्तं तदन्तरं खादृणं धनं धनमृणं भवति~॥ \\
 शून्यविहीनमृणमृणं धनं (धनं) भवति शून्यमाकाशात्~।\\
 शोध्यं यदा धनमृणादृणं धनाद्वा तदा क्षेप्यम्~॥}\end{quote}

 {ऊनो धनराशिरधिकाद्धनराशेः सकाशाद्विशोध्यः (तदन्तरं धनं) भवति~। एवमून}
{ऋणराशिरधिकादृणराशेः शोध्यः\renewcommand{\thefootnote}{१}\footnote{क्रणराशिनार्धकादृणराशेः शोध्यं~।} तथाधिको (ध)नराशिरूनाद्धनराशेः
शोध्यस्तदन्तरमृणं}
{स्यात्~। अधिकऋणराशिरूनादृणराशेः शोध्यस्तदन्तरं धनं स्यात्~। यदि
शून्यादृणराशिः शोध्यते}
{तदा स धनात्मकः स्यात् तथा तस्मादेव शून्याद्यदा धनराशिः\renewcommand{\thefootnote}{२}\footnote{यन*~।} शोध्यते तदा
स ऋणात्मकः}
{स्यात्~। यदि ऋणराशिः शून्येन हीनः क्रियते तदा स ऋणात्मक एव स्यात्, यदि
धनराशिः शून्येन}
{हीनः क्रियते तदा स धनमेव स्यात्~। अथ यदि शून्यं शून्यादूनीक्रियते तदा
तस्मिन् शून्यराशौ शून्यं क्षेप्यम् एव तथा यदा धनमृणाच्छोध्यमानं तदा ऋणराशौ धनराशिः
क्षेप्यः\renewcommand{\thefootnote}{३}\footnote{क्षेप्य~।}  यदा वा ऋणराशिर्धनराशेः शोध्यः स्यात्तदा तस्मिन् धनराशौ\renewcommand{\thefootnote}{४}\footnote{तस्मिन्नयनराशौ~।}  ऋणराशिः क्षेप्य
एवेति व्यवकलितम्\renewcommand{\thefootnote}{५}\footnote{व्यपक*~।}\,।}
\vspace{3mm}

{अथ प्रत्युत्पन्न आर्या\textendash}

\begin{quote}
    
{\qt ऋणमृणधनयोर्घातो धनमृणयोर्धनवधो धनं भवति~।\\
 शून्यर्णयोः खधनयोः खशून्ययोर्वा वधः शून्यम्~॥}\end{quote}

{ऋणराशिधनराश्योर्घात ऋणात्मकः स्यात्~। द्वयोः ऋणराश्योर्घातो\renewcommand{\thefootnote}{६}\footnote{*राश्यौ*~।}  धनं
स्यात् तथा}
{(द्वयोः धनराश्योर्घातः धनं स्यात्)~। शून्यऋणयोर्वधः तथा (शून्यधनयोः)
शून्यशून्ययोश्च शून्यं स्यात्~। समाप्तः प्रत्युत्पन्नः~।}
\vspace{3mm}

{अथ भागहरणे आर्या सार्धा\textendash}

\begin{quote}
    
{\qt धनभक्तं धनमृणहृतमृणं धनं (भवति) खं खभक्तं खम्~।\\
 भक्तमृणेन धनमृणं धनेन हृतमृणमृणं भवति~॥ \\
 खोद्धृतमृणं धनं वा तच्छेदः\renewcommand{\thefootnote}{७}\footnote{तछेदः~।}  खमृणधनविभक्तं खम्~।}\end{quote}

{धनराशौ धनराशिना हृते फलं धनं स्यात्, एवमृणराशौ ऋणराशिना हृते फलं}
{धनमेव स्यात्~। शून्ये शून्येन विभक्ते फलं शून्यं स्यात्~। ऋणराशिना
धनराशौ भक्ते फलमृणं स्यात्, धनराशिना ऋणराशौ हृते फलं ऋणमेव स्यात्~। ऋणराशौ धनराशौ
वा\renewcommand{\thefootnote}{८}\footnote{चा~।}}
{शून्येन ह्रियमाणे\renewcommand{\thefootnote}{९}\footnote{क्रियमाणे~।}  तयोः ऋणधनराश्योश्च्छेदः शून्यमेव, न
किञ्चित्फलमित्यर्थः~। शून्यराशौ}
{ऋणराशिना धनराशिना वा भक्ते फलं शून्यमेवेति भागहारः~।} 

\newpage

 {अथ वर्गे आर्यापरार्धम्\textendash}

\begin{quote}
{\qt धनमृणधनयोर्वर्गः खं खस्य पदं कृतिर्यत्तत्~।}
\end{quote}

 {ऋणराशेर्धनराशेश्च वर्गो धनात्मकः स्यात्~। शून्यस्य मूलं कृतिर्वर्गो
यत्तदित्यनेन घनघनमूले खं शून्यमेव स्यादिति~।}
\vspace{3mm}

 {अथ यत्र पदसङ्कलिते ज्ञायेते\renewcommand{\thefootnote}{१}\footnote{ज्ञायते~।} नादिर्नोत्तरं न चादिरुत्तरं चेष्टतो
व्यवस्थापितं}
{तदितरानयनम् आद्यानयनसूत्रेण
वाशक्यमाद्युत्तरमिश्रदर्शनपारतन्त्र्यात्तत्र\renewcommand{\thefootnote}{२}\footnote{शक्यमाद्यतर*~।}  तयोरानयनार्थं करणसूत्रमार्यामाह\textendash}

\phantomsection \label{88}
\begin{quote}
    
{\bs  विपदपदवर्गदलाहतमिश्रधनात्फलमपास्य परिशिष्टम्~।\\
 व्येकपदार्धेन भजेत् व्येकेन पदाहतेनादिः~॥~८८~॥}\end{quote}

 {पदफलयोरिहोपकरणभावाश्रयेणाद्युत्तरयुतिरेव\renewcommand{\thefootnote}{३}\footnote{*परकरणभावश्रयेणाद्युतितरारेवं~।}  मिश्रधनम्~।} {तस्मान्मिश्रधनाद्विपदस्य}
{सङ्कलितगच्छविरहितस्य पदवर्गस्य सङ्कलितगच्छकृतेर्दलेनार्धेन हतात्
गुणितात्फलं}
{सङ्कलितमपास्य\renewcommand{\thefootnote}{४}\footnote{*मयास्य~।}  विशोध्य परिशिष्टं भजेत् छिन्द्यात्, केनेत्युच्यते~। 
व्येकस्य रूपोनस्य पदस्य}
{सङ्कलितगच्छस्यार्धेन व्येकेन पदाहतेनेत्येष भागहारः, एतदाप्तमादिः
अर्थात्प्रतिपादितं}
{भवति, यदुत आदिवियुक्तं मिश्रं प्रचय इति~।}
\vspace{3mm}

{इदमपि सूत्रं बीजोपजीवनेन स्थितं, तथा
चेहादेरज्ञातत्वात् यावत्तावत् सञ्ज्ञात्वं, तेन तस्य}
{स्थापनम्\textendash \,आ \,या \,१~। उत्तरस्यापि \,(अ)ज्ञातत्वे \,यदि \,तुल्यतादिनियमाभावात् अव्यक्तान्तरत्वे\renewcommand{\thefootnote}{५}\footnote{तुल्यतादनि*~।}}
{कालकादिवर्णान्तरत्वे प्राप्तेऽपि आदिविशुद्धमिश्रशेषमप्युत्तरमतो\renewcommand{\thefootnote}{६}\footnote{*म....राया~।} 
यावत्तावत्कमिश्ररूपं}
{तत्स्थाप्यते यथा उ\renewcommand{\thefootnote}{७}\footnote{३ ऊ ५ या १$+$~।}  ५ या १$+$, गच्छो ज्ञायते ५, सङ्कलितम् ४०~।}
\vspace{3mm}

{एवं स्थिते अव्यक्तराशी एव व्यक्तवत्प्रकल्प्य तयोरानयनं\renewcommand{\thefootnote}{८}\footnote{ततोनानयनं~।}  क्रियते~। 
तेनाव्यक्तसङ्कलितधनेन व्यक्तं चत्वारिंशत्परिमाणं रूपात्मकत्वाद्व्यक्तं सङ्कलितं
स्पर्धयित्वा समीकरणाख्येन प्रथमबीजेनादिरानीयते ततश्चार्थतस्तु
(आदि)विशुद्ध(मिश्र)मुत्तरप्रमाणं लभ्यत इति~।}
\vspace{3mm}

{अव्यक्तपक्षे सङ्कलितानयनार्थो न्यासः\textemdash}
\vspace{2mm}

\hspace{10mm} {आ या १, उ\renewcommand{\thefootnote}{९}\footnote{३ ऊ ५ या १$+$~।} ५ या १$+$, गच्छः ५, सङ्कलितं च\renewcommand{\thefootnote}{१०}\footnote{न~।}  ज्ञायते (४०)~।}
\vspace{3mm}

{अत्र कर्म\textendash \,व्येकस्य\renewcommand{\thefootnote}{११}\footnote{कर्मण्येकस्य~।}  पदस्यार्धं २, चयेन (रू ५ या १$+$ गुणितं रू १०
या २$+$, आदिना या १ सहितं रू १० या १$+$, पदेन ५ सङ्गुणितं) रू\renewcommand{\thefootnote}{१२}\footnote{ऊ ५ .~।}  ५० या ५$+$~। 
एतच्चत्वारिंशता सममिति पक्षयोर्न्यासः\textendash}
\vspace{2mm}

\hspace{1cm}{या\renewcommand{\thefootnote}{१३}\footnote{या ५$+$३४ .} ५$+$ रू ५० एकः पक्षः; या ० रू ४० द्वितीयः\renewcommand{\thefootnote}{१४}\footnote{या . ऊ ४ दितीयः~। या २$+$ ऊ १~।} पक्षः~।}

\newpage

(पञ्चभिः) पक्षावपवर्त्य स्थापनम्\textendash 
\vspace{2mm}

\hspace{2cm}या १$+$ रू १०, (या ० रू ८) 
\vspace{3mm}

\noindent{एवं जाते}

\begin{quote}
    
{\qt 'संशोध्याव्यक्तमेकस्मात्\renewcommand{\thefootnote}{१}\footnote{*कस्मात् ऊ ४~।} पक्षाद्रूपाणि चान्यतः~।\\
 रूपशिष्टप्रमाणं स्याच्छिष्टाव्यक्तस्य तत्\renewcommand{\thefootnote}{२}\footnote{*च्छित्राव्यक्तस्य तं~।}  फलम्~॥'} {इति~।}\end{quote}

 {पक्षशोधनं तावत् क्रियते, तद्यथा\textendash \,प्रथमपक्षस्थिताद्रूपदशकात्
विशोध्यन्ते(ऽष्टौ)}
{तेन तत्र रूपद्वयं शेषं जायते~। ततश्च शुद्धपक्षयोर्न्यासः\textendash \,प्रशे\renewcommand{\thefootnote}{३}\footnote{प्रशे ऊ १~।} 
रू २ द्विशे या १~। अत्राव्यक्तशेषेण भक्तं रूपशेषमादेः प्रमाणम् (२)~। इदानीं यावत्तावदात्मकत्वं
निवृत्तं रूपात्मकत्वं}
{जातम्~। ज्ञातमेतद्यथा आदिः २~। एतस्मिन्मिश्रात्पञ्चभ्यो विशुद्धे शेषं
त्रय इति ज्ञातम् उ ३\renewcommand{\thefootnote}{४}\footnote{ज्ञातं ३३~।}\,।}
\vspace{3mm}

{कथमेतदपि सूत्रेण निबद्धमित्युच्यते~। अव्यक्तादुत्तरपक्षे
तत्पक्षसङ्कलितानयनं यावत्तावदादिना\renewcommand{\thefootnote}{५}\footnote{यावत्तावदिदं~।}  कृतं \hyperref[88]{'विपदपदवर्गदलाहतमिश्रधनात्'} इति\renewcommand{\thefootnote}{६}\footnote{*मिथोध*~।}  राशिद्वयमुत्पद्यते
तत्र धनगतानि रूपाणि\renewcommand{\thefootnote}{७}\footnote{पञ्च रूपाणि~।}}
{पञ्चाशत् विपदपदवर्गदलाहतमिश्रधनं भवति\renewcommand{\thefootnote}{८}\footnote{विपदं पदं व*~।}, तस्मात् \hyperref[88]{'फलमपास्ये'}ति इदं
तदनुगतं रूपाणि}
{चान्यतः \,संशोध्येति, \,ततश्च {\qt 'रूपशिष्टप्रमाणं \,स्याच्छिष्टाव्यक्तस्य \,तत् फलम्'} \,इति\renewcommand{\thefootnote}{९}\footnote{तं फल*~।} यत् \,तदिदं}
{निबद्धं \hyperref[88]{'व्येकपदार्धेन \,भजेत् \,व्येकेन \,पदाहतेने'}ति, \,यो \,हि \,तत्राव्यक्तपक्षे \,ऋणगताः \,यावत्तावत्काः \,पञ्च ते पक्षशोधनक्रियायां\renewcommand{\thefootnote}{१०}\footnote{पक्षसोधनं क्रमायां~।} पक्षान्तरस्थशून्यात्मकयावत्तावद्राशेः
शोध्यमाना धनात्मकतामासाद्य रूपशेषस्य भागहारो जायते~। इयांस्तु विशेषः बैजिकेन
यथादिप्रमाणानयने गौरवप्रसङ्गादिति~।}
\vspace{3mm}

{अथोदाहरणम्~। अत्र पूर्वोक्तप्रथमोदाहरणस्याद्युत्तरयोर्विभागाज्ञाने
मिश्रे ज्ञाते पदसङ्कलितयोश्च ज्ञातयोर्न्यासः\textemdash}
\vspace{2mm}

\hspace{2cm}आ\renewcommand{\thefootnote}{११}\footnote{आ ३\textemdash~।}  ०, मिश्रं ५, गच्छः ५, सङ्कलितम्\renewcommand{\thefootnote}{१२}\footnote{*त ४.~।}  ४०~। \vspace{3mm}

{अत्र कर्म\textendash \,पदस्य ५ वर्गः २५, विपदः २०, अर्धं १०, मिश्रेण ५ गुणितं ५०,}
{सङ्कलितेन ४० हीनं १०, व्येकस्य पदस्यार्धेन २ व्येकेन १ पदगुणितेन ५
भजेदिति लब्ध}
{आदिः २, एतद्विशुद्ध(मिश्र)प्रमाणः ५ प्रचयः (३) उत्तरः~।}
\vspace{3mm}

{अथ राशिगतयुतौ आद्युत्तरपदेषु ज्ञातेषु सङ्कलितानयनार्थं
करणसूत्रमार्यामाह\textendash}

\phantomsection \label{89}
\begin{quote}
    
{\bs निर्विकलपदघ्नचयः सादिरनष्टो मुखान्वितो विचयः~।\\
 निर्विकलपदार्धहतो\renewcommand{\thefootnote}{१३}\footnote{*पदावहतो~।}  विकलघ्नानष्टयुग्गणितम्\renewcommand{\thefootnote}{१४}\footnote{*लम्ना*~।}\,॥~८९~॥}\end{quote}

\newpage

{सङ्कलितक्रमेण क्षेत्रफलक्रमेण च श्रेढ्याश्रितमेव गतमिति पुनः
प्रारम्भसामर्थ्यात्}
{राशि(ग)तम् एतदिति विज्ञायते, तत्रापि (नि)र्विकलपदे
क्षेत्रराशिगणितयोरभेदाद्विकलाश्रयमेव}
{सूत्रमारभ्यते~। एवमुत्तरसूत्रेष्वपि वाच्यम्~। रूपभागस्य विकलसञ्ज्ञा,
सकलत्वाद्रूपस्य~। तत्र}
{विकलमेव\renewcommand{\thefootnote}{१}\footnote{*मेवा~।} पदं स्याद्विकलान्वितमेव वा~। विकलपदे रूपस्थाने शून्यं
कल्प्यं सविकलापत्त्यर्थं}
{विकलपदाश्रयेणैव सूत्रस्य वृत्तत्वात्, तथा च निर्विकलं विकलविहीनं कृतं
यत्पदं तेन गुणितो}
{यश्चयः\renewcommand{\thefootnote}{२}\footnote{यच्चयः~।}  स आदिना सहितो द्वितीयस्थाने स्थाप्यः, स चानष्टसञ्ज्ञ
उत्तरकर्मार्थः, यस्त्वन्यस्थः\renewcommand{\thefootnote}{३}\footnote{*स्य~।}}
{(स) मुखेनान्वितः चयेन रहितो निर्विकलपदस्यार्धेन हतो
विकलहतेनानष्टसञ्ज्ञकेन राशिना}
{युक्तो राशियुतो गणितो सङ्कलितं भवति~।}
\vspace{3mm}

 {उदाहरणम्\textendash}

\begin{quote}
    
{\eg एको लभते त्रीणि द्विरूपवृद्ध्या\renewcommand{\thefootnote}{४}\footnote{*वृद्धा~।}  ततोऽपरे पुरुषाः~।\\
 इत्यर्धपञ्चमनराः कियल्लभन्ते समाचक्ष्व\renewcommand{\thefootnote}{५}\footnote{समानलक्ष्य~।}\,॥~१०६~॥}\end{quote}

 {केचित्पञ्च पुरुषाः कस्यचित्कार्यार्थिनः भृत्यकर्मणि\renewcommand{\thefootnote}{६}\footnote{भृत्याक*~।}  प्रवृत्ताः,
येषामेकस्त्रीणि रूपाणि}
{लभते, द्वितीयस्ततोऽधिकसामर्थ्यः पञ्च लभते, तृतीयोऽपि प्रथमा(त्) द्वितीय
इव द्वितीयाधिकसामर्थ्यः सप्त लभते, एवम् उक्तयोरनन्तराधिकसामर्थ्यादन्ये
द्विद्विरूपवृद्ध्या भृतिं लभन्ते}
{यावच्चत्वारः~। पञ्चमस्तु यावत्प्राप्तभृतिदलभागी\renewcommand{\thefootnote}{७}\footnote{*दलाभा*~।} तावत एव
सामर्थ्यादित्यर्धं तु सङ्ख्यायते}
{यदि वा सोऽपि चतुर्थात् तद्वदधिकसामर्थ्य एव नियमितकालार्धकर्मकरणात्तु
सङ्ख्यायते~। एवं,}
{स्थिते तेषामेवार्धपञ्चमानां नराणां भृतिसङ्कलितं किं भवतीति~।}
\vspace{2mm}

{न्यासः\textendash \,आ ३, च २, गच्छः\renewcommand{\thefootnote}{८}\footnote{गच्छ~।} \begin{tabular}{c|}४\\ १\\ २\\\hline \end{tabular} , सङ्कलितं न ज्ञायते~।}
\vspace{2mm}

{अत्र कर्म\textendash \,पदं\begin{tabular}{c|}४\\ १\\ २\\\hline \end{tabular}\,, विकलेन\begin{tabular}{c}१ \\२\end{tabular}विहीनं ४, एतेन चयः २ गुणितः ८, आदिना
३ सहितः ११, एषोऽनष्टसञ्ज्ञाङ्कितश्च कल्प्यः, यथा न्यासः\textendash \,अनष्टः ११, स तु मुखेन
अन्वितः १४,}
{चयेन २ विरहितः १२, निर्विकलपदस्य ४ अर्धेन २ हतः २४, विकलेन\begin{tabular}{c|}१ \\२\\\hline \end{tabular}
गुणितोऽनष्टसञ्ज्ञको राशिः ११ जातः\begin{tabular}{c}११\\ २\end{tabular}, एतेन २४ संयुतः\renewcommand{\thefootnote}{९}\footnote{एते संयुतः~।}\begin{tabular}{c}५९\\ २\end{tabular}, एतत्
गणितमिति\renewcommand{\thefootnote}{१०}\footnote{गुणित*~।}\,।}
\vspace{2mm}

{तथा चतुर्णां तावत्\renewcommand{\thefootnote}{११}\footnote{तावक्त~।}  पदं ४, व्येकं पदं ३, अर्धं \begin{tabular}{c|}३ \\२\\\hline \end{tabular}\,, चयेन\renewcommand{\thefootnote}{१२}\footnote{क्षयेन~।} 
२ हतं ३, सादिः\renewcommand{\thefootnote}{१३}\footnote{आदिः~।}  ६, पदेन ४}
{गुणितं २४~। पञ्चमस्य सकलस्य धनं ११, तथा च पदं ५, एकहीनं ४,
उत्तर\textendash \,(२)\textendash \,गुणितं (८),}
{संयुक्तमादिना ११, एतदन्त्यस्य पञ्चमस्य धनम्~। तदर्धं \begin{tabular}{c|}११ \\२\\\hline \end{tabular} एतदेवानष्टसञ्ज्ञकस्य राशेः}
{विकलहतस्य प्रमाणं भवति~। एतेन युक्तं चतुर्णां सङ्कलितं २४ इदं \begin{tabular}{c|}५९\\ २\\\hline \end{tabular} 
भवति~।}

\newpage

{अनेन सूत्रेण \hyperref[89]{'निर्विकलपदघ्नचयः सादिरनष्ट'} इति अन्त्यधनमानीय
\hyperref[89]{'विकलघ्नानष्टे'}ति}
{तदर्धभागिधनं साधितं, शेषेण सूत्रेण चतुर्णां सङ्कलितम्~। तत्कथमिति
चेदुच्यते~। अन्त्यधनानयने}
{तावत् {\qt 'पदमेकहीनम्'} इति कर्तव्ये निर्विकलं पदं कृतम्, {\qt 'उत्तरगुणितम्'} इति तु
\hyperref[89]{'पदघ्नचय'}\renewcommand{\thefootnote}{१}\footnote{तुल्यं चय~।}  इति}
{{\qt 'संयुक्तमादिने'}ति च \hyperref[89]{'सादिरि'}ति च न भिन्नार्थम्~। एवमेतदन्त्यधनमानीय
न्यस्तम्~। \hyperref[89]{'मुखान्वितो विचय'} इति चतुर्णां\renewcommand{\thefootnote}{२}\footnote{चतुर्णा~।}  सङ्कलितमुपक्रान्तं, पञ्चमधनादेकादशभ्यो हि
एकगुणचयहीनं\renewcommand{\thefootnote}{३}\footnote{*णमादिहीनं~।}  चतुर्थस्य}
{धनं भवतीत्यस्य \hyperref[89]{'विचय'} इति कृतं, निर्विकलपदार्धेन गुणं मुखान्वितचतुर्थम्
इत्यनेन सङ्कलितार्थं\renewcommand{\thefootnote}{४}\footnote{निर्विकलपदार्धेन गुणयिष्यमाणस्य चतुर्थ इत्यनेनांत्यधनार्धं~।} {\qt 'मध्यधनं पदगुणं\renewcommand{\thefootnote}{५}\footnote{पदगुण~।}   गणितम्'} इति निर्वाहितम्, इदानीमत्र
सङ्कलिते पञ्च(म)धनमनुपात्य \hyperref[89]{'विकलघ्नानष्टयुग्गणितम्'} इति\renewcommand{\thefootnote}{६}\footnote{निर्विकलघ्ना*~।} (कृतम्)~।}
\vspace{3mm}

 {अथ द्वितीयोदाहरणम्\textendash}

\begin{quote}
    
{\eg मासि प्रथमेऽध्यर्धं\renewcommand{\thefootnote}{७}\footnote{*ध्यंर्घ~।}  त्रिभागवृद्ध्या ततो(ऽन्य)मासेषु~।\\
 यदि कर्मकरो लभते तत्किं मासत्रये सार्धे\renewcommand{\thefootnote}{८}\footnote{मासत्रयः सार्धः~।}\,॥~१०७~॥}\end{quote}

{यो भृतकः प्रथमे मासि सार्धं रूपं लभते द्वितीयादिमासेषु
रूपत्रिभागवृद्ध्या,}
{स कस्यचित् मासत्रयं पक्षमेकं च कर्म कृत्वा प्रथममासात्प्रभृति सङ्कलितानि
भृतिरूपाणि}
{कियन्ति लभते इति~।}
\vspace{2mm}

{न्यासः\renewcommand{\thefootnote}{९}\footnote{न्यासः आ \begin{tabular}{c|}११\\ २\\\hline \end{tabular} \begin{tabular}{c|}३१\\ ३\\\hline \end{tabular}  ग
\begin{tabular}{c|}३ \\१\\ २\\\hline \end{tabular}~।}\textendash \,आ\begin{tabular}{c}१\\ १\\ २\end{tabular}, च\begin{tabular}{c}१ \\३\end{tabular}, ग\begin{tabular}{c}३\\ १\\ २\end{tabular}, सङ्कलितं न ज्ञायते~।}
\vspace{2mm}

{तदर्थं कर्म\textendash \,पदं \begin{tabular}{c|}३\\ १\\ २\\\hline \end{tabular}\,, निर्विकलं ३, चयेन \begin{tabular}{c|}१ \\३\\\hline \end{tabular} हतं १, आदिना \begin{tabular}{c|}३\\ २\\\hline \end{tabular}  सहितं\renewcommand{\thefootnote}{१०}\footnote{सहितं\begin{tabular}{c}७\\ २\end{tabular}।} \begin{tabular}{c|}५ \\२\\\hline \end{tabular}\,, एषः अनष्टः,}
{एष मुखान्वितः ४, विचयः\begin{tabular}{c|}११\\ ३\\\hline \end{tabular} , निर्विकलपदार्धे(न)\begin{tabular}{c}३\\ २\end{tabular}हतः\renewcommand{\thefootnote}{११}\footnote{हतं~।}\begin{tabular}{c}११\\ २\end{tabular}, विकलघ्नोऽनष्टो}
{जातः \begin{tabular}{c|}५ \\४\\\hline \end{tabular} एतेन युतः \begin{tabular}{c|}२७ \\४\\\hline \end{tabular}\,, लब्धं सङ्कलितम्~।}
\vspace{4mm}

{अथात्रैव राशिगते (उ)त्तरपदसङ्कलितेषु ज्ञातेषु आदेरज्ञातस्यानयनार्थं
करणसूत्रमार्यामाह\textendash}

\phantomsection \label{90}
\begin{quote}
    
{\bs निर्विकलपदात् व्येकात् दलं सविकलं चयेन सङ्गुणयेत्~।\\
 विकलविहीनपदेन च तदूनधनं\renewcommand{\thefootnote}{१२}\footnote {उदूधनं~।}  पदहृतं\renewcommand{\thefootnote}{१३}\footnote{पदस्यतं~।}  प्रभवः~॥~९०~॥}\end{quote}

{पदाद्विकलहीनाद्विरूपात् अर्धं विकलसहितं चयेन तथा निर्विकलपदेन च
गुणयेत् ,}

\newpage

\noindent{तद्धनादपास्य शेषं\renewcommand{\thefootnote}{१}\footnote{शेष~।} पदेन भक्तं सदादिर्भवति~। इहापि \hyperref[90]{'निर्विकलपदात् व्येकात् दलं सविकलं चयेन सङ्गुणयेत् विकलविहीनपदेन चे'}त्यन्तेन प्राचयिकधनसङ्कलना कृता, तथा
द्वितीयपदात्प्रभृति चयः प्रवृत्त इति \hyperref[90]{'व्येकादि'}(ति) कृतम्~। चयेन सङ्गुणितस्यार्धीकरणेन
तुल्यमर्धीकृतस्य चयेन}
{गुणनमित्यादावेवार्धीकरणं\renewcommand{\thefootnote}{२}\footnote {*मित्याद्यावर्धी*~।}, \hyperref[90]{'सविकलम्'} इति विकले च भागप्राप्त्यर्थं,
\hyperref[90]{'विकलविहीनपदेन'} इत्यतश्च\renewcommand{\thefootnote}{३}\footnote{*पदोन मित्य*~।}  \hyperref[90]{'तदूनधनम्'} इत्यनेनार्धपञ्चमे\renewcommand{\thefootnote}{४}\footnote {तद्रून~। धन*~।}  पदे प्रत्येकं
परिसमाप्यादिसङ्कलितमवशेषयति\renewcommand{\thefootnote}{५}\footnote {परिसमाप्तमादिं कलित्तम*~।}, \hyperref[90]{'पदहृतम्'} इत्यनेनादिलाभः~।}
\vspace{3mm}

{आदावज्ञाते पूर्वोक्तप्रथमोदाहरणे न्यासः\textemdash}
\vspace{2mm}

 \hspace{2cm}{आ\renewcommand{\thefootnote}{६}\footnote {आ ३२ गच्छः \begin{tabular}{c|}४ \\१\\ २\\\hline \end{tabular} फलं \begin{tabular}{c|} २९\\ १\\ २\\\hline \end{tabular}~।}  ०, उ २, गच्छः\begin{tabular}{c|}४ \\१ \\२\\\hline \end{tabular} , फलं\begin{tabular}{c|}२९\\ १\\ २\\\hline \end{tabular}}~। 
\vspace{3mm}

{अत्र कर्म\textendash \,पदात् \begin{tabular}{c|}४\\ १\\ २\\\hline \end{tabular}\,, निर्विकलात्\renewcommand{\thefootnote}{७}\footnote {पदात् ४ विर्विक*~।}  (४), व्येकात् ३, दलं \begin{tabular}{c|}३\\ २\\\hline
\end{tabular}\,, सविकलं २, चयेन २}
{सङ्गुणयेत् ४, तथा निर्विकलपदेन च ४ गुणयेत् १६, तदू(न)धनं \begin{tabular}{c|}१३\\ १\\ २\\\hline
\end{tabular}\,, पदेन हृतम् ३,}
{एष आदिः~।}
\vspace{3mm}

{अथ द्वितीयोदाहरणे\renewcommand{\thefootnote}{८}\footnote {*हरण~।}  न्यासः\textendash}
\vspace{2mm}

\hspace{2cm}{आ ०, च \begin{tabular}{c|}१\\३ \\\hline \end{tabular}\,, पदं \begin{tabular}{c|}७\\ २\\\hline \end{tabular}\,, सङ्कलितम्\begin{tabular}{c|}२७\\ ४\end{tabular}~।\renewcommand{\thefootnote}{९}\footnote {आ ३ \begin{tabular}{c|}१\\ ३\\\hline  \end{tabular} सं\begin{tabular}{c}२७ \\ ४\end{tabular}।}}
\vspace{2mm}

{निर्विकलपदात् ३, व्येकात्\renewcommand{\thefootnote}{१०}\footnote {व्ये १ कात्~।}  २, दलं (१), सविकलं \begin{tabular}{c|}३\\ २\\\hline \end{tabular}\,, चयेन \bigg(\begin{tabular}{c}१\\ ३\end{tabular}\bigg)
सङ्गुणयेत्\begin{tabular}{c}१\\ २\end{tabular}, विकलविहीनपदेन (३) च\renewcommand{\thefootnote}{११}\footnote {च १~।}  सङ्गुणयेत् \begin{tabular}{c|}३\\ २\\\hline \end{tabular}\,,
तदू(न)धनं\begin{tabular}{c}२१\\ ४\end{tabular}, पदहृतम्\begin{tabular}{c}३ \\२\end{tabular}, एष आदिः~।}
\vspace{3mm}

{अथादिपदसङ्कलितेषु ज्ञातेषूत्तरस्याज्ञातस्यानयनार्थं
करणसूत्रमार्यामाह\textendash}

\phantomsection \label{91}
\begin{quote}
    
{\bs मुखपदवधेन हीनं धनं भजेद्रूपरहितगच्छस्य~।\\
 सङ्कलितस्वेनैकादिचयेनावाप्यते\renewcommand{\thefootnote}{१२}\footnote {*लित स्वेनैकादियेनावाप्यते~।}  वृद्धिः~॥~९१~॥}\end{quote}
 
{मुखस्य पदस्य च वधेन हीनं सङ्कलितं भजेत्, केनेति भाजकं\renewcommand{\thefootnote}{१३}\footnote {भाजनं~।}  साधयति,
गच्छस्य}
{रूपरहितस्य पदत्वेन स्थापितस्यैकाद्युत्तरेण यत् सङ्कलितं सोऽस्य भागहारः,
ततो भागहारादाप्तफलं प्रचयो भवति~।}
\vspace{3mm}

{पूर्वोक्तप्रथमोदाहरणे उत्तरेऽज्ञाते न्यासः\textendash \,
\vspace{2mm}

\hspace{2cm}{आ ३, उ ०, प\begin{tabular}{c}९ \\२ \end{tabular}, सङ्कलितम्\begin{tabular}{c}५९\\ २\end{tabular}।\renewcommand{\thefootnote}{१४}\footnote {आ . ३३. ५ \begin{tabular}{c|} ९\\ २\\\hline \end{tabular}~। सङ्क\begin{tabular}{c}५९\\ ३\end{tabular}।}}

\newpage

{मुखस्य\renewcommand{\thefootnote}{१}\footnote{मुख्यस्य~।} ३ पदस्य\begin{tabular}{c}९\\ २\end{tabular}च वधः\begin{tabular}{c}२७\\ २\end{tabular}, एतेन धनं\renewcommand{\thefootnote}{२}\footnote{*धन~।} \bigg(\begin{tabular}{c}५९\\ २\end{tabular}\bigg) हीनम् १६,
अस्य भागहारः}
{तद्यथा\textendash \,रूपरहितो गच्छः\begin{tabular}{c}७ \\२\end{tabular}, अस्मात्पदत्वेन स्थापितादेकादिचयेन
सङ्कलितार्थे न्यासः\textendash}
\vspace{2mm}
    
\hspace{1cm} {आ १, उ\renewcommand{\thefootnote}{३}\footnote{३१~।} १, गच्छः\begin{tabular}{c}७ \\२\end{tabular}, सङ्कलितं न ज्ञायते~।}
\vspace{3mm}

{तत्र कर्म\textendash \,पदं\begin{tabular}{c}७\\ २\end{tabular}, निर्विकलं ३, चय\textendash \,१\textendash \,गुणम् ३, आदिना १ युतम् ४,
एषोऽनष्टसञ्ज्ञः,}
{मुखान्वितः ५, विचयः ४, निर्विकलपदार्धेन\renewcommand{\thefootnote}{४}\footnote{*र्थेन \begin{tabular}{c|}३ \\२\\\hline \end{tabular}~।}\,\begin{tabular}{c|}३ \\२\\\hline \end{tabular} हतः ६, विकलघ्नानष्टेन
२ युतः ८,}
{एतत् सङ्कलितम्~। एष भाग(हारः) स्यात्\renewcommand{\thefootnote}{५}\footnote{स्यत्~।} षोडशानां, ततो लभ्यते (२), एष
उत्तरः~।}
\vspace{3mm}

{द्वितीयोदाहरणे न्यासः\textendash}
\vspace{2mm}

\hspace{1cm}{आ\begin{tabular}{c|}३ \\२ \end{tabular}\,, उत्तरो न ज्ञायते, गच्छः\begin{tabular}{c|}७ \\२\end{tabular}\,, सङ्कलितम्\begin{tabular}{c|}२७ \\४ \end{tabular}~।}
\vspace{3mm}

{कर्म\textendash \,मुखपदवधः\begin{tabular}{c}२१\\ ४\end{tabular}, एते(न) धनं \begin{tabular}{|c|}२७\\ ४\\\hline \end{tabular} 
हीनम्\begin{tabular}{c}३ \\२\end{tabular}, अस्य भागहारोऽयं तद्यथा\textendash \,आ १, उ\renewcommand{\thefootnote}{६}\footnote{३१~।} १, रूपरहितो गच्छः \begin{tabular}{c|}५ \\२\\\hline \end{tabular}\,, सड्कलितं न\renewcommand{\thefootnote}{७}\footnote{संकलितेन~।} ज्ञायते,
तदर्थं कर्म\textendash \,पदं\begin{tabular}{c}५ \\२\end{tabular}, निर्विकलं २,}
{चयघ्नं २, सादिः\renewcommand{\thefootnote}{८}\footnote{सादि~।} ३, अनष्टाख्यः, मुखान्वितः ४, विचयः ३,
निर्विकलपदार्धेन १ हतः ३,}
{विकलघ्नानष्टेन\begin{tabular}{c}३\\ २\end{tabular}युतः\begin{tabular}{c}९ \\२\end{tabular}, एतद्भागादाप्तं प्रचयः \begin{tabular}{|c|} १\\ ३\\\hline \end{tabular}}~। 
\vspace{3mm}

{अथ वासना\textendash \,\hyperref[91]{'मुखपदवधेन हीनं धनम्'} इत्यनेन प्रचयसङ्कलितं पृथक् कृतम्\renewcommand{\thefootnote}{९}\footnote{पृथक् पृथक्तुल्य~।},
प्रथमपदं}
{वर्जयित्वा द्वितीयपदात्प्रभृति चयः प्रवर्तित इति रूपरहितं पदं कृतं
तदेकाद्युत्तरेण सङ्कलितम्,}
{तत्\renewcommand{\thefootnote}{१०}\footnote{उत्तरः~।} प्रचयसङ्कलिताद्यावत्कृत्वः पतति तत्सम्मितेनेहोत्तरेण
भाव्यमिति लाघवार्थं भाज्यभाजकवृत्तिमाश्रितः\renewcommand{\thefootnote}{११}\footnote{लाघवार्थं
भाज्यो भाजक वृत्तमा*~।}\,।}
\vspace{3mm}

{अथ मुखप्रचयसङ्कलितेषु (ज्ञातेषु) अज्ञातपदानयनार्थं
करणसूत्रमार्याद्वयम्\textendash}

\phantomsection \label{92}
\begin{quote}
    
{\bs द्विचयघ्नधनात् चयदलरहितादेः\renewcommand{\thefootnote}{१२}\footnote{द्विचयघ्नधनाश्रयदल*~।} कृतियुतात् समासन्नम्~।\\
 मूलं प्राङ्मूलोनं चयहृतमविकलमनष्टाख्यम्~॥~९२~॥ \\
 व्येकं\renewcommand{\thefootnote}{१३}\footnote{एक~।} चयार्धगुणितं समुखमनष्टाहतं क्षयो गणिते\renewcommand{\thefootnote}{१४}\footnote{*नष्टहतंक्षयो गुणिते~।}\,।\\
 तदनष्टघ्नचयान्वितमुखभक्तमनष्टयुग्गच्छः\renewcommand{\thefootnote}{१५}\footnote{*तदनष्टघ्नेच*~।}\,॥~९३~॥}\end{quote}

{द्वाभ्यां चयेन च गुणिताद्धनाच्चयस्य दलेन रहितो य आदिस्तद्वर्गेण
संयुतात्समासन्नं}
{वर्गमूलं, चयदलरहितादिविरहितं चयेन हृतं\renewcommand{\thefootnote}{१६}\footnote{स्वतं~।} विकलेन रहितमनष्टसञ्ज्ञम्,
तत् रूपोनं}
{चयस्यार्धेन गुणितमादिसहितमनष्टसञ्ज्ञकेन राशिना गुणितं, तेन चोनं गणितं,}

\newpage

\noindent{तदनष्टसञ्ज्ञकराशिगुणितोत्तरोपेतमुखहृतम्\renewcommand{\thefootnote}{१}\footnote{*मुखस्य~।}, अ(न)ष्टसहितं, पदं भवति~।}
\vspace{3mm}

 {उदाहरणम्~। पूर्वोक्तप्रथमोदाहरणेऽज्ञाते\renewcommand{\thefootnote}{२}\footnote{*रणेण*~।} गच्छे न्यासः\textendash \,
\vspace{2mm}

 \hspace{2cm}{आ ३, उ २, ग ०, सङ्कलितधनं\begin{tabular}{c|}५९ \\२\\\hline \end{tabular}~।\renewcommand{\thefootnote}{३}\footnote{आ . ३३२ ग . संकलितधनं \begin{tabular}{c|}५९\\ २\\\hline \end{tabular}~।}}
\vspace{3mm}

{धनात् \begin{tabular}{|c|} ५९\\ २\\\hline \end{tabular}\,, द्विघ्नात्\renewcommand{\thefootnote}{४}\footnote{द्वित्रात् \begin{tabular}{c|}५९\\ १२२\\\hline \end{tabular}~।} ५९, (चयघ्नात् ११८,
चयदलरहितादिकृति\textendash \,४\textendash \,युतात् १२२),}
{समा-सन्नं वर्गमूलं ११, पूर्वस्याः कृतेः ४ मूलम् २, अनेनोनं\renewcommand{\thefootnote}{५}\footnote{पनेनोनं \begin{tabular}{c|}१ \\२\\\hline \end{tabular}~।} ९,
चय\textendash \,(२)\textendash \,हृतम्\begin{tabular}{c}९ \\२\end{tabular}, अविकलं}
{(४, अनष्टाख्यम्, एतत् ४ व्येकं ३, चयार्धेन १ गुणितं ३, समुखम् ६)\renewcommand{\thefootnote}{६}\footnote{स्वच्छेदस्तेन\begin{tabular}{c}१ \\२\end{tabular}रहितं २४ क्षयो मणिते\begin{tabular}{c}५९\\ २ \end{tabular}जातं\begin{tabular}{c}११\\ २\end{tabular}एतदनघ्नचयेन ८ अन्वितस्य स्यादि ९~।}},
अनष्टसञ्ज्ञकेन राशिना}
{४ गुणितं २४, क्षयो गणिते \begin{tabular}{c|}५९\\ २\\\hline \end{tabular} जातम् \begin{tabular}{c|}११\\ २\\\hline \end{tabular}\,, एतदनष्टघ्नचयेन ८ अन्वितस्य
मुखस्य ११}
{भाज्यम्, अतो लब्धम् \begin{tabular}{c|}१ \\२\\\hline \end{tabular}\,, अनष्टेन ४ युतं\begin{tabular}{c|}४ \\१\\ २\\\hline \end{tabular}\,, सवर्णितं जातम्\renewcommand{\thefootnote}{७}\footnote{जातं\begin{tabular}{c}२ \\९\end{tabular}।}\begin{tabular}{c}९ \\२\end{tabular}, एष गच्छः~।}
\vspace{3mm}

{द्वितीयोदाहरणे न्यासः\textemdash}
\vspace{2mm}

\hspace{2cm}{आ \begin{tabular}{c|}३ \\२\\\hline \end{tabular}\,, उ \begin{tabular}{c|}१ \\३\\\hline \end{tabular}\,, ग\renewcommand{\thefootnote}{८}\footnote{ग .~।} ०, सं\begin{tabular}{c|}२७\\ ४\\\hline \end{tabular}}~। 
\vspace{3mm}

{धनात्\begin{tabular}{c}२७\\ ४\end{tabular}, द्विघ्नात्\renewcommand{\thefootnote}{९}\footnote{द्विन्नात्~।}\,\begin{tabular}{c|}२७\\ २\\\hline \end{tabular}\,, चयघ्नात्\,\begin{tabular}{c|}९ \\२\\\hline \end{tabular}\,, चयदलेन\,\begin{tabular}{c|}१ \\६\\\hline \end{tabular}
रहितस्यादेः\,\begin{tabular}{c|}३\\ २\\\hline \end{tabular} जातस्य \begin{tabular}{c|}४ \\३\\\hline \end{tabular}}
{कृत्या \begin{tabular}{|c|}१६\\ ९\\\hline \end{tabular} युतात् \begin{tabular}{c|}११३\\ १८\\\hline \end{tabular}\,, समासन्नमूलं \begin{tabular}{c|}५\\ २\\\hline \end{tabular}\,, प्राङ्मूलेन \begin{tabular}{c|}४ \\३\\\hline \end{tabular} ऊनं\begin{tabular}{c}७\\
६\end{tabular}, चय\textendash \,\bigg(\begin{tabular}{c}१ \\३\end{tabular}\bigg)\textendash \,हृतम्\begin{tabular}{c}७ \\२\end{tabular}, अविकलम्}
{३, अनष्टं\renewcommand{\thefootnote}{१०}\footnote{अति\begin{tabular}{c}११\\ २\end{tabular}कलं ३ अनष्ट~।}, व्येकं २, चयार्ध\textendash \hspace{-2mm} \begin{tabular}{c}१\\३\\\end{tabular}\hspace{-2mm} \textendash गुणितं \begin{tabular}{|c|}१ \\३\\\hline \end{tabular}\,, समुखम्\begin{tabular}{c}११\\ ६\end{tabular},
अनष्टाहतं\begin{tabular}{c}११\\ २\end{tabular}, क्षयो गणिते\begin{tabular}{c}५ \\४\end{tabular}, 
{एतदनष्टघ्नचयेन १ अन्वितमुखस्य\begin{tabular}{c}५ \\२\end{tabular}भाज्यं, लब्धम्\begin{tabular}{c}१\\ २\end{tabular}, अनष्टयुक् \begin{tabular}{|c|}३\\ १\\ २\\\hline \end{tabular}\,, सवर्णितं जातम् \begin{tabular}{c|}७\\ २\\\hline \end{tabular}\,,}
{एतत् पदम्~।}
\vspace{3mm}

{अथात्र करणाभिप्राय उच्यते~। इह सकले पदे क्षेत्रगतराशिगतयोर्भेदो नास्ति
सविकले}
{त्वस्तीत्युपपादितम्\renewcommand{\thefootnote}{११}\footnote{त्वस्त्रीत्यु*~।}\,। तेनात्र\renewcommand{\thefootnote}{१२}\footnote{तेनत्र~।} यस्मिन्सविकले पदे राशिगतं
सङ्कलितं यद्भवति तस्मिन्नेव}
{पदे क्षेत्रगतं ततो न्यूनं भवति धनोत्तरे, (ऋणोत्तरे) त्वधिकम्~। तद्यथा\textendash \,इह आ ३ उ २}
{ग\begin{tabular}{c}९ \\२\end{tabular}, राशिगते धनं\begin{tabular}{c}५९\\ २\end{tabular}; क्षेत्रगते तु करणम्\textendash \,पदं\begin{tabular}{c}९ \\२\end{tabular}, व्येकम्\renewcommand{\thefootnote}{१३}\footnote{एकं~।}\begin{tabular}{c}७
\\२\end{tabular}, अर्धं\begin{tabular}{c}७ \\४\end{tabular}, चयघ्नं\begin{tabular}{c}७ \\२\end{tabular}, सादिः\renewcommand{\thefootnote}{१४}\footnote{सादि~।}}{\begin{tabular}{c}१३\\ २\end{tabular}, पदघ्नम्\begin{tabular}{c}११७\\ ४\end{tabular}, एतत् गणितं राशिगतान्न्यून(मि)ति, सवर्णितं\begin{tabular}{c}११८\\ ४\end{tabular}{चतुर्भागाधिकं चैतत्~।}

\newpage

\noindent{एतद्धनोत्तरे~। ऋणोत्तरे न्यासः\textendash 
\vspace{2mm}

\hspace{20mm} आ ३, उ\renewcommand{\thefootnote}{१}\footnote{३२$+$~।} {२$+$, ग\begin{tabular}{c}९ \\२\end{tabular}।}}
\vspace{3mm}

{अत्र राशिगते कर्म\textendash \,पदं\begin{tabular}{c} ९\\ २\end{tabular}, निर्विकलं ४, चयघ्नं\renewcommand{\thefootnote}{२}\footnote{चयघ्न~।} {\qt 'ऋणं धनर्णयोर्घात'} इति ८$+$,}
{सादिः\renewcommand{\thefootnote}{३}\footnote{सादि~।} ऋणे योज्यमाने धनेऽन्तरमिति\renewcommand{\thefootnote}{४}\footnote{*र १ मिति~।} यथाहुः {\qt 'तयोर्योगे वियोगः स्यात्'} इति ततो जातं ५$+$}
{अनष्टसञ्ज्ञं चैतत्, मुखान्वितं प्राग्वदृणशेषः २$+$, विचयः ऋणधनं
पात्यमधिकीभवति यथाहुः}
{{\qt 'वियोगे सति सङ्गम'} इति ततो जातं ०,\renewcommand{\thefootnote}{५}\footnote{निर्विकलपदार्धहतं . ऋणधनवध 
ऋणमिति ८$+$ विकलघ्नानष्ट \begin{tabular}{c|}$\begin{matrix}
\mbox{{७}}\\
\mbox{{२}}
\end{matrix}+$\\\hline \end{tabular} एतेन युतं एतदत्र गणितम्~।}निर्विकलपदार्धहतं ०,
विकलघ्नानष्टः\begin{tabular}{c}$\begin{matrix}
\mbox{{५}}\\
\mbox{{२}}
\end{matrix}+$\end{tabular}, एतेन}
{युतं जातं\begin{tabular}{c}$\begin{matrix}
\mbox{{५}}\\
\mbox{{२}}
\end{matrix}+$\end{tabular}, एतदत्र गणितम्~।}
\vspace{2mm}

{क्षेत्रगते तु कर्म\textendash \,\renewcommand{\thefootnote}{६}\footnote{एकं पदं\begin{tabular}{c}३\\ २\end{tabular}।}व्येकं पदं\begin{tabular}{c}३ \\१\\ २\end{tabular}, दलितं\begin{tabular}{c}७\\ ४\end{tabular}, चयघ्नं\begin{tabular}{c}$\begin{matrix}
\mbox{{७}}\\
\mbox{{२}}
\end{matrix}+$\end{tabular},
सादिः\begin{tabular}{c}$\begin{matrix}
\mbox{{१}}\\
\mbox{{२}}
\end{matrix}+$\end{tabular},पदसङ्गुणं\begin{tabular}{c}$\begin{matrix}
\mbox{{९}}\\
\mbox{{४}}
\end{matrix}+$\end{tabular}, {(गणितम्), अनेन सवर्णितं प्राच्यं\begin{tabular}{c}$\begin{matrix}
\mbox{{१०}}\\
\mbox{{४}}
\end{matrix}+$\end{tabular}, चतुर्भागैरूनमेतत्\renewcommand{\thefootnote}{७}\footnote{प्राच्यं\begin{tabular}{c}४३\\ ४\end{tabular}त्रयस्त्रिंशता चतु*~।}\,।}
\vspace{3mm}

{एवंगते यदा धनोत्तरं तदा\renewcommand{\thefootnote}{८}\footnote{तुदा~।} सविकलपदे राशिगतसङ्कलितात्\renewcommand{\thefootnote}{९}\footnote{राशिगतं कलित~।}
क्षेत्रगतक्रमेणैव\renewcommand{\thefootnote}{१०}\footnote{क्षत्रग*~।} (ताव)त् पदानयनं कार्यम्~। तच्च बीजमूलमेवेति दर्शितम्~। तथा चेदमपि करणं
तद्वदेव प्रवृत्तम्~।}
\vspace{3mm}

{बीजक्रियया तावत्\textendash}
\vspace{2mm}

{आ ३, उ २, गच्छो न ज्ञायते तदा यावत्तावत्सञ्ज्ञायां कल्पितायां ग या\renewcommand{\thefootnote}{११}\footnote{या २~।} १~।}
\vspace{3mm}

{कर्म\textendash \,व्येकं\renewcommand{\thefootnote}
{१२}\footnote{एक~।} पदं या १ रू\renewcommand{\thefootnote}{१३}\footnote{३१~।} १$+$, दलितं या \begin{tabular}{|c|}१\\ २\\\hline \end{tabular} रू\renewcommand{\thefootnote}{१४}\footnote{ऊ \begin{tabular}{c|}$\begin{matrix}
\mbox{{१}}\\
\mbox{{२}}
\end{matrix}+$\\\hline \end{tabular} क्षेपघ्नं या १ ऊ १~।}\begin{tabular}{c}$\begin{matrix}
\mbox{{१}}\\
\mbox{{२}}
\end{matrix}+$\end{tabular}, चयघ्नं या १ रू १$+$, सादिः$^{\scriptsize{\hbox{{\color{blue}३}}}}$ या १ रू\renewcommand{\thefootnote}{१५}\footnote{ऊ २$+$~।} २, पदसङ्गुणं वर्गं १ या २, एतत् गणितमनेन
गणितेन \begin{tabular}{c|}५९\\ २\\\hline
\end{tabular} सममिति न्यासः\textendash}
\vspace{2mm}

\hspace{20mm} {व\renewcommand{\thefootnote}{१६}\footnote{व १ या २ ऊ . व . या . ऊ \begin{tabular}{c|}५९\\ २\\\hline \end{tabular} पक्षौ सवर्ण्य छेदराशि न्यासः व २ या ४ ऊ अथस्वपक्षे~।} १ या २ रू ० 
\vspace{1mm}

\hspace{20mm}  व ० या ० रू \begin{tabular}{c|}५९\\ २\\\hline \end{tabular}}
\vspace{3mm}

{पक्षौ छेदराशिना सवर्ण्य न्यासः\textendash}
\vspace{2mm}

\hspace{20mm} व २ या ४ रू ० 

\hspace{20mm} व ० या ० रू ५९
\vspace{3mm}

{अस्वपक्षे ५९, चतुर्हते २३६, वर्गहते ४७२, अव्यक्तकृत्या १६ युते ४८८,
मूलमिति}
{\renewcommand{\thefootnote}{१७}\footnote{यो ऊ ५ं४ं~।}मूलयोज-नाङ्कः ४$+$ अन्तर्भूतसङ्कलितमूलानयनाय
प्रवृत्तत्वान्न्यायसिद्धमासन्नमूलग्रहणं तच्चास्य २२,}

\newpage

\noindent{अव्यक्तराशिना ४ रहितं १८, अर्धोनं ९, वर्गभाजितम्\begin{tabular}{c}९\\ २\end{tabular},
एतत् पदप्रमाणम्~। एतदीयक्षेत्रगतसङ्कलितमत्रान्तर्वर्तते,
निर्विकलपदगणितञ्च तुल्यमुभयत्रापीति~।}
\vspace{3mm}

{एतस्मात्पदाद्विकलम(पा)स्य स्थापितात् ४ सड्कलितं यथा\textendash \,व्येकं\renewcommand{\thefootnote}{१}\footnote{एकं~।} पदं ३,}
{अर्धं\renewcommand{\thefootnote}{२}\footnote{अर्धं ३~।}\begin{tabular}{c}३\\२ \end{tabular}, चयघ्नं ३, सादिः\renewcommand{\thefootnote}{३}\footnote{सादि~।} ६, पदसङ्गुणं २४~। एतदुभय(त्र) तुल्यं}
{सङ्कलितं राशिगतादपास्तं शेषं \begin{tabular}{c|}११\\२\\\hline \end{tabular}\,, इदं धनं विकलस्य सिद्धमिति\renewcommand{\thefootnote}{४}\footnote{सैकमिति~।}\,।}
{पञ्चमसम्बन्धिना तावदनेन भवितव्यमिति~। पञ्चमस्य धनमानीयते\textendash \,पदमेकहीनं ४,}
{उत्तरगुणितं ८, संयुक्तमादिना ११, एतदन्त्यस्य पञ्चमस्य धनम्~। तदेतेन}
{प्रमाणेन एकादश द्विभागा अर्धस्य धनमिति लभ्यते अर्धं, तेन युक्ताश्चत्वारः}
{पदमिह राशिगत(म्), एभिः आद्युत्तरसङ्कलितैर्जायत इति~।}
\vspace{3mm}

{मध्यमाहरणाख्यबीजकर्मणोऽनुसारेणेदानीं सूत्रक्रिया व्यञ्ज्यते~। एतत्}
{{\qt 'व्यव्यक्तमर्धोनं\renewcommand{\thefootnote}{५}\footnote{विकृतं~।} प्रमाणं वर्गभाजितम्'} इत्यन्तं बैजिकं कर्म तदिदं}
{\hyperref[92]{'चयहृतम्'} इत्यन्तं\renewcommand{\thefootnote}{६}\footnote{चस्यत*~।} कर्म, ततः सविकलाल्लब्धानुगणित-संवादोऽस्तीति, यतः}
{संवादस्तद्विकलविहीनं\renewcommand{\thefootnote}{७}\footnote{संवातस्त*~।} पदं कृत्वा तदीयं धनमानीतं, विकलानुसरणार्थं\renewcommand{\thefootnote}{८}\footnote{*लानुसार*~।} तच्च}
{\hyperref[92]{'अविकलम्'} इत्यादिना \hyperref[92]{'अनष्टाहतम्'} इत्यन्तेन, \hyperref[92]{'क्षयो\renewcommand{\thefootnote}{९}\footnote{निर्विकलमित्यादिनानष्टहृतमित्यन्तेन चयो~।} गणिते'} इति विकलधनज्ञानाय,}
{\hyperref[92]{'तद-नष्टघ्नचयान्वितमुखभक्तम्'} इति विकललाभः, \hyperref[92]{'अनष्टयुग्गच्छः'}\renewcommand{\thefootnote}{१०}\footnote{*गछब्द~।} इति}
{विवक्षितसविकलपदलाभाय~। तथा च यावत्तावद्गच्छसङ्कलितवर्गाव्यक्तानि योगतो}
{विच्छिन्नानि जायन्ते, पक्षसवर्णने एतेन छेदेनास्वपक्षरूपराशिर्गुणयितव्य}
{इति तदुक्तं द्विघ्नाद्धनादिति~। पक्षशोधनं चात्र नास्ति~।}
\vspace{3mm}

{{\qt 'अस्वपक्षे\renewcommand{\thefootnote}{११}\footnote{अंबपक्षेवर्गादत्तः वर्गहत~।} चतुर्वर्गहत'} इत्यस्मात्प्रकारान्तरेण बीजानयनकरणं यदस्ति (तत्)}

\begin{quote}

{\qt यद्वा\renewcommand{\thefootnote}{१२}\footnote{वित्वा~।} व्यक्तहते वर्गेऽव्यक्तार्धकृतिना\renewcommand{\thefootnote}{१३}\footnote{कृतितां~।} युतम्~। \\
मूलं तेनोनमव्यक्तप्रमाणं\renewcommand{\thefootnote}{१४}\footnote{*मव्यक्तं प्र*~।} वर्गभाजितम्~॥} इति~। \end{quote}

\noindent{तेन {\qt 'व्यक्तहते\renewcommand{\thefootnote}{१५}\footnote{हत~।} वर्गे'} इति यत्कर्तव्यं तदिदं कृतं \hyperref[92]{'चयघ्नधनात्'} इति,
वर्गराशिर्हि तत्प्रमाण एव भवतीति, तथा {\qt 'अव्यक्तार्धकृतिना\renewcommand{\thefootnote}{१६}\footnote{अस्त्य व्वक्तस्य कृत~।} (युत)म्'} इति
यत्तदिदं \hyperref[92]{'चयदलरहितादेः कृतियुतात्'} इति~।}
\vspace{3mm}

{यावत्तावद्गच्छसङ्कलितप्रक्रमे हि \hyperref[85]{'व्येकं$^{\scriptsize{\hbox{{\color{blue}१}}}}$ पदम्'} इति क्रियमाणे यद्रूपं
पृथगेव ऋणात्मकं न्यस्तम्, \hyperref[85]{'अर्धम्'} इति च
दलीकृतांशद्विच्छेदतास्वपक्षसवर्णतावसरे\renewcommand{\thefootnote}{१७}\footnote{*कृतांशा*~।} विनष्टैव\renewcommand{\thefootnote}{१८}\footnote{विनष्टेव~।}, ततः \hyperref[85]{'(अर्धघ्न)चय'}
इति चयरूपता योत्पन्ना साश्रिता चयग्रहणेन\renewcommand{\thefootnote}{१९}\footnote{चयं*~।}, \hyperref[89]{'सादिरि'}ति ग्रहणेन\renewcommand{\thefootnote}{२०}\footnote{रहितादि ग्र*~।}
{\qt 'तयोर्योगे वियोगः स्यात्'} इति न्यायेन (चयदलरहितादिरिति), ततश्च पदसङ्गुणित
इति यावत्तावद्गुणनादव्यक्तीभाव (इति, तस्य) वा {\qt 'कृतिना युतम्'} इति यत्तदिदं
कृतियुतादिति, {\qt 'मूलम्'} इति यत्तत्समासन्नं मूलं, तेनाव्यक्तार्धेनोनम् इति\renewcommand{\thefootnote}{२१}\footnote{*व्यक्तेन*~।}
यत्तत् \hyperref[92]{'प्राङ्मूलोनं'}, {\qt 'वर्गभाजितम्'} इति यत्तत् \hyperref[92]{'चयहृतम्'} इति\renewcommand{\thefootnote}{२२}\footnote{चस्यतं~।}\,।}

\newpage

\noindent{उक्ताभिप्रायात् निर्विकलमिति विकलत्वस्यातात्विकत्वात्त्यागः\renewcommand{\thefootnote}{१}\footnote{*स्यता*~।}
लब्धमविकलं पदं, तदनष्टसञ्ज्ञया}
{स्थापितमुत्तरकर्मार्थं पुनरानयनार्थं च तात्विकविकलमात्रेणैव हि
युयोजयिषत, स च विकलं}
{विक-लधनात्सकलधनं\renewcommand{\thefootnote}{२}\footnote{*धानात्ससक*~।} च \hyperref[92]{'व्येकम्'} इत्यादिना
व्येकपदार्धेत्यनुवादिना\hyperref[92]{'ऽनष्ट(।हत)म्'} इत्यन्तेन\renewcommand{\thefootnote}{३}\footnote{*पदार्थेत्य*~।} लब्धं\renewcommand{\thefootnote}{४}\footnote{लम्बितं~।},}
{\hyperref[92]{'क्षयो गणित'} इति सविकलधनाच्छोधिते शेषं\renewcommand{\thefootnote}{५}\footnote{*धितशेष~।} विकलधनं,
तत्पञ्च(म)धनांशत्वात्तद्धनज्ञानमूलम् इति\renewcommand{\thefootnote}{६}\footnote{ज्ञामूलनमिति~।} तदानयनं \hyperref[92]{'तदनष्टघ्नचयान्वितमुखभक्तम्'} अतस्त्रैराशिकं\renewcommand{\thefootnote}{७}\footnote{*चयार्धेन ततस्त्रै*~।}
कृतमिति\renewcommand{\thefootnote}{८}\footnote{कृतं भक्तमिति~।} लब्धस्तात्विको}
{विकलः, \hyperref[92]{'अनष्टयुगि'}ति सविकलपदलाभः तदुक्तं \hyperref[92]{'गच्छ'} इति~।}
\vspace{3mm}

{करणसूत्रमार्या\textendash}

\phantomsection \label{94}
\begin{quote}
    
{\bs विषमे पदे निरेके गुणं समेऽर्धीकृते कृतिं न्यस्य~।\\
 क्रमशो रूपस्योत्क्रमशो गुणकृतिफलमादिना गुणयेत्~॥~९४~॥}\end{quote}

{गुणोत्तरसङ्कलितमनेन\renewcommand{\thefootnote}{९}\footnote{कृतोत्तरमनेन~।} प्रदर्श्यते, तत्र पदं विषमं वा स्यात् समं वा~। यदा विषमं तदा}
{तन्निरेकं कार्यं गुणशब्दः गुणशब्दाद्यक्षरं 'गु'शब्दो न्यसनीयः\renewcommand{\thefootnote}{१०}\footnote{निस*~।},
एवंकृते शेषमर्धीकृतं ततश्च}
{कृतिशब्दः कृतिशब्दाद्यक्ष(रं) 'कृ'\renewcommand{\thefootnote}{११}\footnote{कृश~।}शब्दो न्यसनीयः, एवमपि कृते
विषमात्मको यदि शेषस्तदा}
{तन्निरेकीकार्यं गुणशब्द इति प्राग्वत्, एवं
तावत्कुर्याद्यावच्छून्यमवशिष्यते, तदनुरूपं क्रमन्यस्तगुणकृत्युपलक्षणाक्षरमालिकायास्तिर्यग्वौत्तराधर्येण वा व्यवस्थापिताया अन्ते
(रूपं) स्थाप्यं, ततस्तस्य}
{रूपस्य समीपवर्त्यक्षरं गुणोपलक्षणं चेत्ततस्तद्रूपं प्रश्नस्थितेन
गुणोत्तरेण गुणयेत्, वर्गोपलक्षणं}
{चेत्ततः तदा तद्रूपं वर्गणीयं, एवं यथाप्राप्तैकतरकर्मणि कृते
तद्रूपलक्षणाक्षरं निवारयेत्,}
{तदनन्तरं च यदुपलक्षणाक्षरं कृतकर्मणो रूपस्य समीपमापद्यते तच्चेत्
गुणोपलक्षणं}
{ततस्तत्प्रश्नस्थितेनेत्यादि वर्गोपलक्षणं चेत्याद्यपि
उपलक्षणाक्षरलोपान्तं कर्म प्राग्वत्, एवं}
{यावत्सम्भवं कृत्वा यज्जायते\renewcommand{\thefootnote}{१२}\footnote{यज्ञायते~।} तत्प्रश्नोक्तेनादिना गुणयेत् तत्
गुणोत्तरसङ्कलितं भवतीति~।}
\vspace{3mm}

{उदाहरणम्\textendash}

\begin{quote}
    
{\eg रूपत्रयं गृहीत्वा लाभार्थं निर्गतो वणिक् कश्चित्~।\\
 प्रतिमासं द्विगुणधनं तस्य भवेत् किं त्रिभिर्वषैः~॥~१०८~॥}\end{quote}

{कश्चिद्वणिक्\renewcommand{\thefootnote}{१३}\footnote{*द्वगिग्~।} रूपत्रयं मूलधनं गृहीत्वा पण्यवीथिकां निर्गतः (ए)तेन
मूलधनेन व्यवहरतो मे लाभोऽस्त्विति, एवं तस्य वर्धयितुकामस्य तन्मूलधनं
प्रतिदिवसव्यवहारात्}
{प्रथममासान्ते द्विगुणं भवेत् मूलधनसमो लाभः स्यात् येन त्रीणि रूपाणि
षड्\renewcommand{\thefootnote}{१४}\footnote{षद्~।} भवेयुरिति}
{सम्भाव्यते, द्वितीये च मासे तद्रूपषट्कं मूलधनं कृत्वा तथैव व्यवहारः तस्य
मासान्ते\renewcommand{\thefootnote}{१५}\footnote{प्रमासान्ते~।} द्विगुणितं}
{भवेत् यथा द्वादशरूपधनं सञ्जायते, इत्येवं यदि वर्षत्रयं स व्यवहरेत् तदा
कियद्धनं\renewcommand{\thefootnote}{१६}\footnote{व्यहारेतिदायि
कियद्धनः~।} स्यादिति~।}

\newpage

{न्यासः\textendash}
\vspace{1mm}

\hspace{15mm} {आ\renewcommand{\thefootnote}{१}\footnote{आ. ३ प्रगु २ गच्छ मासाः ३६~।} ३, गु २, गच्छः मासाः ३६~।}
\vspace{3mm}

{द्विगुणिताया मासावधिकत्वाद्वर्षाणि मासीकृत्वा स्थापितानि~। अतः कर्म\textendash \,पदं ३६,}
{विषमत्वाभावान्निरेकीकरणं सम्प्रति तावन्नास्ति सममित्य(तोऽ)र्धीक्रियते
१८ लब्धमुपलक्षणार्थं}
{वकारः, पुनरपि समम् इत्यतोऽर्धीक्रियते ९ लब्धम् उपलक्षणार्थं वकारः, इदानीं
विषमत्वान्निरेकीक्रियते ८ लब्धम् उपलक्षणाक्षरं गुणकारः, पुनः समत्वादर्धीक्रियते ४ लब्धं
वकारः, पुनरपि}
{समत्वात् अर्धीक्रियते २ लब्धं वकारः, पुनरपि समत्वादर्धीक्रियते १ लब्धं
वकारः, विषमत्वान्निरेकीक्रियते शून्यमवशिष्यते लब्धं गुणकारः, एवं जाते सर्वान्ते रूपन्यासः,
तदेवं लब्धम्}

\begin{center}
    
{व व गु व व व गु १}\end{center}

{अथास्य रूपस्य लब्धाक्षरानुसारोक्तं कर्म\textendash \,रूपं १, गुणः २, एतेन गुणनात्
२ गुणशब्दलोपः, वर्गणात् ४ वकारलोपः, वर्गणात् १६ वकारलोपः, वर्गणात्\renewcommand{\thefootnote}{२}\footnote{वर्गणात् २२५६~।} २५६
वकारलोपः,}
{गुणनात् ५१२ गुकारलोपः, वर्गणात् २६२१४४ वकारलोपः, वर्गणात्\renewcommand{\thefootnote}{३}\footnote{वर्गणात् \begin{tabular}{l}~। \,~। \,~। \,~। \,~। \,~। \,~। \,~। \\ २.६१५८४३०६८७१६८७१९४७६७\end{tabular}।}
६८७१९४७६७३६,}
{एतद्रूपस्योत्क्रमगुणकृतिफलम्, आदिना ३ गुणयेत्\renewcommand{\thefootnote}{४}\footnote{गुणयेत् २.६१५८४३०२०~।} २०६१५८४३०२०८,
तल्लब्धं सङ्कलितम्~।}
{इदं सूत्रपदसम्मितार्वाङ्मूलधनं\renewcommand{\thefootnote}{५}\footnote{*मूले*~।} गुणोत्तरेणावेदिषुरिति\renewcommand{\thefootnote}{६}\footnote{*वेधिषु*~।}
लाघवार्थं तथापि हि सिद्ध्यत्येतत्तथा च}
{प्रथमादिमासेषु द्विगुणं द्विगुणं मूलधनं प्रतिमासं यथागच्छति तथा
लिख्यते\renewcommand{\thefootnote}{७}\footnote{लिख्यते
६~। १२~। २४~। ४८~। ९६~। १९२~। ३४८~। ७६८~। १५~। ३६~। ३०७२~।}\textendash}

\renewcommand*{\arraystretch}{1}
\begin{center}
\begin{tabular}{rrrrrr}
 १. & ६ & ~~~~१३. & २४५७६ & ~~~~२५. & १००६६३२९६ \\
 २. & १२ & १४. & ४९१५२ & २६. & २०१३२६५९२\\
 ३. & २४ & १५. & ९८३०४ & २७. & ४०२६५३१८४\\
 ४. & ४८ & १६. & १९६६०८ & २८. & ८०५३०६३६८\\
 ५. & ९६ & १७. & ३९३२१६ & २९. & १६१०६१२७३६\\
 ६. & १९२ & १८. & ७८६४३२ & ३०. & ३२२१२२५४७२\\
 ७. & ३८४ & १९. & १५७२८६४ & ३१. & ६४४२४५०९४४\\
 ८. & ७६८ & २०. & ३१४५७२८ & ३२. & १२८८४९०१८८८\\
 ९. & १५३६ & २१. & ६२९१४५६ & ३३. & २५७६९८०३७७६\\
 १०. & ३०७२ & २२. & १२५८२९१२ & ३४. & ५१५३९६०७५५२\\
 ११. & ६१४४ & २३. & २५१६५८२४ & ३५. & १०३०७९२१५१०४\\
 १२. & १२२८८ & २४. & ५०३३१६४८ & ३६. & २०६१५८४३०२०८\\
\end{tabular}
\end{center}

\newpage

\renewcommand*{\arraystretch}{0.7}
\noindent{करणसूत्रमार्यापूर्वार्धम्\textemdash}
\renewcommand{\thefootnote}{}\footnote{\hspace{-3.5mm} ९१४४~। १२२८८~। २४५७६~। ४९१५२~। ९८३.४~। १९६६.८~। ३९३२१९~। ७८६४३९~। १५७२६६४~। ३१४५७२८~। ६१९१४५६~। १२५६२९१२~। २५१६५८२४~। ५.३३१६४६~। १००६६३३९६~। २०१३२६५९२~। ४०२६५३१८४~। ६५३०६३८८~। १६६.९१२७३६~। ३२२१२२५४७२~।८४४२२४४.९४४~। १२८८४९.२८८८~। २५७६९८०३७७६~। ५१५३९६०७.५५२~। १०३७९२१४११०४२०९१५८४३.२०८~। \\
\vspace{1mm}}

\phantomsection \label{95.1}
\begin{quote}
    
{\bs  प्राग्वत्फलमाद्यूनं\renewcommand{\thefootnote}{१}\footnote{प्राक्वत्फफमा*~।} निरेकगुणभाजितं भवेत् गणितम्~।}\end{quote}

{पूर्वसूत्रोक्तवत् \hyperref[94]{'विषमे पदे निरेके गुणं समेऽर्धीकृते\renewcommand{\thefootnote}{२}\footnote{समर्धी*~।} कृतिं
न्यस्ये'}त्यादिवद्यत्फलं तत्प्रश्नोक्तेनादिना प्रभवेण विरहितं रूपोनेन\renewcommand{\thefootnote}{३}\footnote{रूपेण~।} गुणोत्तरेण हृतं सङ्कलितं स्यात्~।}
\vspace{3mm}

{उदाहरणम्\textendash}

\begin{quote}
    
{\eg एको लभते त्रीणि द्विगुणं द्विगुणं ततो परे पुरुषाः~।\\
पञ्चैवं लप्स्यन्ते कियद्धनं कथ्यतामाशु\renewcommand{\thefootnote}{४}\footnote{कथिता*~।}\,॥~१०९~॥}\end{quote}

{इदमाशु कथ्यतां (यत्) पञ्च पुरुषाः परस्य वैतनेये\renewcommand{\thefootnote}{५}\footnote{वैनतेये~।} कर्मणि
प्रवृत्तास्तेषां कर्मनिष्पादनविशेषादेकः पुरुषस्त्रीणि रूपाणि लभते द्वितीयः षट् तृतीयो द्वादश
चतुर्थश्चतुर्विंशतिः}
{पञ्चमोऽष्टाचत्वारिंशतं (तदा) तदीयलाभधनपिण्डसङ्ख्या कियतीति~।}
\vspace{3mm}

{आद्युत्तरपदेषु ज्ञातेषु (अ)ज्ञातसङ्कलितानयनार्थो न्यासः\textemdash}
\vspace{-1mm}

\begin{center}
    
{आ ३, गु २, गच्छः\renewcommand{\thefootnote}{६}\footnote{आ ३३२ गच्छ~।} ५, सङ्कलितं न ज्ञायते~।}\end{center}
\vspace{-1mm}

{कर्म\textendash \,\hyperref[95.1]{'प्राग्वत्'} इत्यतिदेशाद्विषमे पदे निरेके ४ लब्धो गुणः,
समेऽर्धीकृते\renewcommand{\thefootnote}{७}\footnote{समर्धीक्रियते~।} २ लब्धो व, पुनः समेऽर्धीकृते १ लब्धो व, विषमे पदे निरेके लब्धो गुणः, अन्ते रूपम्~। एवं
लब्धकस्थापनम्\textendash}
\vspace{-1mm}

\begin{center}
    
{गु व व गु १}\end{center}
\vspace{-1mm}

\noindent{अथास्य रूपस्य लब्धाक्षरानुसारोक्तं कर्म\textendash \,रू १, गुणः\renewcommand{\thefootnote}{८}\footnote{गुणः १~।} २, वर्गः ४,
वर्गः १६, गुणः\renewcommand{\thefootnote}{९}\footnote{गुणः २~।} ३२, आदिना}
{गुणने\renewcommand{\thefootnote}{१०}\footnote{गुणन~।} (९६), इदमातिदेशिकं कर्म, अतो जातं प्राग्वत्फलं ९६,
आद्यूनं\renewcommand{\thefootnote}{११}\footnote{अद्यूनं~।} ९३, निरेकेन\renewcommand{\thefootnote}{१२}\footnote{निरेकेतु~।} गुणेन १}
{भाजितं ९३, लब्धं सङ्कलितम्\renewcommand{\thefootnote}{१३}\footnote{संकलित~।} (९३)~।}
\vspace{3mm}

{पूर्वस्मिन् गणिते प्रथमपद(त) एव वृद्धिप्रवृत्तिः अस्मिंस्तु
द्वितीयपदात्प्रभृतीति\renewcommand{\thefootnote}{१४}\footnote{*यपादा*~।}}
{गणितविशेषारम्भः तुल्येष्वपि मुखोत्तरपदेषु (न) क्षेत्रगतराशिगतवत्\renewcommand{\thefootnote}{१५}\footnote{*गतशशि*~।}\,।}
\vspace{3mm}

{अथ यत्राद्यन्तधनपदानि ज्ञायन्ते तत्र सङ्कलितानयनार्थं
करणसूत्रमार्यापरार्धमाह\textemdash}

\begin{quote}
    
 {\bs आद्यन्तवलयधनयुतिदलेन वलयाहतिर्मूल्यम्~॥~९५~॥}\end{quote}

\newpage

{सीमन्तिनीभुजलतालङ्करणवलयावली नियमत एव परिपाट्या विपुलवृत्तोत्तरवलया}
{भवतीति वलयोपादानं प्रकृत्याद्युत्तरिकानुवृत्तिसूचनार्थम्~। एवमादिविषये
आदिधनान्त्यधनयोगार्धं पदगुणं सङ्कलितं भवति, यथोक्तम्\textemdash}
\vspace{-1mm}

\begin{center}
    
{\qt 'आदियुतान्त्यधनार्धं मध्यधनं पदगुणं गणितम्~।'} \end{center}
\vspace{-1mm}

\noindent{इति~। यद्यपि चान्त्यधना(दा)दिधनमपास्य व्येकेन पदेनाप्तं प्रचयमुपलभ्य
प्राक्तनेनैव}
{\hyperref[85]{'व्येकपदे'}त्यादिना सूत्रेण सङ्कलितावाप्तिरस्ति तथापि लघुकरणार्थो
वचनारम्भः~। उक्तञ्च}
{यत्तत्रापि प्राचयिकम्, आद्यसम्बन्धि च धनमन्त्यधने संयोज्य\renewcommand{\thefootnote}{१}\footnote{धनमन्त्यधनादपास्य~।} दलनेन
मध्यधनतां नीत्वा पदघ्नं}
{विधाय युक्त्या सङ्कलितं साधितमिति च
तस्मादन्त्यधनादानीतेनोत्तरेणान्त्यापत्तिरेव}
{सङ्कलितानुसरणाय कार्येति किमयत्नलब्धस्य\renewcommand{\thefootnote}{२}\footnote{कमयत्न*~।} यत्नेन~।} 
\vspace{3mm}

{उदाहरणम्\textendash}

\begin{quote}
    
 {\eg अष्टाभिर्मुखवलयं पणैस्त्रयोद(श)भिरन्त्यवलयं तु~।\\
 वलयानि चतुर्विंशतिरेषां\renewcommand{\thefootnote}{३}\footnote{*विंशति*~।} किं मूल्यमाचक्ष्व~॥~११०~॥} \end{quote}

{मुखवलयं सूक्ष्मवलयं यत् भुजाग्रवर्ति स हि प्रदेशः सूक्ष्मो भवति,
अन्त्यवलयं स्थू(लव)लयं यत् भुजपूर्वभागवर्ति स हि प्रदेशः स्थूलो भवति, मध्यमानि तु वलयानि
स्थूलसूक्ष्माणि~।}
{तेषां प्रमाणवत् क्रमोपचयिकमूल्यत्वादुपलब्धादिवलयादन्त्यवलये
धनवदुपचयत्वं मूल्यस्य\renewcommand{\thefootnote}{४}\footnote{*वलयांत्यवलये धनवदुपचयस्य मूल्यत्वम्~।}\,। यत्र~चतु-र्विंशतिषु वलयेषु मुखवलयमष्टाभिः प्राप्यते अन्तवलयन्तु
त्रयोदशभिस्तत्र तेषां सर्वेषां}
{मूल्यपिण्ड-सङ्ख्या कियतीति साध्यताम्~।}
\vspace{3mm}

{न्यासः\textendash \hspace{4mm} आ ८, अ १३, ग २४~।}
\vspace{3mm}

{कर्म\textendash \,आदिवलयधनं ८, अन्त्यवलयधनं १३, अनयोर्युतिः \begin{tabular}{c|}२१\\\hline \end{tabular}\,, दलं\begin{tabular}{c}२१\\ २\end{tabular}, वलयसङ्ख्यया २४ गुणं २५२, लब्धं सङ्कलितं सकलवलयावलीमूल्यम्~।}
\vspace{3mm}

{पृथक् पृथक् मूल्यज्ञानं तु यदि वलयान्तराणामिष्यते
तदोत्तरधनमानीयेष्टवलयमन्त्यं}
{परिकल्प्यान्त्यधनानयनमिव\renewcommand{\thefootnote}{५}\footnote{*नमेव~।} तद्धनानयनं कार्यम्~।}
\vspace{3mm}

{अथादिधनं न ज्ञायते तदा यावत्तावत्परिकल्पनानीतगणितेन ज्ञातगणितं
स्पर्धयेत्तत आदिस्वरूपं व्यक्तं भवति~। इत्थं च कर्म\textendash \,आ या १, अन्त्यं\renewcommand{\thefootnote}{६}\footnote{अन्त्य~।} १३,
गच्छः २४, आद्यन्तवलयधनयुतिः या १ रू\renewcommand{\thefootnote}{७}\footnote{ऊ~।} १३, दलं या \begin{tabular}{c|}१ \\२\\\hline \end{tabular} रू$^{\scriptsize{\hbox{{\color{blue}७}}}}$\begin{tabular}{c}१३\\ २\end{tabular}, वलयैः २४ हतं या १२ रू$^{\scriptsize{\hbox{{\color{blue}७}}}}$ १५६, एतदव्यक्तसङ्कलितं}

\newpage

\noindent{व्यक्तेन सङ्कलितेनामुना\renewcommand{\thefootnote}{१}\footnote{*मुना २५२ सममिति न्यासः \begin{tabular}{|c|} या १२ \\  या ०   \\\hline \end{tabular}\begin{tabular}{c|} ऊ १५६ \\ ऊ २५२  \\\hline \end{tabular}~।} २५२ सममिति न्यासः\textendash}

\begin{center}
\begin{tabular}{|c|} या १२ \\  या ०   \\\hline \end{tabular}\begin{tabular}{c|} रू १५६ \\ रू २५२  \\\hline \end{tabular}
\end{center}

{पक्षयोरपवर्तनं द्वादशभिर्विधाय\renewcommand{\thefootnote}{२}\footnote{*धान~।} न्यासः\renewcommand{\thefootnote}{३}\footnote{न्यासः \begin{tabular}{c} या १ \\  या ० \end{tabular}\begin{tabular}{|c|} ऊ १\\ २१~। ३ \\\hline \end{tabular}~।}\textemdash}

\begin{center}
\begin{tabular}{|c|}  या १ \\ या ० \\\hline \end{tabular}\begin{tabular}{c|}  रू १३  \\ रू २१   \\\hline \end{tabular}
\end{center}

{{\qt 'संशोध्याव्यक्तम्'} इत्यादिना (लब्धम)व्यक्तप्रमाणं रू\renewcommand{\thefootnote}{४}\footnote{ऊ~।} ८, एतदादिधनम्~।}
\vspace{2mm}

{अथान्तधनं \,न \,ज्ञायते \,तदा\renewcommand{\thefootnote}{५}\footnote{तदाया~।} यावत्तावत् परिकल्पनयानीतगणितेन \,व्यक्तं\, गणितं \,स्पर्धयेत् ततोऽन्त्यस्वरूपं व्यक्तं भवतीति~। इत्थं च कर्म\textendash \,आ ८, अन्त्य\renewcommand{\thefootnote}{६}\footnote{अत्र
या १६~।} या १,
ग २४, आद्यन्तवलयधन-युतिः या १ रू$^{\scriptsize{\hbox{{\color{blue}४}}}}$ ८, दलं\renewcommand{\thefootnote}{७}\footnote{र्दंल~।} या \begin{tabular}{|c|}१\\ २\\\hline \end{tabular}  रू$^{\scriptsize{\hbox{{\color{blue}४}}}}$
४, वलयै\textendash \,२४\textendash \,राहतं या १२ रू$^{\scriptsize{\hbox{{\color{blue}४}}}}$ ९६~। एतदव्यक्तसङ्कलितं} 
{व्यक्तेनामुना २५२ सममिति न्यासः\textendash}
\vspace{-1mm}

\begin{center}

\begin{tabular}{r} (\,या \\ या \end{tabular}
\begin{tabular}{c} १२ \\ ० \end{tabular}
\begin{tabular}{c} रू   \\ रू \end{tabular}
\begin{tabular}{r}९६  \\ २५२    \end{tabular}
\end{center}
\vspace{-1mm}

{द्वादशभिरपवर्तिते\,) 
\vspace{-1mm}

\begin{center}
{\begin{tabular}{r}(\,या \\ या \end{tabular}
\begin{tabular}{c}	१\\ ०  \end{tabular}
\begin{tabular}{c} रू$^{\scriptsize{\hbox{{\color{blue}४}}}}$  \\ रू  \end{tabular}
\begin{tabular}{c}८ \\ २१\,)  \end{tabular}}
\end{center}
\vspace{-1mm}

{{\qt 'संशोध्याव्यक्तम्'} इत्यादिना लब्धमव्यक्तप्रमाणं १३, एतदन्त्यधनम्~।}
\vspace{2mm}

{अथ पदं न ज्ञायते (तदा) अव्यक्तसङ्कलितानयनं यथा\textendash \,आ$^{\scriptsize{\hbox{{\color{blue}४}}}}$ ८, अ १३, ग या}
{१, आद्यन्तवलयधनयुतिः रू$^{\scriptsize{\hbox{{\color{blue}४}}}}$ २१, दलं \begin{tabular}{|c|}२१\\ २\\\hline \end{tabular}\,, वलयैरिति}
{गच्छेन या १ आहते\renewcommand{\thefootnote}{८}\footnote{आहते \begin{tabular}{|c|} १\\ २\\\hline \end{tabular}~।}
{या\begin{tabular}{c}२१\\ २\end{tabular}, एतदव्यक्तसङ्कलितं व्यक्तेनामुना २५२}
{सममिति समच्छेदीकृत्य छेदराशिना\renewcommand{\thefootnote}{९}\footnote{च्छेदराशि~।} न्यासः\renewcommand{\thefootnote}{१०}\footnote{न्यासः या २१ ऊ . या . ऊ ५४~।}\textendash}
\vspace{-1mm}

\begin{center}
{\begin{tabular}{r} या \\ या \end{tabular}
\begin{tabular}{c} २१  \\ ० \end{tabular}
\begin{tabular}{c} रू$^{\scriptsize{\hbox{{\color{blue}४}}}}$   \\ रू \end{tabular}
\begin{tabular}{r} ०   \\ ५०४    \end{tabular}}
\end{center}
\vspace{-1mm}

{पक्षयोरेकविंशत्यापवर्तनं}
\vspace{-1mm}

\begin{center}
{\begin{tabular}{r}(\,या \\ या \end{tabular}
\begin{tabular}{c}	१\\ ०  \end{tabular}
\begin{tabular}{c} रू  \\ रू  \end{tabular}
\begin{tabular}{c}० \\ २४\,)  \end{tabular}}
\end{center} 

\newpage

\noindent {{\qt 'संशोध्याव्यक्तम्'} इत्यादिना लब्धमव्यक्तप्रमाणं रू\renewcommand{\thefootnote}{१}\footnote{ऊ~।} २४, एतत् पदम्~।}
\vspace{3mm}

{अथाद्यन्तधनयुतिः\renewcommand{\thefootnote}{२}\footnote{*धनयुतिर्न~।}  ज्ञायते न तु तौ पृथगिति तज्ज्ञानं यथा\textendash \,आ\renewcommand{\thefootnote}{३}\footnote{आ या १२ अ २२ या$+$ ग २४~।}
या १, अं रू २१}
{या १$+$, ग २४; आद्यन्तवलयधनयुतिः रू$^{\scriptsize{\hbox{{\color{blue}१}}}}$ २१, दलं रू\renewcommand{\thefootnote}{४}\footnote{ऊ
\begin{tabular}{|c|} २१\\ ३१\\\hline \end{tabular}~।}\begin{tabular}{c}२१ \\२\end{tabular},
वलयैराहतं रू\renewcommand{\thefootnote}{५}\footnote{ऊ २५~।} २५२, एतदव्यक्तसङ्कलितमिति व्यक्ताव्यक्तपक्षयोः साम्याद्यादृच्छिको बीजराशिः
(आदिराशिः),}
{तच्छुद्धो मिश्रराशिः (अन्त्यराशि)रिति~।}
\vspace{3mm}

{द्विविधमिह सङ्कलितं नियतमनियतञ्च, नियतमपि द्विविधं
प्रत्युत्पन्नात्मकमादिप्रचयात्मकं च~। तत्रानन्तरोक्तयोः पदपरिच्छेदेन भाव्यम्~। यदा चैतयोः
पदसाम्यं धनसाम्यं चेष्यते}
{प्रत्युत्पन्नस्य प्रथमज्ञाने आदिप्रचयस्य चाद्युत्तरयोर्ज्ञातयोस्तदा
तत्काललाभाय\renewcommand{\thefootnote}{६}\footnote{तल्लाभाय~।} करणसूत्रमार्यामाह\textendash}

\begin{quote}
    
{\bs  नियतगतेस्त्यक्त्वादिं\renewcommand{\thefootnote}{७}\footnote{*त्यत्कादिं~।} शेषं द्वाभ्यां समाहतं\renewcommand{\thefootnote}{८}\footnote{समाहितं~।} विभजेत्~।\\
 वृद्ध्याप्तमेकसहितं तुल्यगतौ जायते कालः~॥~९६~॥}\end{quote}

{काल इह पदस्योपलक्षणं\renewcommand{\thefootnote}{९}\footnote{पादस्यो*~।}, तुल्यगतावित्यत्र गतिः सङ्कलितस्य,
नियतगतेरित्यत्र (च)}
{प्रत्येक-पदधनम्, उभयोरपि धनयोः सङ्कलितयोर्नित्यत्वे परस्परापेक्षया
प्रत्युत्पन्नः सुतरां}
{नियतत्वात् नियतोऽन्यस्त्वनियत\renewcommand{\thefootnote}{१०}\footnote{*त्वानियतोऽन्यसनि*~।} इति~। वापीह तथोपपत्तौ तदयमर्थः~। 
अनियतगतिभाविनमादिं}
{नियतगतेः अपास्य शेषं द्विगुणय्यानियतगतिभाविनोत्तरेण\renewcommand{\thefootnote}{११}\footnote{द्विगुणयानियतगतिभाविनात्त*~।} विभज्य
लब्धमेकाधिकं कुर्यात्, जायते}
{तुल्य-पदज्ञानलाभः तुल्यधनलाभश्च, तन्मूल्यमुक्तकरणत एव सर्वत्र\renewcommand{\thefootnote}{१२}\footnote{पूर्वत्र~।}\,।
\vspace{3mm}

{उदाहरणम्\textendash}

\begin{quote}
    
{\eg  त्र्याद्येकोत्तरवृद्ध्या\renewcommand{\thefootnote}{१३}\footnote{आद्येको*~।} यात्येकः प्रतिदिनं नरस्त्वन्यः~।\\
 दश योजनानि कियता कालेन तयोर्गतिस्तुल्या~॥~१११~॥}\end{quote}

{एकस्मिन्नेव दिने तुल्यप्रमाणमध्वानं तुल्यकालप्राप्तये द्वौ पथिकौ
प्रपन्नौ, ययोरेकः}
{प्रथमदिने त्रीणि योजनानि गच्छति तदुत्तरदिने एकैकं योजनमधिकीकरोति,
अन्यस्तु प्रतिदिनं}
{दशैव योजनानि याति~। तयोरिदानीं कियता तुल्येन कालेन
तुल्यगन्तव्यप्राप्तिर्भविष्यतीति~।}

\vspace{3mm}
{न्यासः}\textendash \hspace{4mm} आ ~~३, ~उ\renewcommand{\thefootnote}{१४}\footnote{३~।} १, एकपथिकगतिः 

\hspace{13.5mm} आ\renewcommand{\thefootnote}{१५}\footnote{आ १० ३०~।} १०, उ ~\,०, द्वितीयपथिकगतिः\renewcommand{\thefootnote}{१६}\footnote{दितीय*~।}
\vspace{3mm}

अत्र नियतगते\textendash \,१०\textendash \,रनियतगतिसम्बन्धिनमादिं\renewcommand{\thefootnote}{१७}\footnote{नियता गतिर्योनियतगतिसम्बन्धिनं १० अतो आदि*~।} 
{३ त्यक्त्वा शेषं ७, द्वाभ्यां २ समाहतं १४, }

\newpage

\noindent{इदमनियतगतिसम्बन्धिनोत्तरेण १ भक्तं १४, रूपसहितं १५, एष
दिनात्मकस्तुल्यः कालः~।}
{तथा च नियतगतिर्नियतदिनवृन्देन\renewcommand{\thefootnote}{१}\footnote{नियतगतिनियतादिनवृन्देन~।} सपञ्चाशतं (शतं) योजनानि गच्छति,
तस्माद्यद्येकेनाह्ना}
{दश योजनानि याति तदा\renewcommand{\thefootnote}{२}\footnote{तत्~।} पञ्चदशभिः कियन्तीति त्रैराशिकेन लभ्यन्ते\renewcommand{\thefootnote}{३}\footnote{कियन्ति इति लभ्यते एतत्
त्रैराशिकेन~।}
(१५०)~।}
{अनियतगतेरपि तुल्यत्वं तथा च\textemdash}
\vspace{-2mm}

\begin{center}
    
{आ ३, उ\renewcommand{\thefootnote}{४}\footnote{ऊ$^{\scriptsize{\hbox{{\color{black}८}}}}$ १~।}  १, ग १५~।}\end{center}
\vspace{-1mm}

{व्येकं\renewcommand{\thefootnote}{५}\footnote{एक~।} पदं १४, अर्धं ७, चयेन १ हतं ७, सादिः\renewcommand{\thefootnote}{६}\footnote{आदि~।} १०, गच्छेन १५
सङ्गुणितम्\renewcommand{\thefootnote}{७}\footnote{संगुणितं १५.~।} १५०~।}
\vspace{2mm}

{अथवा \hspace{5mm} आ ३, उ\renewcommand{\thefootnote}{८}\footnote{३१ ग.
सं १५.~।} १, ग ०, सङ्कलितम् १५०~।}
\vspace{3mm}

{अष्टोत्तरहतफलतः\renewcommand{\thefootnote}{९}\footnote{*फलतः १२.~।} १२००, द्विघ्नस्यादेः ६ प्रचयस्य १ च विवरं ५,
कृतिः २५, अनया}
{युतात् १२२५, मूलं ३५, द्विगुणमुखेन ६ ऊनं २९, सचयं ३०, द्विचयोद्धृतं १५,
एष गच्छः~।}
\vspace{3mm}

{बीजपक्षे\textendash \hspace{4mm} आ ३, उ\renewcommand{\thefootnote}{१०}\footnote{८~।} १, ग या १~।}
\vspace{3mm}

{व्येकपदस्य\renewcommand{\thefootnote}{११}\footnote{एक*~।} या १ रू\renewcommand{\thefootnote}{१२}\footnote{ऊ ५१$+$~।} १$+$ अर्धेन या \begin{tabular}{|c|} १\\ २\\\hline \end{tabular}  रू\renewcommand{\thefootnote}{१३}\footnote{ऊ~।}
\begin{tabular}{|c|}$\begin{matrix}
\mbox{{१}}\\
\mbox{{२}}
\end{matrix}+$\\\hline \end{tabular}  चयो १ हतः या \begin{tabular}{|c|} १\\ २\\\hline \end{tabular} रू$^{\scriptsize{\hbox{{\color{blue}१३}}}}$ \begin{tabular}{|c|}$\begin{matrix}
\mbox{{१}}\\
\mbox{{२}}
\end{matrix}+$\\\hline \end{tabular} सादिः या \begin{tabular}{|c|} १\\ २\\\hline \end{tabular}  रू\renewcommand{\thefootnote}{१४}\footnote{ऊ ५ च्छे २~।}\begin{tabular}{c}५ \\२\end{tabular}, गच्छेन
सङ्गुणो\renewcommand{\thefootnote}{१५}\footnote{संगुणः\begin{tabular}{c}१\\ २\end{tabular}या ५ च्छे २~।} व\begin{tabular}{c}१ \\२\end{tabular}या\begin{tabular}{c}५ \\२\end{tabular}, एतदेवाव्यक्तसङ्कलितम्\renewcommand{\thefootnote}{१६}\footnote{एतदेकाव्य*~।},}
{अव्यक्तगुणितया\renewcommand{\thefootnote}{१७}\footnote{अव्यक्तच्छेद गुणिताया~।} नियतगत्यानया या १० सममिति\renewcommand{\thefootnote}{१८}\footnote{*गत्याऽनया २०
सममिति~।} (स)दृशच्छेदीकृत्य
न्यासः\renewcommand{\thefootnote}{१९}\footnote{च्छेदराशेर्न्यासः~।}\textendash}
\vspace{3mm}

\hspace{20mm} व ~\;१ \;या \;५ \;इति प्रथमः पक्षः~। 

\hspace{20mm} व\renewcommand{\thefootnote}{२०}\footnote{व.~।} ० या २० एष द्वितीयः पक्षः~। 
 
पक्षशोधनेन \hspace{3.7mm} व ~\;१ \;या १५$+$ ~~~प्र शे~। 

\hspace{20mm} व ~\;० \;या \;० रू ० द्वि शे\renewcommand{\thefootnote}{२१}\footnote{शोधनं व \begin{tabular}{|c|} १\\ २\\\hline \end{tabular}  या १५$+$ प्र शे ऊ. द्विशे~।}\,।
\vspace{3mm}

{अस्वपक्षे चतुर्हते ०, वर्गहते\renewcommand{\thefootnote}{२२}\footnote{चतुर्वर्गहते ऊ~।} ०~। अव्यक्तकृतौ (युते) २२५, मूलं
१५, अव्यक्तोनं\renewcommand{\thefootnote}{२३}\footnote{व्यक्तं~।}}
{३०, अर्धोनं १५, वर्गभाजितं (१५ , पद)प्रमाणं भवति~।} 
\vspace{3mm}

{अथवा पक्षयोर्याव(त्) भागेनापवर्त्य न्यासः\textendash}
\vspace{-1mm}

\begin{center}

\begin{tabular}{c}या\\या\end{tabular}\begin{tabular}{c}१\\०\end{tabular}\begin{tabular}{c}रू$^{\scriptsize{\hbox{{\color{blue}१३}}}}$ \\रू$^{\scriptsize{\hbox{{\color{blue}१३}}}}$ \end{tabular}\begin{tabular}{c}५ \\२०\end{tabular}
\end{center}
\vspace{-1mm}

 अतो लभ्यते १५~। 
\vspace{3mm}

 {इदञ्च सूत्रं युक्तिबलोत्पन्नम्~। तथा हि नियतानियतयोर्धनसाम्ये
नियतगतितुल्यत्वेनानियतगतेरपि गतिः प्रकल्पनीया उच्चावचतायाः समीपकरणेन,}

\newpage

\noindent{तच्च तस्य मध्यधनं}
{ज्ञातव्यम्~। तच्च द्विगुणं (आदिविहीनम्), अन्त्यधनम्,
पुनरादिधनविहीनम्,\renewcommand{\thefootnote}{१}\footnote{आदिधनं*~।}  उत्तरभक्तमेकयुतं पदं भवति~। यदि वा मध्यधनमादिवियुतं
द्विगुणमादिधनविहिनान्त्यधनसम्मितं}
{जायते, तदुत्तरभक्तमेकोनं भवति तदैव सैकं पदमिति स एषोऽन्यः\renewcommand{\thefootnote}{२}\footnote{*ऽन्त्यः~।} प्रकारः
सूत्रकारेण लिखितो\renewcommand{\thefootnote}{३}\footnote{लम्बितो~।}}
{लाघवार्थम्~। पूर्वत्र पञ्चकर्माणीह चत्वारि~।}
\vspace{3mm}

{अथ यत्र द्वावपि सङ्कलितराशी एव तुल्यौ विषमान्यादिप्रचयपदानि
प्रचययोर्वैषम्यमेकत्र च पक्षे\renewcommand{\thefootnote}{४}\footnote{पक्षं~।} प्रचयो निश्चितो न त्वादिः द्वितीये प्रचयोऽपि, ततः
प्रथमसङ्कलितसाम्यानन्तरं}
{द्वितीयसङ्कलितसाम्ये ज्ञाताद्यादिभिः पदानयनार्थं\renewcommand{\thefootnote}{५}\footnote{द्वितीयोऽपि प्रचयोऽपि 
संकलितयोश्च ज्यायानीयो भावो विचितकालः संकलितयोश्च मध्ये धनसम्पद्वयं ततो
ज्ञाताद्यानयनार्थं~।}
करणसूत्रमार्याद्वयमाह\textendash}

\begin{quote}
    
{\bs अधिकः\renewcommand{\thefootnote}{६}\footnote{अभिकः~।} प्रथमस्य चयो\renewcommand{\thefootnote}{७}\footnote{च युतो~।} मुखमिष्टं पदमितिर्द्वितीयस्य\renewcommand{\thefootnote}{८}\footnote{*मितिद्वि*~।}\,। \\
 इष्टात्प्रथमस्य पदादादिः कल्प्यो द्वितीयस्य~॥~९७~॥ \\
 पदहृतफलविश्लेषाच्चयहतपदयोर्विशेषदलहीनात्\renewcommand{\thefootnote}{९}\footnote{*षाश्चय*~।}\,।\\
 चयविवरदलेन हृतादाप्तं दिवसा\renewcommand{\thefootnote}{१०}\footnote{द्विवसा~।} द्वितीययुतौ~॥~९८~॥}\end{quote}

{प्रथमप्रवृत्तस्य पश्चात्प्रवृत्तसङ्कलिते\renewcommand{\thefootnote}{११}\footnote{*कलित~।}  ज्ञातप्रचयादधिकः प्रचयो
मन्तव्यः, आदिस्तु}
{यादृच्छिकः, पश्चात्प्रवृत्तस्य\renewcommand{\thefootnote}{१२}\footnote{*वृतस्य~।} पदपरिमाणं यादृच्छिकम्
अर्थात्\renewcommand{\thefootnote}{१३}\footnote{अर्थेतु~।} पश्चात्प्रवृत्तपदाधिकप्रथमप्रवृत्तगतकालपरिमाणं\renewcommand{\thefootnote}{१४}\footnote{*वृत्तेपदाधिकं प्रथम*~।} पदं भवति~। एवं च सति
प्रथमप्रवृत्तस्यादिप्रचयपदानि ज्ञातानीति~।}
{तेभ्य उक्तकरणेन सङ्कलितं लभ्यते एव, तदेव चापि द्वितीयस्य\renewcommand{\thefootnote}{१५}\footnote{यदेव चाप्य द्वितीय~।}
प्राश्निकेनैवेष्टत्वात्~।}
{स हि प्रथमधनसाम्ये काल\renewcommand{\thefootnote}{१६}\footnote{तदैवकाल~।} इष्यते~। किन्त्वत्रादिरवशिष्यते स च
ज्ञातैरुत्तरपदसङ्कलितैरुक्तानयनकरणेनैव\renewcommand{\thefootnote}{१७}\footnote{*करण एव~।}\,। इतीत्थं प्रथमधनसाम्ये समधिगतपदाभ्यां\renewcommand{\thefootnote}{१८}\footnote{*गतेपदा*~।}
पृथक् पृथक् फलं विभज्य}
{भागाप्तयोरन्तरमासाद्य पृथक् पृथक्\renewcommand{\thefootnote}{१९}\footnote{*मास्यत् पपृथक्यथा~।} प्रचयगुणितयोः
पदयोरन्तरस्यार्धेन न्यूनीकृतं}
{चययोर्विश्लेषस्यार्धेन भजे(त्) लब्धं द्वितीयधनसाम्यकालः,
प्राक्पदयोश्चेप्सितयोजिते\renewcommand{\thefootnote}{२०}\footnote{*योश्चाप्सित्यो*~।}}
{प्रवृत्तिकाललाभो द्वयोरपीति~।}
\vspace{3mm}

{उदाहरणम्\textendash}

\begin{quote}
    
{\eg षड्दिवसैः पुंसि गते केनापि मुखोत्तरेण यातोऽन्यः~।\\
 तेनैव पथा द्विकचयमभूत् कथं मेलकद्वितयम्\renewcommand{\thefootnote}{२१}\footnote{मेकलकद्वितयं~।}\,॥~११२~॥} \end{quote}

\newpage

{कस्मिंश्चित् पथिके केनाप्यविज्ञातेनाद्युत्तरक्रमेण पूर्वं याते
दिवसषट्ककृतप्रस्थाने}
{पश्चादन्यः पथिकः केनापि मुखेन द्वाभ्यामुत्तरेण\renewcommand{\thefootnote}{१}\footnote{*रेणा~।}  यातः, तयोरेवं
प्रवृत्तयोरेवं गच्छतोः पथि}
{द्वयोर्जनयोः\renewcommand{\thefootnote}{२}\footnote{गच्छतो पथि द्वयोर्जनैः~।} (वारद्वयं) सङ्गमो दृष्टः स कथं स्यात्
केनाद्युत्तरपदनियमेनेति~।}
\vspace{3mm}

{न्यासः\renewcommand{\thefootnote}{३}\footnote{न्यासः आ . ६ . ग अती ६
त दिनानि ९ सं आ ८२ ग अतीतदि स १~।}\textendash \,प्रथमस्य आ ०, च ०, ग अतीतदिनानि ६, सं ०~।}

\hspace{10mm} {द्वितीयस्य आ ०, च २ , ग अतीतदिनानि, सं ०~।}
\vspace{3mm}

{अत्र प्रथमस्य द्वितीयादधिकश्चयः षट् च\renewcommand{\thefootnote}{४}\footnote{८~।} ६, मुखमिष्टमिति आदिरेकः आ १~। अथ}
{द्वितीयपदमिष्टं\renewcommand{\thefootnote}{५}\footnote{२१ *मिष्ट~।}  चतुष्कं ग ४ अर्थात् प्रथमस्य जातं पदं दश (१०)~।}
\vspace{3mm}

{इदानीं फलमानीयते~।}
\vspace{2mm}

\hspace{15mm} {आ १, च\renewcommand{\thefootnote}{६}\footnote{८~।} ६, ग १०~।}
\vspace{3mm}

{पदस्य १० व्येकस्य\renewcommand{\thefootnote}{७}\footnote{एकस्य~।}  ९ अर्घेन \begin{tabular}{|c|}९ \\२\\\hline \end{tabular}  चयः ६
गुणितः २७, सादिः २८, गच्छेन १०}
{गुणितः २८० (सङ्कलितम्)~।}
\vspace{3mm}

{एतदेव द्वितीयस्येति उत्तरपदसङ्कलितेषु ज्ञातेषु अज्ञातस्यादेरानयनार्थो
न्यासः\textemdash}
\vspace{2mm}

\hspace{15mm}  आ ०, च २, ग ४, सं २८०~।\renewcommand{\thefootnote}{८}\footnote{आ ६२ ग ४ सं २६०~।}
\vspace{3mm}

{गणितं २८०, गच्छेन ४ हृतं ७०; गच्छो ४ निरेकः\renewcommand{\thefootnote}{९}\footnote{स्यतं \begin{tabular}{c|}\hline ७\end{tabular}  गच्छेनातिरेक ३~।} ३, चयेन २ (नि)घ्नः
(६), दलेन}
{३ ऊनं\renewcommand{\thefootnote}{१०}\footnote{ऊनं ८७~।} ६७, एष द्वितीयस्यादिः, अनेन चादिना फलं प्रथमसममेवेति
तत्प्रदर्शनार्थो न्यासः\textemdash}
\vspace{2mm}

\hspace{15mm} {आ ६७, च\renewcommand{\thefootnote}{११}\footnote{६~।} २, ग ४~।}

\vspace{3mm}
{पदस्य ४ व्येकस्य ३ अर्धेन \begin{tabular}{|c|} ३\\ २\\\hline \end{tabular} चयो २
(नि)घ्नः ३, सादिः\renewcommand{\thefootnote}{१२}\footnote{सादिः ७५~।} ७०, (प)देन (४)}
{सङ्गुणितः २८०, एतत् गणितम्\renewcommand{\thefootnote}{१३}\footnote{गुणितं~।}\,।}
\vspace{3mm}

{कथिताध्वना इयता च कालेनैवं गच्छतोरेको मेलको जातः~। यत्र द्वितीयोऽपीष्यते}
{तत्राद्युत्तरौ द्वयोरपि पक्षयोरानीतावेव कालविशेषात्मकं त्वधिकं पदं
ज्ञेयं तत एव च सङ्कलितविशेषो लभ्यते इति तदर्थं कर्म\textendash}
\vspace{3mm}

{प्रथमपक्षस्थेन पदेन १० हृतं फलं २८, तथा द्वितीयपक्षस्थेन पदेन ४ हृतं
फलं ७०,}
{अनयोरन्तरं ४२; तथा प्रथमपक्षस्थेन चयेन ६ स्वपदं\renewcommand{\thefootnote}{१४}\footnote{स्वपदं १. हतं ६.~।} १० हतं ६०, तथा
द्वितीयपक्षस्थेन}
{चयेन २ स्वपदं\renewcommand{\thefootnote}{१५}\footnote{स्वपपद~।} ४ हतं (८), अनयोर्विश्लेषस्य\renewcommand{\thefootnote}{१६}\footnote{*लेषेण~।} ५२ दलेन २६ हीनात्
(१६), चययो\renewcommand{\thefootnote}{१७}\footnote{दलयो~।}\textendash \,(२, ६)\textendash \,र्विवरस्य (४) (द)लेन २ हृतात् लब्धं\renewcommand{\thefootnote}{१८}\footnote{लब्धं ६~।} ८, द्वितीययुतिकालः~।}

\newpage

{एतत्\renewcommand{\thefootnote}{१}\footnote{एततत्~।} प्राक्पदयोः संयोज्य धनमानीयते~। प्रथमस्य तावत् आ १, उ ६ ग १८~।\renewcommand{\thefootnote}{२}\footnote{प्रथमस्यात्तावत् आ० १३९ २~। १८~।}}
{पदस्य\renewcommand{\thefootnote}{३}\footnote{पदस्य १६~।}  १८ व्येकस्य\renewcommand{\thefootnote}{४}\footnote{एकस्य~।}  १७ अर्धेन \begin{tabular}{|c|} १७\\ २\\\hline \end{tabular}  चयः ६ गुणितः ५१, सादिः
५२, पदसङ्गुणः ९३६~।}
{द्वितीयस्य खल्वपि आ\renewcommand{\thefootnote}{५}\footnote{आ ९७ ८२ पदं १२~।}  ६७, च २, प १२~। पदस्य १२ व्येकस्य$^{\scriptsize{\hbox{{\color{blue}४}}}}$ ११
अर्द्धेन \begin{tabular}{|c|} ११\\ २\\\hline \end{tabular}  चयः (२)}
{गुणितः ११, सादिः ७८, पदसङ्गुणः ९३६~। एतदनयोर्द्वितीयधनसाम्यम्\renewcommand{\thefootnote}{६}\footnote{*धनस्यसा*~।}\,।}
\vspace{3mm}

{अथ प्रथमयुतिकालादनन्तरं धनसाम्यं प्रदर्श्यते~। तत्र प्रथमस्य
तावदेकादशे दिवसे धनं\renewcommand{\thefootnote}{७}\footnote{दिन~।}}
{साध्यते, तच्चाद्यं\renewcommand{\thefootnote}{८}\footnote{तच्चान्त्यं~।}  प्रकल्प्यम्~। पदं ११ ,एकहीनं १०, उत्तरेण ६
गुणितं ६०, आदिना संयुक्तं ६१,}
{एतदेकादशे दिवसे\renewcommand{\thefootnote}{९}\footnote{द्विवसे~।}  धनमादिरुत्तरस्य धनस्य, तेनायं न्यासः\textemdash}
\vspace{2mm}

\hspace{15mm} आ ६१, च ६, प ८~।\renewcommand{\thefootnote}{१०}\footnote{आ० ९१ ८९ ७~। ८~।}
\vspace{3mm}

{पदस्य ८ व्येकस्य$^{\scriptsize{\hbox{{\color{blue}४}}}}$ (७ दलेन)\begin{tabular}{c}७ \\२\end{tabular}चयः ६ गुणितः २१, सादिः\renewcommand{\thefootnote}{११}\footnote{सादिः ६२~।}  ८२,
गच्छेन (८)}
{गुणितः ६५६~। द्वितीयस्य खल्वपि~। तस्य च पञ्चमे दिवसे धनं साध्यते~। पदं
५, व्येकं\renewcommand{\thefootnote}{१२}\footnote{एकं~।}  ४,}
{उत्तरेण २ गुणितं ८, आदिना ६७ संयुक्तं ७५, एष आदिरिति\renewcommand{\thefootnote}{१३}\footnote{आदिना ८७
सयुक्त ७५ एषा 
हादि*~।}  न्यासः\textemdash}
\vspace{2mm}

\hspace{15mm} {आ ७५, च २, ग ८~।\renewcommand{\thefootnote}{१४}\footnote{आ ८२ गत~।}}
\vspace{3mm}

{पदस्य ८ व्येकस्य$^{\scriptsize{\hbox{{\color{blue}४}}}}$ ७ अर्धेन\begin{tabular}{c}७ \\२\end{tabular}चयः २ गुणितः ७, सादिः ८२,
पद\textendash \,८\textendash \,सङ्गुणः\renewcommand{\thefootnote}{१५}\footnote{सादिः ८२५ पदसंगुणः ९५८~।}  ६५६,}
{एतत् पूर्वधनेन (२८०) संयोज्य जायते तदेव ९३६~। इतोऽनन्तरं तु चयोऽधिको
वर्तते तथाप्येवं}
{धनमानयन्ति~। पदमाद्यं १०, द्विगुणं २०, व्येकेनान्त्येन\renewcommand{\thefootnote}{१६}\footnote{एकेना*~।}  ७ संयुक्तं
२७, चयार्ध\textendash \,३\textendash \,गुणितं ८१,}
{युक्तमादिना ८२, अन्त्यहतं ६५६, धनम्~।}
\vspace{3mm}

{अथवा पदं १८, शुद्धे पदे १० न्यस्तं २८, उत्तरेणा\textendash \,(६)\textendash \,हतं\renewcommand{\thefootnote}{१७}\footnote{हतं १८८~।}  १६८,
द्विघ्नादिना (२)}
{युतं १७०, उत्तरोनं\renewcommand{\thefootnote}{१८}\footnote{उत्तरोनं १८४~।}  १६४, पदान्तरस्य दलेन ४ आहतं ६५६ (धनं); तथा पदं
१२, शुद्धे\renewcommand{\thefootnote}{१९}\footnote{शुद्धि~।}  पदे}
{४ न्यस्तं १६, उत्तरेणा\textendash \,(२)\textendash \,हतं ३२, द्विघ्नादिना १३४ युतं १६६,
उत्तरेणो\textendash \,२\textendash \,नं १६४,}
{पदान्तरस्य ८ दलेनाहतं ६५६~।}
\vspace{3mm}

{अथवेत्थं द्वितीययुतिकालः, तद्यथा प्रथमप्रस्थितस्य
प्रथमयुतिदिना(न)न्तरम् एकादशे}
{दिने एकषष्टिर्धनं भवति~। तत् द्वितीययुतावादिधनम्\renewcommand{\thefootnote}{२०}\footnote{*युतेनादि*~।}  उत्तरस्तु\renewcommand{\thefootnote}{२१}\footnote{उत्तस्य~।} 
षडेव, तत्र पदं (न) ज्ञायते}
{इति यावत्तावता\renewcommand{\thefootnote}{२२}\footnote{*वत्~।}  धनमानीयते~।}
\vspace{3mm}

{न्यासः\textendash \hspace{4mm} आ ६१, उ\renewcommand{\thefootnote}{२३}\footnote{३~।}  ६, ग या १~।}
\vspace{3mm}

{पदस्य या १ व्येकस्य\renewcommand{\thefootnote}{२४}\footnote{एकस्य ३१ १$+$~।}  या १ रू १$+$ अर्धेन या\begin{tabular}{c}१ \\२\end{tabular}रू\renewcommand{\thefootnote}{२५}\footnote{ऊ~।}\begin{tabular}{c}$\begin{matrix}
\mbox{{१}}\\
\mbox{{२}}
\end{matrix}+$\end{tabular}चयो ६
निघ्नः\renewcommand{\thefootnote}{२६}\footnote{घ्नः~।}}

\newpage

\noindent{या ३ रू\renewcommand{\thefootnote}{१}\footnote{ऊ~।} ३$+$, सादिः\renewcommand{\thefootnote}{२}\footnote{सादि~।}  या ३ रू$^{\scriptsize{\hbox{{\color{blue}१}}}}$ ५८, पदसङ्गुणः व ३ या ५८, एतत्
प्राक्प्रस्थितस्य धनम्,}
{पश्चात्प्रस्थितस्य खल्वपि~। तस्य द्वितीययुत्यादिदिने आदितः पञ्चमे दिने
धनं भवति}
{पञ्चसप्ततिः आदिधनञ्चात्र तदुत्तरस्तु द्वावेव, तत्र पदं न ज्ञायते इति
यावत्तावता धनमानीयते~।}
\vspace{3mm}

{न्यासः\textendash \hspace{4mm} आ ७५, उ\renewcommand{\thefootnote}{३}\footnote{३~।}  २, ग या १~।}
\vspace{3mm}

{पदस्य या १ व्येकस्य\renewcommand{\thefootnote}{४}\footnote{एकस्य~।}  या १ रू १$+$ अर्धेन (या)\begin{tabular}{c}१ \\२\end{tabular}रू$^{\scriptsize{\hbox{{\color{blue}१}}}}$\begin{tabular}{c}$\begin{matrix}
\mbox{{१}}\\
\mbox{{२}}
\end{matrix}+$\end{tabular}चयो
२ निघ्नः\renewcommand{\thefootnote}{५}\footnote{घ्नः~।}}
{या १ रू$^{\scriptsize{\hbox{{\color{blue}१}}}}$ १$+$, सादिः या १ रू$^{\scriptsize{\hbox{{\color{blue}१}}}}$ ७४, (पदसङ्गुणः व १ या ७४), एतत्
पश्चात्प्रस्थितस्य}
{धनम्~। इदानीम् एतेऽव्यक्तपरिमाणज्ञानाय समीक्रियेते, तदर्थञ्चानयोः\renewcommand{\thefootnote}{६}\footnote{समीक्रियते तदर्थश्चा*~।} 
पक्षयोरव्यक्तभागेनापवर्तितयोर्न्यासः\textemdash}
\vspace{2mm}

\hspace{15mm} \begin{tabular}{c} या \\ या \end{tabular}\begin{tabular}{c} ३\\ १\end{tabular}\begin{tabular}{c} रू$^{\scriptsize{\hbox{{\color{blue}१}}}}$\\ रू$^{\scriptsize{\hbox{{\color{blue}१}}}}$\end{tabular}\begin{tabular}{c} ५८ \\ ७४\end{tabular}
\vspace{2mm}

{{\qt 'संशोध्याव्यक्तम्'} इत्यादिना लब्धमव्यक्तप्रमाणं ८, एतत्पदे
द्वितीययुतिकालः\renewcommand{\thefootnote}{७}\footnote{द्वितीयप्रतिकाला~।}\,।}
{द्वितीयमेलापधनं चेत्येवानीयते उत्तरसूत्रानीतफलानुसारेण~। तद्यथा
प्राक्प्रस्थितस्य द्वे पदे,}
{प्राक्पदं १० अन्त्यपदं\renewcommand{\thefootnote}{८}\footnote{अन्त्यपदं १~।}  ८~। तत्र प्राक्पदमेकविहीनं ९, रूपादिचयेन
(त)त्फलम् ४५, (एषात्रवृद्धिः~। \hyperref[99]{'पूर्वपदैक्ये प्रभवे रूपे प्रचयेऽन्यगच्छानां'} ८ फलम् १००,
एषात्र वृद्धिः)~। \hyperref[99]{'पूर्वपदस्येव फलं सर्वपदैक्यस्य'}\renewcommand{\thefootnote}{९}\footnote{सर्वपदे न्यस्य~।} १८, \hyperref[99]{'प्राक्पदमेकविहीनम्'} इति १७, (रूपादिचयेन) फलं
१५३,}
{हीनपदवृद्ध्या\renewcommand{\thefootnote}{१०}\footnote{हीनादिवृद्ध्या~।}  १०८ द्विगुणितया २१६ ऊनं\renewcommand{\thefootnote}{११}\footnote{ऊनं ३८$+$~।}  ६३$+$, चयेन ६ हतं\renewcommand{\thefootnote}{१२}\footnote{हतं ३७६$+$~।} 
३७८$+$, आदि\textendash \,(१)\textendash \,गुणयोः पदयोः १०, ८, अन्तरेण २ युतं {\qt 'तयोर्योगे वियोगः स्यात्'} इति
३७६$+$,}
\hyperref[99]{'सङ्ख्या क्षयात्मिका चेद्भवति जयो हीनगच्छस्ये'}ति लब्धधनं (३७६),
प्रथममेलापनात् द्वितीयमेलापनमियतातिरिच्यते, तस्य शतद्वयमशीत्यधिकम्~। अतस्तच्चेदं चेह
धनं भवतीति ६५६, इदं च तच्चोभयमेलापे धनं भवति ९३६~।
\vspace{3mm}

{अनुप्रस्थितस्य खल्वपि द्वे (पदे), प्राक्पदं (४) हीनपदं चैतत्, अनुपदं ८
अधिकपदं}
{चैतत्~। तत्र प्राक्पदं ४, एकविहीनं ३, रूपादिचयेन तत्फलं ६, एषात्र\renewcommand{\thefootnote}{१३}\footnote{पषात्र~।} 
वृद्धिः~।}
{\hyperref[99]{'पूर्वपदैक्ये प्रभवे रूपे प्रचयेऽन्यगच्छानां'} ८ फलं\renewcommand{\thefootnote}{१४}\footnote{पूर्वपदस्येव फलं सर्वं पदैक्यस्य
प्रभवे रूपे प्रचयेत्पगच्छाना ८~।}  ६०,
एषात्र$^{\scriptsize{\hbox{{\color{blue}१३}}}}$ वृद्धिः~। \hyperref[99]{'पूर्वपदस्येव फलं (सर्व)पदैक्यस्य'} (१२ \hyperref[99]{'प्राक्पदमेकवि)हीनम्'} इति (११) रूपादिचयेन तत्फलं
६६,}
{हीनपदवृद्ध्या ६ द्विगुणितया १२ ऊनं ५४, चयेन २ हतं १०८, आदिगुणयोः पदयोः
२६८~।}
{५३६ अन्तरेण (२६८) युतं ३७६, \hyperref[99]{'अभ्यधिकपदस्यैवं विजये सङ्ख्या प्रजायते\renewcommand{\thefootnote}{१५}\footnote{प्रजायेत~।} पुंस'} इति~। द्वितीयमेलापस्य पदमिहाभ्यधिकपदमतो लब्धम् ८~।} 

\newpage

{(अथ निश्चितादिप्रचयाभ्यां देवितुं प्रवृत्तयोर्द्वयोः
पुरुषयोर्निर्दिष्टगच्छपर्यन्तं पर्यायवृत्त्या विजयिनोरन्ते कस्य जयः स्यादिति ज्ञानार्थमार्यात्रयमाह\textemdash)}

\phantomsection \label{99}
\begin{quote}
    
{\bs प्राक्पदमेकविहीनं रूपादिचयेन तत्फलं वृद्धिः~।\\
 पूर्वपदैक्ये प्रभवे रूपे प्रचयेऽन्यगच्छानाम्~॥~९९~॥ \\
 पूर्वपदस्येव फलं सर्वपदैक्यस्य हीनपदवृद्ध्या~।\\
 द्विगुणितयोनं\renewcommand{\thefootnote}{१}\footnote{द्विगुणितयोः नं~।}  चयहतमादिगुणपदान्तरेण युतम्~॥~१००~॥ \\
 अभ्यधिकपदस्यैवं विजये सङ्ख्या प्रजायते पुंसः~।\\
 सङ्ख्या क्षयात्मिका चेत् भवति जयो हीनगच्छस्य~॥~१०१~॥}\end{quote}

{आद्यो गच्छो विरूपः कार्यः ततस्तेन पदेनैकाद्युत्तरचयनयात् यत् सङ्कलितं
भवति सा वृद्धिः,}
{हीनपदवृद्ध्या तत्र व्यवहार\renewcommand{\thefootnote}{२}\footnote{सा वृद्धिहीनपदवृद्ध्यै तत्र व्यवहारे~।}   उपयुज्यते~। तथान्येषां द्वितीयादीनां
पदानां निजनिजापेक्षया}
{यानि पूर्वाणि पूर्वाणि पदानि यथासम्भवमेकं द्वे बहूनि तेषामैक्यं
समाहारस्तस्मिन्नाद्ये रूपे}
{उत्तरे यत्\renewcommand{\thefootnote}{३}\footnote{उत्तरयेत्~।}   सङ्कलितं, सा तस्य तस्य यथास्वं वृद्धिः~। उपयोगः
प्राग्वत्~। यथा\renewcommand{\thefootnote}{४}\footnote{यंच~।}   सर्वपदानां}
{वृद्धिरानीता तथा\renewcommand{\thefootnote}{५}\footnote{वृद्धथानीती स~।} पूर्वपदवत् सर्वपदैक्यात्फलमानेतव्यं, यथैवोक्तं
\hyperref[99]{'प्राक्पदमेकविहीनं रूपादिचयेन तत्फलम्'} इति~। ततश्च तेषां पदानां मध्ये हीनं न्यूनसङ्ख्याकं यत्पदं
तस्य या वृद्धिरानीता}
{\hyperref[99]{'पूर्वपदैक्ये प्रभवे'} इति कर्मणा, तया द्विगुणितया तत् सर्वपदैक्यफलं ऊनं
कार्यं, ततश्चयेन}
{प्राश्निकेन प्रचयेन हन्यात्, तदनन्तरमादिना गुणितानां पदानामन्तरेण
योजयेत् पूर्वकृते~।}
{एतस्य फलस्य द्वे गती सम्भवतः धनात्मकता वा ऋणात्मकता वा~। तत्र
धनात्मकतायामभ्यधिकपदस्य विजयो भवति, तावत्सङ्ख्यं धनमसौ जयतीति~। ऋणात्मकतायां तु
हीनपदस्य}
{विजयो विज्ञेयः, तावत्परिमाणं धनमसौ जयतीति~। पृथक् पृथक् सङ्कलितेभ्यो
तत्\renewcommand{\thefootnote}{६}\footnote{तद~।}   विज्ञाय}
{भवत्यपि जयपराजयनिर्णयः, किन्तु लाघवार्थं (सूत्रम्)~।}
\vspace{3mm}

{अथोदाहरणम्\textendash}

\begin{quote}
    
{\eg नवकादिषट्कवृद्ध्या पातास्त्रिंशद्दशाथ शतमष्टौ~।\\
 द्यूते विजिता द्वाभ्यां\renewcommand{\thefootnote}{७}\footnote{द्वयाभ्यां~।} परस्परं कथय (क)स्य जयः~॥~११३~॥}\end{quote}

{द्वौ पुमांसौ देवितुं प्रवृत्तौ यत्र प्रथमः पातो नवभिः रूपैः द्वितीयः
पञ्चदशभिस्तृतीयः}
{पात एकविंशतेरित्येवमादिषूत्तरक्रमेण पाताः जीयन्ते, एवं व्यवस्थायां
प्रथमतस्तावदेकेन}
{त्रिंशत्पाताः जिताः अ(न)न्तरं द्वितीयेन दश पुनः प्रथमेन शतं पुनः
द्वितीयेनाष्टौ, तत्र न}
{ज्ञायते को जयी कियच्च धनं जयतीति~।}

\newpage

{न्यासः\textendash}
\vspace{2mm}

\hspace{15mm} {आ ९, उ ६, ग ३०, ग १०, ग १००, ग ८~।\renewcommand{\thefootnote}{१}\footnote{आ ९ उ ६ ग ३० ग १० ग १० ग १० ग ८~।}}
\vspace{3mm}

{अत्र प्रथमपुरुषस्य पदसमाहारः १३०, द्वितीयस्य १८~। आद्योऽधिकपदः, इतरो
हीनपदः~।}
{तत्र प्राक्पदं ३०, एकविहीनं २९, रूपादिचयेन तत्फलं तद्यथा, न्यासः\textendash \,आ १,
उ १, ग २९,}
{पदस्य २९ व्येकस्य २८ अर्धेन (१४) चयः १ निघ्नः\renewcommand{\thefootnote}{२}\footnote{घ्नः~।}  १४, सादिः १५,
पदसङ्गुणः\renewcommand{\thefootnote}{३}\footnote{यदसंगणः~।}  ४३५,}
{एतत्\renewcommand{\thefootnote}{४}\footnote{पतत्~।}  फलमत्र वृद्धिः~। \hyperref[99]{'पूर्वपदैक्ये प्रभवे रूपे
प्रचयेऽन्यगच्छानाम्'}\renewcommand{\thefootnote}{५}\footnote{*येत्य*~।} इति द्वितीयगच्छस्य वृद्ध्यर्थो} {न्यासः, तत्र यदा पूर्वमेकमेव भवति तदा तत् ऐक्यं, तद्यथा\textendash \,आ\renewcommand{\thefootnote}{६}\footnote{आ २०~।}  ३०, उ
१, ग १०, पदस्य १०}
{व्येकस्य ९ अर्धेन\begin{tabular}{c}९\\ २\end{tabular} चयः १ निघ्नः$^{\scriptsize{\hbox{{\color{blue}२}}}}$\begin{tabular}{c}९ \\२\end{tabular}, सादिः \begin{tabular}{|c|}६९\\ २\\\hline \end{tabular}\,, पदेन १०
सङ्गुणः ३४५, एषा वृद्धिः~। अथ}
{तृतीयगच्छार्थो न्यासः\renewcommand{\thefootnote}{७}\footnote{न्यासः ग १००~।}\textendash \,पूर्वपदैक्ये ४० एष आदिः, तेन आदिः ४०, उ
१, ग\renewcommand{\thefootnote}{८}\footnote{ग १~।}  १००, अतः पदस्य\renewcommand{\thefootnote}{९}\footnote{पदस्य
१००~।}}
{१०० व्येकस्य ९९ अर्धेन \begin{tabular}{|c|}९९\\ २\\\hline \end{tabular} चयो १ निघ्नः$^{\scriptsize{\hbox{{\color{blue}२}}}}$ \begin{tabular}{|c|}९९ \\२\\\hline \end{tabular}, सादिः\begin{tabular}{c}१७९\\ २\end{tabular},
पदसङ्गुणः ८९५०, एषात्र}
{वृद्धिः~। (अथ चतुर्थगच्छार्थो न्यासः)\textendash \,पूर्वपदैक्यं १४० आदिः, तेन
आ १४०, उ १, ग\renewcommand{\thefootnote}{१०}\footnote{पूर्वपदैक्यं १४० उ १ ग १ ग ८~।}  ८,}
{\hyperref[85]{'व्येकपदे'}त्यादिना फलम् ११४८~। अत्र सर्वपदैक्यं १४८, तस्य
पूर्वपदवत्फलं यथोक्तं \hyperref[99]{'प्राक्पदमेकविहीनम्'} इत्यादि, तेन सर्वपदैक्येनैकविहीनेन न्यासः\textendash \,आ १, उ १, ग
१४७, अतः पदस्य}
{१४७ व्येकस्य १४६ अर्धेन ७३ चयः १ निघ्नः$^{\scriptsize{\hbox{{\color{blue}२}}}}$ ७३, सादिः ७४, पदेन १४७
सङ्गुणः १०८७८,}
{एतद्धीनपदवृद्ध्या\renewcommand{\thefootnote}{११}\footnote{एकहीन*~।} १४९३ द्विगुणितया\renewcommand{\thefootnote}{१२}\footnote{द्विगुणतया~।}  २९८६ ऊनं ७८९२, चयेन ६
हतं\renewcommand{\thefootnote}{१३}\footnote{हतं ४७३४२~।}  ४७३५२,}
{प्रथमपुरुषपदे १३० आदिना ९ गुणिते ११७०, द्वितीयपुरुषपदे १८
आदि\textendash \,(९)\textendash \,गुणे १६२,}
{अनयोरन्तरं १००८, एतेन\renewcommand{\thefootnote}{१४}\footnote{१...८ पतेत~।}  युक्तं ४८३६०, लब्धमधिकपदजयसङ्ख्या~।}
\vspace{3mm}

{तथा च प्रथमतस्तावदनेन त्रिंशत्पाता नवकादिषट्कवृद्ध्या जिताः, तत्र
जितधनसङ्ख्याज्ञानार्थो न्यासः\textendash \,प्रा ९, उ\renewcommand{\thefootnote}{१५}\footnote{३६७३०~।}  ६, ग ३०,
व्येकपदेत्यादिना गणितं २८८०~। अत उत्तरे}
{दश पाता\renewcommand{\thefootnote}{१६}\footnote{पातात्~।}  द्वितीयेन जिताः, तस्य जितधनसङ्ख्याज्ञानार्थो न्यासः\textemdash
आ ९, उ ६, आ ग ३०,}
{अन्त्य ग १०, पदमाद्यं ३० द्विगुणितं\renewcommand{\thefootnote}{१७}\footnote{द्विगुणितं १०~।}  ६०, अन्त्येन १० व्येकेन ९
युतं ६९, चयार्धेन ३ गुणितं}
{२०७, आदि\textendash \,(९)\textendash \,युतं २१६, अन्त्य\textendash \,(१०)\textendash \,हतं २१६०, एतत्
प्रथमजितात्पतति पातिते}
{शेषश्चायं ७२०~। अत उत्तरे शतं पाताः प्रथमेनैव जिताः, तस्य
जितधनसङ्ख्याज्ञानार्थो}
{न्यासः\textendash \,तत्र पूर्वपदैक्यमाद्यं पदं यथा आ ९, उ\renewcommand{\thefootnote}{१८}\footnote{८~।}  ६, आ ग ४०, अं
ग १००, आद्यं पदं ४०}
{द्विगुणितं ८०, अन्त्येन १०० व्येकेन ९९ संयुतं १७९, चयार्धेन (३)
गुणितं\renewcommand{\thefootnote}{१९}\footnote{त्रिगुणितं~।}  ५३७, युक्तमादिना ५४६, अन्त्य\textendash \,(१००)\textendash \,हतं\renewcommand{\thefootnote}{२०}\footnote{हितं ५४६~।}  ५४६००, एतत्
प्राग्धनेन\renewcommand{\thefootnote}{२१}\footnote{पतत प्राग्येनेन~।}  (७२०) युतं ५५३२०, प्रथमपुरुष एव\renewcommand{\thefootnote}{२२}\footnote{यव~।}  जयति~। अत उत्तरेऽष्टौ पाता द्वितीयपुरुषेण जिताः, तस्य
जितधनसङ्ख्याज्ञानार्थो}
{न्यासः\textendash \, आ ९, उ\renewcommand{\thefootnote}{२३}\footnote{३~।}  ६, आद्यं पदं\renewcommand{\thefootnote}{२४}\footnote{पदं २४०~।}  १४०, अन्त्यं पदं ८, आद्यं
पदं १४० द्विगुणितं २८०,}

\newpage

\noindent{व्येकेनान्त्येन (७) सहितं २८७, चयार्धं (३) गुणितं ८६१, युक्तमादिना ८७०,
अन्त्य\textendash \,(८)\textendash \,हतं ६९६०, एतदपि\renewcommand{\thefootnote}{१}\footnote{पतदपि~।} प्रथमजितात्पतति शिष्टं चैतत् ४८३६०~।}
\vspace{3mm}

 {अथ हीनपदजयोदाहरणम्\textendash}

\begin{quote}
    
{\eg सप्त त्रि नव द्वादश पाताः पूर्वोक्तमुखचयौ यत्र~।\\
 तत्र भवेत् कस्य जयो गणयित्वा कथय यदि वेत्सि~॥~११४~॥}\end{quote}

{यत्रैकेन द्यूतकेन पुरुषेण सप्त पाताः\renewcommand{\thefootnote}{२}\footnote{पादाः~।}  जिताः, द्वितीयेन
तूत्तरास्त्रयः, पुनः प्रथमेन नव,}
{द्विती-येनापि पुनर्द्वादश, ताभ्यामेव नवषट्काभ्यामाद्युत्तराभ्यां, तत्र
को जयी कियच्च धनं}
{जयतीति~।}
\vspace{-1mm}

{न्यासः\textendash \hspace{4mm} आ ९, उ\renewcommand{\thefootnote}{३}\footnote{३~।}  ६, पाताः ७~। ३~। ९~। १२~।}
\vspace{3mm}

{अत्र प्राक्पदं ७, एकविहीनं ६, रूपादिचयेन तत्फलं २१, वृद्धिः~। 
पूर्वपदैक्ये\renewcommand{\thefootnote}{४}\footnote{सर्व*~।}  ७ प्रभवे}
{रूपे १ प्रचये द्वितीयगच्छस्य ३ फलं २४, वृद्धिः~। पूर्वपदैक्ये\renewcommand{\thefootnote}{५}\footnote{पूर्वपदैक्ये १~।}  १०
प्रभवे रूपे प्रचये तृतीयगच्छस्य ९ फलं १२६, वृद्धिः~। पूर्वपदैक्ये १९ प्रभवे रूपे १ प्रचये
चतुर्थगच्छस्य १२ फलं}
{२९४, वृद्धिः\renewcommand{\thefootnote}{६}\footnote{फले १९४ वृद्धिः~।}\,। अत्र हीनपदवृद्धिः ३१८~। अत्र सर्वपदैक्यं ३१,
प्रथमपदस्येवास्य फलं ४६५}
{हीनपदवृद्ध्या (३१८) द्विगुणितया ६३६ ऊनं न पततीति विपरीतशोधनं विधाय
जातं\renewcommand{\thefootnote}{७}\footnote{विभाय नातं~।}  ऋणं}
{१७१$+$, चयहतं १०२६$+$, पदान्तरेण\renewcommand{\thefootnote}{८}\footnote{१०२६ पक्षान्तरेण~।}  १ आदिगुणेन ९ युते\renewcommand{\thefootnote}{९}\footnote{युत~।} {\qt 'तयोर्योगे वियोगः स्यात्'} इति}
{जातं १०१७$+$~।}
\vspace{3mm}

{इह पदवैषम्येऽपि आद्युत्तरयोः साम्यात् कर्मलाघवार्थं रूपेण कल्पनया
\hyperref[99]{'रूपादिचयेने'}\renewcommand{\thefootnote}{१०}\footnote{चयेनो*~।}त्यादि कृत्वा \hyperref[99]{'चयहतमादिगुणपदान्तरे'}ति कृतं, सर्वपदैक्यधने
जितधनसङ्ख्या तच्छोध्यसङ्ख्या च तत्तुल्यास्तीति\renewcommand{\thefootnote}{११}\footnote{तच्छोभ्यसंख्या च तत्राल्पास्तीति~।}  हीनपदवृद्ध्या द्विगुणितयोनीकरणं ये हि
द्वाभ्यां तुल्याः\renewcommand{\thefootnote}{१२}\footnote{त्त्रल्पाः~।}  पाता}
{जितास्ते परस्परं निवार्याः इति मतिबलेन व्यज्यते\renewcommand{\thefootnote}{१३}\footnote{इति सत्ये वले पातो व्यज्यते~।}\,।  प्राक्पदे
वृद्धिः साधिता २१ सादियुता\renewcommand{\thefootnote}{१४}\footnote{साधियुता~।}}
{सङ्कलितं भवति रूपादिचयेन २८, वस्तुतस्तु सा षड्गुणा सती वृद्धिः\renewcommand{\thefootnote}{१५}\footnote{षड्गुणवृद्धिसती~।} 
यदर्थं चयाहतेति}
{करिष्यते १२६ सादिश्चैषा सङ्कलितमितिवत् नवकेन सप्तहतेन ६३ योगाद्भवति १८९}
{यदर्थमादिगुणेत्यादि\renewcommand{\thefootnote}{१६}\footnote{*गुणोत्पादि~।}  करिष्यते~। तथा च स्पष्टेन यथा पदं\renewcommand{\thefootnote}{१७}\footnote{पदं १~।}  ७,
व्येकं ६, अतोऽर्धं ३, चयेन ६}
{निघ्नं\renewcommand{\thefootnote}{१८}\footnote{घ्नं~।}  १८, सादिः २७, पदसङ्गुणं १८९~। सर्वमुन्नेयमेवम्~।}

\newpage

{अथ द्वितीयहीनपदजयोदाहरणम्\textemdash}

\begin{quote}
    
 {\eg रूपादिद्विकवृद्ध्या पाताः वेदास्त्रयो यमौ दस्रौ~। \\
 द्यूते विजिता द्वाभ्यां परस्परं कथय कस्य जयः~॥~११५~॥}\end{quote}

{यत्र प्रथमः पातो रूपेण द्वितीयस्त्रिभिः तृतीयः पञ्चभिरित्येवमादिद्विद्विवृद्ध्या पाताः}
{जीयन्ते, तत्र प्रथमेन चत्वारः पाताः जिताः द्वितीयेन तदनन्तरं
(त्रयस्तदनु) प्रथमेनैव द्वौ}
{पुनश्च द्वितीयेन द्वौ, तत्र न ज्ञायते तयोः को जयी कियच्च धनं जयतीति~।}
\vspace{3mm}

{न्यासः\textendash }
\vspace{-2mm}

\begin{center}
\begin{tabular}{c}आ \\१\end{tabular}
\begin{tabular}{c}उ\\१\end{tabular}
\begin{tabular}{c}ग\\४\end{tabular}
\begin{tabular}{c}ग\\ ३\end{tabular}
\begin{tabular}{c}ग\\२\end{tabular}
\begin{tabular}{c}ग\\२\end{tabular}
\end{center}
\vspace{-1mm}

{आद्योऽधिकपदः, इतरो हीनपदः~। तत्र प्राक्पदं ४, एकहीनं ३, रूपादिचयेन
तत्फलं ६}
{वृद्धिः~। पूर्वपदैक्ये ४ प्रभवे, रूपे १ प्रचये, द्वितीयगच्छस्य ३ फलं १५,
वृद्धिः~। पूर्वपदैक्ये}
{(७ प्रभवे), रूपे १ प्रचये, तृतीयगच्छस्य २ फलं १५, वृद्धिः~। पूर्वपदैक्ये
९ प्रभवे, रूपे}
{(१) प्रचये, चतुर्थगच्छस्य २ फलं १९, वृद्धिः~। अत्र हीनपदवृद्धिः ३४~। अत्र
सर्वपदैक्यं ११,}
{पूर्वपदस्येवास्य\renewcommand{\thefootnote}{१}\footnote{पूर्वपदेस्येवास्य~।} फलं यथा\textendash \,प्राक्पदं ११, एकविहीनं १०,
रूपादिचयेन तत्फलं ५५, हीनपदवृद्ध्या ३४ द्विगुणितया ६८ ऊनं १३$+$, चयेन २ हतं\renewcommand{\thefootnote}{२}\footnote{हतं २६~।} २६$+$, पदान्तरेण १
आदिगुणेन १ युतं}
{जातं (२५$+$); तथा च प्रथमस्य जितधनसङ्ख्या १६, द्वितीयस्य ३३, (पुनः)
प्रथमस्य ३२,}
{द्वितीयस्य\renewcommand{\thefootnote}{३}\footnote{द्वितीयस्य ४००~।} ४०, योगौ ४८~। ७३, अन्तरं २५, हीनपद एव\renewcommand{\thefootnote}{४}\footnote{यव~।} जयतीति~।}
\vspace{3mm}

{(अथैकस्य राशेः सङ्कलितं वर्गो घनश्च) संयुता (वा) जिज्ञास्यन्ते तस्य
पृथक् पृथक्}
{करणसूत्रैस्तानानीय\renewcommand{\thefootnote}{५}\footnote{*तान् पानीय~।} मिश्रीकृत्य च ज्ञातुं शक्यं यद्यपि\renewcommand{\thefootnote}{६}\footnote{मिश्रीकृत पच ज्ञात्वं यद्यपि~।} तथापि लाघवार्थं किञ्चिन्न्यूनमार्यामाह\textemdash}

\phantomsection \label{102.1}
\begin{quote}
 {\bs द्विगुणितसैकपदघ्नं\renewcommand{\thefootnote}{७}\footnote{द्विगुणितसेकपदं नैक~।} सैकपदं प(द)दलाहतं भवति~। \\
 सङ्कलितकृतिघनैक्यं\renewcommand{\thefootnote}{८}\footnote{*कृतिमनै*~।}}
\end{quote}

{यस्य राशेः सङ्कलितवर्ग(घन)योगो जिज्ञासेत तं द्विगुणयेत्,
 तावत्तत्तो रूपाधिक\renewcommand{\thefootnote}{९}\footnote{रूपादिकं~।}}
{कुर्वीत, अन्यतश्च स एव राशिरेकाधिकः,\renewcommand{\thefootnote}{१०}\footnote{राशित एवैकादयः~।} ततस्तथा कृते च तौ मिथो\renewcommand{\thefootnote}{११}\footnote{मिश्रौ~।}
हन्यात्, अनन्तरं}
{तस्यैव मूलराशेरर्धेन\renewcommand{\thefootnote}{१२}\footnote{*राशेर्येन~।} गुणयेत्, लभ्यते यत्तु तत्\renewcommand{\thefootnote}{१३}\footnote{यत्नुत्~।}
सङ्कलितकृतिघनैक्यं भवेत्~।}
\vspace{3mm}

{उदाहरणम्\textemdash}

\begin{quote}
    
{\eg सङ्कलितकृतिघनैक्यं पञ्चानां किं भवेत् समाचक्ष्व~।}\end{quote}

\newpage

{स्पष्टम्~।}
\vspace{3mm}

{तत्र कर्म\textendash \,पदं त्रिधा ५~। ५~। ५, एकत्र द्विगुणितं सैकमन्यत्र च (सैकं)
इतरत्र दलितं}
{यथा ११~। ६~। \begin{tabular}{c}५\\२\end{tabular} एषां वधः\renewcommand{\thefootnote}{१}\footnote{वधः १९५~।}  १६५, एतत् सङ्कलितकृतिघनैक्यं
पञ्चानाम्, यतः सङ्कलितं}
{१५ कृतिः २५ घनः १२५ योगः सैव~।}
\vspace{3mm}

{कर्मलाघवे युक्तिः\renewcommand{\thefootnote}{२}\footnote{युक्तः~।}\textendash \,तेषां पदं ५, सैकमिति असंयोज्य
रज्जुनीताववस्थापने\renewcommand{\thefootnote}{३}\footnote{रज्जनीता*~।}  ५~।}
{१, एकराशिवच्चैष रज्जुराशिरिष्यते  यथाहुः {\qt 'रज्जुः सा चैकराशिवत्'} इति एवं सति}
{पददलेनाहतमिति\renewcommand{\thefootnote}{४}\footnote{पदेनाद्यांत्यमिति~।} गुणाक्षरोपलक्षितेन स्थापनं यथा ५ १ गु\begin{tabular}{c}५\\२ \end{tabular}, एतत्
सङ्कलितम्\renewcommand{\thefootnote}{५}\footnote{*लक्षिते पदे स्थानं यथापक्षे १५ भृगुपतत् सङ्कलित~।};}
{वर्गस्तु ५ गु ५, (घनः ५ गु ५ गु ५)~। घनः\renewcommand{\thefootnote}{६}\footnote{इनः घनः~।}  सदृशत्रिराशिवधः, वर्गः
सदृशद्विराशिवधः, सङ्कलितं}
{समूलवर्गार्धम्, अतः सूत्रे द्विपदग्रहणं\renewcommand{\thefootnote}{७}\footnote{समूलवर्गात् यतः
सूत्रे त्रिः पदग्रहणे~।}  घनक्रियाभिप्रायेण तथा
हि\renewcommand{\thefootnote}{८}\footnote{द्वि~।}  \hyperref[102.1]{'द्विगुणितसैकपदघ्नं\renewcommand{\thefootnote}{९}\footnote{*पदंन्नं~।} सैकपदम्'} इत्येतावता वक्ष्यमाणपदवधनिष्पत्स्यमानस्वरूपस्य
घनस्याङ्कुरीभावः,\renewcommand{\thefootnote}{१०}\footnote{घनस्यामुरी*~।} (घनो) मूलफलहतो}
{हि वर्गः, स चैकस्य पदस्य पदेन घातादुपांशुनिष्पन्नः, सैकपदपदवधाद्धि
वर्गः समूलो\renewcommand{\thefootnote}{११}\footnote{घाताद्वपां
श्रुतिनिष्पनः सैकपदपदवधांद्विवर्गसम्मूलो~।}  जायते,}
{अनिष्पत्स्यमानवर्गस्याङ्कुरोद्गमोऽयं यदेतया भङ्ग्या वर्गोपरि
तन्मूलजन्म; यदा हि समूलवर्गराशिभूयोऽपि मूलेन ताडयेत् तदा वर्गो घनो\renewcommand{\thefootnote}{१२}\footnote{वर्गी घनी~।}  भवति, समूलवर्गः
३० एषोऽपि\renewcommand{\thefootnote}{१३}\footnote{एवोपि~।}; पञ्चगुणो}
{घनवर्गयोगतामापद्यते\renewcommand{\thefootnote}{१४}\footnote{*तानापं*~।}  ५५०, घनो हि पञ्चानामयं १२५ वर्गश्च (२५)
यथादर्शितं चानयोर्योगः~।}
{तथा सैकपदघ्नं पदमिति कर्मणा जायमानो वर्गो मूलान्वितो
घनवर्गयोरङ्कुराय\renewcommand{\thefootnote}{१५}\footnote{*राया~।}  बोद्धव्यः,}
{पदेन यत् गुणनं तत् सङ्कलिताङ्कुरणाय~। एवं हि$^{\scriptsize{\hbox{{\color{blue}८}}}}$ समूलो वर्गः
सैकमूलाधिको भवति ३६, अत्र}
{घनाङ्कुरः २५ वर्गमूलं ५ सङ्कलितभित्तिः\renewcommand{\thefootnote}{१६}\footnote{*नाकुरः ३५ वर्गमूलः सङ्कलितमिति ९~।} ६ \hyperref[102.1]{'सैकपदे'}ति
सङ्कलितकर्मप्रक्रमात्~। एवं प्राप्ते}
{पदहतमिति कर्मणा वर्गो घनतां\renewcommand{\thefootnote}{१७}\footnote{वर्गतो~।}  मूलं वर्गतां सैकं पदं
द्विगुणितसङ्कलिततां\renewcommand{\thefootnote}{१८}\footnote{*संकुलि*~।}  प्राप्तं, तत्र तत्सूत्रे\renewcommand{\thefootnote}{१९}\footnote{तमसूत्रि~।}}
{\hyperref[14.1]{'सैकपदाहतपददलम्'} इति सङ्कलिते सम्पद्यमाने सहकर्मणा
घनवर्गराश्योरनिष्टदलनापत्तेरिति}
{प्रागेव पदस्य द्विगुणनं, समेन गुणो\renewcommand{\thefootnote}{२०}\footnote{गुणं~।}  भक्तश्च रूप एवावतिष्ठते इति~। 
स्वगुणेन\renewcommand{\thefootnote}{२१}\footnote{अगुणेन~।}  सैकपदे (न)}
{पददलाहतमितीयति कर्मणि फलं स्यात् ९०, यत्र समग्रं सङ्कलितं १५ वर्गार्धं\renewcommand{\thefootnote}{२२}\footnote{वर्गावरी~।}\begin{tabular}{c}२५\\२ \end{tabular}घनार्धं\begin{tabular}{c}१२५\\२ \end{tabular}सर्वेषां च संयोगः ९०; अदलेन तु सैकपदघ्नं सैकपदं\renewcommand{\thefootnote}{२३}\footnote{सेकपदघ्नं सेक*~।} पदाहतमितीयति
कर्मणि फलं स्यात्}
{१८०, यत्र समग्रवर्गः २५ समग्रघनः\renewcommand{\thefootnote}{२४}\footnote{समग्रंघनः~।} १२५ सङ्कलितं तत् द्विगुणं ३०
सर्वेषां च योगः १८०~।}
{अदलेन तत् द्विगुणेन च चतुर्थं फलं स्यात् ३६०, यत्र द्विगुणो\renewcommand{\thefootnote}{२५}\footnote{द्विगुणेन~।} 
वर्गः ५० द्विगुणो घनः २५०}
{चतुर्गुणं सङ्कलितं ६० सर्वेषां च योगः\renewcommand{\thefootnote}{२६}\footnote{योगः ३६~।}  ३६०, तस्मादिष्टफलसिद्धये
यथाकरणमेव साधीयः~।}
{ननु चेदमत्र प्रतिपाद्यं यदुतान्ते क्रियमाणं दलनं सर्वमनिष्टस्थितिः,
वर्गघनयोरनिष्टनिवृत्तये\renewcommand{\thefootnote}{२७}\footnote{*स्थितिवर्गञ्चामयो*~।}} प्रागेव द्विगुणनं क्रियते~। 

\newpage

\noindent{न तु सर्वादौ क्रियमाणेन द्विगुणेन वर्गधनाविव
सङ्कलितप्रकृतिरपि}
{व्याप्येत येन चतुर्गुणं स्यात्, तावति समये स्वतो द्विगुणात्मकत्वे
पुनर्द्विगुणेन तथाभावात्,}
{ततश्च सङ्कलितसिद्धये चत्वारो भाजकः कर्तव्यः\renewcommand{\thefootnote}{१}\footnote{भाजयितव्याः~।} पुनरपि वर्गघनौ दलीकृतावेव
स्यातामिति~।}
{प्रागेव चतुर्भिर्गुणयितव्यं\renewcommand{\thefootnote}{२}\footnote{*गुण*~।} सङ्कलितमिति प्रकृतिमितश्चाष्टगुणो
भवन्नष्टाभिर्विभज्य इत्यादि}
{युगशतेऽपि\renewcommand{\thefootnote}{३}\footnote{र्युग*~।} न कर्म सिद्धम्~। उच्यते, द्विगुणितं पदमिति भविता तावत्
धनं\renewcommand{\thefootnote}{४}\footnote{घनमूल द्विगुणं १०~।},}
{सैकपदमिति पदेन ताडयिष्यमाणं रूपं क्षेपः वर्गस्य~। मूलतां
प्रतिपत्स्यमानम् अपि द्विगुणीकृतम्~।}
{नापि द्विगुणे राशौ योजनाद्द्विगुणीकृतं, यस्मादस्य सङ्क्षेपस्य राशेः पदस्य
च सैकस्य वधे}
{विधीयमाने स्थानविभागप्रत्युत्पन्नेन तावदिदं भवति यत्र न्यासः १०~। १
एकराशिः, ५~। १}
{द्वितीयः, यथोक्तकरणरीत्या फलं\renewcommand{\thefootnote}{५}\footnote{फलं ४०~। १०~। ५~। १~।} ५०~। १०~। ५~। १, किन्तु द्विगुणेन पदेन
पदवधो}
{घनद्वैगुण्याय रूपं वर्गद्वैगुण्याय रूपेण वधः सङ्कलितद्वैगुण्यायेति\renewcommand{\thefootnote}{६}\footnote{रूपे वयं संकलितायेति~।}
नात्र कश्चिद्दोषः, रूपवधजातौ}
{ह्यत्र राशी पञ्चकसङ्कलितसिद्धये भविष्यत इति~।}
\vspace{3mm}

{अथ एकाद्युत्तरेण यावतां राशीनां वर्गसंयोगो लघुकर्मणा
ज्ञातुमिष्यते\renewcommand{\thefootnote}{७}\footnote{ज्ञातमि*} तत्सङ्ख्यापदमालम्ब्य गणितार्थमार्यान्त्यपदमाह\textemdash}

\phantomsection \label{102}
\begin{quote}
    
\hspace{20mm} {\bs  तत् त्रिहृतं वर्गसङ्कलितम्~॥~१०२~॥}\end{quote}

{तदित्यनेनानन्तरसूत्रसिद्धं \;सङ्कलितकृतिघनैक्यं \;परामृश्यते, \,तत् \;त्रिभिर्भक्तं \;वर्गसङ्कलितं}
{भवति~।}
\vspace{3mm}

{उदाहरणम्\textendash}

\begin{quote}
    
 {\eg  एकादिचयपदानां कृतिसङ्कलितं च यदि वेत्सि~॥~११६~॥}\end{quote}

{अत्र पूर्वप्रश्नात्पञ्चानामित्यपेक्ष्यते, एकस्य द्वयोस्त्रयाणां
चतुर्णां पञ्चानां च पृथक्पृथग्वर्गाः १~। ४~। ९~। १६~। २५~।एतस्य, एषां युतिः ५५~। लघुकर्मेदम्\textendash \,तत् सङ्कलितकृतिघनैक्यं १६५, त्रिभिर्हृतं ५५, इदं तत्~।}
\vspace{3mm}

{अथ तथैव घनयोगार्थमार्यापूर्वार्धमाह\renewcommand{\thefootnote}{८}\footnote{घनयोर्योगा*~।}\textendash}

\begin{quote}
    
{\bs सपदपदवर्गतोऽर्धं घनसङ्कलितं स्वसङ्गुणं भवति~।}\end{quote}

{पदवर्गस्य पदेन युतस्यार्धं स्वसङ्गुणं घनसङ्कलितं भवति~। सपदग्रहणस्य
प्राङ्निर्देशे}
{पदवर्गीकरणोत्तरकाले तत्र योगो, द्विगुणपदेत्यकरणात्~।}

\newpage

{उदाहरणम्\textemdash}

\begin{quote}
 {\eg एकादिचयपदानां घनसङ्कलितं सखे कियत् भवति~। \\
 आशु दशानां प्रकथय}
\end{quote}

{एकादिचयेन दशानां राशीनां पृथक् पृथग्घनाः\textendash \,१~। ८~।\renewcommand{\thefootnote}{१}\footnote{१~। २८~।} २७~। ६४~। १२५~।}
{२१६~। ३४३~। ५१२~। ७२९~। १०००~। एषां युतिः\renewcommand{\thefootnote}{२}\footnote{युतिः ३०~।२५~।}  ३०२५~। एतस्या
लघुकर्मेदम्\textendash \,पदस्य १०}
{वर्गः\renewcommand{\thefootnote}{३}\footnote{वर्धः १..~।}  १००, सपदं\renewcommand{\thefootnote}{४}\footnote{सपदं १०~।}  ११०, अतोऽर्धं ५५, एकादिचयेन
श्रेढीफलमित्यर्थः\renewcommand{\thefootnote}{५}\footnote{श्रोशफल*~।}, अस्य वर्गः\renewcommand{\thefootnote}{६}\footnote{वर्गः ३०~।}  ३०२५~।}
\vspace{3mm}

{अथैकादिचयेन तावतां पदानां यानि यथास्वं पृथक् पृथक् सङ्कलितानि तेषां
लघुकर्मणा\renewcommand{\thefootnote}{७}\footnote{कर्मणां~।}  संयोग ज्ञापयितुमार्या(परार्ध)माह\textemdash}

\begin{quote}
    
 {\bs द्वियुतपदेन च गुणितं त्रिहृतं सङ्कलितसङ्कलितम्~॥~१०३~॥}\end{quote}

{सपदपदवर्गतोऽर्धमित्यन्तमपेक्ष्यते, इह गुण्यसाकाङ्क्षत्वात्
पूर्वप्रक्रान्तार्थविशेषद्योतनात्सजातीयस्वगुणवाचनात्~। ततोऽयमर्थः एकादिचयेन (य)च्छ्रेढीफलं तत् द्वियुतेन पदेन गुणितं}
{त्रिभिः भक्तं सङ्कलितसङ्कलितं भवति~।}
\vspace{3mm}

{उदाहरणम्\textemdash}

\begin{quote}
    
\hspace{25mm} {\eg तथैव सङ्कलितसङ्कलितम्~॥~११७~॥}\end{quote}

{एकादिचयेन दशानां पदानां प्रत्येकसङ्कलितसंयोगं प्रकथय~।}
\vspace{3mm}

{तत्र सङ्कलितानां पृथक् पृथक् न्यासः\textendash}
\vspace{-1mm}

\begin{center}

{१~। ३~। ६~। १०~। १५~। २१~। २८~। ३६~। ४५~। ५५~। }
\end{center}
\vspace{-1mm}

\noindent{एषां युतिः\renewcommand{\thefootnote}{८}\footnote{युतिः २२~।}  २२०~। एतस्या लघुकर्मेदम्\textendash \,एकादिचयेन श्रेढीफलं दशानां
५५, एतत् द्वियुतपदेन}
{१२ गुणितं ६६०, त्रिहृतं २२०, एतत् तत्~।}
\vspace{3mm}

{करणसूत्रम्\renewcommand{\thefootnote}{९}\footnote{कारण*~।}\textendash}

\begin{quote}
    
{\bs सैकपदवर्गताडितपदं द्विकोपेतपदगुणं भवति~। \\
 सङ्कलितकृतिघनानां सङ्कलितैक्यं\renewcommand{\thefootnote}{१०}\footnote{*लितं चैक्यं~।}  चतुष्कहृतम्~॥~१०४~॥}\end{quote}

{त्रिधा पदं स्थापयेत् एकत्र सैकं वर्गीकृतम् अन्यत्र शुद्धमेव अपरत्र
द्विकोपेतं, ततस्तेषां}
{घातश्चतुर्भिर्भक्तव्यः,\renewcommand{\thefootnote}{११}\footnote{चातश्च*~।}  तेन सङ्कलितसङ्कलितस्य वर्गसङ्कलितस्य
(घन)सङ्कलितस्य च योगो}
{भवति कर्मलाघवेन~।}

\newpage

{उदाहरणम्\textemdash}

\begin{quote}
    
 {\eg सङ्कलितकृतिघनानां सङ्कलितसमासमानं मे कथय\renewcommand{\thefootnote}{१}\footnote{*समासमध्यं~।}\,।\\
 षण्णां सखे पदानां गणयित्वा यदि विजानासि~॥~११८~॥}\end{quote}

{षष्णां सङ्कलितसङ्कलितस्य वर्गसङ्कलितस्य (घनसङ्कलितस्य) च यः समासस्तं}
{कथय~।}
\vspace{3mm}

{पदं ६, सैकं ७, वर्गः ४९, ताडितपदं\renewcommand{\thefootnote}{२}\footnote{ताडितपदं २६~।} २९४, द्विकोपेतेन पदेन गुणितं\renewcommand{\thefootnote}{३}\footnote{गुणितं २३४२~।}   २३५२,}
{चतुष्कहृतं (५८८), एतत्\renewcommand{\thefootnote}{४}\footnote{पतत~।}   (तत्)~।}
\vspace{3mm}

{एकाद्येकोत्तरेण पदानां वर्गयोगादिकमुक्त्वेष्टाद्युत्तरेण तत्
प्रतिपादयितुमार्यामाह\textemdash}

\phantomsection \label{105}
\begin{quote}
    
{\bs द्विगुणितचयेन गणितं मुखसङ्गुणितं निरेकगच्छस्य~। \\
 कृतिसङ्कलितेन युतं\renewcommand{\thefootnote}{५}\footnote{च युतं~।} चयकृतिगुणितेन वर्गयुतिः~॥~१०५~॥}\end{quote}

{इष्टात्पदादिष्टादिनेष्टेनोत्तरेण द्विगुणेन श्रेढीफलं\renewcommand{\thefootnote}{६}\footnote{श्रेणीफलं~।} साधयेत्, तदनु
तन्मुखेन सङ्गुणितं}
{सदिष्टपदस्य विरूपस्य सम्बन्धिना वर्गसङ्कलितेनोत्तरवर्गगुणितेन संयोजयेत्, एवमिष्टाद्युत्तरगच्छैर्वर्गसङ्कलितं भवति~।}
\vspace{3mm}

{उदाहरणम्\textemdash}

\begin{quote}
    
{\eg द्विकादि(त्रिक)वृद्धीनां पदानां कृतयः क्रमात्~। \\
 षण्णां गणितवित् तासां समासो मम कथ्यताम्~॥~११९~॥}\end{quote}

{आद्यं पदं द्वौ २, द्वितीयं पञ्च ५ तृतीयमष्टौ\renewcommand{\thefootnote}{७}\footnote{पदं द्वौ द्वितीयमष्टौ~।} ८ चतुर्थमेकादश ११
पञ्चमं (चतुर्दश)}
{१४ षष्ठं सप्तदश\renewcommand{\thefootnote}
{८}\footnote{सप्तादश~।} १७ एषां द्विकादित्रिकवृद्ध्या स्थितानां षण्णां
पदानामिमास्तावत् क्रमात्}
{पृथक् पृथक् कृतयो भवन्ति\renewcommand{\thefootnote}
{९}\footnote{भवन्ति ४~। २४~। ६४~। १२१~।  
१९६~। १८९~।} ४~। २५~। ६४~। १२१~। १९६~। २८९, आसां युतिः
६९९~।}
{एतस्या लघुकर्मेदम्\renewcommand{\thefootnote}
{१०}\footnote{लक्तक*~।}\,। द्विगुणितचयेन न्यासः\textemdash}
\vspace{3mm}
 
\hspace{30mm} \begin{tabular}{|c|}आ \\२\\\hline \end{tabular}\begin{tabular}{c|}उ \\६\\\hline \end{tabular}\begin{tabular}{c|}ग\\६\\\hline \end{tabular}
\vspace{3mm}

\noindent {\hyperref[85]{'व्येकपदे'}त्यादिना गणितं १०२, मुखसङ्गुणितं २०४, अथास्यैव गच्छस्य (६
निरेकस्य ५)}
{कृतिसङ्कलितेन \hyperref[102]{'तत् त्रिहृतं वर्गसङ्कलितम्'} इतिसिद्धेन ५५ चयकृत्या ९
गुणिते(न) ४९५ युतं}
{६९९, एषा सा~।}

\newpage

{इदानीमिष्टाद्युत्तरेण सङ्कलितसङ्कलितायार्यामाह\renewcommand{\thefootnote}{१}\footnote{सकलितामार्यां*~।}\textemdash

\begin{quote}
    
{\bs  इष्टाद्युत्तरगच्छैः पूर्ववदानीय वर्गसङ्कलितम्~। \\
 श्रेढीगणितेन युतं दलितं सङ्कलितसङ्कलितम्~॥~१०६~॥}\end{quote}

{\hyperref[105]{'द्विगुणितचयेने'}त्यादिना\renewcommand{\thefootnote}{२}\footnote{*चयेनोत्पादिना~।} वर्गसङ्कलितमानयेत् तदनु
इष्टाद्युत्तरगच्छैः श्रेढीफलमानीय\renewcommand{\thefootnote}{३}\footnote{श्रेणीफल*~।}
{योजयेत्~। सा युतिर्दलिता सतीदं सङ्कलितसङ्कलितं भवति~।}
\vspace{3mm}

{उदाहरणम्\textendash}

\begin{quote}
    
{\eg  त्र्यादिपञ्चकवृद्धीनां\renewcommand{\thefootnote}{४}\footnote{*पंचकवृद्ध्यादि~।} पदानां गणकोत्तम~।\\
 यानि सङ्कलितानि स्युः षण्णां तद्योगमुच्यताम्\renewcommand{\thefootnote}{५}\footnote{यति संकलिताति स्युष्षस्मां तद्योग उच्य*~।}\,॥~१२०~॥}\end{quote}

{आद्यं पदं त्रयः ३ द्वितीयमष्टौ ८ तृतीयं त्रयोदश १३ चतुर्थमष्टादश १८
पञ्चमं}
{त्रयोविंशतिः २३ षष्ठमष्टाविंशतिः\renewcommand{\thefootnote}{६}\footnote{त्रयोर्विशतिः २३
षष्ठस्याष्टा*~।} २८, एषां
त्र्यादिपञ्चकवृद्ध्या\renewcommand{\thefootnote}{७}\footnote{*वृद्ध~।} स्थितानां पदानां पृथक्}
{पृथक् सङ्कलितानीमानि ६~। ३६~। ९१~। १७१~। २७६~। ४०६, एषां युति ९८६~। 
अस्या}
{लघुकर्मेदम्~। पूर्ववद्वर्गसङ्कलितानयनार्थं द्विगुणेन चयेन न्यासः\textemdash}
\vspace{-1mm}

 \begin{center}
 
\begin{tabular}{|c|}आ \\३\\\hline \end{tabular}\begin{tabular}{c|}उ \\१०\\\hline \end{tabular}\begin{tabular}{c|}ग\\६\\\hline \end{tabular}
\end{center}
\vspace{-1mm}

\noindent {अतः \hyperref[105]{'द्विगुणितचयेने'}त्यादि(ना)\renewcommand{\thefootnote}{८}\footnote{*चयेनोत्या*~।} वर्गसङ्कलितं १८७९,
एतच्छ्रेढीगणितेनेष्टाद्युत्तरप्रभवेन\renewcommand{\thefootnote}{९}\footnote{पतेच्छ्रीढगं*~।}
{९३ युतं १९७२, दलितं ९८६, एतत् तत्~।}
\vspace{3mm}

{इष्टादिचयेन जातानां पदानां (ये) घनास्तेषां\renewcommand{\thefootnote}{१०}\footnote{घनयोगास्तेषां~।} योगानयनार्थमाह\textemdash}

\begin{quote}
    
{\bs  श्रेढीफलस्य वर्गे प्रचयहते (चय)विहीनवदनगुणम्~।\\
 मुखफलवधं\renewcommand{\thefootnote}{११}\footnote{*फलबन्धः~।}  निदध्यादिष्टादिचयेन घनयोगः\renewcommand{\thefootnote}{१२}\footnote{च घनयोगः~।}\,॥~१०७~॥}\end{quote}

{इष्टाद्युत्तरगच्छैः श्रेढीफलमानीय वर्गयित्वा ताडयित्वोत्तरेण
चयहीनमुखगुणमादिफलयोर्वधं\renewcommand{\thefootnote}{१३}\footnote{*योर्बन्धं~।} तत्र योजयेत्, एवमिष्टादिचयेन स्थितानां पदानां ये
घनास्तेषां योगो भवति~।}
\vspace{3mm}

{उदाहरणम्\textendash}

\begin{quote}
    
{\eg पञ्चादिद्विकवृद्धीनां पदानां ये क्रमात् घनाः\renewcommand{\thefootnote}{१४}\footnote{घना~।}\,।\\
 चतुर्णां तत्समासेन गणयित्वा निगद्यताम्~॥~१२१~॥} \end{quote}

{आद्यं पदं पञ्च ५ द्वितीयं सप्त ७ तृतीयं नव ९ चतुर्थमेकादश ११, एषा
पञ्चादिद्विकवृद्धीनां पदानां पृथक् पृथक् इमे घनाः }

\newpage

\noindent{१२५~। ३४३~।\renewcommand{\thefootnote}{१}\footnote{२४३~।} ७२९~। १३३१,
एषां युतिः २५२८~। एतस्या लघुकर्मेदं तदर्थो न्यासः\textemdash}
\vspace{-1mm}

\begin{center}
\begin{tabular}{r} अ\\(५\\\end{tabular}
\begin{tabular}{r}उ \\२\\ \end{tabular}
\begin{tabular}{l}ग\\ ४)\\ \end{tabular}
\end{center}
\vspace{-1mm}
 
\noindent {अतः श्रेढीफलं ३२, अस्य वर्गः १०२४, प्रचयेन २ हतः\renewcommand{\thefootnote}{२}\footnote{हतं~।} २०४८; मुखं\renewcommand{\thefootnote}{३}\footnote{मुखं ४~।} ५
फलं ३२ वधः}
{१६०, चय\textendash \,२\textendash \,विहीनवदन\renewcommand{\thefootnote}{४}\footnote{चयः २ विहीनवदनम्~।}\textendash \,३\textendash \,गुणः ४८०, एतेन युतः\renewcommand{\thefootnote}{५}\footnote{युतिः
२४२८~।} २५२८, एष सः~।}
\vspace{3mm}

{अनेन \hyperref[105]{'द्विगुणि(त)चयेने'}त्यादिनेष्टाद्युत्तरविषयेन वर्गयोगाद्यानयनार्थे प्रकरणे रूपाद्युत्तरमपि तदानेतुं शक्यते इति तस्य \hyperref[102]{'तत् त्रिहृतं वर्गसङ्कलितम्'} इत्यादि कर्म लघुकरणार्थम्~।}
\vspace{3mm}

\begin{center}
\textbf{इति व्याख्यातः श्रेढीव्यवहारो द्वितीयः~।} \end{center}

\afterpage{\fancyhead[CO] {\s}}
\afterpage{\fancyhead[CE] {\s}}

\newpage

\phantomsection \label{kshe}
\begin{center}
\textbf{\large अथ क्षेत्रव्यवहारो व्याख्यायते}\end{center}
\vspace{2mm}

{इष्टावधिकः देशैकदेशः क्षेत्रं, तदाश्रयो व्यवहारो
अर्थक्रियासाधनायोद्दिष्टः\renewcommand{\thefootnote}{१}\footnote{*साधनीयेष्टो~।}\,। तत्रेष्टेन}
{परिच्छेदेन नियम्यमानदेशैकदेशे त्र्यश्रचतुरश्र(पञ्चाश्र)वृत्तधनुरादिभेदास्तत्तत्पदार्थस्वरूपसन्निवेशवशात्\renewcommand{\thefootnote}{२}\footnote{*पदार्थसरूप*~।}  प्रादुर्भवन्ति, तथा च तथा व्यपदेशं लभन्ते यथा
सन्निवेशं क्षेत्रं धनुः कार्मुकञ्चापमित्यादि व्यप-दिश्यते~। एवं करिदन्तादिक्षेत्रेषु\renewcommand{\thefootnote}{३}\footnote{पूर्वं करिन्दतादिषु~।}  बोध्यमिति~। 
एतस्मिन् क्षेत्रेण तत्प्रभेदप्रतिभेदविधानाशक्तेः\renewcommand{\thefootnote}{४}\footnote{क्षेत्रेनत प्रभेद*~।}  जातिमात्रमाश्रित्य लक्षणं प्रवर्त्यम्~। तत्र दश
क्षेत्रजातयो भवन्ति, समत्रिभुजं,}
{द्विसम(त्रि)भुजं, विषमत्रिभुजं, समचतुरश्रं, त्रिसमचतुरश्रं,
द्विसमचतुरश्रं, विषमचतुरश्रं,}
{द्विद्विसमचतुरश्रम् आय-तचतुरश्रं, (वृत्तं), धनुरिति~। एषामेव गणितेन
सम्यक् फलोपलब्धिस्तान्येवोद्दिश्य लक्षयितव्यानि नेतराणि, तत्तत्कल्पनया तेषां तेषां तत्करणेन
तत्फलानुसरणं, कल्पना}
{तु समस्तस्यैव समस्तेन यथा गजदन्ताकृतेस्त्रिभुजेन,\renewcommand{\thefootnote}{५}\footnote{*कृते त्रिभु*~।}  अवयववशेन वा यथा
बालेन्दोस्त्रिभुजद्वयेन, तदेवं मुख्यामुख्यकल्पनीयेषु क्षेत्रेषु~। अथैतानि\renewcommand{\thefootnote}{६}\footnote{*यथै*~।} वस्तूनि
व्यवहारं प्रयोजयन्ति यथा क्षेत्रफलं}
{भुजः, भूमिः, मुखं, कोटिः, कर्णः, लम्बः, अवधा\renewcommand{\thefootnote}{७}\footnote{अवधि~।}, हृदयं, परिधिः,
व्यासः, ज्या, शरश्चापमित्यादि\renewcommand{\thefootnote}{८}\footnote{शिर*~।}\,। तत्र त्र्यश्रादिक्षेत्रे\renewcommand{\thefootnote}{९}\footnote{त्यश्रादिक्षेत्रम्~।} तस्य
सर्वाङ्गीणमङ्गुलहस्तादिमानं फलं क्षेत्रफलं, त्र्यश्रचतुरश्रादिसन्निवेशजननानि सूत्राणि भुजा, (यत्र) लम्बपातः स भूमिः, यस्माच्च
तत्तन्मुखं,}
{भुज एवोच्छ्रितः\renewcommand{\thefootnote}{१०}\footnote{यवो*~।}  कोटिः, तन्मूलाश्रितस्तिर्यगवस्थितो भुज एव तदा
भुजः\renewcommand{\thefootnote}{११}\footnote{*वस्थितो भुज एव तदा स एव च भुजः~।}, कोट्यग्रात्}
{आभुजाग्रं प्रसृतं (सूत्रं) तत्र कर्णः\renewcommand{\thefootnote}{१२}\footnote{करणः~।}, तत्र चतुरश्रे कोणात्कोणं गतं
सूत्रं कर्णः, स नास्ति}
{त्र्यश्रवृत्तचापेषु, उपरिष्टात्प्रान्ता(द)वलम्बितगुरुद्रव्यसूत्रभूमिसम्पातावधि\renewcommand{\thefootnote}{१३}\footnote{*द्रव्यसूप्रपातभूमिसंपातो~।}  लम्बः, लम्बविनिपातविभक्ताया भुवो भागाववधे\renewcommand{\thefootnote}{१४}\footnote{भुवोऽवधि~।}, विषमचतुरश्रपञ्चाश्रादेः\renewcommand{\thefootnote}{१५}\footnote{*पंचश्रौदेः~।} 
कोणस्पृग्वृत्तव्यासार्धं\renewcommand{\thefootnote}{१६}\footnote{*व्याःहसार्धं~।}  हृदयं,}
{समग्रशरीरपरिवेष्टनं परिधिः वृत्तवलयोऽतिप्रचुरः, वृत्तस्य परिधिमध्यमानं
व्यासः, धनुराकृतौ}
{क्षेत्रे तु काष्ठरूपस्य भुजस्य प्रान्तद्वयस्पृक्सूत्रं ज्या, घनता च शरः,
एवं तद्वदाकृतियोगात् धनुः}
{यथालोकम्, एषां चान्योन्योपायेनाभ्युपेयभाव इति~।\renewcommand{\thefootnote}{१७}\footnote{धनता
धनुराकृतौ क्षेत्र
चतुष्काठरूपस्य भुजस्य प्रान्तद्वयस्पृक्सूत्रं तद्वदाकृतियोगात् ज्या एवं
धनुः शरे यथा
लोकं एषां चान्योन्यापायेन भ्युपायभाव इति~।}}
\vspace{3mm}

{इदानीम् एवंविधेऽस्मिन् क्षेत्रव्यवहारे
क्षेत्रसम्भवासम्भवक्षेत्रवस्तुलाभालाभं\renewcommand{\thefootnote}{१८}\footnote{क्षेत्रव्यवहारक्षेत्रे
संभवाससभागक्षेत्र*~।}  चोपलक्षयितुं कल्पनीयक्षेत्राणि च विभक्तुं स्थूलं सूक्ष्मं फलञ्चात्र\renewcommand{\thefootnote}{१९}\footnote{फलञ्चाह~।} 
सोदाहरणं प्रदर्शयितुमार्यासप्तकमाह\textemdash }

\afterpage{\fancyhead[CO] {\s क्षेत्रव्यवहारः}}
\afterpage{\fancyhead[CE] {\s क्षेत्रव्यवहारः}}

\newpage

\phantomsection \label{112}
\begin{quote}
    
 {\bs  एकस्मात् भुजतोऽपरबाहुयुतिर्नो\renewcommand{\thefootnote}{१}\footnote{एकस्माद्बजतो यरवाद्रयु*~।} समानहीना वा~।\\
 ऋजुगतितो\renewcommand{\thefootnote}{२}\footnote{*गतिते~।}  वक्रगतिर्यस्मादूना न तुल्या\renewcommand{\thefootnote}{३}\footnote{*दूर्वात तल्या~।} (वा)~॥~१०८~॥ \\
 पार्श्वभुजान्तरसंयुतिवधतो\renewcommand{\thefootnote}{४}\footnote{पार्स्सभु*~।}  मुखहीनभूकृतिर्येषाम्~।\\
 क्षेत्राणामभ्यधिका तेषां लम्बावधावाप्तिः\renewcommand{\thefootnote}{५}\footnote{*धिकास्तेषां लम्बावधाव्याप्तिः~।}\,॥~१०९~॥\\
 आयतसमचतुरश्रे द्वित्रिसमभुजे विषमचतुरश्रम्~।\\
 समविषमद्विसमभुजत्र्यश्राण्यथ वृत्तचापे च~॥~११०~॥ \\
 क्षेत्राणि दशैतानि हि\renewcommand{\thefootnote}{६}\footnote{दशैता विहं~।}  फलमेषां साधयेत् स्वकरणे(न)~।\\
 एतत्परिकल्पनयान्येषां गजदन्तनेमिपूर्वाणाम्~॥~१११~॥ \\
 स्थूलफलं\renewcommand{\thefootnote}{७}\footnote{स्थूलं फलं~।}  त्रिचतुर्भुजबाहुप्रतिबाहुयोगदलघातः\renewcommand{\thefootnote}{८}\footnote{त्रिचतुर्भुजे वाङ्गंप्रतिबाहुयोगः द*~।}\,। \\
 अवलम्बपार्श्वभुजयोर्यस्यान्तरमल्पकं\renewcommand{\thefootnote}{९}\footnote{*भुजयोस्तांतर*~।}  तस्य~॥~११२~॥ \\
 अन्येषां क्षेत्राणां दूरभ्रष्टं यथा त्रयोदशके~।\\
 त्र्यश्रस्य भुजद्वितये त्रिगुणाष्टभुवः फलं स्थूलम्~॥~११३~॥ \\
 षड्युतमध्यर्धशतं\renewcommand{\thefootnote}{१०}\footnote{षड्यतमार्ध्यार्ध*~।}  सूक्ष्मं षष्टिः प्रजायते यस्मात्~।\\
 सूक्ष्मफलस्यैवाहं साधनकरणानि\renewcommand{\thefootnote}{११}\footnote{सायणक*~।}  वक्ष्यामि~॥~११४~॥}\end{quote} 

{त्र्यश्रे\renewcommand{\thefootnote}{१२}\footnote{अश्रे~।}  त्रयाणां भुजानां चतुरश्रे चतुर्णां मध्यात् य एको
भुजस्तस्मादन्यभुजयोगः तस्यैकस्य भुजस्य समं\renewcommand{\thefootnote}{१३}\footnote{सुम~।}  नो भवति, कुत एव तत्तस्मादूनः, साम्यमात्रकेणैव
क्षेत्रासम्भवः~। यथा}
{दशहस्तस्य सूत्रस्य सम्भूय दशहस्ताभ्यां सूत्राभ्यां सह संयोगो\renewcommand{\thefootnote}{१४}\footnote{संयोगे~।} 
निरन्तरालो भवति तथा क्षेत्रप्रदर्शितप्रकारे\renewcommand{\thefootnote}{१५}\footnote{*प्रकार~।}  भुजपरिगतेन च क्षेत्रेण भवितव्यं,
गर्भीकृतक्षेत्रेण च भुजपरिगतेन, अन्यथा}
{तु सूत्रमात्रं तत्~। यदि भुजायोगादधिकस्तु\renewcommand{\thefootnote}{१६}\footnote{भुजान्तरयोगादधिकस्य~।}  भुजो भवति तदा तस्य
कौटिल्यं विना\renewcommand{\thefootnote}{१७}\footnote{विता~।}  न क्षेत्रसम्भवः, न च भुजः कुटिल इष्यते (त्रिभु)जादिक्षेत्रेष्विति वस्तुस्थितिः,
सैव तदनुसरणासमर्थानां}
{गुरुणा प्रदर्श्यते\renewcommand{\thefootnote}{१८}\footnote{प्रदिर्श्यते~।}\,। तथा हि यद्येवं पृच्छेद्यथा\textemdash}

\begin{quote}
    
{\qt  विंशतिरष्टौ द्वादश हस्ता (वि)षमत्रिबाहुनि\renewcommand{\thefootnote}{१९}\footnote{हस्ताच्च समत्रिवाङ्गनि~।}  भुजाः~।\\
 विंशतिरष्टौ दश वा क्षेत्रफलं तत्र किं कथय~॥}\end{quote}

{इति, तदा स एवंविधक्षेत्रासम्भवेनैवोत्तरितव्यः\renewcommand{\thefootnote}{२०}\footnote{*नैवांतरि*~।}, आद्ये प्रश्ने
एकस्य\renewcommand{\thefootnote}{२१}\footnote{एकस्ये~।} भुजस्येतरभुजयोगसाम्याच्छेषेऽधिकत्वात्~।}

\newpage

\noindent{गणितमप्येवमेवाह\renewcommand{\thefootnote}{१}\footnote{*मथेवमेवाहं~।}, तथा हि प्रथमे तावत्क्षेत्रे
भुजाः २०, ८, १२, एषां युतिः}
{४०, अस्याः दलं २०, चतुर्धा २०~। २०~। २०~। २०, भुजहीनं ०~। १२~। ८~।}
{२०, वधः ०, अतः पदम् ०,\renewcommand{\thefootnote}{२}\footnote{भुजहीनं १२~। ८~। २१.. द्विधः अत पदम्~। ०~।} एतत् गणितम्~। फलस्य शून्यत्वं
क्षेत्रासम्भवत्वं\renewcommand{\thefootnote}{३}\footnote{क्षेत्रासंभवं~।} विभावयति~।}
{सम्भवतः क्षेत्रस्य निष्परिमाणत्वाभावात् सपरिमाणमेव क्षेत्रं,
क्षेत्रफलमेव हि क्षेत्रं}
{परिकरस्तु शेषः~। द्वितीयक्षेत्रे खल्वपि भुजाः २०~। ८~। १०, एषां युतिः ३८,
अस्या दलं १९,}
{चतुर्धा १९~। १९~। १९~। १९ , भुजहीनं १$+$~। ११~। ९~। १९, तद्वधः १८८१$+$\renewcommand{\thefootnote}{४}\footnote{भुजहीनं १$+$\,। ११~। १९~। १९~।तद्वधः १८८$+$~।}, अतः
पदमिति नैतत्}
{पदयोग्यमवर्गराशित्वात्~। धनस्य तावद्वर्गो धनम् ऋणस्यापि, {\qt 'ऋणमृणघ्नं धनमेव ह्येवमृणस्य वर्गो\renewcommand{\thefootnote}{५}\footnote{ऋण
ऋणमेव ऋणस्ये वर्गो~।} धनतामियात्'} इति न्यायमूलाद्वचनात्, तत्र कुतो वर्गस्य
ऋणात्मकस्य \,सम्भवः~। \,अथोच्यते\textendash \;धनात्मकतैकस्वभावोऽपि \,वर्गराशिः \,ऋणात्मकेन\renewcommand{\thefootnote}{६}\footnote{वर्गराशि नृणा*~।} रूपेण गुणितः स्यादेव
ऋणात्मक इति~।}
{भवत्येतत् न तु वर्गराशिः; वर्गो ह्यवर्गगुणितो वर्गत्वादपहीयते, न च
वर्गो गुणः, तेन वर्गतां}
{ज्ञातुमिच्छति कश्चित् वर्गयोर्घातः (कर्तव्यः), स एव तु\renewcommand{\thefootnote}{७}\footnote{त~।}
वर्गयोर्वर्गराशित्वमायाति, वर्गवर्गवधस्तु}
{वर्गः\renewcommand{\thefootnote}{८}\footnote{वर्गो वर्गवदस्तु न~।}\,। अथाप्युच्यते रूपस्य वर्गराशित्वात्तेन वर्गत्वं न विहन्तीति~। 
तत्र रूपस्य ऋणात्मकत्वे}
{वर्गत्वासिद्धिः\renewcommand{\thefootnote}{९}\footnote{*सिद्धेः~।} तस्यापि; तत् बाहुघातसमुत्थमिति चैतत्\renewcommand{\thefootnote}{१०}\footnote{चेत्~।}
सर्ववस्तुपर्यनुयोगः~। अथाप्युच्यते\textendash \,वर्गराशिरेवात्मनः\renewcommand{\thefootnote}{११}\footnote{*शितेवात्यतः~।} प्रतिपद्यमानमृणतां व्रजेदिति~। भवत्येवं
वर्गपतितो राशिः, स तु पतितत्वान्निवृत्तार्थो\renewcommand{\thefootnote}{१२}\footnote{प्रतित्वा*~।} विनष्ट इति तस्यानात्मनः कुतो\renewcommand{\thefootnote}{१३}\footnote{ऊतो~।} ऋणत्वं धनत्वं वा~। 
नन्वयोज्यो बीजादिवत्\renewcommand{\thefootnote}{१४}\footnote{नन्वयोग्यो बीजादि~।}}
{व्यक्तं ततो वा तच्छोध्यमप्यशोध्यमविनाशि ऋणात्मकं\renewcommand{\thefootnote}{१५}\footnote{*नाशिं निरात्मकं~।} चेष्यते~। 
सत्यमिष्यते न तु तस्य}
{तथाविधस्यैव पृथक् गुणभागवर्गधनमूलादिकर्माणि\renewcommand{\thefootnote}{१६}\footnote{*वर्गतन्म*~।} सम्भवन्ति,\renewcommand{\thefootnote}{१७}\footnote{संभवति~।}
तयोः आयव्ययराश्योरेकवद्भावस्य न्यायलब्धस्योक्तत्वात् {\qt 'अव्यक्तकरणीनां च (रज्जुः) सा चैकराशिवत्'} इति~। अपि च}
{ऋणात्मकवर्गस्य कीदृशेन मूलेन भवितव्यम्? न धनेन तस्य\renewcommand{\thefootnote}{१८}\footnote{ततः~।}
तथाविधवर्गाप्रत्यायनात्;}
{एवमृणेन च न, न च\renewcommand{\thefootnote}{१९}\footnote{*मृणेन च न
च मूलेन च~।} मूलद्वयमस्ति~। यद्येकं धनात्मकपरं ऋणात्मकं,
तयोश्च घातो वर्गः}
{ऋणं च स्यादिति सम्भवे वा नासौ वर्गः विसदृशयोर्वधात्, मूलयोरपि च तयोः
परस्परं}
{निरवशेषशुद्धः शून्यफलतापत्तिः~। पुनरपि च तथाविधवर्गात् मूलानयनदोषः\renewcommand{\thefootnote}{२०}\footnote{तथाविधवर्गत्पानयन*~।}
क्षेत्रफलाभावश्च,}
{क्षेत्रासम्भवसिद्धिः प्रकृत इति~।}
\vspace{3mm}

{सम्भवतामपि च क्षेत्राणां येषां मुखोनभूवर्गः\renewcommand{\thefootnote}{२१}\footnote{*वर्ग~।}
पार्श्वभुजयोरन्तरयुत्योर्वधादभ्यधिको\renewcommand{\thefootnote}{२२}\footnote{*युत्योवधा*~।}}
{भवति तेषामेव लम्बावधयोरवाप्तिर्भवति\renewcommand{\thefootnote}{२३}\footnote{*वभयो*~।} नेतराणामिति तामेव कल्पनां
प्रतिशिक्षयति~।}

\newpage

\begin{quote}

{\qt  षड्विंशतिस्तु त्र्यंशा\renewcommand{\thefootnote}{१}\footnote{षड्विंशतिस्त्रिंशा~।} (त्र्यंशा)ष्टाविंशतिर्भुजौ यस्य~। \\
त्रिभुजस्य दश धरित्री तस्यावाधे\renewcommand{\thefootnote}{२}\footnote{भरित्री तस्यार्धाभे~।} समाचक्ष्व~॥}\end{quote}

\noindent{इति प्रश्ने, सम्भवति तावत्क्षेत्रं यतोत्र महाभुजो महीदशका\renewcommand{\thefootnote}{३}\footnote{महादशकः~।} 
इतरभुजयुतेरष्टादशिकाया}
{ऊन एव, अवधासम्बन्धस्त्वन्विष्यते यथा पार्श्वभुजयोः \begin{tabular}{|c}२६\\ ३\\\hline \end{tabular}\begin{tabular}{|c|}२८\\ ३\\\hline \end{tabular}\,, कृती \begin{tabular}{|c}६७६\\ ९\\\hline \end{tabular}\begin{tabular}{|c|}७८४ \\९ \\\hline \end{tabular}\,,}
{अनयोर्विवरं\renewcommand{\thefootnote}{४}\footnote{*वरे~।}  १२, भूहृतं\begin{tabular}{c}६\\५\end{tabular}, भूमौ ऋणधनं\begin{tabular}{c}४४\\५\end{tabular}\begin{tabular}{c} ५६\\५\end{tabular},
अनयोर्दले \begin{tabular}{|c|}२२\\५\\\hline \end{tabular}\begin{tabular}{c|}२८\\ ५\\\hline \end{tabular} 
एते अव(धे)~।}
{आभ्यां लम्बो यथा अवधावर्गेण\renewcommand{\thefootnote}{५}\footnote{अवधव*~।}  \begin{tabular}{|c|}४८४\\२५\\\hline \end{tabular} भुजवर्गात्\renewcommand{\thefootnote}{६}\footnote{वमांत्~।} \begin{tabular}{|c|}६७६\\९\\\hline \end{tabular} ऊनात्\renewcommand{\thefootnote}{७}\footnote{अनात्~।} \begin{tabular}{|c|} १२५४४\\२२५\\\hline \end{tabular}  मूलं \begin{tabular}{|c|} ११२\\ १५\\\hline \end{tabular}}\,,
{एष\renewcommand{\thefootnote}{८}\footnote{यष~।}  लम्बः~। लम्बावधानवाप्तिर्यथा\renewcommand{\thefootnote}{९}\footnote{लम्बवेधा*~।} त्रिभूमिके चतुष्पञ्चभुजे
त्र्यश्रे, तथा च पार्श्वभुजयोः ४।५}
{कृत्यो\textendash \,१६।२५\textendash \,र्विवरं\renewcommand{\thefootnote}{१०}\footnote{कृत्य १६।२५
विवरं~।} ९, भूम्या ३ हृतं ३, भूमौ ऋणधनं ०।६, तद्दलं
०।३, एते अवधे\renewcommand{\thefootnote}{११}\footnote{अवभे~।}\,।}
{न चोपपद्यते~। न हि शून्यं भूमेरावाधा भवति, भूखण्डद्वयात्मक(त्वाच्च)
तयोः\renewcommand{\thefootnote}{१२}\footnote{चयोः~।}\,। प्रायेण}
{भुजकोटिकर्णवति क्षेत्रे भुजकोट्योरन्यतरभूमित्वेऽवधालम्बानवाप्तिः, तत्र
हि भुजस्य भूमित्वे}
{कोटितुल्यो लम्बः कोटेश्च भुजतुल्यो, यो यस्य तुल्यः स तत्सूत्रेणैव
सम्भवति~। सम्भवतामपि च}
{तेषां षडेव\renewcommand{\thefootnote}{१३}\footnote{दशैव~।}  स्वकरणानेतव्यफलानि~।}
\vspace{3mm}

{आयतचतुरश्रं यथा द्वादश पञ्च द्वादश\renewcommand{\thefootnote}{१४}\footnote{द्वादशे~।}  पञ्च भुजा यस्येत्याद्यायतं
चतुर्बाहुः,\renewcommand{\thefootnote}{१५}\footnote{मुजाद्यायतं चतुर्वाहु~।}  समचतुरश्रं \,यथा \,{\qt 'वदनं\renewcommand{\thefootnote}{१६}\footnote{वैदनं~।}  पञ्च \,धरित्री\renewcommand{\thefootnote}{१७}\footnote{धत्री~।}  पञ्चैव \,भुजौ \,च \,तादृशौ \,यस्ये'}ति, \,(\,त्रिसमचतुरश्रं यथा\,) \,{\qt 'त्रिंशत् सनव धरित्री शेषभुजाः पञ्चविंशतिर्यस्ये'}त्यादि यन्निबोध्यं
तत्क्षेत्रं त्रिसमचतुरश्रं,}
{विषमचतुरश्रं यथा {\qt 'भूमिः षष्टिर्वदनं पञ्चकृतिः सा द्विसङ्गुणैकभुजः
त्रिंशत् सनवेत्यपरो\renewcommand{\thefootnote}{१८}\footnote{सनवात्प~।}  भुजो यस्ये'}ति तु विषमचतुरश्रं,\renewcommand{\thefootnote}{१९}\footnote{रथं त्रयमपि समं सेम~।}  द्विसमचतुरनं यथा {\qt 'भूवदने चाष्टमिते त्रयो
भुजः सप्तकोऽपरो\renewcommand{\thefootnote}{२०}\footnote{भुजोः स*~।} यस्ये'}ति, भुजत्रयमपि\renewcommand{\thefootnote}{२१}\footnote{भुजे त्रयमपि~।}  (सममिति समत्रिभुजं, भुजत्रयमपि विषममिति)
विषमत्रिबाहुकविषयं,}
{(भूमिः) पञ्चदश\renewcommand{\thefootnote}{२२}\footnote{दशः~।}  त्रयोदशभुजावित्यादि द्विसमबाहुत्र्यश्रमिति\renewcommand{\thefootnote}{२३}\footnote{*बाहुनि त्र्यश्रे~।},
त्रिभुजानि त्रीण्येव}
{पञ्चैव चतुर्भुजानि चेत्यष्टौ,\renewcommand{\thefootnote}{२४}\footnote{चेति अष्ट~।} वृत्तं\renewcommand{\thefootnote}{२५}\footnote{वृतं~।}  नवमं क्षेत्रं, दशमं
धनुः~। एतयोरभिधा(ने) केचित्तु}
{धनुषो वृत्तैकखण्डत्वान्नवैव क्षेत्राणि समाख्यन् यथा तु नैतत्तथा
तत्करणावसर एव}
{भविष्यति~। अपरे तु धनुषि प्रत्याख्याते द्विद्विसमभुजमुपसङ्ख्यातवन्तः,
भूमुखतो\renewcommand{\thefootnote}{२६}\footnote{भूमुख्यतो~।}}
{भुजप्रतिभुजतः साम्यनियमेनोत्पन्नमायतचतुरश्रं पर्युदस्य द्वौ द्वौ भुजौ
यस्य समौ तत्}
{द्विद्विसमभुजं, परिशिष्टं समद्विभुजस्य\renewcommand{\thefootnote}{२७}\footnote{तत् विषमुद्विभुज परिशिष्ट
ममद्विभुजात्~।}  प्रकारान्तरमेवेति
तत्राहुः~। द्विसमभुजस्यायं}
{भेदो न तेन सहासमशीर्षिकामर्हतीति\renewcommand{\thefootnote}{२८}\footnote{सहसमशीर्षिकामर्हतीत्~।}  पक्षयति, यदुत यः कश्चन भुजो
भूः\renewcommand{\thefootnote}{२९}\footnote{भू न~।}  कल्पनीयो यदि लम्बेनार्थः स्यादिति भुवि परिसमाप्यते न तु
क्षेत्रमन्यदित्यनुपसङ्ख्येयम्~।}

\newpage

\noindent{आयतचतुरश्रेऽपि तुल्यमिति चेन्न, तस्य वर्गप्रकृत्यर्थं तत्र विशेषत
उत्पादनात् पृथक् प्रसिद्ध्या}
{प्रधानत्वात्~। तथा च वर्गप्रकृतौ तदुत्पादनं क्रियते\textemdash}

\begin{quote}

{\qt भुजस्य कृतिरिष्टस्य भक्तोनेष्टेन तद्दलम्~। \\
 कोटिरिष्टाधिका कर्णश्चतुरश्रायतस्य ते~॥}\end{quote}

{उदाहरणम्\textendash \,इष्टस्य भुजस्य ३ कृतिः ९, इष्टेन १ भक्ता ९ इष्टेनैव १ ऊना ८,}
{तद्दलं कोटिः ४, इष्टाधिका ५ कर्णः~। एवमेतैर्भुजकोटिकणैः
समुदितैश्चतुरश्रक्षेत्रोत्पत्तिः,}
{उपयो(ग)श्चास्य यथा}

\begin{quote}
    
{\qt  गुणके वर्गयोर्मध्ये तत्पदाधो भुजश्रुती~। \\
 केचित्प्राक्कथिते तत्र वज्रकेणाहती तयोः~। \\
 अन्तरस्य कृतिः क्षेपस्तत्कोटिः प्रथमं पदम्~। \\
 ऋजुहत्यन्तरं\renewcommand{\thefootnote}{१}\footnote{वज्रह*~।} ज्येष्ठं रूपक्षेपेऽन्तरोद्धृते~॥}\end{quote}

{उदाहरणम्\textendash \,कस्य वर्गः पञ्चपञ्चाशता गुणित एकेन युतो मूलदो भवति~।}
\begin{center}


  \begin{tabular}{|c|c|} गु & क्षे\\ ५५ &१\\\hline \end{tabular}

 \end{center}

\noindent{गुणके ५५ अनयोः ३~। ८ वर्गयोः ९~। ६४ मध्ये\renewcommand{\thefootnote}{२}\footnote{अनयोः ३१८ वर्गयोः ९~। ९४ मध्ये~।} तुल्ये सति तत्पदयोरूनाधिकयोः}
{क्रमेण ३~। ८  केचित्प्राक्कथिते भुजश्रुती ३~। ५ अधः स्थाप्यते~। \begin{tabular}{|c|}३\\३\\\hline \end{tabular}\begin{tabular}{c|}८\\५\\\hline \end{tabular} तत्र वज्रकेणाहती}
{१५~। २४, तयोरन्तरं ९, तस्य कृतिः ८१, क्षेपोऽयम्, तत्कोटिः ४ प्रथमं पदं,
ऋजुहत्योः}
{९~। ४० अन्तरं ३१ ज्येष्ठपदम्, एते आद्य द्वितीयपदे ४~। ३१ वज्रहत्यन्तरे(ण)
उद्धृते रूपक्षेपस्य}
{पदे भवतः, अनुद्धृते तु वज्रहत्यन्तरकृतियुक्ते\renewcommand{\thefootnote}{३}\footnote{*कृते भुक्ते~।} एव~। एवं
चतुरश्रस्यामित\renewcommand{\thefootnote}{४}\footnote{*श्रयामिन~।} उपयोगो दर्शितः~।}
{अन्यः चोपयोगस्तत्रैव भूयसोक्तः~। एवं दशैवेति सिद्धमवधारणम्~। यानि तु
गजदन्तादीनि}
{तेषां वक्ष्य-माणानुसारम् एतदन्यतमकल्पनयैतदीयैरेव करणैः फलानयनम्~। तच्च
फलानयनं}
{केचित् द्विधा कुर्वन्ति\renewcommand{\thefootnote}{५}\footnote{कुर्वति~।} यथा\textemdash}

\begin{quote}

{\qt  'स्थूलफलं\renewcommand{\thefootnote}{६}\footnote{स्थूलं फलं~।} त्रिचतुर्भुजबाहुप्रतिबाहुयोगदलघातः\renewcommand{\thefootnote}{७}\footnote{भुजे वाहुप्रतिवाहुंयो*~।}\,।\\
 भुजयोगार्धचतुष्टयभुजोनघातात्\renewcommand{\thefootnote}{८}\footnote{*चतुष्टयेभुजन~।} पदं सूक्ष्मम्~॥'}\end{quote}

\noindent{इति~। तत्र स्थूलं फलं\renewcommand{\thefootnote}{९}\footnote{*स्थूलो फलेत~।} नाम तेन भवितव्यं यत्तु
तात्विकफलप्रत्यासन्नं\renewcommand{\thefootnote}{१०}\footnote{त्वात्वि*~।} न तु दूरान्तरम्\renewcommand{\thefootnote}{११}\footnote{रांतरं~।}\,। न}
{चैतत्स्थूलफलकरणं, तत्तु\renewcommand{\thefootnote}{१२}\footnote{तत्सु~।} बह्वन्तरितं फलं दृश्यते क्षेत्रविशेषे,
यथा त्रयोदशकबाह्वोश्चतुर्विंशतिभूमिकस्य त्रिभुजस्य बाहुप्रतिबाहुयोगदलघातः}

\newpage

\noindent{षड्युतम् अध्यर्धशतं\renewcommand{\thefootnote}{१}\footnote{*घातात्षज्युतमप्यर्ध*~।} १५६,
सूक्ष्मं षष्टिः ६०}
{प्रजायते~। तथा चैवं\renewcommand{\thefootnote}{२}\footnote{च~।} स्थूलफलकरणं, सूक्ष्मं तु \hyperref[117]{'भुजयुतिदलं चतुर्धा भुजहीनं तद्वधात्पदं गणितम्'}~।}
{एकस्तावद्बाहुप्रतिबाहुयोगः २६ अपरः २४, अनयोर्दले १३~। १२, अनयोर्घातः
१५६~।}
{अथ सूक्ष्मफलकर्म\textendash \,भुजानां २४~। १३~। १३ योगस्य ५० दलं २५, चतुर्धा २५~। 
२५~।}
{२५~। २५, भुजैः क्रमेणोनं १~। १२~। १२~। २५, एषां घातात् ३६०० पदं ६०, एतत्
सूक्ष्मम्~।}
{यत एवं तस्मात्सूक्ष्मफलस्यैवानयनकरणोपपत्तिः~।}
\vspace{3mm}

{अथानयोस्तात्विकमुच्यते\textendash \,स्थूलादन्यत्, असम्भवतोऽपि क्षेत्रस्य
स्थूलफलदर्शनात्,}
{विंशतिभूमिकस्य त्रयोदशसप्तकबाह्वोस्त्र्यश्रस्य शतफला(वा)प्तेः, तथा च
बाहुप्रतिबाहुयोगौ}
{२०~। २०, तयोर्दले १०~। १०, तयोर्घातः १००~। यत्र च क्षेत्रस्यैव सम्भवो
नास्ति कस्तत्र}
{क्षेत्रफलावगमः, असम्भवश्च क्षेत्रस्यैकस्माद्विंशतिकात् भुजादपरबाहुयुतेः
समत्वात्~। अवधयोश्च}
{स्वस्वभुजे(न) समत्वाल्लम्बाभावः, तथा हि पार्श्वभुजयोः १३~। ७ कृती\renewcommand{\thefootnote}{३}\footnote{कृति~।} 
१६९~। ४९, अनयोर्विवरं १२०, भूपरिमाणेन\renewcommand{\thefootnote}{४}\footnote{ऊपरि*~।}  २० हृतं ६, ऋणधनं भूमौ १४~। २६, तद्दले\renewcommand{\thefootnote}{५}\footnote{तद्दल~।} 
७~। १३ अवधे,}
{महती १३ बृहद्भुजे १३, परि(शि)ष्टा ७ अल्पभुजे ७, आभ्यां च लम्बाभावः,
यतः अवधावर्गेण}
{१६९ भुजवर्गात् १६९ ऊनात् ०, मूलं ०, लम्बः~। यत्र च क्षेत्रे उभयतोऽपि
लम्बाभावस्तत्र न}
{क्षेत्रफलस्य सूक्ष्मस्यावाप्तिः, नापि तत्क्षेत्रं स्यात्, शून्यलम्बत्वे
मुखभूमिविश्लेषाभावे क्षेत्रत्वाभावात्\renewcommand{\thefootnote}{६}\footnote{क्षेत्रत्रत्वा*~।}\,। सूक्ष्मफलं त्वेवंविधे क्षेत्राभासे नैवायाति, तथा च भुजाः
२०~। १३~। ७~। ०, एषां युतिः}
{४०, अतो दलं २०, भुजैर्हीनं ०~। ७~। १३~। २०, एषां वधः शून्यम्, अतः पदम् अपि
शून्यम् एव~।}
{तस्मात तात्विकम् एतत्तदानुमितस्यैव साधनकरणानि सामान्यविशेषभावापादनेन\renewcommand{\thefootnote}{७}\footnote{विशेषाभा*~।} 
चापपञ्चाश्रवैलक्षण्येन वक्ष्यन्ते~। यैस्तु स्थूलफलमुक्तं
तेषामल्पान्तरावलम्बपार्श्वभुजेषु संवादागमान्मिथ्यावाप्तिर्जाता~। संवादस्थानं चैषां प्रथमं तावत् समचतुरश्रमेव, पञ्चकबाहुके
तस्मिन् बाहुप्रतिबाहुयोगौ १०~। १०, अनयोर्दले ५~। ५, अनयोर्घातः २५, सूक्ष्मं फलं चैतदेव यथा
भुजयुतिः}
{२०, दलं १०, चतुर्धा १०~। १०~। १०~। १०, भुजहीनं ५~। ५~। ५~। ५, तद्वधः ६२५,
अतः}
{पदं २५; आयतचतुरश्रमपि, पञ्चकभूवदने\renewcommand{\thefootnote}{८}\footnote{*कभवंदने~।}  त्रिकबाहौ तस्मिन् भुजयुतिः १६,
दलं ८, चतुर्धा}
{८~। ८~। ८~। ८, भुजहीनं ३~। ५~। ३~। ५, तद्वधात्\renewcommand{\thefootnote}{९}\footnote{उद्वधात्~।}  २२५ पदं १५, एतत् सूक्ष्मम्,
स्थूलमपि}
{च (बाहुप्रति)बाहुयोगौ १०~। ६, दले ५~। ३, अनयोर्घातः १५~। आसन्नता च यथा
सप्तकभूभुजे}
{पञ्चकमुखभुजे च द्विद्विसमचतुरश्रे, तत्र हि सूक्ष्मं फलं भुजयुतिः २४,
दलं १२, चतुर्धा}
{१२~। १२~। १२~। १२, भुजहीनं ५~। ७~। ५~। ७, तद्वधः ११२५, अतः पदं ३५, स्थूलफलं}
{तु बाहुप्रतिबाहुयोगौ १२~। १२, अनयोर्दलं ६~। ६, अनयोर्घातः ३६,
रूपमात्रान्तरितत्वादत्र}
{प्रत्यासन्नता~। दूरान्तरताविषयस्तु दर्शित एव~। अथ कीदृशि स्थाने संवादः
प्रत्यासन्नता}
{दूरान्तरता\renewcommand{\thefootnote}{१०}\footnote{*रत~।}  वा, अवलम्बपार्श्वभुजसाम्याल्पान्तरबह्वन्तरवत्\renewcommand{\thefootnote}{११}\footnote{*वहुंतरवत्स~।} 
क्रमेण~।}

\newpage

{(\,समलम्बचतुरश्रत्र्यश्रयोः क्षेत्रफलानयनायार्यामाह\,)\textemdash}

\phantomsection \label{115}
\begin{quote}
{\bs समलम्बकचतुरश्रे त्र्यश्रे क्षेत्रे\renewcommand{\thefootnote}{१}\footnote{क्षेत्रं~।} च जायते गणित(म्)~। \\    
 भूवदनसमासार्धं मध्यमलम्बेन सङ्गुणितम्~॥~११५~॥}
\end{quote}

{समचतुरश्रं द्वित्रिसमभुजं च तुल्यपार्श्वभुजं समलम्बकं, तेषामेव
समभूखण्डकरणादर्धलम्बनिपातस्य समलम्बत्वम्, तस्मिन् समलम्बके चतुरश्रे यस्मिन्कस्मिंश्चित्
त्र्यश्रे च भूवदनयोगदलं मध्यमेन मध्यभवेन लम्बेन गुणितं सत् गणितं जायते~।}
\vspace{3mm}

{उदाहरणम्\textemdash}

\begin{quote}
{\eg समचतुरश्रे\renewcommand{\thefootnote}{२}\footnote{समरच*~।} क्षेत्रे बाहुसमा वक्त्रभूमिलम्बाः\renewcommand{\thefootnote}{३}\footnote{वक्रभू*~।} स्युः~। \\
 तेऽध्यर्धहस्तसङ्ख्याः\renewcommand{\thefootnote}{४}\footnote{तेप्यर्ध*~।} कथय सखे किं फलं तत्र~॥~१२२~॥}
\end{quote}

{समचतुरश्रे तावद्वक्त्रभूमिलम्बाः\renewcommand{\thefootnote}{५}\footnote{*लम्ब~।} बाहुसमाः स्युः, नात्र
लम्बपरिच्छेदः\renewcommand{\thefootnote}{६}\footnote{*परिछेद~।} आचार्यपरिचर्ययोपयुज्यते,\renewcommand{\thefootnote}{७}\footnote{*परिचर्योप*~।} किन्तु ते यत्र पञ्चापि सार्धैकहस्तसङ्ख्यास्तत्र
सङ्क्षेपमाश्रित्य न तु लौकिकगणनया}
{कलं कथय~।}
\vspace{3mm}

{न्यासः\textendash \,}

\hspace{12mm} \includegraphics[width=5cm, height=3cm]{Images/page-0214as.jpeg}
\vspace{3mm}

{करणम्\textendash \,भूः\begin{tabular}{c}३\\ २\end{tabular}, वदनं\begin{tabular}{c} ३\\ २\end{tabular}}, अनयोः समासः ३, अस्य
दलं\begin{tabular}{c}३\\ २\end{tabular}, मध्यमलम्बेन\begin{tabular}{c}३\\ २\end{tabular}{सङ्गुणितं \begin{tabular}{c}९\\ ४\end{tabular}, अतो\renewcommand{\thefootnote}{८}\footnote{अ २ तो~।} लब्धं हस्तौ २ अङ्गुलानि\renewcommand{\thefootnote}{९}\footnote{अगुलानि~।} षट्~।}
\vspace{3mm}

{द्वितीयोदाहरणम्\textendash}

\begin{quote}
{\eg यत्रायतचतुरश्रे\renewcommand{\thefootnote}{१०}\footnote{यत्रयेत*~।} भूवदने सार्धपञ्चकरसङ्ख्ये~। \\
 पार्श्वभुजमध्यलम्बास्त्रिहस्तकास्तत्र वद गणितम्~॥~१२३~॥}\end{quote}

{यत्राय(त)चतुरश्रे भूमिर्मुखं च पञ्चहस्ता अङ्गुलानि च द्वादश
प्रत्येकं तथा पार्श्वभुजौ}
{लम्बश्चेति त्रयोऽपि प्रत्येकं त्रयो हस्तास्तस्मिन् गणितं वदेति~।}

\newpage

{न्यासः\textendash \,}
\vspace{-4mm}

\hspace{20mm} \includegraphics[width=5cm, height=3cm]{Images/page-0215as.jpeg}
\vspace{1mm}

{करणम्\textendash \,भूः\begin{tabular}
{c}५\\१\\२\end{tabular}वदनं\begin{tabular}
{c}५\\१\\२\end{tabular}अनयोः समासः\renewcommand{\thefootnote}{१}\footnote{समासः\begin{tabular}{c}३१\\२\end{tabular}।} ११, \bigg(अस्य दलं\begin{tabular}
{c}११\\२\end{tabular}\bigg), मध्यमलम्बेन ३}
{सङ्गुणितं \begin{tabular}{|c|} ३३\\२\\\hline \end{tabular}\,, अतो लब्धं हस्ताः १६ अङ्गुलानि\renewcommand{\thefootnote}{२}\footnote{अगुलानि~।} द्वादश १२~।}
\vspace{3mm}

{उभयोरप्युदाहरणवाक्ययोर्भूवदनलम्बमात्रोपयोगे पार्श्वभुजोपादानं
समलम्बकत्वप्रदर्शनार्थम्~। अनयोश्च क्षेत्रयोः स्थूलम् अपि सूक्ष्ममेव फलं, यतः
(प्रथमोदाहरणे) बाहुप्रतिबाहुयोगौ ३, \bigg(दलितौ \begin{tabular}{c|}३\\२\end{tabular}\begin{tabular}{c}३\\२\end{tabular}\bigg), अनयोर्घातः २४, लब्धं तदेव हस्तौ
२ अङ्गुलानि\renewcommand{\thefootnote}{३}\footnote{अं.....६~।} ६;}
{(\,द्वितीये तु बाहुप्रतिबाहुयोगौ ११~। ६\,), अनयोर्दले\begin{tabular}{c|}११\\२\end{tabular} ३,
अनयोर्घातः\begin{tabular}{c}३३\\२\end{tabular}, अतो (ल)ब्धं तदेव}
{हस्ताः षोडश अङ्गुलानि द्वादश~। वक्ष्यमाणेनापि करणेनास्य संवादः\renewcommand{\thefootnote}{४}\footnote{सवादः~।}, यतः
(प्रथमोदाहरणे) भुजयुतिः ६, अतो दलं ३, चतुर्धा ३~। ३~। ३~। ३, भुजैर्हीनं \begin{tabular}{|c|}३\\२\\\hline \end{tabular}\begin{tabular}{c|}३\\२\\\hline \end{tabular} \bigg(\begin{tabular}{c|}३\\२\end{tabular}\begin{tabular}{c}३\\२\end{tabular}\bigg), तद्वधः\renewcommand{\thefootnote}{५}\footnote{तद्वधः·····~।}  \bigg(\begin{tabular}{cc}८१\\१६\end{tabular}\bigg), अतः पदं \begin{tabular}{|c|} ९\\४\\\hline \end{tabular}  अतो लब्धं तदेव हस्तौ २ अङ्गुलानि
षट् ६; तथा (च द्वितीयोदाहरणे) भुजयुतिः}
{१७, दलं \begin{tabular}{|c|} १७\\२\\\hline \end{tabular}\,, चतुर्धा\renewcommand{\thefootnote}{६}\footnote{चत्तुर्धा~।} \begin{tabular}{|c|} १७\\२\\\hline \end{tabular}\begin{tabular}{c|}१७\\२\\\hline \end{tabular}
\bigg(\begin{tabular}{c}१७\\२\end{tabular}\begin{tabular}{|c}१७\\२\end{tabular}\bigg), भुजहीनं
\begin{tabular}{|c|}३\\१\\\hline \end{tabular}\begin{tabular}{c|}११\\२\\\hline \end{tabular}\begin{tabular}{c|}३\\१\\\hline \end{tabular}\begin{tabular}{c|}११\\२\\\hline \end{tabular}\,,
तद्वधः\begin{tabular}{cccl}१०८९\\४\end{tabular}, अतः पदं\begin{tabular}{c}३३\\२\end{tabular}, अतो लब्धं तदेव हस्ताः १६ अङ्गुलानि १२~।
\vspace{3mm}

{उदाहरणम्\textemdash}

\begin{quote}

{\eg पादोनाभ्यधिकचतुस्त्रिहस्तसङ्ख्यौ\renewcommand{\thefootnote}{७}\footnote{नादो*~।} भुजौ धरा त्र्यश्रे~। \\
 सार्धत्रिकरा लम्बस्त्रिहस्तकस्तत्फलं\renewcommand{\thefootnote}{८}\footnote{लम्बे*~।}  किं स्यात्~॥~१२४~॥}
 \end{quote}

{त्रयो हस्ता अष्टादशाङ्गुलान्येको भुजः, त्रयो हस्ताः षडङ्गुलानि
द्वितीयो भुजः,}
{भूमिस्त्रयो हस्ताः द्वादशाङ्गुलानि\renewcommand{\thefootnote}{९}\footnote{द्वितीयः त्रयो हस्ता
द्वादशांगुल....~।}, त्रयो हस्ता लम्बः, एवंवस्तुनि
त्र्यश्रे फलं किं स्यात्~।}
\vspace{3mm}

{न्यासः\textendash \,}
\vspace{-3mm}

\hspace{24mm} \includegraphics[width=4cm, height=3cm]{Images/page-0215bs.jpeg}

\newpage

{कर्म\textendash \,भूः\begin{tabular}{c}७\\२\end{tabular}, वदनं (०), अनयोः समासः\begin{tabular}{c}७\\२\end{tabular}, अर्धं\renewcommand{\thefootnote}{१}\footnote{अर्धं ४~। ....~।}\begin{tabular}{c}७\\४\end{tabular}, लम्बेन ३ सङ्गुणितं \begin{tabular}{|c|} २१\\४\\\hline \end{tabular}\,,}
{(लब्धं ह ५) अं ६~। स्थूलं फलमपि बाहुप्रतिबाहुयोगौ\begin{tabular}{c|}७\\२\end{tabular} ७, अनयोर्दले
\begin{tabular}{|c|}७\\४\end{tabular}\begin{tabular}{c|}७\\२\end{tabular}\,, अनयोर्घातः}
{\begin{tabular}{c}४९\\८\end{tabular}। अत्राधिको पार्श्वभुजौ\renewcommand{\thefootnote}{२}\footnote{त्राधिकौ पार्श्वभुजा~।} लम्बात्, अतः स्थूलमपि
सूक्ष्मभागातिरिक्तमुत्पन्नम्~।}
\vspace{3mm}

{अथात्र लम्बावधानुसरणम्\textendash \,पार्श्वभुजयोरन्तरं\renewcommand{\thefootnote}{३}\footnote{*रन्तरं\begin{tabular}{c}२\\२\end{tabular}।}\begin{tabular}{c}१\\२\end{tabular}, संयुतिः (७),
वधः\begin{tabular}{c}७\\२\end{tabular}, अतोऽधिका मुखहीनभूकृतिः\begin{tabular}{c}४९\\४\end{tabular}, तस्मादस्त्यत्र लम्बावधावाप्तिः~। 
तत्रावधानयनं तावत् त्र्यश्रस्य}
{पार्श्वभु-जयोः\renewcommand{\thefootnote}{४}\footnote{पार्श्वें*~।} \begin{tabular}{|c|} १५\\४\\\hline \end{tabular}\begin{tabular}{c|} १३\\४\\\hline \end{tabular}  कृतिविवरं\begin{tabular}{c}७\\२\end{tabular}, भूहृतं १, ऋणं धनं भूमो \begin{tabular}{|c|c|}५&९\\२&२\\\hline \end{tabular}\,,
तद्दलं \begin{tabular}{|c|c|}५&९\\४&४\\\hline \end{tabular}\,, अवधे~।}
{(लघ्वी लघुभुजे) महती\renewcommand{\thefootnote}{५}\footnote{महती वृहती~।}  बृहद्भुजे~। लम्बः खल्वपि, अवधायाः\begin{tabular}{c}५\\४\end{tabular}वर्गः\begin{tabular}{c}२५\\१६\end{tabular}, अनेन भुजवर्गात्\renewcommand{\thefootnote}{६}\footnote{भुजवर्गात् १६९~।}}{\begin{tabular}{c}१६९\\१६\end{tabular}ऊनात् ९, अतो मूलं ३, एष लम्बः, तथा अवधायाः\renewcommand{\thefootnote}{७}\footnote{अवधायः~।} \begin{tabular}{|c|}९\\४\\\hline \end{tabular}
वर्गेण \begin{tabular}{|c|} ८१\\१६\\\hline \end{tabular} भुजवर्गात्\begin{tabular}{c}२२५\\१६\end{tabular}ऊनात् ९, मूलं ३~।}
\vspace{3mm}

{अत्राप्युदाहरणवाक्ये भुजोपादानं समलम्बताप्रतिपादनार्थम्~। 
वक्ष्यमाणकरणेन}
{क्षेत्रस्य फलसंवादः, (यथा) भुजयुतिः\begin{tabular}{c}२१\\२\end{tabular}, दलं\renewcommand{\thefootnote}{८}\footnote{दल~।}\begin{tabular}{c}२१\\४\end{tabular},
चतुर्धा
\begin{tabular}{c|}२१\\४\end{tabular}\begin{tabular}{c|}२१\\४\end{tabular}\begin{tabular}{c|}२१\\४\end{tabular}\begin{tabular}{c}२१\\४\end{tabular},
भुजहीनं} {\begin{tabular}{c}७\\४\end{tabular}\begin{tabular}{c}३\\२\end{tabular} \begin{tabular}{c}२\\१\end{tabular}\begin{tabular}{c}२१\\४\end{tabular}, तद्वधात् पदं\renewcommand{\thefootnote}{९}\footnote{पदं २१~।}\begin{tabular}{c}२१\\४\end{tabular}, अतो
लब्धं तदेव ह ५ अं ६~।}
\vspace{3mm}

{कथं पुनः स्थूलसूक्ष्मकर्मणोः संवादः\,? पृथक् पृथक्\renewcommand{\thefootnote}{१०}\footnote{पृ~।} स्थाने ह्येते
कर्मणी; बाहुप्रतिबाहुयोगदलघातात्मके हि स्थूलम्, भूवदनसमासार्धमध्यमलम्बवधस्वरूपे\renewcommand{\thefootnote}{११}\footnote{भूवदनसमासार्धं*~।}
तु सूक्ष्मम्, संवादे च
{स्थूलसूक्ष्मत्वानु(प)पत्तिः, व्यतिरेके\renewcommand{\thefootnote}{१२}\footnote{व्यतरे~।} च व्यसंवादे क्वचित्
संवादोऽपि न युज्यते\renewcommand{\thefootnote}{१३}\footnote{युज्येत~।} लक्षणयोः भेदात्~।}
{न हि सङ्कलितकरणयोः \hyperref[102.1]{'सैकपदे'}त्यस्य \hyperref[15.1]{'पदयुतपदवर्गे'}त्यस्य संवादविसंवादौ
स्तः, वर्ग-घनप्रत्युत्पन्नकरणेन विकल्पनैव\renewcommand{\thefootnote}{१४}\footnote{*प्रत्युपन्न करणेन विकल्पना वा~।} उच्यते~। पृथक् पृथक्\renewcommand{\thefootnote}{१५}\footnote{प्र~।} स्थानत्वेऽपि च
कर्मणोः संवादो भवति सङ्कलितप्रत्युत्पन्नवर्गघनकरणविकल्पवत्, लोकेऽपि च धूमस्य दाहकत्वस्य च वह्निज्ञापनेन~। बाहुप्रतिबाहुयोगदलयोर्भूवदनसमासार्धलम्बयोश्च\renewcommand{\thefootnote}{१६}\footnote{*योर्भवदनसमार्ध*~।} समायतचतुरश्रे
तुल्यत्वात्पार्श्वभुजलम्बानां}
{तुल्यत्वे तत्र भुजयोगार्धस्य लम्बेनाविशेषात् (संवादः); यत्र तु लम्बस्य
पार्श्वभुजयोगार्धस्य\renewcommand{\thefootnote}{१७}\footnote{पार्श्वयुगार्ध*~।}}
{विशेषस्तत्र संवादो दूरभ्रष्टः~। ततः\renewcommand{\thefootnote}{१८}\footnote{दूरम्रंशात्तः~।} संवादस्थलदूरभ्रष्टतानां\renewcommand{\thefootnote}{१९}\footnote{*लहूर*~।}
च गतिमाचार्य एव सूचितवान्}
{\hyperref[112]{'अवलम्बपार्श्वभुजे'}त्याद्यावदन्\renewcommand{\thefootnote}{२०}\footnote{*त्यात्यावदन्~।}\,।
अवलम्बपार्श्वभुजयोरल्पान्तरत्वे फलान्तरवचनं हि तदन्तराभावदूरान्तरत्वयोश्च फलसंवाददूरभ्रंशावपि गमयति~।}

\newpage

{उदाहरणम्\textemdash}

\begin{quote}
    
 {\eg  तुल्यत्रिभुजे भूमिः सार्धाष्टौ लम्बकः कराः सप्त~। \\
 अष्टौ तथाङ्गुलानि त्र्यंशौ च कियत्\renewcommand{\thefootnote}{१}\footnote{किययत्~।}  फलं तत्र~॥~१२५~॥}\end{quote}

{सर्वैर्भुजैस्तुल्यं त्रिभुजं तुल्यत्रिभुजं न तु तुल्यास्त्रयो भुजा
यस्येति, एवं हि भूमेः}
{सङ्कीर्तनाच्चतुर्भुजावाप्तिस्तच्च\renewcommand{\thefootnote}{२}\footnote{संकीर्तनाच्चतुर्भजाप्ति*~।}  न प्रकृतं (त्र्य)श्रेणान्तरितत्वात्~। तस्मात्\renewcommand{\thefootnote}{३}\footnote{तस्मा~।}  द्वितीयमिदं}
{त्र्यश्रमेवोदाह्रियत इति~। एवंविधे त्र्यश्रे भूमिरष्टौ हस्ता
द्वादशाङ्गुलयुक्ता\renewcommand{\thefootnote}{४}\footnote{द्वादशांगुलापर्ययुक्ता~।}  भुजावपि तुल्यत्वाभिधानात्,\renewcommand{\thefootnote}{५}\footnote{तल्य*~।}  लम्बश्च सप्त हस्ता अष्टावङ्गुलानि द्वौ
चाङ्गुलत्रिभागौ, तत्र किं क्षेत्रफलम्~।}
\vspace{3mm}

{न्यासः\textendash \,}
\vspace{-1mm}

\hspace{24mm} \includegraphics[width=5.5cm, height=4.5cm]{Images/page-0217a.jpeg}

{कर्म\textendash \,भूः\begin{tabular}{c} १७\\२\end{tabular}, (मुखं ०), अनयोः समासः \bigg(\begin{tabular}{c}१७\\२\end{tabular}\bigg),
अतोऽर्धं\begin{tabular}{c}१७\\४\end{tabular}, मध्यमलम्बेन\renewcommand{\thefootnote}{६}\footnote{*लंबेन \begin{tabular}{|c|}७ \\  २\\ 
 ८\\  २४ \\ २ \\ ३\\\hline \end{tabular}~।}  \begin{tabular}{|c|}७ \\  १\\ 
 ८\\  २४ \\ २ \\ ३\\\hline \end{tabular}}
{\hyperref[41]{'प्राक्छेदांशावि'}ति \,सवर्णितेन\begin{tabular}{c}२६५\\३६\end{tabular}सङ्गुणितम् \,\begin{tabular}{|c|}४५०५\\१४४\\\hline \end{tabular}\,, \,(लब्धं \,हस्ताः \,३१) \,अङ्गुलानि \,६}
{भागाः \begin{tabular}{|c|}५\\६\\\hline \end{tabular} स्थूलं\renewcommand{\thefootnote}{७}\footnote{स्थूले~।}  फलं तु बाहुप्रतिबाहुयोगौ
\begin{tabular}{|c|}१७\\२\\\hline \end{tabular}\begin{tabular}{c|} १७\\ \\\hline \end{tabular}\,, अनयोर्दले\begin{tabular}{c|}१७\\४\end{tabular}\begin{tabular}{c}१७\\२ \end{tabular}\,,}
{अनयोः घातः\renewcommand{\thefootnote}{८}\footnote{*तः\begin{tabular}{r}३८९\\ ८\end{tabular}।}\begin{tabular}{c}२८९\\ ८\end{tabular}, अतो लब्धं (ह) ३६ अं ३, दूरभ्रष्टं
चैतत्~।}
\vspace{3mm}

{अथात्र लम्बावधानुसरणम्\renewcommand{\thefootnote}{९}\footnote{*नुसारणम्~।}\textendash \,पार्श्वभुजान्तरं ०. संयुतिः १७,
अनयोर्वधः ०, अस्मान्मुखहीनभूकृतिः \begin{tabular}{|c|}२८९\\ ४\\\hline \end{tabular} अधिकैव, तस्मादस्त्यत्रावधालम्बलाभ\renewcommand{\thefootnote}{१०}\footnote{*लंवनला*~।} 
इति~। तदानयनम्\textendash \,पार्श्वभुजयोः\renewcommand{\thefootnote}{११}\footnote{पाश्वभु*~।}  \begin{tabular}{|c|}१७\\२\\\hline \end{tabular}\begin{tabular}{c|}१७\\२\\\hline \end{tabular} कृती\begin{tabular}{c}२८९\\४\end{tabular}\begin{tabular}{|c} २८९\\४\end{tabular}, अनयोर्विवरं ०, भूहृतमिति\renewcommand{\thefootnote}{१२}\footnote{भूकुस्यत*~।}  खमेव, ऋणं धनं भूमौ}
{\hyperref[21]{'राशिरविकृतः खयोजनापगम'} इति\renewcommand{\thefootnote}{१३}\footnote{*जनापति~।}  भवति \begin{tabular}{|c|}१७\\२\\\hline \end{tabular}\begin{tabular}{c|}१७\\२\\\hline \end{tabular}\,, तद्दले \begin{tabular}{|c|}१७\\४\\\hline \end{tabular}\begin{tabular}{|c|} १७\\४\\\hline \end{tabular}\,, एते\renewcommand{\thefootnote}{१४}\footnote{यते~।}  अवधे,}

\newpage

\noindent {समत्वाद्भुजयोर्महती बृहद्भुजे इति नास्ति\renewcommand{\thefootnote}{१}\footnote{नस्ति~।}  अवधासाम्यात्~।}
\vspace{3mm}

{अथ \,लम्बानयनम्\textendash \;अवधावर्गेण \,\begin{tabular}{|c|} २८९\\१६\\\hline \end{tabular} \,भुजवर्गात् \,\begin{tabular}{|c|}२८९\\४\\\hline \end{tabular} \,ऊनादिति \,छेदसाम्यं \,कृत्वा}
{शोधितात्\renewcommand{\thefootnote}{२}\footnote{*तात् \begin{tabular}{|c|}२६७\\१६\\\hline \end{tabular}~।}\begin{tabular}{c}८६७\\१६\end{tabular}, मूलमिति करण्या लब्धो लम्बः\renewcommand{\thefootnote}{३}\footnote{लंवः....
\begin{tabular}{|c|}१८७\\१६\\\hline \end{tabular}~।} \begin{tabular}{c}८६७\\१६\end{tabular}। 
अत्र प्रश्ने लम्बप्रमाणं तत् प्रक्षेपसिद्धं, करणीफलस्य गणितगम्यत्वात्~।}
\vspace{3mm}

{तृतीयमुदाहरणम्\textemdash}

\begin{quote}
    
 {\eg  द्विसमत्र्यश्रस्य फलं पञ्चकबाहोस्त्रिहस्तलम्बस्य\renewcommand{\thefootnote}{४}\footnote{*बार्हस्त्रि*~।}\,।\\
 कथयाष्टभूमिकस्य क्षेत्रविधिं\renewcommand{\thefootnote}{५}\footnote{क्षेत्रस्य विधि~।}  यदि विजानासि~॥~१२६~॥}\end{quote}

{यस्य द्विसमत्र्यश्रस्य भूमिरष्टौ हस्ता लम्बस्त्रिकरः बाहू पञ्चकरौ
तस्य फलं क्षेत्रव्यवहारगणितज्ञ\renewcommand{\thefootnote}{६}\footnote{*तज्ञः~।}  त्वं कथय~।}
\vspace{3mm}

{न्यासः\textendash \,}
\vspace{-1mm}

\hspace{24mm} \includegraphics[width=5.5cm, height=3.5cm]{Images/page-0218as.jpeg}

{करणम्\textendash \,भूः ८, वदनं ०, अनयोः समासः ८, अर्ध\renewcommand{\thefootnote}{७}\footnote{अर्धं ८ अर्धं ४~।}  ४, मध्यमलम्बेन ३
सङ्गुणितम्}
{१२~। अथ स्थूलफलम्\textendash \,बाहुप्रतिबाहुयोगौ ८~। १०, अनयोर्दले ४~। ५, अनयोर्घातः
२०,}
{दूरभ्रष्टं चैतत्~।}
\vspace{3mm}

{अथात्र लम्बानुसरणम्\textendash \,पार्श्वभुजान्तरं शून्यं ०, संयुति १०,
अनयोर्वधः शून्यम् ०,}
{अस्मात् मुखहीनभूकृतिः ६४ अधिकैवेत्यस्ति लम्बावधावाप्तिः, प्रश्नोपलब्ध एव
चात्र लम्बः,}
{सम्प्रति चतुरश्रे त्वाचार्येण प्रतिपदमभिधानात्\renewcommand{\thefootnote}{८}\footnote{*मविधां*~।}\,।}
\vspace{3mm}

{उदाहरणम्\textendash}

\begin{quote}
    
{\eg  वदनं सत्र्यंशकरं भूमिः सत्र्यंश(न)वकरा बाहू~। \\
 पञ्चकरौ\renewcommand{\thefootnote}{९}\footnote{*कराः~।} चतुरश्रे लम्बस्त्रिकरः कियत् गणितम्~॥~१२७~॥}\end{quote}

{यत्र चतुरश्रे मुखमेकहस्तोऽष्टावङ्गुलानि\renewcommand{\thefootnote}{१०}\footnote{चतुरश्र मुखमकहस्ता*~।} भूमिर्नव हस्ता अष्टा(व)ङ्गुलानि भुजौ}
{पञ्चकरो\renewcommand{\thefootnote}{११}\footnote{*करा~।} लम्बश्च त्रयो हस्तास्तत्र क्षेत्रे क्षेत्रफलं कियत्
स्यात्~।}

\newpage

{न्यासः\textendash \,}

\hspace{24mm} \includegraphics[width=6cm, height=4cm]{Images/page-0219as.jpeg}


{कर्म\textendash \,भूः \begin{tabular}{|c|} २८\\३\\\hline \end{tabular}\,, वदनं \begin{tabular}{|c|}४\\३\\\hline \end{tabular}\,, अनयोः
समासः \begin{tabular}{|c|}३२\\३\\\hline \end{tabular}\,, अर्धं \begin{tabular}{|c|} ३२\\६\\\hline \end{tabular}\,, मध्यमलम्बेन ३
गुणितं १६,}
{ह १६ फलम्~। स्थूलफलं तु बाहुप्रतिबाहुयोगौ \begin{tabular}{|c}३२\\३\\\hline \end{tabular}\begin{tabular}{|c|}१० \\ \\\hline \end{tabular}\,, अनयोर्दले
\begin{tabular}{|c|}३२\\६\\\hline \end{tabular}\begin{tabular}{|c|}५\\ \\\hline \end{tabular}\,, अनयोर्घातः\renewcommand{\thefootnote}{१}\footnote{तः ८०~।}}{\begin{tabular}{c}८०\\३\end{tabular}, अतो लब्धं ह २६ अङ्गुलानि १६, दूरभ्रष्टं\renewcommand{\thefootnote}{२}\footnote{दूतम्र*~।}  चैतत्~।}
\vspace{3mm}

{अथात्र\renewcommand{\thefootnote}{३}\footnote{तत्रा चात्रा~।} लम्बानुसरणम्\textendash \,पार्श्वभुजान्तरं ०, संयुतिः १०,
अनयोर्वधः\renewcommand{\thefootnote}{४}\footnote{संयुतिः~।}  (०), मुखहीनभूकृतिः \begin{tabular}{|c|}५७६ \\९\\\hline \end{tabular} अधिकैव, तस्मादस्त्यत्र लम्बावधावाप्तिः~। 
पार्श्वभुजयोः ५~। ५ कृती २५~। २५,}
{अनयोर्विवरं ० भूहृतमिति\renewcommand{\thefootnote}{५}\footnote{क्तस्यत*~।}  खमेव, ऋणं धनं भूमौ \hyperref[21]{'राशिरविकृतः खयोजनापगम'} इति भवति}{\begin{tabular}{c}२८\\३\end{tabular}\begin{tabular}{|c}२८\\३\end{tabular} तद्दले\renewcommand{\thefootnote}{६}\footnote{तद्दले १४६~। १४६~।}\begin{tabular}{c|}१४\\३\end{tabular}\begin{tabular}{c}१४\\३\end{tabular}, एते अवधे,
समत्वाद्भुजयोः {\qt 'महती बृहद्भुजे'} इति नास्ति}
{अवधासाम्यात्\renewcommand{\thefootnote}{७}\footnote{*धातासाम्याश्र~।}\,। आभ्यां लम्बानयनं नैव भवति,
त्र्यश्रस्यैवावाधानयनपूर्वकं लम्बानयनात्~।}
{तथा च तत्करणम्\textemdash}

\begin{quote}
    
{\qt त्र्यश्रस्य पार्श्वभुजयोः कृतिविवरं भूहृतमृणं\renewcommand{\thefootnote}{८}\footnote{क्तस्यतं ऋण~।} धनं भूमौ\renewcommand{\thefootnote}{९}\footnote{भूमा~।}\,।\\
 तद्दलमवधे महती बृहद्भुजे लम्बकस्य विनिपातात्\renewcommand{\thefootnote}{१०}\footnote{लंवकनिपाते~।}\,॥\\
 अवधावर्गेणोनात् भुजवर्गान्मूलमिष्यते लम्बः~।} इति~। \end{quote}

\noindent{तस्मादत्र मुखतुल्यभूवदनायतं चतुरश्रं पृथक् परिकल्प्य
परिशिष्टक्षेत्रखण्डाभ्यां सम्पुटीकृताभ्यां}
{त्रयश्रमुद्भाव्य लम्बानयनम्~। तत्र
मुखतुल्यभूवदनायतचतुरश्रनिष्कर्षणान्मुखं शून्यं जातं भूमिश्च}
{मुखोना, भुजौ तु तावेव, भुजसम्पातप्रदेशावलम्बी लम्बश्च~। एवं\renewcommand{\thefootnote}{११}\footnote{एवं च~।}  सति
समपार्श्वत्वात्} {भूभ्यर्धम् एवावधा तथापि करणं, तत्र न्यासः\textemdash}

\newpage

\includegraphics[width=\linewidth, height=3cm]{Images/page-0220as.jpeg}

{पार्श्वभुजौ ५~। ५, ५, कृती\renewcommand{\thefootnote}{१}\footnote{पार्श्वभुज ५ कृतिः~।} २५~। २५, विवरं ०, भूम्या ८ हृतं ०, ऋणं
धनं भूमौ\renewcommand{\thefootnote}{२}\footnote{भुमौ~।}}
{८~। ८, तद्दलं ४~। ४, अवधे~। अतो लम्बः\textendash \,अवधावर्गेण १६ भुजवर्गात् २५
ऊनात् ९ मूलं}
{३, एष लम्बः~।}
\vspace{3mm}

{यथा चात्र पृथक् खण्डत्रयात् क्षेत्रफलानयनं तथा चाग्रिमोदाहरणे\renewcommand{\thefootnote}{३}\footnote{त्राग्नि*~।} 
टीकाकृदेव प्रकटयिष्यतीत्यलम्~। तथा च क्षेत्राणां न्यासः\textendash}

\hspace{-6mm} \includegraphics[width=\linewidth,height=4cm]{Images/page-0220bs.jpeg}

{तत्र प्रथमे लम्बानयनार्थं न्यासः\textemdash}

\includegraphics[width=\linewidth, height=2cm]{Images/page-0220cs.jpeg}
\vspace{3mm}

{\hyperref[112]{'पार्श्वभुजे'}त्यादिनावधे\renewcommand{\thefootnote}{४}\footnote{*दिवधे~।}\begin{tabular}{c|}९\\५\end{tabular}\begin{tabular}{c}१६\\५\end{tabular}, अतो लम्बः  \begin{tabular}{|c|}१२\\५\end{tabular}\,, क्षेत्रफलं ६~। एवं द्वितीये\renewcommand{\thefootnote}{५}\footnote{*द्वितीयः १२~।}}
{(क्षेत्रफलं ६), मध्यमचतुरश्रे ४, एवं\renewcommand{\thefootnote}{६}\footnote{चैवं~।}  १६~।}
\vspace{3mm}

{उदाहरणम्\textemdash}

\begin{quote}
    
{\eg त्रिगुणास्त्रयोदश धरा पञ्चकृतिर्यस्य बाहुवदनानि~। \\
 लम्बस्त्रयोऽष्टगुणितास्तस्य फलं किं भवेत् कथय~॥~१२८~॥}\end{quote}

\newpage

{यस्य समपार्श्वभुजस्य त्रिसमभुजक्षेत्रस्यैको(न)चत्वारिंशद्धस्ता भूमिः
शेषं तु भुजत्रयं}
{प्रत्येकं पञ्चविंशतिर्हस्ता लम्बश्चतुर्विंशतिहस्तस्तस्य किं
क्षेत्रफलमिति~।}
\vspace{3mm}

{न्यासः\textendash \,}
\vspace{-2mm}

\hspace{24mm} \includegraphics[width=5.5cm, height=3.5cm]{Images/page-0221as.jpeg}
\vspace{3mm}

{करणम्\textendash \,भूः ३९, वदनं २५, अनयोः समासः ६४, अतोऽर्धं ३२,
मध्यमलम्बेन २४}
{सङ्गुणितं ७६८ हस्ताः~।}
\vspace{3mm}

{अथ स्थूलं फलम्\textendash \,बाहुप्रतिबाहुयोगौ ६४~। ५०, अनयोर्दले ३२~। २५,
अनयोर्घातः}
{८००, एतत् द्वात्रिंशद्धस्तैरधिकैः\renewcommand{\thefootnote}{१}\footnote{एतद्य द्वात्रिंशता*~।}  सान्तरमिति दूरभ्रष्टम्~। 
वक्ष्यमाणकरणेन भुजयुतिः ११४,}
{अतो\renewcommand{\thefootnote}{२}\footnote{अत~।}  दलं ५७, चतुर्धा ५७~। ५७~। ५७~। ५७, भुजहीनं १८~। ३२~। ३२~। ३२, एषां
वधः\renewcommand{\thefootnote}{३}\footnote{वाधः ५८९८२~।}}
{५८९८२४, अतः पदं तदेव ७६८~।}
\vspace{3mm}

{अत्रापि प्रश्नोपलब्ध एव लम्बः, तदनुसरणं तु पार्श्वभुजयोरन्तरं ०,
संयुतिः ५०,}
{अनयोर्वधः ०, मुखेन २५ हीनायाः भुवः\renewcommand{\thefootnote}{४}\footnote{मुखेन २५
भुवयुतिः ११४ अतोदल ३९ हीनायाः~।}  १४ कृतिः १९६ अधिकैव,
तस्मादस्त्यत्र लम्बावधावाप्तिः\renewcommand{\thefootnote}{५}\footnote{लंवावावधा*~।}\,। अथ लम्बानयनम्~। तत्र लम्बानयनोपकारिण्यावावाधे समभुजत्वात्
क्षेत्रस्य}
{भूम्यर्धभूमी एव भवतः; योऽसौ लम्बो मुखमध्याद्विलम्बितः, चतुरश्रे हि
मुखस्य सम्भवा(त्)}
{तदेकदेशानां बहुत्वात् प्रभूतेष्टलम्बसम्भवः; यावत्र
मुखस्योभयपार्श्वस्थौ\renewcommand{\thefootnote}{६}\footnote{यावत् मुख्यस्यो*~।}  कोणौ मुखपार्श्वभुजसन्धिव्यपदेशान्तरौ~। इह च भूमिरेकोनचत्वारिंशत्, ततोऽर्धमेकोनविंशतिः
सार्धा भवति,}
{तत्र (न) तया आवाधया लम्बानयनमस्ति त्र्यश्रस्यैवावाधानयनपूर्वकं
लम्बानयनकरणात्~।}
{तथा च तत्करणम्\textemdash}

\begin{quote}
    
{\qt त्र्यश्रस्य पार्श्वभुजयोः कृतिविवरं भूहृतमृणं\renewcommand{\thefootnote}{७}\footnote{अस्यतमृण~।}  धनं भूमौ~। \\
 तद्दलमवधे महती बृहद्भुजे लम्बकस्य विनिपातात्\renewcommand{\thefootnote}{८}\footnote{लंबकनिपाते~।}\,॥\\
अवधावर्गेणोनात् भुजवर्गान्मूलमिष्यते लम्बः~।}\end{quote}

\noindent {इति~। तस्मान्मुखतुल्यभूवदनायतचतुरश्रं पृथक्परिकल्प्य
परिशिष्टक्षेत्रखण्डाभ्यां सम्पुटीकृताभ्यां त्र्यश्रमुद्भावयेत्~। तच्चेदम्\textemdash}

 \newpage

\hspace{30mm} \includegraphics[width=3cm,height=2.3cm]{Images/page-0222as.jpeg}
\vspace{3mm}

{मुखतुल्यभूवदनायतचतुरश्रनिष्कर्षणान्मुखं शून्यं जातं भूमिश्च
मुखोना\renewcommand{\thefootnote}{१}\footnote{मुखेना~।} भुजौ तु तावेव}
{भुज-सम्पातप्रदेशावलम्बी लम्बश्च~। एवं च सति
समपार्श्वत्वाद्भूम्यर्धमेवावधा~।}
\vspace{3mm}

{करणं च\textendash \,त्र्यश्रस्य पार्श्वभुजयोः २५~। २५ कृती ६२५~। ६२५
अनयोर्विवरं शून्यं ०}
{भूम्या १४ हृतं शून्यमेव ०, भूमौ द्विष्ठायां १४~। १४ ऋणं धनं \hyperref[21]{'राशिरविकृतः खयोजनापगम'} इति तथैव स्थितं १४~। १४, तयोर्दलं ७~। ७, अवधे एते~। अतो
लम्बः\textendash \,अवधावर्गेण ४९ भुजवर्गात्\renewcommand{\thefootnote}{२}\footnote{वर्गात् ६५२~।} ६२५ ऊनात् ५७६ मूलं २४ एष\renewcommand{\thefootnote}{३}\footnote{यष~।} लम्बः~। अथवा पृथगेव
व्यवस्थितयोः क्षेत्रखण्डयोभूमिर्भुजः\renewcommand{\thefootnote}{४}\footnote{*मिभुजः~।}, पार्श्वभुजः (कर्णः) अत्र कोटिः स एवेति
लम्बः परामृष्टः~। {\qt 'इष्यते लम्बः'}}
{इत्यादि पूर्वप्रक्रान्तकोट्यानयनम्\renewcommand{\thefootnote}{५}\footnote{इति
दि पूर्वप्रक्रान्तं कोत्यानयन~।}\textendash \,भुजस्य ७ कृत्या ४९
कर्णवर्गात् ६२५ हीनात् ५७६ मूलं}
{तदेव २४ एषा कोटिः भुजाग्रात्कृतो\renewcommand{\thefootnote}{६}\footnote{भुजाग्रात्र वृतो~।} लम्बः इति यावत्~। एतेन लम्बाः पञ्च
सिद्धा भवन्ति\textendash \,क्षेत्रखण्डयोर्द्वयोरपि पार्श्वभुजसिद्धिः, अन्तरायतचतुरश्रे\renewcommand{\thefootnote}{७}\footnote{*रश्रो~।}
पार्श्वभुजसिद्धिः, क्षेत्रखण्डसम्पुटात्मके}
{त्र्यश्रे लम्बसिद्धिः~।}
\vspace{3mm}

{अथ क्षेत्रे खण्डत्रयात्म(के) फलत्रयेण सम्भूय
सकलक्षेत्रफलसंवादसम्पादनसन्दर्शनम्~। तत्र}
{क्षेत्राणां न्यासः\textemdash}

\includegraphics[width=\linewidth,height=3cm]{Images/page-0222bs.jpeg}
\vspace{-1mm}

{अत्र त्र्यश्रस्यैवं शिष्यमाणभूभुजस्य लम्बावधावाप्तिर्नास्ति, यतः
पार्श्वभुजान्तरं १,}
{संयुतिः\renewcommand{\thefootnote}{८}\footnote{संयुति~।} ४९, अनयोर्वधः ४९, मुखहीनभूकृतिः ४९ समैव नाधिका~। अतः
पञ्चविंशतिकां}
{भुवं परिकल्प्य सप्तकचतुर्विंशतिकौ च भुजौ लम्बानयनं क्रियते~। सम्भवति
हि तदा लम्बावधा(वा)प्तिः~। तथा च (न्यासः\textemdash)}

\hspace{30mm} \includegraphics[width=3cm,height=1.5cm]{Images/page-0222cs.jpeg}

\newpage

{पार्श्वभुजान्तरं १७, संयुतिः ३१, अनयोर्वधः\renewcommand{\thefootnote}{१}\footnote{अनयोर्वधः ५२ अस्मान्मे नही*~।}  ५२७,
अस्मान्मुखहीनायाः भुवः २५}
{कृतिः ६२५ अधिकैवेति तत्रावधा आनीयते~। त्र्यश्रस्य पार्श्वभुजयोः ७ । २४
कृती}
{४९ । ५७६, विवरं ५२७, भूम्या\renewcommand{\thefootnote}{२}\footnote{कृती ४९ । ५५६ विवरं ५२७
भूम्या~।} २५ हृतं \begin{tabular}{|c|}५२७\\२५\\\hline \end{tabular}\,, ऋणं धनं भूमौ
\begin{tabular}{|c|}९८\\२५\\\hline \end{tabular}\begin{tabular}{c|}११५२\\२५\\\hline \end{tabular}\,, तद्दलं}
{\begin{tabular}{|c|}४९\\२५\\\hline \end{tabular}\begin{tabular}{c|}५७६\\२५\\\hline \end{tabular}\,, एते\renewcommand{\thefootnote}{३}\footnote{तद्दलं
\begin{tabular}{|c|} ५७६\\२५\\\hline \end{tabular}\begin{tabular}{c|}५७६\\२५\\\hline \end{tabular} पते~।} अवधे, महती (बृहत्)भुजे
स्वल्पभुजे\renewcommand{\thefootnote}{४}\footnote{यन्महतीतद्भुजे~। \begin{tabular}{|c|}२४\\\hline ५७६\\२५\\\hline \end{tabular}\begin{tabular}{c|}७ \\\hline ४९\\२५\\\hline \end{tabular}~।} स्वल्पा~। {\qt 'अवधावर्गेणे'}त्यादिना जातो
{लम्बः \begin{tabular}{|c|}१६८\\२५\\\hline \end{tabular}}~। 
\vspace{3mm}

{अतः क्षेत्रफलान(यन)म्\textendash \,भूवदनसमासः २५, अतोऽर्धं \begin{tabular}{|c|}२५\\२\\\hline \end{tabular}\,,
मध्यमलम्बेन\begin{tabular}{c|}१६८\\२५\\\hline \end{tabular}}
{सङ्गुणितं ८४; एवं द्वितीयखण्डेऽपीति ८४, युतं १६८; मध्यचतुरश्रे
भूवदनसमासः ५०, अर्धं}
{२५, मध्यमलम्बेन २४ सङ्गुणितं ६००, पूर्वस्यां फलयुतौ युतं ७६८, तदेव
सकलं\renewcommand{\thefootnote}{५}\footnote{तदेवं ससं~।} क्षेत्रफलम्~।}
\vspace{3mm}

{अथ त्र्यश्रसम्पुटस्य फलानयनम्\textendash \,भूवदनसमासः १४, अतोऽर्धं ७,
मध्यमलम्बेन २४}
{सङ्गुणितमिति खण्डद्वयफलयुतिसममेव १६८~।}
\vspace{3mm}

{उदाहरणम्\renewcommand{\thefootnote}{६}\footnote{उदहर*~।}\textendash}

\begin{quote}
    
{\eg असमानचतुर्बाहुनि दशहस्ता भूर्मुखं च चत्वारः~। \\
 सषडंशाः पार्श्वभुजा नव षट् च यथाक्रमेणैव~॥~१२९~॥\\
 त्र्यंशविरहितार्धयुता\renewcommand{\thefootnote}{७}\footnote{त्र्यशविहितार्थयुता~।} मध्यमलम्बस्तु षट् कराः सार्धाः~। \\
 अङ्गुलषष्ट्यंशोनाः समलम्बे\renewcommand{\thefootnote}{८}\footnote{अगुलाषष्ठंशोनाः समं लंबे~।} तत्र किं गणितम्~॥~१३०~॥}\end{quote}

{यस्मिंश्चतुरश्रे क्षेत्रे\renewcommand{\thefootnote}{९}\footnote{क्षेत्र~।} भूमिर्दश हस्ताः मुखं पुनश्चत्वारो हस्ता
अङ्गुलानि च चत्वारि}
{एकः पार्श्वभुजो नव हस्ता अष्टाभिरङ्गुलैरूनाः द्वितीयो भुजः षट्
हस्ताः\renewcommand{\thefootnote}{१०}\footnote{बहुस्ता~।} द्वादश चाङ्गुलानि,}
{तस्य विषमभुजस्याङ्गुलषष्टिभागोनद्वादशाङ्गुलसहितषड्हस्तसमलम्बकस्य किं
क्षेत्रफलमिति~।}
\vspace{3mm}

{न्यासः\textendash \,}
\vspace{-2mm}

\hspace{24mm} \includegraphics[width=6.5cm, height=4.5cm]{Images/page-0223as.jpeg}

\newpage

{अत्र \hyperref[41]{'प्राक्छेदांशौ गुणयेत्'} इति सवर्णितो लम्बः \begin{tabular}{|c|}९३५९\\
१४४०\\\hline \end{tabular}~।}
\vspace{3mm}

{करणम्\textemdash \;भूवदनसमासः\,\begin{tabular}{c}८५\\६\end{tabular}, \,अर्धं\,\begin{tabular}{c}८५\\१२\end{tabular}, \,मध्यमलम्बेन \,\begin{tabular}{|c|}९३५९\\१४४०\\\hline \end{tabular} , \,सङ्गुणितं \,यथा \begin{tabular}{|c|}१५९१०३
\\३४५६\\\hline \end{tabular}\,, अतो लब्धं हस्ताः ४६ अङ्गुलानि\renewcommand{\thefootnote}{१}\footnote{अगु*~।}  न किञ्चित्
अङ्गुलांशाः\begin{tabular}{c}१२७\\१४४\end{tabular}। अथ}
{स्थूलफला-नयनम्\textendash \,बाहुप्रतिबाहुयोगौ \begin{tabular}{|c|}
८५\\६\\\hline \end{tabular}\begin{tabular}{c|}९१\\६\\\hline \end{tabular}\,, अनयोर्दले \begin{tabular}{|c|}८५\\१२\\\hline \end{tabular}\begin{tabular}{c|}९१\\१२\\\hline \end{tabular}\,, अनयोर्घातः\begin{tabular}{c}७७३५\\१४४\end{tabular},}
{अतो लब्धं हस्ताः ५३ अङ्गुलानि १७ अङ्गुलभागः\begin{tabular}{c}१\\६\end{tabular}, एतत्स्थूलं फलं
सप्तभिर्हस्तैः षोडशभिरङ्गुलैश्चतु-श्चत्वारिंशदधिकशतभक्तैरेकचत्वारिंशद्भिरङ्गुलभागैरधिकत्वाद्दूरान्तरम्\renewcommand{\thefootnote}{२}\footnote{*कत्वादूरारं~।}\,।}
{वक्ष्यमाणकरणेन भुज-युतिः\begin{tabular}{c}८८\\३\end{tabular}, दलं\begin{tabular}{c}४४\\३\end{tabular}, चतुर्धा
\begin{tabular}{|c|}४४\\३\\\hline \end{tabular}\begin{tabular}{c|}४४\\३\\\hline \end{tabular}\begin{tabular}{c|}४४\\३\\\hline \end{tabular}\begin{tabular}{c|}४४\\३\\\hline \end{tabular}\,,
भुजहीनं}
{\begin{tabular}{|c|}४१\\३\\\hline \end{tabular}\begin{tabular}{c|}६\\१\\\hline \end{tabular}\begin{tabular}{c|}६३\\६\\\hline \end{tabular}\begin{tabular}{c|}४९\\६\\\hline \end{tabular}\,, एषां \,(घातस्य) \,पदम् ४९, अदूरान्तरं \,चैतत्~। एषां च \,त्रयाणां फलानां \,कतमं स्फुटम् इत्यग्रे वक्ष्यामः, सम्प्रति त्वत्र लम्बान्वेषणं
क्रियते~। तद्यथा\textendash \,पार्श्वभुजौ}
{\begin{tabular}{|c|}२६\\३\\\hline \end{tabular}\begin{tabular}{c|}१३\\२\\\hline \end{tabular} अनयोरन्तरं संयु-तिश्च\begin{tabular}{c}१३\\६\end{tabular}\begin{tabular}{c}९१\\६\end{tabular}, तद्वधः\begin{tabular}{c}११८३\\३६\end{tabular}, अस्मान्मुखेन\renewcommand{\thefootnote}{३}\footnote{तद्वधः\begin{tabular}{c}११८३\\८६\end{tabular}तस्मात्
मुखेन~।}\begin{tabular}{c}२५\\६\end{tabular}हीनायाः}
{भुवः \begin{tabular}{|c|} ३५\\६\\\hline \end{tabular} कृतिः \begin{tabular}{|c|}१२२५\\३६\\\hline \end{tabular} अधिकैव
तस्माद(त्र) लम्बावधावाप्तिः~।}
\vspace{3mm}

{इदानीमजात्यक्षेत्राणां गजदन्तादीनां\renewcommand{\thefootnote}{४}\footnote{*दीनां करणादीनां~।} करणान्यतिदेष्टुमार्यामाह\textemdash}

\begin{quote}
    
{\bs त्रिभुजं गजदन्ताकृति नेम्याकारं\renewcommand{\thefootnote}{५}\footnote{...स्याकारं~।} चतुर्भुजं क्षेत्रम्~। \\
 बालेन्दौ त्रिभुजे द्वे वज्रे च चतुर्भुजद्वितयम्~॥~११६~॥}\end{quote}

{मुख्यसमनन्तरा(द)न्यस्योपजीवनावसर इति त्र्यश्रचतुरश्रफलकरणानन्तरमेव
तदतिदेशभाजामवसरो दत्तः न दशक्षेत्रफलकरणसमाप्तौ
वृत्तचापयोर्गजदन्तनेमिबालेन्दुवज्रैरनुपजीवनात्~। त्रिभुजचतुर्भुजाभ्यामेव च ते उपजीव्येते इति तदन(न्त)रमेव तयोः
क्षणः}
{प्रणीतः~।}
\vspace{3mm}

{गजदन्ताकृतिक्षेत्रफलानयनं त्रिभुजफलकरणेन साध्यं,
नेम्याकारक्षेत्रफलानयनं चतुर्भुजफलकरणेनेति\renewcommand{\thefootnote}{६}\footnote{चातुर्भु़ज*~।}\,। समस्तस्यासमस्तकल्पना\renewcommand{\thefootnote}{७}\footnote{*स्यसमस्त*~।} तु तथेति पश्चात्क्रियते
बालेन्दुसदृशसन्निवेशवति}
{क्षेत्रे\renewcommand{\thefootnote}{८}\footnote{क्षेत्र~।} विष्कम्भगत्या मध्यतो द्विधाकृते\renewcommand{\thefootnote}{९}\footnote{*कृतेः~।} त्रिभुजद्वयकल्पनया
तत्करणेनेति पूर्ववत्\renewcommand{\thefootnote}{१०}\footnote{पूर्वत्~।}, वज्राकृतिसन्निवेशयुक्ते क्षेत्रे चतुर्भुजद्वयकल्पनया तत्करणेनेति पूर्ववत्~।}
\vspace{3mm}

{उदाहरणम्\textendash}

\begin{quote}
{\eg  गजदन्ते द्विकरधरे त्रिहस्तलम्बे च किं गणितम्~।}
\end{quote}

\newpage

{यस्य भूमिर्द्वौ हस्तौ लम्बश्च त्रयो हस्तास्तत्र गजादन्ताकृतिक्षेत्रे
किं फलम्\,?}
\vspace{3mm}

{न्यासः\textendash \,}
\vspace{-1mm}

\hspace{24mm} \includegraphics[width=2.5cm, height=3.5cm]{Images/page-0225as.jpeg}
\vspace{3mm}

{करणम्\textendash \,भूः २, वदनं\renewcommand{\thefootnote}{१}\footnote{?} ०, समासः २, अर्धं १ मध्यलम्बेन\renewcommand{\thefootnote}{२}\footnote{?} ३
सङ्गुणितं लब्धं}
{हस्ताः ३~।}
\vspace{3mm}

 {उदाहरणम्\textendash}

\begin{quote}
    
{\eg नेम्याकृतिनि त्रिकरकभूमुखदशहस्तलम्बे च~॥~१३१~॥}\end{quote}

{त्रिहस्तभूमिके त्रिहस्तवदने दशहस्तलम्बे च नेम्याकृतिनि क्षेत्रे (किं)
फलं भवतीति~।}
\vspace{3mm}

{न्यासः\textendash \,}
\vspace{-1mm}

\hspace{24mm} \includegraphics[width=4cm, height=5.5cm]{Images/page-0225bs.jpeg}
\vspace{3mm}

{करणम्\textendash \,भूः ३ वदनं ३, समासः ६, अर्धं ३, मध्यलम्बेन १०, सङ्गुणितं ३०~।}
\vspace{3mm}

{उदाहरणम्\textendash}

\begin{quote}
    
{\eg मध्यायामोऽष्टकरो\renewcommand{\thefootnote}{३}\footnote{?} बालेन्दौ मध्यविस्तरस्त्रिकरः~।\\
 त्रिभुजद्वयकल्प(नया) किं (ग)णितं कथय तत्राशु~॥~१३२~॥}\end{quote}

{यस्य\renewcommand{\thefootnote}{४}\footnote{?} बालेन्दुरूपस्य क्षेत्रस्य मध्यायामो दैर्घ्यतो मध्यलम्बोऽष्ट(करः) मध्यविष्कम्भस्त्रिहस्तः तस्य त्रिभुजकल्पने द्वे कृत्वा पृथक् पृथक् फलसमासः कः भवति~।}

\newpage

{न्यासः\textendash }
\vspace{-1mm}

\hspace{24mm} \includegraphics[width=4cm, height=2cm]{Images/page-0226as.jpeg}
\vspace{3mm}

{करणम्\textendash \,अत्र मध्यविस्तरस्य भूमानत्वं, तेन तत एव विभज्य
क्षेत्रद्वयकरणात् न्यासः\textendash }
\vspace{3mm}

\hspace{-20mm} \includegraphics[width=\linewidth,height=3cm]{Images/page-0226bs.jpeg}
\vspace{1mm}

{भूः ३, वदनं ०, समासः ३, अर्धं\begin{tabular}{c}३\\२\end{tabular}, मध्यलम्बेन ४ सङ्गुणितं ६,
एतत्\renewcommand{\thefootnote}{१}\footnote{एत~।} द्विगुणं कृत्वा बालेन्दुक्षेत्रफलम् १२~।}
\vspace{3mm}

{अन्ये मध्यविस्तारस्यार्धं भूः वदनं च प्रकल्प्य
पार्श्वभुजाभ्यामष्टकाभ्यामायतचतुरश्रात्फलम् आनयन्ति, यथा}
\vspace{3mm}

{न्यासः\textendash \,}
\vspace{-3mm}

\hspace{24mm} \includegraphics[width=1.5cm, height=4cm]{Images/page-0226cs.jpeg}
\vspace{2mm}

{लम्बेन चात्र पार्श्वभुजतुल्येन भाव्यं, तेन करणम्\textendash \,भूः\begin{tabular}{c}३\\२\end{tabular}, वदनं\begin{tabular}{c}३\\२\end{tabular}, समासः ३, अर्धं\begin{tabular}{c}३\\२\end{tabular} लम्बेन\renewcommand{\thefootnote}{२}\footnote{बेन~।} ८ सङ्गुणितं १२~।}
\vspace{3mm}

{उदाहरणम्\textemdash}

\begin{quote}
    
{\eg वज्रे मध्यायामो दशहस्तः पञ्चहस्तके वदने~। \\
 मध्यव्यासो\renewcommand{\thefootnote}{३}\footnote{मधूव्यासो~।} द्विकरः कियत् फलं त(स्य) द्विचतुरश्रात्~॥~१३३~॥}\end{quote}

\newpage

{वज्राकृतिक्षेत्रस्य यस्य मध्यतो दैर्घ्यं दश हस्ताः वदनं
पञ्चहस्ताः\renewcommand{\thefootnote}{१}\footnote{दश मासाः....दे वदने पंचहस्त~।} विष्कम्भश्च द्विकरस्तस्मिन् चतुरश्रद्वयकल्पनया\renewcommand{\thefootnote}{२}\footnote{*कल्पनाया~।} पृथक् पृथक् फलसमासः कियान्
भवतीति~।}
\vspace{3mm}

{न्यासः\textendash \,}
\vspace{-2mm}

\hspace{24mm} \includegraphics[width=2.5cm, height=3.5cm]{Images/page-0227as.jpeg}
\vspace{3mm}

{मध्यविभागान्मध्यविष्कम्भस्यार्धत्वे \,लम्बस्यार्धत्वे \,वदनप्रमाणेन\renewcommand{\thefootnote}{३}\footnote{मध्यव्या
किय....तिति~। विष्कंभस्यत्वे लंबस्यार्धत्वे वदनप्रणाणेन~।} \,तथैवार्धे \,स्थिते \,चतुरश्रे न्यासः\textendash \,}
\vspace{-2mm}

\hspace{24mm} \includegraphics[width=2.5cm, height=3.5cm]{Images/page-0227bs.jpeg}
\vspace{3mm}

{करणम् \textendash \,भूः\begin{tabular}{c}७\\२\end{tabular}, वदनं\begin{tabular}{c}७\\२\end{tabular}, समासः ७, अर्धं\begin{tabular}{c}७\\२\end{tabular},
मध्यलम्बेन ५ सङ्गुणितं\begin{tabular}{c}३५\\२\end{tabular}, एतत्\renewcommand{\thefootnote}{४}\footnote{भूः\begin{tabular}{c}५\\२\end{tabular}वदनं\begin{tabular}{c}५\\२\end{tabular}समासः ५ अर्धं\begin{tabular}{c}५\\२\end{tabular}मध्यलंबेन ५\begin{tabular}{c}२५\\२\end{tabular}एत~।}}
{द्विगुणं वज्रफलम्\renewcommand{\thefootnote}{५}\footnote{फलं २५~।} ३५~।
\vspace{3mm}

{एषां गजदन्तादीनां लब्धक्षेत्रफलानि भुजवक्रतया स्थूलानि,
तदन्तःपरिकल्पितत्र्यश्रचतुरश्रचापगणनया तु सूक्ष्माणि साधयितव्यानि~। तथा
चोद्धृतगजदन्तेऽन्तस्त्र्यश्रसाधनम्\textendash \,या तावदत्र भूस्तदर्धमावाधा सा च भुजः, लम्बः कोटिः,
त्रिभुजपार्श्वभुजौ\renewcommand{\thefootnote}{६}\footnote{कोटिः सार्धा कोटिरियं भुज*~।} तु कर्णौ ता(वे)व जातौ~। अतो भूम्यर्धभुजायाः कृतिः १, कोटिकृतिः ९, युतिः १०, अतो मूलं क
१०,}
{एतदुभयोरपि पार्श्वयोः कर्णानयनं\renewcommand{\thefootnote}{७}\footnote{*नयन~।} कर्णेनानीतौ भुजाविति
सिद्धमन्तस्त्र्यश्रक्षेत्रम्~। कर्णसूत्रद्वयोरभ्याशे\renewcommand{\thefootnote}{८}\footnote{*सूत्रदूर्वाभ्यासे~।} च बहिरुभयतश्चापद्वयं जायते ययोस्तौ करणौ ज्यास्थाने
भवतः~। तत्र}
{गजदन्तभवस्य (चापस्य) त्र्यश्रभुजस्य च मध्यान्तरसूत्रं\renewcommand{\thefootnote}{९}\footnote{*त्र्श्रभुजमध्यन्तरो सूत्र~।} शरः, तेन
चापक्षेत्रफलद्वयसंयुतिमन्तस्त्र्यश्रे फले संयोज्य स्फुटं फलं भव(ती)ति~। दिशा
चानयान्यदप्यवगन्तव्यम्~।}

\newpage

{श्रव(ण)लम्बप्रमाणापरिज्ञानेऽपि त्र्यश्रचतुरश्रक्षेत्रफलानयने
करणसूत्रमार्यामाह\textemdash}

\phantomsection \label{117}
\begin{quote}
  
 {\bs भुजयुतिदलं चतुर्धा भुजहीनं तद्वधात्पदं गणितम्~। \\
 सदृशासमलम्बानामसदृशलम्बे विषमबाहौ~॥~११७~॥} \end{quote}

 {यानि सदृशानि\renewcommand{\thefootnote}{१}\footnote{यानि सदृशनि~।} तुल्यसकलभुजानि त्र्यश्राणि चतुरश्राणि च
क्षेत्राणि, यान्यपि}
{चासदृशलम्बानि\renewcommand{\thefootnote}{२}\footnote{*शसेमेल*~।}  भूमुखवर्जं द्विसमभुजानि भूवर्जं\renewcommand{\thefootnote}{३}\footnote{भूवलं~।}  च त्रिसमभुजानि
तथान्यानि यानि}
{विष(म)भुजानि च विषमलम्बानि, यदि वासमलम्बान्यपि यावद्विषमबाहूनि, तेषु
समस्तेषु}
{क्षेत्रेषु इदं सूक्ष्मगणितं यत् त्रिभुजे त्रयाणां चतुर्भुजे चतुर्णां
भुजानां योगस्य दलं चतुर्षु}
{स्थानेषु न्यस्तं त्रिभुजप्रस्तावे त्रिभुजैः यथास्थानं हीनमहीनमेकत्र,
चतुरश्रप्रसङ्गे चतुर्षु}
{स्थानेष्वेकैकस्मादपास्तैकैकभुजं विभागस्थित्या\renewcommand{\thefootnote}{४}\footnote{त्रिभाग*~।}  परस्परहतं तस्य
मूलमिति\renewcommand{\thefootnote}{५}\footnote{स्वमूलविभक्तमिति~।}\,।}
\vspace{3mm}

{उदाहरणम्\textemdash}
\vspace{2mm}

{पूर्वत्रोदाहृतान्ये(व) क्षेत्राणि~।}
\vspace{3mm}

{अथ यत्र पदग्राह्यो राशिरमूलदो भवति न च
वर्गप्रकृतिगणितयोरवसरश्च\renewcommand{\thefootnote}{६}\footnote{*तिर्गणितचरश्च~।}  (तत्र)}
{करण्यन्तं गणितं कार्यमिति रूपेष्वत्र
मन्तव्येष्वासन्नमूलग्रहणज्ञापनार्थमार्यामाह\textemdash}
    
\begin{quote}

{\bs राशेरमूलदस्याहतस्य\renewcommand{\thefootnote}{७}\footnote{*दस्णह*~।} वर्गेण केनचिन्महता~। \\
मूलं शेषेण विना विभजेत् गुणवर्गमूलेन~॥~११८~॥}\end{quote}

{गृहीतशिष्यतो\renewcommand{\thefootnote}{८}\footnote{ग्रहीतशिष्यतो~।} मूलराशेरमूलदस्य सुव्यक्तमासन्नं\renewcommand{\thefootnote}{९}\footnote{*क्तप्रमास*~।} मूलं ग्राह्यं
यथा तस्य राशेः केनचिन्महता वर्गेण हतस्य मूलविधिना मूलं गृहीत्वा शेषं त्यजेत् (प्राप्तं
मूलञ्च गुणवर्गमूलेन}
{विभजेत्)~।}
\vspace{3mm}

{अथ (पञ्चाश्रिणि त्रीणि त्रिभुजानि, तेषां फलानि क ३, क ३, क ३)~। (अत्र)
सङ्ख्या}
{३ {\qt 'करणीनां कृतिः कार्या राशेस्तु तुल्यतां गते'}ति\renewcommand{\thefootnote}{१०}\footnote{करणीनां क्रिया कार्या गतेतिवर्गितां...राशिनेकेन...हता २७~।} वर्गिता ९,
राशिनैकेन ३ हता २७, एषां युतिः}
{क २७, एतत् पञ्चाश्रिणः क्षेत्रफलं, करण्यन्तं गणितं न भवतीति
करणीत्वनिवृत्या प्रत्यासन्नमूलग्रहणार्थं सहस्रवर्गेण प्रयुतेन १०००००० गुणितस्य २७००००००
मूलं\renewcommand{\thefootnote}{११}\footnote{मूलं ५०९६~।} ५१९६, गुणवर्गमूलेन सहस्रेण १००० विभक्तं\begin{tabular}{c}५१९६\\११००\end{tabular}, अतो लब्धं हस्ता ५
अङ्गुलानि ४ अङ्गुलभागाः}
\begin{tabular}{|c|}८८\\१२५\\\hline \end{tabular}~। षडश्रिणि खल्वप्यत्र चत्वारि त्रिभुजानि (सम)द्विबाहुकान्युत्पद्यन्ते, तेन प्राग्वच्चत्वारि}
{त्र्यश्रफलानि क ३, क ३, क ३, क ३, प्राग्वदेव युतिः क ४८, अतः प्राग्वदेव
प्रत्यासन्नं}
{मूलं$^{\scriptsize{\hbox{{\color{blue}११}}}}$ ग्राह्यं स्थूलसूक्ष्मफलयोश्च दूरा(ल्पा)न्तरत्वं
निरूप्यम्~।}

\newpage

{सप्ताश्रिणि मध्यचतुरश्रस्य कर्णानयनेन त्र्यश्रानयनम्, तेन च
सर्वे सम(द्वि)भुजेऽस्तं}
{(न) यान्त्येवेति\renewcommand{\thefootnote}{१}\footnote{न्येवेति~।} पृथक्तयोः फलमानेयम्~। अपरे हृदयात्कोणप्रापिभिः
सूत्रैः त्र्यश्राण्युत्पादयन्ति,}
{तथा-त्वेऽश्रिसम्मितान्येव\renewcommand{\thefootnote}{२}\footnote{तथात्वश्रिम्मि*~।} त्र्यश्राणि द्विभुजसमान्येव जायन्ते
कोणहृदयस्पृशां सर्वेषामपि}
{सूत्राणां सम-त्वादित्यलमतिविस्तरेण\renewcommand{\thefootnote}{३}\footnote{*लमिति*~।}\,॥~शम्~॥}
\vspace{15mm}

\begin{center}
    \rule{7em}{.8pt}
\end{center}


\vfil




\newpage
\thispagestyle{empty}


 \begin{center}  
\textbf{\large शुद्धिपत्रम्}
\end{center}

\renewcommand*{\arraystretch}{0.8}
\begin{tabular}{p{0.5cm} p{0.5cm} p{4cm} p{4.7cm}}
पृष्ठ & पङ्क्ति & अशुद्ध & शुद्ध\\
\\
२१ & १३ & *द्वयम्' & *त्रयम्\\

३६ & २३ &\renewcommand*{\arraystretch}{0.7}\begin{tabular}{c|c|c|c|c|}
१ & १ & १ & १ & १ \\           
२ & ४ & १ & २ & ३ \\ 
{} & १ & ३ & १ & १ \\
{} & ४ &  & २ & २ \\
\end{tabular}   &  \renewcommand*{\arraystretch}{0.7}\begin{tabular}{c|c|c|c||l|}
 १ & १ & १ & १ & १ \\
 २ & ४ & १ & २ & ३ \\
 {} & १ & ३ & १ & १$+$\\
{} & ४ & {} & २ & २ 
\end{tabular}\\

\renewcommand*{\arraystretch}{0.8}३९ & २५ & $\begin{matrix}
\mbox{{१९५}}\\
\mbox{{३९१}}
\end{matrix}$ & $\begin{matrix}
\mbox{{१५५}}\\
\mbox{{३०१}}
\end{matrix}$\\

३९ & २६ & चतुर्भिः कुडवैः प्रस्थ इति चतुर्गुणिते भाज्ये $\begin{matrix}
\mbox{{७८०}}\\
\mbox{{३०१}}
\end{matrix}$ & चतुर्भिः प्रस्थैः आढक इति चतुर्गुणिते भाज्ये $\begin{matrix}
\mbox{{७८०}}\\
\mbox{{३०१}}
\end{matrix}$~। लब्धं प्रस्थौ २ प्रस्थशेषम् $\begin{matrix}
\mbox{{१७८}}\\
\mbox{{३०१}}
\end{matrix}$ प्रस्थाभावात् कुडवा लब्धव्याः~। चतुभिः कुडवैः प्रस्थ इति चतुर्गुणिते भाज्ये $\begin{matrix}
\mbox{{७१२}}\\
\mbox{{३०१}}
\end{matrix}$\\

३९ & २७ & $\begin{matrix}
\mbox{{१७८}}\\
\mbox{{३०१}}
\end{matrix}$
& $\begin{matrix}
\mbox{{११०}}\\
\mbox{{३०१}}
\end{matrix}$\\

४२ & १७ & *च्छेदाव्यय* & *च्छेदायव्यय*\\

४८ & १ & $\begin{matrix}
\mbox{{१६}}\\
\mbox{{३९}}\\
\mbox{{~५}}
\vspace{1mm}
\end{matrix}$  & $\begin{matrix}
\mbox{{१६}}\\
\mbox{{~१}}\\
\mbox{{३९}}\\
\mbox{{~५}}
\end{matrix}$ \\

५० & ४ & द्रव्याथ & द्रव्यार्थे \\

५५ & २५ & मासशेषयुग् & मूलशेषयुग्\\

६० & २३ & फलभक्तो$^{\scriptsize{\hbox{{\color{blue}५}}}}$ & फलभक्तो$^{\scriptsize{\hbox{{\color{blue}६}}}}$\\

६३ & २५ & त्र्यंशशषडांर्धा* & त्र्यंशषडंशार्धा*\\

६९ & ९ & तद्धि शुद्ध* & तद्विशुद्ध*\\

६९ & २४ & कनकाभ्य$^{\scriptsize{\hbox{{\color{blue}३}}}}$ & कनकाभ्यां\\

७३ & १२ & एवं$^{\scriptsize{\hbox{{\color{blue}१०}}}}$ & एष$^{\scriptsize{\hbox{{\color{blue}१०}}}}$\\

७३ & १४ & *प्रक्षेपात्फलेन & *प्रक्षेपान् फलेन\\

७६ & २० & *करणतुल्यानि & *करणमूल्यानि\\

७६ & २३ & *मच्छेनस्य" & *मच्छेदनस्य\\

७७ & १ & *सङ्याया' & *सङ्ख्याया\\

७९ & ९ & समर्घाणि & समर्घाणि\\

८९ & २७ & वेतम* & वेतनम*\\

९३ & १८ & *मोपन्ना* & *मोपपन्ना*\\

१०१ & १८ & दृश्यां & दृश्यं\\

१०५ & १४ & पञ्चवाशिष्टानि & पञ्चावशिष्टानि\\

११३ & १६ & प्रथमपदस्य~। रूपद्वयस्य & प्रथमपदस्य रूपद्वयस्य\\

११४ & ९ & संयोग$^{\scriptsize{\hbox{{\color{blue}६}}}}$ & सङ्गम$^{\scriptsize{\hbox{{\color{blue}६}}}}$\\

११७ & ४ & न्यूने & द्व्यूने\\

१२० & २५ & वा गुणितस्यैव & वागणितस्यैव\\

१३४ & १६ & तद्रूप* & तदुप*\\

१३५ & १४ & सूत्रपद* & सूत्रं पद*\\
 
१३६ & ६ & ततो परे & ततोऽपरे\\

\end{tabular}

\newpage
\thispagestyle{empty}

\phantomsection \label{anu}
\begin{center}
\textbf{ग्रन्थेऽस्मिन् प्रयुक्तपारिभाषिकशब्दानाम्}

\vspace{2mm}

\textbf{\large अनुक्रमणिका}
\end{center}

\begin{center}
\begin{tabular}{p{4.7cm} p{4.7cm}}
अंश २३, २६, २८-३२, ३४, & अवनि १०९\\ 
~~~~ ३५, ७१  & अवलम्ब १०९, १५६\\
अंश (= सुवर्णखण्ड) ७१ & असदृशलम्ब १७५\\
अंश (= भाग) ७५ & असमलम्ब १७५\\
अंशक २९ & असमानचतुर्बाहु १७०\\
अग (=७) ७२ & अहोरात्र ६\\
अगुण ८० & आढक ५\\
अधनपद २१  & आदि १९, ११०, ११३, ११८, १२०,\\
अङ्गुल ६, ४०, १६४, १७० & ~~~~~१२५, १२६, १३०, १३४, १३६\\
अनष्ट १२६, १३०  & ~~~~~१३९, १४१, १४५, १४८, १५३\\
अनुलोममार्ग १३ & आद्य ४२\\
अन्तर्भृति ९६ & आयतसमचतुरश्र १५६, १६१\\
अन्त्य १९, २१, ४२, १३७ & आयराशि २५\\
अन्त्य (अङ्कस्थानसञ्ज्ञा) ५ & आयाम ४४, ४८, ४९\\
अन्त्यपद १६ & आवर्त ६८\\
अन्त्यार्घ ७६ & आहति १३६\\
अन्यजाति ३७ &इच्छा ३७,८०\\
अन्यपक्ष ४५ &इष्ट १६, १०९, १४१\\ 
अपगम १४ &  इष्टादिचय १०७\\
अब्ज ५ &   इष्टावलम्ब १०९\\
अभिन्नजाति ३७ &  ईप्सितकाल ५५\\
अभ्यस्त २६ &  उत्तर ११३, १२०, १३९, १४१, १५३\\
अभ्यास ३०, ३१,४५, १०५ & उत्सार्य १३\\
अयुत ५ & उद्धृत ५८ ७१\\
अर्ध ७७-७९ & उपनय ५५\\
अर्बुद ५ & ऋजुगति १५६\\
अवधा १५६  & ऋण ३५
 
\end{tabular}
\end{center}

\afterpage{\fancyhead[CO] {\s अनुक्रमणिका}}
\afterpage{\fancyhead[CE] {\s अनुक्रमणिका}}

\newpage

\begin{center}
\begin{tabular}{p{5.2cm} p{4.5cm}}
एक ५  & गजदन्त १५६, १७१\\

एकदिनगति ४१ & गजदन्ताकृति १७१\\

एकपत्र ५६, ६० & गणित १, ६, ४२, ११०, १२६, १३०,\\

एकाद्युत्तरविधि ९७ & ~~~~~~१३६\\

कनक ६८ &  गणित (=क्षेत्रफल) १६१, १६५, १७०,\\

कर ६, १६१, १६४, १६५, १७०-१७३ &  ~~~~~१७१, १७२, १७५\\

कर्ष ५,३७ & गतकालफल ५८\\

कलासवर्ण २ & गति ४१, ८४, ८६, ८७, १३९\\

कल्याणसुवर्ण ४७ & गति (= सङ्कलित) १३९\\

कवाटसन्धि १३ & गतिविशेष ८७\\

काकिणी ५, ३५ & गम (=घटाना) १०२\\

काञ्चन ६७ & गुञ्ज ५\\

काल ४६, ५३-५६, ५८, ६०, ८४, ८६ & गुञ्जा ४७\\

कुडव ५, ३९ & गुडिका ७१, ७२\\

कुतप ८८, ११६ & गुण २६, ३२, ६०, ६३, ७१, ७५,८०\\

कृति १६, १७, १९, २७, १०४, १०६, & ~~~~८७, १३४, १३६, १७५\\

~~~~ १२०, १३०, १३४, १४८, १५२, & गुणराशि १३\\

~~~~ १५६, १६७ & गुणित ४२, ७१\\

कृतिसङ्कलित १५२ & गुण्य १३\\

कोटि ५ & घटी ६\\

क्राकच २ & घन २, १९-२१, २८, १४८, १५१-१५३\\

क्रोश ६, ९०, ९६, ११६ & घनपद २१, २८\\

क्षेत्र २, १५६, १६१, १७१ & घनमूल २, २८\\

क्षेत्रविधि १६५ & घनयोग १५३\\

क्षेप १४, १०५ & घनसङ्कलित १५०, १५१\\

ख १४ & घात ५४, ६०, ९५\\

खण्डकरण १३ & चतुरश्र १५६, १६५, १७३\\

खर्व ५ & चतुर्भुज १७१\\

खात २ & चय ६, ७, ११, १६, १९, २४, १०८-\\

खारी ५, ३९, ४८ & ~~~~११०, ११८, १२०, १२६, १२८-\\

गच्छ ८, १०, १२, १०७, १२०, १२९, & ~~~~१३०, १४१, १४५, १४७, १५१-\\

~~~~~१३०, १४५, १५२, १५३ & ~~~~१५३\\

\end{tabular}
\end{center}

\newpage

\begin{center}
\begin{tabular}{p{5cm} p{4.7cm}}
चाप १५६ & निखर्व ५\\
चिति २ & निजकाल ५३\\
छाया २ & निरंशरूप १०३\\

छेद २६, २८, २९, ३१, ३२, ३४, ३५, & निर्युक्तराशि १९\\

~~~~४५, ७८ & निर्विकलपद १२६, १२८\\

छेदगम ३१, ७६ & नेमि १५६\\
छेदन २३ & नेम्याकृति १७२\\
छेदमूल २८ &  नेम्याकार १७१\\

छेदसमत्व ७८ & पक्व ६४\\

जीवविक्रय २ & पङ्क्ति (=१०) ७२\\

& पञ्चसप्तनवराशि २\\

तत्स्थ ३१ & पण ५, ३५, ३७, ३८, ४८, ५०, ७५,\\

ताडित ६८ & ~~~~८८, ११६, १३७\\

तुल्यच्छेद २५, २८, ७८ & पण्य ७५, ७६\\

तुल्यत्रिभुज १६४ & पद (=मूल) १०३, १०५, १०६, १७५\\

तुल्यहर ७६ & पद (=गच्छ) ६, ७, ९, ११, १०८, ११०,\\

त्रिचतुर्भुज १५६ & ~~~~११३, ११८, ११९, १२५, १२८,\\

त्रिभुज १७१, १७२ & ~~~~१२९, १३४, १४१, १४३, १४८,\\

त्रिराशि ३७ & ~~~~१५०, १५३\\

त्रैराशिक २ & पद (=स्थान) १६, २१\\

दण्ड ६ & पदवर्ग १५०, १५१\\

दल ३७ & परार्द्ध ५\\

दश ५  & परिकर्म २\\

दिवाकर (= १२) ७२ & परिपक्ववह्निभव ६६\\

दृश्य १०१, १०३, १०५ & पल ५, ३७, ३८, ६७, ७६, ८८\\

दैर्घ्य ४९ & पाद ३३, ३४\\

द्रोण ३९, ४८, ४९ & पार्श्वभुज १५६, १६१, १७०\\

द्वित्रिसमभुज १५६ & पिण्ड ४९\\

द्विसमत्र्यश्र १६५ & पुराण ५, ३५, ५२\\

घन ३५, १२८-१३० & पूर्वाग्न १०५\\

धनयोग ५८ & प्रक्षेप ७३\\

धरा १०८, १०९, १६२, १६७, १७१, & प्रचय ११९, १२०, १४५, १५३\\
\end{tabular}
\end{center}

\newpage

\begin{center}
\begin{tabular}{p{5cm} p{4cm}}
प्रतिबाहु १५६ & भागापवाहजाति ३४\\

प्रतिलोम १४ & भागोनरूप १०४\\

प्रत्युत्पन्न २, १३, २६ & भाटक ८७, ८८, ९५, ११६\\

प्रत्युत्पन्नविधान १३ & भाण्ड ९५\\

प्रभव १२८, १४५ & भाण्डप्रतिभाण्ड ५०\\

प्रभाग २ & भाव्यक ५५\\

प्रभागजाति ३१ & भास्कर (= १२) ७२\\

प्रमाण ३७, ५४, ६० & भिन्न २\\

प्रमाणराशि ५३ & भुज १५६, १६२, १७५\\

प्रयुत ५ & भू १०८-११०, १५६, १६१, १७०,\\

प्रस्थ ५, ७३, ७४  & ~~~१७२\\

फल ३७, ४५, ४६, ५३-५५, ५८, ६०, & भूमि १०८, १६१, १६४, १६५\\

~~~~~७३, ७४, ८८, १०७, ११०, ११९,& मध्य ४२\\

~~~~~१३४, १३६ & मध्य (अङ्कस्थानसञ्ज्ञा) ५\\

फल (=सङ्कलित) १२०, १२५, १४१, & मध्यमलम्ब १६१, १७०\\

~~~~१४५, १५३ & मध्यलम्ब १६१\\

फल (=क्षेत्रफल) १६१, १६२, १६५, १६७, & मध्यविस्तर १७२\\

~~~~~१७३ & मध्यव्यास १७३\\

बालेन्दु १७१, १७२ & मध्यायाम १७२, १७३\\

बाहु १०८, १५६, १६१, १६५, १६७ & महासरोज ५\\

व्रीहि ४८ & मानान्तर ४२\\

भक्त २६, २७, ४२, ६०, ७१ & मूल २, ८, १०, १२, १८, २१, २७\\

भाग २, १४, ३६, ७१, ६४ & ~~~~२८, ६५, १०३-१०५, १२०,\\

भागजाति २८ & ~~~~१३०, १७५\\

भागभाग २, ३१ & मूल (=मूलधन) ४६, ५३-५५,\\

भागभागविधि ३१ & मूलासन्न १०३\\

भागमाता २,३६ & मूल्य ५०, ७५\\

भागहार २ & यव ६९\\

भागहारविधि १४ & युत ६२\\

भागानुबन्ध २ & युति १९, २३, ७१, ७३, १७५\\

भागानुबन्धजाति ३२ & युतिवर्ण ६८\\

भागापवाह २ & योग ३३, ३४, ६३, ८६, ८७\\

\end{tabular}
\end{center}


\newpage

\begin{center}
\begin{tabular}{p{5cm} p{5cm}}
योजन ६, ४०, ४१, ४८, ८५, ८६, & वर्ष ६, ४६\\

~~~~~~८८, ११६, १३९ & वल्याः सवर्णन ३५\\

रक्तिक ४३ & बहुराशिपक्ष ४५\\

राशि १४ & विकल १२६, १२८, १३०\\

राशि (व्यवहार) २  & विपक्व ६६\\

रूप १० &  विपक्वकनक ६४\\

रूपविभाग १३ & विपरीत १०६\\

रूपादिचय ९७ & विपरीतीकृत ५०\\

लक्ष ५ &  विभाज्य १४\\

लम्ब १०८-११०, १५६, १६१, १६२, &  विलोमगति १३\\

~~~~१६५, १६७, १७१, १७२ &  विवर ११, १४१\\

लम्बक १०७, १०९, १६४ &  विश्लेष १४१\\

लेखक ५५, &  विषम १३४\\

वक्त्र १०८, १६१ & विषमचतुरश्र १५६\\

वक्रगति १५६ & विषमपद १८\\

वज्र १७१, १७३ & विषमबाहु १७५\\
 वज्रवत् १०८ &  विष्कम्भ ४४\\

वदन १०९, १५३, १६१, १६५, १६७, &  विस्तार १०७\\

~~~~~१७३  & वृत्त १५६\\

वध १६, ३०, ३४, ६४, ६७, ६८ & वृत्ति ५५\\

वराटक ५, ३५ & वृद्धि (=चय) १४५\\

वर्ग २, ९, १६, १८, २१, २७, ९५, & वृद्धि १२९, १३९, १४८, १५२, १५३\\

~~~~१०३, १०६, १२५ १५०, १५३,& वृद्धिधन ५३\\

~~~~१७५ &  व्यत्यास ४५\\

{}& व्ययराशि २५\\

वर्गमूल १७५ & व्यवकलित २, ११, २५\\

वर्गयुति १५२  & व्यवकलितपद ११\\

वर्गसङ्कलित १५०, १५३, & व्यवकलितशेष ११, १२\\

वर्ण ६३, ६४, ६६-६९, ७१, ७२ & व्यवहार २ \\

वर्णक ४३, ४४, ४७, ६३, ६४, ६६, & व्यस्त ७१\\

~~~~~~६७, ७१  & व्यस्त (त्रैराशिक) २\\

वर्णमालिका ६९ & व्यास ४८, ४९\\
\end{tabular}
\end{center}

\newpage

\begin{center}
\begin{tabular}{p{5cm} p{5cm}}
शङ्कु ५ & समलम्ब १७०\\

शत ५ & समलम्बकचतुरश्र १६१\\

शर (=५) ७२ & समहर ७६\\

शलाका ६९ & समायोग २९\\

शून्यतत्व २ & समावर्त ६७\\

शेष १०१, १७५ & समास १६, ३३, ३५, ३६, ५८, १६१\\

शेषपद १६, २५ & समुच्छ्रय ४९\\

शेषांश १०५ & सम्भक्त २६\\

श्रीधराचार्य १ & सरितां पतिः ५\\

श्रेढी २, ११० & सहस्त्र ५, ६\\

श्रेढीक्षेत्र १०७, १०९, ११० & सहित ३३\\

श्रेढीगणित १५३ & सुवर्ण ५, ४०, ४३, ४४, ४७, ६४, ६७,\\

श्रेढीफल १५३ & ~~~~~~६८ \\

सङ्गमकाल ८७ & सूक्ष्मफल १५६\\

संयोग ३४ & सेतिक ४३\\

संयोगवर्ण ६८ & स्थानविभाग १३\\

सङ्कलित २, ६-९, ११, २४, २५, १२९, & स्थूलफल १५६\\

~~~~~~~~१४८, १५२, १५३ & स्वपादरहित ३४\\

सङ्कलितकृति १५१ & स्वपादसहित ३३\\

सङ्कलितपद ११, १२ & स्वयुति ५३\\

सङ्कलितसङ्कलित १५१, १५३ & हत ६९\\

सङ्कलितसमास १५२ & हति २८\\

सङ्कलितैक्य १५१ & हनन ३०\\

सङ्गुणन १४ & हर १४, २६, २९, ३०, ३२, ३४, ९७,\\

सङ्गुणित १०, १६१ & ~~~~१०५\\

सदृशच्छेद २३ & हस्त ६, १६१, १६५, १७०-१७३\\

सदृशद्विराशिघात १६ & हीन ३४\\

सम १३४ & हृत ५३, ५८, ६६, ७१, ७३, ७५, १०४\\

समधन ७६ & हेम ६३, ६७-६९, ७१\\

समचतुरश्र १६१ & {}\\

\end{tabular}
\end{center}


\newpage
\thispagestyle{empty}

{\color{white}a} 

\vspace{4cm}
 \begin{center}{\huge{\textbf{English Translation}}}

\vspace{2cm}{\Large{OF THE}}

\vspace{2cm}{\Huge{\textbf{PATIGANITA}}}\end{center}


\newpage

 \begin{center}\textbf{SRIDHARA'S}
\vspace{2mm}

 \textbf{\large PATIGANITA} \end{center}
\vspace{4mm}

\phantomsection \label{hom}

\englishfont\noindent Homage and introduction:
\vspace{2mm}

1. Having paid obeisance to the Unborn God, the cause
of creation, preservation and destruction of the worlds, I shall
briefly state the (\textit{pāṭī})\textit{gaṇita} for the use of the people.
\vspace{4mm}

\phantomsection \label{con}
\noindent Contents:
\vspace{2mm}

 2-6. (The topics dealt with) here are the twenty-nine
{\textit{parikarmas} (logisties) arranged as follows:}
\vspace{2mm}

\hspace{5mm}  (1) \textit{saṅkalita} (addition),

\hspace{5mm}  (2) \textit{vyavakalita} (subtraction),
    
\hspace{5mm}   (3) \textit{pratyutpanna} (multiplication),

\hspace{5mm}   (4) \textit{bhāgahāra} (division),

\hspace{5mm}   (5) \textit{varga} (square),

\hspace{5mm}   (6) \textit{varga-mūla} (square root),

\hspace{5mm}  (7) \textit{ghana} (cube),

\hspace{5mm}   (8) \textit {ghana-mūla} (cube root),

\hspace{1mm}  (9-16) the same (operations) for fractions,

 (17-22) reduction of fractions (\textit{kalā-savarṇa}) of six varieties, viz.,
\vspace{1mm}

\hspace{10mm}  (i) \textit{bhāga} (fractions connected by $+$ or $-$),

\hspace{10mm}  (ii) \textit{prabhāga} (fractions connected by "of"),

\hspace{10mm}  (iii) \textit{bhāga-bhāga}~(a whole number divided by a fraction),

\hspace{10mm}  (iv) \textit{bhāgānubandha} (a whole number increased by a fraction, or a frac-

\hspace{20mm}  tion increased by a fraction of itself),

\hspace{10mm}  (v) \textit{bhāgāpavāha} (a whole number diminished by a fraction, or a fraction

\hspace{20mm}  diminished by a fraction of itself), and

\hspace{10mm}  (vi) \textit {bhāgamātā} (a blending of two or more fractions of previous forms),

\thispagestyle{empty}

\afterpage{\fancyhead[CO] {\small{DEFINITIONS}}}
\afterpage{\fancyhead[CE] {\small{DEFINITIONS}}}
\afterpage{\fancyhead[LE,RO]{\small{\thepage}}}
\cfoot{}

\newpage
\pagenumbering{arabic}
\setcounter{page}{2}


\hspace{5mm} (23) \textit {trairāśika} (rule of three),

\hspace{5mm}  (24) \textit{vyasta-trairāśika}~(inverse rule of three),

\hspace{5mm}  (25) \textit{pañca-rāśika}~(rule of five),

\hspace{5mm}  (26) \textit{sapta-rāśika}~(rule of seven),

\hspace{5mm}  (27) \textit{nava-rāśika}~(rule of nine),

\hspace{5mm}  (28) \textit{bhāṇḍa-prati-bhāṇḍa}~(barter of commodities),

\hspace{5mm}  (29)~\textit{jīva-vikraya} (sale of living beings);

\vspace{0.3cm}\noindent {and the nine \textit{vyavahāras} (determinations) arranged as follows:}

\vspace{0.2cm} 
\hspace{5mm} (1)~\textit{miśraka} (mixtures),

\hspace{5mm}  (2) \textit {śreḍhī} (series),

\hspace{5mm}  (3) \textit{kṣetra} (plane figures),

\hspace{5mm}  (4) \textit{khāta} (excavations),

\hspace{5mm}  (5) \textit{citi}~(piles of bricks),

\hspace{5mm}  (6) \textit{krākaca} (sawn pieces of timber),

\hspace{5mm}  (7) \textit{rāśi} (heaps or mounds of grain),

\hspace{5mm}  (8) \textit{chāyā} (shadow), and then

\hspace{5mm}  (9) \textit{śūnya-tattva~}(the mathematics of zero).
\vspace{4mm}

\phantomsection \label{def}
\begin{center}
DEFINITIONS
\end{center}
\vspace{0.1cm}

\noindent  Names of notational places:
\vspace{0.2cm}

 7-8. \textit{Eka, \,daśa, \,śata, \,sahasra, \,ayuta, \,lakṣa, \,prayuta, \,koṭi, \,arbuda, \,abja, \,kharva, nikharva, mahā-saroja, śaṅku, saritpati, antya, madhya,} and \textit{parārdha} are each stated to be ten times the preceding by those who have a knowledge of them.\renewcommand{\thefootnote}{1}\footnote{\hspace{-2mm} \en \textit{Cf. GT}, p. 1, vv. 2-3; \textit{L} (ASS), vv. 10-11; \textit{GK, I}, p. 1, vv. 2-3.}
\vspace{0.3cm}

{\small That is to say, the units' place is called \textit{eka}, the tens' place
is called \textit{daśa}, the hundreds' place is called \textit{śata}, and so on.
\vspace{0.3cm}

 Lists of names of notational places given by some writers extend
beyond 18 places. Of these, three are as follows:
\vspace{0.3cm}

 Mahāvīra's (850 A.D.) list: (1) \textit{eka}, (2) \textit{daśa}, (3) \textit{śata},
(4) \textit{sahasra},
(5) \textit{daśa-sahasra}, (6) \textit{lakṣa}, (7) \textit{daśa-lakṣa}, (8) \textit{koṭi}, (9) \textit{daśa-koṭi}, (10) \textit{śata-koṭi}, (11) \textit{arbuda}, (12) \textit{nyarbuda}, (13) \textit{kharva}, (14) \textit{mahā-kharva}, (15) \textit{padma},}
 
\newpage

\begin{sloppypar}
{\small \noindent(16)  \textit{mahā-padma,} (17) \textit{kṣoṇī,} (18)  \textit{mahā-kṣoṇī,} (19)  \textit{śaṅkha,} (20)  \textit{mahā-śaṅkha,} (21)  \textit{kṣiti,} (22)  \textit{mahā-kṣiti}, (23)  \textit{kṣobha,} and (24)  \textit {mahā-kṣobha.}\renewcommand{\thefootnote}{1}\footnote{\hspace{-2mm} \en See  \textit{GSS}, i. 63-68.\\}
\vspace{3mm}

Yallaya's (1480 A.D.) list: (1)  \textit{eka,} (2)  \textit{daśa,} (3)  \textit{śata,} (4)  \textit{sahasra,} (5)  \textit{ayuta,} (6)  \textit{lakṣa}, (7)  \textit{prayuta,} (8)  \textit{koṭi,} (9)  \textit{daśa-koṭi,} (10)  \textit{śata-koṭi,} (11) \textit{arbuda,} (12)  \textit{nyarbuda,} (13)  \textit{kharva,} (14)  \textit{mahā-kharva,} (15)  \textit{padma,} (16)  \textit{mahā-padma,} (17)  \textit{śaṅkha,} (18)  \textit{mahā-śaṅkha,} (19 \textit{kṣoṇi} (20)  \textit{mahā-kṣoṇi,} (21)  \textit{kṣiti,} (22) \textit{mahā-kṣiti,} (23)  \textit{kṣobha,} (24)  \textit{mahā-kṣobha,} (25)  \textit{parārdha,} (26)  \textit{sāgara,} (27)  \textit{ananta,} (28)  \textit{cintya,} and (29)  \textit {bhūri}.~~~~\renewcommand{\thefootnote}{\hspace{-4.5mm} 2}\footnote{\hspace{-2mm} \en This list is given in end of Yallaya's commentary on.  \textit{Ā}. ii.\\}
\vspace{3mm}

 Pāvalūri Mallikārjuna's list: (1)  \textit{eka,} (2)  \textit{daśa,} (3) \textit {śata,} (4)
 \textit{sahasra,}
(5)  \textit{daśa-sahasra,} (6)  \textit{lakṣa,} (7)  \textit{daśa-lakṣa,} (8)  \textit{koṭi,} (9)  \textit {daśa-koṭi,}
(10)  \textit{śata-koṭi,} (11)  \textit{arbuda,} (12)  \textit{nyarbuda}, (13)  \textit{kharva}, (14)  \textit{mahā-kharva,} (15)
 \textit{padma,}
(16)  \textit {mahā-padma,} (17) \textit {śaṅkha,} (18)  \textit {mahā-śaṅkha,} (19)  \textit{kṣoṇi,} (20)
 \textit{mahā-kṣoṇi,}
(21)  \textit{kṣiti,} (22)  \textit{mahā-kṣiti,} (23)  \,\textit{kṣobha,} \,(24)  \,\textit{mahā-kṣobha,} \,(25)  \,\textit{nidhi,} \,(26) \,\textit{mahā-nidhi,} \,(27)  \,\textit{parārdha,} \,(28)  \textit {parata,} (29)  \textit{ananta,} 
(30) \textit{sāgara,} (31)  \textit{avyaya,} (32)  \textit{aprameya,} (33)  \textit {atula,} (34)  \textit{ameya,} (35)  \textit{bhūri,} and (36) \textit{mahā-bhūri.}~~~~\renewcommand{\thefootnote}{\hspace{-4.5mm} 3}\footnote{\hspace{-2mm} \en This list is given in Pāvalūri Mallikārjuna's  \textit{Gaṇita-śāstra}.\\}}
\vspace{3mm}

Table of money-measures:
\vspace{2mm}

 9. A  \textit{purāṇa} is equivalent to sixteen \textit{paṇas}; a \textit {paṇa} is
equivalent to four  \textit{kākīṇīs}; and a  \textit{kākīṇī} is equivalent to
twenty \textit {varāṭakas} (cowries).~~~~\renewcommand{\thefootnote}{\hspace{-4.5mm} 4}\footnote{\hspace{-2mm} \en  \textit{Cf. GT}, p. 2, vs. 4;  \textit{L} ( ASS ), vs. 2.\\}
\vspace{3mm}

{\small Śrīpati's (1039 A.D.) \textit{dramma} is equivalent to Śridhara's \textit{purana,}~~~~\renewcommand{\thefootnote}{\hspace{-4.5mm} 5}\footnote{\hspace{-2mm} \en See  \textit{GT}, p. 2, vs. 4.\\}
\vspace{2mm}

 According to \textit {Nārāyaṇa} (1356 A.D.):~~~~\renewcommand{\thefootnote}{\hspace{-4.5mm} 6}\footnote{\hspace{-2mm} \en See  \textit{GK, I}, p. 1, vs. 4.\\}

\begin{center}
\begin{tabular}{lll}
12  \textit {paṇas} & $=$  & 1  \textit{dramma,}\\
 36  \textit {drammas} & $=$  & 1  \textit {niṣka.}\end{tabular}
\end{center}}
\vspace{1mm}

\noindent Table of weights:
\vspace{2mm}

 10. A  \textit{māṣa} is equal (in weight) to five  \textit{guñjās} (Abrus
seeds); a weight of sixteen  \textit{māṣas} is called a  \textit{karṣa;} a  \textit{karṣa} of
gold is called a  \textit{suvarṇa;} and four  \textit{karṣas} make a  \textit{pala.}~~~~\renewcommand{\thefootnote}{\hspace{-4.5mm} 7}\footnote{\hspace{-2mm} \en  \textit{Cf. GK}, 1, p. 2, vs, 5.\\}
\vspace{3mm}

{\small Nārāyaṇa mentions one more weight, viz.,  \textit{tulā,} which is equal to
100  \textit{palas.}~~~~\renewcommand{\thefootnote}{\hspace{-4.5mm} 8}\footnote{\hspace{-2mm} \en See  \textit{GK, I}, p. 2, vs. 5.}}
\end{sloppypar}
 
 \newpage

\noindent Table of measures of capacity:
\vspace{2mm}

 11. A  \textit{khārī} is equal to sixteen \textit{droṇas;} a  \textit{droṇa} is equal
to four  \textit{āḍhakas}; an  \textit{āḍhaka} is equal to four  \textit{prasthas}; and
a \textit{prastha} is equal to four  \textit{kuḍavas}.\renewcommand{\thefootnote}{1}\footnote{\hspace{-2mm} \en \textit{Cf. GSS}, i. 36-37; \textit{L} (ASS), vv. 7-8.\\}
\vspace{3mm}

{\small The terms  \textit{droṇa},  \textit{āḍhaka} and  \textit{kuḍava} are also found to be
mentioned
in the  \textit{Vedāṅga-Jyautiṣa},~~~~\renewcommand{\thefootnote}{\hspace{-4.5mm} 2}\footnote{\hspace{-2mm} \en See  \textit{Ārca-Jyautiṣa}, vs. 17;  \textit{Yājuṣa-Jyautiṣa,} vs. 24.\\} where an  \textit{āḍhaka} is defined to be a
vessel capable
of holding 50  \textit{palas} of water.
\vspace{3mm}

 According to the commentator of the present work there are 3200 \textit{palas} in a  \textit{khārī}, so that

\begin{center}
\begin{tabular}{rll}
 1  \textit{droṇa} & $=$ &200  \textit{palas},\\
 1  \textit{āḍhaka} & $=$ & 50 \textit{palas},\\
 1  \textit{prastha} & $=$ & 12$\frac{1}{2}$ \textit{palas},\\
 and 1 \textit{kuḍava} & $=$ & 3$\frac{1}{8}$ \textit{palas}.
\end{tabular}
\end{center}

Thus we see that the  \textit{āḍhaka} used in the time and locality of the commentator was the same as that used in the time of the
 \textit{Vedānga-Jyautiṣa}.
\vspace{3mm}

 From the  \textit{Trisātikā}~~~~\renewcommand{\thefootnote}{\hspace{-4.5mm} 3}\footnote{\hspace{-2mm} \en See  \textit{rāśi-vyavahāra.}\\} of  Śrīdhara and the  \textit{Līlāvatī}~~~~\renewcommand{\thefootnote}{\hspace{-4.5mm} 4}\footnote{\hspace{-2mm} \en See \textit{L} (ASS), p. 10, vs. 7.\\} of Bhāskara II
(1150 A.D.) it appears that the  \textit{khārī} used in  Magadha was
equivalent to
a cubic cubit. Nārāyaṇa's \textit{khārī} is equivalent to 5 cubic cubits.~~~~\renewcommand{\thefootnote}{\hspace{-4.5mm} 5}\footnote{\hspace{-2mm} \en See \textit{GK, I}, p. 3, vv. 10-13.\\}}
\vspace{4mm}

\noindent Table of linear measures:
\vspace{2mm}

 12. Twenty-four  \textit{angulas} (finger-breadths) make a \textit {hasta}
(cubit); four  \textit{hastas} make a  \textit{daṇḍa} (staff); two thousand of
them make a  \textit{krośa}; and four  \textit{krośas} make a  \textit{yojana}.~~~~\renewcommand{\thefootnote}{\hspace{-4.5mm} 6}\footnote{\hspace{-2mm} \en \textit{Cf. GSS}, i. 29-31(i); \textit{GK, I}, pp. 2-3.\\}
\vspace{-1mm}

{\small  Nārāyaṇa's \textit{daṇḍa} is bigger than Śrīdhara's, being equivalent to 10
cubits; but his other measures are the same as those given here.~~~~\renewcommand{\thefootnote}{\hspace{-4.5mm} 7}\footnote{\hspace{-2mm} \en See \textit{GK, I}, p. 2, vv. 6(ii)-7(i).\\}}
\vspace{4mm}

\noindent Table of time-measures:
\vspace{2mm}

 13. Sixty  \textit{ghaṭīs} make a  nychthemeron (a day-and-night);~~~~\renewcommand{\thefootnote}{\hspace{-4.5mm} 8}\footnote{\hspace{-2mm} \en Thus 1 \textit{ghaṭī} $=$ 24 minutes.}
thirty of them make a month; twelve of them make a year.
\vspace{3mm}

 These are the definitions (used) in this (\textit{pāṭi})\textit{ganita}.

\afterpage{\fancyhead[CO] {\small{LOGISTICS}}}
\afterpage{\fancyhead[CE] {\small{LOGISTICS}}}

\newpage

\phantomsection \label{log}
\begin{sloppypar}

  \begin{center} LOGISTICS (\textit{Parikarma})
\vspace{3mm}

 \textbf{(1) Addition (\textit{saṅkalita})}
\end{center}

\noindent Rule for finding the sum of a series of natural numbers:
\vspace{3mm}

 14(i). When the first term (\textit{ādi}) and the common difference (\textit{caya}) of a series (in arithmetic progression) are (each)
unity, the sum (\textit{saṅkalita}) is equal to half the number of terms
(\textit{pada}) multiplied by the number of terms plus one.\renewcommand{\thefootnote}{1}\footnote{\hspace{-2mm} \en \textit{Cf. GK}, 1, p. 114, lines 15-16.}
\vspace{3mm}

{\small That is,

\hspace{20mm} ${S_n} = 1 + 2 + 3 + ... + n = \dfrac{{n}\,(n + 1)}{2}$}
\vspace{3mm}

\noindent Rule for finding the number of terms of a series of natural 
numbers when the sum is given:
\vspace{3mm}

 14(ii). The number of terms (\textit{gaccha}) is equal to the
(integral) square root of twice the sum of the series, which
must be the same as the residue left (after the extraction
of the square root).
\vspace{3mm}

{\small That is,
\vspace{1mm}

\hspace{20mm} $n =$ integral part of $\sqrt{2S_n}$.}
\vspace{3mm}

Ex. 1. What are the sums of 1 to 10, each multiplied
by 10, terms of the series whose first term and common
difference are each unity\,? Also, from those sums quickly say
the number of terms of the (various) series.
\vspace{3mm}

{\small '1 to 10, each multiplied by 10' means 10, 20, 30, 40, 50, 60, 70, 80, 90 and 100.}
\vspace{3mm}

\noindent Alternative rules:
\vspace{3mm}

 15. Or, the sum of a series (of natural numbers) is 
equal to one-half of\textendash \,the square of the numbers of terms plus
the number of terms.
\vspace{3mm}

 And that (sum) multiplied by 8, (then) increased by 1, (then) reduced to its square root, (then) diminished by 1, and (then) halved, is the number of terms (in the series).

\end{sloppypar}


\newpage

\begin{sloppypar}

{\small  That is,
\vspace{1mm}

\hspace{20mm} ${S_n} =  1 + 2 + 3 +....+ n = \dfrac{{n^2} +\,n} {2}$,
\vspace{2mm}

\hspace{17mm} and $n = \dfrac{\sqrt{8{S_n} + 1} -1} {2}$}
\vspace{4mm}

  \begin{center}
  \textbf{(2) Subtraction ({\emph{vyavakalita}})} \end{center} 

\noindent Rule \,for \,finding \,the \,remainder \,accruing \,on \,subtracting \,one \,series \,of \,natural numbers ('subtrahend series') from another
series of natural numbers ('minuend series'):
\vspace{3mm}

		Having added (\textit{nidhāya}) the number  of terms of the subtrahend series (\textit{vyavakalita-pada}) plus one to the number of terms of the minuend series (\textit{saṅkalita-pada}), multiply that (sum) by the difference of the number of terms (of the two series): that (product), when halved, becomes the residue of subtraction (of the given series).
\vspace{3mm}

{\small That is,
\vspace{1mm}

\hspace{20mm}  {${S_n}-{S_m} = \dfrac{(n - m) [n + (m + 1)]}{2}$} 
\vspace{3mm}

\noindent {where $n\,\textgreater \,m$, and ${S_k}$ stands for}
\vspace{2mm}

\hspace{20mm} {$1 + 2 + 3 + ... + k$.}}
\vspace{3mm}

 Ex. 2. What are the respective remainders obtained
when the sums of 1 to 10, each multiplied by 10, terms of the
series whose first term and common difference are (each)
unity are severally subtracted from the sum of 100 terms (of
the same series)?
\vspace{4mm}

\noindent Rule for finding the number of terms of the subtrahend series,
when the difference of the minuend and subtrahend series as
well as the number of terms of the minuend series is given:
\vspace{3mm}

 17.  Having subtracted the residue of subtraction (i.e.,
the difference of the minuend and subtrahend series) from
the sum of the minuend series, and multiplied the remainder
(obtained) by 2, the square root thereof, which must be equal to the residue left (after the extraction of the square root),
should be declared as the number of terms (of the subtrahend
series).

\end{sloppypar}

\afterpage{\fancyhead[CO] {\small{MULTIPLICATION}}}

\newpage

\begin{sloppypar}

{\small  That is, \hspace{5mm} $m =$ integral part of $\sqrt{2[S_n-D]}$\,,
\vspace{2mm}

 \hspace{7mm} where D $= {S_n}-{S_m}$.}
\vspace{3mm}

  \begin{center}
  \textbf{(3) Multiplication (\textit{pratyutpanna})} \end{center}

\noindent Four methods of multiplication:
\vspace{3mm}

 18-19.  Having placed the multiplicand (\textit{guṇya}) below
the multiplier (\textit{guṇa-rāśi}) as in the junction of two doors,
multiply successively in the inverse or direct order, moving
(the multiplier) each time. This process is known as \textit{kavāṭa-sandhi} ("the door-junction method").
\vspace{3mm}

 When the multiplication (\textit{pratyutpanna}) is performed by
keeping that i.e., the multiplier) stationary, the process is
called \textit{tatstha}. (i.e., "multiplication at the same place") on
that account.\renewcommand{\thefootnote}{1}\footnote{\hspace{-2mm} \en See  Siṃhatilaka Sūri's comm. on  Śripati's \textit{Gaṇita tilaka}, Rule 15-16(i). The \textit{tatstha} method of multiplication has been called  \textit{tatsthāna-guṇana} by Gaṇeśa. See \textit{L} (ASS), p. 17.\\}
\vspace{3mm}

 20. The process of multiplication called  \textit{khaṇḍa} (or \textit{khaṇḍa-guṇana},  "multiplication \,by \,parts") \,is \,of \,two \,varieties \,(called \,\textit{rūpa-vibhāga} \,and  \,\textit{sthāna-vibhāga}), depending on whether
the multiplicand or multiplier is broken up into two or more
parts whose sum or product is equal to it, or the digits standing in the different notational places (\textit{sthāna}) of the multiplicand or multiplier are taken separately.
\vspace{3mm}

 These are the four methods of multiplication.~~~~\renewcommand{\thefootnote}{\hspace{-4.5mm} 2}\footnote{\hspace{-2mm} \en \textit{Cf. GSS}, ii. 1; \textit{GT}, pp. 4-5, vv. 15-16(i). Also see \textit{MSi}, xv. 3; \textit{SiŚe}, xiii. 2; \textit{GK, I} p. 4, vv. 13-14.}
\vspace{3mm}

{\small For details of these methods, see B.\,Datta and A.\,N.\,Singh, \textit{History of the Hindu Mathematics}, Part I, pp. 136-143, and 146-149.}
\vspace{3mm}

 Ex. 3. Multiply 1296 by 21, 896 by 37, and 8065 by 60.
\vspace{4mm}

\noindent Operations with cipher:
\vspace{3mm}

 21. When something is added to cipher, the sum is equal to the additive (\textit{kṣepa});

\end{sloppypar}


\newpage

\begin{sloppypar}

\noindent when cipher is added to or
subtracted from a number, the number remains unchanged.
In multiplication and other operation on cipher, the result is
cipher. Multiplication (of a number) by cipher also yields 
cipher.\renewcommand{\thefootnote}{1}\footnote{\hspace{-2mm} \en \textit{Cf. GSS}, i. 49; \textit{MSi}, xv. 10(ii)-11(i); \textit{GK, I}, p. 13, vs. 30.\\}
\vspace{3mm}

{\small  That is, \hspace{5mm} $0 + a = a,$
\vspace{1mm}

\hspace{15mm} $a \pm 0 = a,$
\vspace{1mm}

\hspace{15mm} $0 \times a$ ~or~ $0 \div a = 0,$
\vspace{1mm}

\hspace{15mm} $a \times 0 = 0$.}
\vspace{3mm}

 \begin{center} {\textbf{(4) Division (\textit{bhāgahāra})}}
 \end{center}

\noindent Rule for division: 
\vspace{3mm}

 22. Having removed the common factor, if any, from
the divisor and the divi-dend, divide (by the divisor the digits 
of the dividend) one after another in the inverse order: this 
is the method of division.~~~~\renewcommand{\thefootnote}{\hspace{-4.5mm} 2}\footnote{\hspace{-2mm} \en \textit{Cf. GSS}, ii. 19; \textit{GT}, p. 6, lines 21-24; \textit{L} (ASS), I, vs. 18(ii); \textit{GK, I}, p. 5, vs. 16(ii).\\}
\vspace{3mm}

{\small Āryabhaṭa II (c. 950 A.D.) gives more details of the process:
\vspace{3mm}

 "Having placed the divisor underneath the (last digits) of the 
dividend, subtract the proper multiple of the divisor from the (last
digits
of the) dividend; the multiple (thus obtained) is the (partial)
quotient. 
Next having moved the divisor (one place to the right) divide what
remains. (Continue this process until all the digits of the dividend
have 
been taken)."~~~~~\renewcommand{\thefootnote}{\hspace{-4.5mm} 3}\footnote{\hspace{-2mm} \en \textit{MSi}, xv. 4. Also see \textit{ŚiSe}, xiii. 3; \textit{L} (ASS), \textit{I}, vs. 18(i); \textit{GK, I}, p. 5, vs. 16(i).\\}}
\vspace{3mm}

 \begin{center} \textbf{(5) Squaring (\textit{varga})} \end{center}

\noindent Rule for squaring:
\vspace{3mm}

 23. To obtain the square of a number (proceed successively as follows): Having squared the last digit (\textit{antya-pada}) (i.e, having written the square of the last digit over the 
last digit), multiply the remaining digits by twice the last,
moving it from place to place (towards right, and set down 
the resulting products over the respective digits); then (rub
out the last digit and) move the remaining digits (one place
to the right).~~~~\renewcommand{\thefootnote}{\hspace{-4.5mm} 4}\footnote{\hspace{-2mm} \en \textit{Cf. GSS}, ii. 31; \textit{GT}, p. 7, lines 24-25; \textit{L} (ASS) \textit{I}, 19; \textit{GK}, p. 6, vs. 17.}

\end{sloppypar}

\afterpage{\fancyhead[CO] {\small{SQUARE-ROOT}}}

\newpage

\begin{sloppypar}

{\small This rule is based on the formula
\vspace{2mm}

\hspace{20mm} ${(a + b)^2} = {a^2} + 2ab + {b^2}$
\vspace{3mm}

 For details of the process, see B. Datta and A. N. Singh, \textit{History
of
Hindu Mathematics}, Part I, pp. 157-160.}
\vspace{4mm}

\noindent Other rules for squaring:
\vspace{3mm}

 24. The square (of a given number) is also equal to
the product of two equal numbers (each equal to the given
number), or the sum of as many terms of the series whose first
term is 1 and common difference 2, or the product of the
difference and the sum of the given number and an assumed
number plus the square of the assumed number.\renewcommand{\thefootnote}{1}\footnote{\hspace{-2mm} \en \textit{Cf. GSS}, ii. 29; \textit{GK, I}, p. 6, vs. 18. Also see  \textit{Ā}. ii. 3(i); \textit{BrSpSi}, xii 63(ii); \textit{MSi}, xv. 6(i); \textit{GT}, p. 8, vs. 21(ii); \textit{SiŚe}, xiii 4(i); \textit{L} (ASS), 20(ii).\\}
\vspace{3mm}

{\small That is, \hspace{5mm} (i) ${n^2} = n \times n$;
\vspace{1mm}

\hspace{15mm} (ii) ${n^2} = 1 + 3 + 5 + ......$ to $n$ terms;
\vspace{1mm}

\hspace{14mm} (iii) ${n^2} = (n - a) (n + a) + {a^2}$.}
\vspace{3mm}

 Ex. 4. Tell (me) the squares of 1 to 9, 25, 36, 63, 432,
and 7802.
\vspace{3mm}

\begin{center} \textbf{(6) Square root (\textit{varga-mūla})} \end{center}

\noindent Rule for finding the square root:
\vspace{3mm}

 25-26. Having subtracted the (greatest possible) square
from the (last) odd place (set down double the square root
underneath the next place). By that double square root, that
has left its place (i.e., which has been set down underneath the
next place), divide the remainder; set down the quotient in
the line (of double the square root), and, having subtracted
the square of that (from the number above), make double of
that (quotient). Then having moved the resulting quantity
(in the line of double the square root) one place forward,
divide by it as before. (Continue this process till all places
are exhausted, and then) halve the doubled quantity (to get
the square root).~~~~\renewcommand{\thefootnote}{\hspace{-4.5mm} 2}\footnote{\hspace{-2mm} \en \textit{Cf. GSS}, ii. 36; \textit{GT}, p. 9, vs. 23; \textit{MSi}, xv. 6(ii)-7; \textit{SiŚe}, xiii. 5; \textit{L} (ASS), p. 21, vs. 22; \textit{GK, I}, p. 7, lines 2-9. Also see \textit{Ā}, ii. 4.}
\vspace{-1mm}

{\small This rule will be clear by the following example:}

\end{sloppypar}

\newpage

\begin{sloppypar}

{\small Example. Find the square root of 186624.
\vspace{3mm}

 Indicating the odd and even places by writing the tachygraphic
abbreviations o and e over the digits, we have
\vspace{3mm}

\hspace{1cm} \begin{tabular}{c} e \\ 1 \end{tabular}
\begin{tabular}{c} o \\ 8  \end{tabular}
\begin{tabular}{c}  e \\ 6\end{tabular}
\begin{tabular}{c} o \\ 6 \end{tabular}
\begin{tabular}{c}e \\2  \end{tabular}
\begin{tabular}{c} o \\ 4 \end{tabular}
\vspace{3mm}
 
 Subtracting the greatest square number (i.e., 16) from the last
odd place (i.e. from 18), and then setting down double the square root
of 16 underneath the next place, we have
\vspace{3mm}

\hspace{1cm} \begin{tabular}{c} o\\ 2\\ \\ \end{tabular}
\begin{tabular}{c} e\\6\\8 \end{tabular}
\begin{tabular}{c} o\\6 \\ \\  \end{tabular}
\begin{tabular}{c} e\\2\\ \\  \end{tabular}
\begin{tabular}{c} o\\4\\ \\ \end{tabular} \hspace{10mm} \begin{tabular}{r} \\ (remainder)\\ (line of double the square root)\\  \end{tabular}
\vspace{3mm}

Dividing out 26 by 8 and setting the quotient 3 in the line of double
the square root, we get
\vspace{3mm}

\hspace{1cm} \begin{tabular}{c} e\\ 2\\ 8  \end{tabular}
\begin{tabular}{c} o\\6\\3  \end{tabular}
\begin{tabular}{c} e\\2\\ \\  \end{tabular}
\begin{tabular}{c} o\\4\\ \\ \end{tabular} \begin{tabular}{c} \\\\ \\ \end{tabular} \hspace{10mm} \begin{tabular}{r} \\ (remainder)\\ (line of double the square root)\\  \end{tabular}
\vspace{3mm}

Subtracting the square of the quotient (i.e., 9) from above, and
doubling the quotient (3), we get
\vspace{3mm}

\hspace{1cm} \begin{tabular}{c} e\\ 1\\ 8  \end{tabular}
\begin{tabular}{c} o\\7\\6  \end{tabular}
\begin{tabular}{c} e\\2\\ \\ \end{tabular}
\begin{tabular}{c} o\\4\\ \\ \end{tabular} \begin{tabular}{c} \\\\ \\ \end{tabular} \hspace{10mm} \begin{tabular}{r} \\ (remainder)\\ (line of double the square root)\\  \end{tabular}
\vspace{3mm}
 
One round of the operation is now over. Now we move 86 one place
forward. Then dividing out 172 by 86 and writing the quotient in the
line of double the square root, we get
\vspace{3mm}

\hspace{1cm} \begin{tabular}{c} o\\ \\ 8  \end{tabular}
\begin{tabular}{c} e\\ \\6  \end{tabular}
\begin{tabular}{c} o\\4\\2  \end{tabular} \begin{tabular}{c} \\ \\ \\ \end{tabular} \begin{tabular}{c} \\ \\ \\ \end{tabular} \hspace{10mm} \begin{tabular}{r} \\ (remainder)\\ (line of double the square root)\\  \end{tabular}
\vspace{3mm}

Finally subtracting the square of the quotient (i.e., 4) from above,
and doubling the quotient 2, we get
\vspace{3mm}

\hspace{1cm} \begin{tabular}{c} o\\ \\ 8  \end{tabular}
\begin{tabular}{c} e\\ \\6  \end{tabular}
\begin{tabular}{c} o\\0\\4  \end{tabular} \begin{tabular}{c} \\ \\ \\ \end{tabular} \begin{tabular}{c} \\ \\ \\ \end{tabular} \hspace{10mm} \begin{tabular}{r} \\ (remainder)\\ (line of double the square root)\\  \end{tabular}
\vspace{3mm}

The process now ends. So we divide 864 by 2, getting 432 as the required square root.
\vspace{3mm}

 As no remainder is left, the square root is exact.}

\end{sloppypar}

\afterpage{\fancyhead[CO] {\small{CUBE}}}

\newpage

\begin{sloppypar}

\begin{center} \textbf{(7) Cube (\textit{ghana})} \end{center}

\noindent Rules for cubing:
\vspace{3mm}

 27-28. [Let the last digit of the given number be called the 'last' (\textit{antya}) and the last-but-one digit the 'first' (\textit{ādi})
or the 'preceding' (\textit{pūrva}).
\vspace{3mm}

 Set down the cube of the 'last'; then set down,
(successively) one place forward (\textit{sthānādhikyam}), (i) the square
of the 'last' as multiplied by thrice the 'preceding', (ii) the
square of the 'first' as multiplied by the 'last' as well as by 3,
and (iii) the cube of the 'first'. This gives the cube of the
combined number (formed by the 'last' and the 'first') (\textit{niryukta-rāśi}), which should now be treated as the 'last' (provided there
be more than two digits in the given number).\renewcommand{\thefootnote}{1}\footnote{\hspace{-2mm} \en \textit{Cf. BrSpSi}, xii. 6; \textit{GSS}, ii. 47; \textit{GT}, p. 11, vs. 25; \textit{L}
(ASS), p. 23,
vs. 24-25(i); \textit{GK, I}, p. 7, lines 17-19, p. 8, lines 1-2.\\}
\vspace{3mm}

 The (continued) product of three equal quantities;~~\,~~\renewcommand{\thefootnote}{\hspace{-4.5mm} 2}\footnote{\hspace{-2mm} \en \textit{Cf. Ā}, ii. 3(ii); \textit{BrSpSi}, xii. 62(ii); \textit{GSS}, ii. 43(i);
\textit{MSi}, xv. 6(i);
\textit{GT}, p. 11, line 10; \textit{SiŚe}, xiii. 4(i); \textit{L} (ASS), p. 23,
vs. 24(i); \textit{GK, I}, p. 7,
line 16.\\} or,
in the series having 1 for the first term and common difference
(considering the last two terms, designating them as the 'first'
or 'preceding' and the 'last' respectively), the last multiplied
by thrice the 'first', and increased by 1, and that added to the
cube of the 'preceding', is also the cube.~~~~\renewcommand{\thefootnote}{\hspace{-4.5mm} 3}\footnote{\hspace{-2mm} \en \textit{Cf. GT}, p. 11, lines 7-9; \textit{GK, I}, p. 8, lines 3-4.}
\vspace{3mm}

{\small Of the three rules stated here, the first one is the main method of
cubing a number. To illustrate it by an example, let us find the
cube of 256.
\vspace{3mm}

 To begin with we treat 2 as the 'last' and 5 as the 'first'. Then
writing the cube of the 'last' (i.e., ${2^3}$, or 8), then in the next place
the
square of the 'last' as multiplied by thrice the 'first' (i.e.,
$3 \times 5 \times {2^2}$, or 60),
then in the next place the square of the 'first' as multiplied by thrice
the
'last' (i.e., $3 \times 2 \times {5^2}$, or 150), and then in the next place the cube
of the
'first' (i.e., ${5^3}$, or 125), we have

\begin{center} \begin{tabular}{c} o\\8\\6\\1\\ \\ \end{tabular}
\begin{tabular}{c} o\\ \\0\\5\\1  \end{tabular}
\begin{tabular}{c} o\\ \\ \\0 \\2  \end{tabular}
\begin{tabular}{c} o\\ \\ \\ \\5  \end{tabular}
  \end{center}}


\end{sloppypar}


\newpage

\begin{sloppypar}

{\small Adding up these numbers, we get 15625. This is the cube of 25.
\vspace{3mm}

 Now we treat 25 as the last and 6 as the first, and proceeding as
above we get 16777216 as the cube of 256.
\vspace{3mm}

 The second and third rules are:
\vspace{2mm}

\hspace{20mm} {(i) ${n^3} = n \times n \times n$

\hspace{19mm} (ii) ${n^3} = (n-1)^3~+ 3n\,(n-1) + 1.$}
\vspace{3mm}

Datta and Singh, assuming

\begin{quote}
{\s खैकादिचये वान्त्ये त्र्यादिहते पूर्वधनयुतिः सैके~।}
\end{quote}

\noindent to be the reading of the second half of verse 28, have given the
following
translation for it:\renewcommand{\thefootnote}{1}\footnote{\hspace{-2mm} \en B. Datta and A. N. Singh, \textit{History of Hindu Mathematics}, Part
I, page 168.\\}
\vspace{3mm}

 "The cube (of a given number) is equal to the series whose terms
are formed by applying the rule, 'the last term multiplied by thrice
the
preceding term plus one,' to the terms of the series whose first term
is zero,
the common difference is one and the last term is the given number."
\vspace{3mm}

 This gives
\vspace{2mm}

\hspace{20mm} ${n^3} = \Sigma\,[3r (1-1) +1]$ 
\vspace{3mm}

\noindent in place of formula (ii) above.~~~~\renewcommand{\thefootnote}{\hspace{-4.5mm} 2}\footnote{\hspace{-2mm} \en This formula is also given in \textit{GSS}, ii. 45.}}
\vspace{3mm}

 Ex. 5. Quickly say what are the cubes of 1 to 9, 15,
256, and 203.
\vspace{3mm}

 \begin{center} \textbf{(8) Cube root (\textit{ghana-mūla})} \end{center}

\noindent Rule for finding the cube root:
\vspace{3mm}

 29-31. (Divide the digits beginning with the units' place
into periods of) one 'cube' place (\textit{ghana-pada}) and two 'non-cube' places (\textit{aghana-pada}). Then subtracting the (greatest possible) cube from the (last) 'cube' place and placing the (cube)
root underneath the third place (to the right of the last 'cube'
place), divide out the remainder up to one place less (than that
occupied by the cube root) by thrice the square of the cube
root, which is not destroyed. Setting down the quotient
(obtained from division) in the line (of the cube root), [and
designating the quotient as the 'first' (\textit{ādima}) and the cube
root as the 'last' (\textit{antya})], subtract the square of that
quotient,
as multiplied by thrice the 'last' (\textit{antya}), from one place less
than that occupied \,by \,the \,quotient \,(\textit{uparima-rāśi}) \,as \,before, \,and \,the \,cube \,of \,the \,'first' (\textit{ādima}) from its own place.

\afterpage{\fancyhead[CO] {\small{CUBE ROOT}}}

\newpage

 \noindent (The number
 now standing in the line of cube root is the cube root of the
given number up to its last-but-one cube place from the left).
 Again apply the rule, '(placing the cube root) under the third
place' etc. (provided there be more than two 'cube' places in
the given number; and continue the process till all cube
 places are exhausted). This will give the (cube) root (of the
given number).\renewcommand{\thefootnote}{1}\footnote{\hspace{-2mm} \en \textit{Cf. Ā}, ii. 5; \textit{BrSpSi}, xii. 7; \textit{GSS}, ii. 53, 54; \textit{MSi},
xv. 9-10(i);
\textit{GT}, p. 13, lines 18-25; \textit{SiŚe}, xiii. 6-7; \textit{L} (ASS), vv. 28-29;
\textit{GK, I}, pp. 8-9,
vv. 24-25.}
\vspace{3mm}

 To illustrate this method, we find out the cube root of
\vspace{2mm}

\hspace{3cm}  277167808. 
\vspace{2mm}

Indicating the 'cube' and 'non-cube' places by writing the tachygraphic abbreviations c and n over the digits, we have
\vspace{3mm}

\hspace{1cm} \begin{tabular}{c} n\\2  \end{tabular}
\begin{tabular}{c} n\\7  \end{tabular}
\begin{tabular}{c} c\\7  \end{tabular}
\begin{tabular}{c} n\\1  \end{tabular}
\begin{tabular}{c} n\\6  \end{tabular}
\begin{tabular}{c} c\\7  \end{tabular}
\begin{tabular}{c} n\\8  \end{tabular}
\begin{tabular}{c} n\\0  \end{tabular}
\begin{tabular}{c} c\\8  \end{tabular}
\vspace{3mm}

Subtracting the greatest possible cube (i.e., ${6^3}$ or 216) from the
last
' cube 'place (i e., from 277) and placing the cube root (i.e., 6)
underneath
 the third place to the right of the last 'cube' place, we have
\vspace{3mm}

\hspace{1cm} \begin{tabular}{c} n\\ \\ \\ \end{tabular}
\begin{tabular}{c} n\\6\\ \\ \end{tabular}
\begin{tabular}{c} c\\1\\ \\ \end{tabular}
\begin{tabular}{c} n\\1\\ \\ \end{tabular}
\begin{tabular}{c} n\\6\\6  \end{tabular}
\begin{tabular}{c} c\\7\\ \\ \end{tabular}
\begin{tabular}{c} n\\8\\ \\ \end{tabular}
\begin{tabular}{c} n\\0\\ \\ \end{tabular}
\begin{tabular}{c} c\\8\\ \\ \end{tabular} \hspace{10mm} \begin{tabular}{r} \\ (remainder)\\ (line of cube root)\\  \end{tabular}
\vspace{3mm}

Dividing out by thrice the square of the cube root (i.e., by $3 \times {6^2}$ or
108)
the remainder up to one place less than that occupied by the cube
root
 (i.e., 611), and setting down the quotient in the line of the cube
root (to
the right of the cube root), we have
\vspace{3mm}

\hspace{1cm} {\begin{tabular}{c} n\\ \\ \\  \end{tabular}
\begin{tabular}{c} n\\ \\ \\ \end{tabular}
\begin{tabular}{c} c\\7\\ \\ \end{tabular}
\begin{tabular}{c} n\\1\\ \\ \end{tabular}
\begin{tabular}{c} n\\6\\6  \end{tabular}
\begin{tabular}{c} c\\7\\5  \end{tabular}
\begin{tabular}{c} n\\8\\ \\ \end{tabular}
\begin{tabular}{c} n\\0\\ \\ \end{tabular}
\begin{tabular}{c} c\\8\\ \\ \end{tabular}} \hspace{10mm} \begin{tabular}{r} \\ (remainder)\\ (line of cube root)\\  \end{tabular}
\vspace{3mm}

 Let now the quotient 5 be called the 'first' and the cube root 6 the
'last'. Then subtracting the square of the 'first' as multiplied by
thrice
the 'last' (i.e., $3 \times 6 \times {5^2}$, or 450) from one place less than that
occupied by
 the quotient (i.e., from 716), we get
\vspace{3mm}

\hspace{1cm} {\begin{tabular}{c} n\\ \\ \\  \end{tabular}
\begin{tabular}{c} n\\ \\ \\ \end{tabular}
\begin{tabular}{c} c\\2\\ \\ \end{tabular}
\begin{tabular}{c} n\\6\\ \\ \end{tabular}
\begin{tabular}{c} n\\6\\6  \end{tabular}
\begin{tabular}{c} c\\7\\5  \end{tabular}
\begin{tabular}{c} n\\8\\ \\ \end{tabular}
\begin{tabular}{c} n\\0\\ \\ \end{tabular}
\begin{tabular}{c} c\\8\\ \\ \end{tabular}} \hspace{10mm} \begin{tabular}{r} \\ (remainder)\\ (line of cube root)\\  \end{tabular}

\end{sloppypar}

\newpage

\begin{sloppypar}

And subtracting the cube of the 'first' (i.e., ${5^3}$ or 125) from
its own place
(i.e., from 2667), we get
\vspace{3mm}

\hspace{1cm}\begin{tabular}{c} n\\ \\ \\ \end{tabular}
\begin{tabular}{c} n\\ \\ \\ \end{tabular}
\begin{tabular}{c} c\\2\\ \\ \end{tabular}
\begin{tabular}{c} n\\5\\ \\ \end{tabular}
\begin{tabular}{c} n\\4\\6  \end{tabular}
\begin{tabular}{c} c\\2\\5  \end{tabular}
\begin{tabular}{c} n\\8\\ \\ \end{tabular}
\begin{tabular}{c} n\\0\\ \\ \end{tabular}
\begin{tabular}{c} c\\8\\ \\ \end{tabular} \hspace{10mm} \begin{tabular}{r} \\ (remainder)\\ (line of cube root)\\  \end{tabular}
\vspace{3mm}

One round of the operation is now over; and the number 65 standing in the line of the cube root is the cube root of the given number
(277167808) upto its last-but-one 'cube' place from the left (i.e., of
277167).
\vspace{3mm}

 As there is one more 'cube' place on the right, the process is
repeated.
Thus, placing the cube-root (i.e., 65) under the third place beginning
with
the last-but-one 'cube' place, we have
\vspace{3mm}

\hspace{1cm}\begin{tabular}{c} n\\ \\ \\ \end{tabular}
\begin{tabular}{c} n\\ \\ \\ \end{tabular}
\begin{tabular}{c} c\\2\\ \\ \end{tabular}
\begin{tabular}{c} n\\5\\ \\ \end{tabular}
\begin{tabular}{c} n\\4\\ \\ \end{tabular}
\begin{tabular}{c} c\\2\\ \\ \end{tabular}
\begin{tabular}{c} n\\8\\6  \end{tabular}
\begin{tabular}{c} n\\0\\5 \end{tabular}
\begin{tabular}{c} c\\8\\ \\ \end{tabular} \hspace{10mm} \begin{tabular}{r} \\ (remainder)\\ (line of cube root)\\  \end{tabular}
\vspace{3mm}

Dividing out 25428 by $3 \times {65^2}$ ($=$ 12675) as before, and placing
the
quotient in the line of the cube root, we have
\vspace{3mm}

\hspace{1cm}\begin{tabular}{c} n\\ \\ \\ \end{tabular}
\begin{tabular}{c} n\\ \\ \\ \end{tabular}
\begin{tabular}{c} c\\ \\ \\ \end{tabular}
\begin{tabular}{c} n\\ \\ \\ \end{tabular}
\begin{tabular}{c} n\\ \\ \\ \end{tabular}
\begin{tabular}{c} c\\7 \\ \\ \end{tabular}
\begin{tabular}{c} n\\ 8\\6  \end{tabular}
\begin{tabular}{c} n\\ 0\\5  \end{tabular}
\begin{tabular}{c} c\\8\\2  \end{tabular} \hspace{10mm} \begin{tabular}{r} \\ (remainder)\\ (line of cube root)\\  \end{tabular}
\vspace{3mm}

Subtracting $3 \times 65 \times {2^2}$ ($=$ 780) from 780 we get
\vspace{3mm}

\hspace{1cm}\begin{tabular}{c} n\\ \\ \\ \end{tabular}
\begin{tabular}{c} n\\ \\ \\ \end{tabular}
\begin{tabular}{c} c\\ \\ \\ \end{tabular}
\begin{tabular}{c} n\\ \\ \\ \end{tabular}
\begin{tabular}{c} n\\ \\ \\ \end{tabular}
\begin{tabular}{c} c\\ \\ \\ \end{tabular}
\begin{tabular}{c} n\\ \\6  \end{tabular}
\begin{tabular}{c} n\\ \\5  \end{tabular}
\begin{tabular}{c} c\\8\\2  \end{tabular} \hspace{10mm} \begin{tabular}{r} \\ (remainder)\\ (line of cube root)\\  \end{tabular}
\vspace{3mm}

Finally subtracting ${2^3}$ from 8, we get
\vspace{3mm}

\hspace{1cm}\begin{tabular}{c} n\\ \\ \\  \end{tabular}
\begin{tabular}{c} n\\ \\ \\ \end{tabular}
\begin{tabular}{c} c\\ \\ \\ \end{tabular}
\begin{tabular}{c} n\\ \\ \\ \end{tabular}
\begin{tabular}{c} n\\ \\ \\ \end{tabular}
\begin{tabular}{c} c\\ \\ \\ \end{tabular}
\begin{tabular}{c} n\\ \\6  \end{tabular}
\begin{tabular}{c} n\\ \\5  \end{tabular}
\begin{tabular}{c} c\\0\\2  \end{tabular} \hspace{10mm} \begin{tabular}{r} \\ (remainder)\\ (line of cube root)\\  \end{tabular}
\vspace{3mm}

The second round of the operation is now over. There being no
more 'cube' places on the right, the process ends. The quantity in the
line
of the cube root, viz., 652, is the cube root of the given number. The
remainder being zero, the cube root is exact.

\afterpage{\fancyhead[CO] {\small{OPERATIONS FOR FRACTIONS}}}

\newpage

\begin{center}  \textbf{(9-16) Operations for fractions} \end{center}

\noindent Rule for the addition of fractions:
\vspace{3mm}

 32(i). Reduce the fractions to a common denominator
and then add the numerators. The denominator of a whole
number is unity.\renewcommand{\thefootnote}{1}\footnote{\hspace{-2mm} \en \textit{Cf. MSi}, xv. 14(i); \textit{GT}, p. 15, lines 20-21;
\textit{SiŚe}, xiii. 8; \textit{L}
(ASS), p. 35, vs. 37; \textit{GK, I}, p. 11, lines 6-7. Also see \textit{BrSpSi},
xii. 2.\\}
\vspace{3mm}

Ex. 6. Say the sum of $\frac{1}{2}, \frac{1}{3}, \frac{1}{6}$ and $\frac{1}{12}$, and  of  2 plus $\frac{1}{2}$.
3 minus $\frac{1}{4}$, and 6.
\vspace{4mm}

 Ex. 7. Friend, if you know the method of calculation
(\textit{gaṇita-vidhi}), quickly say the sum of $1 \frac{1}{2}$ terms, $\frac{1}{2}$ term, and
of
$\frac{1}{3}$ term of the series whose first term (\textit{ādi}) and common
difference (\textit{caya}) are each unity.
\vspace{3mm}

\noindent Rule for the subtraction of fractions:
\vspace{3mm}

 32(ii). Take the difference of the numerators of the
positive and negative fractions reduced to a common denominator.~~~~\renewcommand{\thefootnote}{\hspace{-4.5mm} 2}\footnote{\hspace{-2mm} \en \textit{Cf. BrSpSi}, xii. 2(ii); \textit{MSi}, xv.14(ii); \textit{GT}, p. 18, lines
3-4;
\textit{SiŚe}, xiii. 8; \textit{L} (ASS), p. 35, vs. 37; \textit{GK, I}, p. 11, lines 6-7.}
\vspace{3mm}

 Ex. 8. Subtract $\frac{1}{4}, \frac{1}{3}$ and $\frac{1}{6}$ from 1 and say what remains.
Also, subtract $3 - \frac{1}{2}$ and $2 + \frac{1}{3}$ from 5 (and say the remainder).
\vspace{3mm}

 Ex. 9. Say what remains as the remainder when the
sum of 2 plus $\frac{1}{2}$ terms is subtracted from the sum of 5 plus $\frac{1}{2}$
terms of the series whose first term and common difference are
(each) unity.
\vspace{3mm}

{\small It may be remarked here that, according to the author of the
present work, rule 14(i) applies also to series of natural numbers
having a fractional number of terms. Series in A. P. having fractional number of terms have generally no meaning but they may be
interpreted geometrically by means of figures called series-figures.
See \textit{infra}, 'Determinations pertaining to series'
(\textit{śreḍhī-vyavahāra}).}


\end{sloppypar}

\newpage

\begin{sloppypar}

\noindent Rule for the multiplication of fractions:
\vspace{3mm}

 33(i). The product (of the given fractions) is obtained on
dividing the product of the numerators by the product of the
denominators.\renewcommand{\thefootnote}{1}\footnote{\hspace{-2mm} \en {Cf. BrSpSi}, xii. 3 (ii);  \textit{GSS}, iii. 2;  \textit{MSi}, xv. 15 (i); \textit{GT}, p. 19,
line 26;  \textit{SiŚe}, xiii. 9 (i);  \textit{L} (ASS), p. 36, vs. 39;  \textit{GK, I}, p.12, lines 2-3.\\}
\vspace{3mm}

Ex. 10. \;2 plus $\frac{1}{2}$ is multiplied by 1 plus $\frac{1}{2}$, and 60 plus $\frac{1}{3}$
is multiplied by $\frac{5}{2}$: what are the products, say separately.
\vspace{4mm}

\noindent Rule for the division of fractions:
\vspace{3mm}

 33(ii). Having interchanged the denominator and the
numerator of the divisor, apply the previous rule (i.e., the
rule for multiplication).~~~~\renewcommand{\thefootnote}{\hspace{-4.5mm} 2}\footnote{\hspace{-2mm} \en \textit{Cf BrSpSi}, xii. 4; \textit{GSS}, iii. 8(i);  \textit{MSi}, xv. 15(ii); \textit{GT}, p. 21, line 3-6;  \textit{SiŚe}, xiii, 10; \textit{L} (ASS), p. 37, vs. 41;  \textit{GK, I}, p. 12, lines 11-12.\\}
\vspace{3mm}

 Ex. 11. \;6 plus $\frac{1}{4}$ is divided by $2 + \frac{1}{2}$, and 60 plus $\frac{1}{2}$ is
divided by $3 + \frac{1}{2}$: say the quotients separately.
\vspace{4mm}

\noindent Rule for squaring a fraction:
\vspace{3mm}

 34(i). The square of the numerator divided by the
square of the denominator is the square of the fraction.~~~~\renewcommand{\thefootnote}{\hspace{-4.5mm} 3}\footnote{\hspace{-2mm} \en \textit{Cf. BrSpSi}, xii. 5(i);  \textit{GSS}, iii. 13; \textit{MSi}, xv. 16(i); \textit{GT} p. 22,
line 24;  \textit{SiŚe}, xiii. 9(ii);  \textit{L} (ASS), p. 38, lines 2-3;  \textit{GK},
1, p. 12, lines 17-18.\\}
\vspace{3mm}

 Ex. 12. Say, friend, if you know, the square of 2 plus $\frac{1}{2}$,
of 15 plus $\frac{1}{4}$, of $\frac{1}{2}$, and of $\frac{1}{3}$.
\vspace{4mm}

\noindent Rule for finding the square root of a fraction:
\vspace{3mm}

 34(ii). The square root of the numerator divided by
the square root of the denominator, gives the square root
(of the fraction).~~~~\renewcommand{\thefootnote}{\hspace{-4.5mm} 4}\footnote{\hspace{-2mm} \en \textit{Cf. BrSpSi}, xii. 5(ii);  \textit{GSS}, iii. 13;  \textit{MSi}, xv 16(ii); \textit{GT}, p. 23,
line 25;  \textit{L} (ASS), p. 38, lines 2-3;  \textit{GK, I} p. 12, lines 17-18.\\}
\vspace{4mm}

\noindent Rule for cubing a fraction: 
\vspace{3mm}

 35(i). The cube of the numerator divided by the cube
of the denominator gives the cube (of the fraction).~~~~\renewcommand{\thefootnote}{\hspace{-4.5mm} 5}\footnote{\hspace{-2mm} \en \textit{Cf. GSS}, iii. 13;  \textit{MSi}, xv. 17(i);  \textit{GT}, p. 25, line 13; \textit{L} (ASS), 1,
p. 38, lines 2-3; \textit{GK, I}, p. 12, lines 17-18.}

\end{sloppypar}

\afterpage{\fancyhead[CO] {\small{REDUCTION OF FRACTIONS}}}

\newpage

\begin{sloppypar}

 Ex. 13. Say, if you know, the cube of 7 plus $\frac{1}{2}$, 
of 17 plus $\frac{1}{4}$, of $\frac{1}{4}$, and of $\frac{1}{3}$.
\vspace{4mm}

\noindent Rule for extracting the cube root of a fraction:
\vspace{3mm}

 35(ii). The cube root of the numerator divided by the
cube root of the denominator gives the cube root (of the
fraction).\renewcommand{\thefootnote}{1}\footnote{\hspace{-2mm} \en {\emph{Cf. GSS}}, iii. 13;  \textit{MSi}, xv. 17(ii); \textit{GT}, p. 26, line 27; \textit{L} (ASS), p. 38, lines 2-3;  \textit{GK, I}, p. 12, lines 17-18.\\}
\vspace{3mm}

\begin{center}
 \textbf{(17-22) Reduction of fractions (\textit{kalā-savarṇa})}
\end{center}

\noindent Rule for reducing fractions of the \textit{bhāga} class (i.e., fractions
connected by $+$ and $-$ signs):
\vspace{3mm}

 36. In the \textit{bhāga} class, in order to reduce the (two given)
fractions to a common denominator, remove the common
factor, if any, from the denominators, and then by \,each \,of \,them \,(i.e., \,the \,reduced \,denominators) \,multiply \,the \,denominator \,and numerator of the other fraction.~~~~\renewcommand{\thefootnote}{\hspace{-4.5mm} 2}\footnote{\hspace{-2mm} \en \textit{Cf. BrSpSi}, xii. 2(i);  \textit{MSi}, xv. 13(ii);  \textit{GT}, p. 30, line 16; \textit{SiŚe,} xiii. 11(i); \textit{L} (ASS), p. 28, line 9; \textit{GK, I}, p.
9, vs. 26(i). Also see \textit{GSS}, iii. 55(i).\\}
\vspace{3mm}

 Ex. 14. \;What sum is \,obtained by \,adding together \,the fractions \,having (the integers) 2 to 6 for their denominators,
and 1 for their numerators, and by adding together the fractions having (the integers) 3  to 9 for their denominators and
(the integers) 2 etc. for their respective numerators.
\vspace{4mm}

\noindent Alternative rule for reducing fractions of the \textit {bhāga class:}
\vspace{3mm}

 37. By the lower denominator multiply the upper
numerator, (then) by the upper denominator multiply the
lower denominator, and (then) add the product of the
numerator and the denominator in the middle to the upper
numerator.~~~~\renewcommand{\thefootnote}{\hspace{-4.5mm} 3}\footnote{\hspace{-2mm} \en \textit{Cf. SiŚe}, xiii. 12.}
\vspace{3mm}

{\small Suppose, for example, that we want to reduce the fraction
\vspace{2mm}

\hspace{20mm} $\frac{2}{3} + \frac{4}{5}$}

\end{sloppypar}

\newpage

\begin{sloppypar}

{\small  Writing these fractions one below the other without the fines of
separation as the Hindus used to do, we get
\vspace{1mm}

\hspace{30mm} \begin{tabular}{c} 2\\3\\ 4 \\5\end{tabular}
\vspace{3mm}

[In this scheme, 2 is the upper numerator and 3 the upper denominator; similarly 4 is the lower numerator and 5 the lower
denominator.]
\vspace{3mm}

 Now according to the rule, multiplying the upper numerator by the
lower denominator, and multiplying the lower denominator by the upper
denominator, we get
\vspace{3mm}

\hspace{20mm} \begin{tabular}{c} 2 $\times$ 5 \\ 3\\ 4\\ 5 $\times$ 3\end{tabular}
~~i.e.,~~
\begin{tabular}{c}10\\3\\4\\15\end{tabular}
\vspace{3mm}

 Next, adding the product of the numerator and the denominator in
the middle (viz. 3 and 4) to the upper numerator, we get
\vspace{2mm}

\hspace{20mm} \begin{tabular}{c} $10 + 12$\\15\end{tabular}
~~i.e.,~~
\begin{tabular}{c} 22\\15\end{tabular} 
~~or~~ $\frac{22}{15}$
\vspace{3mm}

\noindent rubbing out the figures in the middle, which are not required.}
\vspace{4mm}

\noindent Rule for reducing fractions of the \textit {prabhāga} class (i.e., fractions
connected by 'of'):
\vspace{3mm}

 38(i). In the \textit {prabhāga} class, one should multiply the numerators together, and the denominators together.\renewcommand{\thefootnote}{1}\footnote{\en \textit{Cf. BrSpSi}  xii. 8(ii);  \textit{GSS} , iii. 99;  \textit{MSi}, xv. 13(i);  \textit{SiŚe}. xiii
11(i);  \textit{L} (ASS), p. 31, vs. 32;  \textit{GK, I}, p. 9, vs. 26(ii).}  
\vspace{3mm}

{\small That is,
\vspace{1mm}

\hspace{20mm} $\frac{a}{b}$  of $\frac{c}{d}$ of $\frac{e}{f}  = \frac{a\,c\,e}{b\,d\,f} $.}
\vspace{4mm}

 Ex. 15. Tell (me) the sum of

\begin{center} $\frac{1}{4}$ of $\frac{1}{2}$ of $\frac{1}{2}$, $\frac{1}{10}$ of $\frac{1}{6}$ of $\frac{1}{5}$ of $\frac{1}{3}$ , and $\frac{1}{7}$ of $\frac{1}{6}$ of \bigg($2 + \frac{1}{2}$\bigg).\end{center}
\vspace{1mm}

\noindent Rule for reducing a fraction of the \textit {bhāga-bhāga} class (i.e., a
whole number divided by a fraction):
\vspace{3mm}

 38(ii). Having multiplied the whole number (\textit{rūpa}) by
the denominator of the fraction, remove the denominator:

\end{sloppypar}

\newpage

\begin{sloppypar}

\noindent this is the process (of reduction) for the \textit{bhāga-bhāga} class.\renewcommand{\thefootnote}{1}\footnote{\hspace{-2mm} \en \textit{Cf. BrSpSi}, xii. 9(i); \textit{GSS}, iii. 99(ii); \textit{MSi}, xv. 19(i).\\} 
\vspace{3mm}

{\small That is,
\vspace{1mm}

\hspace{20mm} $a \div \frac{b}{c} = \frac{ac}{b}$.}
\vspace{4mm}

Ex. 16. Friend, if you know (the method), say after
thinking, what sum will be obtained by adding together (the
fractions):
\vspace{3mm}

\hspace{10mm} $1 \div \frac{1}{6}, \,1 \div \frac{1}{10}, \,1 \div \frac{1}{3}, \,1 \div \frac{1}{9}$, \,and \,$1 \div \frac{1}{2}$ .
\vspace{3mm}

 Ex. 17. 1 has been severally divided by fractions
having (the integers) 3 to 6 for their denominators and (the
integers) 2 etc., for their respective numerators. Say what
sum will be obtained when they are added together.
\vspace{4mm}

\noindent Rule for reducing a fraction of the  \textit{bhāgānubandha} class (i.e., a
whole number increased by a fraction, or a fraction increased by a fraction of itself):
\vspace{3mm}

39. In the \textit {bhāgānubandha} class, the whole number
(\textit{rūpagaṇa}) as multiplied by the denominator (of the fraction)
should be increased by the numerator of
the fraction); or,
the upper denominator having been multiplied by the lower
denominator, the initial numerator (i.e., the upper numerator)
should be multiplied by the sum of the lower numerator and
denominator.~~~~\renewcommand{\thefootnote}{\hspace{-4.5mm} 2}\footnote{\hspace{-2mm} \en \textit{Cf. BrSpSi}, xii 9(ii); \textit{GSS}, iii.113; \textit{MSi}, xv.
11(ii)-12; \textit{GT},
p. 34 lines 15-16; \textit{L} (ASS), p. 32. vs. 34; \textit{GK, I}, p. 9, vs. 27.\\}
\vspace{3mm}

That is,
\vspace{1mm}

\hspace{7mm} (i) $a + \frac{b}{c} = \frac{ac\,+\,b}{c} $,
\vspace{1mm}

\hspace{7mm} (ii) $\frac{a}{b} + \frac{c}{d}$ of $\frac{a}{b}$ \bigg(which was written in the Hindu way as~~\,~~\renewcommand{\thefootnote}{\hspace{-4.5mm} 3}\footnote{\hspace{-2mm} \en Actually it was written as\begin{tabular}{c} a\\b\\ c\\d \end{tabular}}\renewcommand*{\arraystretch}{1.2}\begin{tabular}{c} $\frac{a}{b}$ \\ $\frac{c}{d}$ \end{tabular}\bigg) $= \frac{a\,(d\,+\,c)} {b\,d}$.

\renewcommand*{\arraystretch}{0.7}

\newpage

 Ex. 18. What sum is obtained by adding together
\vspace{2mm}

\hspace{15mm} $1+\frac{1}{2}$, \;$5 + \frac{1}{4}$,\, and \,$8+ \frac{1}{3}$\,?
\vspace{4mm}

Ex. 19. What is the sum of
\vspace{3mm}

\hspace{1cm} $\left(3 + \frac{1}{2}\right) + \frac{1}{4}$ \,of\, $\left(3+ \frac{1}{2}\right) + \frac{1}{6}$ \,of\, $\left\lbrace \left(3 + \frac{1}{2}\right) + \frac{1}{4} \,\textrm{of}\, \left(3+ \frac{1}{2}\right)\right\rbrace$
\vspace{3mm}

and \hspace{5mm} $\frac{1}{2} + \frac{1}{3}$ \,of\, $\frac{1}{2} + \frac{1}{4}$ \,of\, $\left\lbrace\frac{1}{2} + \frac{1}{3} \,\textrm{of}\, \frac{1}{2}\right\rbrace$\,?
\vspace{5mm}

\noindent Rule for reducing a fraction of the  \textit{bhāgāpavāha} class (i.e., a
whole number minus a fraction, or a fraction minus a
fraction thereof):
\vspace{3mm}

 40. In the  \textit{bhāgāpavāha} class, the numerator of the fraction should be subtracted from the whole number multiplied
by the denominator (of the fraction); or having \,multiplied \,the \,upper \,denominator \,by \,the \,lower \,denominator, \,the \,upper
numerator should be multiplied by the lower denominator as
diminished by the lower numerator.\renewcommand{\thefootnote}{1}\footnote{\hspace{-2mm} \en \textit{Cf. BrSpSi}, xii. 9(ii); \textit{GSS}, iii. 126; \textit{MSi}, xv.
11(ii)-12; \textit{GT}, p. 37,
lines 2-5; \textit{L} (ASS), p. 32, vs. 34; \textit{GK, I}, p. 9, vs. 27.\\}
\vspace{3mm}

{\small That is to say,
\vspace{3mm}

\hspace{10mm}(i) $a - \frac{b}{c} = \frac{ac - b} {c}$,
\vspace{1mm}

\hspace{10mm}(ii) $\frac{a}{b} - \frac{c}{d}$ of $\frac{a}{b}$ \bigg(which was written in the Hindu way as~~\,~~\renewcommand{\thefootnote}{\hspace{-4.5mm} 2}\footnote{\hspace{-2mm} \en Actually, it was written as

\begin{tabular}{l}a\\b\\c+\\d\end{tabular}
~or~~
\begin{tabular}{c} a\\b\\$\dot{c}$\\d\end{tabular}

the fractions being written without the separating line, and $+$ or
(\,$\dot{}$\,) being
written to denote subtraction.}\renewcommand*{\arraystretch}{1.2}\begin{tabular}{r} $\frac{a}{b}$ \\ $- \frac{c}{d}$ \end{tabular}\bigg) $= \frac{a\,(d - c)}{b\,d}$.}\\
\vspace{3mm}

\renewcommand*{\arraystretch}{0.7}
 Ex. 20. Say the amount, when $1 - \frac{1}{2}$, \,$5 - \frac{1}{4}$, \,and\, $8 - \frac{1}{3}$ are added together.
\vspace{4mm}

 Ex. 21. What is obtained by adding together:
\vspace{3mm}

 \hspace{1cm} $\left(3 - \frac{1}{2}\right) - \frac{1}{4}$ \,of\, $\left(3 - \frac{1}{2}\right) - \frac{1}{6} \left\lbrace \left(3 - \frac{1}{2}\right) - \frac{1}{4} \,\textrm{of}\, \left(3 - \frac{1}{2}\right)\right\rbrace$
\vspace{2mm}

and \hspace{5mm} $\frac{1}{2} - \frac{1}{3}$ \,of\, $\frac{1}{2} - \frac{1}{4}$ \,of\, $\left(\frac{1}{2} - \frac{1}{3} \,\textrm{of}\, \frac{1}{2}\right)$\,?

\newpage

{\small The commentator interprets '\textit{rūpatrayamardhonam}' as meaning
$3- \frac{1}{2}$ of 3 and not $\left(3 - \frac{1}{2}\right)$, so the first expression according to him
is
\vspace{3mm}

\hspace{8mm} ($3- \frac{1}{2}$ of 3) $- \frac{1}{4}$ of ($3- \frac{1}{2}$ of 3) $- \frac{1}{4}$ of [($3- \frac{1}{2}$ of 3) $- \frac{1}{4}$ of
($3- \frac{1}{2}$ of 3)]
\vspace{3mm}

 The interpretation given by us is more natural and plausible,
especially when we see that a similar interpretation has been given to
the verse of Ex .19, which is analogous to this one.}
\vspace{-0.01mm}

Rule for reducing a chain (\textit{vallī}) of measures:
\vspace{3mm}

41. To reduce a chain (\textit{vallī}), multiply the foremost
denominator and numerator by the lower denominator, and
(then) from or to the foremost numerator subtract or add
(as the case may be) the lower numerator.\renewcommand{\thefootnote}{1}\footnote{\en  \textit{Cf. MSi}, xv. 18; \textit{GT}, p. 39, lines 7-10.}
\vspace{3mm}

Ex. 22. What amount is obtained by reducing
5  \textit{purāṇas}, 3  \textit{paṇas}, 1  \textit{kākiṇī}, $-$1 \textit{varāṭaka}, $- \frac{1}{5}$ of that (i.e., of a  \textit{varāṭaka}) (to  \textit{purāṇas})?
\vspace{3mm}

{\small Converting each measure to the previous one, and writing them one
below the other in the Hindu way, we have
\vspace{3mm}

\hspace{30mm} \begin{tabular}{l} ~~5\\ ~~1\\ ~~3 \\ 16 \\ ~~1\\ ~~4 \\ ~~1$+$ \\ 20\\ ~~1$+$ \\ ~~5\end{tabular}
\vspace{3mm}

\noindent where + indicates subtraction. 
\vspace{3mm}

 Now applying the rule to the uppermost two fractions of this chain, we get
\vspace{3mm}

\hspace{30mm} \begin{tabular}{l} 83\\ 16\\ ~~1\\ ~~4\\ ~~1$+$\\ 20 \\ ~~1$+$\\ ~~5\end{tabular}}

\end{sloppypar}

\newpage

\begin{sloppypar}

{\small  Now applying the rule to the uppermost two fractions of this reduced
chain, we get
\vspace{3mm}

\hspace{30mm} \begin{tabular}{l}333\\ ~~64\\ ~~\,~1$+$\\ ~~20\\ \,~~~1$+$\\ \,~~~5\end{tabular}
\vspace{3mm}

Applying the same rule to this chain, it reduces to
\vspace{3mm}

\hspace{30mm} \begin{tabular}{l} 6659\\ 1280\\ \,~~~~~1$+$\\ ~\,~~~~5 \end{tabular}
\vspace{3mm}

\noindent and then to
\vspace{1mm}

\hspace{20mm} \begin{tabular}{r}33294\\6400\end{tabular} or 
\begin{tabular}{r}16647\\3200\end{tabular}
\vspace{3mm}

Hence the required result is
\vspace{2mm}

\hspace{10mm} $\frac{16647}{3200}$ or $5 \frac {647}{3200}$ \textit{purāṇas}.}
\vspace{4mm}

\noindent Rule for reducing fractions of the  \textit{bhāgamātā} class:
\vspace{3mm}

 42. That (class of fractions) in which fractions of the \textit{bhāga} and other classes occur in combination (of two or
more) is called  \textit{bhāgamātā}. In that (class), the result is obtained
by applying the aforesaid rules separately.\renewcommand{\thefootnote}{1}\footnote{\en {\textit{Cf. GSS}, iii. 138.}}
\vspace{3mm}

 Ex. 23. What amount is obtained by adding together
\vspace{2mm}

\hspace{15mm} $\frac{1}{2}$, $\frac{1}{4}$ of $\frac{1}{4}$, $1 \div \frac{1}{3}$, $\frac{1}{2} + \frac{1}{2}$ of $\frac{1}{2}$, and $\frac{1}{3} - \frac{1}{2}$ of $\frac{1}{3}$\,?
\vspace{4mm}

 Ex. 24. If you know the method of calculation, say
what amount will be obtained by the addition of $\frac{1}{4}$, $\frac{1}{2}$ of $\frac{1}{3}$,
$1 \div \frac{1}{2}$, and $8 + \frac{1}{2}$\,?
\vspace{4mm}

\begin{center} \textbf{(23) Rule of three (\textit{trairāśika})}
\end{center}

\noindent Rule of three;
\vspace{3mm}

 43. In (solving problems on) the rule of three, the
argument (\textit{pramāṇa}) and the requisition (\textit{icchā}), which are
of the same denomination, should be set down in the first and
last places; the fruit (\textit{phala}), which is of a different denomination, should be set down in the middle.

\afterpage{\fancyhead[CO] {\small{RULE OF THREE}}}

\newpage

\noindent (This having been done) that (middle quantity) multiplied by the last
quantity should be divided by the first quantity.\renewcommand{\thefootnote}{1}\footnote{\hspace{-2mm} \en \textit{Cf. Ā}, ii. 26;  \textit{BrSpSi}, xii. 10;  \textit{GSS}, v. 2(i);  \textit{MSi}, xv.
24-25(i);  \textit{GT}, p. 68, vs. 86; \textit{SiŚe}, xiii. 14;  \textit{L} (ASS), p.
71, vs. 73;  \textit{GK, I},
p. 47, vs. 60.\\}
\vspace{3mm}

 Ex. 25. If 1  \textit{pala} and 1  \textit{karṣa} of sandal wood are obtained for ten and a half  \textit{paṇas}, then for how much will
9  \textit{palas} and 1  \textit{karṣa} (of sandal wood of the same quality)
be obtained?
\vspace{3mm}

{\small Here
\vspace{1mm}

\renewcommand*{\arraystretch}{1.4}
\hspace{10mm} \begin{tabular}{rcl} 
argument & $=$ & 1  \textit{pala} and 1  \textit{karṣa}\\
 & $=$ & $1 \frac{1}{4}$ or $\frac{5}{4}$  \textit{palas};\\
fruit & $=$ & $10 \frac{1}{2}$ or $\frac{21}{2}$  \textit{paṇas};\\
 and requisition & $=$ & $9$ \textit{palas} and 1  \textit{karṣa}\\
 & $=$ & $9 \frac{1}{4}$ or $\frac{37}{4}$  \textit{palas} \end{tabular}
\vspace{3mm}

 Writing these quantities as directed in the rule, we have
\vspace{3mm}

\renewcommand*{\arraystretch}{1}
\hspace{20mm} \begin{tabular}{|c|} \hline 5\\4\\\hline \end{tabular}\begin{tabular}{c|} \hline 21\\2\\\hline \end{tabular}\begin{tabular}{c|} \hline 37\\4\\\hline \end{tabular}
\vspace{3mm}

\renewcommand*{\arraystretch}{0.7}

Then applying the rule, the required result
\vspace{3mm}

\hspace{1cm} $= \dfrac{(\frac{21}{2}) \times (\frac{37}{4})}{\frac{5}{4}} = \frac{21 \times 37}{2 \times 5}$~ \textit{paṇas}
\vspace{2mm}

\hspace{1cm} or 4  \textit{purāṇas}, 13  \textit{paṇas}, 2  \textit{kākiṇīs} and 16  \textit{varāṭakas}.}
\vspace{4mm}

 Ex. 26. If 1$\frac{1}{3}$ \textit{palas} of black pepper are obtained for
$\frac{11}{4}$  \textit{paṇas}, then how much of that (black pepper) will be
obtained for ($10 -\frac{1}{3}$) \textit{paṇas}\,? Say quickly.
\vspace{3mm}

 Ex. 27. If one and a half  \textit{droṇas} and three  \textit{kuḍavas} of
grain is obtained for 8, say, if you know, for how much will
one  \textit{khārī} and one  \textit{droṇa} (of that grain) be obtained.
\vspace{-0.01mm}

 Ex. 28. If $60 + \frac{1}{2}$  \textit{khārīs} of grain is obtained for $100 + \frac{1}{3}$ \textit{rūpas}, how much of that (grain) will be obtained for a quarter
of a  \textit{rūpa}\,?~\,~~~\renewcommand{\thefootnote}{\hspace{-4.5mm} 2}\footnote{\hspace{-2mm} \en Examples similar to Exs. 25-28 occur in  \textit{GSS}, iii. 3-6; v. 8,
9, 13,
14;  \textit{GT}, p. 68, vs. 87, p. 69, vs. 88; p. 70, vs. 90; p. 71, vv. 91, 92;
 \textit{L} (ASS), pp. 72-73, vv. 74, 76;  \textit{GK, I}, p. 47, lines 13-14, 17-18,
p. 48,
lines 8-11; and also in Bhāskara I's comm. (629 A.D.) on  \textit{Ā}, ii. 26-27.}

\newpage

{\small The term  \textit{rūpa} stands for any coin whatever. "By the word  \textit{rūpaka} is meant a coin of gold, etc., bearing the name  \textit{paṇa}, etc."\renewcommand{\thefootnote}{1}\footnote{\hspace{-2mm} {\s रूपकशब्देन पणादिसञ्ज्ञितं स्वर्णादिद्रव्यम्} {\en See Paramesvara's commentary on \textit{Ā}, ii 30.}\\}}
\vspace{3mm}

Ex. 29. Where one  \textit{survarṇa} gets $70 + \frac{1}{3}$  \textit{rūpas}, say,
friend, what will 1  \textit{māṣa} as lessened by $\frac{1}{10}$ (of a  \textit{māṣa}) get there.~\,~~~\renewcommand{\thefootnote}{\hspace{-4.5mm} 2}\footnote{\hspace{-2mm} \en \textit{Cf. GT}, p. 73, vs. 95.\\}
\vspace{3mm}

Ex. 30. A certain lame person goes to a distance of $\frac{1}{8}$
of a  \textit{yojana} in $\frac{1}{3}$ of a day, say in how much time will he go
to a distance of 100  \textit{yojanas}.~\,~~~\renewcommand{\thefootnote}{\hspace{-4.5mm} 3}\footnote{\hspace{-2mm} \en \textit{Cf. GSS}, v. 3, 4; \textit{GT}, p. 72, vs. 93.\\}
\vspace{3mm}

Ex. 31. An insect goes to a distance of $\frac{1}{6}$ of an  \textit{aṅgula} in
$\frac{1}{4}$ of a day, in how much time will it go to a distance of 10
and a half  \textit{yojanas}\,?~\,~~~\renewcommand{\thefootnote}{\hspace{-4.5mm} 4}\footnote{\hspace{-2mm} \en \textit{Cf. GSS}, v. 5; \textit{GT}, p. 73, vs. 94; \textit{GK, I}, p. 48, lines 16-19 ff.\\}
\vspace{4mm}

\noindent Forward and backward motion:
\vspace{3mm}

44(i). On subtracting the backward motion per day
from the forward motion per day, the remainder is the (true)
motion per day.~\,~~~\renewcommand{\thefootnote}{\hspace{-4.5mm} 5}\footnote{\hspace{-2mm} \en \textit{Cf. BM}. III, L 1, 60 recto; \textit{GSS}, v. 23; \textit{MS}, xv. 30.\\}
\vspace{3mm}

Ex. 32. The best amongst the elephants goes forward
at the rate of $\frac{1}{2} (1 + \frac{1}{4}) (1- \frac{1}{3}) (1 + \frac{1}{2})$ of a  \textit{yojana} in
$6 \times \frac{1}{5} \times \frac{1}{9} \times \frac{1}{3} (1 + \frac{1}{4})$ of a day, and comes back at the rate of
$2 (1 - \frac{1}{3})$  \textit{yojanas} in $(1 + \frac{1}{2})$ days. Say, friend, in how much
time will he go to a distance of 100  \textit{yojanas}.~\,~~~\renewcommand{\thefootnote}{\hspace{-4.5mm} 6}\footnote{\hspace{-2mm} \en \textit{Cf. GSS}, v. 27. Similar examples occur also in \textit{BM}. See, for
instance, \textit{BM}, III, M1, 21 verso; M4, 36 verso. Also see Pṛthūdaka's
example quoted by Sudhakara. Dvivedi in his comm. on \textit{BrSpSi}, xii.
10-12, p. 179, lines 14-17; and Bhāskara I's example in his comm. on \textit{Ā}, ii. 26-27(i).\\}
\vspace{3mm}

Ex. 33. In how much time will a man, earning at the
rate of $(8- \frac{1}{2})$  \textit{rūpas} in $(1+ \frac{1}{3})$ days and spending on his food
at the rate of $\frac{1}{2}$ per day, be a lord of 100 (\textit{rūpas})\,?~\,~~~\renewcommand{\thefootnote}{\hspace{-4.5mm} 7}\footnote{\hspace{-2mm} \en \textit{Cf. BM}, III, L 1, 60 recto; M 11, 44 recto and verso; M 12, 43 recto; \textit{GSS}, v. 26.}

\end{sloppypar}

\afterpage{\fancyhead[CO] {\small{INVERSE RULE OF THREE}}}

\newpage

\begin{sloppypar}

\begin{center} \textbf{(24) Inverse rule of three (\textit{vyasta-trairāśika})} \end{center}

\noindent Inverse rule of three:
\vspace{3mm}

 44(ii). When there is change in the unit of measurement, the middle quantity multiplied by the first quantity and
divided by the last quantity gives the result.\renewcommand{\thefootnote}{1}\footnote{\hspace{-2mm} \en \textit{Cf. BrSpSi}, xii. 11(i); \textit{MSi}, xv, 25(ii). Also see \textit{L} (ASS), p. 74, vs. 77-78; \textit{GK, I}, p. 49, vs. 61.\\}
\vspace{3mm}

 Ex. 34. When (a given quantity of pearls is) measured
at 8 pearls a necklace, the number of necklaces is twenty;
say, mathematician, what the number of necklaces would be
(when the same quantity of pearls is measured) at 6 pearls a
necklace.~\,~~~\renewcommand{\thefootnote}{\hspace{-4.5mm} 2}\footnote{\hspace{-2mm} \en The same example occurs in \textit{GT}, p. 74, lines 8-9.\\}
\vspace{3mm}

 Ex. 35. Being measured by the  \textit{māṣa} of 5  \textit{raktikās}, a
quantity of gold amounts to 300  \textit{suvarṇas}. Say how much
would that (quantity of gold) amount to, when measured by
the  \textit{māṣa} of 6  \textit{raktikās}.~\,~~~\renewcommand{\thefootnote}{\hspace{-4.5mm} 3}\footnote{\hspace{-2mm} \en \textit{Cf. L} (ASS), p. 76, vs. 81; \textit{GK, I}, P. 50, lines 3-4, 7-8. A
similar example occurs in Bhāskara I's comm. on  \textit{Ā}, ii. 26-27(i).\\}
\vspace{2mm}

{\small \textit{Raktikā} is the same as  \textit{guñjā}.}
\vspace{3mm}

 Ex. 36. How much gold of 11  \textit{varṇas} can be had in
exchange for 168  \textit{suvarṇas} of 16  \textit{varṇas}?~\,~~~\renewcommand{\thefootnote}{\hspace{-4.5mm} 4}\footnote{\hspace{-2mm} \en \textit{Cf. GT}, p. 74, vs. 96. Also  \textit{Cf. GSS}, v. 18(i); vi. 173$\frac{1}{2}$, 174$\frac{1}{2}$;
\textit{L} (ASS), p. 75, vs. 80. For the meaning of varṇa see \textit{infra} p. 36.\\}
\vspace{3mm}

 Ex. 37. Quickly say how many blankets of length 6
(cubits) and breadth 2 (cubits) can be made out of (the yarn
which yields) 200 blankets of length 9 (cubits) and breadth
3 (cubits).~\,~~~\renewcommand{\thefootnote}{\hspace{-4.5mm} 5}\footnote{\hspace{-2mm} \en \textit{Cf. GSS}, v. 19.\\}
\vspace{3mm}

 Ex. 38. How much gold of 10 $\frac{1}{4}$  \textit{varṇas} will be obtained
in exchange for 100  \textit{suvarṇas} and 8  \textit{māṣa} (of gold) of
12 $\frac{1}{2}$  \textit{varṇas}\,?~\,~~~\renewcommand{\thefootnote}{\hspace{-4.5mm} 6}\footnote{\hspace{-2mm} \en \textit{Cf. GSS}, v. 18(ii).}
\vspace{4mm}

\begin{center}
\textbf{(25-27) Rules of five, seven and nine
(\textit{pañca-sapta-nava-rāśika})}
\end{center} 

\noindent Rule of five, seven, and nine:
\vspace{3mm}

 45. After transposing the fruit from one side to the other, and then having transposed the denominators (in like manner)

\end{sloppypar}

\newpage

\noindent and multiplied the numbers (so obtained on either
side), divide the side with larger number of quantities (numerators) by the other.\renewcommand{\thefootnote}
{1}\footnote{\en  \textit{Cf. BrSpSi}, xii. 11(ii)-12; \textit{GSS}, v. 32; \textit{MSi}, xv. 26, 27; \textit{GT}, p.
74, vs. 97; \textit{SiŚe}, xiii, 15; \textit{L} (ASS), pp. 76-77, vs. 82; \textit{GK, I}, p. 50, vs.
62.}
\vspace{3mm}

{\small The two sides referred to in this rule are known as (i) the argument
side (\textit{pramāṇarāśi-pakṣa}) and (ii) the requisition side
(\textit{icchārāśi-pakṣa}). In
Ex. 40, for instance, these sides are as follows:
\vspace{3mm}

\renewcommand*{\arraystretch}{1.2}

\hspace{10mm} (i) argument side $\left\lbrace \begin{matrix}
100\frac{1}{2} ~\textrm{or}~ \frac{201}{2}\\
\frac{1}{3}\\
1\frac{1}{2} ~\textrm{or}~ \frac{3}{2}
\end{matrix} \right. $
\vspace{3mm}

\hspace{9mm} (ii) requisition side \Bigg\{ $\begin{matrix} 
60\frac{1}{4} ~\textrm{or}~ \frac{241}{4}\\
8-\frac{1}{2} ~\textrm{or}~ \frac{15}{2}
\end{matrix}$
\vspace{3mm}

 These are written as
\vspace{3mm}

\renewcommand*{\arraystretch}{1.4}
\hspace{23mm} (i) \hspace{4mm} (ii)

\hspace{20mm} \begin{tabular}{|c|c|} \hline $\frac{201}{2}$ & $\frac{241}{4}$ \\$\frac{1}{3}$ & $\frac{15}{2}$\\$\frac{3}{2}$ & 0\\\hline \end{tabular}
\vspace{3mm}

\noindent 0 being written in place of the desired quantity (unknown).
\vspace{3mm}

 Then the rule is applied as follows:
\vspace{3mm}

 Transposing the fruit, we get
\vspace{3mm}

\hspace{23mm} (i) \hspace{4mm} (ii)

\hspace{20mm} \begin{tabular}{|c|c|} \hline $\frac{201}{2}$ & $\frac{241}{4}$ \\$\frac{1}{3}$ & $\frac{15}{2}$\\ 0 & $\frac{3}{2}$\\\hline \end{tabular}}
 
\afterpage{\fancyhead[CO] {\small{RULE OF FIVE, SEVEN, AND NINE}}}

\newpage

{\small Transposing the denominators, we get
\vspace{3mm}

\renewcommand*{\arraystretch}{1.4}
\hspace{23mm} (i) \hspace{4mm} (ii)

\hspace{20mm} \begin{tabular}{|c|c|} \hline $\frac{201}{4}$ & $\frac{241}{2}$ \\$\frac{1}{2}$ & $\frac{15}{3}$\\ $\frac{0}{2}$ & 3\\\hline \end{tabular}
\vspace{3mm}
 
\renewcommand*{\arraystretch}{0.7}

Now we see that the number of quantities (numerators) in (ii) is
greater than that in (i). Hence the unknown quantity (interest)
\vspace{3mm}

\hspace{10mm} $= \dfrac{241 \times 2 \times 15 \times 3 \times 3}{201 \times 4 \times 1 \times 2 \times 2}$
\vspace{3mm}

\hspace{10mm} $= 20\frac{125}{536}$ }
\vspace{4mm}

Ex. 39. If the interest on 100 for a month be 5, what
is the interest on 60 for a year? From the interest say the
time; and from them both, (say) the unknown principal.\renewcommand{\thefootnote}{1}\footnote{\hspace{-2mm} \en \textit{Cf. GSS}, v. 33, 41; \textit{GT}, p. 75, vs. 98; \textit{L} (ASS), p. 77, vs. 83;
\textit{GK, I}, p. 50. lines 18-21. Similar examples occur in Bhāskara I's
comm.
on \textit{Ā}, ii. 26-27(i).\\}
\vspace{3mm}

 Ex. 40. If 1$\frac{1}{2}$ be the interest on 100$\frac{1}{2}$ for one-third of a
month, what will be the interest on 60$\frac{1}{4}$ for ($8 - \frac{1}{2}$) months.~~~~\renewcommand{\thefootnote}{\hspace{-4.5mm} 2}\footnote{\hspace{-2mm} \en \textit{Cf. GSS}, v. 34; \textit{GT}, p. 76, vs. 99; \textit{GK, I}, p. 51, lines 6-9.
Similar
examples occur in Bhāskara I's comm. on \textit{Ā}, ii. 26-27(i).\\}
\vspace{3mm}

 Ex. 41. When the price of a  \textit{suvarṇa} of gold of 16  \textit{varṇas}
is 60, then say the price of 63  \textit{suvarṇas} of gold of 10 \textit{varṇas}.~~~~\renewcommand{\thefootnote}{\hspace{-4.5mm} 3}\footnote{\hspace{-2mm} \en \textit{Cf. GSS}, v. 35.\\}
\vspace{3mm}

 Ex. 42. One-half of a  \textit{suvarṇa} of pure gold as diminished
by 1  \textit{guñjā} costs 20$\frac{1}{2}$; what will 3  \textit{guñjās} of gold of 11$\frac{1}{2}$ \textit{varṇas} cost\,?
\vspace{3mm}

 Ex. 43. If 8  \textit{droṇas} of rice are carried to a distance of
one  \textit{yojana} for 6  \textit{paṇas}, say for how much will a  \textit{khārī} together
with a  \textit{droṇa} (of rice) be carried to a distance of 3  \textit{yojanas}.~~~~\renewcommand{\thefootnote}{\hspace{-4.5mm} 4}\footnote{\hspace{-2mm} \en \textit{Cf. GSS}, v. 36; \textit{GT}, p. 78, vs. 101.}

\newpage

 Ex. 44. If 3 labourers earn 5  \textit{rūpas} in 2 days, say what
will 8 labourers earn in 9 days\,?\renewcommand{\thefootnote}{1}\footnote{\hspace{-2mm} \en \textit{Cf. GT}, p. 77, vs. 100.\\}
\vspace{3mm}

 Ex. 45. If a blanket, whose breadth is 2 (cubits) and
length 8 (cubits), gets 10, what will 2 other (similar) blankets
of breadth 3 (cubits) and length 9 (cubits) get\,?~~~~\renewcommand{\thefootnote}{\hspace{-4.5mm} 2}\footnote{\hspace{-2mm} \en This example is the same as that given in \textit{GT}, p. 78, vs. 102.\\}
\vspace{3mm}

 Ex. 46. If a (rectangular piece of) stone with length,
breadth, and thickness equal to 9, 5, and 1 cubits respectively
costs 8, what will two other (rectangular pieces of stone) of
dimensions 10, 7, and 2 cubits cost?~~\,~~\renewcommand{\thefootnote}{\hspace{-4.5mm} 3}\footnote{\hspace{-2mm} \en This example is the same as that given in \textit{GT}, p. 79, vs. 103.
Also  \textit{Cf. L} (ASS), p. 79-80, vs. 85; \textit{GK, I}, p. 52, lines 2-5.\\}
\vspace{3mm}

 Ex. 47. If the diet of an elephant of diameter 2 (cubits),
height 6 (cubits) and length 7 (cubits) is one \textit{droṇa}, what
should be the diet of an elephant of diameter 3 (cubits), height
9 (cubits) and length 10 (cubits)?~~\,~~\renewcommand{\thefootnote}{\hspace{-4.5mm} 4}\footnote{\hspace{-2mm} \en Similar examples occur in Bhāskara l's comm. on  \textit{Ā}, ii. 26-27(i)
and in Pṛthūdaka's comm. on \textit{BrSpSi}, xii. 10-12.\\}
\vspace{3mm}

 \begin{center} \textbf{(28) Barter of commodities (\textit{bhāṇḍa-pratibhāṇḍa})}\end{center}

\noindent Rule for problems on the barter of commodities:
\vspace{3mm}

 46(i). In (problems on) the barter of commodities,
having transposed the prices (of the commodities), apply the
previous rule (i.e., rule of five, etc.).~~~~\renewcommand{\thefootnote}{\hspace{-4.5mm} 5}\footnote{\hspace{-2mm} \en \textit{Cf. BrSpSi}, xii. 13(i); \textit{MSi}, xv. 28(i); \textit{GT}, p. 80,
lines 16-17; \textit{SiŚe}, xiii,
16(i); \textit{L} (ASS), p. 83 vs. 88; \textit{GK, I}, p. 53, lines 2-3. Also see
\textit{GSS}, vi. 18.\\}
\vspace{3mm}

 Ex. 48. If 2  \textit{palas} of dry ginger are obtained for 6
and one  \textit{pala} of long pepper for 9, how much of long
pepper will then be obtained for 6  \textit{palas} of dry ginger?~~\,~~\renewcommand{\thefootnote}{\hspace{-4.5mm} 6}\footnote{\hspace{-2mm} \en \textit{Cf. GSS}, v. 37, 38 and vi. 19-20; \textit{GT}, p. 81, lines
6-9.}

\afterpage{\fancyhead[CO] {\small{SALE OF LIVING BEINGS}}}

\newpage

\renewcommand*{\arraystretch}{1}
Setting down the argument and requisition sides, we have
\vspace{3mm}

\hspace{20mm} \begin{tabular}{|c|}\hline 6\\2\\6\\\hline \end{tabular}\begin{tabular}{c|}\hline 9\\1\\ \\\hline \end{tabular}
\vspace{3mm}

Transposing the prices of the commodities, we get
\vspace{3mm}

\hspace{20mm} \begin{tabular}{|c|}\hline 6\\1\\6\\\hline \end{tabular}\begin{tabular}{c|}\hline 9\\2\\ \\\hline \end{tabular}
\vspace{3mm}

\renewcommand*{\arraystretch}{0.7}
Now we have to apply the rule of five, etc. But the fruits (i.e., prices)
have been already transposed, and there are no denominators on any
side. Hence the required answer
\vspace{1mm}

\hspace{10mm} $= \dfrac{6 \times 1 \times 6}{9 \times 2}$
\vspace{2mm}

\hspace{10mm} $=$ 2 \textit{palas} of long pepper.
\vspace{3mm}

 Ex. 49. If 16 mangoes are obtained for 2  \textit{paṇas} and
100 wood-apples for 3 (\textit{paṇas}), say then how many wood
apples will be obtained for 6 mangoes.\renewcommand{\thefootnote}{1}\footnote{\hspace{-2mm} \en \textit{Cf. GSS}, v. 38; \textit{GT}, p. 80, lines 23-26; \textit{L} (ASS), p. 84, vs. 89; \textit{GK, I}, p. 53, lines 5-8. A similar example occurs in Pṛthūdaka's comm. on BrSpSi, xii. 13.\\}
\vspace{3mm}

\begin{center}  \textbf{(29) Sale of living beings (\textit{jīva-vikraya})}\end{center}

\noindent Rule for problems on the sale of living beings:
\vspace{3mm}

 46(ii). In (problems on) the sale of living beings, the same rule (of five, etc.) is applied after transposing the ages
(of the living beings).~~~~\renewcommand{\thefootnote}{\hspace{-4.5mm} 2}\footnote{\hspace{-2mm} \en \textit{Cf. GT}, p. 81, vs. 108; \textit{SiŚe}, xiii. 16(ii); \textit{GK, I}, 53, vs. 63.\\}
\vspace{3mm}

 Ex. 50. If 5 women of 16 years of age get 200, say
then, O mathematician, how much will 2 women of 20 years of age get?~~~~\renewcommand{\thefootnote}{\hspace{-4.5mm} 3}\footnote{\hspace{-2mm} \en \textit{Cf. GSS}, v. 40; \textit{GT}, p. 81, vs. 109; \textit{L} (ASS), p. 74, vs. 79; \textit{GK}, p. 53, lines 15-18.}

\newpage

\renewcommand*{\arraystretch}{1}
Setting down the argument and requisition sides, we have
\vspace{3mm}

\hspace{20mm} \begin{tabular}{|c|}\hline 5\\16\\200\\\hline \end{tabular}\begin{tabular}{c|}\hline 2\\20\\ \\\hline \end{tabular}
\vspace{3mm}

Transposing the ages (of the women), we get
\vspace{3mm}

\hspace{20mm} \begin{tabular}{|c|}\hline 5\\20\\200\\\hline \end{tabular}\begin{tabular}{c|}\hline 2\\16\\ \\\hline \end{tabular}
\vspace{3mm}

Now we have to apply the rule of five, etc. Therefore, transposing
the fruit, we have
\vspace{3mm}

\hspace{20mm} \begin{tabular}{|c|}\hline 5\\20\\ \\\hline \end{tabular}\begin{tabular}{c|}\hline 2\\16\\200 \\\hline \end{tabular}
\vspace{3mm}

\renewcommand*{\arraystretch}{0.7}
There being no denominators on any side, the required amount
\vspace{3mm}

\hspace{15mm} $= \dfrac{2 \times 16 \times 200}{5 \times 20} = 64 $
\vspace{5mm}

Ex. 51. If 3 camels of 10 years of age get 108  \textit{purāṇas},
say then what will 5 camels of 9 years of age get. \renewcommand{\thefootnote}
{1}\footnote{\en A closely similar example occurs in \textit{GT}, p. 82, vs. 110. Also \textit{Cf. GSS}, v. 39.}

\afterpage{\fancyhead[CO] {\small{DETERMINATIONS}}}
\afterpage{\fancyhead[CE] {\small{DETERMINATIONS}}}

\newpage

\phantomsection \label{mix}
\begin{center} {\large DETERMINATIONS (\textit{Vyavahāra})}
\vspace{2mm}

\textbf{(1) Determinations pertaining to mixtures of things}

 \textbf{(\textit{miśraka-vyavahāra})}
\vspace{2mm}

 (i) \textit{SIMPLE INTEREST}
\end{center}

\noindent Rule for finding the capital and the interest, when the
amount and the rate of interest are given:
\vspace{3mm}

 47. Multiply the argument (\textit{pramāṇa-rāśi}) by its time,
and the other time by the fruit (\textit{phala}); divide each of those
(products) by their sum, and multiply by the amount (i.e., capital plus interest). The results (thus obtained) give the
capital (\textit{mūla-dhana}) and the
interest (\textit{vṛddhi-dhana}) respectively.\renewcommand{\thefootnote}{1}\footnote{\hspace{-2mm} \en \textit{Cf. MSi}, xv. 31; \textit{GT}, p. 82, vs. 111; \textit{SiŚe}, xiii. 17; \textit{L} (ASS), p. 85,
vs. 90. Also see \textit{BrSpSi}, xii. 14(ii); \textit{GSS}, vi. 21, 23.\\}
\vspace{3mm}

{\small That is,
\vspace{2mm}

\hspace{20mm} Capital $= \dfrac{\textrm{argument} \times \textrm{time} \times \textrm{amount}}{\textrm{argument} \times \textrm{time} + \textrm{fruit} \times \textrm{other time}}$ 
\vspace{3mm} 

\hspace{20mm} Interest $= \dfrac{\textrm{fruit} \times \textrm{other time} \times \textrm{amount}} {\textrm{argument} \times \textrm{time} + \textrm{fruit} \times \textrm{other time}}$}
\vspace{4mm} 

 Ex. 52. The rate of interest being 5 percent per
month, a certain sum is seen to amount to 96 in a year. Say,
friend, what is the capital and what the interest?~~~~\renewcommand{\thefootnote}{\hspace{-4.5mm} 2}\footnote{\hspace{-2mm} \en The same example occurs in \textit{GT}, p. 83, vs. 112. Also see \textit{GSS},
vi. 22, 24; \textit{L} (ASS), p. 86, vs. 91; \textit{GK, I}, p. 60, lines 10-11. Similar
examples occur also in Pṛthūdaka's comm. on \textit{BrSpSi}, xii. 14.}
\vspace{3mm}

{\small Here argument $=$ 100, time $=$ 1 month, fruit $=$ 5, amount $=$ 96,
and other time $=$ 1 year, i.e., 12 months. Therefore 
\vspace{3mm}

\hspace{20mm} Capital $= \dfrac{100 \times 1 \times 96}{100 \times 1 +5 \times 12} = 60$, 
\vspace{3mm} 

\hspace{20mm} Interest $= \dfrac{5 \times 12 \times 96}{100 \times 1 + 5 \times 12} = 36$.}
\vspace{4mm} 
 
Ex. 53. The interest on $100\frac{1}{2}$ for one month and a
quarter being $1\frac{1}{2}$, a certain sum amounts to $36\frac{1}{2}$ in a period
of $7\frac{1}{2}$ months. (Find the sum and the interest accrued
thereon).
\vspace{4mm}

\noindent Rule for separating the capital, interest, and the commission
of the surety etc., from the given amount:

\newpage

 48. Divide the product of the argument and its time as also the fruit (i.e., interest), etc., as multiplied by the other
time by their sum, and then multiply them by the mixed
amount: then are obtained the capital etc. in their respective
order.\renewcommand{\thefootnote}{1}\footnote{\hspace{-2mm} \en \textit{Cf. GT}, p. 83, vs. 113; \textit{SiŚe}; xiii. 18.\\}
\vspace{3mm}

{\small This rule is similar to the previous one.}
\vspace{3mm}

 Ex. 54. The rate of interest being 5 percent per
month, the commission of the surety (\textit{bhāvyaka}) 1 percent per
month, the fee of the calculator (\textit{vṛtti}) $\frac{1}{2}$ percent per month,
and the charges of the scribe $\frac{1}{4}$ percent per month, a certain
sum amounts to 905 in a year. (Find the capital, the interest,
and the shares of the surety, calculator, and the scribe).~~~~\renewcommand{\thefootnote}{\hspace{-4.5mm} 2}\footnote{\hspace{-2mm} \en The same example occurs in \textit{GT}, p. 83, vs. 114.}
\vspace{3mm}

{\small Here argument $=$ 100, time $=$ 1 month, first fruit $=$ 5, second
fruit $=$ 1, third fruit $= \frac{1}{2}$, fourth fruit $= \frac{1}{4}$; amount $=$ 905, and
other time $=$ 1 year, i.e., 12 months.

\begin{center}
\begin{tabular}{rcl} 
$\therefore$\; Capital & $=$ & $\dfrac{\textrm{argument} \times \textrm{time} \times \textrm{amount}}
{\textrm{argument} \times \textrm{time} + \textrm{(sum of fruits)} \times \textrm{other time}}$\\
\\
 & $=$ & $\dfrac{100 \times 1 \times 905}{100 \times 1 + (5 + 1 + \frac{1}{2} + \frac{1}{4}) \times 12}$\\
 \\
 & $=$ & 500.\\
 \\
Interest & $=$ & $\dfrac{\textrm{first fruit} \times \textrm{other time} \times \textrm{amount}} {\textrm{argument} \times \textrm{time} + \textrm{(sum of fruits)} \times \textrm{other time}}$\\
\\
 & $=$  & $\dfrac{5 \times 12 \times 905}
{100 \times 1 + (5 + 1 + \frac{1}{2} + \frac{1}{4}) \times 12}$\\
\\
 & $=$ & 300.\\
\\
Commission of surety & $=$ & $\dfrac{\textrm{second fruit} \times \textrm{other time} \times \textrm{amount}}{\textrm{argument} \times \textrm{time} + \textrm{(sum of fruits)} \times \textrm{other time}}$\\
\\
 & $=$ & $\dfrac{1 \times 12 \times 905}
{100 \times 1 + (5 + 1 + \frac{1}{2} + \frac{1}{4})\times 12}$\\
\\
 & $=$ & 60.
\end{tabular}
\end{center}

Similarly, fee of the calculator $=$ 30, and charges of the scribe $=$ 15.}
\vspace{4mm}

\noindent Rule for finding the time in which a sum lent out at simple
interest will be paid back by equal monthly installments:
\vspace{3mm}

 49-50. \,To \,get \,the \,desired \,time \,(\textit{ipsitakālopanaye}) \,subtract \,from \,the \,capital (\textit{mūlāt}) the present \,worth of the \,first monthly \,installment, the second \,monthly installment, etc., (\textit{mūla}), one after another (\textit{pṛthak pṛthak}). 

\afterpage{\fancyhead[CO] {\small{MIXTURES. SIMPLE INTEREST}}}

\newpage

\noindent [This will give
the number of complete months elapsed (\textit{gata-māsa}) and a
residue of the capital (\textit{mūla-śeṣa}), if any]. Now calculate
interest for a month on the residue (\textit{śeṣasya māsika-phalam}),
and subtract that from the amount of monthly installment
(\textit{māsikopanayāt}): by the remainder divide the interest for
a month on the residue (\textit{māsikaphala}) as multiplied by the
number of complete months elapsed (\textit{māsa}) and increased \,by \,the \,residue \,of \,the \,capital \,(\textit{mūlaśeṣa}).\renewcommand{\thefootnote}{1}\footnote{\hspace{-2mm} \en The reading  \textit{māsaśeṣayug} occurring in the Sanskrit text should be  \textit{mūlaśeṣayug}.\\}  The \,quotient \,added \,to \,the number of complete months elapsed (\textit{gatamāsa}) gives
the time (in months) of recovery of the capital (together
with interest).~~~~\renewcommand{\thefootnote}{\hspace{-4.5mm} 2}\footnote{\hspace{-2mm} \en \textit{Cf. GK, I}, p. 72, lines 14-15; and p. 73, lines 1-6.}
\vspace{3mm}

{\small Let $C$ be the capital lent out, $i$ the rate of interest percent per month, $I$ the amount of monthly installment of payment, and $T + \frac{1}{t}$,
t\,\textgreater\,1, the time in months in which the money is recovered
with interest.
\vspace{3mm}

 Now suppose that the \textit{P.W.s} of the \textit{1st, 2nd, ..., Tth} installments
having been subtracted one after another from $C$, the residue left is
$R$. Then evidently, $R$ is the \textit{P.W.} of $\frac{I}{t}$ for ($T + \frac{1}{t}$) months, so that

\begin{center}
\begin{tabular}{rcl} 
R & $=$ & $\dfrac{100 \times \frac{I}{t}}{100 + (T + \frac{1}{t}) i}$  \\
\\
whence~~ $\frac{1}{t}$ & $=$ & $\dfrac{\frac{Ri}{100} . T + R}{I - \frac{Ri}{100}} $\\
\end{tabular}
\vspace{3mm}

Hence the required time $T + \frac{1}{t}$ ~~~~~~
\vspace{3mm}

\begin{tabular}{rcl} 
\hspace{26mm} & $=$ & $T + \dfrac{\frac{Ri}{100} . T + R}{I - \frac{Ri}{100}}$\;  months. 
\end{tabular}
\end{center}}
\hspace{2mm}

Ex. 55-56. A rich man lent to somebody a sum of 100
 \textit{rūpas} at 5 percent (per month simple interest) and from him took a house bearing a rent of 40 (\textit{rūpas}) per month.

\newpage

\noindent Say, learned man, after how much time is the debter relieved of
his debt, and what does the rich man get by the gain of bare
accommodation.\renewcommand{\thefootnote}{1}\footnote{\hspace{-2mm} \en A similar example occurs in \textit{GK, I}, p, 73, lines 8-10.\\}
\vspace{4mm}

\noindent Rule for converting several bonds into one:
\vspace{3mm}

 51. The sum (\textit{samāsa}) of the interests (\textit{phala}) (accruing on the given bonds) for the elapsed months (\textit{gatakāla}), being
divided by the sum (\textit{aikya}) of the interests (on the same
bonds) for one month, gives the time (in months for the equivalent single bond); and 100 times the sum of the interests for one month (on the bonds), being divided by the sum of the
capitals (\textit{dhanayoga}) (of the bonds), gives the rate of interest
percent (per month) (for the single bond).~~~~\renewcommand{\thefootnote}{\hspace{-4.5mm} 2}\footnote{\hspace{-2mm} \en \textit{Cf. GT}, p. 86, lines 12-15. Also see \textit{GSS}, vi. 77-77$\frac{1}{2}$.\\}
\vspace{3mm}

{\small Let the \,capitals, \,rates of interest, \,and \,the times elapsed \,for the given \,\textit{n} bonds \,be as follows:
\vspace{2mm}

\hspace{15mm} Capitals: $P_1, P_2, P_3, ..., P_n$.
\vspace{2mm}

\hspace{15mm} Rates of interest: $r_1, r_2, r_3, ..., r_n$\; percent per month
\vspace{2mm}

\hspace{15mm} Times elapsed: $t_1, t_2, t_3, ..., t_n$\; months.
\vspace{3mm}

\noindent Then the time elapsed (t) and the rate of interest (r) for the
equivalent single bond are given by
\vspace{2mm}

\hspace{15mm} $t = \dfrac{\dfrac{P_1\,t_1\,r_1}{100} + \dfrac{P_2\,t_2\,r_2}{100} + ... + \dfrac{P_n\,t_n\,r_n}{100}}{\dfrac{P_1\,r_1}{100} + \dfrac{P_2\,r_2}{100} + \dfrac{P_3\,r_3}{100} + ... + \dfrac{P_n\,r_n}{100}}$\; months
\vspace{4mm}

\hspace{9mm} and~ $r = \dfrac{\left(\dfrac{P_1\,r_1}{100} + \dfrac{P_2\,r_2}{100} + \dfrac{P_3\,r_3}{100} + ... + \dfrac{P_n\,r_n}{100}\right) \times 100}{P_1 + P_2 + P_3 + ... + P_n}$\;  percent per month.
\vspace{3mm}

Āryabhaṭa II puts the above results in the following simplified forms:~~\,~~\renewcommand{\thefootnote}{\hspace{-4.5mm} 3}\footnote{\hspace{-2mm} \en See \textit{MSi}, xv. 33. The reading as emended by S. Dwivedi is
incorrect.}
\vspace{3mm}

\hspace{15mm} $t = \dfrac{P_1\,t_1\,r_1 + P_2\,t_2\,r_2 + ... + P_n\,t_n\,r_n}{P_1\,r_1 + P_2\,r_2 + ... + P_n\,r_n}$\; months
\vspace{4mm}

\hspace{9mm} and~ $r = \dfrac{P_1\,r_1 + P_2\,r_2 + ... + P_n\,r_n}{P_1 + P_2 + ... + P_n}$\;  percent per month.}

\newpage

 Ex. 57-58. (There are 4 bonds on which) capitals
amounting to 100, 200, 300, and 400 are given (to someone
on interest) at the rates of 2, 3, 4, and 5 percent (per month)
in the respective order; and months amounting to 2, 3, 5, and
4 each multiplied by 2, have passed (since the execution of
the respective bonds). Say, how would a single bond
(\textit{eka-patra}) be now made out of these.\renewcommand{\thefootnote}{1}\footnote{\hspace{-2mm} \en \textit{Cf. GSS}, vi. 78$\frac{1}{2}$; \textit{GT}. p. 86, lines 25-28 (contd. on p. 87).\\}
\vspace{3mm}

 Ex. 59. Also say, O learned (mathematician), how a
single bond be made (out of 4 bonds) with the same capitals
as previously stated and with rates percent (per month) of
interest augmented by $\frac{1}{2}$ (in each case) and months elapsed
increased by $\frac{1}{4}$ (in each case).~~~~\renewcommand{\thefootnote}{\hspace{-4.5mm} 2}\footnote{\hspace{-2mm} \en \textit{Cf. GT}, p. 87, lines 22-23.\\}
\vspace{4mm}

\noindent Rule for finding the time in which a capital lent out on interest at a given rate would become a (given) multiple of itself:
\vspace{3mm}

 52(i). The product of the time and the argument,
being divided by the fruit and (then) multiplied by the
multiple minus one, gives the (required) time.~~~~\renewcommand{\thefootnote}{\hspace{-4.5mm} 3}\footnote{\hspace{-2mm} \en \textit{Cf. BrSpSi}, xii. 14(i); \textit{GT}, p. 85, line 19. Both are literally the same.\\}
\vspace{3mm}

 Ex. 60(i). A sum of money is put to interest at 5 percent per month. When will it become twice of itself?~~\,~~\renewcommand{\thefootnote}{\hspace{-4.5mm} 4}\footnote{\hspace{-2mm} \en A similar example occurs in Pṛthūdaka's comm. on \textit{BrSpSi},
xii. 14 The same example occurs in \textit{GT}, p. 85, lines 24-25.\\}
\vspace{3mm}

 Ex. 60(ii). And when will (another sum of money) put to interest at 3$\frac{1}{2}$ percent (per month) become $1\frac{1}{4}$ of
itself?~\,~~~\renewcommand{\thefootnote}{\hspace{-4.5mm} 5}\footnote{\hspace{-2mm} \en A similar example occurs in Pṛthūdaka's comm. on \textit{BrSpSi}, xii. 14. The same example occurs in \textit{GT}, p. 86, lines 4-5.}
\vspace{3mm}

{\small In Ex. 60(i), time $=$ 1 month, argument $=$ 100, fruit $=$ 5, and multiple $=$ 2. Therefore the required time
\vspace{2mm}

\hspace{15mm} $= \dfrac{1 \times 100}{5} \times (2 - 1)$, i.e., 20 months.
\vspace{3mm}
 
Similarly in Ex. 60(ii), time $=$ 1 month, argument $=$ 100, fruit $= 3\frac{1}{2}$, i.e., $\frac{7}{2}$, and multiple $= 1\frac{1}{4}$. Therefore the required time
\vspace{3mm}

\hspace{15mm} $= \dfrac{1 \times 100}{\frac{7}{2}}  \times (1\frac{1}{4} - 1)$, i.e., $\dfrac{50}{7}$ months 
\vspace{2mm}

\hspace{15mm} $=$ 7 months, 4 days, 17 \textit{ghaṭikās}, 8$\frac{4}{7}$  \textit{caṣakas}.}

\newpage

 \begin{center} (ii) \textit{ALLIGATION} \end{center}

\noindent Rule for finding the  \textit{varṇa} of the alloy obtained by melting together a number of pieces of gold of given weights and \textit{varṇas}:
\vspace{3mm}

 52(ii). The sum of the products of weight and  \textit{varṇa} of the several pieces of gold, being divided by the sum of the weights of the pieces of gold, gives the  \textit{varṇa} (of the alloy).\renewcommand{\thefootnote}{1}\footnote{\hspace{-2mm} \en \textit{Cf. BM, III}, H 1, 16 verso; \textit{GSS}, vi. 169(i), $172\frac{1}{2}$; \textit{L} (ASS), p. 99, vs. 103(i); \textit{GK, I}, p. 76, lines 5-7.}
\vspace{3mm}

{\small That is, if \textit{n} pieces of gold of weights ${w_1}, {w_2}, {w_3}$, ...,
$w_n$ and \textit{varṇas} ${v_1}, {v_2}, {v_3}, ..., {v_n}$ respectively are melted
together, then the  \textit{varṇa}
$v$ of the alloy is given by
\vspace{3mm}

\hspace{15mm} $v = \dfrac{w_1\,v_1 + w_2\,v_2 + ... + w_n\,v_n} {w_1 + w_2 + ... + w_n}$
\vspace{3mm}

The term \textit{varṇa} used above is analogous (though not equivalent) to the modern term 'carat'. It is used to denote the fineness of gold. Gold
of I  \textit{varṇa} contains 1 part of pure gold and 15 parts of impurities (in
the
form of baser metals); gold of 2  \textit{varṇas} contains 2 parts of
pure gold and
14 parts of impurities; and so on. Pure gold, called
 \textit{kalyāna-suvarṇa}, is of
16  \textit{varṇas} (or 24 \textit{carats}). Thus the amount of pure gold in a piece of gold of weight $w$ and  \textit{varṇa}  $v$ is equal to $\frac{wv}{16}$.
\vspace{3mm}

 The rationale of the above rule is as follows: The amount of pure gold in the \textit{n} pieces of gold mentioned above is equal to
\vspace{3mm}

\hspace{15mm} $\dfrac{w_1\,v_1}{16} + \dfrac{w_2\,v_2}{16} + ... + \dfrac{w_n\,v_n}{16}$, \hspace{12mm} (1)
\vspace{3mm}

\noindent and the amount of pure gold in the alloy of those pieces of gold is
equal to
\vspace{3mm}

\hspace{15mm} $\dfrac {(w_1 + w_2+ ... + w_n)\,v}{16}$ \hspace{20mm} (2) 
\vspace{3mm}
 
Since (1) and (2) are equal, therefore
\vspace{3mm}

\hspace{15mm} $ \dfrac{(w_1 + w_2+ ... + w_n)\,v}{16} = \dfrac{w_1\,v_1}{16} + \dfrac{w_2\,v_2}{16} + ... + \dfrac{w_n\,v_n}{16}$.
\vspace{3mm}

Hence
\vspace{1mm}

\hspace{15mm} $v = \dfrac{w_1\,v_1 + w_2\,v_2 + ... + w_n\,v_n}{w_1 + w_2 + ... + w_n}$.}
\vspace{4mm}

Ex. 61. Of what \textit{varṇa} do 9, 5, and 17  \textit{māṣas} of
gold of 12, 10 and 11  \textit{varṇas} (respectively) become when
melted together?

\afterpage{\fancyhead[CO] {\small{MIXTURES. ALLIGATION}}}

\newpage

 Ex. 62. Of what  \textit{varṇa} do (the three pieces of gold of)
5 plus $\frac{1}{3}$, 4 plus $\frac{1}{6}$, and 7 plus $\frac{1}{2}$  \textit{māṣas} and 11 plus $\frac{1}{2}$, 10, 8 minus $\frac{1}{2}$  \textit{varṇas} (respectively) become when mixed in one\,?\renewcommand{\thefootnote}{1}\footnote{\hspace{-2mm} \en Similar examples occur in \textit{BM, III}, H 1, 16 verso; H 2, 17 recto and 17 verso; \textit{GSS}, vi. 170-171$\frac{1}{2}$.\\}
\vspace{4mm}

\noindent Rule for finding the  \textit{varṇas} of the refined gold (obtained by mixing up and refining a number of pieces of gold of given weights and \textit{varṇas}), when the weight of the refined gold is known:
\vspace{3mm}

 53(i). The sum of the products of  \textit{varṇa}  and weight of the several pieces of gold, being divided by (the weight of) 
the refined gold, gives the  \textit{varṇa} (of the refined gold).~~~~\renewcommand{\thefootnote}{\hspace{-4.5mm} 2}\footnote{\hspace{-2mm} \en \textit{Cf. MSi}, xv. 39(ii); \textit{GK, I}, p. 76, vs. 18.\\}
\vspace{3mm}

{\small That is, if \textit{n} pieces of gold of weights ${w_1}, {w_2}, ... {w_n}$ and \textit{varṇas} ${v_1}, {v_2}, ..., {v_n}$ respectively are mixed up and refined, then the \textit{varṇa} $v$ of the refined gold is given by
\vspace{3mm}

\hspace{15mm} $v = \dfrac{w_1\,v_1 + w_2\,v_2 + ... + w_n\,v_n}{w}$.
\vspace{3mm}

\noindent where $w$ is the weight of the refined gold.
\vspace{3mm}

 The rationale is similar to that of rule 52(ii).}
\vspace{3mm}

 Ex. 63. 5, 8, and 6  \textit{suvarṇas}  of  \textit{varṇas} 12, 9, and 15 minus
$\frac{1}{2}$ respectively, being mixed up and refined, are seen to be
reduced to 16  \textit{suvarṇas} in all. Quickly say the  \textit{varṇa}  of that
(refined gold).
\vspace{4mm}

\noindent Rule for finding the weight of the refined gold, when the \textit{varṇa}  of the refined gold is known:
\vspace{3mm}

 53(ii). The same (sum of the products of  \textit{varṇa} and
weight of the several pieces of gold), being divided by the
 \textit{varṇa} (of the refined gold), gives the weight of the refined gold.~~~~\renewcommand{\thefootnote}{\hspace{-4.5mm} 3}\footnote{\hspace{-2mm} \en \textit{Cf. BM}, III, H3, 18 recto; \textit{GSS}, vi. $175\frac{1}{2}$, $176\frac{1}{2}$, \textit{MSi}, xv. 40; \textit{L} (ASS), p. 102, vs. 106; \textit{GK, I}, pp. 76-77, vs. 19.}
\vspace{3mm}

{\small That is, if \textit{n} pieces of gold of weights $w_1, w_2, ..., w_n$. and of \textit{varṇas} $v_1, v_2, ..., v_n$ respectively are mixed up and
refined, then the weight
$w$ of the refined gold is given by
\vspace{3mm}

\hspace{15mm} $w = \dfrac{w_1\,v_1 + w_2\,v_2 + ... + w_n\,v_n}{v}$.
\vspace{3mm}

\noindent where $v$ is the  \textit{varṇa}  of the refined gold.}

\newpage

 Ex. 64. (Three pieces of gold of weights) 10, 7, and 5 \textit{māṣas} are seen to be of 9, 8, and 6  \textit{varṇas} (respectively). On
being mixed up and refined, they become of 11  \textit{varṇa}. Say the
weight of the refined gold.
\vspace{4mm}

\noindent Rule for finding the unknown  \textit{varṇa} of one of the pieces of
gold, when the  \textit{varṇa} of the alloy of those pieces of gold is known:
\vspace{3mm}

 54. The sum of the weights of (all) the pieces of gold being multiplied by the  \textit{varṇa} of the alloy (of those pieces
of gold), (then) diminished by the products of weight and \textit{varṇa} (of the pieces of gold of known weight and  \textit{varṇa}), and
(then) divided by the weight of the piece of gold of unknown \textit{varṇa}, gives the (unknown)  \textit{varṇa}.\renewcommand{\thefootnote}{1}\footnote{\hspace{-2mm} \en \textit{Cf. MSi}, xv. 39(i); \textit{GK, I}, p. 76, vs. 18.\\}
\vspace{3mm}

{\small That is, if \textit{n} pieces of gold of weights $w_1, w_2, ..., w_n$ and \textit{varṇas}
$v_1, v_2, ..., v_n$ respectively are mixed up with another piece of gold of weight $w$ and of unknown  \textit{varṇa}, and the \textit{varṇa} of the alloy
is found to be $v'$, then the unknown  \textit{varṇa} $v$ will be given by
\vspace{3mm}

\hspace{10mm} $v = \dfrac{(w_1 + w_2 + ... + w_n + w)\,v' - (w_1\,v_1 + w_2\,v_2 + ... + w_n\,v_n)}{w}$}
\vspace{4mm}

 Ex. 65. 1, 2, and 6  \textit{suvarṇas} of 5, 3, and 4  \textit{kṣayas}
(respectively) being mixed up with a  \textit{pala} of gold of unknown
 \textit{varṇa} become of 12  \textit{varṇa}. (Find the unknown  \textit{varṇa}).~~~~\renewcommand{\thefootnote}{\hspace{-4.5mm} 2}\footnote{\hspace{-2mm} \en Similar examples occur in \textit{BM, III}, H3, 18 recto; \textit{GSS}, vi.
178; 179;
\textit{L} (ASS), p. 102, vs. 107; \textit{GK, I}, p. 77, lines 4-7.}
\vspace{3mm}

{\small Just as the  \textit{varṇa} of gold indicates the fineness of gold, so the
 \textit{kṣaya}
of gold indicates the impurity of gold. Gold of 1  \textit{kṣaya}
contains 1 part of
impurities and 15 parts of pure gold; gold of 2  \textit{kṣayas} contains
2 parts of
impurities and 14 parts of pure gold; and so on. Thus the  \textit{varṇa}
and  \textit{kṣaya}
are related by the formula
\vspace{2mm}

\hspace{25mm} \textit{varṇa} $+$  \textit{kṣaya} $=$ 16.}
\vspace{4mm}

\noindent Rule for finding the unknown weight of one of the pieces of
gold, when the  \textit{varṇa} of the alloy of those pieces of gold is
known:

\newpage

 55. The sum of the weights of the pieces of gold multiplied by the  \textit{varṇa} of the alloy (of those pieces of gold),
(then) diminished by the products of weight and  \textit{varṇa} (of the pieces
of gold of known weights and  \textit{varṇas}), and (then) divided by
the  \textit{varṇa} of the piece of gold of unknown weight minus the
 \textit{varṇa} of the alloy, gives the (unknown weight of) gold.\renewcommand{\thefootnote}{1}\footnote{\hspace{-2mm} \en \textit{Cf. BM, III}, H3, 18 verso; \textit{GSS}, vi. $175\frac{1}{2}$, 180; \textit{MSi}, xv. 41; \textit{L} (ASS), p. 103, vs. 108; \textit{GK, I}, p. 77, vs $20$.\\}
\vspace{3mm}

{\small That is, if a pieces of gold of weights $w_1, w_2, ..., w_n$ and \textit{varṇas} $v_1, v_2, ..., v_n$ respectively are mixed up with a piece of gold of  \textit{varṇa} $v$ but
of unknown weight, and the  \textit{varṇa} of the alloy is found to be $v'$, then
the unknown weight $w$ will be given by

\begin{center} $w = \dfrac{(w_1 + w_2 + ... + w_n)\,v' - (w_1\,v_1 + w_2\,v_2 + ... + w_n\,v_n)}{v - v'}$ \end{center}}

Ex. 66. 2, 3, and 4  \textit{māṣas} of gold of  \textit{kṣayas} 7, 4, and 5 (respectively) being mixed up with some gold of  \textit{kṣaya} 2, the
alloy is found to be of  \textit{kṣaya} 4. Quickly say with how much of that gold (of  \textit{kṣaya} 2 have they been mixed up).~~~~\renewcommand{\thefootnote}{\hspace{-4.5mm} 2}\footnote{\hspace{-2mm} \en \textit{Cf. BM, III}, H3, 18 verso; \textit{GSS}, vi. 181; \textit{L} (ASS), p. 103, vs. 109; \textit{GK, I}, p. 77, lines 16-19.\\}
\vspace{4mm}

\noindent Rule for finding how much of inferior gold (and how much of pure gold) will have to be taken in constructing a number of test sticks of progressively decreasing  \textit{varṇas}:
\vspace{-1mm}

 56. The weight in terms of  \textit{māṣas} (\textit{māṣātmaka māna}) of
one (test) stick being divided by the difference in terms of \textit{yavas} between the  \textit{varṇas} (of pure and inferior gold) and
multiplied one time, 2 times, etc., of the progressive decrease of \textit{varṇa} in terms of  \textit{yavas}, gives the respective amounts of
inferior gold.~~~~\renewcommand{\thefootnote}{\hspace{-4.5mm} 3}\footnote{\hspace{-2mm} \en \textit{Cf. GK, I}, p. 78, lines 18-21, and p. 79, lines 1-2. Also see \textit{GSS}, vi. 192.}
\vspace{3mm}

{\small That is, if \textit{n} test sticks of weight $w$ and \textit{varṇas} $16-k, 16-2k, ..., 16-nk$ are to be made out of pure gold and inferior gold of  \textit{varṇa} $16-K$,
then for the construction of those test sticks we will have to take the
following amounts of inferior gold in the respective order:

\begin{center} $\dfrac{w}{K} \times k, \dfrac{w}{K} \times 2k, ..., \dfrac{w}{K} \times nk$. 
\end{center}} 

\newpage

{\small Subtracting these amounts severally from $w$, we will get the respective amounts of pure gold.
\vspace{3mm}

 The rationale is as follows:
\vspace{3mm}

 Let the required amounts (weights) of gold of  \textit{varṇa} $16-K$ and of \textit{varṇa} 16 be respectively
\vspace{2mm}

\hspace{15mm} ${w_1, w_2, ..., w_n}$
\vspace{2mm}

\hspace{8mm} and~ $(w - w_1), (w- w_2),...(w-w_n)$.
\vspace{2mm}

Then we evidently have
\vspace{2mm}

\hspace{15mm} $(16-K)\,w_r + 16\,(w -w_r) = w\,(16-rk ),$
\vspace{2mm}

\hspace{10mm} where~ $r = 1, 2, ..., n$; whence
\vspace{2mm}

\hspace{15mm} $w_r = \dfrac{w}{K} \times rk$;
\vspace{3mm}

The text does not give the relation between a  \textit{māṣa} and a \textit{yava}, but the commentator takes one  \textit{māṣa} as equivalent to 16 \textit{yavas}.}
\vspace{3mm}

 Ex. 67-68. A series of  \textit{varṇas}, from 0  \textit{kṣaya} to 6 \textit{kṣaya},
uniformly increasing by $\frac{1}{4}$  \textit{kṣaya}, is to be constructed of
sticks of 2  \textit{māṣas} each. O mathematician, if you know then quickly
say after calculating how much gold of  \textit{varṇa} 16 and how
much gold of  \textit{varṇa} 10 will have to be taken (in each case).\renewcommand{\thefootnote}{1}\footnote{\en \textit{Cf. GK, I}, p. 80, lines 2-5.}
\vspace{4mm}

\noindent Rule \,for \,finding \,the \,weights of \,two \,gold \,balls \,of equal \,value \,(i.e., containing equal amounts of pure gold), when the sum of their weights and also the  \textit{varṇas} of the alloys formed by
mixing each of them with given proportions of the other are known:
\vspace{3mm}

 57. The (given)  \textit{varṇas} (severally) divided by the (given)
fractions, each increased by 1, should be mutually diminished
by the given fractions as multiplied by them (i.e., the given \textit{varṇas} severally divided by the given fractions, each increased
by 1). The two (results thus obtained) should be divided by
their (own) sum, then their order should be reversed, and
then they should be multiplied by the sum of the weights of
the gold balls. This will give the weights (of the gold balls)
separately.

 
\newpage

{\small That is, if $w$ be the sum of the weights of the two gold balls, $v_1$ the  \textit{varṇa} of the alloy formed by mixing the first gold ball with
$\frac{a}{b}$ part of the second, and $v_2$ the  \textit{varṇa} of the alloy formed by mixing the
second gold ball with $\frac{c}{d}$ part of the first, then the separate weights $w_1$
and $w_2$ of
the first and second gold balls are given by
\vspace{3mm}

\hspace{15mm} $w_1 = \left(\frac{1}{k}\right) \,\left\lbrace \dfrac{v_2}{1 + \frac{c}{d}} -  \dfrac{\left(\frac{a}{b}\right)\,v_1}{1 + \frac{a}{b}}\right\rbrace \,w$, \hfill (1) \hspace{15mm}
\vspace{3mm}

\hspace{8mm} and~~ $w_2 = \left(\frac{1}{k}\right) \,\left\lbrace \dfrac{v_1}{1 + \frac{a}{b}} - \dfrac{\left(\frac{c}{d}\right)\,v_2}{1 + \frac{c}{d}}\right\rbrace \,w$, \hfill (2) \hspace{15mm}
\vspace{3mm}

\hspace{8mm} where~ $k = \left\lbrace \dfrac{v_2}{1 + \frac{c}{d}} -  \dfrac{\left(\frac{a}{b}\right)\,v_1}{1 + \frac{a}{b}}\right\rbrace + \left\lbrace \dfrac{v_1}{1 + \frac{a}{b}} - \dfrac{\left(\frac{c}{d}\right)\,v_2}{1 + \frac{c}{d}}\right\rbrace$
\vspace{3mm}

The rationale is as follows:
\vspace{3mm}

 Let $V_1$ and $V_2$ be the  \textit{varṇas} of the first and second gold balls respectively. Then we have
\vspace{-1mm}

\hspace{20mm} $w_1 + w_2 = w$, \hfill (3) \hspace{15mm} \vspace{3mm}

\hspace{10mm} $v_1 = \dfrac{w_1V_1 + \left(\frac{a}{b}\right)\,w_2V_2}{w_1 + \left(\frac{a}{b}\right) w_2}$, \hfill (4) \hspace{15mm}
\vspace{3mm}

\hspace{10mm} $v_2 = \dfrac{\left(\frac{c}{d}\right)w_1V_1 + w_2V_2}{\left(\frac{c}{d}\right)w_1 + w_2}$ \hfill (5) \hspace{15mm}
\vspace{3mm}

\noindent and, since the gold balls are of equal value,
\vspace{3mm}

\hspace{20mm} $w_1V_1 = w_2V_2 $. \hfill (6) \hspace{15mm}
\vspace{3mm}

Dividing (4) by (5), and making use of (6), we have
\vspace{3mm}

\hspace{15mm} $\dfrac{v_1}{v_2} = \dfrac{1+ \frac{a}{b}}{1+ \frac{c}{d}} \times \dfrac{\left(\frac{c}{d}\right)w_1 + w_2}{w_1 + \left(\frac{a}{b}\right)w_2}$,
\vspace{3mm}
 
\noindent whence 
\vspace{2mm}

\hspace{15mm} $\dfrac {w_1}{{\dfrac{v_2}{1+\frac{c}{d}} - \dfrac{\left(\frac{a}{b}\right)v_1}{1 + \frac{a}{b}}}} = \dfrac {w_2}{{\dfrac{v_1}{1 + \frac{a}{b}} - \dfrac{\left(\frac{c}{d}\right)v_2}{1+\frac{c}{d}}}}$
\vspace{3mm}

But by virtue of (3), each of these is equal to $\frac{w}{k}$. Hence we
have (1) and (2).}
\vspace{4mm}

\noindent Rule for finding the  \textit{varṇas} of two gold balls of equal value,
when the sum of their weights and also the  \textit{varṇas} of the
alloys formed by mixing each of them with given proportion of the other are known:
\vspace{3mm}

 58. The weights of the two (gold) balls (obtained by applying the previous rule) should each be increased by the
given fraction of the weight of the other.

\newpage

\noindent The two (results thus obtained) being divided by the (same) fractions, each
increased by 1, then multiplied by the (respective)  \textit{varṇas},
and then divided by the (respective) weights of the gold
balls, give the  \textit{varṇas} (of the two gold balls).\renewcommand{\thefootnote}{1}\footnote{\hspace{-2mm} \en See also \textit{GSS}, vi 209-212; and \textit{GK, I}, p. 81, lines 9-18.\\}
\vspace{3mm}

{\small That is, if  $w_1$ and  $w_2$ be the weights of two gold balls whose \textit{varṇas} are $V_1$ and $V_2$ respectively, $ v_1$ the  \textit{varṇa} of the alloy formed by mixing the first gold ball with $\frac{a}{b}$ part of the second, and, $v_2$ the \textit{varṇa} of the alloy formed by mixing the second gold ball with $\frac{c}{d}$ part of the first, then
\vspace{3mm}

\hspace{15mm} $V_1 = \dfrac{w_1 + \left(\frac{a}{b}\right)w_2}{\left(1 + \frac{a}{b}\right)w_1}\,.\,v_1$,
\vspace{2mm}

\hspace{15mm} $V_2 = \dfrac{w_2 + \left(\frac{c}{d}\right)w_1}{\left(1 + \frac{c}{d}\right)w_2}\,.\,v_2$.
\vspace{3mm}

Of these results, the first follows from equations (4) and (6) of the previous rule, and the second from equations (5) and (6).}
\vspace{3mm}

 Ex. 69. There are two small balls of gold of equal
worth (i.e., having equal quantities of pure gold), whose combined weight is 5  \textit{māṣas}. When they are respectively combined with $\frac{2}{3}$ part and $\frac{1}{2}$ part of the other, they become of 10 and 9 \textit{varṇas} respectively. (Find their weights and  \textit{varṇas} separately).
\vspace{3mm}

 Ex. 70. O learned one, there are two small balls of
gold of equal value. When taken together, their weight is 12 \textit{māṣas}; and when they are respectively combined with $\frac{5}{7}$ part
and $\frac{1}{5}$ part of the other, they become of 12 and $10\frac{1}{2}$  \textit{varṇas}
respectively. (Find their weights and  \textit{varṇas} separately).
\vspace{3mm}

\begin{center} \englishfont{(ii) \emph{PARTNERSHIP}}\end{center}

\noindent Rule for finding the shares of the partners in the produce:
\vspace{3mm}

 59(i). To obtain the individual shares (of the partners)
in the produce (\textit{phalāvāptyai}), the seeds (contributed by the
partners) (\textit{prakṣepa}), as divided by their sum, should be severally
multiplied by the produce (\textit{phala}).~~~~\renewcommand{\thefootnote}{\hspace{-4.5mm} 2}\footnote{\hspace{-2mm} \en \textit{Cf. BrSpSi}, xii. 16(i); \textit{SiŚe}, xiii. 19(i); \textit{L}
(ASS), p. 90, vs. 94; \textit{GK, I}, p. 54, lines 11-12. Also see \textit{GSS}, vi. 79$\frac{1}{2}$; \textit{MSi}, xv. 36.}

\afterpage{\fancyhead[CO] {\small{MIXTURES. PURCHASE AND SALE}}}

\newpage

{\small The  \textit{prakṣepa}, according to the commentator, is "that which is
thrown, scattered, or sown as seed" and therefore "seed", and the  \textit{phala} is
"what is produced out of that".}
\vspace{3mm}

 Ex. 71. Two, three, five, and four  \textit{prasthas} of seeds (are
the contributions of the partners) and 210 (\textit{prasthas} of grain)
is the produce; what are the shares of the partners)
separately?
\vspace{3mm}

 Ex. 72. $\frac{1}{2}$  \textit{prastha} is (the contribution) of one, $\frac{1}{3}$
(\textit{prastha}) of another, $\frac{1}{9}$ (\textit{prastha}) of still another, and $1700$ (\textit{prasthas}) is the produce. Say what are their shares in the produce) separately.
\vspace{3mm}

\begin{center} \englishfont{(iv) {\emph{PURCHASE AND SALE}}}
\end{center}

\noindent Rule for obtaining the quantities and prices of each of a
number of commodities separately, when they are purchased
in a specified proportion for a given amount of money:
\vspace{3mm}

 59(ii). Having divided the (given) rate-prices (of the commodities) by the (corresponding) rate-quantities (of those
commodities) and (then) having multiplied them by the \,corresponding \,proportions \,(in which \,those \,commodities \,are
purchased), apply the previous rule.\renewcommand{\thefootnote}{1}\footnote{\hspace{-2mm} \en \textit{Cf. SiŚe}, xiii. 19(ii); \textit{GK, I}, p. 57, vs. 2. Also
see \textit{GSS}, vi. $87\frac{1}{2}$-$89\frac{1}{2}$.\\}
\vspace{3mm}

Āryabhaṭa II states the rule more explicitly. He writes:~~~~\renewcommand{\thefootnote}{\hspace{-4.5mm} 2}\footnote{\hspace{-2mm} \en \textit{MSi}, xi. 37-38(i). Also see \textit{L} (ASS), p. 92, vs. 98.}
\vspace{3mm}

"Set down the rate-prices as multiplied by the respective proportional parts and divided by the rate-quantities of the respective
commodities in two places. Multiply the results set down in one place as well
as the proportional parts by the mixed price and divide by the sum of the
results set down in the other place. These give the prices (paid for respective
commodities) and the quantities (of the respective commodities that
are obtained) respectively."
\vspace{3mm}

 Ex. 73-74. 7  \textit{kuḍavas} of  \textit{mudga} ("seeds of  \textit{Phaseolus mungo}") are obtained for 9  \textit{paṇas}, and $\frac{1}{2}$  \textit{kuḍava} of rice is
obtained for one  \textit{paṇa}.

\newpage

 Then, O merchant, take 3  \textit{paṇas} and a half and quickly
give me one part of rice and two parts of  \textit{mudga}.
\vspace{3mm}

{\small \textit{Solution}. ~Dividing the rate-prices of  \textit{mudga} 
and rice by the
corresponding rate-quantities of those commodities, we get
\vspace{3mm}

\hspace{15mm} $\dfrac{9}{7}$  and $\dfrac{2}{1}$. 
\vspace{3mm}

Multiplying these by the corresponding proportions in which they
are purchased (i.e., by 2 and I), we get
\vspace{3mm}

\hspace{15mm} $\dfrac{18}{7}$  and ${2}$
\vspace{3mm}

Now we apply the previous rule. Thus dividing $\frac{18}{7}$ and 2 by their
sum, and multiplying them by the total price (i.e., 3$\frac{1}{2}$ \textit{paṇas}),
we get
\vspace{3mm}

\hspace{15mm} $\dfrac{63}{32}$ and $\dfrac{49}{32}$  \textit{paṇas} 
\vspace{3mm}

These are the prices of  \textit{mudga} and rice respectively.
\vspace{3mm}

 The quantities of \textit{mudga} and rice obtained for $\frac{63}{32}$  \textit{paṇas} and
$\frac{49}{32}$  \textit{paṇas}
respectively can now be obtained by the rule of three. These come out
to be $\frac{49}{32}$  \textit{kuḍavas} and  $\frac{49}{64}$   \textit{kuḍavas} respectively.}
\vspace{3mm}

 Ex. 75. $\frac{1}{2}$  \textit{pala} of asafoetida (\textit{hiṅgu}), 2  \textit{palas} of long
pepper (\textit{pippalī}), and 7  \textit{palas} of dry ginger (\textit{śuṇṭhī}), are each
obtained for one  \textit{rūpa}. Give me equal quantities (of each
of them) for one  \textit{rūpa}.\renewcommand{\thefootnote}{1}\footnote{\en  \textit{Cf. GK, I}, p. 57, lines 16-19.}
\vspace{4mm}

\noindent Rule for finding the rates of purchase and sale when merchants investing unequal capitals become equally rich by
purchasing and selling articles at the same rates:
\vspace{3mm}

 60. Reduce the capitals (\textit{vitta}) to a common denominator, and then remove the denominators. \,This \,being \,done, \,the \,greatest \,amount \,increased \,by \,an \,optional number (1, 2, 3,
etc.) gives the rate of sale (i.e., the rate at which the
articles are sold for one \textit{rūpa; vikraya}).
\vspace{3mm}

 That (i.e., the rate of sale) multiplied by the price at
which each remnant article is sold (\textit{antyārgha}), being diminished by 1

\newpage

\noindent and then multiplied by the common denominator
(of the capitals), gives the rate of purchase (i.e., the rate at
which the articles are purchased for one  \textit{rūpa; kraya}).\renewcommand{\thefootnote}{1}\footnote{\en Also see \textit{GSS}, vi. $102\frac{1}{2}$; \textit{GK, I}, pp. 94-97, vv. 36
(ii)-38(i).}
\vspace{3mm}

{\small That is, if \textit{n} persons, whose capitals reduced to a common denominator are
\vspace{3mm}

\hspace{20mm} $\dfrac{C1}{D}$, $\dfrac{C2}{D}$, $\dfrac{C3}{D}$, ...., $\dfrac{Cn}{D}$   \textit{rūpas} respectively,
\vspace{3mm}

\noindent invest their capitals in purchasing certain articles at the rate of $x$
articles per \textit{rūpa}, and if by selling those articles in multiples of $y$ at
the rate of $y$
articles per  \textit{rūpa}, and the remnant articles, at the rate of $R$ \textit{rūpas} per
article, they become equally rich, then
\vspace{3mm}

\hspace{20mm} $x = D(Ry - 1)$,
\vspace{3mm}

\hspace{12mm} and~~ $y = C_{\lambda} + k$,
\vspace{3mm}

\noindent where $C_{\lambda}$ is the greatest of all \textit{C}'s, and \textit{k} an arbitrary integer.
\vspace{3mm}

 The rationale of this rule is as follows:
\vspace{3mm}

 Suppose that by selling the articles in multiples of \textit{y}, the \textit{n}
persons earn $P_1, P_2,$ $P_3, ..., P_n$  \textit{rūpas} respectively. Then the
total amounts
earned by the \textit{n} persons by selling all the articles (including the
remnant
articles) are respectively
\vspace{3mm}

\hspace{10mm} $P_1 + R\left(\dfrac{C_1}{D}\,x - P_1\,y\right), \quad P_2 + R\left(\dfrac{C_2}{D}\,x - P_2\,y\right),\, ...,$ 
\vspace{2mm}

\hspace{25mm} and~~ $P_n + R\left(\dfrac{C_n}{D}\,x - P_n\,y\right)$ \;\textit{rūpas},
\vspace{3mm}

\hspace{5mm} or~~ $\dfrac{R{C_1}x}{D} - P_1(Ry -1), \quad \dfrac{R{C_2}x}{D} - P_2(Ry -1)$,\, ...,
\vspace{2mm}

\hspace{25mm} and~~ $\dfrac{R{C_n}x}{D} - P_n(Ry -1)$ \;\textit{rūpas}.
\vspace{3mm}

Since these amounts are equal to one another, therefore we have 
\vspace{3mm}

\hspace{10mm} $\dfrac{R{C_1}x}{D} - P_1(Ry -1) = \dfrac{R{C_2}x}{D} - P_2(Ry -1) =$\, ...
\vspace{2mm}

\hspace{25mm} $= \dfrac{R{C_n}x}{D} - P_n(Ry -1)$
\vspace{3mm}

\hspace{5mm} or~~ $R{C_1}x - P_1D(Ry -1) = R{C_2}x - P_2D(Ry -1) =$\, ...
\vspace{2mm}

\hspace{25mm} $= R{C_n}x - P_nD(Ry -1) = M$ say. \hfill (1) \hspace{5mm}
\vspace{3mm}

To solve (1), ~let $x = M = D\,(Ry - 1)$;}

\newpage

\noindent {\small then
\vspace{1mm}

\hspace{10mm} $R{C_r} - {P_r} = 1$, \hfill (2) \hspace{10mm} 
\vspace{1mm}

\noindent for\; r = 1, 2, ..., n.
\vspace{3mm}

But \qquad $\dfrac{(\frac{{C_r}}{D})x}{y} = {P_r} + \dfrac{A_r}{y}$,
\vspace{3mm}

\noindent where $A_r$ is the remainder when $(\frac{{C_r}}{D})x$ is divided by $y$, so
that, when $x = D\,(Ry-1)$,
\vspace{3mm}

\hspace{10mm} ${P_r} = \dfrac{{C_r}\,x}{Dy} - \dfrac{A_r}{y}$ 
\vspace{3mm}

\hspace{14mm} $= \dfrac{{C_r} (Ry -1 )}{y} - \dfrac{A_r}{y}$
\vspace{3mm}

\hspace{14mm} $= R{C_r} - \dfrac{{C_r} + {A_r}}{y}$ \hfill (3) \hspace{10mm}
\vspace{3mm}

Therefore, we must have
\vspace{3mm}

\hspace{12mm} $y\,\textgreater\,C_r$, for all $r$.
\vspace{1mm}

\hspace{5mm} or \quad $y\,\textgreater\,C_{\lambda}$
\vspace{3mm}

\noindent where $C_{\lambda}$ is the greatest of all $C$'s.
\vspace{3mm}

 Hence a solution of (1) is
\vspace{2mm}

\hspace{10mm} $x = D(Ry - 1)$,
\vspace{1mm}

\hspace{10mm} $y = C_{\lambda} + k$,
\vspace{3mm}

\noindent \textit{k} being an arbitrary integer.}
\vspace{4mm}

\noindent Sub-rule for finding another rate of purchase, when the capitals
are integral and the articles which are left after selling at
the general rate as also the amounts of money acquired by
selling the articles at the general rate have a common factor:
\vspace{3mm}

 61. The articles which are severally left after selling
(at the general rate) (i.e., the remnant articles) as also the
amounts of money which are severally acquired by selling the
articles (at the general rate), being abraded by their common
factor (lit. divisor), the resulting equal amounts of money
(i.e., the equal sale proceeds) give another rate of purchase.
\vspace{3mm}

{\small This rule relates to the case when $D = 1$, and may be expressed
analytically as follows:
\vspace{3mm}

 If $d$ be the common factor of
\vspace{2mm}

\hspace{15mm} ${P_1}, {P_2}, ..., {P_n}$ 
\vspace{2mm}

\noindent and \quad $({C_1}x-{P_1}y), ({C_2}x-{P_2}y), ..... ({C_n}x-{P_n}y)$,}

\newpage

\noindent {\small then another rate of purchase is $x'$ articles per  \textit{rūpa}, where
\vspace{2mm}

\hspace{15mm} $x' = \dfrac{P_1}{d} + \dfrac{R({C_1}x - {P_1}y)} {d}$
\vspace{2mm}

\hspace{19mm} $= \dfrac{P_2}{d} + \dfrac{R({C_2}x - {P_2}y)} {d}$  
\vspace{1mm}

\hspace{25mm} ... ... ... ... ...
\vspace{1mm}

\hspace{19mm} $= \dfrac{P_n}{d} + \dfrac{R({C_n}x - {P_n}y)} {d}$.
\vspace{2mm}

\noindent where $x= {R_y} - 1, y = C_{\lambda} + k, C_{\lambda}$ being the greatest of all \textit{C}'s
and \textit{k} an arbitrary integer; the rate of sale being the same as before
(i.e., $C_{\lambda} + k$ articles per \textit{rūpa}).
\vspace{3mm}

 The rationale of this rule is as follows:
\vspace{2mm}

 When $D = 1$, the equations (1) of the previous rule may be
written as
\vspace{2mm}

\hspace{15mm} $R{C_1}x - {P_1} ({Ry}-1) = R{C_2}x -{P_2} ({Ry}-1) =$\, ...
\vspace{2mm}

\hspace{35mm} $= R{C_n}x-{P_n} ({Ry}-1) = M$. \hfill (1) \hspace{10mm}
\vspace{2mm}

Dividing throughout by \textit{d}, and writing $x'$ for $\frac{x}{d}$, ${P_r}'$ for $\frac{P_r}{d}$,
and $M'$ for $\frac{M}{d}$, we have
\vspace{2mm}

\hspace{15mm} $R{C_1}x' - {P_1}'\,({Ry} -1) = R{C_2}x' - {P_2}' (Ry - 1) =$
\vspace{2mm}

\hspace{35mm} $= R{C_2}x' -{P_n}'\,(Ry - 1) = M'$ \hfill (2) \hspace{10mm}
\vspace{2mm}

The values of $x, y$ satisfying (1) give one set of rates of purchase
and sale, and the values of $x', y$ satisfying (2) give another set of
rates
of purchase and sale.
\vspace{2mm}

 When $M = Ry-1$, a solution of (1) is, as before,
\vspace{2mm}

\renewcommand*{\arraystretch}{1}
\hspace{20mm} \begin{tabular}{l}
$x = Ry - 1$\\ $y = C_{\lambda}  + k$ \end{tabular} \bigg\} \hfill (3) \hspace{10mm}
\vspace{2mm}

\noindent and the corresponding solution of (2) is
\vspace{2mm}

\renewcommand*{\arraystretch}{1.2}
\hspace{20mm} \begin{tabular}{l}
$x' = \dfrac{(Ry - 1)}{d}$, i.e, $M'$\\
$y = C_{\lambda}  + k$ \end{tabular} \Bigg\} \hfill (4) \hspace{10mm}
\vspace{2mm}

\renewcommand*{\arraystretch}{0.7}
The rates of purchase and sale given by (3) are those already given
in Rule 60; those given by (4) are those mentioned in the present
rule.
\vspace{2mm}

Since, from (2),
\vspace{1mm}

\hspace{15mm} $M' = \dfrac{P_1}{d} + \dfrac{R({C_1}x -{P_1}y)}{d} = \dfrac{P_2}{d} + \dfrac{R({C_2}x - {P_2}y)}{d}$
\vspace{2mm}

\hspace{40mm} $=\, ...\, = \dfrac{P_n}{d} + \dfrac{R({C_n}x -{P_n}y)}{d}$
\vspace{2mm}

\noindent it is clear that the value of $x'$ given by (4) is the same as that
stated
in the above rule. It can be easily seen that $M'$ is an integer.
\vspace{2mm}

 \textit{Note.} ~When $D \neq 1$, then another rate of purchase will be $\frac{D (Ry-1)}{d}$
articles per  \textit{rūpa}, where $y = C_{\lambda} + k$.}

\newpage

 Ex. 76. The capitals of (three) men are 1, 3, and 5
(\textit{rūpas}) or $\frac{1}{3},\frac{1}{4}$ and $\frac{1}{2}$ (\textit{rūpas}) (respectively). By purchasing and
selling (certain articles) at the same rates and by selling the
remnant articles at the rate of 1 for 3 (\textit{rūpas}), they become
possessed of equal riches. (Find the rates of purchase and
sale).\renewcommand{\thefootnote}{1}\footnote{\hspace{-2mm} \en \textit{Cf. GSS}, vi. $103\frac{1}{2}$, $104\frac{1}{2}$; \textit{GK, I}, p. 96, lines
2-6, and p. 97, lines 7-9.\\}
\vspace{4mm}

\noindent Sub-rule for finding the rates of purchase and sale when each
of the remnant articles is sold for a fractional price:
\vspace{3mm}

 62. When the remnant-price (i.e., the price at which
each remnant article is sold) is fractional (lit. has a denominator), reduce it along with the capitals to a common
denominator (and remove the denominators as before), and
multiply the rates of purchase and sale resulting from them
(by applying Rule 60) by the common denominator.~~~~\renewcommand{\thefootnote}{\hspace{-4.5mm} 2}\footnote{\hspace{-2mm} \en \textit{Cf. GSS}, vi. $107\frac{1}{2}$. For another rule, see \textit{GSS},
vi. $109\frac{1}{2}$; \textit{GK, I}, p. 98,
lines 2-7.}
\vspace{3mm}

{\small That is, if the capitals of the n persons and the remnant-price
reduced to a common denominator, be as follows
\vspace{3mm}

\hspace{10mm} Capitals: \qquad $\dfrac{C_1}{D}$, $\dfrac{C_2}{D}$, $\dfrac{C_3}{D}$, ...., $\dfrac{C_n}{D}$~ \textit{rūpas}
\vspace{3mm}

\hspace{10mm} Remnant-price: ~$\dfrac{R}{D}$ ~\textit{rūpas},
\vspace{3mm} 

then \hspace{7mm} $x = {D^2}\left(\dfrac{R}{D}\,y - 1\right)$, 
\vspace{2mm}

\hspace{5mm} and \quad $y = D\,(C_{\lambda} + k)$,
\vspace{2mm}

\noindent $C_{\lambda}$ being the greatest of all \textit{C}'s and \textit{k} an arbitrary integer.
\vspace{3mm}

 The rationale of this rule is as follows:
\vspace{3mm}

 As before, equating the wealths of the \textit{n} persons after all the
articles
have been sold away, we get [see equations (1) of Rule 60 (notes)]
\vspace{3mm}

\hspace{10mm} $\dfrac{R}{D}\,.\,{C_1}x - {P_1}D\left(\dfrac{R}{D} y - 1\right) = \dfrac{R}{D}\,.\,{C_2}x - {P_2}D\left(\dfrac{R}{D} y - 1\right)$
\vspace{2mm}

\hspace{30mm} $=\, ...\, = \dfrac{R}{D}\,.\,{C_n}x - {P_n}D\left(\dfrac{R}{D} y - 1\right)$,}

\newpage

{\small or \hspace{10mm} $R{C_1}x - {P_1}{D^2}\,\left(\dfrac{R}{D}\,y - 1\right) = {R}{C_2}x - {P_2}{D^2} \left(\dfrac{R}{D} y - 1\right)$
\vspace{2mm}

\hspace{25mm} $= ... ={R}{C_n}x - {P_n}{D^2} \left(\dfrac{R}{D} y - 1\right) = N$, say. \hfill (1) \hspace{10mm}
\vspace{3mm}

Let \qquad $x = N = {D^2} \left(\dfrac{R}{D} y - 1\right)$, then
\vspace{2mm}

\hspace{30mm} $R{C_r} - {P_r} = 1$, \hfill (2) \hspace{10mm}
\vspace{2mm}

\noindent for~ $r = 1, 2, ..., n $
\vspace{3mm}

But \hspace{5mm} $\dfrac{\left(\frac{C_r}{D}\right)x}{y} = P_r + \dfrac{A_r}{y}$,
\vspace{2mm}

\noindent where $A_r$ is the remainder when $\dfrac{C_r x}{D}$ is divided by $y$, so that,
when ~$x = {D^2} \left(\dfrac{R}{D}\,y - 1\right)$,
\vspace{2mm}

\hspace{20mm} ${P_r} = \dfrac{{C_r}x}{Dy} -\dfrac{A}{y}$
\vspace{3mm}

\hspace{25mm} $= \dfrac{{C_r}{D^2} \left(\dfrac{R}{D} y - 1\right)}{Dy} - \dfrac{A_r}{y}$
\vspace{3mm}

\hspace{25mm} $= R{C_r} - \dfrac{{C_r} D + {A_r}}{y}$ \hfill (3) \hspace{10mm}
\vspace{2mm}

Therefore we must have
\vspace{2mm}

\hspace{20mm} $y \textgreater {C_r}D$, for all $r$,
\vspace{1mm}

\hspace{15mm} or~~ $y \textgreater C_{\lambda}D$,
\vspace{2mm}

\noindent where $C_{\lambda}$ is the greatest of all \textit{C}'s.
\vspace{3mm}

 Hence a solution of (1) is
\vspace{2mm}

\hspace{20mm} $x = {D^2} \left(\dfrac{R}{D}y - 1\right)$,
\vspace{2mm}

\hspace{20mm} $y = D (C_{\lambda} + k)$,
\vspace{2mm}

\noindent \textit{k} being an arbitrary integer.}
\vspace{4mm}

 Ex. 77. The capitals of four men are $1\frac{1}{2}$, 2, 3, and 5
\textit{rūpas}. By purchasing and selling (certain articles) at the same
rates, and by selling the remnant articles at the rate of 1 for $\frac{1}{2}$
(of a  \textit{rūpa}), they become possessed of equal riches. (Find the
rates of purchase and sale).\renewcommand{\thefootnote} {1}\footnote{\en  \textit{Cf. GSS}, vi. 108$\frac{1}{2}$, 110$\frac{1}{2}$; \textit{GK, I}, p. 99, lines 2-5
and 15-18.}

\newpage

\noindent Rule for finding out how a specified number of creatures of
given rates can be bought for a specified price:
\vspace{3mm}

 63-64. By the price of one creature of any variety
multiply the rate-creatures (of other varieties) in the order in
which they have been stated (in the problem) (and also the
number of creatures to be bought). From the products (corresponding to the rate-creatures) severally subtract the respective rate-prices of the creatures (and from the product
corresponding to the number of creatures to be bought subtract the specified price). Now multiply the various remainders, excepting that obtained by subtracting the specified
price, by optional numbers (multipliers) which are to be
chosen in such a way that (i) the resulting products when
added together may yield (the remainder obtained by subtracting) the specified price as sum, and (ii) on taking the
products of those multipliers and the respective rate-prices,
a negative number or zero may not be obtained for the multiplier of the creature which is without a multiplier. (The multipliers for the various creatures, obtained in this way, when
multiplied by the respective rate-creatures, will give the
number of creatures of the different varieties that will be
bought for the specified price; and the same multipliers
when multiplied by the rate-prices of the respective creatures
will give the prices that will be paid for the creatures of the
respective varieties).\renewcommand{\thefootnote}{1}\footnote{\hspace{-2mm} \en \textit{Cf. GK, I}, p. 92, vv. 34(ii)-35. For other rules see
\textit{GSS}, vi. 146$\frac{1}{2}$, 151.\\}
\vspace{3mm}

{\small For the explanation of this rule see under Ex, 78-79.}
\vspace{3mm}

 Ex. 78-79. Pigeons are sold at the rate of 5 for 3 (\textit{rūpas}),
cranes at the rate of 7 for 5 (\textit{rūpas}), swans at the rate of 9 for
7 (\textit{rūpas}), and peacocks at the rate of 3 for 9 (\textit{rūpas}).
Knowing
the rates as stated above, bring 100 birds for 100  \textit{rūpas}
for the amusement of the prince.~~~~\renewcommand{\thefootnote}{\hspace{-4.5mm} 2}\footnote{\hspace{-2mm} \en The same example occurs in \textit{GSS}, vi. 152-153; \textit{BBi} (ASS), p. 163,
vs. 138; \textit{GK, I}, p. 93, lines 2-5. A similar problem occurs in
\textit{BM}, III, E3,
58 verso, which may be stated as: 'A man earns 3  \textit{maṇḍas} in a day,
a
woman $1\frac{1}{2}$  \textit{maṇḍas} in a day, and a  \textit{sūḍha} $\frac{1}{2}$  \textit{maṇḍa} in a
day. If 20 of
them earn 20  \textit{maṇḍas} in a day, how many of each category are there?}

\newpage

{\small Let the number of pigeons, cranes, swans, and peacock bought for
100  \textit{rūpas}, and the price paid for the respective birds be as follows
\vspace{2mm}

\renewcommand*{\arraystretch}{1}
\hspace{5mm} \begin{tabular}{l}\\  Number \\ Price \end{tabular}
\begin{tabular}{c}Pigeons \\$5x$ \\$3x$ \end{tabular}
\begin{tabular}{c}Cranes \\$7y$ \\ $5y$  \end{tabular}
\begin{tabular}{c}Swans \\$9z$ \\$7z$ \end{tabular}
\begin{tabular}{c}Peacocks \\$3u$ \\$9u$ \end{tabular}
\hfill $\Bigg\}$ \hspace{5mm} (1) \hspace{5mm} 
\vspace{3mm}

\noindent Then we have
\vspace{1mm}

\hspace{3cm} $5x + 7y + 9z + 3u =$ 100,  \hfill(2) \hspace{5mm} 
\vspace{1mm}

\hspace{3cm} $3x + 5y + 7z + 9u =$ 100, \hfill  (3) \hspace{5mm}  
\vspace{3mm}

 Multiplying (2) by 3 (which is the price of one peacock) and 
subtracting (3) therefrom, we have
\vspace{1mm}

\hspace{3cm}  $12x + 16y + 20z =$ 200. \hfill (4) \hspace{5mm}
\vspace{3mm}

This is an indeterminate equation having an indefinitely large
number of solutions. But we are in the present case concerned only with
those positive and non-zero solutions of (4), which make
\vspace{2mm}

\hspace{1.5cm}  $5x, 7y, 9z,$ and $3u$
\vspace{2mm}

\noindent positive integers.
\vspace{3mm}

Proceeding by trial, we obtain the following sixteen such solutions 
of (4):

\begin{center}
\renewcommand*{\arraystretch}{1.2}
\begin{tabular}{ll}
 (1)~ $x = 3, ~~y = 4, ~z = 5$. & ~\,(9)~ $x = 4, ~~y = 7, ~z = 2.$ \\

 (2)~ $x = 11, y = 3, ~z = 1.$ & $(10)~ x = 8, ~~y = 4, ~z = 2.$ \\

 (3)~ $x = 1, ~~y = 8, ~z = 3.$ & $(11)~ x = 12, y = 1, ~z = 2.$ \\

 (4)~ $x = 6, ~~y = 3, ~z = 4.$ & $(12)~ x = 5, ~~y = 5, ~z = 3.$ \\

 (5)~ $x = 2, ~~y = 6, ~z = 4.$ & $(13)~ x = 9, ~~y = 2, ~z = 3.$ \\

 (6)~ $x = 4, ~~y = 2, ~z = 6.$ & $(14)~ x = 7, ~~y = 1, ~z = 5.$ \\

 (7)~ $x = 3, ~~y = 9, ~z = 1.$ & $(15)~ x = 1, ~~y = 3, ~z = 7.$ \\

 (8)~ $x = 7, ~~y = 6, ~z = 1.$ & $(16)~ x = 2, ~~y = 1, ~z = 8.$ \\
 \end{tabular}
 \end{center}

\noindent The corresponding values of $u$ are obtained from (2).
\vspace{3mm}

\renewcommand*{\arraystretch}{0.7}
 Substituting these values of $x, y, z,$ and $u$ in (1), we obtained sixteen
valid solutions of the problem. (For actual solutions, see the answer).
\vspace{3mm}

 The above analysis explains the basis of the foregoing rule (Rule
63-64). It may be observed that the multipliers mentioned in the rule
refer to the variables $x, y, z,$ and $u$.}

\newpage

 Ex. 80. (The rates of sale) of pomegranates, mangoes,
and wood-apples are resp-ectively 1 for 2 (\textit{rūpas}), 5 for 3 (\textit{rūpas}),
and 2 fruits for 1 (rūpa). Bring 100 (fruits) for 80 (\textit{rūpas}).\renewcommand{\thefootnote}{1}\footnote{\hspace{-2mm} \en \textit{Cf. GSS}, vi. 147$\frac{1}{2}$-149, 150.\\}
\vspace{3mm}

\begin{center} \englishfont{(v) \emph{MEETING OF TWO TRAVELLERS}}\end{center}

\noindent Rule for finding the time in which the fast traveller, who starts
travelling on the same track after the slow traveller has
already covered a specified distance, would overtake the slow
traveller:
\vspace{3mm}

 65. By the difference between the speeds per day of the
fast and slow travellers divide the distance already travelled
by the slow traveller. This gives the time (in days in which
the fast traveller would overtake the slow one).~~~~\renewcommand{\thefootnote}{\hspace{-4.5mm} 2}\footnote{\hspace{-2mm} \en \textit{Cf. BM, III}, A 13, 3 recto; \textit{GSS}, vi. $326\frac{1}{2}$.\\}
\vspace{3mm}

 Ex. 81-82. When a person, travelling (at the speed of)
8  \textit{yojanas} per 5 minus $\frac{1}{2}$ days, has already travelled for 6 minus
$\frac{1}{4}$ days, another person, who travels (at the speed of) 3  \textit{yojanas}
a day, starts travelling (from the same place) along the same
track. Say, after calculating, when the latter traveller would
overtake the former.~~~~\renewcommand{\thefootnote}{\hspace{-4.5mm} 3}\footnote{\hspace{-2mm} \en \textit{Cf. BM, III} B4, 4 recto.; E 2, 53 recto; E 2, 53 verso;
\textit{GSS}, vi. $327\frac{1}{2}$.\\}
\vspace{4mm}

\noindent Rule for finding the time when two travellers, one fast and
the other slow, who, starting simultaneously from the
same place, are destined to go to a specified distance
and then to come back by the same track, will meet each other
on the way, one going ahead and the other coming back:
\vspace{3mm}

 66(i). The (length of the) track divided by half the
sum of the speeds per day (of the two travellers) gives the
time (in days) at meeting.~~~~\renewcommand{\thefootnote}{\hspace{-4.5mm} 4}\footnote{\hspace{-2mm} \en \textit{Cf. GSS,} vi., 319(ii).}
\vspace{3mm}

{\small The time at meeting is evidently measured since the start of travel.}
\vspace{3mm}

 Ex. 83. One man travels (at the speed of) 8  \textit{yojanas}
(a day) and another (at the speed of) 2  \textit{yojanas} (a day).
(They start simultaneously from the same place, and after
reaching the destination come back by the same track).

\afterpage{\fancyhead[CO] {\small{MIXTURES. WAGES AND PAYMENTS}}}

\newpage

\noindent The (length of the) track is 100   \textit{yojanas}. Say where is the
meeting (of the two), one going ahead and the other coming 
back.\renewcommand{\thefootnote}{1}\footnote{\hspace{-2mm} \en \textit{Cf. BM, III}, 9 verso; \textit{GSS}, vi. 321-321$\frac{1}{2}$.\\}
\vspace{4mm}

\noindent Sub-rule for finding the length of the track:
\vspace{3mm}

 66(ii). The time of the meeting multiplied by half the sum 
of the speeds per day (of the two travellers) gives the length of 
the track.
\vspace{4mm}

\noindent Sub-rule for finding the speed of one of the two travellers:
\vspace{3mm}

 67(i). The (length of the) track as divided by the time
of meeting, being doubled and then diminished by the speed
 (of one of the two travellers) gives the other speed (i.e., the
speed of the other traveller).
\vspace{3mm}

\begin{center} \englishfont{(vi) \emph{WAGES AND PAYMENTS}}\end{center}

\noindent Rule for finding the wages for carrying a bottle of oil when 
some oil falls down on the way due to a hole in the bottle:
\vspace{3mm}

 67(ii). Give half the wages for the quantity fallen down
(on the way) and full for the rest.~~~~\renewcommand{\thefootnote}{\hspace{-4.5mm} 2}\footnote{\hspace{-2mm} \en \textit{Cf. GK, I}, p. 102, lines 6-7.\\}
\vspace{3mm}

{\small It can be easily seen that the wages prescribed in this rule correspond to the average load carried by the porter.}
\vspace{3mm}

 Ex. 84-85. While a leathern oil-bottle (\textit{kutapa}), filled
with 200   \textit{palas} of oil, was being carried (by a porter) to a
distance of 8 \textit{yojanas} for 5  \textit{paṇas} (as wages), a hole happened to occur in the bottom of it through which the oil 
leaked out (on the way) continuously. If 20  \textit{palas} of oil be left
(in the bottle), what wages should be paid (to the porter)?~~~~\renewcommand{\thefootnote}{\hspace{-4.5mm} 3}\footnote{\hspace{-2mm} \en Read  \textit{bhṛte} for  \textit{dhṛte} in the text. For a similar example,
see \textit{GK, I}, 
p. 103, lines 2-5.}
\vspace{4mm}

\noindent Rule for finding the amounts to be paid to a dancing party
by the spectators, who see the dance for different parts of the 
day, the amount to be paid to the dancing party for the whole
day being given:
\vspace{3mm}

 68. The succeeding parts of the day, each diminished
by the preceding one, should be multiplied by the 'fruit' (\textit{phala}); 

\newpage

\noindent then having divided them severally by the respective numbers of spectators, add the preceding results to
the succeeding ones. The results thus obtained being multiplied by the numbers of spectators who go away (at the end
of the respective parts of the day), give the amounts to be paid
(to the dancing party by the spectators who go away at the
end of the respective parts of the day).\renewcommand{\thefootnote} {1}\footnote{\en See \textit{GSS}, vi. 230.}
\vspace{3mm}

{\small By the word 'fruit (\textit{phala}) is meant here the amount to be paid to
the dancing party for the whole day.
\vspace{3mm}

 The above rule is based on the assumption that the spectators
have to pay the dancing party in proportion to the time of seeing the
dance. That is, if the spectators who see the dance for the whole day
pay 96  \textit{rūpas} in all, then the spectators who see the dance for $\frac{1}{4}$
of a day
will have to pay $\frac{96}{4}$, i.e., 24  \textit{rūpas} in all.
\vspace{3mm}

 For the illustration of the rule see under the next example.}
\vspace{3mm}

 Ex. 86-87. One man saw a dance for one quarter of the
day, another for two quarters of the day, another for
three quarters of the day, and (yet) another till the end of the
day. The dancing party has to be paid by them a sum of
ninety six  \textit{rūpas}  in all. If payment is to be made in proportion to the time of seeing the dance, how much of that (sum)
should be paid by each of them separately?
\vspace{3mm}

{\small \textit{Solution.} ~Writing down the parts of the day during which the
various spectators saw the dance, we have
\vspace{3mm}

\hspace{20mm} $\frac{1}{4}$ \qquad $\frac{2}{4}$  \qquad $\frac{3}{4}$  \qquad $\frac{4}{4}$
\vspace{3mm}

Diminishing the succeeding part by the preceding one, we get
\vspace{3mm}

\hspace{20mm} $\frac{1}{4}$  \qquad   $\frac{1}{4}$ \qquad  $\frac{1}{4}$  \qquad   $\frac{1}{4}$\vspace{3mm}

 Multiplying each of these by the fruit (i.e., by 96), we get
\vspace{2mm}

\hspace{20mm}  24 \qquad 24 \qquad 24 \qquad 24
\vspace{2mm}

Dividing these numbers by the number of spectators who saw the
dance during the respective parts of the day, we get
\vspace{2mm}

\hspace{20mm}  6 \qquad 8 \qquad 12 \qquad 24 }

\afterpage{\fancyhead[CO] {\small{MIXTURES. THE CISTERN PROBLEM}}}

\newpage

{\small  Adding the preceding numbers to the succeeding ones, we get
\vspace{2mm}

\hspace{1.5cm} 6 \hspace{1cm} 14 \hspace{1cm} 26 \hspace{1cm} 50 
\vspace{2mm}

 Multiplying by the number of spectators who go away at the end of
the respective parts of the day, we get
\vspace{2mm}

\hspace{1.5cm}  6 \hspace{1cm} 14 \hspace{1cm} 26 \hspace{1cm} 50
\vspace{2mm}

Hence the spectator who saw the dance for $\frac{1}{4}$ of the day will have
to pay 6  \textit{rūpas}; the spectator who saw the dance for $\frac{2}{4}$ of the day
will have
to pay 14  \textit{rūpas}; the spectator who saw the dance for $\frac{3}{4}$ of the day
will
have to pay 26  \textit{rūpas}; and the spectator who saw the dance for the
whole
day who will have to pay 50  \textit{rūpas}.}
\vspace{3mm}

 Ex. 88. A palanquin is to be carried to a distance of
3  \textit{krośas} by 10 men for 100 (\textit{rūpas}  as wages). Of those (men),
2, 3, and 5 stop away after going over 1, 2, and 3  \textit{krośas}  respectively. (Calculate the wages of each of them separately).\renewcommand{\thefootnote}{1}\footnote{\hspace{-2mm} \en \textit{Cf. GSS}, vi. 231-232.\\}
\vspace{3mm}

 Ex. 89-90. Five Brāhmaṇas, enchanters of  \textit{stotras}
("hymns in praise of gods"), were invited by a certain person
to take part in the worship of the five faces of the five-faced
god Śiva on (a remuneration of) 300   \textit{rūpas}. And they, on the
completion of worship of one, two, three, four, and five faces respectively, went away (from the place of worship) one by one.~~~~\renewcommand{\thefootnote}{\hspace{-4.5mm} 2}\footnote{\hspace{-2mm} \en What is meant is this: when the worship of the first face was
completed by all the five Brāhmaṇas, one Brāhmaṇas left the place of
worship;
when the worship of the second face was completed by the remaining
four
Brāhmaṇas, one more Brāhmaṇas left the place of worship; when the
worship of the third face was completed by the remaining three
Brāhmaṇas,
one more Brāhmaṇa left the place of worship, and so on.\\} Say what are their remunerations (separately).
\vspace{3mm}

\begin{center} \englishfont{(vii) \emph{THE CISTERN PROBLEM}}
\end{center}

\noindent Rule for finding the time in which a cistern would be filled
up by a number of drains flowing into it simultaneously, when
the times in which the cistern is filled up by each drain
separately are given:
\vspace{3mm}

 69. Having divided unity severally by the (given)
fractions (of time), take the sum of the quotients and by that
sum divide unity (again): this will give the time in which
the cistern is filled up (when all the drains are simultaneously
opened to flow into it).~~~~\renewcommand{\thefootnote}{\hspace{-4.5mm} 3}\footnote{\hspace{-2mm} \en \textit{Cf. MSi}, xv. 43; \textit{L} (ASS), p. 91, vs. 96; \textit{GK, I}, p. 94, lines
2-3.}

\newpage

{\small That is, if $n$ drains severally fill up a cistern in
\vspace{3mm} 

\hspace{20mm} $\dfrac{a_1}{b_1}$, ~$\dfrac{a_2}{b_2}$, ...,  $\dfrac{a_n}{b_n}$ ~of a day,
\vspace{3mm} 

\noindent then all the drains working together will fill up the cistern in
\vspace{3mm} 
  
\hspace{20mm} $\dfrac{1}{\dfrac{b_1}{a_1} + \dfrac{b_2}{a_2} + ... + \dfrac{b_n}{a_n}}$  ~of a day.}
\vspace{3mm} 

Ex. 91. In what time will the (four) drains, which
severally fill up a cistern in  $\frac{1}{2}$, $\frac{1}{4}$, $\frac{1}{5}$ and $\frac{1}{6}$ of a day, fill
up that cistern if they are opened simultaneously (to flow into
it)?\renewcommand{\thefootnote}{1}\footnote{\hspace{-2mm} \en Similar examples occur in Pṛthūdaka Svāmī's (860 A.D.)
comm.
on \textit{BrSpSi}, xii. 9; \textit{L} (ASS), p. 91, vs. 97; \textit{GK, I}, p. 94,
lines 5-6.\\}
\vspace{3mm} 

\begin{center} \englishfont{(viii) \emph{WAGES PAID FROM THE COMMODITY}}\end{center}

\noindent Rule for finding the wages of a porter who carries a commodity over a part of the stipulated distance, when the wages
are to be paid out of the commodity itself:
\vspace{3mm} 

 70. (To obtain the wages for a part of the stipulated
distance) divide half the product of the commodity and the
stipulated distance as diminished by the square root extracted
from the square of half the product of the commodity and the
stipulated distance as diminished by the (continued) product
of the distance (already) gone over, the distarice to be gone
over, the commodity, and the stipulated wages, by the
distance which is still to be gone over.~~~~\renewcommand{\thefootnote}{\hspace{-4.5mm} 2}\footnote{\hspace{-2mm} \en \textit{Cf. GSS}, vi. 226. For another rule, see \textit{GK, I}, p. 103, lines 10-15.}
\vspace{3mm} 

{\small Assuming that the (stipulated) wages for carrying $a$ articles over the
(stipulated) distance $d$ are $w$ articles (out of those $a$ articles), let
the
wages for carrying those $a$ articles over a distance ${d_1}$, where  ${d_1} \textless  d$, be
$x$ articles (out of those $a$ articles). Then calculating at the same
rate, the 
wages for carrying the remaining $a-x$ articles over the remaining
distance
$d- {d_1}$,  are equal to
\vspace{2mm} 

\hspace{20mm} $\dfrac{(a - x) (d - d_1)x}{ad_1}$ ~articles,
\vspace{3mm} 

But the remaining wages being $w-x$ articles, we have
\vspace{3mm} 

\hspace{20mm} $\dfrac{(a - x) (d - d_1)x}{ad_1}  = w - x$,}
 
\afterpage{\fancyhead[CO] {\small{MIXTURES. WAGES PAID FROM COMMODITY}}}

\newpage

{\small or, \hspace{1cm} $(d-{d_1})\,{x^2} - adx + awd_1 = 0$,
\vspace{3mm} 

whence \hspace{1cm} $x = \dfrac{\frac{ad}{2} \pm \sqrt{\left(\frac{ad}{2}\right)^2 - awd_1 (d-d_1)}}{d - {d_1}}$. 
\vspace{3mm} 

Taking the negative sign before the radical, we have the rule stated
above. The positive sign is rejected because it makes $x \textgreater w$.}
\vspace{3mm} 

 Ex. 92. If for carrying 24 jack-fruits over a distance of 5  \textit{krośas} (a porter) is to get 9 of those jack-fruits, what will
he get if he carries them over a distance of 2  \textit{krośas} (only)\,?\renewcommand{\thefootnote}{1}\footnote{\hspace{-2mm} \en \textit{Cf. GSS}, vi. 227. Also see \textit{GK, I}, p. 103, lines 17-20.\\}
\vspace{4mm} 

Rule for finding the distances over which two porters carry
a commodity in turn, when the wages paid to the porters
out of the commodity carried by them are given:
\vspace{3mm} 

 71. The first and the second loads (i.e., the loads carried
by the first and the second porters), each multiplied by
the wages for the other, being added together, the resulting sum is (to be taken as) the divisor (of the following
results): the results obtained by multiplying the products
of the wages received (for each of the loads) and the
(total) distance (gone over by the porters), each by the other
load. This gives the distances (gone over by the first and the
second porters separately).~~~~\renewcommand{\thefootnote}{\hspace{-4.5mm} 2}\footnote{\hspace{-2mm} \en See \textit{GSS}, vi, 228.}
\vspace{3mm} 

{\small Suppose that the first porter carries $a$ articles over a distance $x$
(unknown) and gets $b$ articles out of those $a$ articles as wages, and
that
the second porter carries the remaining $a - b$ articles over the
remaining
distance $d-x$ and gets $c$ articles as wages. Then, we have
\vspace{3mm} 

\hspace{20mm} $c = \dfrac{b\,(a - b)\,(d - x)}{ax}$
\vspace{3mm} 

whence \hspace{1cm} $\dfrac{x}{(a-b)\,b}  = \dfrac{d-x}{ac} = \dfrac{d}{ac + (a-b)\,b}$
\vspace{3mm} 

Hence the above rule.}
\vspace{3mm} 

 Ex. 93-94. Twenty-four jack-fruits were carried (over
a certain distance) by one (person) for 4 out of those (24) jack-fruits as wages; the remaining jack-fruits were carried (over
the remaining distance) by another person for 5 of them (as
wages). The load was thus carried by the two (person) over a distance of 5  \textit{krośas} (in all).

\newpage

\noindent Say, O learned, how much of
that distance was gone over by each of them?\renewcommand{\thefootnote}{1}\footnote{\hspace{-2mm} \en A similar problem occurs in \textit{GSS}, vi. 229.\\}
\vspace{3mm} 

\begin{center} \englishfont{(ix) \emph{COMBINATIONS OF SAVOURS}}\end{center}

\noindent Rule for finding the number of combinations that can be
formed out of a given number of savours (or things) by
taking 1, 2, 3, ..., all at a time:
\vspace{3mm} 

 72. Writing down the numbers beginning with 1 and
increasing by 1 up to the (given) number of savours in the inverse order, divide them by the numbers beginning with 1 and increasing by 1 in the regular order, and then multiply successively by the preceding (quotient) the succeeding one. (This will give the number of combinations of the savours
taken 1, 2, 3, ..., all at a time respectively).~~~~\renewcommand{\thefootnote}{\hspace{-4.5mm} 2}\footnote{\hspace{-2mm} \en \textit{Cf. GSS}, vi. 218; \textit{MSi}, xv, 45(ii)-46; \textit{L} (ASS), p. 106, vv. 112-113(i); \textit{GK, II}, pp. 318-319, vs. 58.\\}
\vspace{3mm} 

{\small Let the given number of savours be six. Then the numbers 1, 2, 3, ..., 6 are written down in the inverse order thus:
\vspace{2mm} 

\hspace{15mm} 6 \quad 5 \quad 4 \quad 3 \quad 2 \quad 1
\vspace{2mm} 

\noindent These are divided by the numbers 1, 2, 3, ..., 6 respectively, so that we get
\vspace{3mm} 

\hspace{15mm} $\dfrac{6}{1}, \quad \dfrac{5}{2}, \quad \dfrac{4}{3}, \quad \dfrac{3}{4}, \quad \dfrac{2}{5}, \quad \dfrac{1}{6}$. 
\vspace{3mm} 

\noindent Now multiplication is made successively by the preceding quotient of
the succeeding one. Thus we get
\vspace{3mm} 

\noindent $\dfrac{6}{1}, \quad \dfrac{6}{1} \times \dfrac{5}{2}, \quad  \dfrac{6}{1} \times \dfrac{5}{2} \times \dfrac{4}{3}, \quad \dfrac{6}{1} \times \dfrac{5}{2} \times \dfrac{4}{3} \times \dfrac{3}{4}, \quad \dfrac{6}{1} \times \dfrac{5}{2} \times \dfrac{4}{3} \times \dfrac{3}{4} \times \dfrac{2}{5}, \quad \dfrac{6}{1} \times \dfrac{5}{2} \times \dfrac{4}{3} \times \dfrac{3}{4} \times \dfrac{2}{5} \times \dfrac{1}{6}$.
\vspace{1mm} 
 
\noindent These, according to the rule, are respectively the numbers of
combinations
of the six savours taken 1, 2, 3, ..., 6 at a time.}
\vspace{3mm}

 Ex. 95. Friend, a cook prepares varieties of food with
the six savours, pungent, bitter, \,astringent, \,acid, \,saline, \,and \,sweet. \,Say \,what \,is \,the \,(possible) \,number \,of varieties?~~~~\renewcommand{\thefootnote}{\hspace{-4.5mm} 3}\footnote{\hspace{-2mm} \en The same example occurs in \textit{GSS}, vi. 219; \textit{GK, II}, p. 319, Ex. 22.}

\afterpage{\fancyhead[CO] {\small{MIXTURES. SPECIAL PROBLEMS}}}

\newpage

\noindent Rule for writing down the combinations of 2 or more savours
in a serial order:
\vspace{3mm} 

 73. \,To \,get \,the \,2-savour \,combinations \,add \,the \,preceding \,savour \,to \,all \,the succeeding one's in order; and to get
the combinations of 3 or more savours add the preceding
savour to the (succeeding) combinations of 2 or more savours which do not contain the preceding savour.
\vspace{3mm} 

{\small Let the six savours be denoted by
\vspace{2mm} 

\hspace{10mm} $a \quad b \quad c \quad d \quad e \quad f$.
\vspace{2mm} 

\noindent Then the 2-savour combinations are:
\vspace{2mm} 

\hspace{10mm} $ab, ~ac, ~ad, ~ae, ~af; ~bc, ~bd, ~be, ~bf; ~cd, ~ce, ~cf; ~de, ~df; ~ef$.
\vspace{2mm} 

\noindent The 3-savour combinations are:
\vspace{2mm} 

\hspace{10mm} $abc, ~abd, ~abe, ~abf, ~acd, ~ace, ~acf, ~ade, ~adf, ~aef; ~$
\vspace{1mm} 

\hspace{10mm} $bcd, ~bce, ~bcf, ~bde, ~bdf, ~bef; ~cde, ~cdf, ~cef; ~def$.
\vspace{2mm} 

\noindent The 4-savour combinations are:
\vspace{2mm} 

\hspace{10mm} $abcd, ~abce, ~abcf, ~abde, ~abdf, ~abef, ~acde, ~acdf, ~acef, ~adef;$ 
\vspace{1mm} 

\hspace{10mm} $bcde, ~bcdf, ~bcef, ~bdef; ~cdef$.
\vspace{2mm} 

\noindent The 5-savour combinations are:
\vspace{2mm} 

\hspace{10mm} $abcde, ~abcdf, ~abcef, ~abdef, ~acdef, ~bcdef$.
\vspace{2mm} 

\noindent The 6-savour combination is:
\vspace{2mm} 

\hspace{10mm} $abcdef$.}
\vspace{3mm} 

\begin{center} \englishfont{(x) \emph{CERTAIN SPECIAL TYPES OF PROBLEMS}}
\end{center}

\noindent Rule \,for \,solving \,problems \,on \,pillars \,(\textit{stambhoddeśaka}\renewcommand{\thefootnote}{1}\footnote{\hspace{-2mm} \en Mahāvīra calls \textit{sthambha-jāti} by the name \textit{bhāga-jāti} and \textit{Śrīpati}
by the name  \textit{dṛśya-jāti}. See \textit{GSS}, iv 4(i) and \textit{GT}, p. 41,
vs. 55.\\}) \,or \,problems \,involving remainders (\textit{śeṣoddeśaka}):
\vspace{3mm} 

 74(i). To solve problems on pillars or problems involving remainders, divide the visible \,quantity \,(\textit{dṛśya}) \,by \,one \,minus \,(the \,sum \,of) \,the \,fractional \,parts \,(of \,the whole).~~~~\renewcommand{\thefootnote}{\hspace{-4.5mm} 2}\footnote{\hspace{-2mm} \en \textit{Cf. GSS}, iv. 4(i); \textit{GT}, p. 41, vs. 55. Also see \textit{GK, I}, p. 17, lines 14-19. Mahāvīra gives a separate rule for the \textit{śeṣa-jati}. See
\textit{GSS}, iv 4(ii). The
same rule is given by Āryabhaṭa II and Śrīpati. See
\textit{MSi}, xv. 20; \textit{GT} p. 44, line 11.}

\newpage

 Ex. 96. One-fourth, one-third, and one-sixth of a pillar
are respectively buried under the water, mud, and sand of a river, and three cubits (of the pillar) are visible. Give out the measure (of the length) of that (pillar).\renewcommand{\thefootnote}{1}\footnote{\hspace{-2mm} \en \textit{Cf. GSS}, iv. 5; \textit{GT}, p. 41, vs. 56. For other examples
see \textit{GSS}, iv.
6-22; \textit{GT}, p. 42, vs. 57; \textit{L} (ASS), pp. 47-48, vs. 53; \textit{GK, I}, p.
20, lines 4-7,
10-13, 16-19.\\}
\vspace{3mm}

{\small  Here the given fractions (of the pillar) are $\frac{1}{4},\frac{1}{3}$, and $\frac{1}{6}$; and the
visible quantity (i.e., the visible part of the pillar) is equal to '3 cubits.'
Hence the length of the pillar
\vspace{3mm}

\hspace{15mm} $= \dfrac{3}{1- (\frac{1}{4} + \frac{1}{3}  + \frac{1}{6})}$, ~i.e., 12 cubits}
\vspace{3mm}

Ex. 97. After giving away one-half of a quantity, then
$\frac{2}{3}$ of what remains, then $\frac{3}{4}$ of what remains thereafter, and
then $\frac{4}{5}$ of what remains thereafter, the residue left is 3.
(What is that quantity?)~~~~\renewcommand{\thefootnote}{\hspace{-4.5mm} 2}\footnote{\hspace{-2mm} \en For other examples, see \textit{GSS}, iv. 29-30, 31, 32; \textit{GT}, p.44, lines
20-23; p. 45, lines 16-19; \textit{L} (ASS), pp. 49-51, vs. 54.\\}
\vspace{3mm}

{\small  Here the fractions (of the whole quantity) given away are
\vspace{3mm}

\hspace{10mm} $\frac{1}{2}, \frac{2}{3}$ of $(1 - \frac{1}{2}), \frac{3}{4}$ of $(1 - \frac{1}{2})(1 - \frac{2}{3})$, ~and~ $\frac{4}{5}$ of $(1 - \frac{1}{2})(1 - \frac{2}{3}) (1 - \frac{3}{4})$,
\vspace{3mm}

\hspace{2mm}  i.e., \hspace{2mm} $\frac{1}{2}, ~\frac{1}{3}, ~\frac{1}{8},$ ~and~ $\frac{1}{30}$;
\vspace{3mm}

\noindent and the visible quantity (i.e., the residue left) is 3.
\vspace{3mm}

Hence the required quantity
\vspace{3mm}

\hspace{15mm}  $= \dfrac{3}{1 - (\frac{1}{2} + \frac{1}{3} + \frac{1}{8} + \frac{1}{30})}$, ~i.e., 360}
\vspace{5mm}

\noindent Rule for solving problems involving differences (\textit{viśeṣoddeśaka}):
\vspace{3mm}

 74(ii). \,On \,subtracting \,the \,smaller \,quantity \,from \,the \,greater \,quantity \,what remains is called the difference
(\textit{viśeṣa}).~~~~\renewcommand{\thefootnote}{\hspace{-4.5mm} 3}\footnote{\hspace{-2mm} \en Mahāvīra does not differentiate between \textit{bhāga-jāti} and  \textit{viśeṣa-jati} (or \textit{viślesa-jāti})). So he sets problems on the latter under the \textit{bhāga-jati}. See \textit{GSS}, iv. Ex. 23-27.\\} (This having been done) the procedure (to be adopted is the same as the other one (i.e., the previous one).~~~~\renewcommand{\thefootnote}{\hspace{-4.5mm} 4}\footnote{\hspace{-2mm} \en \textit{Cf. GT}, p. 46, lines 10-13.}
\vspace{3mm}

 Ex. 98. Of a herd of cows, one-half went away towards
the east and one-fourth towards the west, the difference of the two as multiplied by 2 and divided by 5 went away towards the north, and three (cows) are left.
 
\newpage

\noindent (What is the numerical strength of the herd?)\renewcommand{\thefootnote}{1}\footnote{\hspace{-2mm} \en For similar examples see \textit{GSS}, iv. 23-27; \textit{GT}, p. 46,
lines 20-25;
\textit{L} (ASS), p. 53, vs. 55.\\}
\vspace{3mm}

{\small Here the fractions (of the herd of cows) that went away in the
various directions are
\vspace{3mm}

\hspace{20mm} $\frac{1}{2}, ~\frac{1}{4},$\; and\; $\frac{2}{5} \left(\frac{1}{2} - \frac{1}{4}\right)$
\vspace{3mm}

\hspace{12mm} i.e., \hspace{2mm} $\frac{1}{2}, ~\frac{1}{4},$\; and\; $\frac{1}{10}$
\vspace{3mm}

\noindent and the visible quantity (i e., the number of cows left) is 3. Hence
the numerical strength of the herd
\vspace{3mm}

\hspace{20mm}  $= \dfrac {3}{1 - \left(\frac{1}{2} + \frac{1}{4} + \frac{1}{10}\right)}$, ~i.e., 20 cows.}
\vspace{5mm}

\noindent Rule for solving problems involving remainders due to subtraction of square root, etc. (\textit{mūlādiśeṣoddeśaka}):~~~~\renewcommand{\thefootnote}{\hspace{-4.5mm} 2}\footnote{\hspace{-2mm} \en Mahāvīra and Śripati have called \textit{mūladiśeṣa-jāti} by the name \textit{śeṣamūla-jāti}.\\}
\vspace{3mm}

 75. When the visible quantity stands near a square
root, multiply the visible quantity by 4, then increase that by the square of the  \textit{pada} (i.e., the co-efficient of the square root
of the unknown), then extract the square root of that and
increase that by the  \textit{pada}, and then take the square of half of
that; (and when the visible quantity stands near a fraction)
divide the visible quantity by one minus the fraction.~~~~\renewcommand{\thefootnote}{\hspace{-4.5mm} 3}\footnote{\hspace{-2mm} \en \textit{Cf. GSS}, iv. 10; \textit{GT}, p. 48, lines 24-25.}
\vspace{3mm}

{\small The first part of this rule relates to the solution of the quadratic
equation of the type
\vspace{2mm}

\hspace{20mm} $x - p\sqrt{x} = d$,
\vspace{2mm}

\noindent where $p$ is the  \textit{pada}, $d$ the visible quantity (\textit{dṛśya}), and $\sqrt{x}$ the
positive
square root of $x$. The solution is correctly stated as
\vspace{3mm}

\hspace{20mm} $x = \left[\dfrac{\sqrt{4d + p^2} +p}{2}\right]^2$.}

\newpage

{\small The second part of the rule relates to the solution of the simple
equation of the type
\vspace{3mm}

\hspace{20mm} $x - \frac{a}{b} x = d'$
\vspace{3mm}

\noindent where $\frac{a}{b}$ is the fraction and $d'$ the visible quantity. The solution is
correctly stated as
\vspace{3mm}

\hspace{20mm} $x = \dfrac{d'}{1 - \frac{a}{b}}$}
\vspace{3mm}

Ex. 99. A number is diminished by its square root,
what remains is diminished by its one-sixth, what remains
after that is diminished by its square root, what remains after
that is diminished by its one-fifth, and what remains after
that is diminished by twice the square root of itself; the
residue now left is 8. (Find the number).\renewcommand{\thefootnote} {1}\footnote{\en  For other examples see, \textit{GSS}, iv. 41, 42-45, 46; \textit{GT}, p. 49,
lines 7-10; 60, lines 2-5.}
\vspace{3mm}

{\small  This problem reduces to the solution of the following equation:
\vspace{3mm}

 $x - \sqrt{x} - \dfrac{1}{6} ( x - \sqrt{x}) - \sqrt{x-\sqrt{x} - \frac{1}{6} (x - \sqrt{x})} - \dfrac{1}{5} \bigg\{x - \sqrt{x} -\frac{1}{6}(x - \sqrt{x})$
\vspace{3mm}

$- \sqrt{x -\sqrt{x}-\frac{1}{6} (x-\sqrt{x})}\bigg\} - 2 \left[{x} - \sqrt{x} - \frac{1}{6} (x-\sqrt{x}) -\sqrt{x-\sqrt{x} - \frac{1}{6} (x-\sqrt{x})}\right. $
\vspace{3mm}

$\left. - \frac{1}{5} \bigg\{x -\sqrt{x} - \frac{1}{6} ( x - \sqrt{x}) - \sqrt{x-\sqrt{x} -\frac{1}{6} (x - \sqrt{x})}\bigg\}\right]^{\frac{1}{2}} = 8 $,
\vspace{4mm}

\noindent which can be written as
\vspace{2mm}

\hspace{3cm} $x - \sqrt{x} = y$, \hfill (1) \hspace{20mm}
\vspace{2mm}

\hspace{1cm} where \hspace{1cm} $y- \frac{1}{6}y \,=\, z$, \hfill (2) \hspace{20mm}
\vspace{1mm}

\hspace{3cm} $z -\sqrt{z} \,= u$, \hfill (3) \hspace{20mm}
\vspace{1mm}

\hspace{3cm} $u - \frac{1}{5}u \,=\, v$ \hfill (4) \hspace{20mm}
\vspace{1mm}

\hspace{3cm} $v - 2\sqrt{v} = 8$, \hfill (5) \hspace{20mm}
\vspace{4mm}

 Solving these equations in the inverse order, we get to $x = 36$.
\vspace{3mm}

\textit{Hindu Method of Solution}. ~Writing down the square roots and
fractions, which have been the successively subtracted from the original
number,
in order and then visible quantity, we get
\vspace{2mm}

\noindent \textit{Square root, $\frac{1}{6}$ part of the remainder, square root of the remainder,
$\frac{1}{5}$ part of the
remainder, 2 times the square root of the remainder, visible quantity
8}.}

\newpage

 Since the visible quantity stands near a square root, therefore performing the operations prescribed in the first part of the rule, we get
\vspace{3mm}

\hspace{20mm} $\left[\dfrac{\sqrt{4 \times 8 + 2^2} + {2}}{2}\right]^2$, ~i.e., 16.
\vspace{3mm}

\noindent This is the number which when diminished by 2 times the square root
of itself yields 8 as remainder.
\vspace{3mm}

 Now treat this (16) as the visible quantity. Since it stands near a
fraction, there-fore performing the operation prescribed in the second
part
of the rule, we get
\vspace{3mm}

\hspace{30mm} $\dfrac{16} {1- \frac{1}{5}}$, ~i.e., 20. 
\vspace{3mm}

\noindent This is the number which when diminished by $\frac{1}{5}$ of itself yields 16
as remainder.
\vspace{3mm}

 Now treat this (20) as the visible quantity. Since it stands near a
square root, therefore performing the operations prescribed in the
first
part of the rule, we get
\vspace{3mm}

\hspace{20mm} $\left[\dfrac{\sqrt{4 \times 20+1} +{1}}{2}\right]^2$, ~i.e., 25 
\vspace{3mm}

\noindent This is the number which when diminished by its square root yields 20
as remainder.
\vspace{-0.1mm}

 Now treat this (25) as the visible quantity. Since this stands near a fraction, therefore performing the operation prescribed in the second
part of the rule, we get
\vspace{-0.1mm}

\hspace{30mm} $\dfrac{25}{1- \frac{1}{6}}$, ~i.e., 30
\vspace{3mm}

\noindent This is the number which when diminished by its $\frac{1}{6}$ yields 25 as
remainder.
\vspace{3mm}

 Now treat this (30) as the visible quantity. Since this stands near a square root, therefore performing the operations prescribed in the
first part of the rule, we get
\vspace{3mm}

\hspace{20mm} $\left[\dfrac{\sqrt{4 \times 30+1} +{1}}{2}\right]^2$, ~i.e., 36 
\vspace{3mm}

\noindent This is the number which being diminished by its square root yields
30 as remainder.
\vspace{-0.1mm}

 Since all quantities in the statement of the problem are now exhausted, no fur-ther operation is needed. The required number is thus
obtained to be 36.

\newpage

\begin{sloppypar}

\noindent Rule for solving problems in which the visible quantity is the
remainder due to the subtraction of a fraction of the unknown
and also a multiple of the square root of the unknown
(\textit{bhāgamūlāgroddeśa}):\renewcommand{\thefootnote}{1}\footnote{\hspace{-2mm} \en Mahāvīra calls \textit{bhāgamūlāgra-jāti} by the name \textit{mūla-jāti}, Śrīpati by
the name  \textit{mūlāgra-bhāga-jāti}, and Nārāyaṇa by
the name  \textit{ṛṇāṃśavimūla-jāti}.\\}
\vspace{3mm}

 76. After having divided the  \textit{pada} (i.e., the co-efficient
of the square root of the unknown) and the visible quantity
(or ultimate remainder,  \textit{agra}) by \textit{one} minus the fraction, add
the square of half the first quotient to the second quotient,
then take the square root of that, then increase that by half
the first quotient, and then multiply that by itself.~~~~\renewcommand{\thefootnote}{\hspace{-4.5mm} 2}\footnote{\hspace{-2mm} \en \textit{Cf. GSS}, iv. 33; \textit{GT}, p. 50 lines 27-28 (contd. on p. 51, lines
1-2). Also see \textit{GK, I}, p. 21, lines 2-6.\\}
\vspace{3mm}

{\small This rule is meant to solve problems reducing to a quadratic equation of the type
\vspace{3mm}

\hspace{20mm} $x- \dfrac{a}{b}x- p\sqrt{x} = d$,
\vspace{3mm}

\noindent where $\frac{a}{b}$ is the 'fraction', \textit{p} the '\textit{pada}', \textit{d} is the visible quantity, and $\sqrt{x}$ the positive square root of \textit{x}.
\vspace{3mm}

The solution is correctly stated as

\begin{center} $x= \left[\dfrac{p}{2\,\left(1 - \frac{a}{b}\right)} + \sqrt{\left\lbrace\dfrac{p}{2\,\left(1 - \frac{a}{b}\right)}\right\rbrace^2 + \dfrac{d}{1-\frac{a}{b}}}\,\right]^ 2$
\end{center}}
\vspace{1mm}

 Ex. 100. One-third of a troop of monkeys together
with one-third of itself has gone to the tank; the square root (of the whole troop) is afflicted with thirst; and the remaining two monkeys are sitting under the mango tree. (What is
the number of monkeys in the troop\,?)~~~~\renewcommand{\thefootnote}{\hspace{-4.5mm} 3}\footnote{\hspace{-2mm} \en For other examples see \textit{GSS}, iv. 34, 35, 36, 37, 38, 39; \textit{GT}, p.
51, lines, 11-14; p. 52, lines 16-19; p. 53, lines 17-20; \textit{GK, I}, p. 23,
lines 15-16, 18-19, and p. 24, lines 1-2.\\}
\vspace{4mm}

\noindent Rule for solving problems involving two visible quantities,
the square root of the original number, and remainders due
to subtraction of fractional multiples of remai-nders (\textit{ubhayāgra-mūlaśeṣoddeśa}):~~~~\renewcommand{\thefootnote}{\hspace{-4.5mm} 4}\footnote{\hspace{-2mm} \en Mahāvīra calls  \textit{ubhayāgramūlaśeṣa-jāti}
by the name  \textit{dviragraśeṣamūla-jāti,} and Śrīpati by the name \textit{ubhayāgradṛśya-jāti}.}
\vspace{3mm}

 77. Take the continued product of units severally
diminished, as before, by the fractional multiples of remainders as the divisors of the  \textit{pada} as well as the (last) visible
quantity;

\end{sloppypar}

\newpage

\noindent to the second quotient also add the first visible
quantity. After that apply the previous rule (starting with
'add the square of half the first quotient').\renewcommand{\thefootnote}{1}\footnote{\hspace{-2mm} \en \textit{Cf. GSS}, iv. 47; \textit{GT}, p. 54, lines 21-24.\\}
\vspace{3mm}

{\small  This rule applies to problems which reduce to the solution of the
quadratic equation of the type
\vspace{1mm}

\hspace{20mm} $(1 - \frac{a}{b}) (1 - \frac{c}{d}) (1 - \frac{e}{f}) (x - d_1) - p\sqrt{x} = {d_2}$, 
\vspace{3mm}

\noindent where $\frac{a}{b}, \frac{c}{d}, \frac{e}{f}$ are the fractional multiples of remainders, \textit{p}
is the  \textit{pada}, ${d_1}, {d_2}$ the first and last visible quantities, and $\sqrt{x}$ denotes
the positive square root of \textit{x}, as before.
\vspace{3mm}

 According to the above rule, this equation should be first
transformed into the form 

\begin{center} $x - \dfrac{p}{(1 - \frac{a}{b}) (1 - \frac{c}{d}) (1 - \frac{e}{f})}\,\sqrt{x} \,=\, \dfrac {d_2}{(1 - \frac{a}{b}) (1 - \frac{c}{d}) (1 - \frac{e}{f})} + {d_1}$, \end{center}

\noindent and then solved by the previous rule.}
\vspace{3mm}

 Ex. 101. After giving away \textit{one} (out of a certain
number), then one-sixth of what remains, then one-fourth of what remains after that, then one-third of what remains after
that, and then the square root of the original number, the
residue left is 5. (What is that number\,?)~\,~~~\renewcommand{\thefootnote}{\hspace{-4.5mm} 2}\footnote{\hspace{-2mm} \en For other examples see \textit{GSS}, iv. 48, 49, 50; \textit{GT}, p. 55, lines
7-10;
p. 56, lines 8-11; p. 5, lines 19-22.\\}
\vspace{4mm}

\noindent Rule of inversion:~
\vspace{2mm}

 78. (Proceeding from the visible quantity backwards,
make) addition subtraction, subtraction addition, multiplication division, division multiplication, square
square-root, and square-root square: this is stated to be the method
of inversion.~~~~\renewcommand{\thefootnote}{\hspace{-4.5mm} 3}\footnote{\hspace{-2mm} \en \textit{Cf. Ā}, ii. 28;  \textit{BrSpSi}, xviii. 14(17);  \textit{GSS}, vi. 286; \textit{MSi}, xv. 23; \textit{GT}, p. 65, vs. 83; \textit{SiŚe}, xiii. 13; \textit{L} (ASS), p. 42, vs. 48; \textit{GK, I}, p.
46, lines 13-16.\\}
\vspace{3mm}

 Ex. 102. Say what (is that number which) being multiplied by $\frac{5}{2}$, then divided by 3, then squared, then increased
by 9, then reduced to its square root, and then diminished by
1, becomes 4.~~~~\renewcommand{\thefootnote}{\hspace{-4.5mm} 4}\footnote{\hspace{-2mm} \en For similar examples see \textit{GSS}, vi 287; \textit{GT}, P. 66, vs. 84; p. 67,
vs. 85; \textit{L} (ASS), p. 43, vs. 50.}

\newpage

\phantomsection \label{ser}
\begin{center} {{\textbf{(2) Determinations pertaining to series} (\textit{śreḍhī-vyavahāra})} 
\vspace{2mm}

\englishfont{(i) \emph{SERIES IN ARITHMETIC PROGRESSION~}}
\vspace{1mm}

\emph{(GEOMETRICAL INTERPRETATION)}}
\end{center}
\vspace{1mm}

\noindent Form of a series-figure (\textit{śreḍhī-kṣetra}): 
\vspace{2mm}

 79. As in the case of an earthen drinking glass
(\textit{śarāva}) the width at the base is smaller and at the top
greater, so also is the case with a series-figure (\textit{śreḍhī-kṣetra}).
\vspace{-0.1mm}

The altitude (\textit{lambaka}) of that (series-figure) is equal
to the number of terms (\textit{gaccha}) of the (corresponding)
series.~ 
\vspace{3mm}

{\small  The series-figure contemplated here is a plane figure resembling
a trapezium with equal flank sides
\vspace{3mm}

 If a series be
\vspace{2mm}

\hspace{20mm}$ a + (a + d) + (a + 2d) + ...$ to \textit{n} terms,
\vspace{3mm}

\noindent then, according to the second part of the verse, the altitude of the
corresponding series-figure $= n$ units, say \textit{n} cubits.}
\vspace{4mm}

 80(i). The (partial) areas (\textit{phala}) of the series-figure
for the successive cubits (\textit{kara}) of the altitude form a series
which begins with the given  \textit{ādi} ('first term of the series')
and increases successively by the given  \textit{caya} ('common
difference of the series').~ 
\vspace{3mm}

{\small That is, the area of the series-figure for the first cubit of the
altitude $= a$, i.e., the first term of the series; the area of the series-figure
for the second cubit of the altitude $= a + d$, i.e., the second term of the series; the
area of the series-figure for the third cubit of the altitude $= a + 2d$;
and so on. }
\vspace{4mm}

\noindent Construction of a series-figure:
\vspace{2mm}

 80(ii). I shall now describe the method for finding
the lengths of the base (i.e., lower side,  \textit{bhū}) and the face
(i.e., upper side,  \textit{mukha}) of the series-figure (corresponding
to the first term of the series).~ 
\vspace{4mm}

 81. The number of terms (\textit{pada}), i.e., one, is the
altitude of the (corresponding) series-figure; the first term of the series (\textit{mukha}) as diminished by half the common
difference of the series is the base (\textit{dharā}); 

\afterpage{\fancyhead[CO] {\small{SERIES IN ARITHMETIC PROGRESSION}}}

\newpage

\noindent and that (base)
increased by the common difference of the series is the face
(\textit{vaktra}). All these should be shown by means of threads. 
\vspace{3mm}

{\small That is,

\hspace{2cm} base $= a - \dfrac{d}{2}$, 
\vspace{3mm}

\hspace{5mm} and \hspace{0.9cm} face $= \left(a - \dfrac{d}{2}\right) + d$, ~i.e.,~ $a + \dfrac{d}{2}$}
\vspace{3mm}

 82. (Two) threads should then be stretched out, one
on either side, joining the extremities of those base and face:
these are the flank sides (\textit{bāhu}) of the series-figure.~ 
\vspace{3mm}

 When the base is negative, these threads should be stretched out crosswise.
\vspace{3mm}

{\small Thus the series-figure will be of one of the following two forms:}
\vspace{3mm}

\includegraphics[width=\linewidth, height=4cm]{Images/page-0304as.jpeg}
\vspace{0.1mm}

{\small Forms (ii) corresponds to the negative base.}
\vspace{4mm}

 83. (When the base is negative the series-figure reduces
to two triangles situated one over the other.) In the upper
triangle, the altitude is equal to the face as divided by
face minus base; and that subtracted from one gives the
altitude in the lower triangle.~ 
\vspace{3mm}

{\small That is,  
\vspace{1mm}

\hspace{1cm} (i) altitude of the upper triangle $= \dfrac{\textrm{face}}{\textrm{face} - \textrm{base}}$, ~i.e.,~ $\dfrac{2a+ d}{2d}$,
\vspace{3mm}

\hspace{1cm} (ii) altitude of the lower triangle $= 1 - \dfrac{\textrm{face}}{\textrm{face} - \textrm{base}}$, ~i.e.,~ $\dfrac{d - 2a}{2d}$}

\newpage

\noindent Rule for finding the face of the series-figure corresponding to
the given series:~
\vspace{3mm}

 84. Having constructed the series-figure (for altitude
unity) in this manner, one should determine the face for the desired altitude (i.e., for the desired number of terms of the series) (by the following rule):~ 
\vspace{3mm}

 The face (for altitude unity) minus the base (for altitude unity), multiplied by the desired altitude, and then increased by the base (for altitude unity), gives the face (for the desired altitude).
\vspace{3mm}

{\small This rule, on simplification, reduces to the following formula:~
\vspace{2mm}

\hspace{20mm} face for altitude ~$n = a + \left(n - \frac{1}{2}\right)\,d$}
\vspace{5mm}

\noindent Rule for finding (i) the sum of a series in A.P. (interpreted geometrically by a series-figure), and (ii) the area of the corresponding series-figure:~ 
\vspace{3mm}

 85. The common difference as multiplied by one-half
of the number of terms minus \textit{one}, being increased by the first term, and then multiplied by the number of terms, gives the sum of the series.\renewcommand{\thefootnote} {1}\footnote{\en \textit{Cf. Ā}, ii. 19; \textit{GSS}, ii. 61, vi. 290; \textit{GK, I}, 105, lines
12-13. The rule of the \textit{GK} is literally the same as above.~ \\

\hspace*{3mm} Brahmagupta (\textit{BrSpSi}, xii, 17) states the result in the modern form:\\
 
\hspace*{10mm} $S_n = a + (a + d) + (a + 2d) +...$ to \textit{n} terms $= \left(\frac{n}{2}\right)\,[2a + (n - 1)d]$. \\

The same form is given in \textit{GSS}, ii. 62; \textit{MSi}, xv, 47; \textit{SiŚe}, xiii. 20; \textit{L} (ASS) p. 114, vs. 121; \textit{GK, I}, p. 105, vs. 1.}
\vspace{3mm}

 And the area of the (corresponding) series-figure is equal to the product of one-half of the sum of the base and the face, and the altitude. 
\vspace{3mm}

{\small That is, the sum of the series~
\vspace{3mm}

\hspace{1cm} $a + (a + d) + (a + 2d) + ...$ ~to \textit{n} terms
\vspace{3mm}

\noindent is equal to \hspace{1cm} $\left\lbrace \dfrac{n - 1}{2}\,d + a \right\rbrace \,n$;\hfill (1) \hspace{10mm}
\vspace{3mm}

\noindent and the area of the corresponding series-figure is equal to~ 
\vspace{3mm}

\hspace{20mm} $\dfrac{\textrm{base} + \textrm{face}} {2} \times $ \,altitude, \hfill (2) \hspace{10mm}}

\newpage

\noindent {\small where, ~according to vv. 80(ii) to 84,  
\vspace{3mm}

\hspace{15mm} base $= a - \dfrac{d}{2}$,
\vspace{3mm}

\hspace{15mm} face $= a + (n - \frac{1}{2})\,d$,
\vspace{3mm}

and \hspace{10mm} altitude $= n$. 
\vspace{3mm}

It may be easily seen that (1) and (2) are the same.}
\vspace{3mm}

 Ex. 103(i). What is the sum of 5 terms of the series
whose first term is 2 and common difference 3\,? And what
of one-half of a term\,?~ 
\vspace{3mm}

 Ex. 103(ii). (Also) say (the sum) of one-fifth of a term
of the series whose common difference is 5 and first term 2.~ 
\vspace{3mm}

 Ex. 104-105. In a leathern oil-bottle (\textit{kutapa}) full of
oil there occurs a minute hole, and the oil leaks through it.
The bottle has to be carried to a distance of 3  \textit{yojanas}. If the
wages for the first  \textit{yojana} be 10  \textit{paṇas} and those for the
subsequent  \textit{yojanas} successively less by 2  \textit{paṇas}, what are the wages
for a  \textit{krośa}\,?~ 
\vspace{4mm}

Sub-rule for finding the first term of a series in A.P.
(interpreted geometrically by a series-figure), when the common
difference, number of terms, and the sum of the series are known:~ 
\vspace{3mm}

 86(i). The sum of the series (\textit{gaṇita}) as divided by the
number of terms of the series (\textit{pada}), being diminished by
half the common difference (\textit{caya}) as multiplied by the
number of terms (\textit{gaccha}) minus 1, gives the first term of the
series (\textit{ādi}).\renewcommand{\thefootnote} {1}\footnote{\en  \textit{Cf. GSS}, ii. 74(i); vi. 292(i); \textit{MSi}, xv. 48; \textit{SiŚe}, xiii.
23(i); \textit{L} (ASS),
p. 116, vs. 124; \textit{GK, I}, p. 106, vs. 2. Also see \textit{GSS}, ii. 73(ii), 76.}
\vspace{3mm}

{\small That is,  
\vspace{1mm}

\hspace{20mm} $a = \dfrac{s}{n} - \dfrac{d}{2}\,(n - 1)$,
\vspace{3mm}

\noindent where $a, d, n,$ and $s$ respectively denote the first term, common difference, number of terms, and the sum of the series.}
 
\newpage

{\small We know (\textit{vide supra} Rule 85) that
\vspace{3mm}

\hspace{20mm} $s = \left[\dfrac{n - 1}{2}\,d + a\right]\,n$
\vspace{3mm}

\noindent whence on solving for $a$, we get
\vspace{3mm}

\hspace{20mm} $a = \dfrac{s}{n} - \dfrac{d}{2}\,(n -1)$.}
\vspace{5mm}

\noindent Sub-rule for finding the common difference of a series in A.P. (interpreted geometrically by a series-figure) when the first term, the number of terms, and the sum of the series are known:~ 
\vspace{3mm}

 86(ii). The sum of the series (\textit{phala}) as divided by the
number of terms of the series (\textit{pada}), being (first) diminished
by the first term (\textit{mukha}) and then divided by half of the
number of terms minus 1, gives the common difference of the
series (\textit{pracaya}).\renewcommand{\thefootnote} {1}\footnote{\en  \textit{Cf. GSS}, ii. 74(ii); vi. 292(ii); \textit{MSi}, xv. 49; \textit{L} (ASS), p. 117,
vs. 126. Also see \textit{GSS}, ii. 73(i); 75; \textit{SiŚe}, xiii. 23(ii).}
\vspace{3mm}

{\small That is,
\vspace{1mm}

\hspace{20mm} $d = \dfrac{\frac{s}{n} - a} {\frac{1}{2}(n - 1)}$,
\vspace{3mm}

\noindent where $a, d, n,$ and $s$ have their usual meanings.
\vspace{3mm}

 This formula is obvious from that of Rule 86(i).}
\vspace{4mm}

\noindent Sub-rule for finding the number of terms of a series in A.P. (interpreted geometrically by a series-figure), when the first term, the common difference, and the sum of the series are given:~
\vspace{3mm}

 87. Multiply the sum of the series (\textit{phala}) by 8 times
the common difference (\textit{uttara}) and (to that product) add
the square of the difference between twice the first term (\textit{ādi})
and the common difference (\textit{pracaya}): take the square root
of that. That (square root) diminished by twice the first
term (\textit{mukha}) and increased by the common difference and
(then) divided by twice the common difference, 

\newpage

\noindent  gives the number of terms of the series (\textit{gaccha}).\renewcommand{\thefootnote}{1}\footnote{\hspace{-2mm} \en \textit{Cf.} BrSpSi, xii. 18; \textit{GSS}, ii. 70. Also see \textit{GSS}, ii. 69.\\}
\vspace{3mm}

{\small That is,
\vspace{1mm}

\hspace{20mm} $n = \dfrac{\sqrt{8ds + (2a - d)^2} - 2a + d}{2d}$,
\vspace{3mm}

\noindent where $a, d, n,$ and $s$ have their usual meanings 
\vspace{3mm}

 We know (\textit{vide supra} Rule 85) that~ 
\vspace{3mm}

\hspace{20mm} $s = \left[\dfrac{n - 1}{2}\,d + a\right]\,n$,
\vspace{3mm}

\hspace{15mm} $\therefore\; dn^2 + (2a - d)\,n - 2s = 0$.
\vspace{3mm}

On solving this quadratic for \textit{n}, we get~\,~~~\renewcommand{\thefootnote}{\hspace{-4.5mm} 2}\footnote{\hspace{-2mm} \en Āryabhaṭa I (\textit{Ā}, ii. 20) puts the result in the
following form: \\

\vspace*{1mm}
\hspace*{20mm} $n = \frac{1}{2} \left[\dfrac{\sqrt{8ds + (2a - d)^2} - 2a}{d} + 1\right]$.\\

\vspace*{1mm}
Āryabhaṭa II (\textit{MSi}, xv. 50), Bhāskara II (\textit{L}, p. 118, vs. 128) and
Nārāyaṇa (\textit{GK, I}, p. 107, lines 4-7) put it in the form: \\

\vspace*{1mm}
\hspace*{20mm}  $n = \dfrac{\sqrt{2ds + (a - \frac{d}{2})^2} - a + \frac{d}{2}}{d}$ \\

\vspace*{1mm}
 Śrīpati (\textit{SiŚe}, xiii. 24) puts it in the form: \\

\vspace*{1mm}
\hspace*{20mm}  $n = \sqrt{\dfrac{s}{\frac{d}{2}} + \left(\dfrac{a - \frac{d}{2}}{d}\right)^2} - \dfrac{a - \frac{d}{2}}{d}$}

\vspace{3mm}
\hspace{20mm} $n = \dfrac{\sqrt{8ds + (2a - d)^2} - 2a + d}{2d}$,
\vspace{3mm}

\noindent taking the positive sign of the radical, because \textit{n} is positive.
\vspace{2mm}

 Hence the above rule.}
\vspace{4mm}

\noindent Sub-rule for finding the first term of a series in A.P. (interpreted geometrically by a series-figure), when the sum of the series, the number of terms, and the sum of the first term and common difference are known:~ 
\vspace{3mm}

 88. Having subtracted the sum of the series (\textit{phala}) from the mixed amount (\textit{miśra-dhana}) (i.e., the sum of the first term and common difference) as multiplied by one-half of (the difference of) the number of terms squared minus the number of terms, divide the residue by one-half of (the difference of) the number of terms minus 1, as diminished by 1 and
multiplied by the number of terms.

\newpage

\noindent Thus is obtained the first term of the series (\textit{ādi}).
\vspace{3mm}
 
{\small That is~ 
\vspace{1mm}

\hspace{20mm} $a = \dfrac{\dfrac{n^2 - n}{2}(a + d) - s}  {\left\lbrace \dfrac{n - 1}{2} - 1\right\rbrace\, n}$,
\vspace{3mm}

\noindent where the symbols have their usual meanings. 
\vspace{3mm}

 We know (\textit{vide supra} Rule 85) that~ 
\vspace{3mm}

\hspace{20mm} $a = \left\lbrace \dfrac{n - 1}{2}d + a\right\rbrace \,n$
\vspace{3mm}

\hspace{22.5mm} $= \left[\dfrac{n - 1}{2}\,(\overline{a + d} - a) + a\right]\,n$ 
\vspace{3mm}

\hspace{22.5mm} $= \dfrac{n^2 - n}{2}\,(a + d) - \left(\dfrac{n - 1}{2} - 1\right)\,na$
\vspace{3mm}

\hspace{16mm} $\therefore\, a = \dfrac{\dfrac{n^2 - n}{2}(a + d) - s}  {\left\lbrace \dfrac{n - 1}{2} - 1\right\rbrace\, n}$}
\vspace{5mm}

\begin{center} \englishfont{(ii) \emph{SERIES IN ARITHMETIC PROGRESSION}~ 

(\emph{SYMBOLICAL INTERPRETATION})}
\end{center}

{\small According to the symbolical interpretation, the series
\vspace{3mm}

\hspace{15mm} $a + (a + d) + (a + 2d) + ...$ to $(n + \frac{p}{q})$ terms 
\vspace{2mm}

\noindent means the sum of \textit{n} terms together with the $\left(\frac{p}{q}\right)^{\textrm{th}}$ part of the  $(n + 1)^{\textrm{th}}$ term.}
\vspace{3mm}

\noindent Rule for finding the sum of a series in A.P. (interpreted
symbolically), when the number of terms is partly integral
and partly fractional, the first term, common difference,
and the number of terms being known:
\vspace{3mm}

89. The \,common \,difference \,(\textit{caya}) \,as \,multiplied \,by \,the \,integral \,part \,of \,the  numbers of terms (\textit{nirvikalapada}) should be increased by the first term  (\textit{ādi}), and the result (obtained) should be kept undestroyed (at one place). The same result (written in another place) being increased by the first term
(\textit{mukha}), (then) diminished by the common difference, (then) multiplied by one-half of the integral part of the number of  terms, and (then) added to the 'undestroyed result' as multiplied by the fractional part of the number of terms (\textit{vikala}),
gives the sum of the series (\textit{gaṇita}).

\newpage

{\small That is to say, if $n + \frac{p}{q}$ be the number of terms, then~ 
\vspace{2mm}

\hspace{20mm} $s = \dfrac{n}{2}\,(dn + a + a - d) + \dfrac{p}{q}\,(dn + a)$, 
\vspace{2mm}

\noindent where the symbols have their usual meanings.
\vspace{2mm}

 It will be noted that~ 
\vspace{2mm}

\hspace{20mm} $\dfrac{n}{2}\,(dn + a + a - d)$
\vspace{2mm}

\noindent denotes the sum of the \textit{n} terms of the series, and~ 
\vspace{2mm}

\hspace{20mm} $\dfrac{p}{q}\,(dn + a)$
\vspace{2mm}

\noindent is equal to the $\left(\frac{p}{q}\right)^{\textrm{th}}$ part of the $(n + 1)^{\textrm{th}}$ term of the series.}
\vspace{3mm}

 Ex. 106. One man gets 3 (\textit{rūpas}), and the other men get 2  \textit{rūpas} more in succession; say, what do (the first) $4\frac{1}{2}$ men get.
\vspace{3mm}

{\small  By saying $4\frac{1}{2}$ men is meant, according to the commentator, that the
fifth man does only half the work due from him.}
\vspace{3mm}

 Ex. 107. If a labourer gets $1\frac{1}{2}$ in the first month and $\frac{1}{3}$
more in succession in the following months, what will he
get in (the first) 3$\frac{1}{2}$ months\,? 
\vspace{4mm}

\noindent Sub-rule for finding the first term of a series in A.P. (interpreted symbolically), when the common difference, the number of terms (which is partly integral and partly fractional), and
the sum of the series are known:~ 
\vspace{3mm}

 90. The integral part of the number of terms
(\textit{nirvikalapada}) minus 1, halved and increased by the fractional part of
the number of terms (\textit{vikala}), should be multiplied by the
common difference (\textit{caya}) and also by the integral part of the
number of terms (\textit{vikala-vihīna-pada}). The sum of the series
(\textit{dhana}) minus that, when divided by the (given) number of
terms, gives the first term of the series (\textit{prabhava}). 
\vspace{3mm}

{\small That is, ~if $n + \frac{p}{q}$ be the number of terms, then 
\vspace{2mm}

\hspace{20mm} $a = \dfrac{s - \left\lbrace \frac{1}{2}(n - 1) + \frac{p}{q}\right\rbrace\, dn} {n + \frac{p}{q}}$ 
\vspace{3mm}

We know (\textit{vide supra} Rule 89) that~ 
\vspace{2mm}

\hspace{20mm} $s = \dfrac{n}{2}(dn + a + a - d) + \dfrac{p}{q}(dn + a)$
\vspace{2mm}

\hspace{22.5mm} $= a\left(n + \frac{p}{q}\right) + \left\lbrace \frac{1}{2}(n - 1) + \frac{p}{q}\right\rbrace\, dn$. 
\vspace{2mm}

\hspace{16mm} $\therefore\; a = \dfrac{s - \left\lbrace \frac{1}{2}(n - 1) + \frac{p}{q}\right\rbrace\, dn} {n + \frac{p}{q}}$}

\newpage

\noindent Sub-rule for finding the common difference of a series in A.P. (interpreted symbolically), when the first term, the number of terms which is partly integral and partly fractional, and the sum of the series are known:~ 
\vspace{3mm}

 91. The sum of the series (\textit{dhana}) as diminished by the product of the first term and the number of terms, should be divided by the sum of the series in which the first term and common difference are each unity and the number of terms
is equal to the given number of terms minus 1. The result which is thus obtained is the common difference. 
\vspace{3mm}

{\small That is, if $n + \frac{p}{q}$ be the number of terms, then~
\vspace{2mm}

\hspace{20mm} $d = \dfrac{s - a\left(n + \frac{p}{q}\right)}{S}$,
\vspace{3mm}

\noindent where S is the sum of $\left(n + \frac{p}{q} - 1\right)$ terms of the series $1 + 2 + 3 +...,$ i.e.,  
\vspace{3mm}

\hspace{20mm} $\dfrac{(n - 1)n}{2} + \dfrac{p}{q}n$.
\vspace{3mm}

\noindent (\textit{Vide supra}, Rule 89).}
\vspace{4mm}

\noindent Sub-rule for finding the number of terms of a series in A.P.
which is partly integral and partly fractional, when the first term, the common difference, and the sum of the series are known, the series being interpreted symbolically:~ 
\vspace{3mm}

 92-93. To \,the \,sum \,of \,the \,series \,(\textit{dhana}) \,as \,multiplied \,by \,twice \,the \,common difference (\textit{caya}), add the square of (the
difference of) the first term minus half the common difference:
of that obtain the nearest (integral) square root. That (square
root), diminished by the square root of the previous square
(i.e., the square of the difference of the first term minus half
the common difference), (then) divided by the common
difference, and (then) rid of its fractional part (\textit{vikala}), is
the so called 'undestroyed quantity.'~
\vspace{3mm}

 That (ʻundestroyed quantity') lessened by 1, being
multiplied by half the com-mon difference, (then) increased by the first term (\textit{mukha}), and (then) multiplied by \,the \,'undestroyed \,quantity', \,should \,be \,subtracted \,from \,the \,sum \,of \,the \,series (\textit{gaṇita}). \,That \,being \,divided \,by \,the \,first \,term \,as \,increased \,by the \,product \,of \,the 'undestroyed quantity' and the common difference and then added to the 'undestroyed quantity', gives the number of terms of the series.

\afterpage{\fancyhead[CO] {\small{GEOMETRIC SERIES}}}

\newpage

{\small That is, if \textit{n} denotes the integral part of~ 
\vspace{3mm}

\hspace{20mm} $\dfrac{\sqrt{2ds + \left(a - \frac{d}{2}\right)^2} - \left(a - \frac{d}{2}\right)}{d}$, 
\vspace{3mm}

\noindent then the number of terms of the series is equal to~ 
\vspace{3mm}

\hspace{20mm} $n + \dfrac{s - \left\lbrace (n - 1)\frac{d}{2} + a\right\rbrace n}{nd + a}$ 
\vspace{3mm}

The rationale of this rule is as follows:~ 
\vspace{3mm}

 Let \textit{N} be the (unknown) number of terms of the series. Then (\textit{vide} Rule 85)
\vspace{3mm}

\hspace{20mm} $s = \left(\dfrac{N - 1}{2}\,d + a\right)N$ 
\vspace{3mm}

\hspace{15mm} or, ~$dN^2 + 2 \left(a - \frac{d}{2}\right)N - 2s = 0$
\vspace{3mm}

giving~ \hspace{10mm} $N = \dfrac{\sqrt{2ds + \left(a - \frac{d}{2}\right)^2} - \left(a - \frac{d}{2}\right)}{d}$,
\vspace{3mm}

Now let \hspace{4mm} $N = n + \frac{p}{q}$. ~Then~ 
\vspace{3mm}

\hspace{10mm} $n =$ integral part of $\dfrac{\sqrt{2ds + \left(a - \frac{d}{2}\right)^2} - \left(a - \frac{d}{2}\right)}{d}$, 
\vspace{3mm}

and $\frac{p}{q}$ is obtained by the formula (\textit{vide} Rule 89)~
\vspace{3mm}

\hspace{20mm} $s = (dn + a + a - d)\frac{n}{2} + (dn + a)\frac{p}{q}$ 
\vspace{3mm}

\hspace{12mm} i.e.,~~ $\dfrac{p}{q} = \dfrac{s - \left\lbrace (n - 1)\frac{d}{2} + a\right\rbrace n}{nd + a}$.}
\vspace{5mm}

\begin{center}\englishfont{(iii) \emph{SERIES IN GEOMETRIC PROGRESSION}}\end{center}

\noindent Rule for obtaining the amount which a given sum increasing in a geometric progression would become after a given period:~ 
\vspace{3mm}

 94. When the number of terms of the series (i.e., the
number denoting the period) is odd, subtract 1 from it
and write 'multiply (by the common ratio)'; and when the
number of terms of the series is even, halve it and write
'square'. (Apply the same rule to the resulting number, and continue the process till the number reduces to zero.)

\newpage

\noindent Having thus written down 'multiply' and 'square' in a
sequence (write 1 thereafter). Then starting from 1
backwards, perform the operations of multiplication and
squaring, and finally multiply the resulting quantity by the
first term of the series (i.e., the given sum).\renewcommand{\thefootnote}{1} \footnote{\en  \textit{Cf. GSS}, ii. 94, vi. 311$\frac{1}{2}$; \textit{MSi}, xv. 52-53(i); \textit{SiŚe}, xiii. 25; \textit{L} (ASS), p. 119, vs. 131; \textit{GK, I}, p. 127, lines 8-11.}
\vspace{3mm}

{\small Given the geometric series to \textit{n} terms~ 
\vspace{3mm}

\hspace{20mm} $a, ar, ar^2, ..., ar^{n-1}$,
\vspace{3mm}

\noindent the object of this rule is to obtain the value of the $(n + 1)^{\textrm{th}}$ term, i.e., $ar^n$.}
\vspace{3mm}

 Ex. 108. Some businessman, taking 3  \textit{rūpas} with him,
went out to make profit. If his capital becomes double after
every month, what will it become after 3 years\,?~ 
\vspace{3mm}

{\small Here the given sum $=$ 3  \textit{rūpas}, increase-ratio $=$ 2, and the period $=$ 3
years, i.e., 36 months.~ 
\vspace{3mm}

 Therefore we have to consider the geometric series, whose first
term $=$ 3, common ratio $=$ 2, and the number of terms $=$ 36.~ 
\vspace{3mm}

 Proceeding according to the rule, we have 

\renewcommand*{\arraystretch}{1.5}
\begin{center}
\begin{tabular}{rrl}
 No. of terms of the series  & Operation & Write\\
36 (even) \hspace{10mm} & $\frac{36}{2} (= 18)$  &  square \\
18 (even) \hspace{10mm} & $\frac{18}{2} (= 9)$ & square\\
9 (odd) \hspace{10mm} & $9 - 1 (= 8)$ &  multiply \\
8 (even) \hspace{10mm} & $\frac{8}{2} (= 4)$ & square\\
4 (even) \hspace{10mm} & $\frac{4}{2} (= 2)$ & square\\
2 (even) \hspace{10mm} & $\frac{2}{2} (= 1)$ & square\\
1 (odd) \hspace{10mm} & $1 - 1 (= 0)$ & multiply\\
\end{tabular}
\end{center}

 Denoting 'square' by \textit{s} and 'multiply' by \textit{m}, and writing them in a
sequence and 1 in the end, we get
\vspace{3mm}

\hspace{1.5cm}\englishfont \textit{s} \hspace{0.5cm}  \textit{s} \hspace{0.5cm}  \textit{m} \hspace{0.5cm}  \textit{s} \hspace{0.5cm}  \textit{s} \hspace{0.5cm}  \textit{s} \hspace{0.5cm}  \textit{m} \hspace{0.5cm} 1}

\afterpage{\fancyhead[CO] {\small{PROBLEMS ON ARITHMETIC SERIES}}}

\newpage

 Proceeding from 1 backwards, and performing the operations of
'multiplication (by the common ratio 2)' and 'squaring', we
successively get~
\vspace{1mm}

\renewcommand*{\arraystretch}{1.2}
\begin{center}
\begin{tabular}{c c c c c c c}
 \textit{s} & \textit{s} & \textit{m} & \textit{s} & \textit{s} & \textit{s} & 2\\

 \textit{s} & \textit{s} & \textit{m} & \textit{s} & \textit{s} & $2^2$ \\

 \textit{s} & \textit{s} & \textit{m} & \textit{s} & $2^4$ \\

 \textit{s} & \textit{s} & \textit{m} & $2^8$ \\

 \textit{s} & \textit{s} & $2^9$ \\

 \textit{s} & $2^{18}$ \\

 $2^{36}$
\end{tabular}
\end{center}
\renewcommand*{\arraystretch}{0.7}
\vspace{1mm}

{\small Finally multiplying $2^{36}$ by the first term of the series, we get $3 \times 2^{36}$, i.e., 206158430208.}
\vspace{4mm}

\noindent Rule for obtaining the sum of a series in G.P. when the first term, common ratio, and number of terms are known:~ 
\vspace{3mm}

 95(i). The result obtained according to the previous
rule, being diminished by the first term of the series, and (then) divided by the common ratio minus 1, gives the sum of the series.\renewcommand{\thefootnote}{1} \footnote{\en  \textit{Cf. GSS}, ii. 94, vi. 311$\frac{1}{2}$; \textit{MSi}, xv. 53(ii); \textit{L} (ASS), pp. 119-120, vs. 130; \textit{GK, I}, p. 127, lines 8-11.}
\vspace{3mm}

 That is,  
\vspace{1mm}

\hspace{20mm} $a + ar + {ar}^{2} + ...$ to \textit{n} terms $= \dfrac{{ar}^n - a}{r - 1}, ~r \textgreater 1$.
\vspace{4mm}

 Ex. 109. One man gets 3, and the other men in
succession get in the ratio of 2. Quickly say, how much
money will (the first) five men get. 
\vspace{3mm}

 \begin{center} \englishfont{(iv) \emph{MISCELLANEOUS PROBLEMS ON SERIES IN \\ARITHMETIC PROGRESSION}}
\end{center}

\noindent Rule for finding the sum of the prices of a number of bangles, which are in an arithmetic progression, when the prices of the first and last bangles are known:

\newpage

 95(ii). The multiplication of the number of bangles by
half the sum of prices of the first and last bangles, gives the
price (of all the bangles).~ 
\vspace{3mm}

{\small This rule is evidently based on the following formula for the sum of
an arithmetic series: 
\vspace{-0.01mm}

 Sum of an arithmetic series $= \dfrac{(\textrm{first term} + \textrm{last term})}{2}\, \times\, $(no. of
terms).}
\vspace{4mm}

 Ex. 110. The first bangle is obtained for 8  \textit{paṇas}, and
the last bangle for 13  \textit{paṇas}. If the total number of the
bangles be 24, say what is the price of all of them.\renewcommand{\thefootnote}{1}\footnote{\hspace{-2mm} \en Similar examples are found to occur in Bhāskara I's comm. on \textit{Ā}, ii. 19.\\}
\vspace{3mm}

{\small The first bangle means the smallest bangle which lies at the wrist
ahead of all the other bangles, and the last bangle means the biggest
bangle which lies behind all the other bangles.}
\vspace{-0.01mm}

Rule for finding the time in which two persons, one travelling with a constant speed and the other accelerating his initial speed by a given quantity per day, would traverse the same distance:~ 
\vspace{3mm}

 96. Having \,subtracted \,the \,initial \,speed \,from \,the \,constant \,speed, \,divide \,the remainder as multiplied by 2 by the acceleration in velocity. The quotient plus 1 gives the time elapsed (in term of days) when the distances traversed are equal.~~~~\renewcommand{\thefootnote}{\hspace{-4.5mm} 2}\footnote{\hspace{-2mm} \en \textit{Cf. BM, III}, B1, 8 recto; \textit{GSS}, vi. 319(i); \textit{GK, I}, p. 112, lines 4-5.\\}
\vspace{3mm}

{\small Suppose that one person travels with a constant speed \textit{u} per day, while the other starts with speed \textit{v} per day on the first day and then accelerates his speed by \textit{f} per day per day. Also suppose that the two persons traverse the same distance in \textit{n} days. Then 
\vspace{3mm}

\hspace{20mm}  $un = \dfrac{n}{2}[2v + (n - 1)f]$. \vspace{3mm}

Hence~ 
\vspace{1mm}

\hspace{20mm} $n = \dfrac{2(u - v)}{f} + 1$.}
\vspace{4mm}

\noindent Ex. 111. One man goes with initial speed 3 (\textit{yojanas}) per day and acceleration 1 (\textit{yojana}) per day per day, and another man goes with the (constant) speed of 10  \textit{yojanas} per
day. In what time will they cover the same distance?~\,~~~\renewcommand{\thefootnote}{\hspace{-4.5mm} 3}\footnote{\hspace{-2mm} \en \textit{Cf. BM, III}, B3, 7 verso; B1, 8 recto; \textit{GSS}, vi. 320.}

\newpage

\noindent Rule for finding the number of days elapsed when two
travellers, starting at a specified interval of time, meet each
other for the second time after their first meeting:~ 
\vspace{3mm}

 97-98. In relation to the first traveller, assume an
arbitrary number, greater than the  \textit{caya} (i.e., acceleration)
for the second traveller, for the   \textit{caya} (i.e., acceleration); and
another arbitrary number (\textit{iṣṭa}) for the  \textit{mukha} (i.e., initial
speed). In relation to the second traveller, assume another
arbitrary number to denote the  \textit{pada} (i.e., the number of days
elapsed at the first meeting); and from the corresponding
 \textit{pada} (i.e., the number of days elapsed at the first meeting) for
the first traveller, calculate the  \textit{ādi} (i.e., the initial speed) for
the second traveller.
\vspace{3mm}

 Now divide the  \textit{phala} (i.e., the distance travelled by each
traveller at their first meeting) severally by the  \textit{padas} (for the
two travellers) and take the difference of the two; diminish
that (difference) by half the difference of  \textit{caya} $\times$ \textit{pada} for
the two travellers; and then divide that by half the
difference between the  \textit{cayas} (for the two travellers): the quotient
gives the days elapsed at the second meeting of the two
travellers) (since the first meeting).\renewcommand{\thefootnote}{1} \footnote{\en  \textit{Cf.} \textit{GK, I}, p. 110, lines 9-14, and p. 111, lines 1-2.}
\vspace{3mm}

{\small That is to say, the number of days (\textit{D}) elapsed at the second meeting of the two travellers since their first meeting is given by~
\vspace{3mm}

\hspace{20mm} $D = \dfrac{\left(\dfrac{s}{n - d} - \dfrac{s}{n}\right) - \frac{1}{2}\{n{f_1} - (n - d){f_2}\}}
{\frac{1}{2}(f_1 - f_2)} $
\vspace{3mm}

\noindent where $f_1$ and $f_2$ are the accelerations of the two travellers, \textit{n} is the
number of days elapsed at the first meeting since the start of the first traveller, $(n - d)$ is the number of days elapsed at the first meeting since the
start of the second traveller, and \textit{s} is the distance travelled by the two travellers at
their first meeting.}

\newpage

{\small  Let the initial speeds, etc., of the two travellers be as follows:
\vspace{1mm}

\renewcommand*{\arraystretch}{1}
\begin{center}
\begin{tabular}{c c c c c}
{} & Initial speed &
 Acceleration &
 \multicolumn{2}{c}{No. of days elapsed since start} \\
{} & {} & {} & at the first &  at the second \\
{} & {} & {} & meeting &  meeting\\
I  & $v_1$ & $f_1$ & $n$ & $n + D$\\
II &$ v_2$ & $f_2$ & $n - d$ & $n - d + D$\\
\end{tabular}
\end{center}
\renewcommand*{\arraystretch}{0.7}
\vspace{1mm}
 
 Then \textit{s}, the distance travelled by the travellers at the first meeting,
and $s_1$, the distance travelled by them at the second meeting, are given by~ 
\vspace{3mm}

$s = \dfrac{n}{2}\,[2{v_1} + (n - 1){f_1}] = \dfrac{n - d}{2}\,[2{v_2} + (n - d - 1) {f_2}],$ \hfill (1) 
\vspace{3mm}

${s_1} = \dfrac{n + D}{2}\,[2{v_1} + (n + D - 1){f_1}] = \dfrac{n- d + D}{2}\,[2{v_2} + (n - d + D - 1){f_2}]$ \hfill (2)
\vspace{4mm}

  Using (1), (2) can be written as~ 
\vspace{4mm}

\hspace{10mm} $\dfrac{n+D}{2}\left[\dfrac{2s}{n} + D{f_1}\right] = \dfrac{n - d + D}{2}\left[\dfrac{2s}{n-d} + D{f_2}\right]$,
\vspace{3mm}

whence 
\vspace{3mm}

\hspace{10mm} $D = \dfrac{\left(\dfrac{s}{n-d} - \dfrac{s}{n}\right) - \frac{1}{2}[n{f_1} - (n - d){f_2}]}{\frac{1}{2}(f1 - f2)}$}
\vspace{5mm}

  Ex. 112. After one man had travelled for 6 days with
some (unknown) initial speed (\textit{ādi}) and acceleration (\textit{uttara}),
another man went by the same track with (an unknown
initial speed and) acceleration 2 per day per day. Say how will they meet each other two times (on the way).\renewcommand{\thefootnote}{1} \footnote{\en   \textit{Cf. GK, I}, p. 111, lines 4-7.}
\vspace{4mm}

\noindent Rule for finding the amount by which one gambler defeats
his opponent in a play with dice, when the moneys staked at the successive casts of dice are in an arithmetic progression:~ 
\vspace{3mm}

 99-101. Diminish \,the \,first \,\textit{pada} \,(i.e., \,the \,number \,of \,casts \,of \,dice \,won \,in \,the beginning by either of the two
persons) by \textit{one}; (taking the remainder as the number
of terms) find the sum of the series whose first term (\textit{ādi})
and common difference (\textit{caya}) are each \textit{one}: this is the  \textit{vṛddhi}
(for the first  \textit{pada}). In regard to the other  \textit{padas} (i.e., the
number of casts of dice won subsequently), take the sum of the preceding  \textit{padas} for the first term (\textit{prabhava}), \textit{one} for the common difference (and the \textit{padas} for the number of terms, and find the sums of the series. These will give the \textit{vṛddhis} for those \textit{padas}).

\newpage

 Now taking the sum of all the  \textit{padas} (for the    \textit{pada}), find
the sum as in the case of the first  \textit{pada}; then diminish that
(sum) by twice the   \textit{vṛddhis} corresponding to the   \textit{padas} of the
lesser group (i.e. the   \textit{padas} corresponding to the person who
wins lesser number of casts); then multiply (the remainder)
by the (given) common difference; and then add that (product) to the product of the (given) first term (\textit{ādi}) and the
difference of the   \textit{padas} (of the greater and lesser groups):
this gives the amount by which the person with greater
number of   \textit{padas} (i.e., casts of dice in his favour) is victorious.
If that quantity be negative, then it gives the amount by
which the person with lesser number of   \textit{padas} is victorious.
\vspace{3mm}

{\small Suppose that two persons \textit{A} and \textit{B} gamble with dice, and that they alternately win $p_1, p_2, p_3$ and $p_4$ casts. If the stake-moneys of the casts be in the arithmetic progression
\vspace{-1mm}

\hspace{20mm} $a, a + d, a + 2d, ... ,$
\vspace{3mm}

\noindent then the amount won by \textit{A}~ 
\vspace{3mm}

\hspace{10mm} $= [a + (a + d) + (a + 2d) + ...$ to $p_1$ terms]
\vspace{3mm}

\hspace{20mm} $+\,[\{a + (p_1 + p_2)\,d\} + \{a + (p_1 + p_2 + 1) d\} + ... $ to $p_3$ terms].
\vspace{3mm}

\hspace{10mm} $= \dfrac{p_1}{2}\,[\,2a + (p_1 - 1)\,d\,] + \dfrac{p_3}{2}\,[\,2\,({a + \overline{p_1 + p_2}d)} + (p_3 - 1)\,d\,]$,
\vspace{3mm}

\noindent and the amount won by \textit{B}
\vspace{3mm}

\hspace{5mm} $= [\{a + p_1d\} + \{a + (p_1 + 1)d\} + \{a + (p_1 + 2)d\} + ....$ to $p_2$ terms]~ 
\vspace{3mm}

\hspace{10mm} $+\,[\{a + (p_1 + p_2 + p_3) d\} + \{a + (p_1 + p_2 + p_3 + 1)d\} + ....$ to $p_4$ terms] 
\vspace{3mm}

\hspace{5mm} $= \dfrac{p_2}{2}\,[2\,(a + p_1d) + (p_2 - 1) d\,] + \dfrac{p_4}{2}\,[2\,\{a + (p_1 + p_2 + p_3)d\,\} + (p_4 - 1)d\,]$.
\vspace{3mm}

Suppose that $p_1 + p_3 \textgreater p_2 + p_4$, then
the person with greater number
of casts in his favour, viz. \textit{A}, is victorious by the amount~ 
\vspace{3mm}

\hspace{5mm} $= \dfrac{p_1}{2}\,[2a + (p_1 - 1)d\,] + \dfrac{p_3}{2} [2(a + \overline{p_1 + p_2}d) + (p_3 - 1)d\,]$ 
\vspace{3mm}

\hspace{10mm} $- \dfrac{p_2}{2}\,[\,2(a + p_1d) + (p_2 - 1)d\,] - \dfrac{p_4}{2}\,[\,2(a + \overline{p_1 + p_2 + p_3}d) + (p_4 - 1)d\,]$
\vspace{3mm}

\hspace{5mm} $= a (p_1 + p_3 - p_2 - p_4) + d\,\left[\,\frac{1}{2}\,(p_1 + p_2 + p_3 + p_4 - 1)(p_1 + p_2 + p_3 + p_4)\right. $
\vspace{3mm}

\hspace{10mm} $\left. - 2\left\lbrace \dfrac{p_2}{2} (2p_1 + p_2 - 1) + \dfrac{p_4}{2} (2(p_1 + p_2
+ p_3) +p_4 - 1)\right\rbrace\, \right].$
\vspace{3mm}

Hence the rule.}

\newpage

 Ex. 113. In a gamble two (persons) alternately won
30, 10, 100 and 8 casts of dice (with stake-moneys) beginning with 9 and increasing (successively) by 6. Say who is the winner.~
\vspace{3mm}

 Ex. 114. If the casts of dice (alternately won by the two persons) be 7, 3, 9 and 12, and the first term and common difference (of the series formed by the stake-moneys) as stated before, then say after calculation who wins, if you know (the method).~ 
\vspace{3mm}

  Ex. 115. In a gamble two (persons) alternately won
from each other 4, 3, 2 and 2 casts of dice (with stake-moneys) beginning with 1 and increasing (successively) by 6. Say who is the winner.
\vspace{3mm}

 \begin{center} \englishfont{(v) \emph{SERIES OF SQUARES, CUBES, AND SUCCESSIVE SUMS OF}\\

 \emph{NATURAL NUMBERS}}
\end{center}
\vspace{1mm} 

\noindent Rule for finding the sum of\, $\smashoperator[r]{\sum_{r=1}^{r=n}} r + n^2 + n^3$\, and of\, $\smashoperator[r]{\sum_{r=1}^{r=n}} r^2$\,:
\vspace{3mm}

 102. The number of terms plus one, as multiplied by
twice the number of terms plus one, being (further) multiplied by half the number of terms, gives the sum of (i) the sum of a series of natural numbers (from 1 up to the given number of
terms), (ii) the square of the number of terms, and (iii) the cube of the number of terms.~ 
\vspace{3mm}

 That divided by 3 gives the sum of a series of squares
of natural numbers.
\vspace{3mm}

{\small That is,  
\vspace{1mm}

\hspace{10mm} (1)\renewcommand{\thefootnote}{1}\footnote{\hspace{-2mm} \en For an alternative formula, see \textit{GSS}, vi. 296; \textit{GK, I}, p. 116,
lines 3-4.\\} \hspace{5mm} $\smashoperator[r]{\sum_{r=1}^{r=n}} r + n^2 + n^3 = \dfrac{(2n + 1)(n + 1)n}{2}$,
\vspace{3mm}

\hspace{10mm} (2)~~~\,~\renewcommand{\thefootnote}{\hspace{-4.5mm} 2}\footnote{\hspace{-2mm} \en \textit{Cf. Ā}, ii. 22; \textit{BrSpSi}, xii. 20(i); \textit{SiŚe}, xiii. 22(i); \textit{L}
(ASS), \textit{I}, p. 113, vs. 19(i); \textit{GK, I}, p. 117, lines 1-2.} \hspace{5mm} $\smashoperator[r]{\sum_{r=1}^{r=n}} r^2 = \dfrac{(2n + 1)(n + 1)n}{(2 \times 3)}$.}

\afterpage{\fancyhead[CO] {\small{SERIES OF SQUARES, CUBES AND SUCCESSIVE SUMS}}}

\newpage

 Ex. 116(i). Say what is the sum of (i) the sum of 
the first five natural numbers, (ii) the square of 5, and (iii) the cube of 5.
\vspace{3mm}

 Ex. 116(ii). Also (say), if you know, the sum of the
squares of 5 terms of the series whose first term and common difference are each unity. 
\vspace{5mm}

Rule for finding the sum of\, $\smashoperator[r]{\sum_{r=1}^{r=n}} r^3$\, and of\, $\smashoperator[r]{\sum_{m=1}^{m=n}}\; \smashoperator[r]{\sum_{r=1}^{r=m}} r$\,:
\vspace{3mm}

 103. One-half of what is obtained by adding the
number of terms to the square of the number of terms, when multiplied by itself, gives the sum of the cubes of natural numbers (from 1 up to the given number of terms); and when multiplied by the number of terms plus 2 and divided by 3, gives the sum of the successive sums of those natural numbers.~ 
\vspace{3mm}

{\small That is,
\vspace{1mm}

\hspace{10mm} (1)\renewcommand{\thefootnote}{1}\footnote{\hspace{-2mm} \en \textit{Cf. Ā}, ii. 22(ii);  \textit{BrSpSi}, xii. 20(ii); \textit{GSS}, vi. 301; \textit{SiŚe}, xiii, 22(ii); \textit{L} (ASS), \textit{I}, p. 113, vs. 119(ii); \textit{GK, I}, p. 117, lines 3-4.\\} \hspace{5mm} $\smashoperator[r]{\sum_{r=1}^{r=n}} r^3 = \left\lbrace \dfrac{n^2 + n}{2}\right\rbrace^2$,
\vspace{3mm}

\hspace{10mm} (2)~~~\,~\renewcommand{\thefootnote}{\hspace{-4.5mm} 2}\footnote{\hspace{-2mm} \en \textit{Cf. Ā}, ii. 21; \textit{BrSpSi}, xii. 19; \textit{SiŚe}, xiii. 21; \textit{L} (ASS), \textit{I}, p. 112, vs. 117.} \hspace{5mm} $\smashoperator[r]{\sum_{m=1}^{m=n}}\; \smashoperator[r]{\sum_{r=1}^{r=m}} r = \dfrac{\left(\dfrac{n^2 + n}{2}\right)(n + 2)}{3}$}
\vspace{4mm}

 Ex. 117. Friend, quickly say what is the sum of the
cubes of 10 terms of the series whose first term and common difference are each unity; and also the sum of the successive sums of those terms.
\vspace{4mm}

\noindent Rule for finding the sum of\, $\smashoperator[r]{\sum_{m=1}^{m=n}}\; \smashoperator[r]{\sum_{r=1}^{r=m}} r + \smashoperator[r]{\sum_{r=1}^{r=n}} r^2 + \smashoperator[r]{\sum_{r=1}^{r=n}} r^3$\,:
\vspace{3mm}

104. The number of terms, as multiplied by the square
of (the sum of) the number of terms plus \textit{one}, when (further)
multiplied by the number of terms plus two and divided by four, gives the sum of (i) the sum of successive sums of natural
numbers (from 1 up to the given number of terms), (ii) the sum of squares of those natural numbers, and (iii) the sum of cubes of those natural numbers.~ 

\newpage

{\small That is,
\vspace{3mm}

\hspace{15mm} $\smashoperator[r]{\sum_{m=1}^{m=n}}\; \smashoperator[r]{\sum_{r=1}^{r=m}} r + \smashoperator[r]{\sum_{r=1}^{r=n}} r^2 + \smashoperator[r]{\sum_{r=1}^{r=n}} r^3 = \dfrac{n\,(n + 1)^2(n + 2)}{4}$ .}
\vspace{4mm}

 Ex. 118. Friend, if you know then say after calculation
the sum of (i) the sum of successive sums of the first 6 natural numbers, (ii) the sum of the squares of the first 6 natural
numbers, and (iii) the sum of the cubes of the first 6 natural numbers.
\vspace{3mm}

 \begin{center}  \englishfont{(vi) \emph{SERIES OF SQUARES AND CUBES, ETC. OF THE TERMS~} \\
 \emph{OF AN ARITHMETIC SERIES}}
\end{center}
 
\noindent Rule for finding the sum of the squares of the terms of the series $\smashoperator[r]{\sum_{r=1}^{r=n}} \{a + (r - 1)d\}^2$\,:
\vspace{3mm}

 105. The sum of the (arithmetic) series with
twice the common difference, when multiplied by the first term and (then) increased by the sum of the squares of the
natural numbers ranging from 1 to one-less the number of
terms, as multiplied by the square of the common difference,
gives the sum of the squares of the terms of the (given
arithmetic) series.\renewcommand{\thefootnote}{1} \footnote{\en The same rule occurs in \textit{GK, I}, p. 119, lines 7-8, and p. 120,
lines 1-2. For further rules, see \textit{GSS}, vi. 298, 299.}
\vspace{3mm}

{\small That is,  
\vspace{1mm}

\hspace{15mm} $\smashoperator[r]{\sum_{r=1}^{r=n}} \{a + (r - 1)d\}^2$
\vspace{3mm}

\hspace{20mm} $= [a + (a + 2d) + (a + 4d) + ...$ to $n$ terms] $\times a$ 
\vspace{3mm}

\hspace{25mm} $+ [1^2 + 2^2 + 3^2 + ... + (n - 1)^2] \times d^2$}
\vspace{3mm}

 Ex. 119. Tell me the sum of the squares of (the first)
six terms of the (arithmetic) series whose first term is two and common difference three. 
\vspace{4mm}

\noindent Rule for finding the sum of the sums of series of natural
numbers each beginning with 1, whose number of terms are the terms of a given arithmetic series:
\vspace{3mm}

 106. Having obtained the sum of the squares of the
terms of the (arithmetic) series with (given) first term, common difference and number of terms, as before, add to it the sum of the same (arithmetic) series,

\newpage 

\noindent  and (then) reduce that (resulting sum) to half: the result (thus obtained) is the
sum of sums of the natural series (beginning with 1 and ending with the terms of the given arithmetic series).\renewcommand{\thefootnote}{1}\footnote{\hspace{-2mm} \en For other rules, see \textit{GSS}, vi. 305-305$\frac{1}{2}$ and \textit{GK, I}, p. 117, lines 11-16.\\}
\vspace{3mm}

{\small That is to say, 
\vspace{1mm}

\hspace{10mm} ($1 + 2 + 3 + ...$ to $a$ terms) $+$ ($1 + 2 + 3 +$ ... to $a + d$ terms) $+$
\vspace{3mm}

\hspace{15mm} ($1 + 2 + 3 +$ ... to $a + 2d$ terms) $+$ ... to $n$ terms.
\vspace{3mm}

\hspace{10mm} $\equiv \smashoperator[r]{\sum_{r=1}^{r=n}} \dfrac{\{a + (r - 1)d\} \{\overline{a + (r - 1)d} + 1\}}{2}$
\vspace{3mm}

\hspace{20mm} $= \dfrac{1}{2} \left[\,\smashoperator[r]{\sum_{r=1}^{r=n}} (a + \overline{(r - 1)d})^2 + \smashoperator[r]{\sum_{r=1}^{r=n}} (a + \overline{(r - 1)d})\,\right]$}
\vspace{5mm}

 Ex. 120. O the best of mathematicians, say the sum of
the sums of series of natural numbers (each beginning with 1), whose number of terms are the first six terms of the arithmetic series with 3 as the first term and 5 as the common difference. 
\vspace{-0.01mm}

\noindent Rule for finding the sum of cubes of the terms of an arithmetic series:~
\vspace{3mm}

 107. To the square of the sum of the (given arithmetic)
series, as multiplied by the common difference, add the
product of the first term and the sum of the series, as
multiplied by the first term minus the common difference: the result is the sum of cubes of the terms of the (arithmetic) series with given first term and common difference.~~~~\renewcommand{\thefootnote}{\hspace{-4.5mm} 2}\footnote{\hspace{-2mm} \en \textit{Cf. GSS}, vi. 30; \textit{GK, I}, p. 121, lines 2-3, and p. 122, lines 1-2.}
\vspace{3mm}

{\small That is,  
\vspace{1mm}

\hspace{10mm} $a^3 + (a + d)^3 + (a + 2d)^3 + ... + \{a + (n - 1)d\}^3$
\vspace{3mm}

\hspace{15mm} $\equiv \smashoperator[r]{\sum_{r=1}^{r=n}} \{a + (r - 1)d\}^3$
\vspace{2mm}

\hspace{20mm} $= S^2 \times d + aS \times (a - d)$,
\vspace{3mm}

where, \hspace{5mm} $S = \smashoperator[r]{\sum_{r=1}^{r=n}} \{a + (r - 1)d\}$}
\vspace{5mm}

  Ex. 121. Say after adding together the cubes of the
four terms, which begin with 5 and increase successively by 2.

\newpage

\phantomsection \label{plane}
\begin{center} 
\textbf{(3) Determinations pertaining to plane figures} \vspace{1mm}

(\textit{Kṣetra-vyavahāra})
\vspace{2mm}

\englishfont{(i) \emph{INTRODUCTION}}
\end{center}
\vspace{1mm}

\noindent Fundamental property of rectilinear figures:~ 
\vspace{3mm}

 108. (In rectilinear figures) the sum of all sides except one, is neither equal to nor less than the side excepted, because a curved path is neither less than nor equal to the straight path.~ 
\vspace{3mm}

{\small What the author means here to say is that a rectilinear figure, such
as a quadrilateral or a triangle, is possible only when each of its
sides is less than the sum of the remaining sides.~ 
\vspace{2mm}

 Nārāyaṇa has made this statement more clearly:~ 
\vspace{2mm}

{\qt "When in a rectilinear figure, the sum of the other sides is less than or equal to the greatest side, it is impossible."}\renewcommand{\thefootnote}{1}\footnote{\hspace{-2mm} \en \textit{GK, II}, p. 48, lines 8-11. So also says Bhāskara II. See \textit{L} (ASS), pp. 151-152, vs. 163.\\}
\vspace{2mm}

Āryabhaṭa II states the same condition as follows:~
\vspace{2mm}

{\qt "When (in a rectilinear figure) each side is less than half the sum
of all the sides, it is possible; when (on the other hard) half the sum
of all the sides is less than (or equal to) any one of the sides, it is
impossible."}~\,~~~\renewcommand{\thefootnote}{\hspace{-4.5mm} 2}\footnote{\hspace{-2mm} \en \textit{MSi}, xv. 64.\\}}
\vspace{4mm}

\noindent Condition for the existence of 'the perpendiculars dropped from the vertices to the base' and 'the segments into which the base is divided by them' in the case of a quadrilateral:~ 
\vspace{3mm}

 109. Only in those (quadrilateral) figures in which the
square of base minus face exceeds the product of the difference and sum of the flank sides, do 'the perpendiculars dropped from the vertices to the base' (\textit{lamba}) and 'the segments into which the base is divided by them' (\textit{abadhā}) exist.
\vspace{3mm}

{\small Nārāyaṇa has criticized this rule.~~~~\renewcommand{\thefootnote}{\hspace{-4.5mm} 3}\footnote{\hspace{-2mm} \en \textit{GK, II}, p. 49.}  His criticism, however, is not well-grounded as it is based on an impossible quadrilateral.}

\afterpage{\fancyhead[CO] {\small{PLANE FIGURES}}}

\newpage

\noindent Enumeration of primary plane figures:~
\vspace{3mm}

 110-111. The rectangular quadrilateral, the equilateral
quadrilateral, the equi-bilateral quadrilateral, the
equi-trilateral quadrilateral, the inequilateral quadrilateral, the
equi-lateral triangle, the scalene triangle, the isosceles triangle, the
circle, and the segment of a circle\textendash \,these are the ten (primary)
plane figures; the areas of these (figures) should be determined
by applying their own rules. And by considering (the shape
in terms of these (figures) should be obtained the areas of other figures, such as (those of the shape of) an elephant's
tusk, a felloe, etc.~
\vspace{3mm}

{\small Nārāyaṇa, too, like Śrīdhara takes the primary plane figures to be 10 in number. But his list of primary plane figures differs from that of
Śrīdhara in so far as "the segment of a circle" has been replaced by him by "the conch-figure."
\vspace{3mm}

 Mahāvīra gives the following list of 16 primary plane figures:
\vspace{1mm}

\renewcommand*{\arraystretch}{1.4}
\hspace{-5mm} \begin{tabular}{p{0.07\textwidth} p{0.87\textwidth}}
(1-3) & Three varieties of triangles, equilateral, isosceles, and scalene;\\
(4-8) & Five varieties of quadrilaterals, equilateral, equidichastic (\textit{dvi-dvi-sama}), equibilateral, equitrilateral, and inequilateral; \\
(9-16) & Eight varieties of curvilinear figures, a circle, a semi-circle, an ellipse, a conchiform area, a concave circle, a convex circle, an
outlying annulus, and an in-lying annulus, 
\end{tabular}
\renewcommand*{\arraystretch}{0.7}
\vspace{1mm}

 Mahāvīra's classification of quadrilaterals deserves special attention, as it differs from that of Śrīdhara. Śrīdhara's commentator has
referred to this classification, and has tried to justify Śrīdhara's classification.}
\vspace{4mm}

\noindent On a formula for the area of a triangle or a quadrilateral:~ 
\vspace{3mm}

 112-114. (It is said that) the product of half the sums
of the sides and counter sides (i.e., the product of half the sum of the base and face and half the sum of the flank sides) of a triangle or a quadrilateral, gives the gross value of the
area.\renewcommand{\thefootnote}{1}\footnote{\hspace{-2mm} \en Reference is evidently to \textit{BrSpSi}, xii. 21(i) where this rule is given.
This rules occurs also in \textit{GSS}, vii. 7(i), \textit{MSi}, xv. 66. and \textit{GK, II}, p. 3,
vs. 8.} But this result is true only for those figures in which the difference between the altitude and the flank sides is small.
In the case of other figures the above result is far removed from the truth; 

\newpage

\noindent as for example, in the case of the triangle having 13
for the two (flank) sides and 24 for the base, the gross area is
156, whereas the correct area is 60.~ 
\vspace{3mm}

 I shall therefore state the methods for obtaining the
accurate results only.
\vspace{3mm}

\begin{center}  
\englishfont{(ii) \emph{AREA OF THE QUADRILATERAL WITH EQUAL} \\
\emph{ALTITUDES AND OF THE TRIANGLE}}
\end{center}
 
\noindent Rule for finding the area of a quadrilateral in which the perpendiculars dropped from the vertices to the base are equal, and that of a triangle:~ 
\vspace{3mm}

 115. In the case of a quadrilateral, in which the perpendiculars dropped from the vertices to the base are equal, and a triangle, half the sum of the base and the face, multiplied by the altitude, gives the area.\renewcommand{\thefootnote}{1} \footnote{\en The same rule is given in \textit{MSi}, xv. 78; \textit{SiŚe}, xiii. 30; \textit{L} (ASS),
p. 162, line 3; \textit{GK, II}, p. 42, lines 10-11; and page 50, lines 13-14.}
\vspace{3mm}

{\small That is, 
\vspace{2mm}

\hspace{10mm} (i) area of a quadrilateral with parallel base and face~ 
\vspace{2mm}

\hspace{20mm} $= \dfrac{(\textrm{base} + \textrm{face})}{2} \times$ altitude; 
\vspace{3mm}

\hspace{10mm}  (ii) area of a triangle = $\frac{1}{2} \times$ base $\times$ altitude.}
\vspace{4mm}

 Ex. 122. In an equilateral quadrilateral, the face, the
base, and the altitude are all equal to the flank sides, each
being 1$\frac{1}{2}$ cubits in length. Say, friend, what is the area of
that (quadrilateral).~ 
\vspace{3mm}

 Ex. 123. Give out the area of that rectangular quadrilateral in which the base and face are each 5$\frac{1}{2}$ cubits, and the
flank sides and altitude each 3 cubits.~
\vspace{3mm}

 Ex. 124. In a triangle the (flank) sides are $4 - \frac{1}{4}$ and 3$\frac{1}{4}$ cubits, the base is 3$\frac{1}{2}$ cubits, and the altitude is 3 cubits.
What is the area of that\,?~ 
\vspace{3mm}

 Ex. 125. In an equilateral triangle the base is 8$\frac{1}{2}$ cubits, and the altitude is 7 cubits and 8$\frac{2}{3}$ \textit{aṅgulas}. What is the area
thereof\,?

\newpage

 Ex. 126. If you know the method of finding the area
of plane figures, say the area of the isosceles triangle, whose (flank) sides are (each) 5 cubits, altitude is 3 cubits, and base 8 cubits.~
\vspace{3mm}

 Ex. 127. In a quadrilateral, the face is 1$\frac{1}{3}$ cubits, the base is 9$\frac{1}{3}$ cubits, the (flank) sides are (each) 5 cubits, and
the altitude is 3 cubits. What is the area of that?~ 
\vspace{3mm}

 Ex. 128. Say what will be the area of the (quadrilateral) figure, whose base is 39 (cubits), (flank) sides and face
are (each) 25 (cubits), and altitude is 24 (cubits).~ 
\vspace{3mm}

 Ex. 129-30. In an inequilateral quadrilateral with equal altitudes, the base is 10 cubits, the face is 4$\frac{1}{6}$ (cubits), the flank sides are $9 - \frac{1}{3}$ and $6 + \frac{1}{2}$ (cubits), and the altitude is 6$\frac{1}{2}$ cubits minus $\frac{1}{60}$ of an  \textit{aṅgula}. What is the area thereof? 
\vspace{4mm}

\noindent Instruction regarding plane figures of the shape of an elephant's tusk, a felloe, a crescent moon, and a thunderbolt:~ 
\vspace{3mm}

 116. A figure of the shape of an elephant's tusk (may
be considered) as a triangle, of a felloe as a quadrilateral, of a crescent moon as two triangles, and of a thunderbolt as two quadrilaterals.\renewcommand{\thefootnote}{1} \footnote{\en Similar statements are made in \textit{MSi}, xv. 101 and \textit{GK, II}, p. 10, lines 2-4.}
\vspace{3mm}

 Ex. 131. What is the area of (the figure of the shape of) an elephant's tusk whose base is 2 cubits and altitude 3 cubits; and also of the figure of the shape of a felloe whose base and
face are (each) 3 cubits and altitude is 10 cubits?
\vspace{3mm}

 Ex. 132. The central length of (a figure of the shape
of) a crescent moon is 8 cubits, and the central width 3
cubits. Treating it as made up of a pair of triangles, quickly say what its area is?
\vspace{3mm}

 Ex. 133. In (a figure of the shape of ) a thunderbolt,
the central length is 10 cubits, the faces are each 5 cubits, and the central width is 2 cubits. What is its area, if it be regarded as made up of two quadrilaterals?~

\newpage

\begin{center}  
\englishfont{(iii) \emph{AREA OF THE QUADRILATERAL WITH UNEQUAL ALTITUDES}}
\end{center}
 
\noindent Rule for finding the area of a quadrilateral with unequal altitudes, when its sides are given:~ 
\vspace{3mm}

 117. Set down half the sum of the (four) sides (of the
quadrilateral) in four places, (then) diminish them
(respectively) by the (four) sides (of the quadrilateral), (then) multiply (the resulting numbers) and take the square root (of the product): this gives the area of quadrilaterals having (two or more) equal sides but unequal altitudes and also of quadrilaterals having unequal sides and unequal altitudes.\renewcommand{\thefootnote}{1} \footnote{\en According \,to \,Brahmagupta \,and \,Mahāvīra, \,this \,rule \,gives \,the \,accurate \,area \,for \,a \,quadrilateral \,in general. See \textit{BrSpSi}, xii. 21(ii); and \textit{GSS}, vii. 50(ii). The same rule occurs also in \textit{SiŚe}, xiii. 28. Āryabhaṭa II has pointed out that this rule gives the accurate area in the case of a triangle only and not in the case of a quadrilateral. See \textit{MSi}, xv. 69. So also write Bhāskara II and Nārāyaṇa. See \textit{L} (ASS), p. 156, vs. 169; and \textit{GK, II}, p. 39, lines 13-14 and p. 40, lines 1-2.}
\vspace{3mm}

{\small That is to say, if $a, b, c, d$ be the sides of a quadrilateral with unequal altitudes, and $s$ half the sum of those sides, then the area of the quadrilateral~ 
\vspace{3mm}

\hspace{15mm} $= \sqrt{(s - a)(s - b)(s - c)(s - d)}$.
\vspace{3mm}

 Our author, like Brahmagupta and other early Hindu
mathematicians, has committed here an error in declaring the above formula as applicable to all quadrilaterals (with unequal altitudes), when in fact
it is applicable to cyclic quadrilaterals only. The formula which is applicable to all quadrilaterals, in general, of which it is only a particular
case, is as follows:
\vspace{3mm}

\hspace{5mm} area of a quadrilateral $= \sqrt{(s - a)(s - b)(s - c)(s - d) - abcd\, \cos^2a}$
\vspace{3mm}

\noindent where $a, b, c, d$ are the sides of the quadrilateral, $s$ is half the sum of those sides, and $a$ is half the sum of a pair of opposite angles of the quadrilateral.~ 
\vspace{3mm}

 The \,earliest \,Hindu \,mathematician \,who \,pointed \,out \,the \,inapplicability \,of \,the \,above formula to a quadrilateral having only its sides
given was Āryabhaṭa II (c. 950 A.D.). He writes:}

\newpage

{\small {\qt "The mathematician who wishes to tell the area or the altitudes of a quadrilateral without knowing a diagonal, is either a fool or a blunderer."}}\renewcommand{\thefootnote}{1}\footnote{\hspace{-2mm} {\s कर्णज्ञानेन विना चतुरश्रे लम्बकं फलं यद्वा~। \\
\vspace*{1mm}

\,वक्तुं वाञ्छति गणको योऽसौ मूर्खः पिशाचो वा~।} {\en (\textit{MS}, xv. 70) \\
\vspace*{1mm}

\,Also see \textit{L} (ASS), pp. 159-160}.\\}
\vspace{4mm}

\noindent Rule for finding an approximate value of the square root of a non-square number:
\vspace{3mm}

 118. Of the non-square number as multiplied by some
large square number, extract the square root, neglecting the remainder; and divide that by the square root of the
multiplier.~~~~\renewcommand{\thefootnote}{\hspace{-4.5mm} 2}\footnote{\hspace{-2mm} \en For similar rules, see \textit{MSi}, xv. 55, or \textit{SiŚe}, xiii. 36; \textit{L} (ASS), p. 132, vs. 140; \textit{GK, II}, p. 33, lines 4-7.}
\vspace{3mm}

{\small That is to say, multiply the non-square number by some large square number (chosen at pleasure); then find out the square root of that product, neglecting the remainder; and then divide that square root by the square root of (the square number, which was taken as) the multiplier.
\vspace{3mm}

 We illustrate this rule by the following two examples:
\vspace{3mm}

 Ex. 1. Show that $\sqrt10 = 3\frac{1}{7}$ approximately.~ 
\vspace{2mm}

 Choose 49 for the square number. Then multiplying 10 by 49, we get 490. Extracting the square root of 490, we get 22 as the square root and 6 as the remainder. Neglecting the remainder, and dividing 22 by the square root of 49, i.e., by 7, we get 3$\frac{1}{7}$. 
\vspace{3mm}

 Ex. 2. Find the square root of 3, correct to 2 decimal places.
\vspace{2mm}

 Choose 1000000 for the square number. Then multipling 3 by 1000000, we get 3000000. Extracting its square root we get 1732 as the square root and 175 as the remainder. Neglecting the remainder, and dividing 1732 by the square root of 1000000, i.e., by 1000, we get
1.732. Hence the square root of 3 correct to 2 decimal places is 1.73.}

\newpage

 It is evident that the larger the square number is chosen, the better will be the approximation. Hence the above rule.
\vspace{3mm}

 In the  \textit{Bakhshālī Manuscript} the following formula is used for finding an approximate value of the square root of a non-square number: 

\begin{center}
$\sqrt{{a}^2 + b} = a + \dfrac{b}{2a} - \dfrac{\left(\dfrac{b}{2a}\right)^2}{2\,\left(a + \dfrac{b}{2a}\right)}$ \; appr.
\vspace{2.5cm}

\rule{5em}{.9pt}
\end{center}

\afterpage{\fancyhead[CO] {\small{}}}
\afterpage{\fancyhead[CE] {\small{}}}

\newpage

\phantomsection \label{ans}
\begin{center}
{\Large \textbf{ANSWERS TO EXAMPLES}}
\end{center}

\begin{sloppypar}

\setstretch{1.2}
\noindent \textbf{1.} 55; 210; 465; 820; 1275; 1830; 2485; 3240; 4095; 5050. 10; 20; 30; 40; 50; 60; 70; 80; 90; 100. \textbf{2.} 4995; 4840; 4585; 4230; 3775; 3220; 2565; 1810; 955; 0. \textbf{3.} 27216; 33152; 483900. \textbf{4.} 1, 4, 9, 16, 25, 36, 49, 64, 81, 625, 1296, 3969, 186624, and 60871204. \textbf{5.} 1; 8; 27; 64; 125; 216; 343; 512; 729; 3375; 16777216; and 8365427. \textbf{6.} 1$\frac{1}{12}$; 11$\frac{1}{4}$. \textbf{7.} $\frac{17}{8}$; $\frac{3}{8}$; $\frac{2}{9}$. \textbf{8.} $\frac{1}{4}$; $\frac{1}{6}$. \textbf{9.} 13$\frac{1}{2}$. \textbf{10.} 3$\frac{3}{4}$; 150$\frac{5}{6}$. \textbf{11.} 2$\frac{1}{2}$; 17$\frac{3}{14}$. \textbf{12.} 6$\frac{1}{4}$; 232$\frac{9}{16}$; $\frac{1}{4}$; $\frac{1}{9}$. \textbf{13.} 421$\frac{7}{8}$; 5132$\frac{61}{64}$; $\frac{1}{64}$; $\frac{1}{27}$. \textbf{14.} 1$\frac{9}{20}$; 5$\frac{1691}{2520}$. \textbf{15.} $\frac{3103}{25200}$. \textbf{16.} 30. \textbf{17.} 5$\frac{17}{60}$. \textbf{18.} 15$\frac{1}{12}$. \textbf{19.} 5$\frac{15}{16}$. \textbf{20.} 12$\frac{11}{12}$. \textbf{21.} 1$\frac{13}{16}$. \textbf{22.} 5$\frac{647}{3200}$ \textit{purāṇas}. \textbf{23.} 4$\frac{23}{48}$. \textbf{24.} 10$\frac{11}{12}$ \textbf{25.} 7$\frac{7}{10}$ \textit{paṇas}; or 4 \textit{purāṇas}, 13 \textit{paṇas}, 2 \textit{kākiṇīs}, and 16 \textit{varāṭakas}. \textbf{26.} 10$\frac{14}{45}$  \textit{palas}; or 10  \textit{palas}, 1 \textit{karṣa}, 3  \textit{māṣas}, and 4$\frac{5}{9}$ \textit{guñjās}. \textbf{27.} 87$\frac{91}{99}$. \textbf{28.} 2 \textit{droṇas}, 1 \textit{āḍhaka}, and 2$\frac{178}{301}$ \textit{prasthas}. \textbf{29.} 3$\frac{153}{160}$ \textit{rūpas}. \textbf{30.} 8 months and 26$\frac{2}{3}$ days. \textbf{31.} 33600 years. \textbf{32.} 9$\frac{343}{373}$ days. \textbf{33.} 19$\frac{21}{41}$ days. \textbf{34.} 26$\frac{2}{3}$ necklaces. \textbf{35.} 250  \textit{suvarṇas}. \textbf{36.} 244  \textit{suvarṇas}, 5  \textit{māṣas}, and 4$\frac{1}{11}$ \textit{guñjās}. \textbf{37.} 450 blankets. \textbf{38.} 122  \textit{suvarṇas}, 8  \textit{māṣas}, and 4$\frac{36}{41}$ \textit{guñjās}. \textbf{39.} Interest = 36. \textbf{40.} 20$\frac{125}{536}$. \textbf{41.} 2362$\frac{1}{2}$. \textbf{42.} 1$\frac{111}{832}$. \textbf{43.} 38$\frac{1}{4}$  \textit{paṇas}; or 2  \textit{purāṇas}, 6  \textit{paṇas}, and 1  \textit{kākiṇī}. \textbf{44.} 60. \textbf{45.} 33$\frac{3}{4}$. \textbf{46.} 49$\frac{7}{9}$. \textbf{47.} 3  \textit{droṇas}, 0  \textit{āḍhaka}, 3  \textit{prasthas}, and 1$\frac{5}{7}$ \textit{kuḍavas}. \textbf{48.} 2 \textit{palas} of long pepper. \textbf{49.} 25 wood-apples. \textbf{50.} 64. \textbf{51.} 200. \textbf{52.} Capital = 60; Interest = 36. \textbf{53.} Capital = 33$\frac{1}{2}$; Interest = 3. \textbf{54.} Capital = 500; Interest = 300; share of surety = 60; share of calculator = 30; and share of scribe = 15. \textbf{55-56.} The debtor is relieved of his debt in 2 months and 21$\frac{1371}{1789}$ days. In this time the rich man gets 109$\frac{39}{1789}$ \textit{rūpas} in all, the profit being 9$\frac{39}{1789}$ \textit{rūpas}. \textbf{57-58.} Principal = 1000; time = 8 months and 3 days; and rate per cent per month = 4. \textbf{59.} Principal = 100; time = 8 months and 17 days; and rate per cent per month = 4$\frac{1}{2}$. \textbf{60(i).} 20 months. \textbf{60(ii).} 7 months, 4 days, 17  \textit{ghaṭikās}, and 8$\frac{4}{7}$ \textit{caṣakas}. \textbf{61.} 11$\frac{4}{31}$ \textit{varṇas}. \textbf{62.} 9$\frac{25}{68}$ \textit{varṇas}. \textbf{63.} 13$\frac{11}{16}$ \textit{varṇas}. \textbf{64.} 16 \textit{māṣas}. \textbf{65.} 11$\frac{3}{4}$ \textit{varṇas}. \textbf{66.} 5  \textit{māṣas}.
\end{sloppypar}

\renewcommand*{\arraystretch}{1.3}
\begin{center}
\begin{tabular}{cccc}
\textbf{67-68.} & \textit{varṇa} & gold of \textit{varṇa} 16 & gold of \textit{varṇa} 10\\
     & 16 & 2  \textit{māṣas} & 0  \textit{māṣas} \\
     & 15$\frac{3}{4}$ & $\frac{23}{12}$~~~ "~~ & $\frac{1}{12}$~~~ "~~\\
     & 15$\frac{1}{2}$ & $\frac{22}{12}$~~~ "~~ & $\frac{2}{12}$ ~~~ "~~\\
     & 15$\frac{1}{4}$ & $\frac{21}{12}$~~~ "~~ & $\frac{3}{12}$~~~ "~~\\
     & 15 & $\frac{20}{12}$~~~ "~~ & $\frac{4}{12}$~~~ "~~ \\
	 & 14$\frac{3}{4}$ & $\frac{19}{12}$~~~ "~~ & $\frac{5}{12}$~~~ "~~\\
     & 14$\frac{1}{2}$ & $\frac{18}{12}$~~~ "~~ & $\frac{6}{12}$ ~~~ "~~\\
     & 14$\frac{1}{4}$ & $\frac{17}{12}$~~~ "~~ & $\frac{7}{12}$~~~ "~~\\
     & ~14 & $\frac{16}{12}$~~~ "~~ & $\frac{8}{12}$~~~ "~~ \\
\end{tabular}
\end{center}

\afterpage{\fancyhead[CO] {\small{ANSWERS}}}
\afterpage{\fancyhead[CE] {\small{ANSWERS}}}

\newpage

\renewcommand*{\arraystretch}{1.3}
\begin{center}
\begin{tabular}{clll}
 \hspace{5mm} & 13$\frac{3}{4}$ \hspace{8mm} & $\frac{15}{12}$~~~ "~~ \hspace{8mm} & $\frac{9}{12}$~~~ "~~\\
     & 13$\frac{1}{2}$ & $\frac{14}{12}$~~~ "~~ & $\frac{10}{12}$ ~~~ "~~\\
     & 13$\frac{1}{4}$ & $\frac{13}{12}$~~~ "~~ & $\frac{11}{12}$~~~ "~~\\
     & ~13 & $\frac{12}{12}$~~~ "~~ & $\frac{12}{12}$~~~ "~~ \\
     & 12$\frac{3}{4}$ & $\frac{11}{12}$~~~ "~~ & $\frac{13}{12}$~~~ "~~\\
     & 12$\frac{1}{2}$ & $\frac{10}{12}$~~~ "~~ & $\frac{14}{12}$ ~~~ "~~\\
     & 12$\frac{1}{4}$ & $\frac{9}{12}$~~~ "~~ & $\frac{15}{12}$~~~ "~~\\
     & ~12 & $\frac{8}{12}$~~~ "~~ & $\frac{16}{12}$~~~ "~~ \\
     & 11$\frac{3}{4}$ & $\frac{7}{12}$~~~ "~~ & $\frac{17}{12}$~~~ "~~\\
     & 11$\frac{1}{2}$ & $\frac{6}{12}$~~~ "~~ & $\frac{18}{12}$ ~~~ "~~\\
     & 11$\frac{1}{4}$ & $\frac{5}{12}$~~~ "~~ & $\frac{19}{12}$~~~ "~~\\
     & ~11 & $\frac{4}{12}$~~~ "~~ & $\frac{20}{12}$~~~ "~~ \\
     & 10$\frac{3}{4}$ & $\frac{3}{12}$~~~ "~~ & $\frac{21}{12}$~~~ "~~\\
     & 10$\frac{1}{2}$ & $\frac{2}{12}$~~~ "~~ & $\frac{22}{12}$ ~~~ "~~\\
     & 10$\frac{1}{4}$ & $\frac{1}{12}$~~~ "~~ & $\frac{23}{12}$~~~ "~~\\
     & ~10 & ~~0~~~ "~~ & ~~2~~~ "~~ \\
\end{tabular}
\end{center}

\begin{sloppypar}

\setstretch{1.2}
\noindent \textbf{69.} One is of weight 2  \textit{māṣas} and  \textit{varṇa} 12; the other is of weight 3 \textit{māṣas} and  \textit{varṇa} 8. \textbf{70.} One is of weight 5 \textit{māṣas} and  \textit{varṇa} 14; the other is of weight 7  \textit{māṣas} and  \textit{varṇa} 10. \textbf{71.} 30  \textit{prasthas}; 45  \textit{prasthas}; 75 \textit{prasthas}; and 60  \textit{prasthas} respectively. \textbf{72.} 900  \textit{prasthas}, 600  \textit{prasthas}, and 200  \textit{prasthas} respectively. \textbf{73-74.} $\frac{49}{32}$ \textit{kuḍavas} of  \textit{mudga} (seeds of \textit{phasolus mungo}) and 49/64  \textit{kuḍavas} of rice, the respective prices being $\frac{63}{32}$ \textit{paṇas} and $\frac{49}{32}$ \textit{paṇas}. \textbf{75.} $\frac{14}{37}$ \textit{palas} of  asafoetida, $\frac{14}{37}$ \textit{palas} of long pepper, and $\frac{14}{37}$ \textit{palas} of dry ginger, the respective prices being $\frac{28}{37}$  \textit{paṇa}, $\frac{7}{37}$ \textit{paṇa}, and $\frac{2}{37}$ \textit{paṇa}. 
\vspace{3mm}

\noindent \textbf{76.} There will be infinite solutions. The commentator obtains two
solutions in the first case, viz., 
\vspace{2mm}

\renewcommand*{\arraystretch}{1}
\begin{tabular}{l} rate of purchase: \\rate of sale: \end{tabular}
\hspace{3mm}
\begin{tabular}{c}17 articles for 1 \\6 articles for 1 \end{tabular} $\bigg\}$\,,
\hspace{3mm}
\begin{tabular}{c}20 articles for 1\\ 7 articles for 1\end{tabular} $\bigg\}$\,,
\vspace{2mm}

\noindent and one solution in the second case, viz.,  
\vspace{2mm}

\begin{tabular}{l} rate of purchase:\\rate of sale:\end{tabular}
\hspace{3mm}
\begin{tabular}{c}240 articles for 1, \\7 articles for 1. \end{tabular} 
\vspace{2mm}

\noindent \textbf{77.} One solution as given by the commentator is: 
\vspace{2mm}

\begin{tabular}{l} rate of purchase:\\rate of sale:\end{tabular}
\begin{tabular}{c}40 articles for 1, \\20 articles for 1. \end{tabular}

\end{sloppypar}

\newpage

\noindent \textbf{78-79.} The following are the 16 valid integral solutions:
\vspace{2mm}

\renewcommand*{\arraystretch}{0.9}
\setlength{\tabcolsep}{4pt}
\hspace{-4mm} \begin{tabular}{lcccccccc}
 & \multicolumn{2}{c}{(1)} & \multicolumn{2}{c}{(2)} & \multicolumn{2}{c}{(3)} & \multicolumn{2}{c}{(4)} \\
 & Number & Price & Number & Price & Number & Price & Number & Price\\
 Pigeons & 15 & \multicolumn{1}{c|}{9} & 55 & \multicolumn{1}{c|}{33} & 5 & \multicolumn{1}{c|}{3} & 30 & 18 \\ 
 Cranes & 28 & \multicolumn{1}{c|}{20} & 21 & \multicolumn{1}{c|}{15} & 56 & \multicolumn{1}{c|}{40} & 21 &  15\\
 Swans & 45  & \multicolumn{1}{c|}{35} & 9 & \multicolumn{1}{c|}{7} & 27 & \multicolumn{1}{c|}{21} & 36 & 28\\
 Peacocks & 12 & \multicolumn{1}{c|}{36} & 15 & \multicolumn{1}{c|}{45} & 12 & \multicolumn{1}{c|}{36} & 13 &  39\\
 &   \multicolumn{2}{c}{(5)} & \multicolumn{2}{c}{(6)} & \multicolumn{2}{c}{(7)} & \multicolumn{2}{c}{(8)} \\
 &   Number & Price & Number & Price & Number & Price & Number & Price\\
 Pigeons & 10 & \multicolumn{1}{c|}{6} & 20 & \multicolumn{1}{c|}{12} & 15 & \multicolumn{1}{c|}{9} & 35 & 21 \\ 
 Cranes & 42 & \multicolumn{1}{c|}{30} & 14 & \multicolumn{1}{c|}{10} & 63 & \multicolumn{1}{c|}{45} & 42 &30\\
 Swans & 36 & \multicolumn{1}{c|}{28} & 54 & \multicolumn{1}{c|}{42} & 9 & \multicolumn{1}{c|}{7} & 9     & 7\\
 Peacocks & 12  & \multicolumn{1}{c|}{36} & 12  & \multicolumn{1}{c|}{36}  & 13  & \multicolumn{1}{c|}{39} & 14 & 42\\
 &   \multicolumn{2}{c}{(9)} & \multicolumn{2}{c}{(10)} & \multicolumn{2}{c}{(11)} & \multicolumn{2}{c}{(12)} \\
 &   Number & Price & Number & Price & Number & Price & Number & Price\\
 Pigeons & 20 & \multicolumn{1}{c|}{12} & 40 & \multicolumn{1}{c|}{24} & 60  & \multicolumn{1}{c|}{36} & 25  & 15 \\ 
 Cranes & 49 & \multicolumn{1}{c|}{35} & 28 & \multicolumn{1}{c|}{20} & 7  & \multicolumn{1}{c|}{5} & 35  &25\\
 Swans & 18 & \multicolumn{1}{c|}{14} & 18  & \multicolumn{1}{c|}{14} & 18 & \multicolumn{1}{c|}{14} & 27  &21\\
 Peacocks & 13 & \multicolumn{1}{c|}{39} & 14 & \multicolumn{1}{c|}{42} & 15  & \multicolumn{1}{c|}{45} & 13 & 39\\
 &   \multicolumn{2}{c}{(13)} & \multicolumn{2}{c}{(14)} & \multicolumn{2}{c}{(15)} & \multicolumn{2}{c}{(16)} \\
 &   Number & Price & Number & Price & Number & Price & Number & Price\\
 Pigeons & 45 & \multicolumn{1}{c|}{27} & 35  & \multicolumn{1}{c|}{21} & 5  & \multicolumn{1}{c|}{3} & 10 & 6 \\ 
 Cranes & 14  & \multicolumn{1}{c|}{10} & 7  & \multicolumn{1}{c|}{5} & 21  & \multicolumn{1}{c|}{15} & 7 & 5\\
 Swans & 27  & \multicolumn{1}{c|}{21} & 45  & \multicolumn{1}{c|}{35} & 63  & \multicolumn{1}{c|}{49} & 72 & 56\\
 Peacocks & 14  & \multicolumn{1}{c|}{42} & 13  & \multicolumn{1}{c|}{39} & 11  & \multicolumn{1}{c|}{33} & 11 & 33\\
 \end{tabular}
 \vspace{3mm}
 
\begin{sloppypar}

 Of these sixteen solutions, the first four have been given by the commentator. 
\vspace{3mm}

\noindent \textbf{80.} The commentator gives the following solutions:~ 
\vspace{3mm}

\renewcommand*{\arraystretch}{0.9}
\setlength{\tabcolsep}{2pt}
\hspace{-4mm} \begin{tabular}{lcccccccccc}
 & \multicolumn{2}{c}{(1)} & \multicolumn{2}{c}{(2)} & \multicolumn{2}{c}{(3)} & \multicolumn{2}{c}{(4)} & \multicolumn{2}{c}{(5)}\\
 &   Fruits & Price & Fruits & Price & Fruits & Price & Fruits & Price & Fruits & Price\\
Pomegranates & 16  & \multicolumn{1}{c|}{32} & 17  & \multicolumn{1}{c|}{34} & 19   & \multicolumn{1}{c|}{38} & 18  & \multicolumn{1}{c|}{36} & 15 & \multicolumn{1}{c|}{50}\\
Mangoes & 60  & \multicolumn{1}{c|}{36} & 45  & \multicolumn{1}{c|}{27} & 15  & \multicolumn{1}{c|}{9}  & 30  & \multicolumn{1}{c|}{18} & 75 & \multicolumn{1}{c|}{45}\\
Wood-apples & 24  & \multicolumn{1}{c|}{12} & 38  & \multicolumn{1}{c|}{19} & 66   & \multicolumn{1}{c|}{33} & 52  & \multicolumn{1}{c|}{26} & 10 & \multicolumn{1}{c|}{5}\\
\end{tabular}
\vspace{3mm}

\setstretch{1.2}
\noindent \textbf{81-82.} After 8 days, 21$\frac{9}{11}$ \textit{ghaṭikās}. \textbf{83.} The meeting takes
place
after 20 days when the travellers are at a distance of 40 \textit{yojanas} from
the
starting place. \textbf{84-85.} 2 \textit{paṇas} and 3 \textit{kākiṇīs}. \textbf{86-87.} 6, 14, 26,
and
50 respectively. \textbf{88.} The two persons stopping away after going  over 1 \textit{krośa} receive 6$\frac{2}{3}$; the three persons stopping  away after going
over
2 \textit{krośas} get 22$\frac{1}{2}$; and the five persons who go over 3 \textit{krośas} receive
70$\frac{5}{6}$.
\textbf{89-90.} 12, 27, 47, 77, and 137 respectively. \textbf{91.} $\frac{1}{17}$ of a day.

\end{sloppypar}

\newpage

\begin{sloppypar}

\setstretch{1.2}
\noindent \textbf{92.} 4 jack-fruits. \textbf{93-94.} 2  \textit{krośas} and 3  \textit{krośas}
respectively.
\textbf{95.} 63. \textbf{96.} 12 cubits. \textbf{97.} 360. \textbf{98.} 20 cows. \textbf{99.}
36.
\textbf{100.} 9. \textbf{101.} 25. \textbf{102.} $\frac{24}{5}$. \textbf{103(i).} 40; $\frac{5}{8}$
\textbf{103(ii).} 0.
\textbf{104-105.} 2 \textit{paṇas} and $\frac{23}{4}$ \textit{kākiṇīs}. \textbf{106.} 29$\frac{1}{2}$.
\textbf{107.} $6\frac{3}{4}$ \textbf{108.}
206158430208. \textbf{109.} 93. \textbf{110.} 252 \textit{paṇas}. \textbf{111.} 15 days.
\textbf{112.}
If the first traveller travels with speed 1 and acceleration 6 and meets
the second traveller for the first time after 10 days, then the second
meeting of the two travellers will occur 8 days after the first
meeting. \textbf{113.} The first man wins, the amount won being 48360. \textbf{114.} The
second man wins, the amount  won being 1017. \textbf{115.} The second
man wins, the amount won being 25. \textbf{116(i).} 165. \textbf{116(ii).} 55.
\textbf{117.} 3025; 220. \textbf{118.} 588. \textbf{119.} 699. \textbf{120.} 986.
\textbf{121.}
2528. \textbf{122.} 2$\frac{1}{4}$ square cubits. \textbf{123.} 16$\frac{1}{2}$ square
cubits. \textbf{124.}
5$\frac{1}{4}$ square cubits. \textbf{125.} 31$\frac{41}{144}$ square cubits. \textbf{126.}
12 square
cubits. \textbf{127.} 16 square cubits. \textbf{128.} 768 square cubits.
\textbf{129-130.} 46$\frac{127}{3456}$ square cubits. \textbf{131.} 3 square cubits;
30 square
cubits. \textbf{132.} 12 square cubits. \textbf{133.} 35 square cubits.

\end{sloppypar}
\vspace{2.5cm}


\begin{center}
\rule{5em}{.9pt}
\end{center}

\end{document}
