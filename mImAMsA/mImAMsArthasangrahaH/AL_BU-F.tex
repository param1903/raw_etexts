\documentclass[11pt, openany]{book}
\usepackage[text={4.65in,7.45in}, centering, includefoot]{geometry}
\usepackage[table, x11names]{xcolor}
\usepackage{fontspec,realscripts}
\usepackage{polyglossia}
\setdefaultlanguage{sanskrit}
\setotherlanguage{english}
\setmainfont[Scale=.9]{Shobhika}
\newfontfamily\bl[Script=Devanagari, Scale=1.1]{Shobhika-Bold}
\newfontfamily\br[Script=Devanagari]{Shobhika-Bold}
\newfontfamily\bld[Script=Devanagari, Scale=5]{Shobhika-Bold}
\newfontfamily\al[Script=Devanagari, Scale=1.2, Color=purple]{Shobhika-Bold}
\newfontfamily\regular{Linux Libertine O}
\newfontfamily\en[Language=English, Script=Latin]{Linux Libertine O}
\newfontfamily\ab[Script=Devanagari, Scale=0.9, Color=purple]{Shobhika-Bold}
\newfontfamily\qt[Script=Devanagari, Scale=0.9, Color=violet]{Shobhika-Regular}
\newfontfamily\qtl[Script=Devanagari, Scale=1.1, Color=violet]{Shobhika-Bold}
\newcommand{\devanagarinumeral}[1]{%
	\devanagaridigits{\number \csname c@#1\endcsname}} % for devanagari page numbers
%\usepackage[Devanagari, Latin]{ucharclasses}
%\setTransitionTo{Devanagari}{\s}
%\setTransitionFrom{Devanagari}{\regular}
\XeTeXgenerateactualtext=1 % for searchable pdf
\usepackage{enumerate}
\pagestyle{plain}
\usepackage{fancyhdr}
\pagestyle{fancy}
\renewcommand{\headrulewidth}{0pt}
\usepackage{afterpage}
\usepackage{multirow}
\usepackage{multicol}
\usepackage{wrapfig}
\usepackage{vwcol}
\usepackage{microtype}
  \usepackage{amsmath,amsthm, amsfonts,amssymb}
\usepackage{mathtools}% <- new package for rcases
\usepackage{graphicx}
\usepackage{longtable}
\usepackage{setspace}
%%%%%%%%%%%%%%%%%%%%%%%%%%%%%%%%%%%\footnote
\usepackage{manyfoot}
\DeclareNewFootnote{A}
\newcommand\alfootnote[1]{%
  \begingroup
  \renewcommand\thefootnoteA{}\footnoteA{#1}%
  \addtocounter{footnoteA}{-1}%
  \endgroup
}
\usepackage{footnote}
\usepackage{pfnote}
\makeatletter
\def\blfootnote{\gdef\@thefnmark{}\@footnotetext}
\makeatother
\usepackage{perpage}
\MakePerPage{footnote}
%\usepackage[splitrule]{footmisc}
%\usepackage{dblfnote}
%%%%%%%%%%%%%%%%%%%%%%%%%%%%%%%%%%%%%%%%%%%%%%%%%%%
\usepackage{xspace}
\usepackage{array}
\usepackage{emptypage}
\usepackage{hyperref}% Package for hyperlinks
\hypersetup{colorlinks,
citecolor=black,
filecolor=black,
linkcolor=blue,
urlcolor=black}
\usepackage{tikz}
\usetikzlibrary{decorations.pathmorphing, patterns,shapes}
\begin{document}
\pagestyle{empty}
\begin{center}
    \rule{\linewidth}{2pt}\\
    \vspace{-3mm}
    \rule{\linewidth}{1pt}\\
\begin{tikzpicture}
\draw[decoration = {zigzag,segment length = 4mm, amplitude = 1mm},decorate] (-1,0)--(10.6,0);
\end{tikzpicture}
\rule{\linewidth}{1pt}\\
\vspace{-3mm}
\rule{\linewidth}{2pt}\\
\end{center}
\begin{center}
\textbf{श्रीलौगाक्षिभास्करप्रणीतः}\\
\vspace{5mm}
\textbf{\Huge अ\;र्थ\;सं\;ग्र\;हः}\\
\vspace{2.5cm}
\textbf{श्रीमत्परमहंसरामेश्वरशिवयोगिभिक्षुवरप्रणीत\textendash}\\
\vspace{2mm}
\textbf{मीमांसार्थसंग्रहकौमुदीव्याख्यायुतः}\\
\vspace*{\fill}
{\Large 105}\\
{\bld .}\\
\vspace*{\fill}
{\bl नि\;र्ण\;य\;सा\;ग\;र\;\textendash\;मु\;द्र\;णा\;ल\;य\;म्, मुं\;ब\;ई}\\
\vspace{2mm}
\hspace{-5.5cm}\textbf{मूल्यम् १॥ रूप्यकः}
\end{center}
\begin{center}
    \rule{\linewidth}{2pt}\\
    \vspace{-3mm}
    \rule{\linewidth}{1pt}\\
\begin{tikzpicture}
\draw[decoration = {zigzag,segment length = 4mm, amplitude = 1mm},decorate] (-1,0)--(10.6,0);
\end{tikzpicture}
\rule{\linewidth}{1pt}\\
\vspace{-3mm}
\rule{\linewidth}{2pt}\\
\end{center}
\newpage
%%%%%%%%%%%%%%%%%%%%%%%%%%%%%%%%%%%%%%%%%%
\hfill
\newpage
%%%%%%%%%%%%%%%%%%%%%%%%%%%%%%%%%%%%%%%
\begin{center}
\doublespacing
\textbf{\large श्रीः\\}
\textbf{\large श्रीलौगाक्षिभास्करप्रणीतः\\}
\vspace{-3mm}
\textbf{\Huge अर्थसंग्रहः\\}
{\Large (\,A\,R\,T\,H\,A\,S\,A\,N\,G\,R\,A\,H\,A\,) \\}
\vspace*{\fill}
{\bld .}\\
\vspace*{\fill}
\textbf{\Large श्रीमत्परमहंसरामेश्वरशिवयोगिभिक्षुवरप्रणीत\textemdash\\  }
\textbf{\Large मीमांसार्थसंग्रहकौमुदीव्याख्यायुतः }\\
\vspace*{\fill}
{\bld .}\\
\vspace*{\fill}
{\large श्रीमदिन्दिराकान्ततीर्थचरणान्तेवासिभिः}\\
\textbf{\Large नारायण राम आचार्य}\\
{\qt काव्यतीर्थ} इत्येतै\textendash\\
ष्टिप्पण्यादिभिः समलंकृत्य संशोधितः \\
\rule{.35\linewidth}{1.5pt}\\
\vspace{-3mm}
\textbf{{\small पञ्चमं संस्करणम् : १९५०}} \\
\vspace{-5mm}
\rule{.35\linewidth}{1.5pt}\\
{\bl नि\;र्ण\;य\;सा\;ग\;र\;\textendash\;मु\;द्र\;णा\;ल\;य\;म्, मुं\;ब\;ई}\\
\hspace{-5.5cm}\textbf{मूल्यम् १॥ रूप्यकः}
\end{center}





\newpage
%%%%%%%%%%%%%%%%%%%%%%%%%%%%%%%%%%%5

\vspace*{\fill}
\begin{center}
\en
{\Large [ All rights reserved by the publisher ]} \\
\rule{.25\linewidth}{.5pt}\\
\vspace{3mm}
\begin{tabular}{lll}
   Publisher:\textendash\ Satyabhamabai Pandurang,&\multirow{2}{*}{$\bigg\}$} &For the Nirnaya Sagar Press, \\
Printer\textendash\ Ramchandra Yesu Shedge,& &26\textendash\ 28,Kolbhat Street,Bombay. \\
\end{tabular}
\end{center}
\vspace*{\fill}
\newpage
%%%%%%%%%%%%%%%%%%%%%%%%%%%%%%%%%%%5
\onehalfspacing
\begin{center}
\textbf{\Large उपोद्धातः}\end{center}

सुविदितमेवेदं षड्दर्शनीमरण्यानीमवगाहमानानां पण्डितपञ्चाननानां यद् वेदार्थप्रतिपादनपरा  द्वादशलक्षणी पूर्वमीमांसा नाम शास्त्रं तत्रभवच्छ्रीमज्जैमिनिमुनिपादैः करामलकवन्निखिलमर्थजातं दिव्येन चक्षुषा प्रत्यक्षीकुर्वाणैर्दुरधिगमं वेदार्थं बुभुत्सून् जनाननुग्रहीतुं प्राणायीति~।\\

 अथातिक्रामति काले {\qt हासदर्शनतो हासः संप्रदायस्य मीयताम्} इति नयं चरितार्थयति भारतीयानां सर्वविधाभ्युदयिकसाधनेन सह मन्दतामुपेयुषि बुद्धिवैभवे, रजोदूषिततया पदार्थप्रतिबिम्बग्रहाक्षमे च प्रतिभादर्पणे, विरलतामुपगतायां शास्त्राध्ययनपरिपाठ्यां, तिरोभवत्यदृष्टवादे, विलुप्तायां यागादिक्रियायां, त्रिविधसन्तापोत्थधूम्ययाऽधिकृते यज्ञधूमस्थाने, बौद्धाद्यवैदिकमतप्राबल्येन नास्तिक्यपङ्कनिमग्नायां जनतायां, सर्वतो रूपान्तरं प्रतिपन्नायां धर्मभूमौ भारतभुवि,परमकारुणिकैर्वैदिकधर्मोद्धरणप्रवणैः पुनरपि पूर्ववदेव प्रधानं धर्मसाधनं मीमांसादर्शनं तद्द्वारा यागादिकर्म च प्रचारयितुं बद्धपरिकरैः श्रीशबरमुनि-कुमारिलभट्ट-प्रभृतिभिर्मीमांसकमूर्धन्यैर्भाष्यवार्तिकादिरूपेण भूयांसो निबन्धा उपनिबद्धाः~। कालान्तरे च तेभ्योऽपि दुर्बोधतया पराङ्मुखानालस्योपहतचेतसो मन्दधिषणान् जनानवलोक्य अर्वाचीनैर्विचक्षणैः श्रीमत्कृष्णयज्वाऽऽपोदेव- लौगाक्षिभास्करप्रमुखैरपरे प्रबन्धाः प्रणीताः~। तत्र च तत्रभवता श्रीमल्लौगाक्षिभास्करेण संग्रथितोऽर्थसंग्रहः सुतरां सरलसरण्या निखिलमीमांसापदार्थनिरूपणपरतया बालोपयोगेन सर्वानप्यतिशेते प्रक्रियाग्रन्थान् , वस्तुतो नास्ति जैमिनीयतन्त्रं प्रविविक्षूणां जनानामेवंविध उपकारको यावदर्थपरिचायकः कश्चिदन्यः सन्दर्भः~।\\

 निर्माता चास्य श्रीलौगाक्षिभास्करः कस्मिन्समये कतमं देशं निजजनुषाऽलञ्चकारेत्यादिविषयेषु तु संस्कृतसन्दर्भान्तरनिबन्धका इवायमपि वाचंयम एव~। अनेन निर्मितं तर्ककौमुदीनाम वैशेषिकमतप्रतिपादकं पुस्तकं निर्णयसागरयन्त्रालये मुद्रितमुपलभ्यते, एतेन शास्त्रान्तरेऽपि सरलान् मनोरमान् बालोपयोगिनो ग्रन्थानयं न्यभान्त्सीदिति शक्यतेऽनुमातुम्~। टीकाकारोऽप्यस्य शिवयोगिभिक्षुः श्रीरामेश्वरः स्वकीयं वाराणसीवासगौरवमागूरयन्नपि निजसमयादिनिर्णये षोडशकलां शालीनतां बिभर्ति~।
\newpage
%%%%%%%%%%%%%%%%%%%%%%%%%%%%%%%%%%%5
\pagestyle{fancy}
\cfoot{}
\chead{२}

{ इदानीं पुनः सर्वतः प्रसृतायां वेदचर्चायां सुतरां प्रचारमर्हति मीमांसालोचनम्~।~तदन्तरेण हि न सुशकं केनाप्युपायेन वेदार्थज्ञानम्~।~ये हि मीमांसाज्ञानशून्या वैदिकंमन्याः स्वबुद्धिबलेनैव मीमांसन्ते वेदार्थं, ते खलु निरवलम्बे विहायसि विहर्तुमनसो बाहुमात्रसाधना धीविधुरा इव अन्धतमसाच्छन्ने चक्षुष्मतामपि दुःसंचरे पिच्छिले पथि धावितुकामा अन्धा इव च न केवलं प्रेक्षावतामुपहास्यास्पदं भवन्ति परं पातयन्त्यात्मानमपि निरयनिमित्तेऽनर्थावटे~।}\\

{ तदेवं लोकोपहासं निरयनिपातं च परिजिहीर्षोर्वेदार्थमधिजिगांसोर्जनस्य नूनमामननीयमिदं जैमिनीयं तन्त्रमिति व्यक्तमेव~।~तत्र चार्थसंग्रहः सरलतया
हृदयङ्गमप्रणालीप्रतिबद्धपार्थकतया च बालोपयोगाय सर्वातिशायीति न परोक्षं प्रेक्षावताम्~।}\\

 सोऽयं पूर्व वारद्वयं वाराणस्यां मुद्रितोऽपि स्खलनबाहुल्येन मुद्रणादिदोषप्राचुर्येण चाध्यापकानामपि ललाटन्तपोऽजनिष्ट; टीकात्रुटिभिस्तु बहुशो मूलमप्युदमूल्यतेति न खलु क्षेमङ्करोऽप्ययं ग्रन्थस्तावत्प्रियङ्करोऽभूदध्येतृवर्गस्य, तमिमं विद्वद्दारिद्र्यं दूरीकर्तुं दुष्प्रापपुस्तकमुद्रणादिभिः संस्कृतसाहित्यं समुचितमुपसेवितवद्भिः श्रीमद्भिनिर्णयसागरयन्त्रालयाधिपतिभिर्ग्रन्थोऽयं निखिलतन्त्रापरतन्त्रप्रतिभैर्विद्याधिनाथ- श्री ६ गुरुवरश्रीकाशीनाथशास्त्रिपादैः सम्पाद्य संस्कार्य च लोकोपकाराय
प्राकाश्यं प्रापितः~।~वेदार्थाधिगमस्य जीवातुभूतोऽयं ग्रन्थोऽध्यापकानां छात्राणां मीमांसारसिकानामन्येषां च निःसंशयमुपकारकोऽवश्यं संग्रहणीय इति
खलु न वर्णनीयतामर्हति पूर्वमुद्रितपुस्तकावलोकनव्याकुलमनोभिर्विशुद्धग्रन्थानुशीलनकुतूहलैश्चास्यालोचनेन नूनमधिगमनीय आह्लाद इत्याशास्यते~।~एतत्पुस्तकमाद्रियमाणैश्च प्राचीनानां दुरधिगमानां दर्शनपुस्तकानां प्रकाशनाय प्रोत्साहनीयाः श्रीनिर्णयसागरयन्त्रालयाध्यक्षा इत्यभ्यर्थयते\textemdash\ \begin{table}[h!]
    \centering
    \begin{tabular}{ccp{1cm}c}
        हरिद्वारोपान्तवर्तिज्वालापुरीय\textemdash\ &\multirow{3}{*}{$\Bigg\}$} &&\textbf{विदुषां वशंवदः}\\
महाविद्यालयः~। शुद्ध वैशाख सुदि ७&& &\textbf{पद्मसिंहशर्मा}\\
(संवत् १९७२)& &&\textbf{(भारतोदय\textemdash\ सम्पादकः)}
    \end{tabular}
\end{table}
\newpage
%%%%%%%%%%%%%%%%%%%%%%%%%%%%%%%%
\chead{}
\begin{center}
\textbf{\Large अर्थसंग्रहविषयानुक्रमणिका}\\
\rule{.15\linewidth}{1pt}
\end{center}



\begin{longtable}{lcccr|lcccr}
 \multicolumn{2}{c}{विषयः} &&&पृष्ठम्&\multicolumn{2}{c}{विषयः} &&&पृष्ठम्\\
\endhead
\multicolumn{2}{l}{मङ्गलाचरणम्} &$\ldots$&$\ldots$&१&\multicolumn{4}{l}{द्वितीयाविनियोक्त्र्या उदाहरणम्} &४२\\
\multicolumn{3}{l}{तन्त्रारम्भकसूत्रावतरणम्} &$\ldots$&३&\multicolumn{4}{l}{सप्तमीविभक्तिविनियोक्त्र्या उदाह-} &\\
\multicolumn{4}{l}{धर्मविचारशास्त्रस्यावश्यकता}& ५ &रणम्&$\ldots$&$\ldots$&$\ldots$&४३\\
\multicolumn{2}{l}{धर्मलक्षणप्रश्नः}& $\ldots$&$\ldots$&६&\multicolumn{4}{l}{अमूर्ताया अपि भावनाङ्गत्वम्} &४४ \\
\multicolumn{3}{l}{वेदस्य धर्मप्रतिपादकत्वम्}&$\ldots$& ८&\multicolumn{4}{l}{भावनाया आख्यातवाच्यत्वम्} &४६\\
\multicolumn{2}{l}{भावनाविचारः} &$\ldots$&$\ldots$&१०&\multicolumn{2}{l}{लिङ्गनिर्वचनम्}&$\ldots$&$\ldots$&४९\\
\multicolumn{2}{l}{शाब्दी भावना} &$\ldots$&$\ldots$&११&\multicolumn{2}{l}{वाक्यनिर्वचनम्} &$\ldots$&$\ldots$&५२\\
\multicolumn{3}{l}{शाब्द्या लौकिकवैदिकभेदौ}&$\ldots$& १२&\multicolumn{3}{l}{प्रकृतिविकृतिलक्षणम्} &$\ldots$&५३\\
\multicolumn{3}{l}{आर्थीभावनालक्षणम्} &$\ldots$&१९&\multicolumn{2}{l}{प्रकरणनिरूपणम्} &$\ldots$&$\ldots$&५५\\
\multicolumn{3}{l}{आर्थीभावनाया अंशत्रयम्} &$\ldots$&२१&\multicolumn{2}{l}{प्रकरण्द्वैविध्यम्} &$\ldots$&$\ldots$&५४\\
\multicolumn{2}{l}{वेदलक्षणविचारः} &$\ldots$&$\ldots$&२७&\multicolumn{2}{l}{महाप्रकरणम्} &$\ldots$&$\ldots$&५५\\
\multicolumn{2}{l}{विधिमीमांसा} &$\ldots$&$\ldots$&२८&\multicolumn{2}{l}{अवान्तरप्रकरणम्} &$\ldots$&$\ldots$&५७\\
\multicolumn{3}{l}{वाक्यभेददोषपरिहारः} &$\ldots$&२९ &\multicolumn{2}{l}{संदंशलक्षणम्} &$\ldots$&$\ldots$&५७ \\
\multicolumn{2}{l}{गुणविध्यादिभेदाः} &$\ldots$&$\ldots$&३०&\multicolumn{2}{l}{स्थाननिरूपणम्} &$\ldots$&$\ldots$&६१\\
\multicolumn{2}{l}{उभयविधित्वम्} &$\ldots$&$\ldots$&३०&\multicolumn{3}{l}{पाठसादेश्येन विनियोगः} &$\ldots$&६१\\
\multicolumn{2}{l}{विधिश्चतुर्विधः} &$\ldots$&$\ldots$&३२&\multicolumn{4}{l}{अनुष्ठानसदेश्येन विनियोगः}&६३\\
उत्पत्तिविधिः. &$\ldots$&$\ldots$&$\ldots$&३२&\multicolumn{2}{l}{समाख्यानिरूपणम्} &$\ldots$&$\ldots$&६६\\
\multicolumn{2}{l}{यागस्य रूपद्वयम्} &$\ldots$&$\ldots$&३४&\multicolumn{3}{l}{विनियोगविधिबोधिताङ्गानि}&$\ldots$&६७\\
\multicolumn{2}{l}{विनियोगविधिः} &$\ldots$&$\ldots$&३५&\multicolumn{2}{l}{संनिपत्योपकारकाणि} &$\ldots$&$\ldots$&६८\\
\multicolumn{3}{l}{विधेः श्रुत्यादिषट्प्रमाणानि}&$\ldots$&३८&\multicolumn{2}{l}{आरादुपकारकाणि} &$\ldots$&$\ldots$&६८\\
\multicolumn{2}{l}{श्रुतिनिर्वचनम्}&$\ldots$&$\ldots$&३९&प्रयोगविधिः &$\ldots$&$\ldots$&$\ldots$&७०\\
\multicolumn{3}{l}{विनियोक्त्री श्रुतिस्त्रिधा}&$\ldots$&४०&क्रमस्वरूपम् &$\ldots$&$\ldots$&$\ldots$&७२\\
\multicolumn{4}{l}{तृतीयाविभक्तिरूपाया उदाहर०}&४१&\multicolumn{2}{l}{श्रुत्यादिषट्प्रमाणानि} &$\ldots$&$\ldots$&७३\\
\multicolumn{4}{l}{द्वितीयारूपाया विनियोक्त्र्या उदा-}&४१&श्रुतिलक्षणम् &$\ldots$&$\ldots$&$\ldots$&७३\\
हरणम्&$\ldots$&$\ldots$&$\ldots$&४१&\multicolumn{2}{l}{अर्थक्रमलक्षणम्} &$\ldots$&$\ldots$&७६\\
\pagebreak
%%%%%%%%%%%%%%%%%%%%%%%%%%%%%%%%%%%5
\multicolumn{2}{l}{पाठक्रमलक्षणम्} &$\ldots$&$\ldots$&७७ &\multicolumn{4}{l}{तद्यपदेशेन कर्मनामधेयत्वम्} &१०३\\
\multicolumn{2}{l}{\multirow{2}{*}{स्थानलक्षणम्}} &\multirow{2}{*}{$\ldots$}&\multirow{2}{*}{$\ldots$}&\multirow{2}{*}{७९}&\multicolumn{4}{l}{कर्मनामधेयत्वे उत्पत्तिशिष्टगुण\textendash} &\\
&&&&&\multicolumn{2}{l}{बलीयस्त्वम्} &$\ldots$&$\ldots$&१०५\\
\multicolumn{2}{l}{मुख्यक्रमलक्षणम्} &$\ldots$&$\ldots$&८२&\multicolumn{2}{l}{निषेधमीमांसा} &$\ldots$&$\ldots$&१०६\\
\multicolumn{2}{l}{\multirow{2}{*}{प्रवृत्तिक्रमलक्षणम्}} &\multirow{2}{*}{$\ldots$}&\multirow{2}{*}{$\ldots$}&\multirow{2}{*}{८४} &\multicolumn{4}{l}{लिङर्थशब्दभावनाया नञर्थेना\textendash}&\\
&&&&&\multicolumn{2}{l}{न्वय:} &$\ldots$&$\ldots$&१०७\\
\multicolumn{2}{l}{अधिकारविधिलक्षणम्} &$\ldots$&$\ldots$&८६&\multicolumn{2}{l}{नञ्स्वभावकथनम्} &$\ldots$&$\ldots$&१०८\\
\multicolumn{2}{l}{अथ मन्त्रमीमांसा} &$\ldots$&$\ldots$&९२&\multicolumn{2}{l}{बाधकं द्विविधम्} &$\ldots$&$\ldots$&१०९\\
नियमविधिः &$\ldots$&$\ldots$&$\ldots$&९२&\multicolumn{4}{l}{पर्युदासपक्षे नेक्षेतेत्यस्य वा०} &११२\\
\multicolumn{2}{l}{परिसंख्याविधिः} &$\ldots$&$\ldots$&९४&\multicolumn{4}{l}{\multirow{2}{*}{विकल्पप्रसक्तौ पर्युदासाश्रयणम्}} &\multirow{2}{*}{११३}\\
\multicolumn{4}{l}{परिसंख्यायाः श्रौतीत्वलाक्ष\textendash} &&\multicolumn{5}{l}{}\\
\multicolumn{2}{l}{णिकीत्वभेदौ}& $\ldots$&$\ldots$&९५ &\multicolumn{2}{l}{बाधायोगोपसंहारः} &$\ldots$&$\ldots$&११५\\
\multicolumn{3}{l}{परिसंख्याया दोषत्रयम्} &$\ldots$&९५&\multicolumn{4}{l}{पर्युदासोपसंहारयोर्भेदवर्णनम्} &११७\\
\multicolumn{2}{l}{नामधेयमीमांसा} &$\ldots$&$\ldots$&९६ &\multicolumn{4}{l}{विकल्पे प्रतिषिध्यमानस्यानर्थ}&\\
\multicolumn{4}{l}{नामधेयत्वे निमित्तचतुष्टयम्} &९७&हेतुत्वम्&$\ldots$&$\ldots$&$\ldots$&११९\\
\multicolumn{4}{l}{नामधेयत्वस्य वाक्यभेदप्रसङ्ग\textemdash\ } &&\multicolumn{2}{l}{अर्थवादमीमांसा} &$\ldots$&$\ldots$&१२१\\
\multicolumn{4}{l}{रूपद्वितीयनिमित्तोदाहरणम्}&९९&\multicolumn{2}{l}{अर्थवादविभागः} &$\ldots$&$\ldots$&१२१\\
\multicolumn{3}{l}{तत्प्रख्यशास्त्रान्नामधेयत्वम् }&$\ldots$&१०१&\multicolumn{3}{l}{अर्थवादस्य भेदत्रयम्} &$\ldots$&१२३\\
\multicolumn{4}{l}{देवतारूपेणाग्निप्रापकशास्त्रप्रश्नः}&१०१&\multicolumn{2}{l}{ग्रन्थोपसंहारः} &$\ldots$&$\ldots$&१२३\\
\end{longtable}

\vspace{1cm}
\begin{center}
    \textbf{\large इत्यर्थसंग्रहविषयानुक्रमणिका~।}\\
\vspace{1cm}
    \rule{.2\linewidth}{1pt}
\end{center}
\chead{(\,२\,)}

\newpage
%%%%%%%%%%%%%%%%%%%%%%%%%%%%%%%%%%%%%%%%%
\thispagestyle{empty}
\onehalfspacing
\begin{center}
श्रीः\\
\vspace{3mm}
महोपाध्यायलौगाक्षिभास्करप्रणीतः\\
\vspace{3mm}
\textbf{\Huge अर्थसंग्रहः}\\
\vspace{3mm}
\textbf{\large परमहंसरामेश्वरभिक्षुकृतार्थसंग्रहकौमुदीसहितः}\\
\vspace{3mm}
{\bld .}\\
\vspace{3mm}
\textbf{मङ्गलाचरणम्}
\end{center}
\begin{quote}
    \al
वासुदेवं रमाकान्तं नत्वा लौगाक्षिभास्करः~।~\\
कुरुते जैमिनिनये प्रवेशायार्थसंग्रहम्~॥~१~॥   
\end{quote}
\hrule
\begin{quote}
\qtl  
आद्यो~यो~हेतुर्विश्वसर्गे~महेशो~यज्ञादीनां (को)~यो~हव्यनिक्षेपदेवः~।\\
भूतानां भर्ता सर्वभूतान्तरात्मा हृद्यं मे कार्यं तत्प्रणामः करोतु~॥~१
॥\\

श्रीजैमिनिनये ग्रन्थः प्रवेशाय निरूपितः~।\\
विदुषा तत्र बालानां कौमुदीयं वितन्यते~॥~२~॥
\end{quote}
 
इह खलु परमकारुणिकेन मुनिना जैमिनिना धर्माधर्मविवेकाय द्वादशलक्षणी मीमांसा प्रणीता~। तत्र हि प्रवेशाय शिशूनामर्थसंग्रहाख्यं प्रकरणं प्रारभमाणो लौगाक्षिभास्करः शिष्टाचारपरिप्राप्तं प्रचयगमनादिफलकमिष्टदेवतानमस्कारलक्षणं मङ्गलमाचरति- {\br वासुदेवमित्यादिना} ~। वासुदेवं श्रीनारायणं सर्वनिवासाधिष्ठानं प्रकाशात्मकं ब्रह्मेत्यर्थः
~। रमाकान्तं रमाया लक्ष्म्याः कान्तमित्यर्थः~। न चेश्वरानङ्गीकारो द्रव्यत्यागोद्देश्यविष्णुदेवतायाः स्वीकृतत्वादित्यन्यत्~। {\br जैमिनिनय इति ~।} जैमिनिप्रणीते द्वादशाध्यायात्मके पूर्ववेदभागविचारात्मके तन्त्र इत्यर्थः~। प्रवेशाय बालानामिति शेषः~।~{\br अर्थसंग्रहमिति~।} अर्थानां द्वादशाध्यायप्रतिपाद्यप्रमाणादिपदार्थानां संक्षिप्तशब्दरचनया लक्षणादिकथनमित्यर्थः~।
\newpage
%%%%%%%%%%%%%%%%%%%%%%%%%%%%%%%%%
\fancyhead[LE,RO]{\thepage}
\renewcommand{\thepage}{\devanagarinumeral{page}}
\setcounter{page}{2}
\fancyhead[CE]{अर्थसंग्रहः}
\fancyhead[RE]{[द्वादशाध्यायीपदार्थक्रमः]}
\begin{center}
\textbf{द्वादशाध्यायीपदार्थक्रमः}    
\end{center}
\renewcommand{\thefootnote}{\devanagarinumeral{footnote}}

तत्र हि {\br प्रथमे लक्षणे} विध्यादेः प्रामाण्यं निरूपितम्~।~{\br द्वितीये} तद्विधेयकर्मभेदो निरूपितः~। {\br तृतीये} विहितानां शेषशेषिभावः~। {\br चतुर्थे} ऋतुप्रयुक्तानुष्ठेयानां पुरुषार्थप्रयुक्तानुष्ठेयानां च पदार्थानां परिमाणं चिन्तितम्~।~{\br पञ्चमे}ऽनुष्ठेयपदार्थानामनुष्ठानक्रमो निरूपितः~। {\br षष्ठे} विहितकर्मफलभोक्तृत्वरूपाधिकारनिरूपणम्~। {\br सप्तमे} प्रकृतावुपदिष्टाङ्गानां विकृतौ सामान्यातिदेशो निरूपितः~। {\br अष्टमे} {\qt आग्नेयोऽष्टाकपालः} इत्यादिप्रकृत्यङ्गानां {\qt सौर्यं चरुं निर्वपेत्} इत्यादिविकृतौ \blfootnote{पाठा०- $^{१}${\qt सप्तदशद्रव्य}.}\footnotemark सप्रपञ्चं द्रव्यदेवतादिद्वारेण विशेषातिदेशः~।
{\br नवमे} प्रकृतावुपदिष्टानां मन्त्र-साम-संस्कारकर्मणां विकृतावतिदेशप्राप्तानां प्रकृतिविकृत्योरग्निसूर्यादिदेवतादिभेदे प्रकृतिगतं देवतादिवाचकं पदं विहाय विकृतौ देवतादिवाचकस्य पदस्याध्याहार ऊहो निरूपितः~। यथा {\qt अग्नये जुष्टम्} इति मन्त्रे प्रकृत्युपदिष्टे विकृतावतिदेशप्राप्ते {\qt अग्नि}पदपरित्यागेन {\qt सूर्य} पदाध्याहारः~। यथा च {\qt गिरागिरा च दक्षसे} इत्यत्र साम्नि {\qt गिरा}पदस्य परित्यागेनेरापदाध्याहारः साम्नामूहः, व्रीह्यादिद्रव्यान्तरसंबन्धिनश्चावघातादेर्नीवारादिद्रव्यान्तरसंबन्धः, संस्कारकर्मणामूहश्च~। {\br दशमे} विकृतौ चोदकप्राप्तानां प्राकृताङ्गानां प्रकृतौ सावकाशानां विकृतौ ह्युपदिष्टविशेषाङ्गादिना बाधो निरूपितः~। यथा  प्रकृतेः सकाशाद्विकृतावतिदिष्टानां बर्हिषां {\qt शरमयं बर्हिः} इत्युपदिष्टेन शरमयबर्हिषा विकृतौ बाधः~। {\br एकादशे} चानेकाङ्गिविधिप्रयुक्तानामङ्गानां सकृदनुष्ठानात् सर्वाङ्गिनामुपकारसाम्यं तन्त्रं निरूपितम्~।~यथा {\qt आग्नेयोऽष्टाकपालः, उपांशुयाजमन्तरा यजति, अग्नीषोमीयमेकादशकपालम्} इत्यादिपौर्णमासादिकर्मप्रयुक्तानां
प्रयाजाद्यङ्गानां सकृदनुष्ठानात्सर्वाङ्ग्युपकारः~।~{\br द्वादशे} त्वेकाङ्गिप्रयुक्तस्याङ्गानुष्ठानस्य तत्प्रयोजकसामर्थ्यरहितेऽङ्ग्यन्तरेऽप्युपकारः प्रसङ्गो निरूपितः~। यथा
{\qt अग्नीषोमीयं पशुमालभेत} इति पशुविधौ प्रयुक्तानां प्रयाजाद्यङ्गानां पशुपुरोडाशेऽप्युपकार इति~। तथा च ते पदार्थाः केचिदत्रापि संक्षेपेण निरूपिताः, केचित्तु सूचितास्तत्र कांश्चित्पदार्थांस्तत्र तत्र प्रदर्शयिष्याम इत्यर्थसंग्रहमित्यस्योपपत्तिः~। तथा सति यानि च \footnote{{\qt चानुबन्धादीनि}.}विषयादीनि जैमिनितन्त्रस्य तान्येवास्यापि तत्प्रकरणत्वात्~। तस्य च धर्म एव विषयः, अधर्मस्तु निरसनीयतया विचारितः~। सोऽपि अधिकारी अधीतवेदवेदाङ्गो धर्मजिज्ञासुः~। श्रेयोऽर्थः प्रयोजनं च विचारितधर्मानुष्ठानेन स्वर्गादि~। संबन्धो बोध्यबोधकभावलक्षणो धर्मतन्त्रयोरिति~॥~१~॥
\newpage
%%%%%%%%%%%%%%%%%%%%%%%%%%%%%%%
\fancyhead[CO]{कौमुदीव्याख्यासहितः}
\fancyhead[LO]{[तन्रारम्भकसूत्रावतरणम् ]} 
\begin{center}
 \textbf{तन्त्रारम्भकसूत्रावतरणम्}    
\end{center}

{\bl
अथ परमकारुणिको भगवाञ्जैमिनिर्धर्मविवेकाय द्वादशलक्षणीं प्रणिनीय तत्रादौ धर्मजिज्ञासां सूत्रयामास - {\al अथातो धर्मजिज्ञासा} इति~।~अत्र {\al अथ} शब्दो
वेदाध्ययनानन्तर्यवचनः~।~{\al अतः} शब्दो हि वेदाध्ययनस्य दृष्टार्थत्वं ब्रूते
~।}\\
\hrule
\vspace{3mm}
तत्र तावत्प्रतिज्ञातमर्थसंग्रहं निरूपयितुं स्वप्रकरणस्य तन्त्रारम्भाधीनारम्भकत्वात्तन्त्रारम्भार्थकं सूत्रमवतारयति {\br अथेत्यादिना~।} अत्र च {\qt अथ}शब्दः सौत्राथशब्दसमानार्थकः ~। वेदाध्ययनानन्तरं तदर्थविचारः कर्तव्य इति धर्मजिज्ञासां सूत्रयामासेति तदर्थः~। परमकारुणिको निरुपधिकरुणायुक्तः~। भगवान् कीर्त्यादिमान्~।{\qt भगं श्रीकाममाहात्म्यवीर्ययत्नार्ककीर्तिषु} इत्यमरात्~। जैमिनिरादौ धर्मजिज्ञासां सूत्रयामासेत्यन्वयः~। कस्यादाविति वीक्षायामाह\textendash {\br द्वादशलक्षणीमिति~।} द्वादशानां लक्षणानामध्यायानां समाहारो द्वादशलक्षणी, तां प्रणिनीय बुद्धौ समारोप्य; तत्र तस्यां तस्या वेत्यर्थः~। लोकेऽपि जनः स्वकृत्यं समाकलय्य तत्करणे प्रवर्तत इति प्रसिद्धेर्न प्रणिनीय सूत्रयामासेति विरोधः~। {\qt  प्रणिनाय} इति पाठान्तरे तु प्रणयनं कृतवानित्यर्थः~।~प्रणिनायेत्युक्ते केन क्रमेणेति वीक्षायां
तत्रेत्याद्युत्तर, धर्मविवेकायेत्यधर्मस्याप्युपलक्षणं, तयोर्विवेकाय निर्णयज्ञानायेत्यर्थः~। कुत्र धर्मजिज्ञासां सूत्रयामासेति वीक्षायामाह\textendash {\br अथेति~।} सूत्रं व्याचष्टे {\br अत्रेत्यादिना~।} अत्र सूत्रे, अस्य च प्रथमसूत्रस्य {\qt चोदनालक्षणोऽर्थौ धर्मः} इत्यारभ्य {\qt अन्वाहार्ये च दर्शनात्} इत्यन्तं जैमिनिप्रणीतं धर्मविचारशास्त्रं  विषयः~।
तत्र संशयः - किमस्य धर्मविचारतन्त्रस्यारम्भोऽध्ययनविध्यप्रयोज्यस्तत्प्रयोज्यो वा ? इति~। तत्र यदि स्वाध्यायाध्ययनविधिनार्थज्ञानाय दृष्टप्रयोजनाय वेदाध्ययनं विधीयेत तदा तस्य शास्त्रारम्मे भवेदपि प्रयोजकत्वम्~। नैतदस्ति, अन्यथा सिद्धत्वात्~। तथा हि किमत्यन्ताप्राप्तार्थज्ञानहेतुमध्ययनं तद्विधिर्विधत्ते, किंवा पक्षे प्राप्तस्यावघातावन्नि\blfootnote{पाठा०\textemdash\ $^{१}${\qt नियामक (?)}.}\footnotemark यमेन? नाद्यः; विवादास्पदं वेदाध्ययनमर्थज्ञानहेतुः, अध्ययनत्वात् ,
भारताद्यध्ययनवत्-इत्यनुमानेनेवाध्ययनस्यार्थज्ञानहेतुत्वप्राप्तेः~।
\newpage
%%%%%%%%%%%%%%%%%%%%%%%%%%%%%%%%%%%%%%%%%%5
\fancyhead[RE]{[ तन्रारम्भकसूत्रावतरणम् ]}
\noindent
नापि द्वितीयः ; अवघातवैषम्यात्~। यथाऽवघातनिष्पन्नैरेव तण्डुलैरनुष्ठीयमानौ दर्शपूर्णमासाववान्तरापूर्वद्वारेण परमापूर्वं जनयतः तदपूर्वमेवावघातनियमहेतुः, तथा लिखितपाठेन गुरुपूर्वकाध्ययनेन वाऽर्थज्ञानसंभवात्पक्षे प्राप्ताध्ययननियमहेतुर्वक्तव्यः, स च नास्ति ; लिखितपाठजन्यार्थज्ञानेनैव क्रत्वनुष्ठानसिद्धेः प्रयोजकाभावात्~। तस्मादुक्तविधिद्वयासंभवादर्थज्ञानहेतुविचारशास्त्रारम्भस्य न विधिप्रयोज्यत्वमिति प्राप्तम्~। अत्रोच्यते\textendash यदुक्तम् अध्ययनस्यार्थज्ञानहेतुत्वमनुमानसिद्धमिति नात्यन्ताप्राप्तविधिरिति तत्तथैव~। नियमविधिस्तु भवत्येव~। न च प्रयोजकाभावः~। सकलक्रत्वपूर्वस्यैव प्रयोजकत्वाद्दर्शपूर्णमासजन्यपरमा\textendash
पूर्वस्यावघातनियमजन्यापूर्वकल्पकत्व\textemdash\ वदेव च क्रतुजन्यापूर्वजातस्य क्रतुज्ञानसाधनाध्ययननियमजन्यापूर्वकल्पकस्य सत्त्वान्नियमादृष्टस्य
कल्पनादर्थज्ञानसाधनयोर्लिखितपाठगुरुपूर्वकाध्ययनयोः पक्षे प्राप्तत्वात् यदा गुरुपूर्वकाध्ययनं परित्यज्य लिखितपाठादिनार्थज्ञानं संपादयितुं व्युत्पन्नः पुरुषः प्रवर्तते तदा नियमादृष्टाय गुरुपूर्वकाध्ययनमेवार्थज्ञानसाधनं विधीयते, क्रतुजन्यापूर्वप्रयुक्तनियमादृष्टस्यास्वीकारे च श्रूयमाणो विधिरनर्थकः स्यात्~। न च नानर्थको
लिखितपाठगुरुपूर्वकाध्ययनयोरक्षरग्रहणमात्रेऽप्यविशेषसाधनत्वात् यदा गुरुपूर्वकाध्ययनं परित्यज्य लिखितपाठेनाक्षरग्रहणाय प्रवर्तते तदा गुरुपूर्वकाध्ययनमेव नियमादृष्टाय विधीयत इति वाच्यम् ; तत्फलस्य कल्प्यत्वप्रसङ्गात्~। अर्थज्ञानरूपदृष्टप्रयोजनायाध्ययनस्य विधेयत्वे तु नियमादृष्टस्य क्रतुजन्यापूर्वे श्रुतफले ह्युपयोगो भविष्यति~। न च
{\qt यदृचोऽधीते पयसः कुल्या अस्य पितृृन्स्वधा वहन्ति} इत्यार्थवादिकं श्रुतमेव फलमिति वाच्यम् ; तस्य नित्याध्ययनविधिफलत्वेन प्रथमाध्ययनविधिफलत्वाभावात्~। किंचाध्ययनव्यापारस्य संभवत्यर्थज्ञानरूपदृष्टफलकत्वे केवलादृष्टार्थकत्वानुपपत्तेः~। तथा चोक्तम्\textendash {\qt लभ्यमाने फले दृष्टे नादृष्टपरिकल्पना~। विधेस्तु
नियमार्थत्वान्नानर्थक्यं भविष्यति~॥} इति~। किंच विधिनैव वेदाध्ययनस्य तदर्थज्ञानपर्यवसायित्वं तदर्थनिर्णयहेतुविचारकर्तव्यता चाक्षिप्यते~। तथा हि\textendash {\qt स्वाध्यायोऽध्येतव्यः} इत्यत्र तव्यप्रत्ययः शाब्दभावनामभिधत्ते~। सा च स्वभाव्यं विनानुपपद्यमाना किंचिद्भाव्यं कल्पयति~। तत्र चैकप्रत्ययोपात्तत्वेनार्थभावनैव भाव्यत्वेन समन्वेति~। सापि स्वभाव्यमन्तरेणानुपपद्यमाना किंचिद्भाव्यमाक्षिपति~। तत्रापि फलपदस्याश्रवणात्समभिव्याहृतः स्वाध्यायः कर्मभूत एव भाव्यत्वेन संबध्यते~।
तस्य च फलवदर्थावबोधपर्यन्तत्वाभावे 
\newpage
%%%%%%%%%%%%%%%%%%%%%%%%%%%%%%%%%%%
\fancyhead[LO]{[धर्म०शास्त्रस्यावश्यकता ]} 
\begin{center}
धर्मविचारशास्त्रस्यावश्यकता     
\end{center}

{\bl
{\qtl स्वाध्यायोऽध्येतव्य} इत्यध्ययनविधौ तदध्ययनस्यार्थज्ञानरूपदृष्टार्थकत्वेन व्यवस्थापनात्~। तथा च वेदाध्ययनानन्तरं यतोऽर्थज्ञानरूपदृष्टार्थकं तदध्ययनमतो हेतोर्धर्मस्य
वेदार्थस्य जिज्ञासा कर्तव्येति शेषः~। जिज्ञासापदस्य विचारे लक्षणा~। अतो धर्मविचारशास्त्रमिदारम्भणीयमिति शास्त्रारम्भसूत्रार्थः~।}\\
\hrule
\vspace{3mm}
\noindent
भाव्यतानुपपत्त्या फलवदर्थावबोधपर्यवसायित्वमापतति~। अर्थनिर्णयमन्तरेण च फलवदर्थावबोधस्यासंभवेनार्थनिर्णयहेतुविचारकर्तव्यतामप्यध्ययनविधिराक्षिपतीति
~। तस्मादर्थज्ञानरूपदृष्टप्रयोजनायैवेदमध्ययनं विधीयते नाक्षरग्रहणमात्रायेति सिद्धान्तमभिप्रेत्य, {\qt अथ} शब्दं वेदाध्ययनानन्तर्यार्थकत्वेन, {\qt अतः} शब्दं च वेदाध्ययनस्य
वेदार्थज्ञानरूपदृष्टार्थकत्वपरत्वेन च व्याचष्टे  {\br अथशब्द इत्यादिना~।} \\

 तत्र हेतुमाह\textendash {\br स्वाध्याय इत्यादिना~। अध्ययनविधाविति~।} अध्ययनविध्यनुकूलविचारात्मके प्रमाणलक्षणस्य प्रथमाधिकरण इत्यर्थः~। वेदः
तच्छब्दार्थः~। कर्तव्यपदाध्याहारेण सूत्रं योजयति {\br तथा चेत्यादिना~। तथा चेति~।} वेदाध्ययनस्य दृष्टार्थत्वे च सतीत्यर्थः~। तस्य वेदस्याध्ययनं यतोऽर्थज्ञानरूपदृष्टार्थकमतो हेतोर्वेदाध्ययनानन्तरं वेदार्थस्य धर्मस्य जिज्ञासा पदार्थनिर्णयहेतुविचारः कर्तव्य इत्यध्ययनविधिप्रयुक्त्यैव शास्त्रमारम्भणीयमिति भावः~। {\br ननु} जिज्ञासा हि ज्ञानेच्छा~। न च सा कर्तुं शक्यते~। तस्या व्यापारागोचरत्वात्, इच्छामात्रेणानुष्ठानोपयोगिधर्मज्ञानासंभवाच्चेत्यत आह\textendash {\br जिज्ञासापदस्येति ~।} तथा च जिज्ञासेत्यत्र प्रकृत्या
ज्ञानमात्रशक्तिमत्यानुष्ठानोपयोगि ज्ञानमजहल्लक्षणया प्रत्ययेन च साध्यसाधनभावसंबन्धेनेच्छासाध्यो विचारो जहल्लक्षणया च बोध्यत इत्यर्थः~। समर्थितं शास्त्रारम्भमुपसंहरति {\br अत इति~।} स्वाध्यायाध्ययनविधेः शास्त्रारम्भे प्रयोजकत्वं {\qt अतः} शब्दार्थः~।~\\

  {\br ननु} धर्मविचारशास्त्रमारम्भणीयमित्ययुक्तम्~। विचारविषयधर्मस्यानिरूपणात्, तदनिरूपणं च लक्षणप्रमाणाभावात्~। लक्षणप्रमाणाभ्यामेव हि वस्तुसिद्धिर्नान्यथा~। अत एवोक्तम्\textendash {\qt मानाधीना मेयसिद्धिर्मानसिद्धिश्च लक्षणात्} इति~।
\newpage
%%%%%%%%%%%%%%%%%%%%%%%%%%%%%%%%%%%%%%%%%%ः
\fancyhead[RE]{[ धर्मलक्षण\textemdash\ }
\begin{center}
\textbf{धर्मलक्षणप्रश्नः}
\end{center}

{\bl अथ को धर्मः, किं तस्य लक्षणमिति चेत्, उच्यते\textendash {\al  यागादिरेव धर्मः}~। तल्लक्षणं च वेदप्रतिपाद्यः प्रयोजनवदर्थो धर्म इति~। प्रयोजनेऽतिव्याप्तिवारणाय प्रयोजनवदिति~। भोजनादावतिव्याप्तिवारणाय वेदप्रतिपाद्य इति~। अनर्थफलकत्वादनर्थभूते श्येनादावतिव्याप्तिवारणाय अर्थ
इति~।}\\
\hrule
\vspace{3mm}
\noindent
सजातीयविजातीयवस्त्वन्तरेभ्यः स्वलक्ष्यस्य व्यावर्तको लोकप्रसिद्धः कश्चिदाकारविशेषो लक्षणम्~। तेन च लक्षणेन लक्ष्ये संभाविते सति ततः प्रमाणेन तदवगच्छति~। यथा {\qt सास्नादिमती गौः} इति गोलक्षणलक्षितपदार्थमन्विष्य {\qt इयं गौः} इति चक्षुरादिना तदवगच्छति, तथा धर्मस्य नास्ति लक्षणम् , अलौकिकत्वात्~। न च विहितक्रियात्वं धर्मत्वमिति वाच्यम् ; विहितद्रव्याव्याप्तिप्रसङ्गात्~। न च तस्यालक्ष्यत्वं, फलार्थं गुणानुाष्ठातरि धार्मिकोऽयमिति व्यवहारस्याभावप्रसङ्गात्~। न चेष्टापत्तिः~। इन्द्रियकामाद्यधिकरणस्य धर्मविचारात्मकत्वाभावेन शास्त्रसंगत्यभावप्रसङ्गात् , फलार्थं विहितस्य दध्यादिगुणस्य धर्मत्वाभावे तज्जन्यादृष्टस्यापि धर्मत्वासंभवेन तस्य {\qt धर्मः क्षरति कीर्तनात्} इति श्रुतकीर्तननाश्यत्वाभावप्रसङ्गाच्च~। न च विहितत्वमात्रं लक्षणमिति वाच्यम्; विवाहार्थमनृतवदनादेरभ्यनुज्ञाविधिविषयस्य धर्मत्वप्रसङ्गादित्यभिप्रायेण
चोदयति - {\br अथ क इति~।} क इति किं यागादिरेव धर्मः किंवा चैत्यवन्दनादिकमपीत्यर्थः~। किमिति धर्मलक्षणस्याक्षेपः स च निर्दिष्टः~। समाधत्ते - {\br उच्यत इत्यादिना~।} यागादिरेवेत्येवकारेण चैत्यवन्दनादेर्धर्मत्वं वारयति~। न चैत्यवन्दनादिर्धर्मः, तत्र प्रमाणाभावादित्यर्थः~। धर्मस्य लक्षणमाह \textendash\ {\br तल्लक्षणमिति~। प्रयोजन इति~।} वेदप्रतिपाद्ये स्वर्गादिफलेऽर्थरूप इत्यर्थः~।~{\br प्रयोजनवदितीति~।} स्वर्गादिफलस्य सुखादिरूपत्वेन तत्प्रयोजनान्तराभावाद्भवति वारणमिति भावः~। {\br भोजनादाविति~।} तृप्त्यादिप्रयोजनवत्यर्थरूप इत्यर्थः~। {\br वेदप्रतिपाद्य इतीति~।} भोजनादेः रागादिनैव प्राप्तत्वात्~।~{\qt अष्टौग्रासा मुनेर्भक्ष्याः षोडशारण्यवासिनाम्~॥ द्वात्रिंशत्तु गृहस्थस्य यथेष्टं ब्रह्मचारिणाम्} इत्यादिवचनस्य च ग्रासादिनियमपरत्वाद्भवति तेन तद्वारणमिति भावः~।~{\br श्येनादाविति~।}
\newpage
%%%%%%%%%%%%%%%%%%%%%%%%%%%%%%%%%%5
\fancyhead[LO]{प्रश्नः ]}
\noindent
{\qt श्येनेनाभिचरन् यजेत} इत्यादिवेदप्रतिपाद्ये वैरिमरणानुकूलशस्त्रघातादिरूपहिंसात्मकाभिचारस्वरूपप्रयोजनवतीत्यर्थः~।~\\

 {\br ननु} {\qt अर्थ} पदस्य श्येनकर्मणि न धर्मलक्षणस्यातिव्याप्तिवारकत्वं, श्येनस्यार्थत्वात्~। न हि श्येनो नरकं जनयति, येनानर्थः स्यात्, नरकजनकस्यैवानर्थत्वात् , श्येनस्य तु शत्रुवधमात्रजनकत्वात्~। किंच चतुर्थ श्येनस्येष्टसाधनत्वेन वेदबोधितत्वाद्धर्मत्वमेवोक्तम्~। तत्फलस्यैव हिंसात्मकाभिचारस्य नरकरूपानिष्टजनकत्वेनाधर्मत्वमुक्तम्~। न च तत्रैवातिव्याप्तिवारकं भवत्विति सांप्रतम्~। फले विध्ययोगेन तस्य चोदनागम्यत्वाभावात्~। अन्यथा {\qt विधिस्पृष्टे निषेधानवकाशात्} इति न्यायेन तस्य {\qt न हिंस्यात्} इति निषेधाविषयत्वेन नरकजनकत्वानापत्तिः, निषिद्धस्यैव तज्जनकत्वात्~। तस्मादर्थपदं व्यर्थमेवेत्यत आह\textendash\ {\br अनर्थफलकत्वादनर्थभूत इति~।} श्येनफलस्य शत्रुवधस्य नरकजनकत्वेनानर्थत्वाच्छ्येनोऽपि तद्वारानर्थ एव, तस्यापि शत्रुवधद्वारा नरकजनकत्वादिति भावः~। न च चतुर्थविरोधः ; तत्र साक्षादिष्टसाधनत्वेन वेदबोधितवधमात्रमभिप्रेत्य धर्मत्वस्योक्तत्वात्~। अन्यथा सौत्रार्थशब्दविरोधापत्तिः,
व्यावर्त्यान्तराभावात्~। नहि व्यवधानेन कार्यजनके जनकत्वव्यवहाराभावः~। व्यवधानेनाप्यनुमित्यादिजनके व्याप्त्यादिज्ञाने तद्दर्शनात्~।~\\

 {\br ननु} न श्येनस्यानर्थरूपत्वं संभवति ; तस्य चोदनागम्यत्वात्~। न च सौत्रार्थशब्दस्य वैयर्थ्यमिति वाच्यम् ; तत्फलव्यावर्तकत्वेनाप्युपपत्तेः ~। अन्यथा फलस्य वेदप्रतिपाद्यन्वाभावे तदुद्देशेन यागादिषु प्रवृत्त्यभावप्रसङ्गः~। न च प्रमाणान्तरोपस्थितफलोद्देशेन तत्र सेति वाच्यम् ; इन्द्रियागोचरेऽर्थे  प्रमाणान्तराभावात्~। तस्माच्छ्येनफलस्यापि वेदप्रतिपाद्यत्वेन शत्रुराज्यादिग्रहणप्रयोजनवत्त्वेन च ग्रन्थकारोक्तधर्मलक्षणलक्षितत्वादर्थपदेन वारणं युक्तमिति चेत्,- न ; {\ab  चोदनालक्षणोऽर्थो धर्मः} इति सौत्रधर्मलक्षणे {\qt चोदनाः} पदेन तद्वारणात्~। वस्तुतस्तु शत्रुवधरूपाभिचारस्य श्येनफलस्य लोकतः प्राप्तत्वात्तत्र रागतः प्रवृत्तं पुरुषं प्रति श्येनस्य तत्साधनत्वमात्रं
वेदेन बोध्यत इति न तस्य वेदप्रतिपाद्यत्वम्~। ततश्च तस्य तेनैव वारणेऽर्थपदस्य श्येनवारकत्वेनैव सार्थक्यमिति ध्येयम्~। यत्तु-श्येनादौ धर्मत्वाभावे तामसधर्मत्वकथनानुपपत्तिरिति,-तन्न ; तस्य तामसत्वकथने नैवानर्थकत्वोपपत्तेः~।~प्रसिद्धं हि लोके क्रौर्यादिपुरःसरं तामसक्रियाया अनर्थरूपत्वमित्यलम्~।
तस्माद\blfootnote{पाठा०\textemdash\ $^{१}${\qt अर्थवत्त्वे}.}\footnotemark र्थत्वे प्रयोजनवत्त्वे च सति
वेदप्रतिपाद्यत्वं धर्मत्वमिति धर्मलक्षणमुपपन्नम्~।
\newpage
%%%%%%%%%%%%%%%%%%%%%%%%%%%%%%%%%%%
\fancyhead[RE]{[ वेदस्य धर्म\textemdash\ }
\begin{center}
\textbf{वेदस्य धर्मप्रतिपादकत्वम्}
\end{center}

{\bl न च {\al चोदनालक्षणोऽर्थो धर्मः} इति सौत्रतल्लक्षणविरोधः चोदनापदस्य विधिरूपवेदैकदेशपरत्वादिति वाच्यम्~। तत्रापि चोदनाशब्दस्य वेदमात्रपरत्वात्~। वेदस्य सर्वस्य धर्मतात्पर्यवत्त्वेन धर्मप्रतिपादकत्वात्~।}\\
\hrule
\vspace{3mm}

यत्तु-विहितद्रव्यादावव्याप्तिरिति, -तन्न; दध्यादेरर्थत्वस्येन्द्रियादिप्रयोजनवत्त्वस्य वेदप्रतिपाद्यत्वस्य च सत्त्वात्~। यत्तु {\qt स्त्रीषु धर्मविवाहेषु वृत्त्यर्थे प्राणसंकटे~।
गोब्राह्मणार्थे हिंसायां नानृतं स्याज्जुगुप्सितम्} इत्याद्यभ्यनुज्ञाविधिविषयेऽनृतवदनादौ धर्मत्वापत्तिरिति,  -तदपि न; तत्र निरुक्तस्य धर्मलक्षणस्यापत्त्यभावात्~।
अभ्यनुज्ञाविधिना हि दोषाभावमात्रस्याक्षेपात्प्रयोजनवत्त्वस्य चानाक्षेपाद्रागप्राप्तप्रयोजनसाधनत्त्वस्याप्यनृतवदनादेर्वेदबोधितप्रयोजनसाधनताकत्वाभावान्न धर्मत्वापत्तिः~। तस्माल्लक्षणे न कोऽपि दोष इति सिद्धम्~। इदमधर्मस्याप्युपलक्षणम् ; तस्यापि प्रासङ्गिकशास्त्रविषयत्वात्~। तथा चोक्तम्\textendash\ {\qt धर्मस्योपक्रान्तत्वेऽपि प्रसङ्गात्प्रतिषेधचोदनार्थो निरूपितः} इति~। तथा खण्डदेवेनाप्युक्तम्\textendash यद्यपि {\qt धर्मः क्षरति कीर्तनात्}  इत्यादौ, वैशेषिकतन्त्रे च {\qt क्रियाजन्यादृष्टे
धर्माधर्मशब्दप्रयोगस्तथापि {\qt धर्मः स्वनुष्ठितः पुंसाम्} इत्यादौ तज्जनकविहितनिषिद्धक्रियादावपि तच्छब्दप्रयोगात्ताविह प्राधान्येन विचार्येते} इति तथा च वेदबोधितानिष्टसाधनताकत्वमधर्मत्वमित्यधर्मलक्षणं सिद्धम्~। अत्रानिष्टसाधनताकत्वं विषभक्षणादेरप्यस्तीति तद्वारणाय-{\br वेदेति~।} ब्रह्मयागादेरपि वेदबोधितत्वमस्तीति तद्वारणाया{\br निष्टेति~।} एवं धर्मस्य लक्षणमुक्तम्~।~\\

 तत्र च सौत्रचोदनापदपरित्यागेन वेदपदप्रदानं सूत्रविरुद्धमित्याशङ्क्य परिहरति-{\br न चेत्यादिना~।} न च वाच्यमित्यत्र हेतुमाह\textendash\ {\br तत्रापीति~।}
तत्रापि सूत्रेऽपि~।~{\br वेदमात्रपरत्वादिति~।} चोदनाप्रकरणपठितकृत्स्नवेदपरत्वादित्यर्थः~।~तेन न ब्रह्ममीमांसाविरोधः ~। नापि सौत्रचोदनापदविरोधः, चोदनाशेषार्थवादादेर्वेदस्य स्वप्रकरणपठितस्य तया गृहीतत्वात्प्रकरणान्तरपठितस्य ब्रह्मवाक्यस्य ग्रहीतुमशक्यत्वेऽपि~।  ननु {\qt सोऽरोदीद्यदरोदीत्तद्रुद्रस्य रुद्रत्वं, स प्रजापतिरात्मनो वपामुदस्विदत्} इत्यादिवाक्यानां धर्मप्रतिपादकत्वादर्शनात् कथं 
\newpage
%%%%%%%%%%%%%%%%%%%%%%%%%%%%%%
\fancyhead[LO]{प्रतिपादकत्वम् ]}
{\bl 
स च यागादिः {\al यजेत स्वर्गकाम} इत्यादिवाक्येन स्वर्गमुद्दिश्य पुरुषं प्रति विधीयते~। तथा हि\textendash यजेतेत्यत्रास्त्यंशद्वयं}\\
\hrule
\vspace{3mm}

\noindent चोदनापदस्य यागादिधर्मविधायकस्य वेदपरत्वमित्याशङ्क्य, विधिशेषस्य स्तुत्यादेः प्रतिपादकत्वेन सर्वस्यापि तादृशवेदवाक्यस्य धर्मतात्पर्यकत्वान्मैवमित्याह \textendash {\br सर्वस्येति~।~ननु} {\qt चोदनालक्षणोऽर्थो धर्मः} इत्यत्र सूत्रेऽर्थत्वे सति चोदनागम्यत्वं धर्मलक्षणं, प्रत्यक्षाद्यगोचरेऽपि धर्मे चोदनागम्ये गमकं चोदनावाक्यमेव प्रमाणमिति प्रतीयते~। तच्चायुक्तम् ; एकसूत्रवाक्यस्य स्वरूपप्रमाणपरत्वे वाक्यभेदप्रसङ्गादिति चेत् सूत्रस्यार्थतो धर्मलक्षणत्वेऽपि मुखतः प्रमाणपरत्वात्~। तथा चोक्तम्\textendash {\qt धर्मलक्षणपरं सूत्रमर्थात्प्रमाणप्रतिज्ञा} इति प्राभाकराः~। {\qt मुखतः प्रतिज्ञार्थाद्धर्मलक्षणत्वमिति वार्तिककारीया} इति~।\\

 {\br ननु} पूर्वं प्रयोजनवत्त्वमपि धर्मलक्षणे विशेषणं दत्तं, ततश्च किं तत्प्रयोजनं ? किंच तदुद्देशेन धर्मविधायकं चोदनावाक्यमिति वीक्षायामाह\textendash {\br स चेत्यादिना~।} यद्यपि यथा प्रत्यक्षादीनां धर्मे न प्रामाण्यं तथा चोदनावाक्यस्यापि न तत्र प्रामाण्यं संभवति, यतः शक्तिग्रहणपूर्वकं लोके ह्याप्तवाक्यस्य प्रामाण्यं दृष्टम् , शक्तिश्च लोकप्रसिद्धे गवादौ गृह्यते, धर्मस्य चालौकिकत्वात्तत्र शक्तिग्रहणं न संभवति, शक्तिग्रहणमन्तरेण च हुंफडादिवच्चोदनावाक्यस्यापि धर्माबोधकत्वान्न तत्र प्रामाण्यम्~। तथापि {\qt प्रभिन्नकमलोदरे मधूनि मधुकरः पिबति} इत्यत्र वाक्ये यथा मधुकरपदस्यार्थमजानन् तदन्यपदार्थांश्च जानन् तत्समभिव्याहारात्कमलमध्यगते मधुपानं कुर्वति दृश्यमाने भ्रमरे मधुकरशब्दस्य सङ्गतिं गृहीत्वा वाक्यार्थं प्रतिपद्यते~। तथा {\qt कारीर्या वृष्टिकामो यजेत} इत्यादौ लोकप्रसिद्धार्थवृष्ट्यादिपदसमभिव्याहारादलौकिकेऽपि भावनापदार्थे चोदनायाः सङ्गतिं गृहीत्वा चोदनावाक्यार्थं प्रतिपद्यत इति धर्मबोधकत्वाच्चोदनाया धर्मे नाप्रामाण्यमस्ति~। तथा धर्मस्यालौकिकत्वेन प्रमाणान्तरागोचरत्वाद्वेदस्य च तत्र स्वतःप्रामाण्याभ्युपगमान्न चोदनाया धर्मबोधने मानान्तरसापेक्षत्वमपि~। तस्मादप्रामाण्यकारणयोरबोधकत्वसापेक्षत्वयोरसंभवाच्चोदनायाः सिद्धं स्वतःप्रामाण्यं धर्मे~। ततश्च
विधायकत्वमुपपन्नमित्यभिप्रेत्योक्तं\textendash {\br स्वर्गमुद्दिश्य पुरुषं प्रति विधीयत इति~।} एतेन विध्यादेर्धर्मे
\newpage
%%%%%%%%%%%%%%%%%%%%%%%%%%%%%%%%
\fancyhead[RE]{[ भावनाविचारः ]}
{\bl \noindent  यजिधातुः प्रत्ययश्च~। प्रत्ययेऽप्यस्त्यंशद्वयं आख्यातत्वं लिङ्त्वं च~। तत्राख्यातत्वं दशलकारसाधारणं, लिङ्त्वं पुनर्लिङ्मात्रे~।~}
\begin{center}
 \textbf{भावनाविचारः}
\end{center}
 
{\bl उभाभ्यामप्यंशाभ्यां भावनैवोच्यते \textendash {\al भावना नाम भवितु} }\\
\hrule
\vspace{3mm}
\noindent
प्रामाण्यं प्रथमाध्यायार्थो ध्वनितः~। मानसविषयत्वाकारेण स्वर्गं सिद्धवन्निर्दिश्य तत्साधनत्वेनाज्ञातस्य यागस्यानुष्ठेयत्वं प्रतिपाद्यत इति तदर्थः~। तथा चोक्तम् {\qt फलस्योद्देश्यत्वं नाम मानसापेक्षो विषयत्वाकारः} इति~।\\

 {\br ननु} {\qt यजेत स्वर्गकामः} इत्यादौ साधनत्ववाचकशब्दस्यादर्शनात्कथं स्वर्गसाधनत्वेन वेदेन यागस्यानुष्ठेयत्वं प्रतिपाद्यत इत्याशङ्क्य प्रकृतिप्रत्यययोर्विभागपुरःसरं प्रत्ययस्यांशविवेकेन भावनां प्रतिपादयन् तत्सामर्थ्येन यागस्य स्वर्गसाधनत्वं दर्शयति {\br तथा ही}त्यारभ्य {\br अथ क} इत्यतः प्राक्तनेन ग्रन्थेन~।~यद्वा {\br ननु} न यागादीनां स्वर्गसाधनत्वं स्वर्गस्य कालान्तरभावित्वादपूर्वमन्तरेण तन्निष्पादकत्वासंभवात्~। न च यागादीनामपूर्वनिष्पादकत्वं स्यादिति वाच्यम्~। सिद्धस्यैव लोके
साध्यनिष्पादकत्वदर्शनात्साध्यस्वभावस्य यागदानादिरूपस्य भावार्थस्यापूर्वनिष्पादकत्वासंभवात्~। तस्मान्न यागादेः स्वर्गसाधनत्वम्~। ततश्च न तदुद्देशेन यागादिविधिरिति चेन्न~। क्रियामन्तरेण द्रव्यादेः सिद्धस्यापि लोके फलविशेषसाधनत्वादर्शनात्~। नहि पचिक्रियामन्तरेण काष्ठस्थाल्यादीनामोदनसाधनत्वं दृश्यते~। माभूत्तर्हि न तावता
भवदिष्टसिद्धिरिति चेत्~। साध्यस्यापि भावार्थस्य यागादेरेकपदोपात्तत्वेन भावनाभाव्यनिर्वृत्तिद्वारेण भावनाकरणस्य स्वसाधननिष्पादितस्य सतोऽपूर्वद्वारा भावनाभाव्यस्वर्गनिष्पादकत्वादित्यभिप्रायेण भावनां निरूपयितुं विधेर्विधायकत्वप्रकारं प्रदर्शयितुं च प्रकृत्यादिकं विभजते- {\br तथा हीत्यादिना}~। {\br तत्रेति~।} आख्यातत्वलिङ्त्वयोर्मध्य इत्यर्थः~।\\

 {\br उभाभ्यामिति~।} आख्यातत्वलिङ्त्वाभ्यामित्यर्थः~। भावनैवेत्येवकारेण कर्त्रादिवाचकत्वमाख्यातस्य वारयति~। भावनासामान्यं लक्षयति\textendash  {\br भवितुरिति~।} भवितुरुत्पद्यमानस्योत्पत्त्यनुकूलो भावयितुरुत्पादयितुः प्रयोजकस्य व्यापारविशेषो भावनेत्यर्थः~। प्रयोजकव्यापारत्वादेव णिजन्तेन {\qt भावना}शब्दे
\newpage
%%%%%%%%%%%%%%%%%%%%%%%%%%%%%%%%
\fancyhead[LO]{[ शाब्दी भावना ]}
{\bl\noindent 
{\al र्भवनाकूलो भावयितुर्व्यापारविशेषः}~। सा द्विधा - शाब्दीभावना आर्थी भावना चेति~।}
\begin{center}
 \textbf{शाब्दीभावना   }
\end{center}
 
{\bl तत्र {\al पुरुषप्रवृत्त्यनुकूलो भावयितुर्व्यापारविशेषः शाब्दीभावना}~। सा च लिङंशेनोच्यते~। लिङ्श्रवणेऽयं मां प्रवर्तयतिमन्प्रवृत्त्यनुकूलव्यापारवानयमिति नियमेन प्रतीतेः~। यद्यसाच्छब्दान्नियमतः प्रतीयते तत्तस्य वाच्यम्~।~यथा गामानये\textendash\ }\\
\hrule
\vspace{3mm}
\noindent
नोच्यते~। यथोत्पद्यमानस्यौदनस्योत्पत्त्यनुकूलो देवदत्तस्य व्यापारविशेषो भावनेत्यर्थः~। यथा चोत्पद्यमानाया देवदत्तप्रवृत्तेरुत्पत्त्यनुकूलः प्रवर्तकस्य चैत्रस्याभिप्रायविशेषः~। यथा वा {\qt यजेत स्वर्गकामः} इत्यत्रोत्पद्यमानस्य धात्वर्थस्य स्वर्गस्य वोत्पत्त्यनुकूलः स्वर्गकामस्य व्यापार उत्पद्यमानायाश्च स्वर्गकामप्रवृत्तेरुत्पत्त्यनुकूलो लिङ्गो व्यापारविशेषः~।तथा चान्योत्पादनानुकूलो भावुकस्य व्यापारविशेषो धात्वर्थादन्यः सर्वधात्वर्थसंबद्धाकारेण भासमानो भावनासामान्यमिति सिद्धम् ~। तथा चोक्तम् \textendash {\qt अन्योत्पादानुकूलात्मा भावना साध्यरूपिणी} इति~। तथाचार्यैरप्युक्तम् \textendash {\qt धात्वर्थव्यतिरेकेण\blfootnote{पाठा०\textemdash\ $^{१}${\qt अनुभाव्य}.}\footnotemark\ यद्यप्येषा न लभ्यते~। तथापि सर्वसामान्यरूपेणैवावगम्यते~॥} इति~।~भावनां विभजते - {\br सा द्विधेति~।} तत्र शब्दभावनां लक्षयति {\br तत्रेति~।} तत्र तयोः शब्दभावनार्थभावनयोर्मध्य इत्यर्थः~। परिस्पन्दपरिणामविलक्षणः पुरुषप्रवृत्त्यात्मकार्थभावनोत्पत्त्यनुकूलो लिङादिशब्दस्य व्यापारविशेषः
शब्दभावनेत्यर्थः~। शब्दभावनैव लिङ्त्वादिना लिङाद्यर्थ इत्याह\textendash\ {\br सा चेति~।} तस्या लिङाद्यर्थत्वेऽनुभवं प्रमाणयति- {\br लिङ्श्रवण इति~।}
अनुभवमभिनयति-\ {\br मदिति~।} यद्वा, {\br ननु} कथमननुभूयमानत्वाल्लिङादिवाच्यत्वं भावनाया लिङादेः प्रवर्तकत्वेऽपि तत्र भावनारूपव्यापारस्याननुभवादित्याशङ्क्य तं व्यापारं स्पष्टतयानुभावयति- {\br मदित्यादिना~।} पुरुषप्रवृत्त्यनुकूलं व्यापारं लिङादिशब्दनिष्ठतयानुभावयित्वा तस्य लिङादिशब्दान्नियमेन प्रतीयमानत्वाल्लिङादिशब्दवाच्यत्वमित्यनुमानप्रदर्शनाय व्याप्तिं दर्शयति\textendash {\br यदित्यादिना~।} तत्रोदाहरणमाह\textendash {\br यथेत्यादि~।} सा
\newpage
%%%%%%%%%%%%%%%%%%%%%%%%%%%%%%%%%%%
\fancyhead[RE]{[ शाब्द्या लौकिक\textemdash\ }
{\bl\noindent त्यस्मिन्वाक्ये गोशब्दस्य गोत्वम्~॥ }
\begin{center}
 \textbf{शाब्द्या लौकिकवैदिकभेदौ    }
\end{center}
 
{\bl स च व्यापारविशेषो लौकिकवाक्ये पुरुषनिष्ठोऽभिप्राय }\\
\hrule
\vspace{3mm}
\noindent
च शब्दभावना लोके वेदे च प्रवर्तनात्वेनैव लिङादिशब्दवाच्या तत्त्वेनैव च पुरुषप्रवृत्तिहेतुरिति स्वीकर्तव्यम्~। अन्यथा प्रैषादेरनेकस्य पुरुषाशयविशेषस्य विधिवाच्यत्वानुपपत्तिः स्यात्, आनन्त्यव्यभिचारदोषप्रसङ्गात्~। प्रवर्तनात्वं च प्रवृत्त्यनुकूलव्यापारत्वम्~। अनुकूलत्वं च जनकत्वम्~। तच्च लोके पुरुषाशयवृत्ति, वेदे तु पुरुषाभावात्पुरुषाशयभिन्नस्यैव कस्यचिल्लिङादिशब्दनिष्ठव्यापारविशेषस्य प्रवर्तनात्वमित्याशयेनाह\textendash  {\br स चेत्यादिना~। अत्रेदं बोध्यम् }  तस्य व्यापारविशेषस्य प्रवृत्तिविषयस्येष्टसाधनत्वानुमानद्वारा प्रवृत्तिजनकत्वमङ्गीकरणीयम्~। अन्यथा प्रवृत्तिविषयस्येष्टसाधनत्वानाक्षेपे प्रवृत्त्यनुपपत्तिः स्यात्~। तथा चानुमानम् {\qt विमतमिदमिष्टसाधनम् , अप्यत्र प्रवर्ततामित्याकारकासप्तेष्टव्यापारविशेषविषयत्वात्, यन्नैवं तन्रैवं, यथा प्रतारकवाक्योपस्थितम्~}। {\qt अत्र सुखे सुखं मे जायताम्} इत्युदासीनस्य कस्यचिदिच्छाविषयत्वेऽपीष्टसाधनत्वाभावात्तत्र व्यभिचारव्यावृत्तये हेतावाकारकान्तम्~।~प्रतारकस्य तादृशेच्छाविषये व्यभिचारवारणायाप्तेति~। आप्तत्वं च लोकवेदसाधारणं प्रतारणाद्यजन्यहिताहितोपदेशर्कर्तृत्वे सति तद्भिन्नोपदेशाकर्तृत्वम्~। प्रतारणया तु सर्वदा हिताहितोपदेशकर्तर्यनाप्तेऽतिप्रसङ्गवारणायाजन्यान्तम्~।
कदाचित्प्रतारणाद्यजन्यतत्कर्तरि तद्दोषवारणायोत्तरदलम्~। हितस्योपदेशस्तत्संग्रहायाहितस्योपदेशश्च तत्परिहाराय बोध्यः~। ततश्च तत्कर्तृत्वं लोके पुरुषविशेषे वेदे च {\qt यजेत स्वर्गकामः, न कलज्जं भक्षयेत्} इत्यादिवाक्ये भवति~। तथा च लौकिकवैदिकव्यापारयोर्व्यापारविशेषत्वेन संग्रहाय हेतौ विशेषपदम्~। वैदिकश्च स व्यापारविशेषः प्रवर्तनाप्रेषणाविध्यपरपर्याया भावनैव नञ्रहिते वाक्ये लिङाद्यर्थः~। लौकिकस्तु प्रैषोऽतिसर्गः प्रेषणाज्ञाध्येषणानुज्ञानुमितिरित्यादिबहुविधो भवति~। प्रमाणान्तरप्रमितेऽर्थे पुरुषनिष्ठा पुरुषप्रवर्तना प्रैषः~। अतिसर्गः कामचारः~। उत्कृष्टस्य निकृष्टं प्रति प्रवर्तना प्रेषणाऽऽज्ञा चोच्यते~। निकृष्टस्योत्कृषटं प्रति प्रवर्तना प्रार्थनाध्येषणा चोच्यते समं प्रति
समस्य प्रवर्तनोत्कर्षनिकर्षौदासीन्येन जातानु- \newpage
%%%%%%%%%%%%%%%%%%%%%%%%%%%%%%%
\fancyhead[LO]{वैदिकभेदौ ]}
\noindent
ज्ञानुमतिश्चोच्यते~। ते च प्रैषादयो ज्ञानविशेषा इच्छाविशेषा वा चेतनधर्माः~। पुरुषस्याशयविशेषा एवेत्यभिप्रायेण ग्रन्थकारेणाप्युक्तम्\textendash लौकिकवाक्ये पुरुषनिष्ठोऽभिप्रायविशेष इति~।~तथा च ते एव लोके लिङाद्यर्थाः~। तस्माल्लोके वेदे च व्यापार एव प्रवर्तनाख्यो लिङादिवाच्योऽर्थ इति फलितम्~। {\br ननु} किमत्र वाच्यताख्यं- शक्यतावच्छेदकं शक्ततावच्छेदकं च ? अन्यथातिप्रसङ्गापत्तेरिति चेत् , अत्रोच्यते\textendash लौकिके हि प्रैषादौ वैदिके च भावनारूपे व्यापारे साधारणव्यापारत्वमेव पूर्वोक्तप्रवर्तनात्वरपं शक्यतावच्छेदकमस्ति च लौकिके लिङादिप\footnotemark दोपस्थाप्येतत्प्रवृत्तिहेतुभूतेष्टसाधनताद्यनुमितिजनकं
पुरुषाशयविशेषे प्रवृत्तिप्रयोजकव्यापारत्वं वेदेऽपि लिङादिशब्दश्रवणात्तदुत्तरकाले यागादिप्रवृत्तिदर्शनेनेयं देवदत्तस्य यागादिप्रवृत्तिः व्यापाराख्यप्रवर्तनाज्ञानपूर्विका
अन्यप्रेरितप्रवृत्तित्वाच्चैत्राशयज्ञानजन्यमैत्रगवानयनप्रवृत्तिवदित्यलौकिकमेव व्यापारमपौरुषेये वेदे इच्छदेर्बाधादनुमाय तत्प्रतीतेर्लिङादिज्ञानान्वयव्यतिरेकानुविधायित्वेन तत्र लिङादिवाच्यत्वं च परिकल्प्य तस्य पूर्वोक्तविधया प्रवृत्तिहेतुभूतेष्टसाधनताद्यनुमापकतया बालस्तत्रापि प्रवृत्तिप्रयोजकव्यापारत्वं प्रतिपद्यते~। कथं तर्हि लोकेऽपि
व्यापारप्रतिपत्तिरिति चेत् ,-इत्थम्\textendash उत्तमवृद्धस्य सविधिकवाक्यश्रवणोत्तरकालभाविनीं मध्यमवृद्धस्य गवानयनप्रवृत्तिमुपलभ्य बालो लोकेऽपि तत्प्रयोजकव्यापारमनुमिमीते~। तथा हि\textendash गवानयनानुकूलोत्तमवृद्धवाक्यश्रवणोत्तरभाविनी मध्यमवृद्धप्रवृत्तिः प्रवर्तनाज्ञानपूर्विका अन्यप्रेरितप्रवृत्तित्वान्मद्रोदनपूर्वकमदीयभोजनादौ मदभिप्रायजन्यमन्मातृप्रवृत्तिवत्~। अत्र स्वतःसिद्धप्रवृत्ताविष्टसाधनताज्ञानजन्यस्वप्रवृत्तौ च व्यभिचारवारणायान्यप्रेरितेति विशेषणम्~। किंच प्रवर्तनाज्ञानमुत्तमवृद्धवाक्यजन्यं तदन्वयव्यतिरेकानुविधायित्वाद्दण्डान्वयव्यतिरेकानुविधायिघटवत्~। किंच यस्माच्छब्दाद्यत्प्रतीयते तत्तद्वाच्यं
घटपदवाच्यघटत्ववदित्युत्तमवृद्धवाक्यस्य मुग्धाकारां शक्तिं व्यापाराख्यप्रवर्तनायामवधार्य तत्र चावापोद्वापाभ्यां विधिशक्तिं तस्यामवधारयति, एवं सर्वत्रोह्यम्~। तस्माल्लोकवेदसाधारण्येन व्यापारत्वमेव शक्यतावच्छेदकमिति सिद्धम्~। शक्ततावच्छेदकं तु लिङ्त्वलेट्त्वलोट्त्वादिकं बोध्यम्~। किंच यद्यपि निरुक्तविधया व्यापारवेनैव सामान्यरूपेण व्यापारज्ञानं तच्च विशेषज्ञानसापेक्षमतिप्रसङ्गवारणाय तथापि प्रमाणान्तरेण विशेषबोधः सुलभः~। यथा घटवद्भूतलमित्यादौ घटपदाद्घटत्वावच्छिन्नघटमात्रप्रतीतावपि तस्य योग्यसंसर्गेण भूतला- 
\blfootnote{पाठा०\textemdash\ $^{१}${\qt ०पदे यस्याप्येतत्}}
\newpage
%%%%%%%%%%%%%%%%%%%%%%%%%%%%%%%%%
\fancyhead[RE]{[ शाब्द्या लौकिक\textemdash\ }
\noindent
दावन्वये बुद्धे तत्संसर्गस्य तत्त्वेन जिज्ञासायां प्रत्यक्षादिप्रमाणान्तरेणैव संयोगत्वादिना संयोगादिरूपसंसर्गप्रतीतिर्भूतलवृत्तिघटविशेषप्रतीतिश्च भवति, तथा
लिङादिपदाद्व्यापारत्वावच्छिन्नव्यापारमात्रस्योपस्थितावपि तस्याख्यातोपात्तार्थभावनाया योग्यसंसर्गेणान्वये बुद्धे पश्चात्तत्संसर्गस्य तत्त्वेन जिज्ञासायां प्रवृत्तिप्रयोजकत्वानुपपत्त्यादिना प्रमाणेनैव विशेषरूपेण संसर्गविशेषप्रतीतिर्यागादिप्रवृत्तिसंबन्धिव्यापारविशेषप्रतीतिश्च भवति, संसर्गविशेषस्तु तत्तत्प्रवृत्तिप्रागभावकाले यल्लिङादिपदज्ञानं तेनोत्पादितं यत्प्रेरणाज्ञानं तज्जन्येष्टसाधनताद्यनुमितिप्रयोज्यत्वम्~। तेन च व्यापारवती यागादिप्रवृत्तिरिति भवति विशेषनिर्णयः~। एवं च लोके वेदे
च लिङादिश्रवणे प्रैषादिरूपस्य वक्त्रभिप्रायस्य भावनारूपस्य च व्यापारविशेषस्य व्यापारत्वेनैव रूपेण प्रतीतिर्न विशेषरूपेण, तथैव शक्तिग्रहात् विशेषरूपेण प्रतीतिस्तु लोकेऽजहल्लक्षणयैव~। वेदे तु विशेषरूपाकाङ्क्षायां प्रैषादिरूपस्य वक्त्राशयविशेषस्यापौरुषेये वेदेऽसंभवेन लिङादिशब्दनिष्ठ एव प्रेरणापरपर्यायः कश्चिद्व्यापारो विशेषरूप
इत्युक्तमेव, तस्मान्निरुक्तव्यापार एव लिङाद्यर्थो नेष्टसाधनत्वादिरिति सिद्धम्~। {\br नन्वि}ष्टसाधनत्वमेव प्रवर्तनात्वेन रूपेण वेदे लिङाद्यर्थ इति मण्डनमिश्रा वदन्ति~। अर्थभावनाभिधानानुकूलाया लिङादिनिष्ठशक्तेरेवाभिधाख्यायाः प्रवर्तनात्वेन रूपेण लिङादिवाच्यत्वं परिकल्प्य तज्ज्ञानस्य प्रवृत्तिं प्रति कारणत्वमात्रं वेदे कल्प्यत इति तु
पार्थसारथिर्वदति लिङादिश्रवणानन्तरं प्रवृत्तिदर्शनात्प्रवृत्तिसामग्रीजननद्वारा लिङादिज्ञानस्य प्रवृत्तावुपयोग इति तावदविवादम्~। तत्सामग्री च कृतिसाध्यत्वप्रकारकेच्छारूपा चिकीर्षा, तस्याश्च स्वरूपसत्याः कारणत्वात्तत्र लिङादिज्ञानस्यानुपयोगेऽपि चिकीर्षाकारणीभूतज्ञाने तदुपयोगः कल्प्यते~। तच्च बलवदनिष्टाननुबन्धित्वज्ञानं कृतिसाध्यत्वज्ञानमिष्टसाधनत्वज्ञानं च अन्यतमाभावे इतरद्वयसत्त्वेऽपि मधुविषान्नभोजने चन्द्रस्पर्शे मण्डलीकरणादौ वा प्रवृत्त्यनुत्पतेः~। तस्माल्लिङादिज्ञानेन त्रितयज्ञानजननाल्लिङादेर्बलवदनिष्टाननुबन्धित्वे कृतिसाध्यत्वे इष्टसाधनत्वे च शक्तिरिति तार्किकाः, तस्मात्कथं व्यापारस्य लोकवेदसाधारणस्य लिङाद्यर्थत्वमिति चेत्, अत्रोच्यते\textendash न तावदिष्टसाधनत्वं लिङाद्यर्थः, इष्टसाधनत्वज्ञानादेव प्रवृत्त्युपपत्तौ गुरुप्रेरितोऽहं जलमानयामीत्यादौ गुर्वादेः प्रवर्तकत्वव्यवहारानुपपत्तेः~। न च
प्रवृत्तिजनकेष्टसाधनताबोधकलिङुच्चारयितृत्वात्तस्य प्रवर्तकत्वव्यवहार इति वाच्यम्~। राजप्रेरितपदातेस्तादृशलिङुच्चारयितृत्वेन 
\newpage
%%%%%%%%%%%%%%%%%%%%%%%%%%%%%%%
\fancyhead[LO]{वैदिकभेदौ ]}
{\bl\noindent
विशेषः~। वैदिकवाक्ये तु पुरुषाभावाल्लिङादिशब्दनिष्ठ एव~। अत एव शाब्दी भावनेति व्यवह्रियते~।~}
\begin{center}
 \textbf{शाब्दीभावनाया अंशत्रयम्    }
\end{center}
 
{\bl सा च भावनांशत्रयमपेक्षते साध्यं साधनमितिकर्तव्यतां  च, {\qtl किं भावयेत् , केन भावयेत्, कथं भावयेदिति}~। तत्र साध्याकाङ्क्षायां वक्ष्यमाणांशत्रयोपेता आर्थीभावना साध्यत्वेनान्वेति एकप्रत्ययगम्यत्वेन समानाभिधानश्रुतेः~।}\\
\hrule
\vspace{3mm}
\noindent
प्रवर्तकत्वापत्तौ पदातिप्रेरितोऽहं न गामानयामि किंतु राजप्रेरित इति पदातौ प्रवर्तकत्वाभावव्यवहारानुपपत्तेः~। न चान्याप्रेरितत्वे सति तादृशलिङुच्चारयितृत्वं प्रवर्तकत्वमिति वाच्यम् ; पिशुनप्रेरिते राज्ञि प्रवर्तकत्वानुपपत्तेः~। ततश्च तादृशव्यवहारबलात्प्रवृत्तिकारणीभूतज्ञानविषयाशयविशेषाश्रयत्वेनैव राजादेः प्रवर्तकत्वं वाच्यं न चाशयविशेषस्य लिङादिवाच्यत्वाभावे ततस्तज्ज्ञानं संभवति तस्माल्लिङादिवाच्यत्वं तस्येति~।नाप्यर्थभावनाभिधानानुकूलायाः शक्तेर्लिङाद्यर्थत्वं संख्याभिधानानुकूलशक्त्या लिङ्त्वादिनैव वा
विनिगमनाविरहात्~। नापि बलवदनिष्टाननुबन्धित्वादेर्लिङाद्यर्थत्वं, शक्तित्रयकल्पनायां गौरवप्रसङ्गात्~। बलवदनिष्टाननुबन्धित्वज्ञानाभावेऽपि बलवदनिष्टानुबन्धित्वज्ञानाभावमात्रेणेष्टसाधनत्वज्ञानादिनैव प्रवृत्तेरनुभवसिद्धत्वाच्च~। तस्माद्बलवदनिष्टानुबन्धित्वज्ञानं प्रवृत्तिप्रतिबन्धकं तदभावश्च स्वरूपसन्नेव तत्कारणमिति स्वीर्कर्तव्यम्~। तस्मान्निरुक्तस्य व्यापारस्यैव लिङाद्यर्थत्वं सर्वत्रनेष्टसाधनत्वादेरिति सिद्धम्~।~{\qt अभिधाभावनामाहुरन्यामेव लिङादयः} इति वार्तिकानुरोधेनाह\textendash\ {\br अत एवेति~।} शब्दनिष्ठत्वादेवेत्यर्थः~। {\qt अभिधा} शब्देनाभिधीयतेऽनेनेति व्युत्पत्त्या शब्द उच्यते, तस्य व्यापारविशेषो भावना, तां स्वनिष्ठामन्यामर्थभावनाभिन्नां लिङादय आहुरिति वार्तिकवचनार्थः~।~\\

 सा च निरुक्ता शब्दभावना साध्याद्यंशत्रयापेक्षया तादृशांशत्रयवती भवतीत्याह\textendash {\br सा चेत्यादिना~।} साध्याकाङ्क्षामभिनयति\textendash {\br किमित्यादिना ~। तत्रेति~।} साध्यादिभावनांशेऽपीत्यर्थः~। {\br वक्ष्यमाणेति~।} वक्ष्यमाणा या स्वर्गादिरूपसाध्याद्यंशत्रयोपेतेत्यर्थः~। {\br एकप्रत्ययगम्यत्वेन समानाभिधान- }
\newpage
%%%%%%%%%%%%%%%%%%%%%%%%%%%%%%%%
{\bl 
संख्यादीनामेकप्रत्ययगम्यत्वेऽप्ययोग्यत्वान्न साध्यत्वेनान्वयः~। साधनाकाङ्क्षायां लिङादिज्ञानं करणत्वेनान्वेति~।}\\
\hrule
\vspace{3mm}
\noindent
{\br श्रुतेरिति~।} अर्थभावनाया एव शब्दभावनासाध्यत्वमनयोरेकलिङादिप्रत्ययगम्यत्वेनैकलिङादिप्रत्ययशब्दात्मिकायाः समानाभिधानश्रुतेः सत्त्वादित्यर्थः~। एतेन यागादेः साध्यत्वाशङ्का निरस्ता, तस्यैकपदादिश्रुतिगम्यत्वेनैकप्रत्ययरूपसमानाभिधानश्रुत्यगम्यत्वात्~। शब्दार्थभावनयोस्त्वेकाभिधानश्रुतिगम्यत्वेन संनिकृष्टयोर्भवति विवक्षितः
संबन्ध इति द्रष्टव्यम्~।~\\

 {\br ननु} संख्यादीनामपि शब्दभावनाभाव्यत्वं स्यात्तेषामप्येकप्रत्ययगम्यत्वेन  समानाभिधानश्रुतेरविशेषादितिचोद्यमुद्भाव्य परिहरति- {\br संख्यादीनामित्यादिना~।} आदिना कालादिपरिग्रहः~। संख्यादीनां साध्यत्वेनान्वयाभावे हेतुमाह\textendash {\br अयोग्यत्वादिति~।} अपुरुषार्थत्वेन तत्साधनताशून्यत्वेन च संख्यादीनां साध्यत्वयोग्यतानाश्रयत्वादित्यर्थः~। शब्दभावनाया भाव्यसाकाङ्क्षत्वाद्भाव्यान्तरस्य चादर्शनादर्थभावनायाश्च विधिप्रयोज्यत्वात्पुरुषार्थानुबन्धित्वाच्च तस्या एव समानाभिधानश्रुतेः
शब्दभावनाभाव्यत्वमिति समुदायतात्पर्यम्~।\\

 तदेवमर्थभावनायाः पुरुषार्थहेतुतया शब्दभावनाभाव्यत्वमुक्तम्~।~तत्र च करणाकाङ्क्षायां करणान्तरस्यादर्शनात्करणांशं लिङादिविधिशब्दज्ञानमेवाह\textendash {\br साधनेत्यादिना~।} यद्वा,
लिङादिशब्दव्यापारस्य सर्वदा पुरुषप्रवृत्तिजनकत्वं किं न स्यादित्याशङ्क्य
करणरूपसहकार्यभावान्मैवमित्याशयेनाह \textendash {\br साधनेत्यादिना~।} अत एव ग्रन्थकारेणाप्युक्तं भावना ज्ञापकत्वेनेति~। स्वज्ञानस्यैव स्वज्ञापकत्वम्
~। लिङादीत्यादिना लेट्लोडापरिग्रहः~। तथा च लिङादिज्ञानं शब्दभावनायां करणत्वेनैवान्वयं लभते, न तु शब्दभावनाकरणत्वेनापि कल्पनायां गौरवप्रसङ्गात्~। ज्ञानस्य पुरुषनिष्ठत्वेन शब्दनिष्ठत्वाभावाच्छब्दभावनात्वासंभवप्रसङ्गाज्ज्ञानस्य तृतीयक्षणवृत्तिध्वंसप्रतियोगित्वेन शब्दभावनाया नित्यत्वव्याघातप्रसङ्गच्च ~। शब्दभावना तु निरुक्त एव व्यापारविशेष इति भावः~। लिङादिज्ञानमित्यत्र लिङादिधर्मस्य भावनारूपस्य वा ज्ञानं विवक्षितम्~। तदेव च शब्दभावनायाः करणांशम्~। तथा चोक्तम्\textendash  लिङादिशब्दव्यापारः पुरुषप्रवृत्तिलक्षणार्थभावनालक्षणभाव्यनिष्ठः स्वज्ञानकरणक इति द्रष्टव्यम्~।ज्ञापकत्वमत्र प्रकाशकत्वमेव, तथा च तादृशधर्मज्ञाने सत्येव
\newpage
%%%%%%%%%%%%%%%%%%%%%%%%%%%%%%%%%%%%%%%
\fancyhead[LO]{वैदिकभेदौ ]}
{\bl\noindent
तस्य च करणत्वं न भावनोत्पादकत्वेन, तत्पूर्वमपि तस्याः शब्दे सत्त्वात्~। किंतु भावनाज्ञापकत्वेन शब्दभावनाभाव्य }\\
\hrule
\vspace{3mm}
\noindent
पुरुषप्रवृत्तिदर्शनेन तस्य वक्ष्यमाणपुरुषप्रवृत्तिनिर्वर्तकत्वेन शब्दभावनाकरणत्वं नानुपपन्नमिति भावः~। अन्ये त्वाहुः\textendash विधिशब्दस्य पुरुषप्रवृत्तिरूपार्थभावनाज्ञानहेतुर्व्यापारस्तद्वाचकशक्तिमत्तया विधिशब्दज्ञानं स एव च तस्य प्रवृत्तिहेतुर्व्यापार इति प्रवर्तनाभिधानीयकं लभते, ज्ञानद्वारेणैव शब्दस्य प्रवृत्तिजनकत्वात् ज्ञानजनकव्यापारातिरिक्तव्यापारकल्पने मानाभावात्~। ज्ञानकरणकश्च व्यापारः, तस्य स्वज्ञानं शक्तिज्ञानं शक्तिविशिष्टस्वज्ञानं च~। तत्राद्ययोरन्यतरस्य
शब्दभावनात्वम् , तृतीयस्य तु तत्र करणत्वमिति विवेक इति~। तस्य च लिङादिज्ञानस्य न शब्दभावनोत्पादकत्वेन तत्करणत्वं संभवति~। तस्माल्लिङादिज्ञानात्पूर्वमपि तस्याः शब्दभावनायाः शब्दे विद्यमानत्वेन तदुत्पादकत्वासंभवादित्याह\textendash {\br तस्य चेत्यादिना~। ननु} कथं लिङादिज्ञानस्य भावनासाधकत्वेन तत्करणत्वं न स्वीक्रियते ?
चक्षुरादेस्तत्संनिकर्षस्य वा रूपादिज्ञानसाधकत्वेनैव तत्करणत्वदर्शनादिति चोदयति {\br किंत्विति~?} समाधत्ते-\ {\br भावनेत्यादिना~।} यथा कुठारस्य
छिदिक्रियाभाव्यद्वैधीभावनिर्वर्तकत्वेन छिदिभावनाकरणत्वं तथा लिङादिज्ञानस्य शब्दभावनाभाव्यार्थभावनानिर्वर्तकत्वेन शब्दभावनाकरणत्वमिति भावः~। तथा चोक्तं कुठारादीनामपि छिदिक्रियाभाव्यद्वैधीभावनिर्वर्तनद्वारेण छिदिभावनाकरणत्वदर्शनादिति~। किंच रूपादिज्ञानस्य चक्षुःसंनिकर्षादेः प्रागसत्त्वेन तस्य तत्साधकत्वेन तत्करणत्वं, शब्दभावनायास्तु प्रागपि सत्त्वेन लिङादिज्ञानस्य तत्साधकत्वेन तत्करणत्वासंभवेऽपि तद्भाव्यार्थभावनानिर्वर्तकत्वेन तस्य शब्दभावनाकरणत्वं कुठारवदुपपद्यते~। {\br ननु} कुठारस्य तु छिदिक्रियानिर्वर्तकत्वमपि भवतीति चेत्- सत्यम् ; दृष्टान्तस्तु- {\qt यथा कुठारस्य छेदननिर्वर्तकत्वेऽपि तस्याफलत्वेन छिदिक्रियाभाव्यद्वैधीभावफलनिर्वर्तकत्वेनैव तत्करणत्वमिति द्रष्टव्यम्~}। {\br ननु} पुरुषो लिङादिज्ञानेन स्वप्रवृत्तिं भावयेदित्युक्ते सर्वेषामेव यागादौ प्रवृत्तिः किं न स्यादित्याशङ्कय {\br सर्वेषां} प्राशस्त्यज्ञानाभावेन न
सर्वेषां प्रवृत्त्यापत्तिः, किंतु यस्य पुरुषस्य कर्मप्राशस्त्यज्ञानं भवति तस्यैव तत्फलरागादिना तत्र प्रवृत्तिरित्याशये-
\lfoot{२ अ०}
\newpage
%%%%%%%%%%%%%%%%%%%%%%%%%%%%%%%%%%%%
\lfoot{}
\fancyhead[RE]{[शाब्द्या विशेषविचारः]}
{\bl निर्वर्तकत्वेन वा~ । इतिकर्तव्यताकाङ्क्षायामर्थवादज्ञाप्यप्राशस्त्यमितिकर्तव्यतात्वेनान्वेति~।}\\
\hrule
\vspace{3mm}
\noindent
नाह\textendash\ {\br इतिकर्तव्यतेत्यादिना~।} कर्तव्यस्येतिप्रकार इतिकर्तव्यता, {\qt इति} शब्दस्य प्रकारवाचकत्वात्~। प्रकारश्च सामान्यस्य भेदको विशेष इत्यर्थः~।
तथा च कर्तव्यसामान्यस्य भेदकः कर्तव्यविशेष एव प्राशस्त्यरूपः शब्दभावनायामितिकर्तव्यतात्वेनान्वयं लभते, तच्च कर्तव्यसामान्यं लिङादिज्ञानरूपं भावनाकरणमेव,
करणगतप्रकाराकाङ्क्षापूरकस्येतिकर्तव्यतात्वात्~। लिङादिज्ञानेन भावयेत्कथमित्याकाङ्क्षायां कर्मप्राशस्त्यविशिष्टेनेति प्रकारान्वयात् तस्य च प्राशस्त्यविशेषस्य ज्ञापकोऽर्थवादविशेष एवेति भावः~। स चार्थवादः {\qt स प्रजापतिरात्मनो वपामुदखिदत्} इत्यादिः, तं चार्थवादं चतुर्विधविभागेन निरूपयिष्यामोऽर्थवादनिर्णये~।
\begin{center}
 \textbf{शाब्द्या विशेषविचारः}
\end{center}

{\br ननु} किं नाम प्राशस्त्यं यच्छब्दभावनायामितिकर्तव्यतात्वेनान्वेतीति चेत् विधेयतावच्छेदकसामानाधिकरण्येन बलवदनिष्टाननुबन्धित्वे सति क्रियाजन्यदुः(सु)खापेक्षयाऽधिकेष्ट-जनकत्वं प्राशस्त्यं, तदेव च विध्यर्थवादेषु लक्ष्यते~। निषेधार्थवादेषु तु निषेध्यतावच्छेदकसामानाधिकरण्येन क्रियाफलापेक्षयाधिकदुःखसाधनत्वमप्राशस्त्यं लक्ष्यत इति बोध्यम्~। लक्षणा च सर्वत्र वाक्ये, लक्षणायां बाधकाभावादर्थवादस्थपदसमुदाये वैकस्मिन्नेव वा पदे भवतीतरपदानि तात्पर्यग्राहकाणीत्यनाग्रहः~। तच्च प्राशस्त्यादिकमितिकर्तव्यतात्वसंबन्धेन शब्दभावनायामन्वेतीति बहवो वदन्ति~। स्वरूपसंबन्धेन धात्वर्थादावेवान्वेतीति केचित्~।
वस्तुतस्तु - प्राशस्त्यं स्वविषयकज्ञानजन्येष्टविषयकोत्कटरागजन्यत्वसंबन्धेन प्रवृत्तावन्वेति~। अप्राशस्त्यमपि स्वविषयकज्ञानजन्यानिष्टविषयकोत्कटद्वेषप्रयोज्याभावप्रतियोगित्वसंबन्धेन प्रवृत्तावेवान्वेतीत्यन्यत्र विस्तरः~। इत्थं च पुरुषप्रवृत्तिलक्षणार्थभावनाभाव्यको लिङादिज्ञानकरणकः स्वज्ञानकरणको वा स्तुतिनिन्दार्थवादबोधितप्राशस्त्यादीतिकर्तव्यताको लिङादिशब्दस्य व्यापारविशेषः शब्दभावना लिङादिशब्देन लिङ्त्वांशेनोच्यते~। तत्र
चार्थवादबोधितप्राशस्त्यादिनोपकारं संपाद्य लिङादिज्ञानेन पुरुषप्रवृत्तिलक्षणामर्थभावनां भावयेद्यागविषयप्रवृत्तिलक्षणां कुर्यादिति फलितम्~।
\newpage
%%%%%%%%%%%%%%%%%%%%%%%%%%%%%%%%%%
\fancyhead[LO]{[आर्थीभावनालक्षणम् ]}
\begin{center}
\textbf{आर्थीभावनालक्षणम्}    
\end{center}

{\bl {\al प्रयोजनेच्छाजनितक्रियाविषयव्यापार आर्थी भावना~}।~सा}\\
\hrule
\vspace{3mm}

 अर्थभावनां लक्षयति- {\br प्रयोजनेच्छेत्यादिना~।} प्रयोजनस्य स्वर्गादिरूपफलस्य येच्छा रागविशेषः {\qt फलेच्छासाधनमुपसंक्रामति} इति न्यायात् तेन च रागविशेषेण जनितो यो यागादिक्रियाविषयः पुरुषस्य व्यापारविशेषः साऽऽर्थी भावनेत्यर्थः~। {\br ननु} कोऽयं व्यापारो नाम योऽर्थभावनात्वेनोच्यते ?~।~न तावत्प्रयत्नमात्रं, रथो
गच्छतीत्यत्राव्याप्त्यापत्तेः~। स्पन्द इति चेत् {\qt देवदत्तो ग्रामं गच्छति, स्वर्गकामो यजेत, यागेन स्वर्गं कुर्यात्} इत्यादौ गमनाद्यनुकूलकृतावव्याप्तेः, कृतेश्चेतनधर्मत्वेन तत्र स्पन्दत्वाभावात् , गच्छतीत्यादौ गमनं करोतीति करोतिप्रयोगदर्शनात् स्पन्दत इति स्पन्दिप्रयोगादर्शनाच्चेति चेत्- केचिदत्राहुः\textendash स्वर्गादिफलेच्छाजनितो
यागादिक्रियाविषयः प्रयत्न एव सव्यापारोऽर्थभावनात्वेनोच्यते~। न च रथो गच्छतीत्यत्राव्याप्त्यापत्तिः; रथवोढृृणामश्वानां प्रयत्नं रथे समारोप्य रथो गच्छतीति प्रयोगोपपत्तेः~। येषां मतेऽन्योत्पादनानुकूलव्यापारसामान्यमेवार्थभावना तेषामपि रथे गमनव्यतिरिक्तस्य व्यापारविशेषस्यानुपलब्धेः रथो गच्छतीति प्रयोगस्यौपचारिकत्वमन्तरेणानिर्वाहात्तस्य तत्त्वं स्वीकर्तव्यम्~। स च प्रयत्न आख्यातसामान्येन कथ्यते, {\qt यजेत पचति गच्छति} इत्याद्याख्यातश्रवणे प्रयत्नप्रतीतेः~। यागेन कुर्यात् , पाकं करोति, गमनं करोतीति
प्रयत्नार्थककरोतिना आख्यातस्य विवरणदर्शनात्प्रयत्न एवाख्यातसामान्यस्यार्थो न स्पन्दादिः ; प्रयत्नपूर्वकगमनादिकर्तरि गमनं करोति पाकं करोतीति प्रयत्नार्थककरोतिप्रयोगदर्शनात्~। वायुवेगादिना स्पन्दमाने त्वयं वायुवेगादिना स्पन्दते न किंचित्करोतीति तत्प्रतिषेधदर्शनाच्चेति~।\\

 अन्ये त्वाचार्या आहुः\textendash अन्योत्पादनानुकूलात्मा स व्यापारो योऽर्थभावनात्वेनोच्यते, यस्मिन्व्यापारे कृते करणस्य फलोत्पादनसामर्थ्यं भवति तादृशो व्यापार इति यावत्~। स एव च व्यापार आख्यातसामान्यस्यार्थः~। {\qt यजेत स्वर्गकामः} इत्याख्यातश्रवणे हि यागेन तथा व्याप्रियेत यस्मिन्व्यापारे कृते यागः स्वर्गजननसमर्थो भवतीति भवति बोधः~। कुठारेण छिन्द्यादिति कुठारेण तथा व्याप्रियेत यस्मिन्व्यापारे कृते कुठारश्छेदनसमर्थो भवतीतिवत्~। स च
\newpage
%%%%%%%%%%%%%%%%%%%%%%%%%%%%%
\fancyhead[RE]{[आर्थीभावनालक्षणम् ]} 
\noindent
व्यापारोऽन्योत्पादनानुकूलत्वेन सामान्येन रूपेणाख्यातादेवावगम्यते; पश्चात्तु कथंभावाकाङ्क्षायां विशेषरूपेण क्वचिदुद्यमननिपतनादिरूपेण क्वचित्त्वग्न्यन्वाधानादिब्राह्मणतर्पणान्तप्रवृत्तिरूपेण चावगम्यते~। न च {\qt रथो ग्रामं गच्छति} इत्यत्राव्याप्तिः, अन्योत्पादनानुकूलव्यापारस्य रथे गमनव्यतिरिक्तस्याभावादिति वाच्यम् ; रथस्तथागमनेन व्याप्रियते यस्मिन्व्यापारे कृते गमनेन ग्रामप्राप्तिर्भवतीति प्रतीतेः~। कोऽसौ व्यापार इत्याकाङ्क्षायां पूर्वोत्तरावान्तरदेशविभजनसंयोजनरूप इति
पश्चादवगम्यते~। उद्यम्य निपात्य कुठारेण छिनत्तीतिवत् पूर्वप्रदेशेन विभज्योत्तरप्रदेशेन च संयोगं लब्ध्वा रथो ग्रामं गच्छतीति प्रत्ययात्~। नचात्र गमनमात्रमाख्यातार्थ इति वाच्यम् ; {\qt अनन्यलभ्यो हि शब्दार्थ} इति न्यायात्तस्य धातुलभ्यत्वेनाख्यातार्थत्वायोगात्~। एवं {\qt चैत्रः प्रयतते} इत्यत्रापि चैत्रस्तथा व्याप्रियते यथा प्रयत्नो भवतीति
प्रयत्नानुकूलव्यापार एवाख्यातस्यार्थो नतु प्रयत्न एव, प्रयत्नस्य धातुमात्रलभ्यत्वात्~। कोऽसौ व्यापार इत्याकाङ्क्षायां पश्चाज्ज्ञानेच्छादिकमवगम्यते~। न च रथो गच्छतीत्यादिप्रयोगस्यौपचारिकत्वं स्यादिति वाच्यम् ; मुख्ये संभवत्यौपचारिकत्वस्यान्याय्यत्वात्~। न च गच्छति- {\qt गमनं करोति}, पचति- {पाकं करोतीति} प्रयत्नार्थककरोतिनाख्यातस्य समानार्थकत्वदर्शनात्प्रयत्न एवाख्यातस्यार्थ इति वाच्यम् ; {\qt चैत्रः प्रयतते} इत्यादौ व्यभिचारेण प्रयत्नस्याख्यातार्थत्वासंभवात् , औपचारिकत्वस्य निरस्तत्वाच्च~। करोत्यर्थोऽपि व्यापारविशेष एव, नतु प्रयत्नः, चैत्रो गच्छति, रथो गच्छतीति चेतनाचेतनकर्तृकाख्यातसमानार्थकस्य करोतेः प्रयत्नार्थकत्वासंभवात् , तस्मादन्योत्पादनानुकूलव्यापार एवाख्यातसामान्यस्यार्थ इति सिद्धमिति~। \\

 {\br ननु} प्रयोजनेच्छाजनितक्रियाविषयव्यापार इति मूलग्रन्थे स्वर्गादिप्रयोजनेच्छाजनितस्य यागादिविषयकव्यापारस्य प्रयत्नत्वमेव कल्प्यते, चेतनव्यापारत्वात्~। ततश्च प्रयत्न एवाख्यातार्थ इति चेत् ! सत्यम्; तथापि न सर्वत्र प्रयत्नार्थकत्वमाख्यातस्य संभवति, रथो गच्छतीत्यादौ व्यभिचारस्य दर्शितत्वात्~। तत्र यागादिविषयकपुरुषप्रयत्नादेरपि
{\qt कृति} शब्दाभिधेयस्यान्योत्पादनानुकूलव्यापारत्वेनैव सामान्याकारेणाख्यातार्थत्वं, नतु विशेषरूपेण; विशेषरूपेण तु पश्चादवगमोभवति- कोऽसौ व्यापार इति वीक्षायाम्~। तस्मात् सर्वत्रान्योत्पादनानुकूलव्यापारसामान्यमेवाख्यातार्थ इति सिद्धम्~।
\newpage
%%%%%%%%%%%%%%%%%%%%%%%%%%%%%%%
\fancyhead[LO]{[ आर्थीभावनांशत्रयम् ]} 
{\bl\noindent
चाख्यातत्वांशेनोच्यते आख्यातसामान्यस्य व्यापारवाचित्वात्~।~}
\begin{center}
\textbf{आर्थीभावनाया अंशत्रयम्}    
\end{center}

{\bl साप्यंशत्रयमपेक्षते साध्यं साधनमितिकर्तव्यतां च {\qtl किं भावयेत् , केन भावयेत् , कथं भावयेत्} इति~।~तत्र साध्याकाङ्क्षायां}\\
\hrule
\vspace{3mm}

किंच यागादिकरणेन धात्वर्थेन स्वसाधनद्रव्यादिनिष्पादितेन स्वर्गादिफलोत्पत्तौ येयमनुकूलव्यापारस्वरूपा {\qt कृति} शब्दाभिधेया फलोत्पादनाऽऽर्थी भावना, सेयं न यज्यादिधातूनामन्यतमेन केनचिदप्यभिधीयते; सर्वधात्वर्थानुगतरूपत्वात्~। नापि धात्वर्थसामान्यमेव सा, प्रतिधात्वर्थं विलक्षणत्वात्~। तथा हि\textendash {\qt स्वर्गकामो यजेत} इत्यत्र यागेन स्वर्गं भावयेदिति बोधः~। तत्र च यागविषयकव्यापारस्य स्वर्गं  प्रति विलक्षणमानुकूल्यं भवति~। {\qt ओदनकामः पचेत्} इत्यत्र च पाकेनौदनं भावयेदिति बोधः~। पाकव्यापारस्य चोदनं प्रति विलक्षणमेवानुकूल्यम्~।~{\qt नैरोग्यकामो भेषजपानं कुर्यात्} इत्यत्र भेषजपानेन नैरोग्यं कुर्यादिति बोधो भेषजपानव्यापारस्य च नैरोग्यं प्रति विलक्षणमेव चानुकूल्यं भवति~। तथा च विलक्षणानुकूल्यविशिष्टस्य व्यापारविशेषस्य प्रतिधात्वर्थविलक्षणरूपत्वमेव; अन्यथा फलविभागानुपपत्तिः स्यात्~।~ततश्च भावनात्वसामान्यं तु भिन्नासु भावनाव्यक्तिष्वनुवर्ततां नाम, नैतावता प्रकृत्यर्थसामान्यं भावना~। तस्माद्यज्यादिधात्वर्थाद्विशेषरूपात्सामान्यरूपाच्च भिन्नैवाख्यातप्रत्ययसामान्यार्थभूताऽऽर्थी भावना~। ततश्च सा तेनैवोच्यत इत्याशयेनाह\textendash {\br सा चाख्यातत्वांशेनोच्यत इति~।} तत्र हेतुमाह\textendash {\br आख्यातेत्यादिना~।~व्यापारवाचित्वादिति~।} अन्योत्पादनानुकूलव्यापारसामान्यवाचित्वादित्यर्थः~।\\

 सा निरुक्ताऽऽर्थी भावना व्यापारविशेषात्मिका व्यापारविशेषाणां च छिदिभावनारूपव्यापारस्य द्वैधीभावरूपफलादिस्वापेक्षत्ववत्फलादिस्वापेक्षत्वाद्भाव्याद्यंशत्रयमपेक्षत
इत्याशयेनाह \textendash {\br साऽप्यंशत्रयमपेक्षत इति~।} अपेक्षामभिनयति- {\br किं भावयेदित्यादिना~। तत्रेति~।} साध्यादीनां भावनांशानां त्रयाणां मध्य इत्यर्थः~। तस्यां च भावनायां स्वर्गादिफलमेव पुरुषविशेषणमपि साध्यत्वेनान्वेति, पुरुषार्थत्वात् , नतु धात्वर्थः; समानपदोपात्तोऽप्यपुरुषार्थत्वात्तत्त्वेनान्वयं लभत इत्याह\textendash {\br स्वर्गादिफलमिति~।} किंच तस्यां पुनः फलभावनायां प्रत्यय-
\newpage
%%%%%%%%%%%%%%%%%%%%%%%%%%%%%%%%%
\fancyhead[RE]{[ आर्थीभावनाया }
{\bl\noindent स्वर्गादिफलं साध्यत्वेनान्वेति~। साधनाकाङ्क्षायां यागादिः करणत्वेनान्वेति~। इतिकर्तव्यताकाङ्क्षायां प्रयाजाद्यङ्गजातमितिकर्तव्यतात्वेनान्वेति~।}\\
\hrule
\vspace{3mm}
\noindent
वाच्यभूतायामेकपदोपात्तः प्रकृत्यर्थ एव करणत्वेनान्वेति संनिकृष्टत्वात्, नतु पदान्तरोपात्तं द्रव्यादि, विप्रकृष्टत्वात्~। न च साध्यरूपस्य प्रकृत्यर्थस्य कथं फलसाधकत्वमिति वाच्यम्, द्रव्यादिस्वसाधननिष्पादितस्य साध्यस्यापि प्रकृत्यर्थस्य फलं साधयितुं शक्यत्वादित्युक्तमेवेत्याशयेनाह\textendash {\br यागादिः करणत्वेनान्वेतीति~।} किंच {\qt कुठारेण छिनत्ति} इत्यादौ कथमिति कथंभावाकाङ्क्षायाम् {\qt उद्यम्य निपात्य} इत्युद्यमननिपतनादेरितिरकर्तव्यतात्वेनान्वयवत् {\qt यागेन स्वर्गं भावयेत्} इत्यत्रापि कथमिति कथंभावाकाङ्क्षायामग्न्यन्वाधानप्रयाजावघातादिभिरुपकारं संपाद्येति प्रयाजाद्यङ्गजातमितिकर्तव्यतात्वेनान्वयं भजत इत्याशयवानाह\textendash {\br इतिकर्तव्यताकाङ्क्षायामित्यादिना~।} कथंभावाकाङ्क्षापूरकत्वमितिकर्तव्यतात्वम्~। निरुक्तो वेतिकर्तव्यताशब्दार्थो बोधः~। भवति च प्रयाजादिषु लक्षणसमन्वयो {\qt यागेन
स्वर्गं कुर्यात्} इति~। ततः कथमिति कथंभावाकाङ्क्षायां प्रयाजादिभिरुपकारं संपाद्येति कथंभावाकाङ्क्षापूरणात्, कर्तव्यसामान्यस्य यागादिरूपस्य भेदकविशेषरूपत्वाच्च~। तच्च प्रकरणप्रमाणनिरूपणावसरे प्रदर्शयिष्यामः~। तथा च {\qt यजेत स्वर्गकामः} इत्यत्राग्न्यन्वाधानावघातप्रयाजादिभिरुपकारं संपाद्य यागेन स्वर्गं भावयेत् - स्वर्गं कुर्यादिति वाक्यार्थः~।~यथा {\qt ओदनकामः पचेत्} इत्यत्र लिङा भावनाऽभिधीयते, तत्र च किं भावयेत् , कथं भावयेदिति भाव्याद्याकाङ्क्षायां तृणफूत्कारादिभिरुपकारं संपाद्य पाकेन
तेजःसंयोगलक्षणेनौदनं भावयेदोदनं कुर्यादिति भाव्याद्यन्वयेन वाक्यार्थः संपद्यते तद्वदिति~।~\\

 {\br ननु} पूर्वं साध्यस्यापि यागादेः स्वसाधननिष्पादितस्य सतोऽपूर्वनिष्पादकत्वं संभवति, तद्द्वारेण च विनश्वरस्याप्यचिरं स्वर्गसाधनत्वमितरस्य संभवतीत्युक्तम्~। तच्च यागेन कथमुत्पाद्यते ? तत्र वक्तव्यं\textendash {\qt यागेन स्वर्गं कुर्यादिति तावत्फलवाक्येन यागस्य फलसाधनत्वं बोध्यते }~। तत्र च कथं विनश्वरेण स्वर्गः कर्तव्यः,? {\qt तस्य
कालान्तरभावित्वादित्याकाङ्क्षायामपूर्वं निष्पाद्येत्युच्यते~}।~पुनः कथमपूर्वं निष्पादनीयमित्याकाङ्क्षायां प्राच्योदीच्याङ्गविशिष्टस्य यागस्यानुष्ठानप्रकारेणेत्यु- 
\newpage
%%%%%%%%%%%%%%%%%%%%%%%%%%%%%%%%%%%%%
\fancyhead[LO]{अंशत्रयम् ]} 
\noindent
च्यते~। तच्चापूर्वं दर्शपूर्णमासयोरनेकविधं - फलापूर्वं समुदायापूर्वमुत्पत्त्यपूर्वमङ्गापूर्वं चेति । तत्र येन स्वर्गः क्रियते तत् फलापूर्वमित्युच्यते; फलजनकत्वात्, तच्च समुदायापूर्वेण जन्यते~। समुदायश्च द्विविधः {\qt अमावास्यायां त्रयाणां यागानामेकः समुदायः, पौर्णमास्यां च त्रयाणां यागानामपरः समुदायश्च}, ताभ्यां जन्यं यदपूर्वं तत्समुदायापूर्वमित्युच्यते~। समुदाययोश्च भिन्नकालवर्तिनोः संहत्य फलापूर्वजननायोगात्तज्जननाय समुदायद्वयजन्यस्यापूर्वद्वयस्यावश्यं कल्पनीयत्वात्~। अमावास्यायां समुदायस्तु {\qt ऐन्द्रं दध्यमावास्यायामैन्द्रं  पयोऽमावास्यायाम्} इति वाक्यविहितौ सांनाय्ययागौ {\qt यदाग्नेयोऽष्टाकपालः} इति वाक्यविहित आग्नेयश्च तेषां त्रयाणां भवति~। पौर्णमास्यां
समुदायस्तु {\qt यदाग्नेयोऽष्टाकपालोऽमावास्यायां च पौर्णमास्यां चाच्युतो भवति} इत्याग्नेययागो विहितः, {\qt ताभ्यामेतमग्नीषोमीयमेकादशकपालं पूर्णमासे
प्रायच्छत्} इत्यम्नीषोमीययागो विहितः~।~{\qt उपांशुयाजमन्तरा यजति} इत्युपांशुयागः {\qt तावब्रूतामग्नीषोमावाज्यस्यैव तावुपांशु पौर्णमास्याम्} इति वाक्येन विहितः~। तेषां त्रयाणां च भवति, तयोस्तु समुदाययोर्मध्य एकसमुदायवर्तिनां त्रयाणां यागानां भिन्नक्षणवर्तिनां संहत्य समुदायद्वयजन्ययोरपूर्वयोरेकापूर्वजननायोगात्तज्जननाय यागत्रयजन्यानि
त्रीण्युत्पत्त्यपर्वाणि कल्पनीयानि~। तेषां चाङ्गोपकारमन्तरेणानुत्पत्तेरङ्गानां चानेकक्षणवर्तित्वेन संहत्योत्पत्त्यपूर्वारम्भायोगात्तदारम्भायाङ्गापूर्वाणि संनिपत्योपकारकादीनि कल्पनीयानि~।~तत्र चाऽयं विभागः - संनिपत्योपकारकाण्यवघातप्रोक्षणादीनि द्रव्यदेवतासंस्कारद्वारेण यागस्वरूपस्यैवातिशयजननेन यागोत्पत्त्यपूर्वोत्पत्तौ, तद्द्वारेण हि
फलापूर्वे च व्याप्रियन्ते~। संनिपत्योपकारकाङ्गापूर्वे यागोत्पत्त्यपूर्वस्य प्रयोजकत्वमिति केचित्~।~फलापूर्वस्यैव तादृशाङ्गापूर्वेऽपि प्रयोजकत्वं स्वीर्कर्तव्यम् , तस्यैव
सर्वापूर्वप्रयोजकत्वे लाघवादित्यन्ये~। आरादुपकारकाणि तु प्रयाजादीनि यागोत्पत्त्यपूर्वेभ्यः सकाशाज्जायमानं फलापूर्वमेव साक्षाज्जनयन्तीति~।~एवं प्रकारभेदेऽपि
सर्वाण्यङ्गान्यपूर्वोत्पत्तावनुग्राहकाणीत्येकरूपेणैवेत्थंभावेन गृह्यन्ते ~। इत्थंभाव इतिर्कर्तव्यता चानर्थान्तरम्~।~तथा च प्रधानानामाग्नेयादीनां षण्णां स्वरूपेण सर्वाङ्गसाहित्याभावेऽपि स्वस्वोत्पत्त्यपूर्वद्वारेण सर्वाङ्गसाहित्यं तेषामङ्गानां, प्रयाजादीनामपि स्वरूपेण सर्वप्रधानसाहित्यासंभवेऽपि स्वस्वोत्पत्त्यपूर्वप्रधानसाहित्यं चोपपन्नम्
~। तच्च साहित्यं
\newpage
%%%%%%%%%%%%%%%%%%%%%%%%%%%%%%%
\fancyhead[RE]{[ आर्थीभावनाया}
\noindent
विहितमिति वक्ष्यते~।~एवं च यदेव प्रधानोत्पत्त्यपूर्वाणां प्रयाजाद्युत्पत्त्यपूर्वैः साहित्यं तदेव प्रधानानामङ्गवैशिष्ट्यरूपं साङ्गत्वमित्युच्यते~। तस्मात्ताभ्यां समुदायापूर्वाभ्यामाग्नेयादिप्रधानोत्पत्त्यपूर्वत्रितयत्रितयजन्याभ्यां प्रयाजाद्यङ्गापूर्वसहिताभ्यां फलजनकीभूतं फलापूर्वापरनामकं महापूर्वं जन्यते, तेन च फलमिति यागस्यापूर्वद्वारेण फलसाधनत्वमुक्तमुपपन्नतरं भवतीति सर्वं समञ्जसम्~।\\

 पर्यायत्वप्रसक्त्यैकस्यैव पदस्यैकापूर्ववाचकत्वम्~।~{\br ननु} {\qt दर्शपूर्णमासाभ्यां स्वर्गकामो यजेत, चित्रया यजेत पशुकामः, उद्भिदा यजेत पशुकामः, ज्योतिष्टोमेन स्वर्गकामो यजेत} इत्यादिषु सर्ववाक्येषु कस्य पदस्यापूर्वप्रतिपादकत्वमिति वक्तव्यम्~। न च भावनावाचकस्य यजतिददात्याख्यातान्तस्यापूर्ववाचकत्वं भवत्विति वाच्यम् ; अपूर्वस्य साध्यत्वेन प्रधानत्वात्सर्वेषां पदानां प्रधानान्वयलाभाय तेषां सर्वेषामेव क्रियाकारकसंबन्धमनादृत्य प्रत्येकमपूर्ववाचकत्वात् , अन्यथा तेषां प्रधानान्वयित्वं न स्यादिति
चेत्, अत्रोच्यते\textendash अपूर्वस्यात्यन्तादृष्टरूपत्वादेकापूर्वकल्पनयैव वाक्यस्योपपत्तावनेकापूर्वकल्पनायां गौरवप्रसङ्गः स्यात्~। सर्वेषां च पदानां तद्वाचकत्वे पर्यायत्वप्रसक्त्यैकस्यैव पदस्यैकापूर्ववाचकत्वं स्वीकर्तव्यं, पदान्तरं तु तद्गुणतयान्वेति, तच्चापूर्ववाचकं पदमाख्यातान्तमेव नतु कर्मनामधेयादिकं, तस्य भावार्थसामानाधिकरण्यादिनाप्युपपत्तेः~।\\

 {\br ननु} {\qt सोमेन यजेत, हिरण्यमात्रेयाय ददाति तस्मात्सुवर्णं हिरण्यं भार्यम्} इति हि श्रुतम्~। तत्र च {\qt सोम-हिरण्य} शब्दौ द्रव्यवाचकौ~। {\qt सुवर्ण} शब्दस्तु, शोभनवर्णरूपगुणवाचकः~। तैरेव द्रव्यादिशब्दैरपूर्वमवगम्यते~। द्रव्यादीनां सिद्धस्वरूपाणामेव साध्यापूर्वसाधनत्वादिति चेत् ! न; द्रव्यादिसिद्धस्वरूपाणां
यागदानादिरूपभावार्थशेषत्वेनाप्युपपत्तेर्भावार्थस्यैवापूर्वसाधनत्वात्~। क्रियां विना द्रव्यादीनां न फलसाधनत्वं संभवति, पचिक्रियामन्तरेण काष्ठस्थाल्यादीनामोदनसाधनत्वादर्शनादित्युक्तमेव~। तस्माद्भावनावाचकस्यैवाख्यातान्तपदस्यापूर्वगमकत्वं न द्रव्यादिपदस्येति
सिद्धम्~।\\

 {\br नन्वेवं} भवतु भावनावाचकस्याख्यातपदस्यापूर्वगमकत्वं, भवतु च भावनैवाख्यातसामान्यार्थः, तथापि {\qt सोमेन यजेत, हिरण्यमात्रेयाय ददाति, दाक्षिणानि जुहोति} इत्यादिषु वाक्येषु भावनावाचकस्याख्यातस्यैकत्वाद्भावनाया अप्येकत्वं युक्तम्~। न च धातुभेदाद्भावनाया भेद इति वाच्यम् ; धातोर्भावनावाचक-
\newpage
%%%%%%%%%%%%%%%%%%%%%%%%%%%%%%%%%%%
\fancyhead[LO]{अंशत्रयम् ]}
\noindent
त्वाभावात्तस्य तद्भेदाप्रयोजकत्वादिति चेत् , अत्राभिधीयते- अस्तु तावदाख्यातस्यैव भावनावाचकत्वं, तच्चाख्यातं न प्रतिधातुव्यक्त्येकव्यक्तिरूपं भवति~। न हि सर्वासां
धातुव्यक्तीनामुपर्येकाख्यातप्रत्ययव्यक्तिः श्रूयते~। व्याकरणेनापि न धातुसमूहादेकाख्यातव्यक्तिर्विहिता, तस्माद्बहूनामाख्यातव्यक्तीनामेकैकधातुविशेषानुषक्तत्वेनोत्पन्नानां
भावनावाचकत्वाद्यागदानहोमभावनाः परस्परं भिद्यन्त इति भावनाभेदे तत्करणस्यापि भावार्थस्यापर्यायशब्दान्तराद्भेदः स्पष्ट एवेति सिद्धम्~।\\

 {\br ननु} {\qt समिधो यजति, तनूनपातं यजति, इडो यजति, बर्हिर्यजति, स्वाहाकारं यजति} इति दर्शादिप्रकरणे पञ्च प्रयाजाः श्रूयन्ते, तत्र च पञ्चकृत्वः श्रुते यजतिपदे यजति-ददाति-जुहोत्यादिषु पूर्वोक्तपदेष्विव धातुभेदाभावात्तदनुषक्ताख्यातस्याप्यभेद एव, ततश्चाख्यातैक्यप्रयुक्तभावनैक्यमपि दुर्वारमिति चेत्- न; यजतिपदाभ्यासाद्भावनाभेदस्य स्वीकर्तव्यत्वात्~। तस्मात्करणभेदः~। अन्यथा कर्मैकत्वेऽभ्यासो निरर्थकः स्यात्~। तस्मादविशेषपुनःश्रुतिरूपयजतिपदाभ्यासात्कर्मभेदः सिद्ध इत्यन्यत्र विस्तरः~।\\

 {\br ननु} {\qt तिस्र आहुतीर्जुहोति} इत्यत्र {\qt जुहोति}इत्याख्यातं {\qt समिधो यजति} इत्यादिवन्नाभ्यासेनाम्नातं किंतु सकृदेव, ततश्च भावनैक्येन कर्मैक्यमेवेति
चेत् , अत्र वक्तव्यम्\textendash किमिदमाख्यातं पदान्तरान्वयनिरपेक्षस्वरूपं सदेव भावनैक्ये प्रमाणमुत पदान्तरान्वयसापेक्षस्वरूपम् ? नाद्यः; पदमात्रस्य
वाक्यांशरूपस्य स्मारकत्वेन वाक्यकार्यरूपप्रमितिजनकत्वासंभवात्~। न द्वितीयः; त्रित्वसंख्यया विशेषितेनाख्यातेन कर्मबहुत्वावगमाद्भावनाबहुत्वावगमे तस्य भावनैक्ये प्रमाणत्वाभावात्~। तस्मात् पदाभ्यासाभावेऽपि जुहोत्यर्थे होमे त्रित्वसंख्यान्वयात् परस्परं त्रयो होमा भिद्यन्त इति भावानानां त्रित्वमेवेति सिद्धम्~।\\

 {\br ननु} {\qt अथैष ज्योतिरथैष विश्वज्योतिरथैष सर्वज्योतिरेतेन सहस्रदक्षिणेन यजेत} इत्यत्र हि प्रकृतं ज्योतिष्टोमम् {\qt एष ज्योतिः} इत्यादिनाऽनूद्य, तस्मिन्सहस्रदक्षिणादानलक्षणो गुणो विधीयत इति नात्र कर्मभेदेन भावनाभेद इति चेत्- न; प्रकृतस्य ज्योतिष्टोमस्य {\qt अथ} इत्यनेन विच्छेदात्~। ततश्च ज्योतिष्टोमप्रकरणे श्रूयमाणानामपि त्रयाणां यागानां ज्योतिष्टोमसंज्ञापेक्षया पृथक्संज्ञात्रयकरणाज्ज्योतिष्टोमाद्भिन्नसंज्ञावशादेव त्रयाणां च परस्परं भेद इति भावनानां भिन्नत्वमिति सिद्धम्~।
\newpage
%%%%%%%%%%%%%%%%%%%%%%%%%%%%
\fancyhead[RE]{[आर्थीभावनाया अंशत्रयम्]}

{\br ननु} {\qt तप्ते पयसि दध्यानयति सा वैश्वदेव्यामिक्षा वाजिभ्यो वाजिनम्} इति हि श्रूयते~। तत्र च घनीभूतः पयःपिण्ड आमिक्षा~। जलं वाजिनम्~। तथा चामिक्षाद्रव्यभाजो ये विश्वेदेवास्तान्वाजिभ्य इत्यनेनानूद्य तत्र {\qt वाजिन}द्रव्यरूपो गुणो विधीयते; वाजोऽन्नमामिक्षारूपमेषामस्तीति तदुत्पत्तेः~।\footnotemark\ तच्च द्रव्यमामिक्षाद्रव्येण सह समुच्चीयतां विकल्प्यतां वेति चेत्,\textemdash\ न; वैश्वदेवयागस्य पूर्वमेवामिक्षारूपगुणावरुद्धत्वेन तत्र वाजिनगुणस्य प्रवेशायोगात्~। न हि व्रीहियवयोरिव वाजिनामिक्षयोर्विकल्पः; समशिष्टत्वाभावात्~। आमिक्षारूपो गुणस्तु वैश्वदेवयागस्योत्पत्तिवाक्य एव शिष्यते विधीयत इत्युत्पत्तिशिष्टः ~। वाजिनगुणस्य तूत्पन्ने कर्मणि विधिः कल्प्यत इत्युत्पन्नशिष्टः~। उत्पत्तिशिष्टोत्पन्नशिष्टयोर्मध्य उत्पत्तिशिष्टः कर्मोत्पत्तिकाल एव तदङ्गत्वेन प्रमितत्वात्प्रबलः~। उत्पन्नशिष्टस्तु तदनन्तरं प्रमितोऽपि विलम्बितत्वेन , दुर्बलत्वात्तत्र प्रवेशमलभमानो {\qt वाजि} शब्दार्थस्य देवतान्तरत्वमापाद्य तद्दैवत्यकर्मान्तरे प्रविशति, तस्माद्द्रव्यदेवतालक्षणस्य रूपस्य भेदात्कर्मभेदेन भावनाया भेद इति
सिद्धम्~।\\

 {\br ननु} {\qt उपसद्भिश्चरित्वा मासमग्निहोत्रं जुहोति} इत्यत्र हि न कर्मान्तरभावनाया विधिः, किंतु नित्याग्निहोत्रमनूद्य तत्र मासरूपो गुणो विधीयते प्राप्तत्वादिति चेत् ,- न; वाक्यभेदप्रसङ्गात्~। तथा हि\textendash किं तदनुवादेन मास एव विधीयते, किमुतोपसदोऽपि ? न प्रथमः; उपसदामपि नित्याग्निहोत्रेऽप्राप्तानां त्वन्मते विधेयत्वात्~। नापि द्वितीयः; प्राप्ते कर्मणि मासोपसद्रूपानेकगुणविधाने वाक्यभेदस्य दुर्वारत्वात्~। स चाष्टदोषदुष्टः~। दोषांश्चोपरिष्टात् प्रदर्शयिष्यामः~। तस्मादपूर्वकर्मविधायकात्कुण्डपायिनामयनरूपात्प्रकरणान्तरान्नित्याग्निहोत्रधर्मकं तन्नामकं च कर्मान्तरमत्र विधीयत इति कर्मान्तरभावना सिद्धा~। तथा च\textendash शब्दान्तराभ्याससंख्यासंज्ञागुणभेदप्रकरणान्तरैः कर्मभेदोऽपि द्वितीयाध्यायस्यार्थो मूले भावनाप्रदर्शनेनैव सूचितो वेदितव्यः~। अग्रे वोत्पत्तिविधिनिरूपणेन स ध्वनितोऽस्माभिस्त्वत्रैव निरूपितः~। तस्मात् स्वर्गमुद्दिश्य तत्साधनत्वेन स्वर्गकामविधिर्यागादिकं विधत्ते~। ततश्च तादृशवेदस्य धर्मादौ प्रामाण्यं निरवग्रहमुपपन्नमिति सर्वं निरवद्यम्~।
\blfootnote{टिप्प०\textemdash\ $^{१}$अत्र तद्व्युत्पत्तेरिति पाठो युक्त इति भाति~।}
\newpage
%%%%%%%%%%%%%%%%%%%%%%%%%%%%%%%%%%%%%%%%%%%%%%%
\fancyhead[LO]{[वेदलक्षणविचारः]}
\begin{center}
 \textbf{वेदलक्षणविचारः}   
\end{center}
 
{\bl अथ को वेद इति चेत्,  उच्यते\textendash\ {\al अपौरुषेयं वाक्यं वेदः~}।}\\
\hrule
\vspace{3mm}

तस्य च वेदरूपप्रमाणस्य प्रमेयोऽर्थस्त्रिविध इति प्रसङ्गाच्चिन्त्यते {\qt क्रत्वर्थः पुरुषार्थ उभयार्थश्चेति}~।~तत्र प्रयाजादिः प्रोक्षणादिश्च केवलं क्रत्वर्थः~।~फलं यागादिरूपं तत्करणं च पुरुषार्थः, यथा स्वर्गादिदर्शपूर्णमासादिश्च~।~दध्यादि तूभयार्थ {\qt दध्ना जुहोति} इति फलासंयुक्तवाक्येन तस्य क्रत्वर्थत्वावगमात् , {\qt दध्नेन्द्रियकामस्य
जुहुयात्} इति फलसंयोगपरवाक्येन पुरुषार्थत्वावगमाच्च~।~एकस्य तूभयार्थत्वे विनियोजकप्रमाणभेदस्य नियामकत्वात्~।~तथा चोक्तम्\textendash {\qt एकस्य तूभयत्वे संयोगपृथक्त्वम्} इति~।~किंच क्रत्वर्थे प्रयाजादौ क्रतुः प्रयोजकः, पुरुषार्थे च दर्शादौ फलं, प्रयोजकत्वं चानुष्ठापकत्वम्~।~तथा च विधिर्यदर्थं यदनुष्ठापयति स तत्र प्रयोजकः~।~यथा दर्शादिविधिः स्वर्गार्थं दर्शादिकमनुष्ठापयतीति स्वर्गादिदर्शादौ प्रयोजकः~।~यथा च प्रयाजादिविधिः प्रयाजादीन् दर्शाद्यर्थमनुष्ठापयतीति दर्शादिः प्रयाजादिषु प्रयोजको दध्यानयनविधिश्च दध्यानयनमामिक्षार्थमनुष्ठापयत्यामिक्षा दध्यानयने प्रयोजिका~।~वाजिनं तु दध्यानयनानुष्ठानेनामिक्षायां जायमानायामनुनिष्पाद्यमानत्वान्मधुररसस्य चामिक्षायामेव विशेषेणोपलभ्यमानत्वान्न दध्यानयने प्रयोजकमित्यादिका चतुर्थाध्यायार्थचिन्ता स्वयमूहितव्या~।\\

 तदेवं सामान्यतः प्रयोजनवदर्थावबोधकत्वेन वेदस्य धर्मे प्रामाण्यं धर्मविधायकत्वप्रदर्शनेन चोपपाद्य, इदानीं तस्य विध्यादिरूपविभागेन तत्र प्रामाण्यमुपपादयितुं तल्लक्षणं पृच्छति {\br अथेति~।} उत्तरं प्रतिजानीते {\br उच्यत इति~।} तत्र सामान्यलक्षणमाह\textendash\ {\br अपौरुषेयं वाक्यमिति~।} {\qt वेदः} इति लक्ष्यनिर्देशः~।~तत्र भारतादावतिव्याप्तिवारणाय {\qt अपौरुषेयम्} इति विशेषणम्~।~आत्मादौ तद्दोषवारणाय {\qt वाक्यम्} इति विशेष्यम्~। प्रमाणान्तरेणार्थमुपलभ्य विनिर्मितत्वं पौरुषेयत्वं
तद्भिन्नवाक्यत्वमिति फलितम्~।~ईश्वरो वेदमपि प्रमाणान्तरेणार्थमुपलभ्य विरचयति~।~तस्मात्कथं वेदस्य पौरुषेयाद्भिन्नत्वम् ? न हि वेदं परमेश्वरस्तदर्थं प्रमाणान्तरेणोपलभ्य विरचयति~।~वेदाध्ययनं गुर्वध्ययनपूर्वकं,
\newpage
%%%%%%%%%%%%%%%%%%%%%%%%%%%%%%%%
\fancyhead[RE]{[विधिमीमांसा ]}
{\bl\noindent स च
{\al विधि- मन्त्र-नामधेय-निषेधा-र्थवादभेदात् पञ्चविधः~}।}
\begin{center}
\textbf{विधिमीमांसा}    
\end{center}

{\bl तत्राज्ञा\footnotemark तार्थज्ञापको वेदभागो विधिः~। स च तादृशप्रयोजनवदर्थविधानेनार्थवान् यादृशं चार्थं प्रमाणान्तरेणाप्राप्तं विधत्ते- यथा {\qtl अग्निहोत्रं जुहुयात् स्वर्गकामः} इति विधिर्मानान्तरेणाप्राप्तं स्वर्गप्रयोजनवद्धोमं विधत्ते~। अग्निहोत्रहोमेन स्वर्गं भावयेदिति वाक्यार्थबोधः~। यत्र कर्म मानान्तरेण प्राप्तं तत्र तदुद्देशेन गुणमात्रं विधत्ते- यथा {\qtl दध्ना जुहोति} इत्यत्र होमस्याग्निहोत्रं जुहुयादित्यनेन प्राप्तत्वाद्धोमोद्देशेन दधिमात्रविधानं दध्ना होमं भावयेदिति~। यत्र तूभयमप्राप्तं तत्र विशिष्टं विधत्ते- यथा
{\qtl सोमेन यजेत} इत्यत्र सोमयागयोरप्राप्तत्वात्सोमविशिष्टयागा}\\
\hrule
\vspace{3mm}
\noindent
वेदाध्ययनत्वात् , वर्तमानवेदाध्ययनवदित्यनुमानेन वेदस्यापौरुषेयत्वसाधनात्~। {\qt यः कल्पः स कल्पपूर्वकः} इति न्यायेन संसारस्यानादित्वात् {\qt परमेश्वरस्य च सर्वज्ञत्वात्परमेश्वरो गतकल्पीयं वेदमस्मिन्कल्पे स्मृत्वोपदिशतीत्येतावन्मात्रेणोपपत्तौ वेदपौरुषेयत्वस्यानौचित्याच्चेति भावः~}। वेदं विभजते {\br स चेत्यादिना~।}\\

तत्र विधिं लक्षयति- {\br तत्रेत्यादिना~।} तत्र पञ्चविधेषु वेदभागेषु मध्य इत्यर्थः~। {\br अज्ञातेति~।} अज्ञातस्य प्रयोजनवतोऽर्थस्य यागादिनामकस्य ज्ञापकत्वं विधित्वमित्यर्थः । तदेवाह\textendash {\br स चेत्यादिना~।} स विधिर्यादृशं प्रमाणान्तरेणाप्रा\footnotemarkA[1]प्तं भवति तादृशमर्थं विधत्ते, तेन चाज्ञातप्रयोजनवदर्थविधानेनार्थवानिति योजना~। तत्रोदाहरणमाह\textendash\ {\br यथाऽग्निहोत्रमिति~।} गुणविधिस्तु मानान्तरेणाप्राप्तं गुणमात्रं विधत्ते~।~प्रधानकर्मणस्तु मानान्तरेण प्राप्तस्यानुवाद
एवेत्याह\textendash {\br यत्र कर्मेत्यादिना~। उभयमिति~।} शेषशेषिलक्षणमुभयमित्यर्थः~। {\br विशिष्टमिति~।} शेषविशिष्टं शेषिणमित्यर्थः~। तत्रोदाहरणमाह\textendash {\br यथा सोमेन यजेतेति~।} ननु सोमरूपस्यैव यागस्यात्र विधिरस्तु, किं विशिष्टविधानेनेत्या- 

\blfootnote{पाठा०\textemdash\ $^{१}$तत्राज्ञातज्ञापकः~।}
\alfootnote{टिप्प०\textemdash\ $^{1}$अप्राप्तामिति सामान्ये नपुंसकम्}
\newpage
%%%%%%%%%%%%%%%%%%%%%%%%%%%%%%%%%%
\fancyhead[LO]{[वाक्यभेददोषपरिहारः]}
{\bl\noindent विधानम्~।~सोमपदे मत्वर्थलक्षणया सोमवता यागेनेष्टं भावयेदिति वाक्यार्थबोधः~।}
\begin{center}
 \textbf{वाक्यभेददोषपरिहारः}   
\end{center}
 
{\bl न चोभयविधाने वाक्यभेदः, प्रत्येकमुभयस्याविधानात् ,}\\
\hrule
\vspace{3mm}
\noindent
शङ्कय  , सोमस्य द्रव्यत्वेन यागत्वासंभवात् सोमपदे मत्वर्थलक्षणास्वीकारेण विशिष्टस्यैव विधेर्युक्तत्वान्मैवमित्याह- {\br सोमपदे इत्यादिना~।}\\

 {\br ननु} शेषशेषिलक्षणस्योभयपदार्थस्य विधाने वाक्यभेदः स्यादेव~। न च स इष्ट इति वाच्यम् ; यत्र वाक्यभेदस्तत्राष्टदोषप्रसङ्गात्~। तथा हि\textendash {\qt व्रीहिभिर्यजेत यवैर्वा} इत्यत्र~। तत्र च प्रथमप्रयोगे व्रीह्यनुष्ठाने यवशास्त्रप्रामाण्यस्य स्वार्थानुष्ठापकत्वरूपस्य परित्यागः~।~स्वार्थाननुष्ठापकत्वरूपस्याप्रामाण्यस्य च स्वीकारो भवति~। ततो द्वितीयप्रयोगे यवानुष्ठाने तु पूर्वपरित्यक्तस्य यवशास्त्रप्रामाण्यस्य स्वीकारः, स्वीकृतस्य च तदप्रामाण्यस्य परित्यागश्चेति यवशास्त्रे चत्वारो दोषा भवन्ति~। तथा प्रथमप्रयोगे यवानुष्ठाने व्रीहिशास्त्रप्रामाण्यस्य स्वार्थानुष्ठापकत्वलक्षणस्य परित्यागः, स्वार्थाननुष्ठापकत्वखरूपस्य चाप्रामाण्यस्य स्वीकारः; ततो द्वितीयप्रयोगे व्रीह्यनुष्ठाने तु व्रीहिशास्त्रप्रामाण्यस्य पूर्वं परित्यक्तस्य स्वीकारः, स्वीकृतस्य च तदप्रामाण्यस्य परित्यागश्चेति व्रीहिशास्त्रे चत्वारो दोषा भवन्तीत्यष्टदोषदुष्टो विकल्पो यथा व्रीहियववाक्ये प्रसिद्धस्तथाऽत्रापि स्यात्~।\\

 न च तद्वदत्रापीष्ट एवेति वाच्यम्~।~तत्र व्रीहियवयोः पुरोडाशरूपैककार्यकारित्वेन विकल्पस्येष्टत्वाद्गत्यन्तराभावाच्च~।~प्रकृते तु गुणविधिमात्रस्वीकारेणाप्युपपत्तौ नोभयविधिः साधुरित्याशङ्क्य परिहरति{\br न चेत्यादिना~}। यद्वा,- ननु {\qt यागेनेष्टं भावयेत् , सोमेन च यागं भावयेत्} इत्यावृत्त्योभयविधावावृत्तिलक्षणो वाक्यभेदः स्यात्~। यदा
तु युगपत्तन्त्रेण यागेनेष्टं भावयेत्सोमेन यागं भावयेदित्युभयविधानं तदा तु विरुद्धत्रिकद्वयप्रसङ्ग इत्याशङ्क्य परिहरति {\br न चेत्यादिना~।} एकवाक्यस्य प्रत्येकमुभयपदार्थे व्यापारभेदेनोभयविधायकत्वे वाक्यभेदो भवति, तथाऽत्रास्वीकारादित्यर्थः~।~{\br ननु} विधेयस्योभयत्वे तद्विधायकवाक्यस्यापि व्यापार-
\newpage
%%%%%%%%%%%%%%%%%%%%%%%%%%%%%%%%%%%%%
\fancyhead[RE]{[ उभय\textemdash\ }
{\bl\noindent
किंतु विशिष्ट\footnotemark स्यैव विधानात्~।}
\begin{center}
\textbf{गुणविध्यादिभेदः} 
\end{center}
 
 \blfootnote{पाठा०\textemdash\ $^{१}$विशिष्टस्यैकस्यैव~।}
{\bl न च {\al ज्योतिष्टोमेन स्वर्गकामो यजेत} इति विधि\footnote{विधिना प्राप्त~।}प्राप्तयागोद्देशेन सोमरूपगुणविधानमेवास्तु, सोमेन यागं भावयेदिति किं मत्वर्थलक्षणयेति वाच्यम्~। तस्याधिकारविधित्वेनोत्पत्तिविधित्वासंभवात्~।}
\begin{center}
 \textbf{उभयविधित्वम्~।}
\end{center}

{\bl ननु {\qtl उद्भिदा यजेत पशुकामः} इत्यस्येव ज्योतिष्टोमेनेत्य-\\ }
\hrule
\vspace{3mm}
\noindent
भेदेनैव तद्विधायकत्वं संभवति, नान्यथेति शङ्कते- {\br किंत्विति~।} विशिष्टस्यैकरूपत्वेन विधेयस्योभयत्वासिद्धेर्न व्यापारभेदेन तद्विधानं ततश्च न वाक्यभेदप्रसङ्ग इति परिहरति- {\br विशिष्टस्येति~।} विशिष्टविधौ विशेषणविधेरार्थिकत्वेन श्रूयमाणविधिना गुणस्य पृथगविधेयत्वादिति भावः~।\\

 {\br ननु} {\qt सोमेन यजेत} इत्यस्य गुणविधित्वमेवास्तु, सोमयागस्वरूपविधिस्तु {\qt ज्योतिष्टोमेन स्वर्गकामो यजेत} इत्ययमेव भवतु~। ततश्च ज्योतिष्टोमेनेत्यादिना प्राप्ते कर्मणि गुणविधायकत्वेनापि सोमवाक्यस्योपपत्तौ न मत्वर्थलक्षणा पददोषरूपत्वेनान्याय्या स्वीकर्तव्येत्याशङ्क {\br न च ज्योतिष्टोमेनेत्यादिना~।} न च वाच्यमित्यत्र हेतुमाह \textendash\ {\br तस्याधिकारविधित्वेनेति~।} तर्हि भवतु तस्याधिकारविधित्वम् , उत्पत्तिविधित्वमपि कुतो न स्यादित्याशङ्क्याह \textendash {\br उत्पत्तिविधित्वासंभवादिति~।} कर्मस्वरूपमात्रबोधकस्यैव विधेरुत्पत्तिविधित्वं व्यवह्रियते, उत्पत्तिविधिविहितस्य च कर्मणः फलविशेषेण सह संबन्धमात्रमधिकारविधिः करोति~। यथा {\qt आग्नेयोऽष्टाकपालो भवति} इत्येतदुत्पत्तिविधिविहितस्य कर्मणः स्वर्गरूपफलविशेषेण संबन्धो {\qt दर्शपूर्णमासाभ्यां स्वर्गकामो यजेत} इत्यनेन
विधिना क्रियत इति तस्याधिकारविधित्वमेव नोत्पत्तिविधित्वम् , तद्वदस्याप्यधिकारविधेर्नोत्पत्तिविधित्वं संभवतीति भावः~।\\

 {\br ननु} {\qt उद्भिदा यजेत पशुकामः} इत्यस्य विधिवाक्यस्य यथोद्भिन्नामक-
\newpage
%%%%%%%%%%%%%%%%%%%%%%%%%%%%%%%%
\fancyhead[LO]{विधित्वम् ]}
{\bl\noindent स्याप्युत्पत्यधिकारविधित्वमस्त्विति चेत्,- न; दृष्टान्ते उत्पत्तिवाक्यान्तराभावेनान्यथानुपपत्त्या तथात्वाश्रयणात्~। किंच ज्योतिष्टोमेनेत्यस्योभयविधित्वेऽनेनैव\footnotemark\ यागस्तस्य फलसंबन्धोऽपि बोधनीय इति सुदृढो वाक्यभेदः~।~तद्वरं सोमपदे मत्वर्थ-\\} 
\hrule
\vspace{3mm}
\noindent
कर्मरूपबोधकत्वरूपोत्पत्तिविधित्वेऽपि तस्य कर्मणः पशुरूपफलसंबन्धबोधकत्वरूपाधिकारविधित्वमपि भवति तद्वदस्याप्युभयविधित्वमविरुद्धमित्याशङ्कते {\br नन्वित्यादिना~।} दृष्टान्ते तु कर्मस्वरूपबोधकोत्पत्तिवाक्यान्तराभावेनैकस्य वाक्यस्योभयविधित्वमाश्रितं कर्मणः स्वरूपबोधनमन्तरेण तस्य फलसंबन्धबोधनानुपपत्त्या तस्यैव वाक्यस्योभयविधित्वाश्रयणादिति परिहरति {\br दृष्टान्त इत्यादिना~।}  ननु {\qt ज्योतिष्टोमेन स्वर्गकामो यजेत} इत्यस्य वाक्यस्योभयविधित्वस्वीकारे {\qt सोमेन यजेत} इत्यस्य मत्वर्थलक्षणामन्तरेणैव गुणमात्रविधायकत्वनिर्वाहे को दोषोऽन्यथा मत्वर्थलक्षणाप्रसङ्गादित्याशङ्क्य परिहरति {\br किंचेत्यादिना~। उभयविधित्व इति~।} सोमयागस्योत्पत्तिविधित्वेऽधिकारविधित्वे चेत्यर्थः~। {\qt ज्योतिष्टोमेन स्वर्गकामो यजेत} इत्यनेनैव सोमयागस्वरूपं तस्य स्वर्गरूपफलसंबन्धश्च बोधनीय इत्यर्थः~।~{\br सुदृढो वाक्यभेद इति~।} गौरवलक्षणो वाक्यभेदो दृढतरो भवेदित्यर्थः~।~तथा चोक्तम् \textendash  {\qt श्रौतव्यापारनानात्वे शब्दानामतिगौरवम्~।~एकोक्त्यवसितानां तु
नार्थाक्षेपो विरुध्यते} इति~।~{\br ननु} {\qt सोमेन यजेत} इत्यत्रापि गुणस्य कर्मस्वरूपस्य च विधाने वाक्यभेदस्य सुदृढत्वादिति चेत् , न; विशिष्टविधौ विशेषणविधेरार्थिकत्वेन श्रूयमाणविधिना गुणस्य पृथगविधेयत्वादिति निरस्तत्वात्~। न च तस्योत्पत्तिविधित्वे वाक्यभेदस्याभावेऽपि लक्षणादोषस्तु दुर्वार इति वाच्यम्~। तस्या वाक्यभेदादल्पदोषत्वात् ,
लक्षणायाः पददोषत्वाद्वाक्यभेदस्य तु वाक्यदोषत्वात् ; पदवाक्यदोषयोर्मध्ये पददोषस्यैव कल्पनीयत्वाद् {\qt गुणे त्वन्याय्यकल्पना} इति न्यायात्~। तस्मात् {\qt ज्योतिष्टोमेन स्वर्गकामो यजेत} इत्यस्योभयविधित्वे वाक्यभेदप्रसङ्गाद्वरं सोमपदे मत्वर्थलक्षणां स्वीकृत्य गुणविशिष्टकर्मस्वरूपविधानमेवेत्याशयेनोपसंहरति {\br तद्वरमित्यादिना~।}
\blfootnote{पाठ॰\textemdash\ $^{१}$तेनैव~।}
\newpage
%%%%%%%%%%%%%%%%%%%%%%%%%%%%%%%%%%%
\fancyhead[RE]{[ उत्पत्ति\textemdash\ }
{\bl\noindent
लक्षणया विशिष्टविधानम्~।}
\begin{center}
 \textbf{विधिश्चतुर्विधः }   
\end{center}
 
{\bl {\al विधिश्चतुर्विधः - उत्पत्तिविधि-विनियोगविधि-रधिकारविधिः-प्रयोगविधिश्चेति}
~।}
\begin{center}
 \textbf{उत्पत्तिविधिः}    
\end{center}

{\bl {\al तत्र कर्मस्वरूपमात्रबोधको विधिरुत्पत्तिविधिः~}। यथा {\qtl अग्निहोत्रं जुहोति} इति~। अत्र च विधौ कर्मणः करणत्वेनान्वयः, अग्निहोत्रहोमेनेष्टं भावयेदिति~।}\\
\hrule
\vspace{3mm}

विधिं विभजते {\br विधिश्चतर्विध इति~।} तत्रोत्पत्तिविधिं लक्षयति {\br तत्रेत्यादिना~।} तत्र चतुर्णां विधीनां मध्य इत्यर्थः~। कर्मस्वरूपमात्रेत्यत्र मात्रपदेनोत्पत्तिविधेः कर्मणः फलादिना सह संबन्धबोधकत्वं वारयति {\qt उद्भिदा यजेत पशुकामः} इति विधेस्तु श्रौतमधिकारविधित्वमेव, उत्पत्तिविधित्वं तु कर्मस्वरूपबोधकविध्यन्तराभावेनार्थिकमेवेति न दोषः~। तत्रोदाहरणमाह\textendash\ {\br यथाग्निहोत्रं जुहोतीति~।} एतदुपलक्षणं {\qt सोमेन यजेत, तप्ते पयसि दध्यानयति सा वैश्वदेव्यामिक्षा वाजिभ्यो वाजिनं,
यदाग्नेयोऽष्टाकपालोऽमावास्यायां च पौर्णमास्यां चाच्युतो भवति} इत्यादीनाम्~। तथा च तत्र द्वितीयाध्याये कर्मोत्पत्तिविधीनामेव भेदो निरूपितः~ अत्राप्युत्पत्तिविधिलक्षणप्रदर्शनेनैव तेषां भेदोऽपि ध्वनित एव शब्दान्तरादिभिर्हेतुभिः, अस्माभिस्तु पूर्वमेव तेषां भेदः संक्षेपेण निरूपित इत्युपरम्यते~।~{\br ननु} चात्रोत्पत्तिविधौ कर्मणः साध्यत्वेनान्वयो भवतु {\qt अग्निहोत्रं होमं कुर्यात्} इति, {\qt व्रीहीन्प्रोक्षति} इतिवत् {\qt अग्निहोत्रं जुहोति} इत्यत्रापि द्वितीयायाः साध्यत्ववाचकत्वात्~। तथा च
साध्यस्य साध्यत्वस्वभावादेव साध्यन्तरसाधनत्वासंभवेन साध्यान्तरान्वयायोगादधिकारविध्यवगतफलसंबन्धानुपपत्तिः स्यादित्याशङ्क्याह\textendash {\br अत्र च विधावित्यादिना~।} अग्निहोत्रहोमेनेष्टं भावयेदिति करणत्वेनान्वये तु किं तदिष्टमिति वीक्षायाम् {\qt अग्निहोत्रं जुहुयात्स्वर्गकामः} इत्यधिकारविध्यवगतफलसंबन्धोपपत्तेः,
स्वसाधननिष्पादितस्य सिद्धस्वभावस्यैव करणत्वेनान्वयाच्च न कोऽपि दोष इति भावः~। ननूत्पत्तिविधाविष्ट-
\newpage
%%%%%%%%%%%%%%%%%%%%%%%%%%%%%%
\fancyhead[LO]{विधिः ]}
\hrule
\vspace{3mm}\noindent
बोधकपदस्याभावादग्निहोत्रहोमेनेष्टं भावयेदिति कथं वाक्यार्थः स्यादिति चेत् न~। विधेरेवेष्टाक्षेपकत्वात्~। अन्यथा तस्यापुरुषार्थभूते कर्मणि पुरुषप्रवर्तकत्वानुपपत्तिः स्यात्~।~\\

 {\br ननु} गुरुमते विधिः स्वसिद्ध्यर्थमेव नित्यादिषु कर्मसु पुरुषं प्रवर्तयति~। तथा च काम्यकर्मोत्पत्तिविधीनामिष्टान्तराक्षेपकत्वेऽपि न नित्यादिकर्मोत्पत्तिविधीनामिष्टान्तराक्षेपकत्वं संभवति~। न च लिङादिशब्दव्यापारस्य विधेर्नित्यत्वेन कथं तस्य स्वसिद्ध्याक्षेपकत्वमिति वाच्यम् ; गुरुमतापरिज्ञानात्~। तथा
हि\textendash विधिशब्दस्तावल्लिङादिशब्दवचनस्तदर्थवचनश्च भवति~। तत्र भट्टमते तदर्थस्तु शब्दव्यापारविशेषो भावनैव~। गुरुमते तु नियोगाख्यमपूर्वमेव
लिङादिशब्दार्थभूतो विधिः~। तस्य च साध्यस्वभावत्वेन स्वसिद्ध्याक्षेपकत्वमुपपद्यते~। तथा च लोके लिङः कार्यव्युत्पत्त्यनुरोधेनाग्निहोत्रं जुहुयादित्यादावपि लिङा नियोग एव
प्रतीयते~। नियोगश्चाधिकारिविषयादिसापेक्ष एव~। तत्र कस्य नियोग इत्यधिकार्याकाङ्क्षायां जीवनादिमत इति जीवनादिविशिष्टोऽधिकारित्वेन संबध्यते~। कुत्र नियोग इति विषयाकाङ्क्षायां तु होमादाविति होमादिविषयत्वेन संबध्यते~। होमादिविषयश्च नियोगः कृतिसाध्यतयैव प्रतीयते~। तस्य च साक्षात्कृतिसाध्यत्वासंभवेन स्वस्य कृतिसाध्यत्वनिर्वाहार्थं विषयतया संबद्धस्य होमादेः करणत्वमप्याक्षिपति~। तदुक्तं शालिकायां\textendash  {\qt कृतितत्साध्यमध्यस्थो यागादिविषयो मतः~।~कार्येऽसङ्घटिताकारे करणत्वेन संमतः} इति~। तस्मात् काम्यकर्मोत्पत्तिविधेः {\qt विश्वजिता यजेत} इत्यादिविधेश्च स्वप्रवर्तकत्वान्यथानुपपत्त्या सामान्यतः
प्रयोजनविशेषाक्षेपकत्वेऽपि नित्यकर्मोत्पत्तिविधेः स्वसिद्धेरेव प्रयोजनत्वान्न प्रयोजनान्तराक्षेपकत्वमिति चेत्, न~। नियोगाख्यविधेः स्वरूपेणाप्रयोजनत्वेन स्वसिद्धये पुरुषप्रवर्तकत्वानुपपत्तेः~। अन्यथा {\qt विश्वजिता यजेत} इत्यादावपि विधेः स्वसिद्ध्यर्थमेव पुरुषप्रवर्तकत्वापत्तेः, तच्चानिष्टम् , {\qt स वर्गः स्यात्सर्वान्प्रत्यविशेषात} इति न्यायविरोधात्~। तस्मान्नित्यादिष्वपि कर्मसु स्वर्गः, धर्मेण पापमपनुदतीत्यादिशास्त्रानुरोधेन पापक्षयादिकं वा प्रयोजनं स्वीकर्तव्यमित्यलमतिप्रसङ्गेन~। प्रकृतमनुसरामः~। तस्मात् साधूक्तमुत्पत्तिविधौ कर्मणः करणत्वेनान्वय इति~।
\lfoot{३ अ.}
\newpage
%%%%%%%%%%%%%%%%%%%%%%%%%%%%%%%%%
\lfoot{}
\fancyhead[RE]{[ यागस्य रूपद्वयम् ]}
\begin{center}
\textbf{यागस्य रूपद्वयम्}   
\end{center}

{\bl ननु यागस्य द्वे रूपे द्रव्यं देवता च~। तथा च रूपाश्रवणेऽग्निहोत्रं जुहोतीति कथमुत्पत्तिविधिः? अग्निहोत्रशब्दस्य तु तत्प्रख्यन्यायेन नामधेयत्वादिति चेत्~। न~। रूपाश्रवणेऽप्यस्योत्पत्तिविधित्वात्~। अन्यथा रूपश्रवणात् {\al दध्ना जुहोति} इत्ययमेवोत्पत्तिविधिः स्यात्~। तथा च {\qtl अग्निहोत्रं जुहोति} इति वाक्यमनर्थकं स्यात्~।}\\
\hrule
\vspace{3mm}

{\br ननु} {\qt अग्निहोत्रं जुहोति} इत्यस्य नोत्पत्तिविधित्वं द्रव्यदेवतात्मकस्य कर्मरूपस्यात्राश्रवणादित्याशङ्कते {\br नन्विति~।} {\br ननु} {\qt अग्नये होत्रमत्र}
इत्यग्निहोत्रशब्देनाग्निदेवतात्मकस्य कर्मरूपस्य श्रवणात्कथं रूपाश्रवणमित्याशङ्क्याह\textendash {\br अग्निहोत्रशब्दस्येति~।} अग्निहोत्रशब्दस्य {\qt तत्प्रख्यं चान्यशास्त्रम्} इति तत्प्रख्यन्यायेन नामधेयत्वस्य वक्ष्यमाणत्वान्न कस्यापि कर्मरूपस्य श्रवणमत्रेत्यर्थः~। यद्यप्यत्र कर्मणो रूपं न श्रूयते, तथापि विध्यन्यथानुपपत्त्या तत्कल्प्यते, तच्च सामान्यतः कल्प्यमानं द्रव्यदेवतात्मकं कर्मणो रूपं विशेषाकाङ्क्षया गुणविधिमन्त्रवर्णाभ्यां विशेषेण चावगम्यमानमत्र संभवति~। ततश्च {\qt अग्निहोत्रं जुहोति} इत्यत्र द्रव्यदेवतात्मकस्य कर्मरूपस्य श्रवणाभावेऽप्यस्य होमरूपकर्मस्वरूपमात्रबोधकत्वरूपमुत्पत्तिविधित्वं संभवतीत्याशयेन परिहरति {\br नेति~।} तत्र निरुक्ताशयं हेतुमाह\textendash {\br रूपाश्रवण इति~।} विपक्षे बाधकमाह\textendash\ {\br अन्यथेति~।} {\qt अग्निहोत्रं जुहोति} इत्यस्य रूपाश्रवणमात्रेणोत्पत्तिविधित्वानङ्गीकारे {\qt दध्ना जुहोति} इत्यस्यैवोत्पत्तिविधित्वं स्यात्, अत्र कर्मस्वरूपस्य श्रवणादित्यर्थः~। {\qt दध्ना जुहोति} इत्यस्याग्निहोत्रकर्मोत्पत्तिविधित्वे वाक्यान्तरस्यानर्थकत्वमनिष्टमापादयति {\br तथा चेति~।} न चाग्निरूपगुणविधित्वेनाप्यस्योपपत्तिरिति वाच्यम् ; {\qt अग्निर्ज्योतिः}
इत्यादिमन्त्रवर्णेनाग्निरूपगुणस्य प्राप्तत्वात् , कर्मनामधेयत्वस्य वक्ष्यमाणत्वाच्च; तस्मादनर्थकमिति साधूक्तम्~।~किंच {\qt दध्ना जुहोति} इत्यस्योत्पत्तिविधित्वे {\qt पयसा जुहोति} इत्यस्यापि वैयर्थ्यं कर्मान्तरविधायकत्वं वा स्यात् , होमस्योत्पत्तिशिष्टदध्यवरुद्धत्वेन तत्र पयोरूपगुणविधित्वासंभवात्~।~तथा चानेकादृष्टकल्पनापत्तिः~।~{\qt अग्निहोत्रं
जुहोति} इत्यस्य होमो-
\newpage
%%%%%%%%%%%%%%%%%%%%%%%%%%%%%%%%%%%%%%%%
\fancyhead[LO]{[ विनियोगविधिः]}
\begin{center}
  \textbf{विनियोगविधिः} 
\end{center}

{\bl अङ्गप्रधानसंबन्धबोधको विधिर्विनियोगविधिः~। यथा {\qtl दध्ना जुहोति} इति~। स हि तृतीयया प्रतिपन्नाङ्गभावस्य दध्नो होमसंबन्धं विधत्ते दध्ना होमं भावयेदिति~।}\\
\hrule
\vspace{3mm}
\noindent
त्पत्तिविधित्वे तु {\qt दध्ना जुहोति, पयसा जुहोति} इत्यादिवाक्यस्य सर्वस्यापि तत्र खले कपोतन्यायेन युगपद्दध्यादिगुणविधायकत्वेनाप्युपपत्त्या नानेकादृष्टकल्पनाप्रसङ्ग इत्यग्निहोत्रं जुहोतीत्यस्यैवोत्पत्तिविधित्वं न्याय्यमित्यलमतिविस्तरेण~।\\

 तृतीयाध्यायस्यार्थभूतं शेषशेषिभावं निरूपयितुमिदानीं विनियोगविधिं लक्षयति- {\br अङ्गप्रधानेति~।} अङ्गानां द्रव्यदेवतादिलक्षणानां प्रधानैर्वाक्यान्तरविहितैः सह संबन्धस्य शेषत्वलक्षणस्य बोधको विधिरित्यर्थः~। तत्रोदाहरणमाह\textendash {\br यथा दध्नेति~।}  {\qt दध्ना जुहोति} इति, {\qt पयसा जुहोति} इत्यादेरुपलक्षणार्थम्~। स हि विधिस्तृतीयया प्रतिपन्नाङ्गभावस्य दध्यादेरग्निहोत्रं जुहोतीति विहितहोमसंबन्धं विधत्त इत्याह\textendash {\br स हीत्यादिना~।}  {\br ननु} {\qt दध्ना होमं भावयेत् , होमेनेष्टं भावयेत्} इति साध्यत्वेन करणत्वेन च होमस्यान्वयः स्यात्~। तथा च विरुद्धत्रिकद्वयप्रसङ्गः~। तथा हि\textendash {\qt दध्ना जुहोति} इत्यत्र सकृदुच्चरितस्य
जुहोतीत्याख्यातस्य दधिरूपगुणे किंचिदिष्टे च तन्त्रेण संबन्धाङ्गीकारे सत्युपादेयत्वं, विधेयत्वं, गुणत्वं, चेत्येकं त्रिकम् ; उद्देश्यत्वं, अनुवाद्यत्वं, प्राधान्यं चेत्यपरं त्रिकं होमे संपद्यते कथम् ? शृणु {\qt फलमुद्दिश्य होम उपादीयते, फलमनूद्य होमो विधीयते, फलं प्रधानं होम उपसर्जनम्~}।~एवं होममुद्दिश्य दध्युपादीयते, होममनूद्य दधि विधीयते, होमः
प्रधानं दध्युपसर्जनम्~। ततश्च होमे फलापेक्षयोपादेयत्वं विधेयत्वं गुणत्वं दधिरूपगुणापेक्षया चोद्देश्यत्वमनुवाद्यत्वं प्राधान्यं च संपद्यते~। न चात्र न तन्त्रेण संबन्धः किंतु पृथग्घोमावृत्त्या संबन्धो भवतीति वाच्यम् ; वाक्यभेदप्रसङ्गात् , {\qt दध्ना होमं भावयेत् , होमेन चेष्टं भावयेत्} इति वाक्यद्वयप्राप्तेः~।~तस्मान्न {\qt दधि}शब्दो गुणपर इति चेत्, न~। भ्रान्तिमत्त्वात् , तथा हि\textendash यत्र हि साध्यत्वेन करणत्वेन चैकस्य युगपत्तन्त्रेण संबन्ध आशङ्क्यते तत्रैव विरुद्धत्रिकद्वयस्य प्रसक्तिर्भवति\textendash यथा {\qt वाजपेयेन}
\newpage
%%%%%%%%%%%%%%%%%%%%%%%%%%%%%%%%%%%
\fancyhead[RE]{[विनियोग\textemdash\ }
{\bl\noindent गुणविधौ च धात्वर्थस्य \blfootnote{पाठा०\textemdash\ $^{१}$साध्यत्वेनैव टीका.}\footnotemark साध्यत्वेनान्वयः~। क्वचिदाश्रयत्वेनापि यथा
{\qtl दध्नेन्द्रियकामस्य जुहुयात्} इत्यत्र दधिकर\footnoteA{करणकत्वेन.}णत्वेने- }\\
\hrule
\vspace{3mm}
\noindent
{\qt स्वाराज्यकामो यजेत्} इत्यत्र~। तत्र च वाजपेयशब्दस्य पेयसुराद्रव्यवाचित्वेन गुणविध्याशङ्कायां मत्वर्थलक्षणाप्रसङ्गाक्षेपे तन्त्रेण युगपत्स्वाराज्यफलवाजपेयगुणसंबन्धो यागस्य पूर्वपक्षितः~। तत्र विरुद्धत्रिकद्वयप्रसङ्गापादनेन यथोक्तद्रव्यनिमित्तं वाजपेयशब्दस्य नामधेयत्वं राद्धान्तितम्~। विरुद्धत्रिकद्वयप्रसङ्गश्च फलमुद्दिश्य याग उपादीयते फलमनूद्य यागो विधीयते~। फलं प्रधानं याग उपसर्जनम्~। एवं यागमुद्दिश्य वाजपेय उपादीयते, यागमनूद्य वाजपेयो विधीयते~। यागः प्रधानं वाजपेय उपसर्जनम्~।~फलस्योद्देश्यत्वं च मानसापेक्षो विषयत्वाकारः~। यागस्योपादेयत्वं त्वनुष्ठीयमानत्वाकारः~। तौ चोभौ मनःशरीरोपाधिकौ धर्मौ भवतः~। अनुवाद्यत्वविधेयत्वे तु शब्दोपाधिकौ धर्मौ स्तः~। अनुवाद्यत्वं नाम मानान्तरज्ञातस्यानुकथ्यमानत्वम्~।~विधेयत्वं चाज्ञातस्यानुष्ठेयत्वेन प्रतिपाद्यमानत्वम्~। फलस्य प्राधान्यं नाम साध्यत्वेन, यागस्योपसर्जनत्वं
च साधनत्वेन बोध्यम्~। तथा यागस्योद्देश्यत्वं नाम मानान्तरसिद्धस्य विधेयान्वयितया निर्देश्यत्वमन्यद्यथोक्तम्~। न चैवं होमे विरुद्धत्रिकद्वयप्रसङ्गः~। तस्य साध्यत्वेन साधनत्वेन चात्र विधौ युगपत्तन्त्रेणान्वयानङ्गीकारात्, किंतु साध्यत्वेनैव~। तथा च होमस्योद्देश्यत्वमनुवाद्यत्वं प्रधानत्वमेव, नतूपादेयत्वादिकम्~। तस्मान्न
विरुद्धत्रिकद्वयापत्तिरित्याशयेनाह\textendash\ {\br गुणविधौ चेति~।~धात्वर्थस्य साध्यत्वेनैवान्वय इति~।} धात्वर्थस्य यागदानहोमादेः साध्यत्वेनैवेत्यनेन
साधनत्वेनान्वयं वारयति~।~साधनत्वेनान्वयस्तु धात्वर्थस्योत्पत्तिविधावधिकारविधौ च भवति, नतु गुणविधावित्यर्थः~।\\

 {\br ननु} गुणविधौ धात्वर्थस्य साध्यत्वेनैवान्वये {\qt दध्नेन्द्रियकामस्य जुहुयात्} इत्यत्र दध्ना होमं भावयेदिन्द्रियकामस्येति वाक्यार्थः स्यात्~। तथा चेन्द्रियस्य साध्यत्वेनानन्वये तस्याफलत्वप्रसङ्गः~। न च नात्र गुणविधिः, गुणपदस्यानर्थकत्वप्रसङ्गात्~। होमस्योभयरूपेणान्वये तु पूर्वोक्तदोषापत्तिश्चेत्याशङ्क्याह\textendash\ {\br क्वचिदित्यादिना~।}  यद्वा, {\qt गुणविधौ धात्वर्थस्य साध्यत्वेनैवान्वयो नान्यथा}~। यत्र तु
\newpage
%%%%%%%%%%%%%%%%%%%%%%%%%%%%%%%%%%%%%%%
\fancyhead[LO]{विधिः ]}
{\bl\noindent न्द्रियं भावयेत्~। तच्च किंनिष्ठमित्याकाङ्क्षायां संनिधिप्राप्तहोम आश्रयत्वेनान्वेति~।}\\
\hrule
\vspace{3mm}
\noindent
तृतीययोपात्तं दध्यादिगुणकरणत्वं तस्य प्रत्ययार्थत्वेन दध्यादिगुणादपि प्रधानत्वात्फलभावनायां करणत्वत्वेन विधीयते, तत्र तु धात्वर्थस्याश्रयत्वेनैवान्वय इत्याह\textendash {\br क्वचिदित्यादिना~।}  क्वचिदिति यत्र दध्यादिगुणकरणत्वस्य फलभावनायां करणत्वेन विधानं तत्रेत्यर्थो न तु गुणविधावित्यर्थः~। {\br तच्चेति~।}  तृतीययोपात्तं दधिकरणत्वं चेत्यर्थः~। होमाश्रयदधिकरणत्वेनेन्द्रियं भावयेदिति वाक्यार्थः~।~तथा च करणस्य कर्तृव्यापारव्याप्यत्वनियमात् केवलदध्नः कर्तृव्यापारानाविष्टस्य
करणत्वानुपपत्तेर्होमस्य च वाक्यान्तरप्राप्तत्वात्तयोर्विध्यनुपपत्तेः~। होमस्य गुणसंबन्धविधाने फलपदस्यानर्थकत्वप्रसङ्गात्तस्य फलसंबन्धविधौ च गुणपदस्यानर्थक्यापातात्फलगुणोभयसंबन्धविधौ च प्राप्ते कर्मण्यनेकपदार्थविधाने वाक्यभेदप्रसङ्गात् प्राप्ते कर्मण्यनेकपदार्थविधानस्य च वाक्यभेदापादकस्य {\qt प्राप्ते कर्मणि नानेको विधातुं शक्यते गुणः~। अप्राप्ते तु विधीयन्ते बहवोऽप्येकयत्नत} इति वचनविरोधेन स्वीकर्तुमशक्यत्वात्तृतीययोपात्तस्य दधिकरणत्वस्य होमनिरूपितत्वेन फलभावनायां करणत्वमत्र विधीयत इति भावः~। प्राप्ते कर्मणीत्यत्र कर्मणो द्रव्याधुपलक्षणत्ववद्गुणस्यापि प्रधानोपलक्षणत्वमेकोद्देशेनानेकविधावेव वाक्यभेदात्~। अत एव
ग्रहैकत्वाधिकरणे {\qt ग्रहं संमार्ष्टि} इत्यत्र ग्रहोद्देशेनैकत्वसंमार्जनविधौ वाक्यभेदाद्ग्रहैकत्वमविवक्षितमित्युक्तम्~। तेन {\qt दध्नेन्द्रियकामस्य जुहुयात्} इत्यत्रेन्द्रियसंबन्धस्य
प्रधानसंबन्धत्वेऽपि न क्षतिः~। {\br ननु} कर्मणो द्रव्योपलक्षणत्वं भवतु, ग्रहस्य द्रव्यत्वाद्गुणस्य प्रधानोपलक्षणत्वं कुत्र चरितार्थमिति चेत्, न ~। रेवत्यधिकरणे चरितार्थत्वात्~।~तत्र हि {\qt एतस्यैव रेवतीषु वारवन्तीयमग्निष्टोमसाम कृत्वा पशुकामो ह्येतेन यजेत्} इति~। तत्र च {\br ननु} पशुकामस्य {\qt रेवतीर्नः सधमाद} इत्यादिरेवतीष्वृक्षु वारवन्तीयं साम गातव्यं, तथा चात्र रेवतीनामृचां वारवन्तीयनामकेन साम्ना यः संबन्धः सोऽयं पशुफलायाग्निष्टुति गुणो विधीयते~। एतस्यैवेत्यत्र प्रकृतपरामर्शकेनैतच्छब्देनान्यव्यावर्तकेन चैवकारेणाग्निष्टुतः समर्प्यमाणत्वादिति चेत् , न~। रेवत्यृगाधारकवारवन्तीयसाम्नोऽग्निष्टुत्कर्मसाधनत्वं पशुफलसाधनत्वं चेत्युभयस्य विधाने वाक्यभेदप्रसङ्गात् । ततश्च पशुफलकं रेवत्यृगाधारकवारवन्तीयगुणविशिष्टं कर्मान्तर-
\newpage
%%%%%%%%%%%%%%%%%%%%%%%%%%%%%%%%%%%%
\fancyhead[RE]{[विधेः श्रुत्यादि\textemdash\ }
 \begin{center}
     \textbf{विधेः श्रुत्यादिषट्प्रमाणानि~।~}
 \end{center} 

{\bl एतस्य विधेः सहकारिभूतानि {\al षट्प्रमाणानि - श्रुति-लिङ्ग-वाक्य-प्रकरण-स्थान-समाख्यारूपाणि~।} एतत्सहकृतेनानेन विधिनाङ्गत्वं परोद्देशप्रवृत्तकृतिसाध्यत्वरूपं
पारार्थ्यापरपर्यायं ज्ञाप्यते~।}\\
\hrule
\vspace{3mm}
\noindent
मत्र विधीयते न प्रकृते गुणः~। एतच्छब्दस्य तु बुद्धिस्थपरामर्शकत्वेनाप्युपपत्तेः~। एवकारस्य चायोगव्यवच्छेदकत्वमन्ययोगव्यवच्छेदकत्वं वोपपद्यत इत्युक्तम्~।
तस्मात्प्राप्ते होमे गुणफलसंबन्धोभयविधौ वाक्यभेदो दुष्परिहर इत्यलम्~।\\

 {\br एतस्य विधेरिति~।}  लक्षितस्य विनियोगविधेरित्यर्थः~। {\br एतत्सहकृतेनेति~।}  एतत्प्रमाणषट्कसहकृतेनेत्यर्थः~। विनियोगविधिसहकारित्वं च तेषां
विनियोगविधिकृतविनियोगे प्रमाणत्वाद्भवतीति बोध्यम्~। अनेन विधिनाङ्गत्वं ज्ञाप्यत इत्यन्वयः~। अङ्गत्वं लक्षयति\textendash {\br परोद्देशेत्यादिना~।}  यद्वा,
{\br ननु} न शेषत्वापरनामकस्याङ्गत्वस्य विनियोगविधिबोध्यत्वं संभवति तस्यानिरूपणात्~। तदनिरूपणं च लक्षणप्रमाणाभावान्न तावदविनाभूतत्वं तत्त्वम् , आग्नेयादीनां षण्णामविनाभूतत्वेन परस्परं शेषत्वापत्तेः~। नापि प्रयोज्यत्वं, शेषत्वं {\qt पुरोडाशकपालेन तुषानुपवपति} इत्यत्र तुषोपवपनं प्रति शेषत्वेन श्रुतस्यापि पुरोडाशकपालस्योपवापाप्रयोज्यत्वात्~। न च विध्यन्तरविहितत्वं शेषत्वमिति वाच्यम्~। {\qt इषे त्वेति च्छिनत्ति} इत्यत्र पलाशशाखाच्छेदनस्य सत्यपि शेषत्वे विध्यादिविहितत्वेनाव्याप्तिप्रसङ्गात्~। नापि तत्र प्रत्यक्षं प्रमाणं, लोके तत्त्वेऽपि वेदे तस्य शब्दैकगम्यत्वोपगमात्~। नापि शब्दः, केनापि शब्देन शेषशेषिभावस्याप्रतीयमानत्वात्~। लोके क्रियाकारकान्वयस्यैव व्युत्पत्तिप्रयोजकत्वदर्शनेन तत्र कस्यापि शब्दस्य व्युत्पत्त्यग्रहात्~। तत्र च न हेतुरप्युपलभ्यत इति चेत् , अत्रोच्यते\textendash न तावदस्य लक्षणासंभवः, पारार्थ्यस्यैव निर्दुष्टशेषत्वलक्षणत्वात्~। नापि तत्र प्रमाणाभावः, शब्दगम्यत्वात्~। न च तत्र व्युत्पत्त्यभावात्कथं शब्दगम्यत्वमिति वाच्यम्; शेषशेषिभावस्यान्वयेऽन्तर्भावात्तत्र व्युत्पत्त्युपपत्तेः~। न हि गुणप्रधानभावमन्तरेणान्वयः संभवति, द्वयोः प्रधानयोर्गुणयोर्वा
परस्पराकाङ्क्षारहितत्वेनान्वययोग्यत्वाभावात्~। ततो यथा क्रियाकारकतदन्वयाः शब्द- 
\newpage
%%%%%%%%%%%%%%%%%%%%%%%%%%%%%%%%%%%%%
\fancyhead[LO]{षट्प्रमाणानि ]}
{\bl
तत्र {\al निरपेक्षो रवः श्रुतिः~}।~सा च त्रिविधा - {\al विधात्री,}}\\
\hrule
\vspace{3mm}
\noindent
गम्यास्तथा तदन्वयान्तर्गतः शेषशेषिभावोऽपि शब्दगम्य एव~। नापि तत्र हेतोरभावः, विवादास्पदः प्रयाजानुयाजप्रोक्षणादिः शेषो भवितुमर्हति पारार्थ्यात् भृत्यादिवदिति हेतुसत्त्वात्~। न च पारार्थ्यस्यैव लक्षणत्वे हेतुत्वे च सांकर्यप्रसङ्ग इति वाच्यम् ; सेनानां महारथिवदाकारभेदेन तद्भेदोपपत्तेः~। दृष्टान्ते गृहीतव्याप्तिं हि सहायीकृत्य बोधकत्वाकारो हेतुरितरव्यावृत्त्या बोधकत्वाकारश्च लक्षणमिति तद्भेद इत्यभिप्रेत्याङ्गत्वं लक्षयति {\br परोद्देशेति~।}  परं स्वर्गादिरूपमुत्कृष्टं साध्यं फलं तदुद्देशेन संकलनया मनसि सिद्धवत्करणेन तत्साधनयागादिषु प्रवृत्तस्य पुरुषस्य कृतिसाध्यत्वरूपं कृतिव्याप्यत्वरूपमित्यर्थः~।~तथा च स्वर्गफलोद्देशेन दर्शादिषु प्रवृत्तपुरुषकृतिव्याप्यत्वं प्रयाजानुयाजावघातप्रोक्षणादीनां सुप्रसिद्धमिति तेषां शेषत्वम्~।~यद्वा, {\qt पर}शब्दो दर्शादिपरः~। तथा च तदुद्देशेन प्रवृत्तपुरुषकृतिव्याप्यत्वं प्रयाजादीनां भवतीति तेषां
तत्त्वम् , दर्शादेस्तु प्रयाजाद्युद्देशेन प्रवृत्तपुरुषकृतिव्याप्यत्वाभावान्न तत्रातिव्याप्तिः~। केवलप्रयाजाद्युद्देशेन कस्यचिदपि पुरुषस्य प्रवृत्त्यभावादिति ध्येयम्~। {\br ननु} {\qt कर्माण्यपि जैमिनिः फलार्थत्वात्, तत्फलं च पुरुषार्थत्वात्, पुरुषश्च कर्मार्थत्वात्} इत्यत्र जैमिनिसूत्रत्रये कर्मफलपुरुषाणामपि शेषत्वमुक्तं, तत्र च कर्मणां फलोद्देशेन प्रवृत्तकृतिसाध्यत्वेऽपि फलपुरुषयोस्तु पुरुषकर्मोद्देशेन प्रवृत्तपुरुषकृत्यसाध्यत्वात्कथं शेषत्वमिति चेत् , अत्र वक्तव्यम्\textendash स्वर्गस्य साक्षात्कृतिसाध्यत्वाभावेऽपि
यदा पुरुषः स्वस्वर्गोद्देशेन दर्शादिषु प्रयतते तदैव कालान्तरेऽपूर्वद्वारेण स्वर्गो जायते, नान्यथेति भवति पुरुषकृतिव्याप्यत्वं तस्य, तथा स्वर्गफलोद्देशेन दर्शाद्युद्देशेन वा प्रवृत्तः पुरुषो यदैव तदनुष्ठानानुकूलं स्वं प्रयत्नेन संपादयति तदैव कर्म निष्पद्यते नान्यथेति तस्यापि भवति पुरुषकृतिव्याप्यत्वमिति सर्वमनवद्यम्~।\\

 तत्र श्रुति लक्षयति\textendash\ {\br तत्र निरपेक्षो रव इति~। तत्र} षण्णां श्रुत्यादिप्रमाणानां मध्ये इत्यर्थः~। स्वकरणीये शेषत्वबोधे प्रमाणान्तरनिरपेक्षः शब्दः श्रुतिरित्यर्थः~। रव इत्युक्ते वाक्यादावतिप्रसङ्गस्तद्वारणाय निरपेक्ष इत्युक्तम्~।~तां च श्रुतिं विभजते\textendash\ {\br सा च त्रिविधेति~।} विधात्री विधानकर्त्री, अभि- 
\newpage
%%%%%%%%%%%%%%%%%%%%%%%%%%%%%%%%%%%%
\fancyhead[RE]{[ विनियोक्त्री  श्रुतिस्त्रिधा ]}
{\bl\noindent {\al अभिधात्री, विनियोक्त्री च~}। तत्राद्या लिङाद्यात्मिका~। द्वितीया व्रीह्यादिश्रुतिः~। यस्य च शब्दस्य श्रवणादेव संबन्धः प्रतीयते सा विनियोक्त्री~।}
\begin{center}
 विनियोक्त्री श्रुतिस्त्रिधा   
\end{center}
 
{\bl सापि त्रिविधा\textendash विभक्तिरूपा, एकाभिधानरूपा, एकपदरूपा चेति । तत्र विभक्तिश्रुत्या अङ्गत्वं यथा {\qtl व्रीहिभिर्यजेत} इति तृतीयाश्रुत्या व्रीहीणां यागाङ्गत्वम्~। तदपि पुरोडाशप्रकृतितया~। यथा पशोर्हृदयादिरूपहविःप्रकृतितया यागाङ्गत्वम्~।}\\
\hrule
\vspace{3mm}
\noindent
धात्री अभिधानकर्त्री, विनियोक्त्री विनियोगकर्त्री~। तत्राद्यामुदाहरति {\br तत्राद्या लिङाद्यात्मिकेति~। तत्र} तिसृणां श्रुतीनां मध्य इत्यर्थः~। आदिना लेडादिग्रहः~।
अभिधात्रीमुदाहरति{\br  द्वितीया व्रीह्यादीति}~। तृतीयां विनियोक्त्रीं लक्षयति {\br यस्य चेत्यादिना~। संबन्ध इति~।} विनियोज्यविनियोजकभावः संबन्ध इत्यर्थः~। शेषशेषिणोरिति वा शेषः\footnotemark~~। \\

 विनियोक्त्रीमपि श्रुतिं विभजते{\br सापि त्रिविधेति~।} तत्र विभक्तिरूपां  विनियोक्त्रींमुदाहरति {\br तत्रेति~।} यद्वा, {\br ननु} विनियोक्त्र्याः श्रुतेर्लक्षणविभागावनुपपन्नौ तस्या अङ्गत्वाबोधकत्वादित्याशङ्क्य तत्र विभक्तिश्रुतेरङ्गत्व बोधकत्वं दर्शयति {\br तत्रेति~।} तिसृणां विनियोक्त्रीणां मध्य इत्यर्थः~। {\br विभक्तिश्रुत्याङ्गत्वं }बोध्यत एव तथोपलब्धेरिति शेषः~। कुत्रोपलम्भ इति वीक्षायां तत्रोदाहरणमाह\textendash {\br यथा व्रीहिभिरिति~। तदपीति~।} व्रीहीणां यागाङ्गत्वमपीत्यर्थः~। व्रीहीणां
पुरोडाशप्रकृतितया यागाङ्गत्वे दृष्टान्तमाह\textendash {\br यथा पशोरिति~।} {\qt \noindent अथ\textendash  हृदयस्याग्रेऽवद्यत्यथ वक्षस} इत्यादिशास्त्रात्,
पशोर्हृदयादिरूपहविरुपादानतयैव यागाङ्गत्वं न साक्षात्, तद्वदत्रापीत्यर्थः~। क्वचित्तु साक्षात्पशोरेव यागाङ्गत्वं यथा पात्नीवतयागे~। तत्र हि पशोराग्नेयस्य जीवत एवोत्सर्गः क्रियते, {\qt पर्यग्निकृतं पत्नीवन्तमुत्सृजन्ति} इति वाक्यात्~। तथा च यत्र पशोर्विशसनं हृदयाद्य-
\blfootnote{टिप्प \textemdash\ $^{१}$शेष इति पूरणीय इत्यर्थः~।}
\newpage
%%%%%%%%%%%%%%%%%%%%%%%%%%%%%%%%%%%%%%
\fancyhead[LO]{[द्वि०रू०विनि०उदा०]}
  \begin{center}
  \textbf{तृतीयाविभक्तिरूपाया उदाहरणम्}    
  \end{center}
  
{\bl {\qtl अरुणया एकहायन्या गवा सोमं क्रीणाति} इत्यस्मिन्वाक्ये आरुण्यस्यापि तृतीयाश्रुत्या क्रयाङ्गत्वम्~। तदपि गोरूपद्रव्यपरिच्छेदद्वारा न तु साक्षात् , अमूर्तत्वात्~।}
\begin{center}
\textbf{द्वितीयारूपाया विनियोक्त्र्या उदाहरणम्}
\end{center}

{\bl {\qtl व्रीहीन्प्रोक्षति} इति प्रोक्षणस्य व्रीह्यङ्गत्वं द्वितीयाश्रुत्या~। तच्च}\\
\hrule
\vspace{3mm}
\noindent
वदानं च भवति तत्र हृदयादिप्रकृतिरेव पशुरिति सिद्धम्~। तृतीयाविभक्तिरूपाया विनियोक्त्र्या उदाहरणान्तरमाह\textendash {\br आरुण्यस्यापीत्यादिना~।} {\qt अरुणया पिङ्गाक्ष्या एकहायन्या\blfootnote{पाठा०\textemdash\ $^{१}$गवा.}\footnotemark\ सोमं क्रीणाति} इति ज्योतिष्टोमप्रकरणे श्रुतस्यारुणिमगुणस्येत्यर्थः~। क्वचित्पुस्तके मूलग्रन्थ एवेदं वाक्यमुदाहृतम्~। अत्र {\qt अरुणा} शब्दोऽरुणिमानं गुणमाचष्टे, तस्य गुणिविषयतया प्रयुक्तस्यापि {\qt नागृहीतविशेषणाविशिष्टबुद्धिः} इति न्यायेन गुणबोधकत्वादन्वयव्यतिरेकसिद्धगुणमात्रव्युत्पत्तिकत्वाच्च~। {\qt पिङ्गाक्षी} शब्दस्तु पिङ्गलवर्णविशिष्टाक्षिमद्द्रव्यवाचको भवति~। {\qt एकहायनी} शब्दश्चैकसंवत्सरविशिष्टगोद्रव्यवाचकः~। तौ च शब्दौ यद्यप्येकगोवाचकौ भवतस्तथापि विशेषणभूतधर्मभेदाच्छब्दद्वयमुपपद्यते, तच्च युगपत्प्रवृत्तं सद्धर्मद्वयविशिष्टं गोद्रव्यं क्रयसाधनत्वेन विधत्ते~। अरुणशब्दस्त्वरुणिमगुणस्य कारकाणां क्रियान्वयनियमेन सोमक्रयणान्वयेऽपि साक्षात्तत्साधनत्वेनामूर्तमरुणिमानं गुणं न विदधाति, अमूर्तस्य
साक्षात्तत्साधनत्वासंभवात् , किंतु सोमक्रयणसाधनीभूतगोद्रव्यपरिच्छेदकत्वेन तत्साधनम्~। तथा च परस्परमनन्वितानामेवारुण्यपिङ्गाक्षीत्वादीनां कारकाणां क्रियान्वयनियमात्करणविभक्तिभिः सोमक्रयणेऽङ्गत्वेनान्वये सत्यारुण्यादेश्चगुणत्वेनामूर्तस्य स्वतः सोमक्रयणसाधनत्वायोगात्तत्साधनगोद्रव्यपरिच्छेदकत्वेन पश्चात्परस्परं
पार्ष्णिकान्वयो भवति एकहायनी गौः, सा पिङ्गाक्ष्यरुणा चेत्याशयवानाह\textendash {\br तदपीति~।} क्रयाङ्गत्वमपीत्यर्थः~।\\

 द्वितीयारूपां विनियोक्त्रीं श्रुतिमुदाहरति {\br व्रीहीन्प्रोक्षतीति~।  तच्चेति~।}  प्रोक्षणस्य व्रीह्यङ्गत्वं चेत्यर्थः~। तच्चेत्यस्यापूर्वसाधनत्वप्रयुक्तमित्यनेनान्वयः~।
\newpage
%%%%%%%%%%%%%%%%%%%%%%%%%%%%%%%%%
\fancyhead[RE]{[द्वि०विनि०उदा०]}
{\bl\noindent प्रोक्षणं न व्रीहिस्वरूपार्थम् , तस्य तेन विनाप्युपपत्तेः~। किंत्वपूर्वसाधनत्वप्रयुक्तम्~। व्रीहीनप्रोक्ष्य यागानुष्ठानेऽपू\blfootnote{पाठा०\textemdash\ $^{१}$अपूर्वानुत्पत्तेः.}\footnotemark र्वानुपपत्तेः~। एवं स\footnote{सर्वेष्वङ्गेषु.}र्वेष्वपूर्व-प्रयुक्तमङ्गत्वं बोध्यम्~।}
\begin{center}
 \textbf{द्वितीयाविनियोक्त्र्या उदाहरणम्}   
\end{center}
 
{\bl एवम् {\qtl इमामगृभ्णन्रशनामृतस्येत्यश्वाभिधानीमादत्त} इत्यत्र द्वितीयाश्रुत्या मन्त्रस्याश्वाभिधान्यङ्गत्वम्~।}\\
\hrule
\vspace{3mm}
\noindent
प्रोक्षणं व्रीहिस्वरूपार्थमेव स्यादिति कुतस्तस्यापूर्वप्रयुक्तं व्रीह्यङ्गत्वमित्यत आह\textendash {\br प्रोक्षणं नेति~।} प्रोक्षणस्य व्रीहिस्वरूपार्थत्वाभावे हेतुमाह\textendash {\br तस्येत्यादिना~।} तस्य व्रीहिस्वरूपस्य~। तेन विना प्रोक्षणेन विनेत्यर्थः~।~{\br ननु} यागानुष्ठानेनैवापूर्वसिद्धौ किं प्रोक्षणेन व्रीहिस्वरूपानुपयोगिनेत्यत आह\textemdash\ {\br व्रीहीनप्रोक्ष्येति~।} अनुपनीतानुष्ठितवेदाध्ययनस्यापूर्वाजनकत्ववद्व्रीहीणां प्रोक्षणं न कृत्वा तैरनुष्ठितस्य यागस्यापूर्वजनकत्वानुपपत्तेरित्यर्थः~। प्रोक्षणे
सिद्धमपूर्वप्रयुक्तं व्रीह्यङ्गत्वमन्यत्रातिदिशति {\br एवमित्यादिना~।} एवं दर्शपूर्णमासप्रकरणसहकृतया द्वितीयाश्रुत्या प्रोक्षणस्य तण्डुलनिष्पत्तिप्रणालिकया व्रीहीणामपूर्वसाधनत्वप्रयुक्ताङ्गत्ववत्सर्वेषामङ्गानामवघातादीनां तत्तत्प्रमाणवशादपूर्वसाधनत्वप्रयुक्तमेवाङ्गत्वमित्यर्थः~।\\

 द्वितीयारूपाया विनियोक्त्र्याः श्रुतेरुदाहरणान्तरमाह\textendash {\br एवमित्यादिना~। एवम्,} द्वितीयाश्रुत्या प्रोक्षणस्य व्रीह्यङ्गत्ववदित्यर्थः~। {\br ऋतस्येति~।} सत्यफलस्येत्यर्थः~। तथा च सत्यफलसंबन्धिनीं  रशनामिमां गृहीतवन्त इति मन्त्रार्थः~। {\br अश्वाभिधान्यङ्गत्वमिति~।} अश्वाभिधान्या अश्वरशनाया अङ्गत्वमित्यर्थः~। केचित्तु वाक्यीयमिमं विनियोगमाहुः~। अन्ये तु वाक्याल्लिङ्गस्य बलीयस्त्वेन यावद्वाक्यादश्वाभिधान्यङ्गत्वं भवति तावद्रशनामात्राङ्गत्वमेव स्यादिति तत्र दूषणमाहुः~। वाक्यविनियोगवादिनस्तु मन्त्रस्य न रशनामात्रप्रकाशकत्वं किंतु ऋतस्य सत्यफलसाधनभूतस्याश्वस्येमां रशनां गृहीतवन्त इत्यश्वरशनाप्रकाशकत्वमेवेति,  लिङ्गसहकृतादश्वाभिधानीमादत्त इति वाक्यान्मन्त्रस्याश्वरशनादाने
\newpage
%%%%%%%%%%%%%%%%%%%%%%%%%%%%%%%%%%%
\fancyhead[LO]{[ स०वि०उदा० ]}
\begin{center}
\textbf{सप्तमीविभक्तिविनियोक्त्र्या उदाहरणम् }    
\end{center}

{\bl {\qtl यदाहवनीये जुहोति} इत्याहवनीयस्य होमाङ्गत्वं सप्तमीश्रुत्या~। एवमन्योऽपि विभक्तिश्रुत्या विनियोगो ज्ञेयः~।\\

{\qtl पशुना यजेत} इत्यत्रैकत्वपुंस्त्वयोः समानाभिधानश्रुत्या कारकाङ्गत्वम्~। यजेतेत्याख्याताभिहितसंख्याया भा\footnotemark वनाङ्गत्वं समानाभिधानश्रुतेरेव
पदश्रुत्या च यागाङ्गत्वम्~।}\\
\hrule
\vspace{3mm}
\noindent
विनियोग इति स्वाशयमाहुः~। रशनाविशेषस्य न लिङ्गमिति तु मूलग्रन्थतात्पर्यम्~।\\

 सप्तमीविभक्तिरूपां विनियोक्त्रीं श्रुतिमुदाहरति {\br यदाहवनीय इति~। एवमन्योऽपीति~।} {\qt दध्ना जुहोति, पयसा जुहोति} इत्यादौ होमानुवादेन दध्यादेस्तदङ्गत्वेन तृतीयाविभक्तिश्रुत्या विनियोग इत्यर्थः~।~\\

 एकाभिधानरूपामेकपदरूपां च विनियोक्त्रीं श्रुतिमुदाहरति  {\br पशुना यजेतेतीति~।} पशुनेत्यत्रैकपदश्रुत्या ह्येकत्वपुंस्त्वयोः पशुद्रव्याङ्गत्वमेकाभिधानश्रुत्या च कारकाङ्गत्वं च भवति~। यजेतेत्यत्राप्याख्याताभिहितसङ्ख्याया एकाभिधानश्रुत्या भावनाङ्गत्वमेकपदश्रुत्या च यागाङ्गत्वं च भवतीति भावः~। यजेतेत्यभिहितसङ्ख्यायाः समानाभिधानश्रुतेरेव भावनाङ्गत्वमिति संबन्धः~। {\br समानाभिधानश्रुतेरेवेति~।} {\qt एक} शब्दरूपप्रत्ययश्रुतेरेवेत्यर्थः~। अभिधीयतेऽनेनेति व्युत्पत्त्या {\qt अभिधान} शब्देन शब्द उच्यते~। {\br पदश्रुत्या चेति~।} यजेतेत्येकपदश्रुत्या चेत्यर्थः~।\\

 {\br अत्रेदं बोध्यम्}\textendash यथा {\qt दशापवित्रेण ग्रहं संमार्ष्टि} इत्यत्रैकत्वमुद्देश्यगतत्वेनाविवक्षितं तथा प्रकृते नोद्देश्यगतमेकत्वं किंतु विधेयगतत्वेन विवक्षितमेव~। न च ग्रहं संमार्ष्टीत्यत्र नोद्देश्यगतमेकत्वं किंतु स्वयं विधेयमिति वाच्यम् ; {\qt ग्रह संमृज्याद्यं संमृज्यात्तं चैकम्} इति वाक्यभेदप्रसङ्गात्~। तथा च ग्रहमिति द्वितीयया
ग्रहस्येप्सिततमत्वेनोद्देश्यत्वात्प्रयोजनत्वाच्च प्राधान्यं गम्यते~। संमार्गस्तु
\blfootnote{पाठा०\textemdash\ $^{१}$अर्थभावनाङ्गत्वम्}
\newpage
%%%%%%%%%%%%%%%%%%%%%%%%%%%%%%%%%
\fancyhead[RE]{[ अमूर्ताया अपि}
\begin{center}
 \textbf{अमूर्ताया अपि भावनाङ्गत्वम् }    
\end{center}

{\bl न चामूर्तायास्तस्याः कथं भावनाङ्गत्वमिति वाच्यम्~। कर्तृ- }\\
\hrule
\vspace{3mm}
\noindent
ग्रहं प्रति गुणभूत एव~। ततश्च {\qt प्रतिप्रधानं गुण आवर्तनीयः} इति न्यायाद्यावन्तो ग्रहास्तावतां  सर्वेषां संमार्ग इति निश्चये सति कति ग्रहाः संमार्जनीया इति बुभुत्साया अभवादुद्देश्यगतमेकत्वं श्रूयमाणमपि न विवक्ष्यते~। यस्य च विशेषणस्य विवक्षामन्तरेणोद्देश्यप्रत्ययो न पर्यवस्यति तादृशं विशेषणमुद्देश्यगतमपि विवक्ष्यत एव, यथा तत्रैव ग्रहत्वं
तद्विवक्षामन्तरेण चोद्देश्यस्वरूपं ज्ञातुमशक्यमेव उद्देश्यतावच्छेदकनिर्णयमन्तरेण चमसेष्वपि संमार्गप्रसङ्गात्~। तेन च तत्र तद्वारणं फलम्~। {\qt पशुना यजेत} इत्यत्र तु यागं प्रति पशोर्विधेयत्वेन गुणभूतत्वात् {\qt प्रतिप्रधानं गुण आवर्तनीयः} इति न्यायप्रवेशो नास्ति~। यागस्य च प्रधानत्वादिति कियता पशुना यागः कर्तव्य इति बुभुत्सायाः
सत्त्वादेकवचनेन प्रतीयमानं विधेयगतमेकत्वं विवक्षितमेव~। किंच लिङ्गसंख्याविशेषितस्यैकपदोपात्तस्य पशुद्रव्यरूपकारकस्य विधेयपशुद्वारा तत्तल्लिङ्गसंख्यादेरपि क्रियाङ्गत्वादेकत्वादिकं विवक्षितम्~। किंच तृतीयया विभक्त्याऽभिहितयोर्लिङ्गसंख्ययोर्विभक्त्यभिहिततया करणकारकशक्त्यात्मसात्कृतयोः प्रातिपदिकार्थपशुद्रव्येण सह संबन्धमनादृत्य पशुवदेव साक्षाद्यागक्रियाङ्गत्वेन विधौ पश्चादरुणैकहायनीन्यायेन वा परस्परमन्वयोऽपि भवति {\qt यो यागाङ्गत्वेन विहितः पशुः स एकः पुमांश्चेति~}।
तस्मात् सर्वथा पश्वेकत्वादिकं विवक्षितमेवेति~।\\

{\doublespacing {\br ननु} संख्याया न यागाङ्गत्वं भावनाङ्गत्वं च भवति, गुणत्वेनामूर्तत्वात्~। न ह्यमूर्तस्य रूपादेः कुत्रचिदङ्गत्वं दृश्यते, व्रीह्यादिद्रव्यस्यैव मूर्तभूतस्य ह्यङ्गत्वमित्याशङ्क्य परिहरति {\br न चेत्यादिना~।} भावनाया यागस्याप्युपलक्षणत्वम्~। न चेति प्रतिषेधे हेतुमाह\textendash {\br कर्त्रिति~।} कर्तुरर्थभावनारूपव्यापारवतो यागकर्तुः परिच्छेदद्वारा संख्याया भावनाद्यङ्गत्वोपपत्तेरित्यर्थः~। {\br नन्वेवं} समानाभिधानश्रुत्या साक्षादेव कर्तरि संख्याख्यातार्थभूतान्वये तु
तस्यैवाख्यातार्थत्वात् भावना तु कर्तृव्यापारभूता धातुनापि लभ्यते~। तथा चोक्तम्\textendash {\qt फलव्यापारयोर्धातुराश्रये तु तिङः स्मृताः} इति फलं विक्लित्त्यादि व्यापारस्तु भावनाभिधस्तत्र धातुः स्मृतः~। आश्रये तु फलाश्रये कर्मणि व्यापाराश्रये कर्तरि च तिङः स्मृता}
\newpage
%%%%%%%%%%%%%%%%%%%%%%%%%%%%%%%%%
\fancyhead[LO]{भावनाङ्गत्वम् ]}
{\bl\noindent 
परिच्छेदद्वारा तदुपपत्तेः~। कर्ता चाक्षेपलभ्यः~। }\\
\hrule
\vspace{3mm}
\noindent
इति वैयाकरणवृद्धवचनपदार्थः~। न च कर्तुराख्यातवाच्यत्वमप्रामाणिकमिति वाच्यम्~। {\qt लः कर्मणि च भावे चाकर्मकेभ्यः} ( ३।४। १९) इति मुनिप्रणीतसूत्रस्यैव चकारात् {\qt कर्तरि कृत्} इति सूत्रोक्तकर्तरीत्यनुकर्षणेन लकाराणां सकर्मकेभ्यो धातुभ्यः कर्मणि कर्तरि चाकर्मकेभ्यश्च भावे कर्तरि च विधायकस्य प्रमाणत्वात्~। तस्मात् कर्तर्येव
समानाभिधानश्रुत्याख्यातार्थसंख्या संबध्यते, न भावनायामित्याशङ्क्याह\textendash {\br कर्ता चाक्षेपलभ्य इति~।} आक्षेपोऽनुमानमर्थापत्तिर्वा तादृशाक्षेपेण लभ्योऽनुमेयः कल्प्यो वेत्यर्थः ~। तथा च भावना साश्रया गुणत्वाद्व्यापारविशेषत्वाच्च संप्रतिपन्नगुणवत्तादृशव्यापारविशेषवच्च~। तथा लोके व्यापारविशेषस्य निराश्रयस्यादर्शनाद्भावनारूपो व्यापारविशेषोऽप्यन्यथानुपपद्यमानः स्वाश्रयमाक्षिपतीत्यर्थापत्तिः~। न चाचेतनस्यैव कस्यचिदाश्रयस्य लाभ इति वाच्यम्~। कृतिशब्दाभिधेयस्य भावनारूपव्यापारविशेषस्याचेतनाश्रयत्वानुपपत्तेः~। किंच भावनाक्षिप्ते च कर्तर्याख्यातस्य लक्षणास्वीकारान्न शाब्द्याः संख्याया अशाब्देन कर्त्रान्वयानुपपत्तिः~। तस्यापि लक्षणया शाब्दत्वात्~। शब्दगम्यत्वस्य हि शाब्दत्वात्~। {\br न च} कर्तुरेवाख्यातवाच्यत्वेन भावनाया आक्षेपलभ्यत्वं स्यादिति वाच्यम्~। कृतिमत एव कर्तृत्वेन कृतेरेव भावनापरनामधेयाया आकृत्यधिकरणन्यायेनाख्यातवाच्यत्वसंभवे तद्वतः कर्तुराख्यातवाच्यत्वकल्पनायां गौरवात्~। किंच यस्य हि प्रकारान्तरेणालाभः स एव शब्दस्यार्थो भवति, न तु तदन्यः शब्दवाच्यार्थः; {\qt अनन्यलभ्यः शब्दार्थः} इति न्यायात्~। अत एव न तीरं गङ्गापदार्थः तस्य लक्षणयैव प्रतिपन्नत्वात्~। एवं चोक्तन्यायेन भावनाया आख्यातवाच्यत्वे संप्रतिपन्ने तया च कर्तुराक्षेपेण लाभे पुनरपि तस्याख्यातवाच्यत्वकल्पनं न न्यायविदां शोभते~। {\br न च} {\qt लः कर्मणि च भावे चाकर्मकेभ्यः} (३।४।६९) इति मुनिस्मरणबलादेव कर्तुराख्यातवाच्यत्वं न स्वयं कल्पितमिति युक्तम्~। वाच्यवाचकभावस्योक्तन्यायसहकृतान्वयव्यतिरेकगम्यत्वेन तस्य मुनिस्मरणाप्रयोज्यत्वात्~। तत्र कर्तृकर्मपदयोः कर्तृत्वकर्मत्वपरत्वेन तस्य कृतिकर्मत्वयोर्लकारविधायकत्वसंभवाच्च~। किंच नास्य स्मरणस्य कर्तुराख्यातवाच्यत्वे प्रमाणत्वं संभवति {\qt द्व्येकयोर्द्विवचनैकवचने}
(१।४।२२)
\blfootnote{टिप्प०\textemdash\ $^{१}$वचनार्थ इत्यर्थः~।}
\newpage
%%%%%%%%%%%%%%%%%%%%%%%%%%%%%%%5
\fancyhead[RE]{[ भावनाया आख्यात\textemdash }
\begin{center}
  \textbf{भावनाया आख्यातवाच्यत्वम्}  
\end{center}
  
{\bl आख्यातेन हि भावनोच्यते~।~सा च कर्तारं विनाऽनुपपन्नेति}\\
\hrule
\vspace{3mm}
\noindent
{\qt बहुषु बहुवचनम्} (१।४।२११) इत्यनेन सूत्रेण तस्य स्मरणस्यैकवाक्यत्वेन कर्तुरेकत्व एकवचनात्मको लकारो द्वित्वे द्विवचनात्मको बहुत्वे बहुवचनात्मको लकारो भवेदित्यस्मिन्नेवार्थे प्रमाणत्वात्~। {\br न च} कर्तुराख्यातानभिधेयत्वे देवदत्तेन पचतीति प्रयोगापत्तिः~। अनभिहितयोः कर्तृकरणयोस्तृतीयाया एव विहितत्वादिति वाच्यम्~। उक्तन्यायेन कर्तुर्भावनयैवाक्षेपात्; तद्गतसंख्यायाश्चाख्यातेनैव प्रतीतेस्तृतीयायाः कर्तृप्रतिपत्त्यर्थत्वतद्गतसंख्याप्रतिपत्त्यर्थत्वयोरसंभवात् तथा चोक्तं वृद्धैः\textendash {\qt संख्यायां कारके वा धीर्विभक्त्या हि प्रवर्तते~। उभयं चात्र तत्सिद्धं भावनातिङ्विभक्तितः} इति~। न च कृतामपि ण्वुल्तृजादीनां कर्तृवाचकत्वं न स्यात्तत्र मानस्य वक्तव्यत्वादिति वाच्यम्~।~{\qt कर्तरि कृत्} इति मुनिप्रणीतसूत्रस्यैव तेषां तत्र शक्तिग्राहकत्वात्~।\\

 {\br ननु} तर्हि {\qt कर्तरि कृत्} इति सूत्रादेव {\qt लः कर्मणि} इत्यत्र कर्तृपदमनुवर्तते~। तथा च तत्र तस्य धर्मिपरत्वेऽत्रापि तत्परता न्याय्या~। अत्र धर्मपरतायां तु तत्रापि धर्मपरतैव स्यादिति चेन्न~। शब्दाधिकाराश्रयणादनुवर्तितस्य कर्तृपदस्य धर्मपरतायां बाधकाभावात्~। न च शब्दाधिकाराश्रयस्य गमकमन्तरेणासंभवात्कृतामिवाख्यातस्यापि
कर्तृवाचित्वमेवास्त्विति वाच्यम्~। कृतिमत इत्यादिना तत्र गमकस्य दर्शितत्वात्~। तथा च कृतिशब्दाभिधेयस्य कर्तृधर्मस्याख्यातवाच्यत्वे सिद्धेऽनुवर्तितस्य कर्तृपदस्य धर्मपरत्वमेव न्याय्यम्~। समुच्चयार्थकेन {\qt लः कर्मणि च} इति सूत्रस्थचकारेणैव वा कर्तृधर्मस्यैव लाभो भवति, न कर्तृपदानुवृत्तिः कर्तव्या, कल्पनालाघवात् पूर्वोक्तन्यायेन कर्तृधर्मस्य भावनारूपस्याख्यातवाच्यत्वसिद्धेश्चेत्यलमतिविस्तरेण~।\\

 तस्माद्भावनैवाख्यातवाच्या न कर्ता तद्वाच्य इत्यभिप्रेत्याह\textendash {\br आख्यातेन हि भावनोच्यत इत्यादिना~।} सा कर्तृव्यापाररूपा भावना कर्तारं विनानुपपद्यमाना पूर्वोक्तप्रकारेण कर्तारमाक्षिपतीत्यर्थः~। एवं श्रुतेः प्रमाणान्तरनिरपेक्षविनियोजकत्वं निरूप्य तस्या लिङ्गादिभ्यः पञ्चभ्यः प्रमाणेभ्यः प्राबल्य-
\newpage
%%%%%%%%%%%%%%%%%%%%%%%%%%%%%%%%%%%%%%
\fancyhead[LO]{वाच्यत्वम् ]}
{\bl\noindent तमाक्षिपति~। सेयं श्रुतिर्लिङ्गादिभ्यः प्रबला~। लिङ्गादिषु न प्रत्यक्षो विनियोजकः शब्दोऽस्ति किंतु कल्प्यः~। यावच्च तैर्विनियोजकशब्दः कल्प्यते तावत्प्रत्यक्षया श्रुत्या विनियोगस्य कृतत्वेन तेषां कल्पकत्वशक्तेर्व्याहतत्वात्~। अत एवैन्द्र्या लिङ्गान्नेन्द्रोपस्थानार्थत्वम्~। किंतु {\qtl ऐन्द्र्या गार्हपत्यमुपतिष्ठते} इत्यत्र गार्हपत्यमिति द्वितीयाश्रुत्या गार्हपत्योपस्थानार्थत्वम्~।~}\\
\hrule
\vspace{3mm}
\noindent
माह\textendash {\br सेयं श्रुतिर्लिङ्गादिभ्यः प्रबलेति~।} कुत इत्याकाङ्क्षायां तत्र हेतुमाह\textendash {\br लिङ्गादिषु न प्रत्यक्ष इत्यादिना~।} यावच्च लिङ्गादिभिर्विनियोजकः शब्दोऽर्थप्रकाशनादिना कल्प्यते तावत्प्रत्यक्षोपलब्धया श्रुत्या शेषिणा शेषसंबन्धबोधस्य कृतत्वेन लिङ्गादीनां विनियोजकश्रुतिकल्पनद्वारेण विनियोगशक्तेः प्रतिबद्धत्वादित्यर्थः~। तत्र लिङ्गाच्छ्रुतेः प्राबल्यसिद्ध्यैव तस्मादपि दुर्बलेभ्यो वाक्यादिभ्यस्तस्याः प्राबल्यसिद्धये तावल्लिङ्गात्प्राबल्यमुदाहरणं प्रदर्शयन् साधयति {\br अत एवेत्यादिना~।} अत एव लिङ्गादिभ्यः श्रुतेः प्रबलत्वादेव~। ऐन्द्र्या इति षष्ठी {\qt नेन्द्र सश्चसि} इतीन्द्रप्रकाशनसमर्थाया अपि ऋचो न लिङ्गादिन्द्रोपस्थानाङ्गत्वमित्यर्थः~। {\br ननु} यदीन्द्रप्रकाशनसामर्थ्यरूपाल्लिङ्गादैन्द्र्या ऋच इन्द्रोपस्थानार्थत्वं न भवति तर्हि तत्रान्यप्रकाशनसामर्थ्यादर्शनेन तस्याः
कस्यचिदप्युपस्थानार्थत्वासंभवाद्वैयर्थ्यमेव स्यादित्याशङ्कते {\br किंत्विति~।} यद्यपीन्द्रशब्दस्य गार्हपत्येऽग्नौ रूढिर्नास्ति तथापि तस्यैश्वर्यगुणयोगेन यागसाधनत्वेन वा
मुख्येन्द्रसदृशत्वाद्गुणवृत्तिरस्ति~। तथा चैन्द्र्यास्तत्प्रकाशनसामर्थ्यस्यापि सत्त्वेन द्वितीयाश्रुत्या तदुपस्थानार्थत्वं निर्विघ्नमुपपद्यत इति समाधत्ते {\br ऐन्द्र्या गार्हपत्यमित्यादिना~।} अत्रायमाशयः ऐन्द्र्या गार्हपत्यमुपतिष्ठते इति श्रूयते~। तत्र च {\qt कदाच-नस्तरीरसि-नेन्द्र सश्चसि दाशुषे} इत्यसावृगैन्द्री तत्रेन्द्रस्य प्रकाशनात्~। भो इन्द्र ! कदाचिदपि न सश्चसि घातको न भवसि किंत्वाहुतिं दत्तवते यजमानाय प्रीयसे इति तस्या अर्थः~। तत्रेन्द्रप्रकाशनसामर्थ्यरूपाल्लिङ्गान्मन्त्रस्येन्द्रविषयक्रियासाधनत्वं
गम्यते~। यद्यसौ मन्त्र इन्द्रप्रधानक्रियायाः
\blfootnote{पाठा०\textemdash\ $^{१}$विनियोजकः कल्प्यते~।}
\newpage
%%%%%%%%%%%%%%%%%%%%%%%%%%%%%%%%%%%%%%%%%
\fancyhead[RE]{[ भाव०आ०वाच्यत्वम् ]}
\hrule
\vspace{3mm}
\noindent 
साधको न भवेत् तदानीमनेन मन्त्रेणेन्द्रप्रकाशनं व्यर्थं स्यात्~। तस्मादेतन्मन्त्रकरणकक्रियां प्रति इन्द्रः प्रधानमित्येतादृशबुद्ध्युत्पादनं लिङ्गविनियोगः~। कासौ क्रियेति विशेषजिज्ञासायामैन्द्र्योपतिष्ठते इत्यनेनाविरुद्धपदद्वयरूपेण वाक्येनोपस्थानक्रियायां पर्यवसानं क्रियते~। तथा सत्यैन्द्रमन्त्रेणेन्द्रमुपतिष्ठेतेत्ययमर्थः पर्यवस्यति~। तथा
गार्हपत्यमित्यनया द्वितीयान्तपदरूपया श्रुत्या गार्हपत्यस्य प्राधान्यं गम्यते~। तच्च गुणभूतां यत्किंचित्करणकक्रियामन्तरेण न संभवति~। ततस्तादृशीं कांचित्क्रियां प्रति गार्हपत्यः
प्रधानमित्येतादृशबुद्ध्युत्पादनं श्रुतिविनियोगः, ऐन्द्र्योपतिष्ठत इति पदद्वये मन्त्रविशेषक्रियाविशेषयोः पर्यवसानं भवति~। तथा सत्यैन्द्रमन्त्रेण गार्हपत्यमुपतिष्ठते इत्यर्थो भवति~। यद्यपि प्रमाणत्वविशेषाच्छ्रुतिलिङ्गयोर्विरोधे प्राप्ते व्रीहियववद्विकल्पः स्यात्~। इन्द्रगार्हपत्ययोः प्रधानत्वविशेषादुपस्थानस्य च गुणत्वात् {\qt प्रतिप्रधानं गुणावृत्तिः} इति न्यायेनोपस्थानावृत्त्या श्रुतिलिङ्गयोः समुच्चयो वा स्यात्~। श्रुतिर्विनियुञ्जाना वस्तुसामर्थ्यमनुसृत्यैव विनियुङ्क्ते, अन्यथा वह्निना सिञ्चेद्वारिणा दहेदित्यपि विनियुज्येत~।
तस्मात्तदुपजीव्यत्वेन लिङ्गस्य प्रबलत्वादिन्द्र एव मन्त्रेणोपस्थयो वा स्यात्~। तथाप्यैन्द्रमन्त्रस्य गार्हपत्येऽग्नौ मुख्यवृत्त्या सामर्थ्याभावेऽपि गुणवृत्त्या सामर्थ्यस्योक्तत्वात्सामर्थ्याभावकृतप्रतिबन्धाभावान्निविना श्रुतिराशु विनियुङ्क्ते~। लिङ्गं तु विलम्बेन विनियुङ्क्ते~। तथा हि \textendash  तत्र प्रथमं मन्त्रपदानि स्वाभिधेयार्थ प्रतिपादयन्ति, तत ऊर्ध्वं मन्त्रस्य वस्तुप्रकाशनसामर्थ्यं निरूप्यते, तत ऊर्ध्वं च तत्सामर्थ्यवशात्साधनत्ववाचिनी प्रधानत्ववाचिनी च श्रुतिः कल्प्यते~। कल्पिता च श्रुतिः पश्चादैन्द्रमन्त्रेणेन्द्रमुपतिष्ठेतेति विनियुङ्क्त इति मन्त्रपदाभिधेयप्रतिपादनविनियोगयोर्मध्यवर्तिनौ सामर्थ्यनिरूपणश्रुतिकल्पनव्यापारौ भवतः~। प्रत्यक्षश्रुतिविनियोगपक्षे तु
मन्त्रपदाभिधेयप्रतिपादनमात्रेण तस्या विनियोजकत्वसंभवात् , न तौ मध्यवर्तिनौ व्यापारौ भवत इति श्रुतेः प्राबल्यात्तया लिङ्गं बाध्यते~। न च प्रत्यक्षश्रुतिविनियोगवेलायामलब्धात्मकत्वेनाप्राप्तं लिङ्गं श्रुत्या कथं बाध्यत इति वाच्यम्~। तार्तीयबाधत्वेन भविष्यत्प्राप्तिप्रतिबन्धस्यैवात्र बाधत्वात्~। बाधश्च द्विविधः {\qt प्राप्तबाधोऽप्राप्तबाधश्च}~। तत्र {\br दशमे} प्राकृतानामङ्गानां चोदकेन विकृतौ प्राप्तानां प्रत्याम्नानादर्थलोपात्प्रतिषेधाद्वा यो बाधः स प्राप्तबाधः~। यथा चोदकप्राप्तानां प्राकृतानां कुशानां शरमयं बर्हिरिति प्रतिकूलशराम्नानाद्बाधः~। यथा चावघातप्रयोजनस्य
\newpage
%%%%%%%%%%%%%%%%%%%%%%%%%%%%%%%%
\fancyhead[LO]{[ लिङ्गनिर्वचनम् ]}
\begin{center}
 \textbf{लिङ्गनिर्वचनम् }   
\end{center}

{\bl {\al शब्दसामर्थ्यं लिङ्गम्}~। यथाहुः\textendash\ {\qtl सामर्थ्य सर्वशब्दानां}}\\
\hrule
\vspace{3mm}
\noindent
वैतुष्यरूपस्य लोपात्कृष्णलेष्ववघातस्य बाधः~। यथा च पित्र्येष्टौ {\qt न होतारं वृणीत} इति प्रतिषेधाद्धोतृवरणस्य बाधः~। अप्राप्तबाधस्तु तृतीयाध्याये यो बाधः~। तत्र हि यावद्दुर्बलप्रमाणेन विनियोगः कर्तुमारभ्यते तावत्प्रबलप्रमाणेन विनियोगस्य कृतत्वादेव, तेन वा तद्बोधितेन वेतरबाधोऽप्राप्तबाध एव दुर्बलप्रमाणस्याप्रवृत्तत्वादेव\footnotemark~~। ततश्च कथंचित्प्राप्तभाविप्रवृत्तिप्रतिबन्धस्यैवात्र बाधत्वमिति सिद्धम्~। तस्माद्वितीयया श्रुत्या विनियुक्तस्यैव मन्त्रस्य पुनर्विनियोगाकाङ्क्षाया अनुदयाद्विनियोजकं तस्यैव लिङ्गं न प्राप्स्यति~। तस्माद्गार्हपत्योपस्थान एव मन्त्रः प्रत्यक्षश्रुत्यैव विनियुज्यत इति सिद्धम्~। एवमग्निचयने समाम्नाता {\qt निवेशनः संगमनः} इत्यादिकापि काचिदृगैन्द्री भवति~। तस्या उत्तरार्ध {\qt इन्द्रो न तस्थौ} इति पठनात्~। सापि {\qt निवेशनः संगमनो वसूनामित्यैन्द्र्या गार्हपत्यमुपतिष्ठते} इति ब्राह्मणे तृतीययैन्द्र्येति श्रुत्या गार्हपत्योपस्थाने विनियुज्यते; तस्या अपि पूर्वोक्तन्यायेन गार्हपत्यप्रकाशनसामर्थ्यस्य सत्त्वात्~। न च मन्त्रस्थेन्द्रपदस्य गुणवृत्त्या गार्हपत्येऽग्नौ वृतत्वेऽपि रूढ्या तु स्वर्गाधिपतौ सहस्राक्ष इन्द्रे वृतत्वान्मन्त्रेण प्रकाशिते मुख्येन्द्रे मन्त्रब्राह्मणयोर्विवादपरिहाराय ब्राह्मणस्थ {\qt गार्हपत्य} शब्देन निरुक्तगुणसंबन्धद्वारेणेन्द्रो लक्षणीय इति वाच्यम् ; {\qt इन्द्रगार्हपत्य} शब्दयोरन्यतरस्य गौणत्वेऽवश्यंभाविनि सति मन्त्रस्य प्राप्तार्थत्वेनानुवादकत्वसंभवान्मन्त्रस्थेन्द्रपदस्यैव गौणत्वस्य न्याय्यत्वात् ब्राह्मणवाक्यस्याप्राप्तार्थत्वेन विधायकत्वाद्विधौ लक्षणाया अन्याय्यत्वाच्च, तस्माद्गार्हपत्यप्रकाशने समर्थमेव मन्त्रमैन्द्र्येति तृतीयाश्रुतिर्गार्हपत्योपस्थाने विनियुङ्क्त इति
सिद्धम्~। एवं श्रुतेर्लिङ्गस्य दौर्बल्यवल्लिङ्गादेः पूर्वस्मात्पूर्वस्मात्परस्य परस्य वाक्यादेरेकविनियोज्यविषयत्वेन विरोधे दौर्बल्यं, स्वार्थावबोधने परस्य पूर्वव्यवधानेन प्रवृत्तेः~। तथा च
सूत्रं\textendash {\ab  श्रुतिलिङ्गवाक्यप्रकरणस्थानसमाख्यानां समवाये पारदौर्बल्यमर्थविप्रकर्षात्} इति~।\\

 एवं श्रुतिं निरूप्य लिङ्गं लक्षयति\textendash {\br शब्दसामर्थ्यमिति~।} लिङ्गमिति
\blfootnote{टिप्प० \textemdash\ $^{1}$एवकाररहितः पाठः समञ्जसो भाति~।}
\lfoot{४ अ०}
\newpage
%%%%%%%%%%%%%%%%%%%%%%%%%%%%55
\fancyhead[RE]{[ लिङ्ग\textemdash\ }
\lfoot{}
{\bl\noindent {\qtl लिङ्गमित्यभिधीयते} इति~। सामर्थ्यं रूढिरेव~। तेन समा\footnotemark ख्यातोऽस्याभेदः~। यौगिकशब्दसमाख्यातो रूढ्यात्मकलिङ्गशब्दस्य भिन्नत्वात्~। तेन {\qtl बर्हिर्देवसदनं दामि} इति मन्त्रस्य कुशलवनाङ्गत्वं न तूलपादिलवनाङ्गत्वम् ; तस्य बर्हिर्दामीति लिङ्गात्तल्लवनं प्रकाशयितुं समर्थत्वात्~। एवमन्यत्रापि लिङ्गाद्विनि-\\}
\hrule
\vspace{3mm}
\noindent
लक्ष्यनिर्देशः~। सामर्थ्यं लिङ्गमित्युक्तेऽङ्कुरादिजननानुकूलबीजादिसामर्थ्ये तत्प्रसङ्गः स्यात्तद्वारणाय - शब्देति विशेषणम्~। सामर्थ्यं द्विविधं {\qt शब्दगतमर्थगतं
चेति}~। तत्राद्यस्य लक्षणमत्रोक्तं, तत्रैव वृद्धसंमतिं प्रदर्शयंस्तदर्थमाह\textendash {\br यथाहुरित्यादिना~।} अन्त्यं तु यथा {\qt स्रुवेणावद्यति} इत्यवदानसामान्यशेषत्वावगमेऽपि स्रुवस्य लिङ्गात्सामर्थ्यरूपादाज्यसांनाय्यादिद्रवद्रव्यावदानविशेषाङ्गत्वं स्रुवेण मांसादिद्रव्यावदानस्य कर्तुमशक्यत्वादिति~। {\br ननु} रूढिरूपस्य सामर्थ्यस्य समाख्यातो न भेदः स्यात्~। तथा च {\qt बर्हिर्देवसदनं दामि} इति मन्त्रस्योलपादिलवनाङ्गत्वमपि स्यादित्याशङ्क्याह\textendash {\br समाख्यातो भेद इति~।} रूढिरूपस्य सामर्थ्यस्येति शेषः~। यौगिकशब्दरूपसमाख्यातो रूढात्मकस्य लिङ्गशब्दस्य भेद एव प्रसिद्धेः~। तथा च न तयोरभेद इत्यर्थः~। समाख्यातो नाभेद इति पाठेऽपि स एवार्थः~। रूढिरूपस्य सामर्थ्यस्य समाख्यातो भेदे फलितमाह\textendash {\br तेनेति~।} रूढिसमाख्ययोर्मिथोभेदेन पुरोडाशसदनभूतं बर्हिः खण्डयामीति श्रुतपदसामर्थ्यात् {\qt बर्हिर्देवसदनं दामि} इति मन्त्रस्य कुशकाशादिरूपाणां मुख्यबर्हिषां लवनाङ्गत्वं न तु समाख्यातो दर्भसदृशानामुलपादिरूपतृणविशेषाणां गौणबर्हिषां लवनाङ्गत्वमित्यर्थः~। उक्तेऽर्थे हेतुमाह\textendash\ {\br तस्येत्यादिना~।} तस्येति बर्हिर्देवेत्यादिमन्त्रस्येत्यर्थः~।~{\br तल्लवनमिति~।} कुशलवनमित्यर्थः~।
{\br एवमन्यत्रापीति~।} {\qt बर्हिर्देवसदनं दामि} इति मन्त्रस्य कुशलवनाङ्गत्ववदग्नये जुष्टं निर्वपामीत्यादिमन्त्रस्यापि निर्वापादिप्रकाशनसामर्थ्यरूपाल्लिङ्गान्निर्वापादौ विनियोग इत्यर्थः~। अत्रेदं बोध्यम् \textendash तस्य हि मन्त्रस्य निर्वापाङ्गत्वं संभवति यस्य यत्प्रकाशनसामर्थ्यं तस्य मन्त्रस्य तदङ्गत्वमिति न्यायात्~। तच्च लिङ्गं
द्विविधं, सामान्यसंबन्ध-
\blfootnote{पाठ०\textemdash\ $^{१}$ख्यातो नामाभेदः}
\newpage
%%%%%%%%%%%%%%%%%%%%%%%%%%%%%%%%%%%%%%%%%%%%%
\fancyhead[LO]{निर्वचनम् ]}
{\bl\noindent योगो द्रष्टव्यः~। तदिदं लिङ्गं वाक्यादिभ्यो बलवत्~। अत एव}\\
\hrule
\vspace{3mm}
\noindent
बोधकप्रमाणान्तरानपेक्षं तत्सापेक्षं चेति~। तत्र यद्विना यन्न संभवति तस्य तदङ्गत्वम्~। सामान्यसंबन्धबोधकप्रमाणान्तरानपेक्षकेवललिङ्गाद्भवति~। यथार्थज्ञानस्य कर्मानुष्ठानाङ्गत्वम् ; अर्थज्ञानं विना तदनुष्ठानासंभवात्~। यद्विना यत्संभवति तस्य तदङ्गत्वं तत्सापेक्षलिङ्गाद्भवति यथोक्तमन्त्रस्य निर्वापाङ्गत्वं, निर्वापस्य हि विनापि मन्त्रमुपायान्तरेण स्मृत्वा कर्तुं शक्यत्वान्न मन्त्रो निर्वापस्वरूपार्थः संभवति, किंत्वपूर्वसाधनीभूतनिर्वापप्रकाशनार्थ एव~। तादृशनिर्वापप्रकाशनार्थत्वं च न सामर्थ्यमात्राल्लभ्यते, सामर्थ्यस्य निर्वापप्रकाशनमात्रे सत्त्वात्~। ततश्चावश्यं  प्रकरणादिसामान्यसंबन्धबोधकं प्रमाणं स्वीकर्तव्यम् , तथा च दर्शपूर्णमासप्रकरणे तस्य मन्त्रस्य पाठादेवानेन मन्त्रेण
दर्शपूर्णमासापूर्वसंबन्धिकिंचित्प्रकाश्यत इत्यवगम्यते~। अन्यथा दर्शादिप्रकरणपाठस्य वैयर्थ्यापत्तेः~। किं तद्दर्शपूर्णमासापूर्वसंबन्धि प्रकाश्यमित्याकाङ्क्षायां निर्वापप्रकाशनसामर्थ्यात्पुरोडाशनिर्वाप इत्यवगम्यते~। निर्वापश्च पुरोडाशसंस्कारद्वाराऽपूर्वसंबन्धीति लिङ्गान्मन्त्रस्य निर्वापार्थत्वे सति न तस्यानर्थक्यापत्तिः~। तस्मात्
{\qt अग्नये जुष्टं निर्वपामि} इति मन्त्रस्य प्रकरणाद्दर्शपूर्णमाससंबन्धितयाऽवगतस्य निर्वापप्रकाशनसामर्थ्यरूपाल्लिङ्गान्निार्वापाङ्गत्वमिति~। तदिदं निरुक्तं\footnotemark\ सामर्थ्यरूपं लिङ्गं वाक्यप्रकरणादिभ्यः प्रबलमित्याह\textendash {\br तदिदमित्यादिना~।} {\br अत एवेति~।} लिङ्गस्य वाक्यादिभ्यो बलवत्त्वादेवेत्यर्थः~। {\qt स्योनं ते सदनं कृणोमि} इत्यत्र {\qt घृतस्य धारया सुशेवं कल्पयामि तस्मिन्सीदामृते प्रतितिष्ठ व्रीहीणां मेध सुमनस्यमानः} इति वाक्यशेषः~। भोः पुरोडाश ! ते तव स्योनं समीचीनं सदनं स्थानं कृणोमि करोमि, तच्च स्थानं घृतस्य धारया सुशेवं सुष्ठु सेवितुं योग्यं कल्पयामि~। भो व्रीहीणां मेध व्रीहिसारभूत ! त्वं सुमनस्यमानः समाहितमनस्कः सन् तस्मिन्नमृते समीचीने स्थाने सीद उपविश प्रतितिष्ठ; तत्र स्थिरो भवेत्यर्थः~। तत्र च तस्मिन्नित्यनेन तच्छब्देन प्रकृतवाचकेन पूर्वोत्तरार्धयोरेकवाक्यत्वे सिद्धे  मन्त्रद्वयस्याभावात्सर्वोऽप्ययं
मन्त्रः स्थानकरणस्याङ्गं पुरोडाशस्थापनस्य चाङ्गं भवति, तत्र विनियोजिका श्रुतिश्चैवं कल्पनीया सर्वेणानेन मन्त्रेण स्थानं कर्तव्यमिति~। तथा सर्वेणानेन मन्त्रेण
\blfootnote{टिप्प०\textemdash\ $^{1}$निरनुस्वारः पाठः सम्यगिति प्रतीयते~।}
\newpage
%%%%%%%%%%%%%%%%%%%%%%%%%%%%%%%%%%%%%%%%%%%
\fancyhead[RE]{[ वाक्यनिर्वचनम् ]}
{\bl\noindent {\qtl स्योनं ते सदनं कृणोमि} इति मत्रस्य पुरोडाशसदनकरणाङ्गत्वं सदनं कृणोमीति लिङ्गात्, न तु वाक्यात्~।}
\begin{center}
 \textbf{वाक्यनिर्वचनम्}   
\end{center}
 
{\bl समभिव्याहारो वाक्यम्~। समभिव्याहारश्च साध्यत्वादिवाचकद्वितीयाद्यभावेऽपि~। वस्तुतः शेषशेषिवाचकपदयोः सहोच्चारणम्~। यथा {\qtl यस्य पर्णमयी जुहूर्भवति न स पाप्ँ श्लोकं शृणोति}~। अत्र पर्णताजुह्वोः समभिव्याहारादेव पर्णताया जुह्वङ्गत्वम्~। न चानर्थक्यम् , अन्यथापि जुह्वाः सिद्धत्वादिति वाच्यम् ; {\qtl जुहू} शब्देन तत्साध्यापूर्वलक्षणात्~। तथा च }\\
\hrule
\vspace{3mm}
\noindent
तत्र पुरोडाशः स्थापनीय इति च~। तथा च सदनकरणपुरोडाशस्थापनयोरस्य मन्त्रस्य विकल्पः समुच्चयो वा स्वेच्छया भविष्यतीति पूर्वपक्षः~। तत्र हि \textendash यदेतत्पूर्वोत्तरार्धयोः परस्परान्वयेनैकवाक्यं संपन्नं तदेतदुत्तरार्धस्य सदनकरणे शक्तिमकल्पयित्वा सकलं मन्त्रं सदने विनियोक्तुं नार्हति~। तथा तदेव वाक्यं पूर्वार्धस्य पुरोडाशस्थापने शक्तिमकल्पयित्वा न पुरोडाशस्थापने कृत्स्नं मन्त्रं विनियोक्तुं प्रभवति~। ततश्च लिङ्गकल्पनव्यवधानेन वाक्यं श्रुतिं प्रति विप्रकृष्यते~। प्रत्यक्षं तु
लिङ्गद्वयं तां श्रुतिं प्रति न विप्रकृष्यते~। तथा सति लिङ्गेन वाक्यस्य बाधान्मन्त्रस्यार्धद्वयं सदनकरणस्थापनयोर्व्यवस्थितमिति राद्धान्तः~। तमेतमभिप्रेत्याह\textendash {\br मन्त्रस्येत्यादिना~। लिङ्गादिति~।} सदनकरणप्रकाशनसामर्थ्यरूपादित्यर्थः~। {\br न तु वाक्यादिति~।} मन्त्रस्य पुरोडाशसदनसादनरूपार्थकरणाङ्गत्वं, न तु
सदनसादनलक्षणशेषशेषिरूपार्थप्रतिपादकात्सदनं कृणोमि तस्मिन्सीदेत्येवं सहोच्चारणरूपाद्वाक्यादित्यर्थः~।\\


 वाक्यं लक्षयति\textendash {\br समभिव्याहार इति~।} वाक्यमिति लक्ष्यनिर्देशः~। {\qt समभिव्याहार}शब्दं व्याचष्टे- {\br समभिव्याहारश्चेत्यादिना~।} तत्रोदाहरणमाह \textendash  {\br यथेत्यादिना~। पर्णताजुह्वोरिति~।} शेषशेषिलक्षणयोरिति शेषः~। {\br ननु} काष्ठान्तरेणैव जुह्वाः सिद्धत्वात्पर्णताया आनर्थक्यं स्यादित्याशङ्क्य परिहरति {\br न चेत्यादिना~।} न च वाच्यमित्यत्र हेतुमाह\textendash\ {\br जुहूशब्देनेत्यादिना~।} यदि
\newpage
%%%%%%%%%%%%%%%%%%%%%%%%%%%%%%%%%
\fancyhead[LO]{[प्रकृतिविकृतिलक्षणम् [}
{\bl\noindent वाक्यार्थः पर्णतयावत्तहविर्धारणद्वारा जुह्वपूर्वं भावयेदिति~। एवं च पर्णतया यदि जुहूः क्रियते तदैव तत्साध्यमपूर्वं भवति, नान्यथेति गम्यत इति न पर्णताया वैयर्थ्यम्~। अवत्तहविर्धारणद्वारेति चावश्यं वक्तव्यम् ; अन्यथा स्रुवादिष्वपि पर्णतापत्तेः, सेयं पर्णता अनारभ्याधीतापि सर्वप्रकृतिष्वेवान्वेति न विकृतिषु~। तत्र चोदकेनापि
तत्प्राप्तिसंभवात्पौनरुक्त्यापत्तेः~।}
\begin{center}
 \textbf{\textbf{प्रकृतिविकृतिलक्षणम् }}   
\end{center}
 
{\bl यत्र समग्राङ्गोपदेशः सा प्रकृतिः, यथा\textemdash\ दर्शपूर्णमासादिः, तत्प्रकरणे सर्वाङ्गपाठात्~। यत्र न सर्वाङ्गोपदेशः सा विकृतिः~। यथा- सौर्यादिः~। तत्र कतिपयाङ्गानामतिदेशेन प्राप्तत्वात्~।}\\
\hrule
\vspace{3mm}
\noindent
जुहूः पर्णतयैव क्रियते तदैव जुह्वाऽपूर्वं जन्यते नान्यथेति भावः~। फलितं वाक्यार्थमाह\textendash {\br तथा चेत्यादिना~।} क्वचित्पुस्तकेऽवत्तेत्यतः प्रागितितः परत्रैवं च पर्णतया यदि जुहूः क्रियते तदैव तत्साध्यमपूर्वं भवति नान्यथेति गम्यत इति न पर्णताया वैयर्थ्यमिति पाठः~। अवद्यते इत्यवत्तं अवत्तं च तद्धविश्च तस्य धारणद्वारेत्यर्थः~।
अवत्तेत्यादेर्व्यावृत्तिमाह\textendash {\br अन्यथेत्यादिना~।} अन्यथाऽवत्तेत्यादेरनुपादान इत्यर्थः~। स्रुक्स्रुवादिभिराज्यहविर्धारणद्वारैवोपक्रियत इति भावः~।~{\br ननु} पर्णताया अनारभ्याधीतत्वादनारभ्यविधेश्च सामान्यविधित्वात्पर्णता सर्वेषु प्रकृतिविकृतियागेषु समन्वेत्वित्याशङ्क्याह\textendash {\br सेयमित्यादिना~।} एवकारव्यावर्त्यमाह \textendash {\br न विकृतिष्विति~। तत्रेति~।} विकृतिष्वित्यर्थः~। {\br चोदकेनेति~।} प्रकृतिवद्विकृतिः कर्तव्येति चोदकशब्दितातिदेशवाक्येनेत्यर्थः~। {\br तत्प्राप्तिसंभवादिति~।} पर्णताप्राप्तिसंभवादित्यर्थः {\br पौनरुक्त्यापत्तेरिति~।} {\qt प्रकृतौ वाऽद्विरुक्तत्वात्} इति न्यायविरोधेन पौनरुक्त्यरूपदोषापत्तेरित्यर्थः~।\\

 केयं प्रकृतिर्विकृतिश्चेत्याकाङ्क्षायां तल्लक्षणमाह\textendash {\br यत्रेत्यादिना~।  अतिदेशेनेति~।} उक्तचोदकशब्दितातिदेशवाक्येनेत्यर्थः~। किंच, यत्र
चोदकेनाङ्गाप्राप्तिस्तत्रानारभ्याधीतानां संनिवेशः, दर्शपूर्णमासयोरपि चोदकप्रमाणेनाङ्गानामप्राप्तेः प्रकरणपठितैरेवाङ्गैर्नैराकाङ्क्ष्यात् तत्र पर्णतायाः संनिवेशो न्याय्य एव~। गृह-
\newpage
%%%%%%%%%%%%%%%%%%%%%%%%%%%%%%%%%
\fancyhead[RE]{[प्रकृतिविकृतिलक्षणम् ]}
{\bl\noindent अनारभ्यविधिः सामान्यविधिः~। तदिदं वाक्यं प्रकरणादिभ्यो बलवत्~। अत एव {\qtl इन्द्राग्नी इदं हविः} इत्यादेरेकवाक्यत्वाद्दर्शाङ्गत्वं, न तु प्रकरणाद्दर्शपूर्णमासाङ्गत्वम्~।}\\
\hrule
\vspace{3mm}
\noindent
मेधीययागाद्यद्यपि कुत्रापि विकृतौ नाङ्गानामतिदेशेन प्राप्तिस्तथापि तस्य क्लृप्तोपकारैराज्यभागादिभिरेव नैराकाङ्क्षयेण तत्रापि चोदकादङ्गाप्राप्तेः सत्त्वात्तत्रापि पर्णतासंनिवेशो भवत्येव~। किंच योऽनारभ्यविधिः स सामान्यविधिरित्युच्यते~। सामान्यविधेश्चास्पष्टत्वात्तस्य विशेषेणोपसंहारो भवति~। तथा चोक्तं \textendash  {\qt सामान्यविधिरस्पष्टः संह्रियेत विशेषतः} इति~। तथा च सामिधेन्यृचां साप्तदश्यस्यानारभ्याधीतत्वेऽपि न प्रकृतिषु गमनं तस्य प्रकृतियागानां सामिधेन्यृक्पाञ्चदश्यावरोधात्~। नापि विकृतिमात्रे तद्गमनं तत्र चोदकप्राप्तपाञ्चदश्यबाधप्रसङ्गात् , किंतु प्रत्यक्षश्रुतसाप्तदश्यासु मित्रविन्दादिष्वेव विकृतिषु तस्य साप्तदश्यस्य गमनं भवति~। तथा चोक्तम \textendash  {\qt एवं च प्रकृतावेतत्पाञ्चदश्यं प्रतिष्ठितम्~। विकृतौ च न यत्रास्ति साप्तदश्यपुनःश्रुतिः} इति~। न च वाक्यवैयर्थ्यम् ; अनारभ्याधीतस्यैव साप्तदश्यस्य मित्रविन्दादिप्रकरणस्थवाक्येनोपसंहारात्~। उपसंहारश्च नाम सामान्यप्राप्तस्य विशेषे नियमनम्~। ततश्च साप्तदश्यस्यानारभ्यविधिः सामान्यविधिः~। मित्रविन्दादिप्रकरणस्थश्च विशेषविधिरिति सर्वमभिप्रेत्याह\textendash\ {\br अनारभ्यविधिः सामान्यविधिरिति~।} किंच तद्धि वाक्यं प्रकरणाद्बलीयो भवति, स्थानादितस्तु सुतराम्~। तस्य प्रकरणादपि दुर्बलत्वात् वाक्यस्य प्रकरणादिभ्यो बलवत्त्वादेव {\qt इन्द्राग्नी इदं हविः} इत्यादिमन्त्रस्य दर्शवाक्यत्वादर्शाङ्गत्वं, न तु दर्शपूर्णमासप्रकरणाद्दर्शपूर्णमासोभयाङ्गत्वमित्याह\textendash {\br तदिदमित्यादिना~।} \noindent अत्र \textendash चेदं बोध्यम् {\qt अग्नीषोमाविदं हविरजुषेतामवीवृधेतां महोज्यायोऽक्राताम् , इन्द्राग्नी इदं हविरजुषेतामवीवृधेतां महोज्यायोऽक्राताम्} इति सूक्तवाके श्रूयते~। तत्र च देवतावाचकं पदमग्नीषोमादिरूपं पौर्णमास्यादिकाले यथादेवतं विभज्य प्रयोक्तव्यमिति {\br तृतीये} स्थितम्~। इदं हविरित्यादिपदमवशिष्टं तु यथोक्ताग्नीषोमेन्द्राग्निमन्त्रद्वयगतमपि यथाक्रमममावास्यायामग्नीषोमपदपरित्यागेन पौर्णमास्यामिन्द्राग्निपदपरित्यागेन च पठनीयम्~। तथा च सति तेषां मन्त्रभागानां सर्वशेषत्वबोधको दर्शपूर्णमासप्रकरणपाठोऽनुगृहीतो भवतीति प्राप्तेऽभिधीयते- अग्नीषोममन्त्रशेषस्येदं हविरित्यादिरूपस्येन्द्राग्निपदान्वयाश्रवणात्
\newpage
%%%%%%%%%%%%%%%%%%%%%%%%%%%%%%%%%%
\fancyhead[LO]{[महाप्रकरणम्]}
 \begin{center}
\textbf{प्रकरणनिरूपणम् }     
 \end{center}

{\bl उभयाकाङ्क्षा प्रकरणम्~। यथा प्रयाजादिषु {\qtl समिधो यजति} इत्यादिवाक्ये फलविशेषस्यानिर्देशात्समिद्यागेन भावयेदिति बोधानन्तरं किमित्युपकार्याकाङ्क्षा~। दर्शपूर्णमासवाक्येऽपि {\qtl दर्शपूर्णमासाभ्यां स्वर्गं भावयेत्} इति बोधानन्तरं कथमित्युपकारकाकाङ्क्षा ~। इत्थं चोभयाकाङ्क्षया  प्रयाजादीनां दर्शपूर्णमासाङ्गत्वम्~।}
\begin{center}
\textbf{प्रकरणद्वैविध्यम्}    
\end{center}

{\bl तच्च प्रकरणं द्विविधम्\textendash {\al महाप्रकरणमवान्तरप्रकरणं चेति~}।}
\begin{center}
 \textbf{महाप्रकरणम्}   
\end{center}
 
{\bl {\al तत्र मुख्यभावनासंबन्धिप्रकरणं महाप्रकरणम्~}। तेन च}\\
\hrule
\vspace{3mm}
\noindent
प्रकरणेन प्रथमं तदन्वयरूपं वाक्यं कल्पनीयं, तेन च वाक्येनेन्द्राग्निप्रकाशनसामर्थ्यरूपं लिङ्गं कल्पनीयम्~। तच्च लिङ्गमनेन मन्त्रभागेनेन्द्राग्निविषया काचित्क्रियानुष्ठेयेति विनियोजिकां तृतीयां श्रुतिं कल्पयति~। ततः प्रकरणविनियोगयोर्मध्ये त्रिभिर्व्यवधानं भवति~। अग्नीषोमपदान्वयरूपं वाक्यं तु श्रूयमाणत्वाल्लिङ्गश्रुतिभ्यामेव व्यवधीयते
~। एवमिन्द्राग्निमन्त्रशेषस्याप्यग्नीषोमपदान्वयाश्रवणात्प्रकरणेन प्रथमं तदन्वयरूपं वाक्यं कल्पनीयमित्यादि स्वयमूह्यम्~। तस्माद्वाक्येन स्वस्माद्दुर्बलस्य प्रकरणस्य बाधितत्वात्तत्तन्मन्त्रशेषस्तत्र तत्रैव व्यवतिष्ठत इति~।\\

 प्रकरणं लक्षयति\textendash {\br उभयेत्यादिना~।} आकाङ्क्षात्वं चेल्लक्षणं शाब्दबोधकारणीभूताकाङ्क्षायामतिप्रसङ्गस्तद्वारणायउभयेति विशेषणम्~। उभयत्वं चेत्तदोभयत्वावच्छिन्ने घटपटादावतिव्याप्तिस्तद्वारणायोत्तरं दलम्, परस्परमुभयाकाङ्क्षेत्यर्थः~। प्रकरणमिति लक्ष्यनिर्देशः~। तत्रोदाहरणद्वारोभयाकाङ्क्षां प्रदर्शयन् लक्षणसमन्वयं करोति- {\br यथेत्यादिना~।~इत्थं चेति~।} अनेन प्रकारेण चेत्यर्थः~।\\

 प्रकरणं विभजते {\br तच्च प्रकरणमित्यादिना~। तच्चेति~।}  उक्तलक्षणलक्षितं चेत्यर्थः~।\\

 {\br तत्रेति~।} महाप्रकरणावान्तरप्रकरणयोर्मध्य इत्यर्थः~। {\br मुख्यभावनेति~।}
\newpage
%%%%%%%%%%%%%%%%%%%%%%%%%%%%%%%%%%%
\fancyhead[RE]{[ महाप्रकरणम् ]}
{\bl\noindent प्रयाजादीनां दर्शपूर्णमासाङ्गत्वम्~। एतच्च प्रकृतावेव उभयाकाङ्क्षायाः संभवात् , न तु विकृतौ~। तत्र {\qtl प्रकृतिवद्विकृतिः कर्तव्या} इत्यतिदेशेन कथंभावाकाङ्क्षाया उपशमेनापूर्वाङ्गानामप्युभयाकाङ्क्षया विनियोगासंभवात्~। तस्मादपूर्वाङ्गानां स्थानादेव विकृत्यर्थत्वमिति~।}\\
\hrule
\vspace{3mm}
\noindent
फलभावनेत्यर्थः~। {\br तेन चेति~।} महाप्रकरणेन चेत्यर्थः अङ्गत्वम्~। बोध्यत इति शेषः~। तद्धि महाप्रकरणं प्रयाजादीन्यङ्गानि दर्शपूर्णमासयोर्विनियुङ्क्त इत्यर्थः~। {\br एतच्च}~। महाप्रकरणं चेत्यर्थः~। तत्र हेतुमाह\textendash {\br उभयेति~।} प्रकृतावेवेत्यनुषङ्गः~। एवकारव्यावर्त्यमाह\textendash {\br न त्विति~।} विकृतावुभयाकाङ्क्षाया असंभवादिति भावः~। {\br तत्रेति~।} विकृतावित्यर्थः~। {\br अतिदेशेनेति}~। अतिदेशवाक्यप्राप्तप्राकृताङ्गजातेनेत्यर्थः~। {\bl कथंभावाकाङ्क्षायाः~।} इतिकर्तव्यताकाङ्क्षाया इत्यर्थः~। तथा च यद्यप्युपहोमादीनामपूर्वाङ्गानामुपहोमादिभिर्भावयेत्किं भावयेदित्यस्त्याकाङ्क्षा तथापि सौर्यादिविकृतियागस्यातिदेशवाक्यप्राप्तप्राकृताङ्गैरेव नैराकाङ्क्षयेण नापूर्वाङ्गानामप्युपहोमादीनामुभयाकाङ्क्षया विकृतौ विनियोगः संभवतीति भावः~। {\br ननु} प्राकृताङ्गानां विकृतावपठितत्वादप्रत्यक्षाणां कथं विकृत्याकाङ्क्षोपशमहेतुत्वं वैकृताङ्गानां तूपहोमादीनामत्र पठितत्वेन प्रत्यक्षाणां संभवत्याकाङ्क्षोपशान्तिहेतुत्वमिति चेत् , न; तेषां विकृतौ
पठितत्वेन प्रत्यक्षत्वेऽप्यन्यत्राक्लृप्तोपकारकत्वाच्छीघ्रं विकृत्याकाङ्क्षोपशान्तावसामर्थ्यात्~। अतिदिष्टानां तु प्रकृतौ क्लृप्तोपकारकत्वेन संभवति तदाकाङ्क्षोपशमनसामर्थ्यम्~। न च तेषामेव प्राकृताङ्गानां विकृतौ प्रकरणाद्ग्रहणं स्यादिति वाच्यम्~; तेषामपि प्रकृतावुपकारकत्वेनोपक्षीणाकाङ्क्षत्वात्~। न च प्राकृताङ्गानामत्रोपस्थापकाभावेनानुपस्थितत्वमिति वाच्यम् ; उपमानप्रमाणस्योपस्थापकस्य सत्त्वेन तेषामत्रोपस्थितत्वात्~। तथा हि\textendash सौर्यवाक्यस्य दर्शने ह्यौषधद्रव्यत्वस्यैकदेवत्यत्वस्य च सादृश्यस्य दर्शनेनानेन सदृशमाग्नेयवाक्यमित्याग्नेयवाक्यमुपमीयते गवयदर्शनादनेन सदृशी मदीया गौरिति गोरुपमानवत्~। आग्नेयवाक्ये चोपमिते तेन तदर्थो ज्ञायते~। स च त्र्यंशभावनारूपस्तस्मिंश्च ज्ञाते सौर्यवाक्ये भावनाया भाव्यस्य करणस्य च विद्यमानत्वेन तत्राकाङ्क्षाभावेऽपीतिकर्तव्यताकाङ्क्षायां प्राकृतोपकारपृष्ठभावेनाग्नेयेतिकर्तव्यतामतिदिश्य सौर्ययागेन ब्रह्मवर्चसं
\newpage
%%%%%%%%%%%%%%%%%%%%%%%%%%%%%%%%%%%
\fancyhead[LO]{[ संदंशलक्षणम् ]}
\begin{center}
 \textbf{अवान्तरप्रकरणम्}
\end{center}
 
{\bl अङ्गभावनासंबन्धिप्रकरणमवान्तरप्रकरणम्~। तेन चाभिक्रमणादीनां प्रयाजाद्यङ्गत्वम्~। तच्च संदंशेनैव ज्ञायते~। तदभावे चाविशेषात्सर्वेषां फलभावनाकथंभावेन ग्रहणप्रसङ्गेन प्रधानाङ्गत्वापत्तेः~।}
\begin{center}
 \textbf{संदंशलक्षणम्}   
\end{center}
 
{\bl एकाङ्गानुवादेन विधीयमानयोरङ्गयोरन्तराले विहितत्वं संदंशः~। यथा- अभिक्रमणे~।~तद् हि {\qtl समानयते जुह्वाम् उपभृत-\\ }}
\hrule
\vspace{3mm}
\noindent
भावयेदाग्नेयवदिति सिद्ध्यति~। तस्मादाग्नेयेतिकर्तव्यतयैव विकृत्याकाङ्क्षोपशमने सौर्यादौ विकृतावुभयाकाङ्क्षारूपप्रकरणाभावात्स्थानादेवोपहोमादीनामपूर्वाङ्गानां
विकृतिशेषत्वमित्युपसंहरति \textendash {\br तस्मादिति~।}\\

 अवान्तरप्रकरणं लक्षयति\textendash {\br अङ्गभावेनेति~। तेन चेति~।} अवान्तरप्रकरणेन चेत्यर्थः~। अत्रापि बोध्यत इति शेषः~। {\br तच्चेति~।}  तेन तेषां तदङ्गत्वं
चेत्यर्थः~। {\br संदंशेनेति~।} संदंशो लोहकण्टकविद्धलोहशलाकाद्वयरूपस्तेनेत्यर्थः~। तन्न्यायेनेति यावत्~। तदनङ्गीकारे दोषमाह\textendash {\br तदभाव इति~।} संदंशाभाव इत्यर्थः~।~{\br अविशेषात्} प्रकरणाविशेषात्~। {\br सर्वेषां} प्रयाजादीनामभिक्रमणादीनां च {\br कथंभावेन} इतिकर्तव्यतारूपेण~। {\br प्रधानाङ्गत्वापत्तेः}
दर्शादिप्रधानाङ्गत्वापत्तेः~। तथा चाभिक्रमणादीनां प्रयाजाद्यङ्गत्वग्रहे संदंशस्याभावे सति दर्शादिप्रधानयागप्रकरणपाठाविशेषात्प्रयाजादिवदभिक्रमणादीनामपि दर्शादिफलभावनाया इतिकर्तव्यताकाङ्क्षया ग्रहणप्रसङ्गेन दर्शादिप्रधानयागाङ्गत्वापत्तेस्तत्संदंशेनैव ज्ञातुं शक्यत इति समुदायार्थः~। \\


 संदंश लक्षयति\textendash {\br एकाङ्गानुवादेनेति~।} तत्रोदाहरणमाह\textendash {\br यथाभिक्रमण इति~।} अभिक्रमणे ह्येकस्य प्रयाजरूपस्याङ्गस्यानुवादेन विधीयमानयोः प्रयाजाङ्गयोरन्तराले विहितत्वं भवत्येव~। तदेवाह\textendash {\br तद्धीत्यादिना~।} तद्धि, अभिक्रमणं हीत्यर्थः~। {\br समानयते जुह्वामुपभृत इति~।} उपभृतो घृतपात्र-
\newpage
%%%%%%%%%%%%%%%%%%%%%%%%%%%%%%%%%%
\fancyhead[RE]{[कथंभावाकाङ्क्षा]}
{\bl \noindent  स्तेजो वा इत्यादिना
प्रयाजानुवादेन किंचिदङ्गं विधाय\blfootnote{पाठा०\textemdash\ $^{१}$विधाय तदनन्तरमपि.}\footnotemark विधीयते {\qtl यस्यैवंविदुषः प्रयाजा
इज्यन्ते प्रैभ्यो लोकेभ्यो भ्रातृव्यान्तुदतेऽभिकामञ्जुहोत्यभिजित्यै}
इति, तदन\footnote{तदनन्तरमपि.}न्तरं {\qtl यो वै प्रयाजानां मिथुनं वेद} इत्यादिना किंचिदङ्गं विधीयते
~। अतः प्रयाजाङ्गमध्ये विहितमभिक्रमणं तदङ्गम्~।}
\begin{center}
 \textbf{कथंभावाकाङ्क्षा}   
\end{center}
 
{\br प्रयाजैरपूर्वं कृत्वा यागोपकारं भावयेदिति ज्ञाते कथमेभिरपूर्वं कर्तव्यमिति कथंभावाकाङ्क्षायाः सत्त्वात्~। सा च संदंशपठितैरभिक्रमणादिभिःशाम्यति~। न चाङ्गभावनायाः कथं-\\}
\hrule
\vspace{3mm}
\noindent
विशेषाज्जुह्वां जुहूरूपपात्रविशेषे घृतं समानयत इति मन्त्रार्थः~। {\br किंचिदङ्गमिति~।} उपभृतः पात्राज्जुह्वां प्रयाजार्थं घृतानयनरूपमङ्गमित्यर्थः~। {\br तदनन्तरमपीति~।} अभिक्रमणानन्तरमपीत्यर्थः~। तदनन्तरमपीत्यस्य {\qt किंचिदङ्गं विधीयत} इत्युत्तरेणान्वयः~। {\qt प्रयाजाङ्गम्} इति पाठे तु तस्यापि किंचिदङ्गमित्यनेनोत्तरेणैवान्वयः~। {\br भ्रातृव्यानिति~।} शत्रूनित्यर्थः~। {\br तुदत इति~।} व्यथयति अपतुदतीति वार्थः~। जयतीति यावत्~। {\br अभिक्रामञ्जुहोत्यभिजित्या इति~।} विजयायाहवनीयं सर्वतः संचरणं कृत्वा जुहुयादित्यर्थः~। {\br मिथुनं वेदेति~।} युगलं जानातीत्यर्थः~। {\br अत इति~।} प्रयाजानुवादेन विहिततदङ्गमध्ये विहितत्वादित्यर्थः~।\\

 {\br ननु} प्रयाजानामितिकर्तव्यताकाङ्क्षाभावान्न तत्र संदंशेनाप्यभिक्रमणस्याङ्गत्वेनान्वयः । {\qt साकाङ्क्षस्यैव गुणेऽन्वेषणा} इति न्यायादित्यत आह\textendash {\br प्रयाजैरित्यादिना~।} सा च प्रयाजानामितिकर्तव्यताकाङ्क्षा संदंशलब्धैरेवाभिक्रमणादिभिः शाम्यतीत्याह\textendash {\br सा चेत्यादिना~।} तथा चोक्तम् {\qt परप्रकरणस्थानामङ्गे श्रुत्यादिभिस्त्रिभिः~। ज्ञाते पुनश्च तैरेव संदंशेन तदिष्यते} इति~। {\br ननु} प्रयाजभावनाया अङ्गभावनात्वेन कथंभावाकाङ्क्षाभावान्न प्रयाजभावनाकथंभावेनाभिक्रमणं गृह्यत
\newpage
%%%%%%%%%%%%%%%%%%%%%%%%%%%%%%%%%%%%%%%%%%%%%%%%%%
\fancyhead[LO]{[कथंभावाकाङ्क्षा]}
{\bl\noindent
भावाकाङ्क्षाऽभावः, भावनासामान्येन\footnotemark\ तत्रापि तत्संभवात्~। तदिदं प्रकरणं क्रियाया एव साक्षाद्विनियोजकं द्रव्यगुणयोस्तु तद्द्वारा~। तथा हि\textendash {\qtl यजेत स्वर्गकामः} इत्यत्र फलभावनायां कथंभावाकाङ्क्षायां संनिधिपठिताऽश्रूयमाणफलकं क्रियाजातमुपकार्याकाङ्क्ष्येतिकर्तव्यतात्वेनान्वेति~। क्रियाया एव लोके
कथंभावाकाङ्क्षायामन्वयदर्शनात्~। न हि कुठारेण छिन्द्यादित्यत्र कथं-\\}
\hrule
\vspace{3mm}
\noindent
 इत्याशय परिहरति {\br न चेत्यादिना~।} प्रयाजादिभावना कथंभावाकाङ्क्षाशून्या अङ्गभावनात्वादित्यत्र हेतोरसाधारणानैकान्तिकत्वात्साकाङ्क्षत्वसाधने हेतुसत्त्वाच्च न
प्रयाजाद्यङ्गभावनायाः कथंभावाकाङ्क्षाशून्यत्वमित्याह\textendash {\br भावनासाम्येनेति~।} प्रयाजाद्यङ्गभावना कथंभावसाकाङ्क्षा भावनात्वाद्दर्शादिभावनावदिति
प्रयोगोऽत्र बोध्यः~। नचावहननादिभावनायां व्यभिचारः; तस्याः पक्षसमत्वात्~। {\br तत्रापि तत्संभवादिति~।} प्रयाजाद्यङ्गभावनायामपि कथंभावाकाङ्क्षासंभवादित्यर्थः~। निरूपितप्रकरणस्य साक्षाद्विनियोज्यविषयमुपन्यस्यति {\br तदिदमित्यादिना~।} प्रकृतस्यैव व्यवहितविनियोज्यमादर्शयति {\br द्रव्येति~। तद्य्वारेति~।} क्रियाद्वारेत्यर्थः
~। अत्र च द्रव्यस्य क्रियाद्वारैव प्रकरणं विनियोजकं, गुणस्य तु द्रव्यक्रियोभयद्वारा विनियोजकमिति भावः~।~{\br फलभावनायामिति~।} फलभावनायां क्रियाजातमितिकर्तव्यतात्वेनान्वेतीति संबन्धः~। भिन्नप्रकरणस्थक्रियाया अन्यत्रक्रियायामन्वयापत्तिवारणाय संनिधिपठितमित्युक्तम्~। श्रूयमाणफलस्य
क्रियाजातस्येतिकर्तव्यतात्ववारणायाश्रूयमाणफलमित्युक्तम्~। प्रधानस्योपकारकाङ्क्षाभावे तत्र तदन्वयादर्शनात्कथंभावाकाङ्क्षायामित्युक्तम्~।
गुणस्याप्युपकार्याकाङ्क्षाभावेऽप्युपकारकत्वेनान्वयादर्शनादुपकार्याकाङ्क्षयेत्युक्तम्~। इतिकर्तव्यताकाङ्क्षायां क्रियाया एव साक्षादन्वये लोकप्रसिद्धिं हेतुत्वेन दर्शयति {\br क्रियाया एवेत्यादिना~।} क्रियाया एवान्वयदर्शनादित्यन्वय इत्यर्थः~। कुठारेण छिन्द्यादित्यत्र कथमिति कथंभावाकाङ्क्षायां हस्तेनेत्युच्चार्यमाणोऽपि हस्तो न हि च्छिदिभावनायां
साक्षादन्वेतीत्यभिप्रेत्याह\textendash {\br न}
\blfootnote{पाठा०\textemdash\ $^{१}$साम्येन.}
\newpage
%%%%%%%%%%%%%%%%%%%%%%%%%%%%%%%%%%%%
\fancyhead[RE]{[कथंभावाकाङ्क्षा ]}
{\bl\noindent भावाकाङ्क्षायामुच्चार्यमाणोऽपि हस्तोऽन्वेति किंतु हस्तेनोद्यम्य निपात्येति उद्यमननिपातने एव, हस्तश्च तद्वारैवान्वेतीति सार्वजनीनमेतत्~। इदं च स्थानादिभ्यो बलवत्~। अत एवाक्षैर्दीव्यति }\\
\hrule
\vspace{3mm}
\noindent
{\br हीत्यादिना~।} उद्यमननिपातने एवान्वित इति शेषः~। उद्यमनं च निपातनं चोद्यमननिपातने इति द्विवचनं~। {\br सन्धिस्तु लेखकप्रमादतः (?)~।} यद्वा, {\qt उद्यमनेन निपातन उद्यमननिपातन इति व्याख्येयम्~}।~\noindent अत्र \textendash  चान्वेतीत्यनुषङ्गः~। {\br तद्द्वारैवेति~।} उद्यमननिपातनद्वारैवेत्यर्थः~। {\br सार्वजनीनमिति~।} सर्वेषु जनेषु भवमित्यर्थः~। सर्वजनप्रसिद्धमिति यावत्~। किंच कथंभावाकाङ्क्षा नाम करणगतप्रकाराकाङ्क्षा थमुनः प्रकारवाचित्वात् , सामान्यस्य भेदको विशेषः प्रकारः, सामान्यं च क्रियारूपमेवाख्यातेनोच्यते~। {\qt यजेत स्वर्गकामः} इत्यस्य ह्ययमर्थः\textendash यागेन तथा कर्तव्यं यथा स्वर्गो भवतीति, क्रियासामान्यस्य च विशेषः क्रियैव भवति, न हि ब्राह्मणविशेषः परिव्राजकादिरब्राह्मणो भवति, एवं च करणगतक्रियाविशेषाकाङ्क्षापरनामधेयायां कथंभावाकाङ्क्षायां क्रियैवान्वेतीति युक्तम्~। स च करणगतः क्रियाविशेषोऽन्वाधानादिब्राह्मणतर्पणान्तः क्रियारूप एवेति युक्तं तस्य प्रकरणेनैव ग्रहणं तस्य च करणगतत्वं तदुपकारकत्वमेव तेन विना यागेनापूर्वाजननात्~। न ह्युद्यमननिपातनव्यतिरेकेण कुठारेण द्वैधीभावो जन्यते~। तत्सिद्धं कथंभावाकाङ्क्षया क्रियैवान्वेतीति~। तदिदं प्रकरणं स्थानादितो बलीयो भवति~। तथा हि\textendash राजसूयप्रकरणे पश्विष्टिसोमभागा बहवः समप्रधानभूताः पठ्यन्ते~। तत्र च कश्चिदभिषेचनीयसंज्ञकः सोमयागः पठितः~। तस्य हि संनिधौ विदेवनादयो धर्माः {\qt अक्षैर्दीव्यति राजन्यं जिनाति शौनःशेपमाख्यापयती} ति श्रूयन्ते~। जिनाति जयतीत्यर्थः~। बह्वृचब्राह्मणे समाम्नातं शुनःशेपविषयमुपाख्यानं शौनःशेपं तदाख्यापयत्तीत्यर्थः~। तत्र च
विदेवनादीनां संनिधिबलादभिषेचनीयाङ्गत्वमिति प्राप्ते राद्धान्तः, राजसूयेतिकर्तव्यताकाङ्क्षायामनुवृत्तायां विहिता विदेवनादयः प्रकरणेन राजसूयशेषा एव भवन्ति~। राजसूयश्च बहुयागात्मको भवति~। ततश्च तत्रत्यसर्वयागशेषत्वं विदेवनादीनां सिद्ध्यति~। किंचाभिषेचनीयस्य काचिदप्याकाङ्क्षा विदेवनादिषु नास्त्येव~। तस्य ज्योतिष्ठोमविकृतित्वेनातिदिष्टैरेव प्राकृताङ्गैस्तदाकाङ्क्षानिवृत्तेः~।~{\br ननु} संनिहितविधिबलादाकाङ्क्षोत्थाप्यत इति चेत् तर्ह्या-
\newpage
%%%%%%%%%%%%%%%%%%%%%%%%%%%%%%%%%%%%%%%
\fancyhead[LO]{[पाठसादेश्येन विनि०]}
{\bl\noindent
राजन्यं जिनातीति देवनादयो\footnotemark\ धर्मा अभिषेचनीयसंनिधौ पठिता अपि स्थानान्न तदङ्गं, किंतु प्रकरणाद्राजसूयाङ्गमिति~।}
\begin{center}
\textbf{स्थाननिरूपणम्}    
\end{center}

{\bl देशसामान्यं स्थानम्~। तद्द्विविधम् पाठसादेश्यमनुष्ठानसादेश्यं चेति~।स्थानं क्रमश्चेत्यनान्तरम्~। पाठसादेश्यमपि द्विविधम् {\al यथासङ्ख्यपाठः संनिधिपाठश्चेति~}।~}
\begin{center}
\textbf{पाठसादेश्येन विनियोगः }    
\end{center}

{\bl तत्र {\qtl ऐन्द्राग्नमेकादशकपालं निर्वपेत्, वैश्वानरं द्वादशकपालं निर्वपेत्} इत्येवं क्रमविहितेषु {\qtl इन्द्राग्नी रोचना दिवः} इत्यादीनां याज्यानुवाक्यामन्त्राणां यथासंख्यं प्रथमस्य प्रथमं द्वितीयस्य द्वितीयमित्येवंरूपो विनियोगो यथासंख्यपाठात्~। प्रथम-\\} 
\hrule
\vspace{3mm}
\noindent
काङ्क्षारूपमन्तराले प्रकरणमादौ परिकल्प्य तद्द्वारा वाक्यलिङ्गश्रुतिकल्पनया संनिधिर्विप्रकृष्यते, राजसूयाकाङ्क्षारूपं महाप्रकरणं तु क्लृप्तत्वादेकयाऽऽकाकाङ्क्षया संनिकृष्यते, ततश्च प्रकरणेन संनिधेबोधात्सर्वयागशेषा विदेवनादयो धर्मा इत्यभिप्रेत्य प्रकरणस्य स्थानादिभ्यो बलवत्त्वमाह\textendash {\br इदं चेत्यादिना~।} \\

 स्थानं लक्षयति\textendash  {\br देशसामान्यमिति~।} संनिधिविशेषत्वमिति लक्षणान्तरम्~। स्थानं क्रमश्चेति लक्ष्यत्वेन निर्दिश्यते~। पाठसमानदेशत्वेनानुष्ठानसमानदेशत्वेन च स्थानं विभजते {\br तद्द्विविधमित्यादिना~।} यथासंख्यपाठत्वेन संनिधिपाठत्वेन च पाठसमानदेशत्वमपि विभजते {\br पाठसादेश्यमित्यादिना~।}\\

 तत्र यथासंख्यपाठत्वेन समानदेशत्वं तमुदाहरति\textendash\ {\br तत्रैन्द्राग्नमित्यादिना~। प्रथमस्य प्रथममिति~।} प्रथमस्यैन्द्राग्नमित्यैन्द्राग्नेष्टियागस्य {\qt इन्द्राग्नी रोचना दिव} इति प्रथमं याज्यानुवाक्यायुगलमङ्गं, द्वितीयस्य वैश्वानरमिति वैश्वा
\blfootnote{पाठा०\textemdash\ $^{१}$विदेवनादया.}
\newpage
%%%%%%%%%%%%%%%%%%%%%%%%%%%%%%%%%%%%%%%%%
\fancyhead[RE]{[ पाठसादेश्येन विनि० ]}
{\bl\noindent पठितमन्त्रस्य हि कैमर्थ्याकाङ्क्षायां प्रथमतो विहितं कर्मैव प्रथममुपतिष्ठते, समानदेशत्वात्~। एवं द्वितीयमन्त्रस्यापि~। वैकृताङ्गानां प्राकृताङ्गानुवादेन विहितानां संदंशपतितानां विकृत्यर्थत्वं संनिधिपाठात्~। यथा {\qtl आमनहोमानाम्~}। तेषां हि कैमर्थ्याका-\\}
\hrule
\vspace{3mm}
\noindent
नरेष्टियागस्य {\qt वैश्वानरोऽजीजनत्} इत्यादि द्वितीयमङ्गमित्यर्थः~। एवंरूपे विनियोगे हेतुमाह\textendash {\br प्रथमपठितमन्त्रस्येत्यादिना~। कैमर्थ्याकाङ्क्षायामिति~।} किमर्थोऽयं मन्त्र इति कैमर्थ्याकाङ्क्षायामित्यर्थः~।~{\br एवं द्वितीयमन्त्रस्यापीति~।} किमर्थोऽयं मन्त्र इति द्वितीयमन्त्रस्यापि कैमर्थ्याकाङ्क्षायां द्वितीयस्थाने विहितमेव कर्मोपतिष्ठते समानदेशत्वादित्यर्थः~। कोऽर्थो यस्य स किमर्थः, तस्य भावः कैमर्थ्यं तस्याकाङ्क्षायामिति निरुक्त्या किमनेन मन्त्रेण भाव्यमिति सिध्यति~। अत्र च क्रमस्योदाहरणान्तरमपि वर्तते~। तथा हि\textendash दर्शपूर्णमासयोराध्वर्यवे काण्डे आग्नेयोपांशुयाजाग्नीषोमीययागाः क्रमेणाम्नाताः~। याजमाने च काण्ड आग्नेयादिविषयास्तेनैव क्रमेण मन्त्रा आम्नाताः {\qt अग्नेरहं देवयज्ययान्नादो भूयासं दब्धिरस्यदब्धो भूयासममुं दभेयमग्नीषोमयोरहं देवयज्यया वृत्रहा भूयासम्}इति~।
तत्राद्यन्तयोराग्नेयाग्नीषोमीययोः कर्मणोराद्यन्तौ मन्त्रौ, मध्यमस्य चोपांशुयाजस्य कर्मणो मध्यमो {\qt दब्धिरस्यदब्धो भूयासममुं दभेयम्} इति मन्त्रोऽङ्गं, तस्य च ब्राह्मणवाक्यमेवमाम्नायते {\qt एतया वै दब्ध्या देवा असुरान् दभ्रुवन् तयैव भ्रातृव्यं दभ्नोति}इति~। दब्धिर्घातकमायुधमित्यर्थः~। न चाग्नेयाग्नीषोमीययोरप्यनिष्टनिवारकत्वेन साधारणलिङ्गेन दब्धिमन्त्रस्याङ्गत्वं किं न स्यादिति वाच्यम्~। यथा वाक्यद्वयानुसंधानेन संपन्नं प्रकरणं पृथक्प्रमाणं तथा प्रकरणद्वयानुसंधानसंपन्नेन क्रमप्रमाणेनोपांशुयाजे तस्य मन्त्रस्य विनियोगात्~। न च क्रमस्य प्रकरणेऽन्तर्भाव इति वाच्यम्~। द्वयोर्वाक्ययोरिव प्रकरणयोरेकवाक्यत्वाभावात्~। तस्माक्रमप्रमाणेन मध्यवर्तिन उपांशुयाजस्य
मध्यवर्ती मन्त्रोऽङ्गं समानदेशत्वादिति यथासंख्यपाठाद्विनियोग उक्तः~। संनिधिपाठात्तु यानि वैकृतान्यङ्गानि प्राकृताङ्गानुवादेन विहितानि संदंशपतितानि तेषां विकृतौ विनियोग इत्याह\textendash {\br वैकृताङ्गानामित्यादिना~।} तत्रोदाहरणमाह\textendash {\br यथा आमनहोमानामिति~।} {\qt अमनसे स्वाहा, रेतस्विने स्वाहा} इत्यादय आमनहोमाः~। यद्वा, {\qt अम}- 
\newpage
%%%%%%%%%%%%%%%%%%%%%%%%%%%%%%%%%%%%
\fancyhead[LO]{[अनु०सा०विनियोग ]}
{\bl\noindent ङ्क्षायां फलं विकृत्यपूर्वमेव भाव्यत्वेन संबध्यते, उपस्थितत्वात्, स्वतन्त्रफलकत्वे विकृतिसंनिधिपाठानर्थक्यापत्तेः~।}\\

\begin {center}
\textbf {अनुष्ठानसादेश्येन विनियोगः}
\end {center}

{\bl पशुधर्माणामग्नीषोमीयार्थत्वमनुष्ठानसादेश्यात्~।औपवस-\\}
\hrule
\vspace{3mm}
\noindent
नोहोमानामिति पाठः प्रामादिक एव, किंत्वामनहोमानामिति सकाररहित आपूर्व एव पाठः साधुः~। तत्र च {\qt वैश्वदेवीं सांग्रहणीं निर्वपेद्ग्रामकामः} इत्यस्य काम्येष्टियागस्य संनिधौ श्रूयन्त आमनहोमा {\qt आम\footnotemark नदेवा इति~। तिस्र आहुतीर्जुहोति} इति~।~{\br नन्वा}मनहोमानां फलाकाङ्क्षायां फलवद्विकृत्यपूर्वमेव भाव्यत्वेन संबध्यत इत्यसत् , तेषां मुख्ययागत्वे विरोधाभावात् , न ह्याग्नेयादीनां षण्णामनुमत्यादीनां च बहूनां मुख्यत्वं विरुद्धमित्याशङ्क्य यथा {\qt दर्शपूर्णमासाभ्यां स्वर्गकामः} इति
वाक्येनाग्नेयादीनां फलसंबन्धावगमस्तथाऽऽमनहोमानां फलसंबन्धावगमाभावान्न प्राधान्यं युज्यते, {\qt सांग्रहणीं निर्वपेद्ग्रामकामः} इति वाक्यस्य तु सांग्रहण्या एव फलसंबन्धबोधकत्वेनामनहोमानां तत्संबन्धबोधकत्वाभावात्~। तस्मात् {\qt फलवत्संनिधावफलं तदङ्गम्} इति न्यायात्फलवत्याः सांग्रहण्याः संनिधावाम्नाता अफला आमनहोमास्तदङ्गमित्यभिप्रेत्य किमर्था इमे किलेति कैमर्थ्याकाङ्क्षायां फलवत्तया विकृत्यपूर्वस्यैव भाव्यत्वेन संबन्धे हेतुमाह\textendash {\br उपस्थितत्वादिति~।} संनिधिप्रमाणेनोपस्थितत्वादित्यर्थः~। {\br नन्वा}मनहोमानां सांग्रहणीसंनिधिपाठेऽपि विश्वजिन्न्यायेन स्वतन्त्रफलकत्वमेव किं न स्यादित्यत आह\textendash {\br स्वतन्त्रफलकत्व इत्यादिना~।} फलवत्कर्मासंनिधौ पठितस्यैवाश्रूयमाणफलस्य विश्वजिन्न्यायेन स्वतन्त्रफलं कल्प्यते~। अन्यथा प्रयाजादीनामपि तन्न्यायेन स्वतन्त्रफलकत्वापत्तिः
स्यात्~। अफलस्य फलवत्संनिधौ पाठस्तु तदङ्गत्वायैव~। तदभावेऽनर्थकत्वमेव तस्यापद्येतेति भावः~।\\

 संनिधिपाठसादेश्येन विनियोगं निरूप्यानुष्ठानसादेश्येन विनियोगं निरूपयति  {\br पशुधर्माणामित्यादिना~।} पशुधर्मा ह्युपाकरणपर्यग्निकरणादयस्तत्र प्रजापतेर्जायमानाः, इदं पशुमित्याभ्यामृग्भ्यां पशोरुपस्पर्शनमुपाकरणम् , दर्भज्वालया त्रिःप्रदक्षिणीकरणं पर्यग्निकरणम् , यूपे रज्ज्वा बन्धनं यूपनियो-
\blfootnote{पाठा०\textemdash\ $^{१}$ब्रामनमस्यानस्यदेवा इति क्वचित्पाठः~।}
\newpage
%%%%%%%%%%%%%%%%%%%%%%%%%%%%%%%%%%%%%
\fancyhead[RE]{[अनुष्ठानसादेश्येन\textemdash\ }
{\bl\noindent थ्येऽह्नि अग्नीषोमीयः पशुरनुष्ठीयते तस्मिन्नेव दिने ते धर्माः पठ्यन्ते~। अतस्तेषां कैमर्थ्याकाङ्क्षायामनुष्ठेयत्वेनोपस्थितं पश्वपू-\\ }
\hrule
\vspace{3mm}
\noindent
जनमित्यादयो बोध्याः, तेषामनुष्ठानसमानदेशत्वादग्नीषोमीयपशुशेषत्वमेव नतु सवनीयादिशेषत्वमित्यर्थः~। किंच ज्योतिष्टोमप्रकरणे त्रयः पशवः समाम्नाता अग्नीषोमीयः सवनीयोऽनुबन्ध्यश्चेति~। तत्राग्नीषोमीयः पशुः सौत्यनामकादह्नः प्राचीने औपवसथ्यनामकेऽह्नि धिष्ण्यनिर्माणादूर्ध्वं समनुष्ठीयते तत्र चैवाह्नि ते धर्माः समाम्नाताः~। ततश्च तेषां
शेष्याकाङ्क्षायामनुष्ठेयतयोपस्थितमग्नीषोमीयपश्वपूर्वमेव संनिधितो भाव्यत्वेन संबध्यते न तु सवनीयानुबन्ध्यात्पूर्वं तत्संनिधिविरहात्~। \noindent अथ \textendash
कथमिति चेन्न, सवनीयपशोः सौत्यनामकेऽह्नि समाम्नानादाश्विनं ग्रहं गृहीत्वा त्रिवृता यूपं परिवीयाग्नेयं पशुमुपाकरोतीति~। अनुबन्ध्यस्य त्ववभृथान्ते श्रूयमाणत्वाच्च~। तस्मादुपपद्यते ह्येतेषां धर्माणां संनिधिनाग्नीषोमीयार्थत्वं, सवनीयानुबन्ध्ययोस्तु चोदकात्तेऽतिदिश्यन्ते~। न च पाठसादेश्यादेव तेषां तदर्थत्वं किं न स्यादिति वाच्यम् ;
अग्नीषोमीयपशोः सोमक्रयसमीपे पाठात्तेन तत्त्वासंभवात्~। न च क्रयसंनिधावेव तस्यानुष्ठानमपि किं न स्यादिति वाच्यम् ; {\qt स एष द्विदैवत्यः पशुरौपवसथ्येऽहनि आलब्धव्यः} इति वाक्यात्तस्य तत्रानुष्ठानानुपपत्तेः~। न च स्थानात्प्रकरणस्य बलीयस्त्वात्तेन तद्धर्माणां ज्योतिष्टोमार्थत्वमेव किं न स्यादिति वाच्यम् ; तस्य सोमयागत्वात्तद्धर्मग्रहणायोगात्~। सोमो ह्यभिषवादीन्धर्मानाकाङ्क्षति नतु यूपनियोजनविशसनादीन्~। तस्मात् {\qt आनर्थक्यप्रतिहतानां विपरीतं बलाबलम्} इति न्यायेन प्रकरणं प्रधानात्प्रच्याव्य स्थानात्पशुयागार्थत्वमेव पशुधर्माणां युक्तं भवति~। न चाङ्गिन्यनर्थकाः सन्तस्ते धर्मा अङ्गेषु त्रिष्वपि पशुयागेष्वविशेषेणावतिष्ठन्त्विति वाच्यम्~। संनिधिरूपस्य विशेषस्य दर्शितत्वात्~। न चाग्नीषोमीयेऽपि तद्धर्माणां प्रकरणादेव विनियोगः किं न स्यादिति वाच्यम् ; क्लृप्तोपकारैरेव प्राकृतधर्मैरग्नीषोमीयस्येतिकर्तव्यताकाङ्क्षाया उपशान्तत्वात्~। अग्नीषोमीयो हि सांनाय्ययागप्रकृतिको भवति तयोरुभयोरपि पशुप्रभवद्रव्यकत्वाविशेषात् सांनाय्यं दधिपयसी तत्र पशुयागः पयोयागप्रकृतिकः, साक्षात्पशुप्रभवद्रव्यकत्वात्~। तच्चोक्तं\textendash {\qt सांनाय्यं वा तत्प्रभवत्वात्} इति~। तस्माच्चोदकप्राप्तैस्तद्धर्मैर्निराकाङ्क्षत्वान्न पशुयागे तद्धर्माणां प्रकरणं विनियोजकं, किंतु स्थानमेवेत्यभिप्रेत्याह\textendash
\newpage
%%%%%%%%%%%%%%%%%%%%%%%%%%%%%%%%%%
\fancyhead[LO]{विनियोगः ]}
{\bl\noindent
र्वमेव भाव्यत्वेन संबध्यते~। तच्च स्थानं समाख्यातः प्रबलम्~।}\\
\hrule
\vspace{3mm}
\noindent
{\br औपवसथ्येऽह्नीत्यादिना~।} एवं स्थानं निरूप्य तस्य समाख्यातः प्राबल्यमाह\textendash {\br तच्च स्थानमित्यादिना~। तच्चेति~।}  उक्तलक्षणलक्षितं चेत्यर्थः~। तन्त्रे तृतीये चेदं स्थितम्~। {\qt शुन्धध्वं दैव्याय कर्मणे} इत्ययं मन्त्रः पौरोडाशिकमिति याज्ञिकैः समाख्याते काण्डे पठितस्तस्य च समाख्यया पुरोडाशकाण्डोक्तानामुलूखलजुह्वादीनामपि शोधनेऽङ्गत्वमिति प्राप्ते राद्धान्तः, न समाख्यया मन्त्रस्य पुरोडाशपात्राङ्गत्वम् , पदार्थयोभिन्नदेशस्थत्वेन संबन्धस्याप्रत्यक्षत्वात् , स्थानविनियोगे तु पदार्थयोर्देशसामान्यलक्षणः संबन्धः प्रत्यक्ष एव~। न च सा पदार्थयोः संबन्धवाचिका भवति, यौगिकशब्दानां द्रव्यवाचकत्वेन पदार्थसंबन्धावाचकत्वात्
, तथात्वे वा तस्याः सा संबन्धमात्रवाचिका तद्विशेषवाचिका वा स्यात्~। नाद्यः; तन्मात्रोक्तो प्रयोजनाभावात्~। सर्वेषां यौगिकवचसां पर्यायतापत्तेश्च~। द्वितीये तु संबन्धविशेषस्य
संबन्धिविशेषनिरूप्यत्वादवश्यं संबन्धिनौ वक्तव्यौ~। तथा च संबन्धिप्रतिपत्त्यैव वाक्यार्थप्रतिपत्तिन्यायेन : संबन्धप्रतिपत्तिसंभवे तत्रापि शक्तिकल्पने गौरवान्न समाख्यायाः संबन्धवाचित्वम्~। तथा चोक्तम्\textendash {\qt सर्वत्र यौगिकैः शब्दैर्द्रव्यमेवाभिधीयते~। न हि संबन्धवाचित्वं संभवत्यतिगौरवात्}~। तथाऽन्यच्च {\qt पाकं तु पचिरेवाह कर्तारं प्रत्ययोऽप्यकः~। पाकयुक्तः पुनः कर्ता वाच्यो नैकस्य कस्यचित्} इति~। किंच पौरोडाशिकमिति समाख्यायां प्रकृतिः पुरोडाशमात्रमभिधत्ते, तद्धितप्रत्ययस्तु
पुरोडाशस्येदमिति व्युत्पत्त्या काण्डम्~। न चैतावता कृत्स्नपुरोडाशपात्राणां मन्त्रसंनिधिः प्रत्यक्षो भवति, किंत्वर्थापत्त्या स कल्प्यते~। कथम् ? शृणु, यद्युक्तः संनिधिर्न स्यात्तदा
शुन्धनप्रतिपादकमन्त्रस्य पौरोडाशिकसमाख्या न स्यात्~। न ह्यग्न्यसंनिहितानां {\qt इषे त्वा} इत्यादिमन्त्राणामाग्नेयकाण्डसमाख्या भवति~। भवति च सा संनिहितानां {\qt युञ्जानः प्रथमं मनः} इत्यादिमन्त्राणाम्~। ततश्च काण्डसमाख्यया संनिधिं परिकल्प्य कल्पितकाण्डसंनिध्यन्यथानुपपत्त्या परस्पराकाङ्क्षारूपं कृत्स्नपात्रप्रकरणं कल्पयित्वा तद्वारा
वाक्यलिङ्गश्रुतीश्च कल्पयित्वा तया श्रुत्या विनियोग इति स्थानापेक्षया विनियोगे समाख्याया विप्रकर्षः~। सांनाय्यपात्राणां तु कुम्भीशाखापवित्रादीनां शोधनमन्त्रसंनिधिः प्रत्यक्षो भवति~। कथम् ? शृणु, इध्माबर्हिःसंपादनस्य मुष्टिनिर्वापस्य चान्तरालं सांनाय्यपात्राणां देश उक्तः शोधनमन्त्रश्चायमिध्माबर्हिर्निर्वापविषय-

\lfoot{५ अ०}
\newpage
%%%%%%%%%%%%%%%%%%%%%%%%%%%%%%
\lfoot{}
\fancyhead[RE]{[ समाख्यानिरूपणम् ]}
{\bl\noindent अत एव शुन्धनमत्रः सांनाय्यपात्राङ्गं पाठसादेश्यात्, न तु पौरोडाशिकमिति समाख्यया पुरोडाशपात्राङ्गम्~।}
\begin{center}
 \textbf{समाख्यानिरूपणम्}   
\end{center}
 
{\bl समाख्या यौगिकः शब्दः~। सा च द्विविधा {\qtl वैदिकी लौकिकी च~}। तत्र {\qtl होतुश्चमसभक्षणाङ्गत्वम् , होतृचमस इति वैदिक्या समाख्यया~।  अध्वर्योस्तत्तत्पदार्थाङ्गत्वम् , लौकिक्या}}\\
\hrule
\vspace{3mm}
\noindent
योर्मन्त्रानुवाकयोर्मध्यमेऽनुवाके पठ्यते~। तेन च प्रत्यक्षसंनिधिना प्रकरणादीनां चतुर्णामेव कल्पनात्संनिधिः समाख्यापेक्षया संनिकृष्यते~। तस्मात्क्रमेण समाख्यां बाधित्वा सांनाय्यपात्राणां शोधने शेषोऽयं मन्त्रो भवतीत्यभिप्रेत्याह\textendash {\br अत एवेत्यादिना~।} अत एव स्थानस्य समाख्यातः प्रबलत्वादेव~। शुन्धनमन्त्रः {\qt शुन्धध्वं दैव्याय कर्मण} इति मन्त्रः~। सांनाय्यपात्राङ्गमिति सांनाय्ययागयोरैन्द्रदध्यैन्द्रपयसोः पात्राणां कुम्भीशाखापवित्रादीनामङ्गमित्यर्थः~।~{\br पाठसादेश्यादिति~।}
संनिधिपाठादित्यर्थः~। {\br पुरोडाशपात्राङ्गमिति~।} पुरोडाशपात्राणामुलूखलादीनामङ्गमित्यर्थः~।\\

 एवं स्थानं निरूप्य समाख्यां लक्षयति\textendash {\br समाख्या यौगिकः शब्द इति~।} अत्र यौगिकशब्दत्वं लक्षणम् , समाख्येति लक्ष्यम्~। शब्दश्चतुर्धा {\ab यौगिको
रूढो योगरूढो यौगिकरूढश्चेति~}। तत्राध्वर्यवं पाचक इत्यादिर्यौगिकः स एव समाख्यति चोच्यते~। यत्रावयवार्थं एव ज्ञायते स यौगिक इति तन्निर्वचनम्~। योऽवयवशक्तिनिरपेक्षया समुदायशक्त्यैवार्थं बोधयति स रूढः, यथा गवादिशब्दः ~। यस्त्ववयवशक्तिविषये समुदायशक्त्यापि प्रवर्तते स योगरूढः~। यथा पङ्कजादिशब्दस्यैवावयवशक्त्या पङ्कजनिकर्तृत्वेन समुदायशक्त्या च पद्मत्वेन रूपेण पद्मबोधकत्वात्~। यस्त्ववयवशक्तिसमुदायशक्तिभ्यां रूढार्थं यौगिकार्थं च स्वातन्त्र्येण बोधयति स यौगिकरूढः, यथा {\qt उद्भिदादिशब्दः~}। स चोर्ध्वभेदनकर्तृतरुगुल्मादिकं बोधयति यागविशेषमपि चेति~। सा, यौगिकशब्दात्मिका समाख्या~। तत्र, तयोः समाख्ययोर्मध्य इत्यर्थः~। {\br होतुरिति~।} भक्षणस्य क्रियात्मकत्वेन प्राधान्यात्तत्कर्तुर्होतुर्भवति तदङ्गत्वमित्यर्थः~। {\br तत्तत्पदार्थाङ्गत्वमिति~।} {\qt पुरोऽध्वर्युविभजति}
इत्यादिना यजुर्वेदेन विहितानां पदार्थानामङ्गत्वमित्यर्थः~।~{\br लौकि- }
\newpage
%%%%%%%%%%%%%%%%%%%%%%%%%%%%%%%%%%%%%%%%%%%
\fancyhead[LO]{[विनि० वि० बोधि० ]}
{\bl\noindent आध्वर्यवमिति समाख्ययेति संक्षेपः~। तदेवं निरूपितानि संक्षेपतः श्रुत्यादीनि षट्प्रमाणानि~।}
\begin{center}
\textbf{विनियोगविधिबोधिताङ्गानि}    
\end{center}

{\bl एतत्सहकृतेन विनियोगविधिना समिदादिभिरुपकृत्य {\qtl दर्शपूर्णमासाभ्यां यजेत} इत्येवंरूपेण यानि विनियोज्यन्ते तान्यङ्गानि द्विविधानि {\qtl सिद्धरूपाणि
क्रियारूपाणि चेति~}।~तत्र {\qtl सिद्धानि जातिद्रव्यसंख्यादीनि~}।~तानि च दृष्टार्थान्येव~।~क्रियारूपाणि च द्विविधानि {\qtl गुणकर्माणि प्रधानकर्माणि च} ~। एतान्येव संनिपत्योपकारकाणि आरादुपकारकाणीति चोच्यन्ते~।\\}
\hrule
\vspace{3mm}
\noindent
{\br क्येति~।} याज्ञिकैः परिकल्पितयेत्यर्थः~। {\br तृतीये} स्थितं {\qt प्रैतु  होतुश्चमसः} इत्यत्र समाख्या भक्षहेतुः {\qt हरिरसि हारियोजनः} इत्यनेन मन्त्रेण गृह्यमाणो ग्रहो हारियोजनः~। तत्र वाक्यं {\qt वषट्कर्तुः प्रथमभक्षः} इत्यत्र वषट्कार इत्येवमुक्तत्वात् त्रय एव भक्षहेतव इति प्राप्तेराद्धान्तः - हविर्धाने ग्रावभिरभिषुत्याहवनीये हुत्वा प्रत्यञ्चः परेत्य सदसि भक्षान्भक्षयन्तीति श्रूयते~। उत्तरवेद्याः प्रतीचीने सदसः प्राचीने मण्डपेऽभिषवः, उत्तरवेद्यां होमः, सदसि भक्षणं, तत्राभिषवहोमयोर्वचनान्तरप्राप्तयोरविधेयतया तयोर्निमित्तत्वेनापूर्वभक्षणं विधीयते~। तस्मात्स्थानादिवदेतयोरपि\alfootnote{टिप्प०\textemdash\ $^{1}$अभिषवहोमयोरित्यर्थः~।}\footnotemarkA[1]
भक्षणहेतुत्वमस्तीत्यभिप्रेत्याह\textendash {\br संक्षेप इति~।} श्रुत्यादिप्रमाणस्य विनियोगविधिसहकारिणो निरूपणमुपसंहरति\textendash {\br तदेवमिति~।}\\

 एतैः षड्भिः प्रमाणैः सहकृतेन विनियोगविधिना यान्यङ्गानि विनियोज्यन्ते तानि विनियोगविधेः स्वरूपमनुवदन्विभजते {\br एतदित्यादिना~।} तत्र सिद्धरूपाणि निर्दिशति  {\br तत्रेति~।} सिद्धरूपक्रियारूपाणां मध्य इत्यर्थः~। जातिः पशुत्वादिः, द्रव्यं व्रीह्यादि, संख्या एकत्वादिः, तेषां यागस्वरूपोपकारकत्वेन दृष्टार्थत्वमेवेत्याह\textendash {\br तानि चेति~।} क्रियारूपाणां विभागद्वारा लक्षणमाह\textendash\ {\br क्रियारूपाणीत्यादिना~।} गुणस्य कर्माङ्गस्य द्रव्यादेः संस्कारकराणि क्रियाविशेषरूपाणि गुणकर्माणि यथाऽवघातादीनि~। प्रधानस्य फलापूर्वस्योपकारकाणि तानि प्रधानकर्माणि यथा
प्रयाजादीनि~। तेषामेव लक्षणान्तरमाह\footnoteA[2]{संशान्तरमित्यर्थः॥}\textemdash\ {\br एतान्येवेत्यादिना~।}
\newpage
%%%%%%%%%%%%%%%%%%%%%%%%%%%%%%%%%%%%%%%%%%%
\fancyhead[RE]{[ आरादुप\textemdash\ }
\begin{center}
 \textbf{संनिपत्योपकारकाणि }   
\end{center}
 
{\bl {\al कर्माङ्गद्रव्याद्युद्देशेन विधीयमानं कर्म संनिपत्योपकारकम्~}। यथाऽवघातप्रोक्षणादि~। तच्च-{\qtl दृष्टार्थं अदृष्टार्थं दृष्टादृष्टार्थं चेति}~। तत्र दृष्टार्थमवघातादि, अदृष्टार्थं प्रोक्षणादि, दृष्टादृष्टार्थं पशुपुरोडाशादि~। तद्धि द्रव्यत्यागांशेनैव अदृष्टं देवतोद्देशेन च देवतास्मरणं दृष्टं करोति~।~}
\begin{center}
 \textbf{आरादुपकारकाणि }    
\end{center}

{\bl द्रव्याद्यनुद्दिश्य केवलं विधीयमानं कर्म आरादुपकारकम्~। यथा प्रयाजादि~। आरादुपकारकं च परमापूर्वोत्पत्तावेवोप-\\}
\hrule
\vspace{3mm}
\noindent
संनिपत्योपकारकाणां लक्षणमाह\textendash {\br कर्माङ्गेत्यादिना~।} यानि चाङ्गानि साक्षात्परम्परया\alfootnote{टिप्प०\textemdash\ $^{1}$निष्पाद्येत्यत्रत्यं निष्पादनान्वयि~।}\footnotemarkA[1] वा विहितफलसाधनयागशरीरं निष्पाद्य तद्द्वारा तदुत्पत्त्यपूर्वोपयोगीनि तानि संनिपत्योपकारकाणीति फलितम्~। द्रव्यादीत्यादिना देवतादिपरिग्रहः~। क्रियारूपत्वं सर्वेषां कर्मणां गुणकर्मत्वमवघातादीनां प्रधानकर्मत्वं प्रयाजादीनां संनिपत्योपकारकत्वं अवघातादीनामारादुपकारकत्वं प्रयाजादीनां च लक्षणमिति समुदायतात्पर्यम्~। तत्रोदाहरणमाह\textendash {\br यथावघातेति~}।~संनिपत्योपकारकाणां पुनरपि त्रिविधं विभागं करोति {\br तच्चेत्यादिना~।} तच्च संनिपत्योपकारकं चेत्यर्थः~। दृष्टार्थमुदाहरति {\br दृष्टार्थमवघातादीति~।} अवघातादेस्तुषविमोकादिरूपदृष्टद्वारा यागस्वरूपतदुत्पत्त्यपूर्वहेतुत्वात्संभवति तस्य दृष्टार्थत्वमित्यर्थः~। अदृष्टार्थमुदाहरति {\br अदृष्टार्थमिति~।} प्रोक्षणादेर्व्रीहिगतसंस्काररूपातिशयादृष्टद्वारा प्रधानयागोत्पत्त्यपूर्वहेतुत्वात्संभवति केवलादृष्टार्थत्वं
प्रोक्षणमन्तरेणापि यागस्वरूपानुपपत्त्यभावात् दृष्टोपकारासंभवात्~। दृष्टादृष्टोभयार्थमुदाहरति {\br दृष्टादृष्टार्थमिति~।} पशुपुरोडाशादीत्यादिना यागादिसंग्रहः~।\footnoteA[2]{अत्र गुणादीतिपाठो भाति, तेन चारुण्यादेः संग्रहः~।} {\br तद्धि} पशुपुरोडाशादि हीत्यर्थः~। {\br अदृष्टमिति~।} यागोत्पत्त्यपूर्वद्वारा फलापूर्वमित्यर्थः~।
 आरादुपकारकाणां लक्षणमाह\textendash\ {\br द्रव्याद्यनुद्दिश्येति~।}आत्मसमवेतापूर्वजनकान्यारादुपकारकाणीत्यपि वदन्ति~। तत्रोदाहरणमाह\textendash\ {\br यथेति~।} प्रयाजा-
\newpage
%%%%%%%%%%%%%%%%%%%%%%%%%%%%%%%%%%%%
\fancyhead[LO]{कारकाणि ]}
{\bl\noindent युज्यते~।~संनिपत्योपकारकं तु द्रव्यदेवतासंस्कारद्वारा यागस्वरूपेऽप्युपयुज्यते~। इदमेव चाश्रयिकर्मेत्युच्यते~। तदेवं निरूपितः संक्षेपतो विनियोगविधिः~।}\\
\hrule
\vspace{3mm}
\noindent
दीत्यादिनाज्यभागानुयाजादिपरिग्रहः~। अत्र परमापूर्वोत्पत्तावेवेत्येवकारेण द्रव्यदेवतागतसंस्कारजननद्वारा यागस्वरूपोपयोगित्वमारादुपकारकाणां निरस्यते~। यागस्वरूपेऽपीति {\qt अपि} ना संनिपत्योपकारकाणां द्रव्यदेवतासंस्कारद्वारा यागोत्पत्त्यपूर्वेऽप्युपकारकत्वं समुच्चीयते~। अत्र च व्रीह्यादीनां पिष्टद्वारा पुरोडाशनिर्वर्तकत्वं तद्द्वारा यागशरीरतदुत्पत्त्यपूर्वजनकत्वं च भवति~। याज्यानुवाक्यादेर्देवतासंस्कारद्वारा यागोत्पत्त्यपूर्वादावुपयोगित्वं देवतायाश्च साक्षाद्यागशरीरनिर्वर्तकत्वं तद्द्वारा
तदुत्पत्त्यपूर्वनिष्पादकत्वं च भवति~। यागस्य देवतोद्देशेन द्रव्यत्यागस्वरूपत्वाद्द्रव्यदेवते खलु यागस्वरूपमित्यभ्युपगमाच्चेति ध्येयम्~। {\br इदमेव चेति~।} संनिपत्योपकारकमेव च न त्वारादुपकारकमित्यर्थः~। {\br आश्रयि} द्रव्यदेवतादिरूप आश्रयोऽस्यास्तीत्याश्रयि~। पारिभाषिकी वाश्रयिकर्मेति संज्ञा~। इदमेव सामवायिककर्मेत्यपि वदन्ति~। किंच सामान्यतः कर्म द्विविधमर्थकर्म गुणकर्म चेति~। तत्रात्मसमवेतापूर्वजनकं कर्मार्थकर्म-यथाऽग्निहोत्रदर्शपूर्णमासप्रयाजानुयाजादिकं
तस्यात्मगतफलापूर्वजनकत्वात्~। द्रव्यादिसंस्कारजनकं द्वितीयम्~। तदेव संनिपत्योपकारकमित्युक्तम्~। तदपि द्वितीयमुपयोक्ष्यमाणसंस्कारकमुपयुक्तसंस्कारकं चेति~।
तत्रावघातप्रोक्षणादिकमुपयोक्ष्यमाणसंस्कारकं, यागे व्रीहीणामुपयोक्ष्यमाणत्वात्~। प्रतिपत्तिकर्म चात्वालकृष्णविषाणप्रासनेडाभक्षणादिकमुपयुक्तकृष्णविषाणपुरोडाशादिसंस्कारकं द्वितीयम्~।~उपयुक्तस्याऽऽकीर्णकरस्यावशिष्यमाणद्रव्यादेर्विहितदेशे
संस्कारविशेषहेतुः प्रक्षेपः प्रतिपत्तिकर्म~। उपयोक्ष्यमाणसंस्कारभिन्नसंस्कारकर्मत्वं प्रतिपत्तिकर्मत्वमित्यपि वदन्ति
~। {\br ननु} भवतु कथंचिदिडाभक्षणस्य प्रधानयागोपयुक्ताकीर्णकरपुरोडाशद्रव्यप्रक्षेपात्मकत्वात्प्रतिपत्तिकर्मत्वं, चात्वाले कृष्णविषाणप्रासनस्याग्निहोत्रादिकर्मवद्विहितबहिर्देशविशेषे प्रक्षेपक्रियात्मकत्वात्कथं प्रतिपत्तिकर्मत्वं, तथा च तद्वदर्थकर्मत्वमेव तस्य~।~तदित्थं श्रूयते ज्योतिष्टोमे-नीतासु दक्षिणासु चात्वाले कृष्णविषाणां प्रास्यतीति~।~यजमानेन दक्षिणा दत्ता ऋत्विग्भिर्यदा नीतास्तदा यजमानः स्वस्ते धृतं कृष्णमृगस्य शृङ्गं ,चात्वालनामकगर्ते परित्यजेत्~। सोऽयं
\newpage
%%%%%%%%%%%%%%%%%%%%%%%%%%%%%%%%%%%
\fancyhead[RE]{[ प्रयोग\textemdash\ }
\begin{center}
 \textbf{प्रयोगविधिः}   
\end{center}
 
{\bl प्रयोगप्राशुभावबोधको विधिः प्रयोगविधिः~। स चाङ्गवाक्यैकवाक्यतापन्नः प्रधानविधिरेव~। स हि साङ्गं प्रधानमनुष्ठापयन्विलम्बे प्रमाणाभावादविलम्बापरपर्यायं प्रयोगप्राशुभावं विधत्ते~। न च तदविलम्बेऽपि प्रमाणाभाव इति वाच्यम्~। विलम्बे हि अङ्गप्रधानविध्येकवाक्यतावगततत्साहित्यानुप -\\} 
\hrule
\vspace{3mm}
\noindent
परित्यागोऽर्थकर्म~।~कुतः ? सप्रयोजनत्वलाभात्~। प्रतिपत्तिकर्मत्वेऽपूर्वाभावे निरर्थकः स्यात्~। तस्मादपूर्वलाभायार्थकर्मत्वमेव न तु प्रतिपत्तिकर्मत्वमिति चेन्न~। {\qt कृष्णविषाणया कण्डूयति} इति तृतीयाश्रुत्या यजमानशिरःकण्डूतावुपयुक्तस्य कृष्णविषाणस्य प्रतिपत्त्यपेक्षत्वात्~। न च प्रतिपत्तिकर्मत्वेऽत्यन्तमपूर्वाभावः~। चात्वाल एव प्रासनमित्येवंविधस्य नियमस्य वैधत्वेन प्रासनक्रियाप्रयुक्तापूर्वाभावेऽपि नियमापूर्वसद्भावात्~। तस्मात्प्रासनं प्रतिपत्तिकर्मेति~। तमेतं सर्वमभिप्रेत्य विनियोगविधिनिरूपणमुपसंहरति {\br तदेवमिति~।}\\

 इदानीं प्रयोगविधिं निरूपयति {\br प्रयोगेत्यादिना~।} साङ्गप्रधानकर्मप्रयोगप्राशुभावबोधक इत्यर्थः~। स च प्रयोगविधिः प्रयाजाद्यङ्गजातवाक्यैकवाक्यतामापन्नो {\qt दर्शपूर्णमासाभ्यां स्वर्गकामो यजेत} इत्यादिप्रधानविधिरेवोक्तवक्ष्यमाणविधित्रयमेलनरूपश्चतुर्थोऽयं विधिर्नतु विध्यन्तरमित्याह\textendash {\br स चेति~।} तत्र हेतुं प्रदर्शयन्
प्रयोगप्राशुभावशब्दं व्याचष्टे {\br स हीत्यादिना~।}  प्रयोगविधिः साङ्गप्रधानकर्मप्रयोगाविलम्बं विधत्ते तद्विलम्बे प्रमाणाभावात्संप्रतिपन्नवत्~। तत्र प्रयोगविधिः प्रयोगविलम्बं विधत्ते, तदविलम्बे प्रमाणाभावादिति सप्रतिसाधनतामाशङ्क्य समाधत्ते {\br न चेत्यादिना~।} अङ्गप्रधानविध्योरेकवाक्यतयावगतस्याङ्गप्रधानयोः साहित्यस्यानुपपत्तेरेव तदविलम्बे प्रमाणत्वान्न तत्र प्रमाणाभाववचनं शोभतेतरामिति~। तत्रार्थापत्तिं प्रमाणं दर्शयति {\br विलम्ब इत्यादिना~।} {\qt हि} शब्देन तत्साहित्यनिश्चयं द्योतयति~। प्रयाजाद्यङ्गजातमङ्गशब्दार्थः ~। प्राधानशब्देन च दर्शादिः प्रधानयागो ह्युच्यते~। तावेव तच्छब्दार्थः~। न  च तत्साहित्यमेव नास्तीति वाच्यम् ; अङ्गप्रधानविध्येकवाक्यत्वानुपपत्तेः, तस्य प्रयोजनान्तरानुपलब्धेश्च~। {\br नन्व}विलम्बमन्तरेण नाङ्गप्रधानयोः साहित्या-
\newpage
%%%%%%%%%%%%%%%%%%%%%%%%%%%%%%%
\fancyhead[LO]{विधिः ]}
{\bl\noindent पत्तिः~। विलम्बेन क्रियमाणयोः पदार्थयोरिदमनेन सहकृतमिति साहित्यव्यवहाराभावात्~। स चाविलम्बो नियते क्रमे आश्रीयमाणे भवति~। अन्यथा हि किमेतदनन्तरमेतत्कर्तव्यमेतदनन्तरं वेति प्रयोगविक्षेपापत्तेः।अतः प्रयोगविधिरेव स्वविधेय\\}
\hrule
\vspace{3mm}
\noindent
नुपपत्तिः, तेन विनापि तथा व्यवहारादित्याशङ्क्यमैवमित्याह\textendash  {\br विलम्बेनेत्यादिना~।} पूर्वेऽह्नि परेऽह्नि  च क्रियमाणयोर्भाोजनयोरिदमनेन भोजनेन सहकृतमिति व्यवहारादर्शनान्न विलम्बेन क्रियमाणयोरङ्गप्रधानयोः साहित्यव्यवहारोऽभ्रान्तस्योपपद्यते ततश्च सा तदवस्थेत्यर्थः~। सोऽप्यङ्गप्रधानयोरविलम्बस्तन्नियतक्रमसमाश्रयणे संभवति यथाग्नेयहविरभिघारणं, ततश्चैन्द्रदधिहविरभिघारणं, तत आग्नेययागानुष्ठानं, तदनन्तरमैन्द्रदधियागानुष्ठानमिति नियतः क्रमः~। ततश्चैकान्तरितव्यवधानेनाविलम्ब इति सर्वमुपरिष्टात्स्पष्ठीभविष्यतीत्यभिप्रेत्याह \textendash  {\br स चेत्यादिना~।} क्रमानङ्गीकारे दोषमाह\textendash {\br अन्यथेत्यादिना~। प्रयोगविक्षेपापत्तेरिति~।} किमाग्नेयहविषोऽभिघारणमैन्द्रदधिहविरभिघारणानन्तरं कर्तव्यं किंवैन्द्रदधिहविषोऽभिघारणमाग्नेयहविरभिघारणानन्तरमित्येवं सर्वत्र संदेहेन सर्वानुष्ठानोच्छेदापत्तेरित्यर्थः~।~{\qt संशयात्मा विनश्यति} इत्यादिना भगवद्वचनेन संशयात्मन ऐहिकभोगसाधनेऽपि प्रवृत्त्यभावोपलम्भान्न संदिहानस्यामुत्रभोगसाधने प्रवृत्तिरुपपद्यत इति भावः~।~{\br ननु} नाङ्गप्रधानयोर्नियतः क्रम उपपद्यते, तद्विधायकाभावात्~। न च प्रयोगविधिरेव तद्विधायकः, वाक्यभेदापातात्~। कथं न प्रयोगविधेरेव
प्रयोगप्राशुभावविधायकत्वे गुणप्रधानयोर्नियतक्रमविधायकत्वे च वाक्यभेदापात इति चेन्मैवमित्याह\textendash\ {\br अत इत्यादिना~।}यतोऽङ्गप्रधानयोर्नियतक्रममन्तरेणाविलम्बापरनामधेयप्रयोगप्राशुभावो नोपपद्यतेऽतः स्वविधेयप्रयोगप्राशुभावसिद्ध्यर्थं नियतं क्रममपि प्रयोगविधिरेव विधत्ते इत्यर्थः~। वाक्यभेदनिरासाय पदार्थविशेषणतयेत्युक्तं, व्यापारभेदेन हि विधाने वाक्यभेदो भवति, न तु विशिष्टविधान इत्यवोचामेति भावः~। पदार्थश्चात्र क्रियारूपो ग्राह्यः, अन्यथा क्रमविधानानुपपत्तेः~। तदुक्तं माधवेन\textendash {\qt यद्यपि क्रमस्य क्रियात्वाभावात्स्वरूपेण न विधियोग्यता तथापि दध्ना जुहोतीत्यादावक्रियारूपं दधिद्रव्यं यथा क्रियाविशेषणं}
\newpage
%%%%%%%%%%%%%%%%%%%%%%%%%%%%
\fancyhead[RE]{[ क्रमस्वरूपम् ] }
{\bl\noindent
प्रयोगप्राशुभावसिद्ध्यर्थं नियतं क्रममपि पदार्थविशेषणतया विधत्ते~। अत एवाङ्गानां क्रमबोधको विधिः प्रयोगविधिरित्यपि लक्षणम्~।}
\begin{center}
 \textbf{क्रमस्वरूपम्}   
\end{center}
 
{\bl तत्र क्रमो नाम विततिविशेषः, पौर्वापर्यरूपो वा~।~}\\
\hrule
\vspace{3mm}
\noindent
{\qt सद्विधीयते दधिसाधनकं होमं कुर्यादिति, एवमनेन क्रमेण कर्तव्यमिति क्रियाविशेषणतया क्रमो विधीयताम्} इति~। यतः प्रयोगविधिरेव पदार्थविशेषणतया तन्नियतक्रममपि विधत्तेऽत एव कारणात्तस्य लक्षणान्तरमपीत्याह\textendash {\br अत एवेति~।}\\

 {\br ननु} क्रमबोधक इत्यत्र किंलक्षणः क्रमोऽभिमत इत्याकाङ्कायां क्रमं लक्षयति {\br तत्र क्रम इति~।} यद्वा, वाक्यार्थबोधे पदार्थज्ञानस्य हेतुत्वात्क्रमपदार्थज्ञानमन्तरेण प्रयोगविधिलक्षणवाक्यार्थबोधो न स्यात्ततश्च क्रमपदार्थो वक्तव्य इत्याकाङ्क्षायां क्रमलक्षणमाह\textendash {\br तत्र क्रम इति~।} तत्र, प्रयोगविधिलक्षणवाक्यार्थघटकपदार्थानां मध्य इत्यर्थः~। वितननं वितानो वा विततिः~। {\qt तनु विस्तारे} इति धातोः स्त्रियां भावे क्तिन्प्रत्यये विततिरिति रूपम्~। तथा च बहुभिः कर्तृभिर्युगपत्कृतानामपि पदार्थानां वितानविशेषो भवत्येव न तु तत्र क्रमव्यवहार इत्यरुच्या लक्षणान्तरमाह\textendash  {\br पौर्वापर्यरूपो वा इति~।} यद्वा, युगपत्कृतानां क्रमेण कृतानां वा पदार्थानां वितानाविशेषेऽपि विशेषपदसूचितं पौर्वापर्यरूपं विवक्षितं लक्षणार्थं लक्षणान्तरव्याजेन वाशब्दमनास्थायां निश्चयार्थं वा
मत्वा प्रकटयति\textendash {\br पौर्वापर्यरूपो वेति~।} तदुक्तं तत्तत्पदार्थानन्तरं तत्तत्पदार्था इत्येवमनेकपदार्थवृत्तिपौर्वापर्यसमुदायरूपविततिरेव क्रम इति
~। अत्र भाट्टदीपिकाकारास्तु तत्र क्रमो नामाव्यवहितोत्तरत्वरूपमानन्तर्यम्~। तच्चैकप्रतियोगिकमेकवृत्ति यथा {\qt वेदं कृत्वा वेदिं करोति} इत्यत्र वेदकरणप्रतियोगिकमानन्तर्यं वेदिकरणवृत्ति~। अत्र दर्शपूर्णमासोत्तरत्वस्यासोमाङ्गत्वात्तद्व्यावृत्त्यर्थमव्यवहितेति विशेषणम्~। तत्र दर्शपूर्णमासपूर्वकालताकत्वमात्रं क्त्वाप्रत्ययार्थः, न त्वव्यवहितांशोऽपि न च सोमविधेस्तदपेक्षा, येनाव्यवधान एव तत्पर्यवस्येत, सोमविधेर्मिन्नप्रयोगविधिविधेयदर्शपूर्णमासप्रतियोगिकक्रमानपेक्ष
\newpage
%%%%%%%%%%%%%%%%%%%%%%%%%%%%%%%%%%%%%
\fancyhead[LO]{[ श्रुतिलक्षणम् ]}
\begin{center}
 \textbf{श्रुत्यादिषट्प्रमाणानि}   
\end{center}
 
{\bl तत्र षट् प्रमाणानि-श्रुति-अर्थ-पाठ-स्थान-मुख्य-प्रवृत्त्याख्यानि
~।\\
\begin {center}\textbf {श्रुतिलक्षणम्}\end {center}

\subsection{तत्र {\al क्रमपरवचनं श्रुतिः}~।~तच्च द्विविधम् {\al केवलक्रमपरं} तद्विशिष्टपदार्थपरं} चेति~। तत्र {\qtl वेदं कृत्वा वेदिं करोति} इति केवलक्रमपरं, वेदिकरणादेर्वचनान्तरप्राप्तत्वात्~। {\qtl वषट्कर्तुः प्रथम-\\}}
\hrule
\vspace{3mm}
\noindent
त्वात्~। अतस्तत्रोत्तरकालत्वमेव विधेयं न क्रमः~। क्रमश्च सर्वत्रोत्तरपदार्थाङ्गं, तस्यैव क्वाहं कर्तव्य इत्यपेक्षणान्न तु पूर्वपदार्थाङ्गं मदुत्तरं कः पदार्थः कर्तव्य इत्यपेक्षायाः
क्वचिदप्यदर्शनात्~। पूर्वपदार्थस्तु प्रतियोगितया क्रमविशेषणं दर्शपूर्णमासादिरिव पूर्वकालतायाः~। एतेन क्रमः पदार्थद्वयाङ्गमिति केषांचिदुक्तमपास्तम्, प्रमाणाभावात्~। अस्तु वा {\qt प्रथमभक्षः} इत्यादौ प्राथम्यं पूर्वपदार्थाङ्गमेवेत्याहुः~।\\

 तन्नियमे च षट् प्रमाणानि श्रुत्यादीनि भवन्तीत्याह\textendash {\br तत्रेति~।} तत्र, क्रम इत्यर्थः~।\\

 तत्र श्रुतिं लक्षयति\textendash {\br तत्रेति~।} तत्र, षट्सु प्रमाणेषु मध्य इत्यर्थः~। क्रमपरवचनं, वृत्त्या क्रमबोधकं क्लृप्तशब्द इत्यर्थः~। तच्च {\qt अथ} शब्दादिकम्~। तत्राथशब्दस्यानन्तर्यवाचित्वं शक्त्यैव, क्त्वाप्रत्ययादीनां तु पूर्वकालादिवाचिनामपेक्षानुरोधात्क्रमपरत्वं लक्षणया, अर्थादिषु कल्प्यशब्दस्यैव क्रमबोधकत्वात् क्लृप्तेति विशेषणम्~। श्रुतिं विभजते {\br तच्चेति~।}  क्रमपरवचनं चेत्यर्थः~। तत्र केवलक्रमपरं वचनमुदाहरति {\br तत्रेत्यादिना~।} तत्र, द्वयोर्द्वयोर्मध्य इत्यर्थः~। वेदः दर्भमुष्टिविशेषः~। वेदिः आहवनीयगार्हपत्ययोर्मध्ये चतुरङ्गुलं निखातं भूतलं हविर्निधानस्थानविशेषरूपम्~। तस्य केवलक्रमपरत्वे हेतुमाह\textendash {\br वेदीत्यादिना~।} दर्शपूर्णमासयोर्हविरधिवासनोत्तरं वेदिकरणविधिवाक्येनैव वेदिकरणस्य प्राप्तत्वात्तदनुवादेन क्त्वाप्रत्ययोक्तक्रममात्रमत्र विधीयत इति भावः~। अत्र
क्त्वाप्रत्ययोक्तस्य क्रमस्य वाक्यादेव वेदिकरणाङ्गत्वं तत्त्वेनैव तद्विधिरिति बोध्यम्~। क्रमविशिष्ट पदार्थपरं वचनमुदाहरति\textendash {\br वषट्कर्तुरिति~।} {\br ननु} भक्षस्य
\newpage
%%%%%%%%%%%%%%%%%%%%%%%%%%%%%%%%%%%%% 
\fancyhead[RE]{[ श्रुति\textemdash\ }
{\bl\noindent {\qtl भक्ष} इति तु क्रमविशिष्टपदार्थपरम्~। एकप्रसरताभङ्गभयेन भक्षानुवादेन क्रममात्रस्य विधातुमशक्यत्वात्~। सेयं श्रुतिरितर\\} 
\hrule
\vspace{3mm}
\noindent
कथंचित्संभवप्राप्तिकत्वसंभवात्प्रथमशब्दोक्तक्रममात्रस्यात्रापि विधानं भविष्यतीत्याशङ्क्य तत्र हेतुमाह\textendash {\br एकेत्यादिना~।} तथा च
प्राथम्यविशिष्टैकभक्षपदार्थप्रसरभावस्य विशिष्टोपस्थितिरूपस्य भङ्गभिया न क्रममात्रं विधातुं शक्यते~। अन्यथा यो वषट्कर्तुर्भक्षः स प्रथम इत्युपस्थितिः स्यान्न तु वषट्कर्तृत्वोपाधिविशिष्टसंबन्ध्यभिन्नः प्राथम्यविशिष्टभक्ष इति भावः~। यद्वा, एकप्रसरताभङ्ग एकवाक्यताभङ्गो बोध्यः~। भक्षस्याप्यनेनैव विधानाभ्युपगमात्~। तथा चानेन विहितभक्षानुवादेन प्रथमशब्दोपात्तक्रमविधान एकवाक्यताभङ्गेनावृत्त्यात्मको वाक्यभेदः स्यादिति भावः~। अत्रापि क्रमविशिष्टभक्षविधानात्प्रथमपदोक्तक्रमस्य
वाक्याद्भक्षाङ्गत्वमिति ध्येयम्~। {\br अत्रेदं बोध्यम्} यत्र धात्वर्थस्य क्लृप्तप्रमाणेन प्राप्त्यभावेऽपि कथंचित्संभवत्प्राप्तिकस्य पुनर्विधानेन किंचित्प्रयोजनं विधेयान्तरं च नान्यत्किंचित्तत्र क्रम एव श्रुत्युक्तो विधीयते~। यथा सत्रात्मके द्वादशाहे, अध्वर्युर्गृहपतिं दीक्षयित्वा ब्रह्माणं दीक्षयति, तत उद्गातारं, ततो होतारं, ततस्तं प्रतिप्रस्थाता
दीक्षयित्वाऽर्घिनो दीक्षयति ब्राह्मणाच्छंसिनं ब्रह्मणः, प्रस्तोतारमुद्गातु, मैत्रावरुणं होतुस्ततस्तं नेष्टा दीक्षयित्वा तृतीयिनो दीक्षयति आग्नीध्रं ब्रह्मणः, प्रतिहर्तारमुद्गातु, रच्छावाकं
होतुस्ततस्तमुन्नेता दीक्षयित्वा पादिनो दीक्षयति पोतारं ब्रह्मणः, सुब्रह्मण्यमुद्गातु, र्ग्रावस्तुतं होतुस्ततस्तमन्यो ब्राह्मणो दीक्षयति ब्रह्मचारी वाचार्यप्रेषितः इति श्रुते वाक्ये, अत्र हि न दीक्षायाः स्वरूपेण तत्तत्संस्कारकत्वेन वा विधिरतिदेशप्राप्तत्वात्~। प्रकृतौ हि यजमानसंस्कारार्था दीक्षातिदेशेनैव सत्रे प्राप्यते~। सत्रे च ये यजमानास्त्र ऋत्विज इति वचनेन ऋत्विक्कार्योद्देशेन यजमानविधानाद्ब्रह्मादीनां यजमानत्वेनैव प्रतिप्रधानगुणावृत्तिन्यायेन तत्तत्संस्कारकत्वप्राप्तेः, ततश्च दीक्षारूपं धात्वर्थमनूद्याख्यातेन, क्त्वाप्रत्ययोक्तस्ततःशब्दोक्तश्च क्रमो विधीयते~। अत्र चायमर्थः\textendash अध्वर्युर्यजुर्वेदोक्तं करोति तस्य पुरुषास्त्रयः प्रतिप्रस्थाता नेष्टोन्नेता चेति~। एते चत्वारो दीक्षयितारः~। ब्रह्मा वेदत्रयोक्तस्य प्रत्यवेक्षणं करोति, तस्य पुरुषास्त्रयः ब्राह्मणाच्छंसी अग्नीत् पोता चेति~। उद्गातोद्गानं करोति, तस्य पुरुषास्त्रयः प्रस्तोता प्रतिहर्ता सुब्रह्मण्यश्चेति~। होता शंसनं करोति, तस्य पुरुषास्त्रयः
\newpage
%%%%%%%%%%%%%%%%%%%%%%%%%%%%%%%%%%%%%%%
\fancyhead[LO]{लक्षणम् ]}
{\bl\noindent प्रमाणापेक्षया बलवती~। तेषां
वचनकल्पनद्वारा क्रमप्रमाणत्वात्~। अत एवाश्विनग्रहस्य\footnotemark\ पाठक्रमात्तृतीयस्थाने ग्रहणप्रसक्तौ}\\
\hrule
\vspace{3mm}
\noindent
मैत्रावरुणोऽच्छावाको ग्रावस्तुच्चेति~। चतुर्षु  वर्गेषु ये प्रथमास्ते दक्षिणां संपूर्णामाप्नुवन्ति~। ये द्वितीयास्ते तदर्धं प्राप्नुवन्ति ततोऽर्धिन उच्यन्ते~। ये तृतीयास्ते तृतीयांशं प्राप्नुवन्तीति तृतीयिनः~। ये चतुर्थास्ते चतुर्थांशं प्राप्नुवन्तीति पादिनः~। तानेतानुक्तक्रमेण स स पुरुषः करोतीति~। किंच {\qt इतरमन्यस्तेषां यतो विशेषः स्यात्} इति न्यायेनाध्वर्युपुरुषाणामाद्यः प्रतिप्रस्थातैवाध्वर्युदीक्षायां प्राप्नोति~। ब्राह्मणाच्छंस्यादिदीक्षासु त्वध्वर्योर्न पूतः पावयेदिति वचनेन सत्रप्रकरणपठितेन दीक्षासु दीक्षाख्यसंस्काररहितपुरुषकर्तृविधायकेन पर्युदासात्प्रतिप्रस्थातृप्राप्तिः सुलभैव~। एवं प्रतिप्रस्थात्रादिदीक्षासु नेष्टुः प्रतिप्रस्थात्रनन्तरस्य नेष्ट्रादिदीक्षासु चोन्नेतुर्नेष्ट्रनन्तरस्य प्राप्तिर्न्यायादेवेति न विनेयान्तराशङ्का~। अतः क्रम एवात्राप्राप्तः, स एव तत्तद्दीक्षोद्देशेन विधीयते~। अत एव चैतानि द्वादशवाक्यानिं  श्रौतक्रमविधायकानि
अर्धित्वाद्युद्देश्यतावच्छेदकमङ्गीकृत्य षडेव वा~। उन्नेतृदीक्षावाक्ये तु वैकल्पिकब्रह्मचारिविधानात् हृदयादिन्यायेन\footnotemarkA[1] पाठादेव क्रमसिद्धेस्तत इत्यनुवादः~। निपातत्वाच्च वाशब्दस्य ब्रह्मचारिविशेषणत्वेन न वाक्यभेदः~। ब्राह्मणानामेवार्त्विज्यविधानाद्ब्राह्मण इति चानुवादः~। ब्रह्मचारिणश्चाचार्याधीनत्वस्मृतेराचार्यप्रेषित इत्यप्यनुवादः~। यत्तु {\qt यद्यप्यत्र क्रमस्य वाचकः शब्दो नास्ति तथापि वाक्येन प्रतीयते~}। स च प्रतीयमानः क्रमो मानान्तरेण कर्तव्यतया प्राप्त्यभावादिह विधीयत इति~। तन्न, क्त्वाप्रत्ययादेः क्रमवाचकस्योक्तत्वात्~। अतः क्रमः एवात्राप्राप्तस्तत्तद्दीक्षोद्देशेन विधीयते क्त्वाप्रत्ययोक्तस्ततःपदोक्तश्चेति भाट्टदीपिकाविरोधाच्चेति~।
इदानीमुक्तश्रुतेरर्थपाठादिप्रमाणापेक्षया प्राबल्यमाह\textendash {\br सेयं श्रुतिरिति~।} अर्थादिप्रमाणानां श्रुतिकल्पनद्वारा क्रमे प्रामाण्यादिति~। तत्र हेतुमाह\textendash {\br तेषामित्यादिना~।} ज्योतिष्टोम ऐन्द्रवायवादिग्रहेष्वाश्विननामकग्रहस्तृतीयस्थाने पठितः~। ततश्च तृतीयस्थाने ग्रहणप्रसक्तावपि तस्य दशमस्थानत्वमाश्विनो दशमो गृह्यत इति शब्देनैवाम्नायत इति तत्र गमकमप्याह\textendash {\br अत एवेत्यादिना~।} अत एव, इतरप्रमाणापेक्षया 
\blfootnote{पाठा०\textemdash\ $^{१}$ग्रहणस्य. }
\alfootnote{टिप्प०\textemdash\ $^{1}$हृदयावदानादिन्यायेनेत्यर्थः}
\newpage
%%%%%%%%%%%%%%%%%%%%%%%%%%%%
\fancyhead[RE]{[अर्थक्रमलक्षणम् ]}
{\bl\noindent आश्विनो दशमो गृह्यत इति वचनाद्दशमस्थाने ग्रहणमित्युक्तम्~। \\
\begin{center}\textbf {अर्थक्रमलक्षणम्}\end{center}

यत्र प्रयोजनवशेन क्रमनिर्णयः सोऽर्थक्रमः~। यथा {\qtl अग्निहोत्रं जुहोति, यवागूं पचति} इत्यग्निहोत्रयवागूपाकयोः~। अत्र हि यवाग्वा होमार्थत्वेन तत्पाकः प्रयोजनवशेन पूर्वमनुष्ठीयते~। स चायं पाठक्रमाद्बलवान्~। यथापाठं ह्यनुष्ठाने क्लृप्तप्रयोजनबाधोऽदृष्टार्थत्वं च स्यात्~। न हि होमानन्तरं क्रियमाणस्य पाकस्य किंचिदृष्टं प्रयोजनमस्ति~।}\\
\hrule
\vspace{3mm}
\noindent
श्रुतेः प्राबल्यादेवेत्यर्थः~। {\br इत्युक्तमिति~।} पञ्चमाध्याये चतुर्थपादस्य प्रथमाधिकरण इति शेषः~। पाठो हि न क्रमस्याभिधायकः किं त्वन्यथानुपपत्त्या क्रमं कल्पयति~। दशम इत्येषा श्रुतिस्तु साक्षादेव क्रममभिधत्ते~। ततः पाठादपि श्रुतिः प्रबलेति भावः~।\\

 इदानीमर्थक्रमं लक्षयति\textendash {\br यत्रेत्यादिना~।} प्रयोजनवशेन प्रयोजनानुपपत्त्या, तद्धि व्युत्क्रमेऽनुपपद्यमानं क्रमे प्रमाणमिति भावः~ । तत्रोदाहरणमाह\textendash {\br यथाग्निहोत्रमिति~।} {\qt अग्निहोत्रयवागूपाकयोः} इत्यत्र प्रयोजनवशेन क्रमनिर्णय इत्यनुषङ्गः~। {\br अत्र हीति~।} अत्र प्रकृते,
यवागूहोमयोर्मध्य इति वार्थः~। तत्पाकः पूर्वमनुष्ठीयत इत्यन्वयः, होमादिति शेषः, यवागूः तच्छब्दार्थः~। यवागूपाकस्य पाठक्रमेण पश्चात्करणे पाकसंस्कृताया यवाग्वा होमरूपप्रयोजनस्य {\qt यवाग्वाग्निहोत्रं जुहोति} इति वचनसिद्धस्यानिष्पत्तेः~। पाकस्य च यवागूत्पादकत्वेऽपि तस्य अनुपयुक्तायाः प्रयोजनत्वानुपपत्तेस्तदन्यथानुपपत्त्या पूर्वं पाकः पश्चाद्धोमोऽनुष्ठीयत इति भावः~। अर्थक्रमश्च पाठक्रमाद्बलीयान्भवतीत्याह\textendash {\br स चायमिति~।} यवागूपाकस्य यथापाठमनुष्ठानेऽदृष्टार्थत्वं स्यादिति विपक्षे
बाधकमाह\textendash {\br यथापाठमिति~।} \footnotemarkA[1]क्रमप्रयोजनबाधस्तु दर्शितो यवागूपाकस्येत्यारभ्यास्माभिरिति भावः~। दृष्टमेव किंचित्संभवत्प्राप्तिकं
भवेदित्यत आह\textendash {\br न हीत्यादिना~।} न हि भक्षणमन्तरेण किंचिद्दृष्टप्रयोजनमुपलभ्यते, तच्च न प्रकृतयवाग्वा, अदर्शनात्प्रमाणाभावाच्चेति भावः~।
\alfootnote{टिप्प०\textemdash\ $^{1}$पाठक्रमेत्यर्थः~।}
\newpage
%%%%%%%%%%%%%%%%%%%%%%%%%%%%%%%%%%
\fancyhead[LO]{[पाठक्रमलक्षणम् ]}
\begin{center}
\textbf{पाठक्रमलक्षणम्}    
\end{center}

{\bl पदार्थबोधकवाक्यानां यः क्रमः स पाठक्रमः~। तस्माच्च पदार्थानां क्रम आश्रीयते~। येन हि क्रमेण वाक्यानि पठितानि तेनैव क्रमेणाधीतान्यर्थप्रत्ययं जनयन्ति~। यथाप्रत्ययं च पदार्थानामनुष्ठानम्~। स च पाठो द्विविधः {\al मन्त्रपाठो ब्राह्मणपाठ}-\\}
\hrule
\vspace{3mm}
\noindent
पाठक्रमं लक्षयति\textendash {\br पदार्थेति~। तस्माच्चेति~।} पाठक्रमाच्चेत्यर्थः~। पाठक्रमात्पदार्थक्रममुपपादयति {\br येनेत्यादिना~।} वाक्यानीति पदस्याप्युपलक्षणम्~। पदार्थप्रत्ययस्य क्वात्रोपयोग इत्यत आह\textendash {\br यथाप्रत्ययं चेति~।} पाठक्रमं विभजते {\br स च पाठो द्विविध इति~।}
मन्त्रपाठक्रममुदाहरति\textendash {\br तत्रेत्यादिना~।} तत्र मन्त्रपाठब्राह्मणपाठयोर्मध्य इत्यर्थः~। किंच अग्नीषोमीययागस्तैत्तिरीयब्राह्मणपञ्चमप्रपाठके द्वितीयानुवाके
समाम्नातः {\qt ताभ्यामेतमग्नीषोमीयमेकादशकपालं पूर्णमासे प्रायच्छत्} इति~। आग्नेययागस्तु षष्ठप्रपाठके तृतीयानुवाके आम्नातः {\qt यदाग्नेयोऽष्टाकपालोऽमावास्यायां च पौर्णमास्यां चाच्युतो भवति} इति~। तत्रानुष्ठानस्य ब्राह्मणोक्तविध्यधीनत्वादग्नीषोमीयस्य प्रथममनुष्ठानमित्याशङ्क्य मन्त्रकाण्डे पूर्वं पठिता आग्नेयमन्त्राः~। तथा
हि\textendash हौत्रकाण्डे आज्यभागमन्त्रानुवाकादुत्तरस्मिन्ननुवाके प्रथमम् {\qt अग्निर्मूर्धा} इत्यादिके आग्नेय्यौ याज्यानुवाक्ये आम्नाते~। ततः {\qt प्रजापते नत्वदेतानि} इत्यादिके प्राजापत्ये~। ततो {\qt ऽग्नीषोमासवेदसा} इत्यादिके अग्नीषोमीये~। आध्वर्यवे काण्डेऽपि {\qt अग्नये जुष्टं निर्वपाम्यग्नीषोमाभ्याम्} इत्याग्नेयः पूर्वमाम्नातः~। यजमानकाण्डेऽपि {\qt अग्नेरहं देवयज्ययान्नादो भूयासम्} इत्याग्नेयस्य पश्चाद् {\qt अग्नीषोमयोहं देवयज्यया वृत्रहा भूयासम्} इत्याम्नायते~। मन्त्रक्रमः प्रबलः मन्त्रैः
स्मृत्वा पश्चादनुष्ठेयत्वाद्ब्राह्मणपाठस्त्वप्राप्तपदार्थे विधिनापि चरितार्थः~। अतोऽनुष्ठानस्मरणायैवोत्पन्नान्मन्त्रान्बाधितुं नालमिति मन्त्रक्रमेणाग्नेयस्यैव प्रथममनुष्ठानमिति समाधानमभिप्रेत्य वा मन्त्रपाठस्य ब्राह्मणपाठाद्बलीयस्त्वमाह\textendash {\br स चायमित्यादिना~।} किंच पाठयोस्तु मन्त्रब्राह्मणगतयोर्मन्त्रपाठो बलीयान् , न तु ब्राह्मणपाठः तस्योत्पत्तिविनियोगविधिगतत्वेन प्रथमोपस्थितत्वेऽपि पाठस्य स्मारकक्रमविधयैव क्रमनियामकत्वोक्तेर्मन्त्रसत्त्वे च तस्यैव स्मारकतया विधानोपयुक्तस्यासमर्थस्य च विधेः स्मारकत्वा-
\newpage
%%%%%%%%%%%%%%%%%%%%%%%%%%%%%%%%%%%%
\fancyhead[RE]{[पाठक्रमलक्षणम् ]}
{\bl\noindent श्चेति~। तत्राग्नेयाग्नीषोमीययोस्तत्तद्याज्यानुवाक्यानां पाठाद्यःक्रम आश्रीयते स मन्त्रपाठात्~। स चायं मन्त्रपाठो ब्राह्मणपाठाद्बलीयान्, अनुष्ठाने
ब्राह्मणवाक्यापेक्षया मन्त्रपाठस्यान्तरङ्गत्वात्~। ब्राह्मणवाक्यं हि प्रयोगाद्बहिरेवेदं कर्तव्यमित्यवबोध्य कृतार्थम्~। मन्त्राः पुनः प्रयोगकाले व्याप्रियन्ते, अनुष्ठानक्रमस्य स्मरणक्रमाधीनत्वात्~। तत्क्रमस्य च मन्त्रक्रमाधीनत्वाद् अन्तरङ्गोऽयं मन्त्रपाठ इति~। प्रयाजानां {\qtl समिधो यजति,तनूनपातं यजति} इत्येवंविधपाठक्रमाद्यः क्रमः स
ब्राह्मणपाठक्रमात्~। यद्यपि ब्राह्मणवाक्यान्यर्थं\\}
\hrule
\vspace{3mm}
\noindent
भावान्मन्त्रपाठक्रम एव बलीयान्~। तेन च याज्यानुवाक्यादिमन्त्रपाठक्रमादाग्नेयस्य प्रथमानुष्ठानं पश्चाच्चोपांशुयाजोत्तरमग्नीषोमीयस्य, न तु ब्राह्मणपाठकमादग्नीषोमीयस्य
प्रथमं, पश्चादुपांशुयाजोत्तरमाग्नेयस्येत्यभिप्रेत्य मन्त्रपाठस्य ब्राह्मणपाठाद्बलीयस्त्वमाह\textendash {\br स चायमित्यादिना~। याज्यानुवाक्यानामिति~।}\footnotemarkA[1] यजेति प्रैषानन्तरमृग् या ब्रह्मणा समुच्चार्यते सा याज्येत्युच्यते~। अनुब्रूहीति प्रैषानन्तरमृग् या तेनैव समुच्चार्यते सानुवाक्येत्युच्यत इत्यर्थः~। तस्य तद्बलीयस्त्वे हेतुमाह\textendash {\br अनुष्ठान इत्यादिना~।} मन्त्रपाठस्य ब्राह्मणपाठादनुष्ठानेऽन्तरङ्गत्वमुपपादयति {\br ब्राह्मणवाक्यमित्यादिना~। स्मरणक्रमाधीनत्वादिति~।} अनुष्ठेयपदार्थस्मरणक्रमाधीनत्वादित्यर्थः~। तत्क्रमस्येति~। अनुष्ठेयपदार्थस्मरणक्रमस्येत्यर्थः~। {\br मन्त्रक्रमाधीनत्वादिति~।} अनुष्ठेयपदार्थस्मारकमन्त्रकमाधीनत्वादित्यर्थः~। प्रयाजानामनुष्ठानक्रमः {\qt समिधो यजति, तनूनपातं यजति, इडो यजति, बर्हिर्येजति, स्वाहाकारं यजति} इति ब्राह्मणपाठक्रमात्स्वीक्रियते इत्याह\textendash {\br प्रयाजानामित्यादिना~।  ननु} {\qt समिधोऽग्न आज्यस्य व्यन्तु} इत्यादिमन्त्राणां प्रयाजक्रमस्मारकाणां
सत्त्वेन कथं तेषां ब्राह्मणवाक्यक्रमात्क्रमः स्वीक्रियते, तद्वाक्यानां विधानमात्रे चरितार्थत्वादित्याशयेनाशङ्कते {\br यद्यपीति~।} मन्त्रपाठस्यान्यादृशत्वेन वा, मन्त्राणां देवतामात्रस्मारकत्वेन कर्मस्मारकत्वाभावाद्वा, प्रयाजक्रमो ब्राह्मणपाठक्रमादेवेत्याशयेन-
\alfootnote{टिप्प०\textemdash\ $^{1}$सिंहावलोकितन्यायेन याज्यानुवाक्यपदे व्याचष्टे~।}
\newpage
%%%%%%%%%%%%%%%%%%%%%%%%%%%%%
\fancyhead[LO]{[ स्थानलक्षणम् ]}
{\bl\noindent विधाय कृतार्थानि तथापि प्रयाजादीनां क्रमस्मारकान्तरस्याभावात्तान्येव क्रमस्मारकत्वेन स्वीक्रियन्ते~।}
\begin{center}
\textbf{स्थानलक्षणम्}    
\end{center}

{\bl स्थानं नामोपस्थितिः~। यस्य हि देशे योऽनुष्ठीयते तत्पूर्वतने पदार्थे कृते स एव प्रथममुपस्थितो भवतीति युक्तं तस्य प्रथममनुष्ठानम्~। अत एव
साधस्के\textendash अग्नीषोमीय-सवनीया-नुब- }\\
\hrule
\vspace{3mm}
\noindent
परिहरति  {\br तथापीति~।} तथा च येन क्रमेण ब्राह्मणवाक्यान्यधीतानि तेनैक  क्रमेणार्थस्मरणं जनयन्तीति युक्तं तेषां तेनैव क्रमेणानुष्ठानमिति भावः~। वस्तुतस्तु प्रयाजक्रमो न ब्राह्मणपाठक्रमात्स्वीक्रियते {\qt समिधोऽग्न आज्यस्य व्यन्तु} इत्यादिभिः क्रमप्रकरणप्राप्तैर्मन्त्रैर्देवता गुणत्वेन समर्प्यन्त इति नवमतन्त्ररत्नविरोधप्रसङ्गात्~। अन्यथा
मन्त्राणामन्यादृशक्रमत्वे तदनुपपत्त्यापत्तेः, किं तु मन्त्रपाठक्रमात्क्रम एव, ब्राह्मणपाठक्रमात्क्रमस्तु यत्रार्थस्मारका मन्त्रा न सन्त्येव, तेषामेव, यथा तूष्णी विहितानां कर्मणां क्रमो
ब्राह्मणपाठक्रमाद्भवति तत्र तेषामेव प्रयोगसमवेतार्थस्मारकत्वात्~। प्रयाजोदाहरणं तु कृत्वाचिन्तया, तत्र ब्राह्मणवाक्यानां प्रयोगसमवेतार्थस्मारकत्वाभावात्~। तथा चार्थवादपादे वार्तिकवचनम् {\qt प्रयाजादिवाक्यान्यर्थं समर्प्य चरितार्थानि स्वरूपसंस्पर्शे सत्यपि प्रयोज्यतां न प्रतिपद्यन्ते} इति, तस्मान्मन्त्रक्रमादेव प्रयाजक्रम इति सिद्धमिति ध्येयम्~।\\

 इदानीं स्थानं लक्षयति\textendash {\br स्थानं नामोपस्थितिरिति~।} प्रकृतौ नानादेशानां पदार्थानां विकृतौ चोदकवचनादेकस्मिन्देशेऽनुष्ठाने कर्तव्ये यस्य देशे तेऽनुष्ठीयन्ते तस्य प्रथममनुष्ठानमितरयोस्तु पश्चादयं यः क्रमः स स्थानक्रमः तेन चोपस्थितिविशेषेण योऽनुष्ठानक्रमः स एव स्थानक्रम इत्युच्यत इति भावः~। उपस्थितिं व्यनक्ति {\br यस्येत्यादिना~। यस्येति~।} ज्योमिष्टोमादिप्रकृतेः साद्यस्क्रादिविकृतौ चोदकप्राप्तस्य सवनीयादेरित्यर्थः~। {\br य इति~।} अग्नीषोमीयानुबन्ध्यादिरित्यर्थः~। {\br तत्पूर्वतन इति~।} तस्मात्सवनीयादेः पूर्वतन आश्विनग्रहणादौ पदार्थे कृते सतीत्यर्थः~। {\br स एवेति~।} सवनीयादिरेवेत्यर्थः~। {\br प्रथममिति~।} अग्नीषोमीयानुबन्ध्यापेक्षया प्रथममित्यर्थः~। {\br तस्येति~।}
\newpage
%%%%%%%%%%%%%%%%%%%%%%%%%%%%%%%%
\fancyhead[RE]{[ स्थान}
{\bl\noindent न्ध्यानां सवनीयदेशे सहानुष्ठाने कर्तव्ये आदौ सवनीयपशोरनुष्ठानमितरयोः पश्चात्~। तस्मिन्देशे आश्विनग्रहणानन्तरं सवनीयस्यैव प्रथममुपस्थितिः~। तथा
हि ज्योतिष्टोमे त्रयः पशुयागा:अग्नीषोमीयः सवनीय आनुबन्ध्यश्चेति~। ते च भिन्नदेशा:अग्नीषोमीय औपवसथ्येऽह्नि, सवनीयः सुत्याकाले, आनुबन्ध्यस्त्वन्ते~। साद्यस्को नाम याग\footnotemarkA[1]विशेषः~। स चाव्यक्तत्वाज्ज्योतिष्टोमविकारः~। अतस्ते त्रयोऽपि पशुयागाः साद्यस्के चोदकप्राप्ताः~। तेषां च तत्र साहित्यं श्रुतं {\qtl सह पशूनालभेत} इति~। तच्च साहित्यं सवनीयदेशे, तस्य प्रधानप्रत्यासत्तेः, स्थानातिक्रमसाम्याच्च~। सवनीयदेशे ह्यनुष्ठानेऽग्नीषोमीयानुबन्ध्ययोः \\}
\hrule
\vspace{3mm}
\noindent
सवनीयादेरेवेत्यर्थः~। {\br अत एवेति~।} प्रथमोपस्थितप्रथमानुष्ठानस्य युक्तत्वादेवेत्यर्थः~। {\br इतरयोरिति~।} अग्नीषोमीयानुबन्ध्ययोरित्यर्थः~। {\br तस्मिन्देश इति~।} त्रयाणामपि पशूनां विकृतौ प्राप्तसवनीयदेश इत्यर्थः~। आश्विनग्रहणानन्तरमित्यत्र सवनीयस्यैव विकृतौ प्रधानप्रत्यासत्तिबलाचोदकप्राप्तत्वेनोपस्थितियोग्यत्वादिति शेषः~।
{\br प्रथममिति~।} अग्नीषोमीयानुबन्ध्यापेक्षया प्रथममित्यर्थः~। विकृतावाश्विनग्रहणानन्तरमेव सवनीयस्थानप्रदर्शनाय प्रथमं प्रकृतौ तत्स्थानं प्रदर्शयति {\br तथा हीत्यादिना~।} औपवसथ्येऽह्नीत्युक्तार्थ एव~। {\br अन्त इति~।} अवभृथादूर्ध्वकाल इत्यर्थः~। {\br अव्यक्तत्वादिति~।} स्वार्थचोदितदेवतारहितत्वादित्यर्थः~। तदुक्तं न्यायप्रकाशे {\qt अव्यक्तत्वं च स्वार्थचोदितदेवताराहित्यम्} इति~। {\br अत इति~।} ज्योतिष्टोमविकारत्वादित्यर्थः~। {\br तत्रेति~।} साद्यस्क इत्यर्थः~। {\br तच्चेति~।}  श्रुतं चेत्यर्थः~। तस्येति~। सवनीयस्येत्यर्थः~। {\br प्रधानेति~।} यथा प्रकृतिभूते ज्योतिष्टोमे सुत्याकालिकः सवनीयः प्रधानप्रत्यासन्नस्तथैव तद्विकृतिविशेषे साद्यस्केऽपि तात्कालिकः स प्रधानप्रत्यासन्न एवेति भावः~। सवनीयदेशे साहिये त्रयाणां स्वस्वस्थानातिक्रमसाम्यं हेत्वन्तरमाह\textendash {\br स्थानेति~।} तदेव स्ववस्थानातिक्रमसाम्यं प्रदर्शयति\textendash {\br सवनीयदेश} इत्यारभ्य {\br तथा- }
\alfootnote{टिप्प०\textemdash\ $^{1}$सोमयागविशेषः~।}
\newpage
%%%%%%%%%%%%%%%%%%%%%%%%%%%%%%%%%%%%%%%%%
\fancyhead[LO]{लक्षणम् ]}
{\bl\noindent स्वस्वस्थानातिक्रमो\blfootnote{पाठा०\textemdash\ $^{१}$स्वस्वस्थानातिक्रममात्रं भवति.}\footnotemark\ भवति (प्रधानप्रत्यासत्तिलाभश्च~।~)
अग्नीषोमीयदेशे त्वनुष्ठाने सवनीयस्य स्वस्थानातिक्रममात्रम्~। अग्नीषोमीयस्य सवनीयस्थानातिक्रमः, अनुबन्ध्यस्य तु स्वस्थानातिक्रमः, सवनीयस्थानातिक्रमश्च स्यादिति त्रयाणां स्वस्वस्थानातिक्रमः~। एवमनुबन्ध्यदेशेऽग्नीषोमीयस्य द्रष्टव्यः स्थानातिक्रमः~। तथा च सवनीयदेशे सर्वेषामनुष्ठाने कर्तव्ये सवनीयस्य प्रथममनुष्ठानम्~। आश्विनग्रहणानन्तरं
हि सवनीयदेशः~। प्रकृता {\qtl वाश्विनग्रहं \footnote{गृहीत्वा. }कृत्वा त्रिवृता यूपं परिवीय आग्नेयं सवनीयं पशुमुपा-\\}}
\hrule
\vspace{3mm}
\noindent
{\br चेति} प्राक्तनेन ग्रन्थेन~। {\br स्वस्वस्थानातिक्रमो भवतीति~।} यथा प्रकृतौ तथैव विकृतावपि चोदकप्राप्तस्याग्नीषोमीयस्यौपवसथ्याहोरूपस्य स्वस्थानमात्रस्यातिक्रमो भवति, आनुबन्ध्यस्य च स्वान्तस्थानमात्रस्यातिक्रमो भवतीत्यर्थः~। सवनीयस्य तु सुत्याकालरूपस्वस्थानस्य नातिक्रमो भवतीत्यत्र लाघवमिति भावः~। {\br इति त्रयाणां स्वस्वस्थानातिक्रम इति~।} अग्नीषोमीयस्य देश औपवसथ्येऽह्नि सर्वेषामनुष्ठानेऽग्नीषोमीयस्य सवनीयप्रधानसोमप्रत्यासत्तिबलात्साहित्यविधिप्राप्तसुत्याकालरूपसवनीयस्थानरूपस्वस्थानातिक्रमो भवतिआनुबन्ध्यस्य तु तादृशसवनीयस्थानरूपस्वस्थानातिक्रमः
स्वीयान्तस्थानातिक्रमो भवतीति त्रयाणां स्वस्वस्थानातिक्रमः~। आनुबन्ध्यस्य देशे तु तेषामनुष्ठानेऽग्नीषोमीयस्यौपवसथ्याहोरूपस्य स्वस्थानस्यातिक्रमस्तादृशसवनीयस्थानरूपस्वस्थानस्यातिक्रमश्च भवति, आनुबन्ध्यस्य तु निरुक्तसवनीयस्थानरूपस्वस्थानस्यातिक्रमः, सवनीयस्य तु
स्वस्थानमात्रस्यातिक्रमश्च भवतीति त्रयाणां स्वस्वस्थानस्यातिक्रम इत्यर्थः~। {\br तथा चेति~।} उक्तयुक्त्या सवनीयदेशे त्रयाणामनुष्ठाने
कर्तव्ये चेत्यर्थः~। वैकृतसवनीयस्थाननिर्णयोपायतया प्राकृतं सवनीयस्थानं श्रुत्या प्रदर्शयति\textendash {\br प्रकृतावित्यादिना~।} त्रिवृता त्रिगुणितरज्ज्वा
परिवीय परिवेष्टनं कृत्वा आश्विनः सोमग्रहः तद्ग्रहणानन्तरं ज्योतिष्टोमे सव- 
\lfoot{६ अ०}
\newpage
%%%%%%%%%%%%%%%%%%%%%%%%%%%%
\lfoot{}
\fancyhead[RE]{[मुख्यक्रम\textemdash\ }
{\bl {\qtl\noindent करोति} इत्याश्विनग्रहणानन्तरं सवनीयो विहित इति साद्यस्केऽप्याश्विनग्रहणे कृते सवनीय एवोपस्थितो भवति~।अतो युक्तं तस्य स्थानात्प्रथममनुष्ठानम् , इतरयोस्तु पश्चादित्युक्तम्~।}
\begin{center}
\textbf{मुख्यक्रमलक्षणम्}    
\end{center}

{\bl प्रधानक्रमेण योऽङ्गानां क्रमः स मुख्यक्रमः~। येन हि}\\
\hrule
\vspace{3mm}
\noindent
नीयो विहित इति भावः~। {\br इति~।} यतः प्रकृतावाश्विनग्रहणानन्तरं स विहितोऽतो हेतोरित्यर्थः~। {\br आश्विनग्रहणेति~।} अत्राप्याश्विनग्रहणानन्तरमेव तस्य चोदकप्राप्तत्वेव तदनन्तरमेव तदुपस्थितिरिति भावः~। {\br तस्येति~।} सवनीयस्येत्यर्थः~। {\br स्थानादिति~।} उपस्थितेरित्यर्थः~। {\br इतरयोरिति~।} अग्नीषोमीयानुबन्ध्ययोरित्यर्थः~। {\br ननु} तयोः पश्चादनुष्ठानेऽपि कस्य प्रथममनुष्ठानं कस्य पश्चादित्यनिर्णये प्रयोगविक्षेपापत्तिरित्यत आह\textendash {\br उक्तमिति~।}
पञ्चमाध्यायस्य प्रथमे पाद इति शेषः~। तत्र हि स्थानात्सवनीयस्य प्राथम्ये निश्चिते स्थानभ्रष्टयोस्तयोरग्नीषोमीयस्य प्रथममनुष्ठानमानुबन्ध्यस्य पश्चादित्युक्तमित्यर्थः~। उक्तं हि {\br न्यायमालाविस्तरे}\textendash साद्यस्कनामकः कश्चित्सोमयागः, तत्र श्रूयते सहपशूनालभत इति~। प्रकृतावग्नीषोमीयपशुरौपवसथ्ये दिने आलभ्यते, सवनीयपशुः सुत्यादिने
प्रातःसवन आश्विनग्रहादूर्ध्वमालभ्यते, आनुबन्ध्यः पशुरवभृथादूर्ध्वमालभ्यते, इह तु त्रयोऽपि पशवः सहालब्धव्याः; सोऽयं सहालम्भः सुत्यादिन आश्विनग्रहादूर्ध्वं सवनीयस्थाने
भवतीत्येतदवश्यमभ्युपेतव्यं, तथा सति प्रधानसोमप्रत्यासत्तिलाभादिति~। सवनीयो हि स्वस्थान एव वर्तते, आश्विनग्रहसमीपस्य सवनीयस्थानत्वात्~। आश्विने गृहीते सति सवनीय एव बुद्धिस्थो भवति, प्रकृतौ तदानन्तर्यस्य, क्लृप्तत्वात्~। ततः सवनीयस्य प्राथम्ये स्थानानिश्चिते सति स्थानभ्रष्टयोरग्नीषोमीयानुबन्ध्ययोः प्रकृताविव पूर्वोत्तरभावो द्रष्टव्य
इति~। यद्वा {\br उक्तमिति~।} आदौ सवनीयपशोरनुष्ठानमितरयोः पश्चादित्यत्र पूर्वत्रोक्तमित्यर्थः~।\\

 अथ मुख्यक्रमं लक्षयति\textendash {\br प्रधानक्रमेणेति~।} तेनैव क्रमेणेत्यत्र यदेति शेषस्तदेत्यनुरोधात्~। यत्र ह्यनेकेषां साङ्गानां प्रधानानां सहकर्तव्यता
तत्र प्रयोगविधिनाङ्गप्रधानयोः साहित्यावगतावपि प्रधानान्तरसाहित्यानुरोधेन याव- 
\newpage
%%%%%%%%%%%%%%%%%%%%%%%%%%%%%%%%%%
\fancyhead[LO]{लक्षणम् ]}
{\bl\noindent क्रमेण प्रधानानि क्रियन्ते, तेनैव क्रमेण तेषामङ्गान्यनुष्ठीयन्ते चेत् तदा सर्वेषामङ्गानां स्वैः स्वैः प्रधानैस्तुल्यं व्यवधानं भवति~। व्युत्क्रमेणानुष्ठाने
केषांचिदङ्गानां स्वैः प्रधानैरत्यन्तमव्यवधानं केषांचिदत्यन्तं व्यवधानं स्यात् , तच्चायुक्तम्~। प्रयोगविध्यवगतसाहित्यबाधापत्तेः~। अतः प्रधानक्रमोऽप्यङ्गक्रमे हेतुः~। अत एव प्रयाजशेषेणादावाग्नेयहविषोऽभिधारणं पश्चादैन्द्रस्य दध्नः, आग्नेयैन्द्रयागयोः पौर्वापर्यात्~। एवं च द्वयोरभिधारणयोः स्वस्वप्रधानेन तुल्यमेकान्तरितं व्यवधानं, व्युत्क्रमेणाघारे त्वाग्नेयहविरभिधारणाग्नेययागयोरत्यन्तमव्यवधानम्, ऐन्द्रदध्यभिघारणैन्द्रयागयोर्द्व्यन्तरितं व्यवधानं तच्चा\footnotemark युक्तमित्युक्तमेव~।}\\
\hrule
\vspace{3mm}
\noindent
दनुज्ञातव्यवधानस्वीकारेऽपि तदधिकव्यवधाने प्रमाणाभावात्प्रधानप्रत्यासत्त्यनुग्रहाय मुख्यक्रमेणैवाङ्गक्रमनियमः~। अत एव प्रवृत्तौ अङ्गनिरूपितप्रत्यासत्त्यनुग्रहो बीजं, मुख्यक्रमे तु प्रधाननिरूपितप्रत्यासत्त्यनुग्रहो बीजमिति तयोर्भेद इति भावः~। प्रधानक्रमव्युत्क्रमेणाङ्गानुष्ठाने बाधकमाह\textendash {\br व्युत्क्रमेणेत्यादिना~।}
तत्रापीष्टापत्तिमाशङ्क्य प्रयोगविध्यवगततत्साहित्यबाधापत्तिरूपमनिष्टं बाधकमाह\textendash {\br तच्चायुक्तमिति~।} केषांचिदङ्गानां तैरत्यन्तमव्यवधानं
केषांचिदत्यन्तव्यवधानं चेति शेषः~। {\br अभिघारणमिति~।} क्षरद्धृतेनाभिषेक इत्यर्थः~। {\br एवं चेति~।} मुख्यक्रमेण हविरभिघारणरूपाङ्गक्रमे चेत्यर्थः~।
{\br एकान्तरितं व्यवधानमिति~।} आग्नेयहविरभिघारणाग्नेययागयोरैन्द्रहविरभिघारणेन व्यवधानादैन्द्रहविरभिघारणैन्द्रयागयोश्चाग्नेययागेन व्यवधानादित्येकान्तरितं व्यवधानमित्यर्थः~। तथा चादावाग्नेयहविरभिघारणं, तत ऐन्द्रस्य हविषोऽभिघारणं तत आग्नेययागः ततश्चैन्द्रयाग इत्येव क्रमो मुख्यक्रमात्सिद्धो भवतीति भावः~। यदि त्वादावैन्द्रहविषोऽभिघारणं तत आग्नेयहविषस्तत्क्रियते ततश्च याज्यानुवाक्याक्रमवशादाग्नेययागस्यानुष्ठानं तत ऐन्द्रयागस्यानुष्ठानमिति क्रमः स्वीक्रियते तदा कस्यचिदत्यन्तमव्यवधानं कस्यचिदत्यन्त-
\blfootnote{पाठा०\textemdash\ $^{१}$तच्चायुक्तमेव.}
\newpage
%%%%%%%%%%%%%%%%%%%%%%%%%%%%%%%%%%%%%
\fancyhead[RE]{[मुख्यक्रमलक्षणम् ]}
{\bl\noindent स च मुख्यः क्रमः पाठक्रमाद्दुर्बलः~। मुख्यक्रमो हि प्रमाणान्तरसापेक्षप्रधानक्रमप्रतिपत्तिसापेक्षतया विलम्बितप्रतिपत्तिकः~। पाठक्रमस्तु निरपेक्षस्वाध्यायपाठक्रममात्रसापेक्षतया न तथेति बलवान्~। स चायं मुख्यः क्रमः प्रवृत्तिक्रमाद् बलवान्~। प्रवृत्तिक्रमे हि बहूनामङ्गानां प्रधानविप्रकर्षात् , मुख्यक्रमे तु संनिकर्षात्~।}\\
\hrule
\vspace{3mm}
\noindent
व्यवधानं च स्यात् , तच्चायुक्तं,   प्रयोगविध्यवगततत्साहित्य-बाधापत्तेरित्यभिप्रेत्याह \textendash {\br व्युत्क्रमेणेत्यादिना~। उक्तमेवेति~।} {\qt तच्चायुक्तं प्रयोग} इत्यादौ दूषणमुक्तमेवेत्यर्थः~। इदानीं पाठकमान्मुख्यक्रमस्य दौर्बल्यमाह\textendash {\br स चेति~।} तत्र हेतुमाह\textendash {\br मुख्यक्रमो हीति~।} यतो मुख्यक्रमः प्रमाणान्तरसापेक्षा या प्रधानक्रमस्य प्रतिपत्तिस्तत्सापेक्षत्वेन विलम्बितप्रतिपत्तिकः, अतः पाठक्रमाद्दुर्बल एवेत्यर्थः~। किंच निरपेक्षो यः स्वाध्यायपाठक्रमस्तन्मात्रसापेक्षत्वेन यतो न दुर्बलः पाठक्रमः, अतो बलवानिति पाठक्रमस्य ततो वैषम्यमाह\textendash {\br पाठक्रम स्त्विति~।} अत्रेदं बोध्यम्\textendash दर्शपूर्णमासयोरुपांशुयाजोऽग्नीषोमीयश्चेत्येतदुभयं पौर्णमास्यामाम्नातम्~। तत्रोपांशुयाजस्याज्यं द्रव्यम् , आज्यस्य धर्मा उत्पवनचतुगृहीतत्वादयः, अग्नीषोमीयस्य पुरोडाशो द्रव्यं, तस्य धर्मा निर्वापावघातादयः, तत्र चायं पूर्वपक्षः {\qt मुख्यौ यागावुपांशुयाजाग्नीषोमीयौ पूर्वोत्तरभाविनौ भवतः, तथा च सति अङ्गक्रमस्य प्रधानक्रमेणैवाश्रयणीयत्वात्प्रथममाज्यधर्माणामेवानुष्ठानं न निर्वापादीनामिति~}~। तत्र सिद्धान्तः {\qt औषधधर्मा निर्वापादयः पूर्वमाम्नाताः}~। आज्यधर्मास्तु पश्चात् , तत्र मुख्यक्रमप्रयुक्तमाज्यधर्माणां प्राथम्यं बाधित्वा पाठक्रमानुरोधेनौषधधर्मा एव प्रथमतोऽनुष्ठेयाः, पाठक्रमो हि वैदिकैः शब्दैः सहसा प्रतीयते, मुख्यक्रमानुसारी तु क्रम उपपत्त्या
कल्पनीयः~। तस्मादग्नीषोमीयपुरोडाशार्था औषधधर्माः प्रथममनुष्ठेयाः, आज्यधर्मास्तु पश्चादिति~। प्रवृत्तिक्रमापेक्षया तु मुख्यक्रमस्य प्राबल्यमेवेत्याह\textendash {\br प्रवृत्तीति~।} तत्र हेतुमाह\textendash {\br प्रवृत्तिकमे हीति~।} अत्रेदं बोध्यम् -दर्शपूर्णमासयोरादावाग्नेययागस्यानुष्ठानं, ततः सांनाय्ययागस्य, सांनाय्यधर्माश्च केचिद्वत्सापाकरणदोहनादयः पूर्वमेवा-
\newpage
%%%%%%%%%%%%%%%%%%%%%%%%%%%%%%%%%
\fancyhead[LO]{[प्रवृत्तिक्रमलक्षणम्]}
\begin{center}
 \textbf{प्रवृत्तिकमलक्षणम्}   
\end{center}
 
{\bl सहप्रयुज्यमानेषु प्रधानेषु संनिपातिनामङ्गानामावृत्त्यानुष्ठाने कर्तव्ये हि द्वितीयादिपदार्थानां प्रथमानुष्ठितपदार्थक्रमाद्यः क्रमः स प्रवृत्तिक्रमः~। यथा प्राजापत्यपश्वङ्गेषु~। प्राजापत्या हि {\qtl वैश्वदेवीं कृत्वा प्राजापत्यैश्चरन्ति} इति वाक्येन तृतीयानिर्देशात्सेतिकर्तव्यताका एककालत्वेन विहिताः, अतस्तेषां तदङ्गानां चोपाकरणनियोजनप्रभृतीनां साहित्यं संपाद्यम्~। तच्च प्राजापत्यपशूनां संप्रतिपन्नदेवताकत्वेन युगपदनुष्ठानादुपपद्यते~। तदङ्गानां चोपाकरणादीनां युगपदनुष्ठानमशक्यम्~।}\\
\hrule
\vspace{3mm}
\noindent
नुष्ठीयन्ते, तत्र यदि प्रवृत्तिक्रममाश्रित्य सांनाय्याधर्मा अवदानाभिधाराणहविरासादनादयोऽपि सर्वे पूर्वमेवानुष्ठीयेरन् तत आग्नेयधर्मा अवदानादयस्तदनुष्ठानं\footnotemarkA[1] तदा सांनाय्यधर्माणां {\br सर्वेषां} स्ववप्रधानेन सह द्वाभ्यामाग्नेयधर्म-तदनुष्ठानाभ्यां विप्रकर्षः स्यात्~। यदा तु
सांनाय्यधर्माणां केषांचिद्वत्सापाकरणादीनां पूर्वमनुष्ठानेऽप्यन्ये सर्वेऽवदानादयस्तद्धर्मा मुख्यक्रममाश्रित्याग्नेयधर्मानुष्ठानानन्तरमनुष्ठीयन्ते तदा
सर्वेषामाग्नेयधर्मसांनाय्यधर्माणामेकैकेन विजातीयेन व्यवधानं भवति~। आग्नेयधर्माणां स्वप्रधानेन सह सांनाय्यधर्मैर्व्यवधानात्सांनाय्यधर्माणां च स्वप्रधानेन सहाग्नेयानुष्ठानेन व्यवधानादिति न विप्रकर्षः~। तस्मान्मुख्यक्रमः प्रवृत्तिकमाद्बलवानिति~।। \\

 इदानीं प्रवृत्तिक्रमं लक्षयति\textendash {\br सहप्रयुज्येत्यादिना~।} तत्रोदाहरणमाह\textendash {\br यथेति~।} प्राजापत्या हि सेतिकर्तव्यताका एककालत्वेन विहिता इत्यन्वयः~।~{\br अत इति~।} तेषामेककालत्वेन विहितत्वादित्यर्थः~। {\br तेषामिति~।} प्राजापत्यानामित्यर्थः~। उपाकरणेत्याद्युक्तार्थ एव बोध्यः~। {\br तच्चेति~।}  साहित्यं चेत्यर्थः~। संप्रतिपन्नदेवताकत्वेनेत्यत्र {\qt संप्रतिपन्नदेवताकालत्वेन} इति पाठः~। तत्कालस्तु वैश्वदेव्यनुष्ठानानन्तरकालो देवता च प्रजापतिरेव~। {\br तदङ्गानामिति~।} प्राजापत्याङ्गानामित्यर्थः~। {\br अशक्यमिति~।} अनेकेषां पशुनामुपाकरणं नियोजनं
\alfootnote{टिप्प०\textemdash\ $^{1}$आग्नेययागानुष्ठानमित्यर्थः~।}
\newpage
%%%%%%%%%%%%%%%%%%%%%%%%%%%%%%%
\fancyhead[RE]{[ अधिकारविधि\textemdash\ }
{\bl\noindent अतस्तेषां
साहित्यमव्यवहितानुष्ठानात्संपाद्यम्~। तचैकस्योपाकरणं विधायापरस्योपाकरणं विधेयम्~। एवं नियोजनादिकमपि~। तथा च प्राजापत्येषु कस्माच्चित्पशोरारभ्य एकं सर्वत्रानुष्ठाय द्वितीयादिपदार्थस्तेनैव क्रमेणानुष्ठेयः स प्रवृत्तिक्रमः~। सोऽयं श्रुत्यादिभ्यो दुर्बलः~। तदेवं संक्षेपतो निरूपितः षड्विधक्रमनिरूपणेन प्रयोगविधिः॥}
\begin{center}
\textbf{अधिकारविधिलक्षणम्}    
\end{center}

{\bl कर्मजन्यफलस्वाम्यबोधको विधिरधिकारविधिः~। कर्मजन्यफलस्वाम्यं कर्मजन्यफलभोक्तृत्वम्~। स च {\qtl यजेत स्वर्गकामः}}\\
\hrule
\vspace{3mm}
\noindent
चैकस्मिन्काल एकेन कर्त्रा  कर्तुमशक्यमित्यर्थः~। {\br अत इति~।} उपाकरणादीनां युगपदनुष्ठानानुपपत्तेरित्यर्थः~। {\br तेषामिति~।} उपाकरणादीनामित्यर्थः~।
{\br तच्चेति~।}  अव्यवधानेन साहित्यं चेत्यर्थः~। एवमेकस्य पशोर्नियोजनं विधायापरस्य पशोर्नियोजनं विधेयमित्यतिदिशति {\br एवमिति~।} प्राजापत्येष्वेकस्य पदार्थस्य सर्वत्रानुष्ठेयत्वे यं पशुमारभ्यैकः पदार्थोऽनुष्ठितस्तमेव पशुमारभ्य द्वितीयादिः पदार्थोऽनुष्ठेय इत्याह\textendash {\br तथा चेत्यादिना~। सोऽयमिति~।} प्रवृत्तिक्रम इत्यर्थः~।
{\qt आदि} शब्देनार्थक्रमादयो गृह्यन्ते~। {\br अत्रेदं बोध्यम्} {\qt सप्तदश प्राजापत्या भवन्ति, सप्तदश प्राजापत्यान्पशूनालभेतेति तद्विधौ तथैव श्रवणात्~}। तथा च
प्राजापत्येषु तेषु यं कंचित्पशुमारभ्योपाकरणं सप्तदशसु पशुषु कृत्वा तमेव पशुमारभ्य नियोजनं कर्तव्यम्~। एवं च तत्तत्पशूपाकरणानां तत्तत्पशुनियोजनैस्तुल्यं षोडशक्षणैर्व्यवधानं भवति~। तथा सति {\qt सप्तदश प्राजापत्यान्पशूनालभेत} इत्युत्पत्तिवाक्ये {\qt वैश्वदेवीं कृत्वा प्राजापत्यैश्चरन्ति} इति प्रयोगवाक्ये च श्रुतं साङ्गानां
सप्तदशपशुयागानां साहित्यमुपपद्यते~। अन्यथा तेष्वेककस्मिन्नुपाकरणनियोजनादिसर्वसंस्काराणां समापने प्रत्यक्षवचनावगतपशुसाहित्यं बाधितं भवेदिति~। एवं निरूपितं
षड्विधक्रमनिरूपणेन प्रयोगविधिमुपसंहरति {\br तदेवमिति~।}\\

 इदानीं क्रमप्राप्तमधिकारविधिं निरूपयति\textendash {\br कर्मजन्येति~।} कर्मजन्यफलस्वाम्यपदं व्याचष्टे {\br कर्मजन्यफलस्वाम्यमिति~। स चेति~।} अधिकार-
\newpage
%%%%%%%%%%%%%%%%%%%%%%%%%%%%%%%%%%%%%
\fancyhead[LO]{लक्षणम् ]}
{\bl\noindent इत्यादिरूपः~। स्वर्गमुद्दिश्य यागं विदधताऽनेन स्वर्गकामस्य यागजन्यफलभोक्तृत्वं प्रतिपाद्यते~। {\qtl यस्याहिताग्नेरग्निर्गृहान्दहेत् सोऽग्नये क्षामवतेष्टाकपालं निर्वपेत्} इत्यादिनाऽग्निदाहादौ निमित्ते कर्म विदधता निमित्तवतः कर्मजन्यपापक्षयरूपफलस्वाम्यं प्रतिपाद्यते~। एवं {\qtl अहरहः सन्ध्यामुपासीत} इत्यादिना शुचिविहितकालजीविनः संध्योपासनजन्यप्रत्यवायपरिहाररूपफलस्वाम्यं बोध्यते~।\blfootnote{पाठा०\textemdash\ $^{१}$चोद्यते.}\footnotemark\ तच्च
फलस्वाम्यं तस्यैव योऽधिकारविशिष्टः, अधिकारश्च स एव यद्विधिवाक्येषु पुरुषविशेषणत्वेन श्रूयते~। यथा काम्ये कर्मणि फलकामना, नैमित्तिके कर्मणि
निमित्तनिश्चयः, नित्ये संध्योपासनादौ शुचिविहितकालजीवित्वम्~। अत एव {\qtl राजा राजसूयेन स्वाराज्यकामो यजेत} इत्यनेन विधिवाक्येन स्वाराज्यमुद्दिश्य
विदधतापि न स्वाराज्यमात्रकामस्य\footnote{राज्यकाममात्रस्य.} तत्फलभोक्तृत्वं प्रतिपाद्यते, किंतु राज्ञः सतः }\\
\hrule
\vspace{3mm}
\noindent
विधिश्चेत्यर्थः~। {\br अनेनेति~।} {\qt यजेत वर्गकामः} इत्यादिवाक्येनेत्यर्थः~। तत्रोदाहरणान्तरमाह\textendash {\br यस्येति~। क्षामवत इति~।} क्षामवत्त्वगुणविशिष्टायेत्यर्थः~। {\br निमित्तवत इति}~। अग्निना गृहदाहादिरूपनिमित्तवतः पुरुषस्येत्यर्थः~। इत्यादिना कर्म विदधता विधिनेत्यन्वयः~। {\br कर्मजन्येति~।} अग्निदेवताककर्मजन्येत्यर्थः~। तत्रैवोदाहरणान्तरमाह\textendash {\br एवमिति~।} {\br शुचिविहितकालजीविन इति~।} शौचविशिष्टत्वे सति विहितकालजीविन इत्यर्थः~। {\br तच्चेति~।}  फलविधिबोधितं चेत्यर्थः~। को ह्यधिकारो यद्विशिष्टस्य पुंसः कर्मजन्यफलभोक्तृत्वरूपं फलस्वाम्यं विधिना बोध्यत इत्यत आह\textendash {\br अधिकारश्चेति~। अत एवेति~।} विधिवाक्येषु पुरुषविशेषणत्वेन श्रूयमाणस्याधिकारत्वादेवेत्यर्थः~। स्वाराज्यमुद्दिश्येत्यत्र राजसूयमिति शेषः~। {\br राज्ञः सत इति~।} क्षत्रियस्य सत इत्यर्थः~।
\newpage
%%%%%%%%%%%%%%%%%%%%%%%%%%%%%%%%%%%%%%%%%
\fancyhead[RE]{[ अधिकारविधि\textemdash\ }
{\bl\noindent स्वाराज्यकामस्यैव, राजत्वस्याप्यधिकारिविशेषणत्वेन श्रवणात्~। क्वचित्तु पुरुषविशेषणत्वेनाश्रुतमप्यधिकारिविशेषणम्~। यथा\\} 
\hrule
\vspace{3mm}
\noindent
तत्र हेतुमाह\textendash {\br राजत्वस्येति~। श्रवणादिति}~। {\qt राजा राजसूयेन} इत्यत्र श्रवणादित्यर्थः~। अत्र हि {\qt राज} शब्देन क्षत्रिय एवोच्यते, नतु राज्यसंबन्धमात्रेण तदन्योऽपि~। तेन क्षत्रियस्यैव राजसूयेऽधिकारः, नतु तदन्यस्य ब्राह्मणादेरित्यन्यत्र विस्तर इति भावः~। इदमत्र चिन्त्यते {\qt दर्शपूर्णमासाभ्यां स्वर्गकामो यजेत} इति श्रूयते~। तत्र क्रियानिष्पादकत्वं कर्तृत्वं फलभोक्तृतया स्वामित्वमधिकारः~। तादृशोऽधिकारो योगकर्तुर्नास्ति~। कुतः ? फलभोगाभावात्~। तथा हि\textendash यजेतेत्यत्राख्यातेन भावनाभिधीयते~। तस्यां च धात्वर्थो भाव्यः, एकपदोपात्तत्वात्~। वर्गस्तु पदान्तरोपात्तत्वाद्वाक्येन भाव्यतयाऽन्वेतव्यः~। तच्च वाक्यमेकपदरूपया श्रुत्या बाध्यते~। स्वर्गस्य भाव्यत्वाभावे सति गुणत्वमभ्युपेयम्~। स्वर्गशब्दो नात्र सुखवाची किंतु सुखसाधनं चन्दनादिद्रव्यं ब्रूते~। लोके तथा व्यवहारात्~। तच्च कामयितुं
योग्यम्~। तेन द्रव्येण विना यागानिष्पत्तेः~। तस्मादस्मिन्वाक्ये फलानभिधाने तद्भोगाभावात्कर्तुर्यागे कर्तृत्वमेव नत्वधिकार इत्यधिकारलक्षणं नारब्धव्यमिति प्राप्ते ब्रूमः {\qt यजेतेत्यत्र प्रत्ययस्य केवलमाख्यातरूपत्वमेवेति न च मन्तव्यं, किंतु लिङ्प्रत्ययत्वेन विधिरूपत्वमप्यस्ति तत्राख्यातत्वाकारेण भावनामाचष्टे, विधित्वाकारेण पुरुषं प्रवर्तयति, पुरुषश्च स्वाभिमतफलमन्तरेण न प्रवर्तते इति तदपेक्षितं वर्गमेव भाव्यतया विधिरुपादत्ते~}। स्वर्गशब्दश्वोत्कृष्टे सुखे रूढः~। द्रव्ये तु लाक्षणिकः~। तस्मात्सुखस्य भाव्यत्वं विधिश्रुत्या सिद्धम्~। धात्वर्थस्य तु भाव्यत्वमेकपदेन प्रतीयमानमपि प्रत्ययेन नावगम्यते, किंतु प्रकृत्या~। तथा सति स्वर्गभाव्यत्वं भावनायां प्रत्यासन्नमेकेनैव विधिरूपेणाख्यातेनावगमात्कमियोगादपि स्वर्गस्यैव भाव्यत्वम्~। तस्मात्फलभोगसंभवेन कर्तुरधिकारोऽस्तीत्यधिकारलक्षणमारब्धव्यमिति~। विधिवाक्येष्वश्रुतमपि किंचिदधिकारिविशेषणत्वेनान्यथानुपपत्त्या तत्समाश्रयणेन व्यवहारोपपत्तिरित्याशयेनाह\textendash {\br क्वचित्त्विति~।} तत्रोदाहरणमाह\textendash {\br यथेति~।} विद्येत्यत्रविधिवाक्येष्वश्रुतमप्यधिकारिविशेषणत्वेन तद्विशेषण-
\alfootnote{टिप्प०\textemdash\ $^{1}$आश्रीयते इति शेषः~।}
\newpage
%%%%%%%%%%%%%%%%%%%%%%%%%%%%%%%%%%%%%%%%%%%%%
\fancyhead[LO]{लक्षणम् ]}
{\bl\noindent ऽध्ययनविधिसिद्धा विद्या,
क्रतुविधीनामर्थज्ञाना\footnotemarkA[1]पेक्षणीयत्वेनाध्ययनविधि\textendash सिद्धार्थज्ञानवन्तं प्रत्येव प्रवृत्तेः~। एवमग्निसाध्यकर्मसु आधानसिद्धाग्निमत्ता~।
अग्निसाध्यकर्मणामग्यपेक्षत्वेन तद्विधीनामाधानसिद्धाग्निमन्तं प्रत्येव प्रवृत्तेः~। एवं सामर्थ्यमपि}\\
\hrule
\vspace{3mm}
\noindent
मिति शेषः~। तत्र हेतुमाह\textendash {\br क्रत्वित्यादिना~।} तत्रैवोदाहरणान्तरमाह\textendash {\br एवमिति~।} अग्निमत्तेत्यत्रापि पूर्ववदेव शेषो बोध्यः~। तत्रापि हेतुमाह\textendash {\br अग्निसाध्येति~। तद्विधीनामिति~।} अग्निसाध्यकर्मविधीनामित्यर्थः~। अनेन च निरुक्ताधिकारिविशेषणेन शूद्रस्य यागेऽनधिकारो ध्वनितः~। तस्याध्ययनविधिसिद्धविद्याया अभावादाधानसिद्धाग्निमत्ताया अभावाच्च~। किंच अध्ययने ह्युपनीतस्यैवाधिकारात् उपनयनेऽपि च {\qt अष्टवर्षं ब्राह्मणमुपनयीत} इत्यादिना
त्रैवर्णिकस्यैवाधिकारविधानात्~। अग्न्याधानेऽपि {\qt वसन्ते ब्राह्मणोऽग्नीनादधीत} इत्यादिना त्रैवर्णिकमात्रस्याधिकारविधानाच्च~। यद्यपि {\qt वर्षासु रथकारोऽग्नीनादधीत} इत्यनेन रथकारस्य सौधन्वनापरनामकस्याग्न्याधानं विहितं योगाद्रूढेर्बलीयस्त्वात् तथापि नास्योत्तरकर्मस्वधिकारः तस्याध्ययनविधिसिद्धविद्याया अभावादित्यन्यत्र विस्तर इति भावः~।~{\br ननु} तत्र रथं करोतीति व्युत्पत्त्या त्रैवर्णिक एव रथकारो नतु शूद्रस्य तत्राप्यधिकार इति चेत्, {\br न}; संकीर्णजातिविशेषे {\qt रथकार} शब्दस्य रूढत्वात्~। तथा हि\textendash वैश्यायां क्षत्रियादुत्पन्नो माहिष्यः~। शूद्रायां वैश्यादुत्पन्ना करणी~। तस्यां करण्यां माहिष्यादुत्पन्नो रथकारः~। तथा च याज्ञवल्क्यः  {\qt माहिष्येण करण्यां तु रथकारः प्रजायते} इति~। तस्मान्न तादृशव्युत्पत्त्या त्रैवर्णिको {\qt रथकार} शब्देन ग्रहीतुं  शक्यत इति~। किंच कुत्रचिद्यागेऽपि कस्यचिच्छूद्रस्याधिकारो भवति; {\qt वास्तुमयं रौद्रं चरुं निर्वपे}दिति प्रकृत्य {\qt एतया निषादस्थपतिं याजयेत्} इति श्रवणात्~। {\qt वास्तु} शब्दः किंचित्प्रकृतिद्रव्यविशेषमाह~। एतस्यामिष्टावधिकारी निषादस्थपतिशब्दवाच्यत्रैवर्णिक एव~। कुतः ? निषादानां स्थपतिरिति षष्ठीसमासस्य त्रैवर्णिके संभवात्~। तस्य ह्यधीतवेदत्वेन विद्यासंभवाच्चेति प्राप्ते ब्रूमः{\qt निषाद}
\alfootnote{टिप्प०\textemdash\ $^{1}$पेक्षत्वेनेति पाठो भाति~।}
\newpage
%%%%%%%%%%%%%%%%%%%%%%%%%%%%%%%%%%
\fancyhead[RE]{[अधिकारविधि\textemdash\ }
{\bl\noindent {\qtl आख्यातानामर्थं ब्रुवतां शक्तिः सहकारिणी} इति न्यायात्}\\
\hrule
\vspace{3mm}
\noindent
श्चासौ स्थपतिश्चेति कर्मधारयसमासस्य मुख्यत्वान्न षष्ठीसमासेन त्रैवर्णिको {\qt निषादस्थपति} शब्दार्थः~। षष्ठीसमासे तु संकीर्णजातिविशेषवाचिना निषादशब्देन तत्संबन्ध उपलक्ष्येत; नत्वयं कर्मधारये दोषोऽस्ति~। तस्मात् तात्कालिकाचार्योपदेशादिना विद्यां संपाद्य धनिको निषादो रौद्रं यागं कुर्यादिति राद्धान्तः~। अध्ययनविधिसिद्धविद्यादिवत्पुरुषसामर्थ्यमपि विधिवाक्येष्वश्रुतमप्यधिकारिविशेषणमित्याह\textendash {\br एवमिति~। सामर्थ्यमिति~।} आज्यावेक्षणादिकं लौकिकपुंसामर्थ्यमित्यर्थः~। वैदिकसामर्थ्यस्याध्ययनविधिसिद्धविद्यादेः पूर्वमेवोक्तत्वादित्यर्थः~। वृद्धोक्तन्यायं विनिगमकं समुदाहरन् तत्र हेतुमाह\textendash {\br आख्यातानामिति~।} अनेन च विशेषणेनान्धादेरनधिकारो ध्वनितः~। इदमत्र विचार्यते {\qt अन्धः पङ्गुर्बधिरो मूको गवाश्वादयश्च तिर्यञ्च इत्यादीनां चेतनत्वेन
निरतिशयसुखरूपे स्वर्गे कामना संभवति~}। अथोच्येत,\textendash केषुचिदङ्गेषु तेषां शक्तिर्नास्ति~। तथा हि\textendash अन्धो नाज्यमवेक्षितुं क्षमः,
पङ्गुर्विष्णुक्रमेष्वशक्तः, बधिरो नाध्वर्युप्रोक्तं शृणोति~। तथा च {\qt क्लृप्तीर्वाचयति} इति विहितस्यानुष्ठानं न सिद्ध्येत्~। मूकोऽनुमन्त्रणादावसमर्थः~। तिर्यञ्चो बहुष्वसमर्था इति, तन्न; यथाशक्त्यङ्गानामनुष्ठेयत्वात्~। {\qt स्वर्गकामो यजेत} इत्यनेन प्रधानवाक्येन सर्वाधिकारः प्रतीयते~। स चाज्यावेक्षणाद्यङ्गवाक्यानुसारेण न संकोचयितुं
युक्तः, किंतु प्रधानानुसारेणाङ्गानुष्ठानमेव संकोचयितुं युक्तं, तस्मादन्धादेरप्यधिकार इति प्राप्ते ब्रूमः {\qt यदाज्यावेक्षणादयः पुरुषार्थतया विधीयेरन् तदा तल्लोपयितुर्न क्रतोर्वैकल्यम् , इह तु क्रत्वङ्गतया ते विहिता इति तल्लोपे क्रतुरेव न निष्पद्येत, तस्मादसमर्थस्य नास्त्यधिकार इति सिद्धम्~}। किंच ज्योतिष्टोमे श्रूयते {\qt यद्युद्गाताऽपच्छिद्येतादक्षिणो यज्ञः संस्थाप्योऽथान्यश्चाहर्तव्यस्तत्र तद्दद्याद्यत्पूर्वस्मिन्दास्यन्स्यात् यदि प्रतिहर्तापच्छिद्येत सर्वस्वं दद्यात्} इति~। अस्यायमर्थः {\qt प्रातःसवने
बहिष्पवमानेन स्तोष्यमाणा ऋत्विजः शालाया बहिः प्रसर्पन्ति तदानीमेकस्य पृष्ठतोऽन्य इत्येवं पिपीलिकावत्पङ्क्त्याकारेण गन्तव्यम्~}।~तत्र पुरतो गन्तुः कच्छं गृहीत्वैव पृष्ठतोऽन्यो गच्छेत्~। एवं सति यदि प्रमादादुद्गाता गृहीतं कच्छं मुञ्चेत्तदा दक्षिणामदत्वा प्रक्रान्तो यज्ञः समापनीयः~। तं समाप्य पुनरपि स यज्ञः प्रयोक्तव्यः~। तस्मिन्प्रयोगे पूर्वं
यद्दित्सितं द्रव्यं
\newpage
%%%%%%%%%%%%%%%%%%%%%%%%%%%%%%%%%
\fancyhead[LO]{लक्षणम्]}
\hrule
\vspace{3mm}
\noindent
तद्दद्यात्~। यदा प्रतिहर्ता मुञ्चेत् तदा तस्मिन्नेव प्रयोगे सर्वस्वं दद्यादिति~। तत्र यद्युद्गातृप्रतिहर्तारौ युगपत्तन्मुञ्चेतां तदानीमुक्तं प्रायश्चित्तनिमित्तं विहन्येत~। एककर्तृको ह्यपच्छेदो निमित्तत्वेन श्रुतः अयं तूभयकर्तृकत्वान्नैकेन व्यपदेष्टुं शक्यते, तस्माच्छ्रूयमाणस्य निमित्तस्य विहितत्वान्नास्ति प्रायश्चित्तमिति प्राप्ते ब्रूमः {\qt द्वौ ह्यत्रापच्छेदौ तयोरेकैकस्यैकैक एव कर्तेति निमित्तस्य नास्ति विघातः कालमात्रैक्यादेकापच्छेदभ्रान्तिः तस्मान्निमित्तविघाताभावादस्ति प्रायश्चित्तम्~}। किंच अदक्षिणत्वं सर्वस्वदक्षिणत्वं चेति यत्प्रायश्चित्तद्वयं निमित्तभेदेन श्रुतं तन्निमित्तद्वयसंनिपाते समुच्चेतव्यम्, यद्यदक्षिणत्वसर्वस्वदानयोरन्योन्यविरोधस्तर्हि प्रयोगभेदेन व्यवस्थापनीयम् , अपच्छेदयुक्ते प्रथमप्रयोगे दक्षिणा न दातव्या, उत्तरप्रयोगे सर्वस्वं दातव्यं, सत्यपि प्रयोगभेदे कर्मण एकत्वात् समुच्चय इति प्राप्ते ब्रूमः{\qt  न ह्युत्तरप्रयोगेऽपच्छेदो विद्यते~}। न चासति निमित्ते प्रायश्चित्तं युक्तम्~। तस्मात्प्रथमप्रयोग एव निमित्तद्वयवशात्प्रायश्चित्तद्वयं प्राप्त तच्चान्योन्यविरुद्धं विकल्प्यते~। किंच उद्गातृप्रतिहर्तृकर्तृकयोरपच्छेदयोर्यौगपद्ये समानबलत्वादस्तु प्रायश्चित्तयोर्विकल्पः~। यदा तु क्रमेणापच्छेदौ स्यातां तदानीमसंजातविरोधित्वेन पूर्वस्य प्रबलत्वाच्छ्रुतिलिङ्गादाविवोत्तरस्य प्रवृत्तिर्विरुध्यत इति चेत्, {\qt मैवम्} ; श्रुतिलिङ्गादावुत्तरस्य पूर्वसापेक्षत्वात्पूर्वेण विरोधे सत्युत्तरस्योत्पत्तिरेव नास्ति~। इह तु ज्ञानद्वयमन्योन्यनिरपेक्षं वाक्यद्वयादुत्पद्यत इत्युत्पत्तिप्रतिबन्धो नास्ति~। उत्पद्यमानं चोत्तरज्ञानं स्वविरुद्धस्य पूर्वज्ञानस्य बाधेनैवोत्पद्यते~। {\br ननु} निरपेक्षत्वस्य समानत्वात्पूर्वज्ञानमेवोत्तरस्य बाधकमस्त्विति चेत् , {\qt न}; पूर्वज्ञानोत्पत्तिदशायामविद्यमानस्योत्तरज्ञानस्य बाध्यत्वायोगात्~। उत्तरकाले तु स्वयं बाधितं पूर्वज्ञानं कथमुत्तरस्य बाधकं भवेत् , नान्यत्किंचिदुत्तरस्य बाधकं पश्यामः~। तस्मादुत्तरकालीनापच्छेदनिमित्तं प्रायश्चित्तमनुष्ठेयम्~। किंच यद्युद्गाता पश्चादपच्छिद्यते तदा तस्यापच्छेदस्य प्रबलत्वात्तन्निमित्तं प्रायश्चित्तमनुष्ठेयम्~। तच्च प्रायश्चित्तमीदृशम् {\qt प्रथमं प्रयोगं  दक्षिणारहितमनुष्ठाय द्वितीयप्रयोगे पूर्वं दित्सिता दक्षिणा दातव्येति~}। पूर्वं च गवां द्वादशाधिकं\footnotemarkA[1] शतंदित्सितं तस्य ज्योतिष्टोमदक्षिणारूपेण विहितत्वात्तस्मादुत्तरप्रयोगे द्वादशशतं देयमिति प्राप्ते ब्रूमः  {\qt प्रतिहर्तुः प्रथममपच्छेदे सति तन्निमित्तकं सर्वस्वदानरूपं प्रायश्चित्तं प्रथमप्रयोगे प्राप्तं, तेन च
क्रतुस्वभावप्रयुक्तस्य द्वादशशतस्य बाधात्स}-
\alfootnote{टिप्प०\textemdash\ $^{1}$द्वादशाधिकं शतमित्यर्थः, एवमुत्तरत्रापि~।}
\newpage
%%%%%%%%%%%%%%%%%%%%%%%%%%%%%%%%%%%
\fancyhead[RE]{[नियम\textemdash\ }
{\bl\noindent समर्थं प्रत्येव विधिप्रवृत्तेः~। तदेवं निरूपितो विधिः~।}
\begin{center}
    \rule{.15\linewidth}{1pt}\\
    \vspace{2mm}
 \textbf{अथ मन्त्रमीमांसा}   
\end{center}

{\bl प्रयोगसमवेतार्थस्सारका मन्त्राः~। तेषां च तादृशार्थस्सारकत्वेनैवार्थवत्त्वम्~। नतु तदुच्चारणमदृष्टार्थं संभवति, दृष्टफलकत्वेऽदृष्टकल्पनाया अन्याय्यत्वात् , न च दृष्टस्यार्थस्मरणस्य प्रकारान्तरेणापि संभवान्मन्त्राम्नानं व्यर्थमिति वाच्यम् ; मन्त्रैरेव स्मर्तव्यमिति नियमविध्याश्रयणात्~।}
\begin{center}\textbf{नियमविधिः}\end{center}
{\bl नानासाधनसाध्यक्रियायामेकसाधनप्राप्तावप्राप्तस्यापरसाध-\\} 
\hrule
\vspace{3mm}
\noindent
र्वस्वं दित्सितं, न चोद्गात्रपच्छेदेन पश्चाद्भाविना सर्वदित्सा बाध्यत इति शङ्कनीयम्~। बाधकस्य दक्षिणान्तरस्य तत्रानुक्तत्वात्~। यद्दित्सितं तदुत्तरप्रयोगे देयमित्येतावदेव तत्रोच्यते~। दित्सितं च सर्वस्वमित्युक्तम्~। अत उत्तरकालीनोद्गात्रपच्छेदनिमित्तेऽपि पुनःप्रयोगे पूर्वकालीनप्रतिहर्त्रपच्छेदप्रयुक्तं सर्वस्वमेव दातव्यमित्यादिकमर्थजातं
निरूपणीयमभिप्रेत्य विधिनिरूपणमुपसंहरति {\br तदेवमिति~।} तस्मिन्नतीते ग्रन्थ उक्तप्रकारेणेत्यर्थः~।
\begin{center}
    \rule{.15\linewidth}{1pt}
\end{center}
\begin{quote}
    \ab यः सर्वकर्ता सकलात्मरूपश्चन्द्रार्कवह्नीक्षणकश्चिदात्मा~।\\
साम्बो हि सोमार्धविभूषणाढ्यस्तं नौमि देवार्चितपादपीठम्~॥~१॥
\end{quote}

 इदानीं मन्त्ररूपं वेदभागं निरूपयति {\br प्रयोगेति~। तेषामिति~।} मन्त्राणामित्यर्थः~। {\br तादृशार्थेति~।} प्रयोगसमवेतार्थेत्यर्थः~। {\br अर्थवत्त्वमिति~।} प्रयोजनवत्त्वमित्यर्थः~। {\br ननु} मन्त्रोच्चारणस्यादृष्टार्थकत्वेनाप्युपपत्तेः कुतस्तेषां प्रयोगसमवेतार्थस्मारकत्वेनैव प्रयोजनवत्त्वमित्यत आह\textendash {\br नत्विति~। तदुच्चारणमिति~।} मन्त्रोच्चारणमित्यर्थः~। तत्र हेतुमाह\textendash {\br दृष्टफलकत्व इति~।} {\br ननु} दृष्टस्य देवताद्यर्थस्मरणस्य ब्राह्मणवाक्यादिनापि
संभवान्मन्त्रोच्चारणस्यादृष्टार्थकत्वानङ्गीकारे तदाम्नानस्य वैयर्थ्यापत्तिरित्याशङ्क्य {\qt मन्त्रैरेव सोऽर्थः स्मर्तव्यः} इति नियमविध्यङ्गीकारान्न मन्त्राम्नानस्य वैयर्थ्यमिति परिहरति\textemdash\ {\br न चेत्यादिना~।} \\

 {\br ननु} किंलक्षणको नियमविधिर्यदाश्रयणादन्यस्यार्थस्मारकस्य व्यवच्छेदो
\newpage
%%%%%%%%%%%%%%%%%%%%%%%%%%%%%%%%
\fancyhead[LO]{विधिः ]}
{\bl\noindent नस्य प्रापको विधिर्नियमविधिः~। यथाहुः {\qtl विधिरत्यन्तमप्राप्तौ नियमः पाक्षिके सति~। तत्र चान्यत्र च प्राप्तौ परिसंख्येति गीयते} इति~। अस्यार्थः {\qtl प्रमाणान्तरेणाप्राप्तस्य प्रापको विधिरपूर्वविधिः}, यथा {\qtl यजेत स्वर्गकामः} इत्यादिः; स्वर्गार्थकयागस्य प्रमाणान्तरेणाप्राप्तस्यानेन विधानात्~। पक्षेऽप्राप्तस्य प्रापको विधिर्नियमविधिः~। यथा {\qtl व्रीहीनवहन्ति} इत्यादिः~। कथमस्य पक्षेऽप्राप्तप्रापकत्वमिति चेत् , {\qtl इत्थम्}~। अनेन ह्यवघातस्य वैतुष्यार्थत्वं न प्रतिपाद्यते; अन्वयव्यतिरेकसिद्धत्वात्~। किंतु नियमः, \\}
\hrule
\vspace{3mm}
\noindent
लभ्यत इत्याशङ्क्य नियमविधिलक्षणमाह\textendash {\br नानासाधनेति~।} तत्र संमतिमाह\textendash {\br यथाहुरिति~।} तां व्याचष्टे {\br अस्यार्थ इत्यादिना~।}
तत्रापि {\qt विधिरत्यन्तमप्राप्तौ} इति प्रथमपादमपूर्वविधिपरं व्याचष्टे {\br प्रमाणान्तरेणेति~।} प्रमाणान्तरेण यदर्थत्वेनाप्राप्तस्य तदर्थत्वेन प्रापको यो विधिः सोऽपूर्वविधिरित्यर्थः~। तत्रोदाहरणमाह\textendash {\br यथेति~।} तस्यापूर्वविधित्वे हेतुमाह\textendash {\br स्वर्गेति~।} स्वर्गार्थकत्वेन प्रमाणान्तरेणाप्राप्तस्य यागस्य तदर्थत्वेन {\qt यजेत स्वर्गकामः} इत्यनेन विधानाद्भवत्ययमपूर्वविधिरित्यर्थः~। नियमः पाक्षिके सतीति द्वितीयपादं प्रकृते संमतिरूपं व्याचष्टे {\br पक्षेऽप्राप्तस्येति~।} तत्रोदाहरणमाह\textendash  {\br यथेति~। ननु} {\qt व्रीहीनवहन्ति} इत्यस्यावघातविधेः कथं पक्षेऽप्राप्तस्यावघातस्य प्रापकत्वं स्वीक्रियते ? वैतुष्यार्थत्वेन
प्रमाणान्तरेणाप्राप्तस्य तस्य तदर्थत्वेनानेन विधानोपपत्तेरित्याशङ्क्य, अनेन हि विधिनावघातस्य वितुषतार्थत्वं न प्रतिपाद्यते, तस्य तदर्थत्वेनान्वयव्यतिरेकसिद्धत्वात्, अवघातादिसत्त्वे व्रीहीणां वैतुष्यं जायते, तदसत्त्वे तदभाव इति सार्वजनीनमिति परिहरति {\br कथमित्यादिना~। इत्थमिति~।} अनेन वक्ष्यमाणप्रकारेणेत्यर्थः~। {\br ननु} यद्यवघातादीनां सर्वेषामेव वैतुष्यार्थत्वमन्वयव्यतिरेकाभ्यां सिद्धं तदा {\qt व्रीहीनवहन्ति} इत्यवघातविधेरन्वयव्यतिरेकसिद्धावघातप्रापकत्वेन वैयर्थ्यमेवापद्येतानुवादकत्वादित्याशङ्कते {\br किंत्विति~।} नास्य विधेरनुवादकत्वेन वैयर्थ्यं संभवति, नियमविधायकत्वेनानुपपत्तेरिति परिहरति {\br नियम इति~।} व्रीहीणां वैतुष्यमवघातेनैव
\newpage
%%%%%%%%%%%%%%%%%%%%%%%%%%%%%%
\fancyhead[RE]{[परिसंख्याविधिः ] }
{\bl\noindent
स चाप्राप्तांशपूरणम्~। वैतुष्यस्य हि नानोपायसाध्यत्वाद्यदावघातं परित्यज्य उपायान्तरं ग्रहीतुमारभते, तदावघातस्याप्राप्तत्वेन तद्विधाननामकमप्राप्तांशपूरणमेवानेन विधिना क्रियते~। अतश्च नियमविधावप्राप्तांशपूरणात्मको नियम एव वाक्यार्थः~। पक्षेऽप्राप्तावघाततस्य\footnotemark\ विधानमिति यावत्~।}
\begin{center}
 \textbf{परिसंख्याविधिः}   
\end{center}

{\bl {\al उभयोश्च युगपत्प्राप्तावितरव्यावृत्तिपरो विधिः परिसंख्याविधिः~}। यथा {\qtl पञ्च पञ्चनखा भक्ष्याः} इति~। इदं हि वाक्यं न पञ्चनखभक्षणपरं, तस्य रागतः प्राप्तत्वात्; नापि नियमपरं, पञ्चनखापञ्चनखभक्षणस्य युगपत्प्राप्तेः पक्षेऽप्राप्त्यभावात्~। अत इदमपश्चनखभक्षणनिवृत्तिपरमिति भवति परिसंख्याविधिः~।\\}
\hrule
\vspace{3mm}
\noindent
संपादनीयमिति नियमः प्रतिपाद्यत इत्यर्थः~। {\br ननु} तादृशनियमप्रतिपादनेऽत्र कीदृशो वाक्यार्थो भवतीत्यत आह\textendash {\br स चेति~।} स एवाप्राप्तांशपूरणरूपो नियमोऽत्र वाक्यार्थ इत्यर्थः~। {\br ननु} कथमत्राप्राप्तांशो लभ्यते यस्य पूरणेनास्य विधेः सार्थक्यमित्यतोऽत्राप्राप्तांशपूरणरूपमेव वाक्यार्थमुपपादयति {\br वैतुष्येत्यादिना~।} उपपादितमप्राप्तांशपूरणरूपं वाक्यार्थमुपसंहरति {\br अतश्चेति~।} अस्य विधेरप्राप्तांशपूरणरूपनियमप्रतिपादकत्वाच्चेत्यर्थः~। पर्यवसितार्थमाह\textendash 
{\br पक्ष इति~। यावदिति~।} पर्यवसन्नमित्यर्थः~।\\

तत्र चेत्याद्युत्तरार्धं परिसंख्याविधिपरं व्याचष्टे {\br उभयोश्चेति~।} तत्रोदाहरणमाह\textendash {\br यथेति~।} अस्यात्यन्तमप्राप्तविधित्वरूपमपूर्वविधित्वमाशङ्क्य निराचष्टे {\br इदं हीति~।} तत्र हेतुमाह\textendash {\br तस्येति~।} पश्चनखभक्षणस्येत्यर्थः~। अस्य वाक्यस्य नियमविधित्वमपि निराचष्टे{\br नापीति~।} तत्रापि हेतुमाह\textendash {\br पञ्चनखेति~। अत इति~।} विधिद्वयासंभवादित्यर्थः~। {\br इदमिति~।} पञ्च  पञ्चनखा भक्ष्या इति वाक्यमित्यर्थः~। अपञ्चनखभक्षणवृत्तिस्तु न केनापि 
\blfootnote{पाठा०\textemdash\ $^{१}$वघातविधानम्.}
\newpage
%%%%%%%%%%%%%%%%%%%%%%%%%%%%%%%%%%%%%%%%%
\fancyhead[LO]{[परिसं० दोषत्रयम् ]}
\begin{center}
 \textbf{परिसंख्यायाः श्रौतीत्व -लाक्षणिकीत्वभेदौ}
\end{center}
 
{\bl सा च द्विविधा  {\al श्रौती लाक्षणिकी } चेति~। तत्र {\qtl अत्र ह्येवावयन्ति}  इति श्रौती परिसंख्या~। एवकारेण पवमानातिरिक्तस्तोत्रव्यावृत्तेरभिधानात्~। {\qtl पञ्च पञ्चनखा भक्ष्याः} इति तु लाक्षणिकी ; इतरनिवृत्तिवाचकपदाभावात्~। अत एवैषा त्रिदोषग्रस्ता~।~}
\begin{center}
 \textbf{परिसंख्याया दोषत्रयम् }   
\end{center}
 
{\bl दोषत्रयं च  {\qtl श्रुतहानिः, अश्रुतकल्पना, प्राप्तबाधश्चेति~}।तदुक्तम्\textendash {\qtl श्रुतार्थस्य परित्यागादश्रुतार्थप्रकल्पनम्\footnotemarkA[1]~। प्राप्तस्य
बाधादित्येवं परिसंख्या त्रिदूषणा} इति ; श्रुतस्य पञ्चनखभक्षणस्य हानात्, अश्रुताऽपञ्चनखभक्षणनिवृत्तेः कल्पनात्, प्राप्तस्य  चापश्चनखभक्षणस्य बाधनादिति~। अस्मिंश्च दोषत्रये 
दोषद्वयं शब्दनिष्ठम्~। प्राप्तबाधस्त्वर्थनिष्ठ इति दिक्~।~}\\
\hrule
\vspace{3mm}
\noindent
प्राप्ता~। ततश्च तद्विधायकत्वेन नास्य वाक्यस्यानुवादकत्वमपि~। तथा च सैवात्र वाक्यार्थ इति भावः~।~\\

 सा च परिसंख्या द्विविधेत्याह  {\br सा चेति~।} आद्यामुदाहरति {\br तत्रेति~।}  द्वयोः परिसंख्ययोर्मध्य इत्यर्थः~। {\br अत्रेति~।} प्रकृत इत्यर्थः~। अवयन्तीति~। अवजानन्तीत्यर्थः, गायन्तीति यावत्~। श्रौत्याः परिसंख्यायास्तत्त्वे हेतुमाह\textendash {\br एवकारेणेति~।} द्वितीयामुदाहरति  {\br पञ्चेति~।} पञ्च पञ्चनखास्तु {\ab  पञ्च पञ्चनखा भक्ष्या ब्रह्मक्षत्रेण राघव~॥ शशकः शल्लकी गोधा खङ्गी कूर्मोऽथ पञ्चमः}  इत्यादिवचनोदाहृता बोध्याः~। अस्या अपि तत्त्वे हेतुमाह \textendash {\br इतरनिवृत्तीति~। अत एवेति~।} इतरनिवृत्तिवाचकपदाभावादित्यर्थः~। {\br एषेति~।} लाक्षणिकीत्यर्थः~।~\\

 दोषत्रयं प्रदर्शयति  {\br दोषत्रयं चेति~।} तत्र श्रुतहानौ हेतुमाह \textendash {\br श्रुतस्येति~।} अश्रुतकल्पनायां हेतुमाह \textendash {\br अश्रुतेति~।} प्राप्तस्य बाधेऽपि हेतुमाह\textendash {\br प्राप्तस्येति~।} रागतः प्राप्तस्येत्यर्थः~। अस्य दोषत्रयस्य व्यवस्थाया वृत्तित्वमाह\textendash
\alfootnote{टिप्प०\textemdash\ $^{1}$प्रकल्पनादिति पाठो भाति~।~}
\newpage
%%%%%%%%%%%%%%%%%%%%%%%%%%%%%%%%%%%%%%%%%%%%
\fancyhead[RE]{[नामधेयमीमांसा ]}
{\bl\noindent येषां तु प्रयोगसमवेतार्थस्मारकत्वं न संभवति तदुच्चारण  स्यानन्यगत्याऽदृष्टार्थकत्वं कल्प्यत इति नानर्थक्यमिति~।}
\begin{center}
  \textbf{अथ नामधेयमीमांसा }  
\end{center}

{\bl नामधेयानां च विधेयार्थपरिच्छेदकतयार्थवत्त्वम्~। तथा हि\textendash {\qtl उद्भिदा यजेत पशुकामः} इत्यत्रोद्भिच्छब्दो यागनामधेयम् ,  तेन च विधेयार्थपरिच्छेदः क्रियते~। तथा हि \textendash अनेन वाक्येनाप्राप्तत्वात्फलोद्देशेन यागो विधीयते~। यागसामान्यस्याविधेय  त्वात् यागविशेष एव विधीयते~। तत्र कोऽसौ यागविशेष 
इत्यपेक्षायामुद्भिच्छब्दादुद्भिद्रूपो याग इति ज्ञायते~। उद्भिदा  यागेन पशुं भावयेदित्यत्र सामानाधिकरण्येन नामधेयान्वयात्~।~}\\
\hrule
\vspace{3mm}
\noindent
{\br अस्मिंश्चेति~।} दोषत्रयमध्य इत्यर्थः~।~{\br ननु} कथं {\br सर्वेषां} मन्त्राणां प्रयोगसमवेतार्थस्मारकत्वेनैवार्थवत्त्वमुपपद्यते ? हुंफडादिमन्त्राणां प्रयोगसमवेतार्थस्मारकत्वासंभवादित्याशङ्क्याह \textendash {\br येषामित्यादिना~। तदुच्चारणस्येति~।} हुंफडादिमन्त्रोच्चारणस्येत्यर्थः~। अनन्यगत्यादृष्टार्थत्वमित्यत्राऽदृष्टार्थत्वमिति पदच्छेदः~। {\br इति नानर्थक्यमिति~।} अतो हेतोर्हुफडादिमन्त्राणां नानर्थक्यमित्यर्थः~।\\

 मन्त्रभागस्य यथायथं प्रयोजनवत्त्वमुपपाद्येदानीं क्रमप्राप्तं नामधेयानां  सार्थक्यमुपपादयति  {\br नामधेयानामिति~। विधेयार्थपरिच्छेदकतयेति~।} विजातीयव्यावर्तकत्वेन विधेयार्थनिश्चायकतयेत्यर्थः~। विधेयार्थस्यैव  समर्थकतयेति यावत्~। एतदेव प्रदर्शयति  {\br तथा हीत्यादिना~।} तेन चेति~। उद्भिच्छब्देन चेत्यर्थः~। उद्भिच्छब्दस्य विधेयार्थपरिच्छेदकतया नामधेयत्वप्रदर्शनाय भूमिकामारचयति {\br तथा हीत्यादिना~। अनेनेति~।} {\qt उद्भिदा जयेत पशुकामः} इत्यनेनेत्यर्थः~। {\br फलोद्देशेनेति~।}
पशुरूपफलोद्देशेनेत्यर्थः~। यागेति~। साधनवैलक्षण्यमन्तरेण फलवैलक्षण्यानुपपत्तेर्नात्र यागसामान्यं विधीयते~। ततश्च यागविशेष एव विधीयत इत्यर्थः~।~{\br तत्रेति~।} 
यागविशेषस्य विधेयत्व इत्यर्थः~। {\br क इति~।} कोऽसौ यागविशेष इति यागविशेषापेक्षायामित्यर्थः~। {\br उद्भिच्छब्दादुद्भिद्रूपो याग इति ज्ञायत इति~।}
\newpage
%%%%%%%%%%%%%%%%%%%%%%%%%%%%%%%%%%%%%%
\fancyhead[LO]{[नामधेयत्वे नि० ]}
\begin{center}
 \textbf{नामधेयत्वे निमित्तचतुष्टयम् }   
\end{center}
 
{\bl नामधेयत्वं च निमित्तचतुष्टयात्~। मत्वर्थलक्षणाभयाद्वाक्यभेदभयात्तत्प्रख्यशास्त्रात्त्यद्य्वपदेशाच्चेति~। तत्र {\qtl उद्भिदा यजेत पशुकामः} इत्यत्र {\qtl उद्भित्}  शब्दस्य यागनामधेयत्वं मत्वर्थलक्षणा-\\ }
\hrule
\vspace{3mm}
\noindent
उद्भिच्छब्दात्पुनरुद्धिन्नामको यो यागः स एवात्र यागविशेष इति विज्ञायत  इत्यर्थः~। एवं च सिद्धमुद्भिच्छब्दस्य धात्वर्थसामानाधिकरण्येनान्वयं फलितमाह\textendash {\br उद्भिदेति । ननू}द्भिच्छब्दस्य नीलमुत्पलमित्यत्र नीलपदस्योत्पपदसामानाधिकरण्यवद्यजिसामानाधिकरण्यं भवेत्, किं नामधेयत्वेनेति चेत्, {\qt न}; वैषम्यात्~। तथा हि\textendash तत्र हि {\qt नील} पदस्यार्थो नीलगुण उत्पलपदार्थादुत्पलरूपद्रव्यादतिरिक्तो भवति, लक्षणया तु नीलपदस्य तादृशद्रव्यपरत्वेनोत्पलपदसामानाधिकरण्यमुपपद्यते~। उद्भिच्छब्दस्य तु यज्यवगतयागविशेषान्नातिरिक्तोऽर्थोऽस्ति, तस्यैव तत्र विशेषत्वसमर्पकत्वात्~।
ततश्चार्थान्तरवाचकत्वाभावेन नोद्भिच्छब्दस्य नीलशब्दस्योत्पलशब्दसामानाधिकरण्यवद्यजिसामानाधिकरण्यमुपपद्यते, किं तर्हि {\qt वैश्वदेव्यामिक्षा} इत्यत्रामिक्षापदस्य
वैश्वदेवीशब्दसामानाधिकरण्यवत्, वैश्वदेवीशब्दस्य हि देवतातद्धितत्वात् तस्य च {\qt सास्य देवता}  इति सर्वनामार्थे स्मरणात् सर्वनाम्नां चोपस्थितविशेषवाचित्वेन
विशेषपरत्वम्~। तत्र कोऽसौ वैश्वदेवीशब्दोपात्तो विशेष इत्यपेक्षायाम् आमिक्षापदसांनिध्यादामिक्षारूपो विशेष इत्यवगम्यते~। यथाहुः\textendash {\qt आमिक्षां देवतायुक्तां वदत्येवैष तद्धितः~।~आमिक्षापदसांनिध्यादस्यैव विषयार्पणम्}  इति~। तस्माद्यथा वैश्वदेवीशब्दोपात्तविशेषसमर्पकत्वेनामिक्षापदस्य वैश्वदेवीशब्देन सामानाधिकरण्यमेवं सामान्यस्याविधेयत्वाद्यज्यवगतयागविशेषसमर्पकत्वेनैवोद्भिच्छब्दस्य  यजिसामानाधिकरण्यमित्युक्तप्रकारेणैव नामधेयानामन्वयः साधुरिति~। तथा चोक्तं  {\qt तदधीनत्वाद्यागविशेषसिद्धेः} इति~।~\\

 तच्च निमित्तचतुष्ट्याद्भवतीत्याह\textendash {\br नामधेयत्वं चेति~।} निमित्तचतुष्ट्यं  निर्दिशति\textendash {\br मत्वर्थेत्यादिना~।} तत्राद्यनिमित्तविषयमुदाहरति {\br तत्रोद्भिदा यजेतेति~। तत्रेति~।} चतुर्षु मध्य इत्यर्थः~। अत्र मत्वर्थलक्षणापत्तिप्रदर्शनाय 
\lfoot{७ अ०}
\newpage
%%%%%%%%%%%%%%%%%%%%%%%%%%%%%%%%%%%%%%%
\lfoot{}
\fancyhead[RE]{[ नामधेयत्वे\textemdash\ }
{\bl\noindent
भयात्~। तथा हि न तावदनेन वाक्येन फलं प्रति यागविधानम्,  तं प्रति च गुणविधानं युज्यते, वाक्यभेदापत्तेः~। उद्भिच्छब्दस्य  गुणसमर्पकत्वे च यागस्याप्यप्राप्तत्वात् गुणविशिष्टकर्मविधानं }\\
\hrule
\vspace{3mm}
\noindent
तावद्वाक्यभेदमापादयति {\br न तावदिति~।} अस्मिन्पक्षे तूद्भिद्यते भूमिरनेनेति व्युत्पत्त्या खनित्रवाच्यसावुद्भिच्छ्दो भवेत्, तथा च {\qt उद्भिदा यजेत  पशुकाम;} इत्यनेन वाक्येन यागेन पशुं भावयेद्यागं च खनित्रेण भावयेदिति फलं प्रति यागविधानं यागं प्रति च गुणविधानं क्रियेत, तच्च न युज्यत इत्यर्थः~। तत्र हेतुमाह\textendash {\br वाक्येति~।} आवृत्तिरूपवाक्यभेदापत्तेरित्यर्थः~। नन्वनेन वाक्येन खनित्ररूपो गुण एव विधीयते~। {\qt दध्ना जुहोति}  इत्यनेन गुणविधिना समानत्वात्~। न चात्र पशुफलकः कश्चिद्यागो विधीयत इति वाच्यम्~। पशूनां  गुणफलत्वात्~। यथा गोदोहनेन पशुकामस्येत्यत्र पशवो गोदोहनगुणस्य फलं  तथेह खनित्रगुणस्य फलमस्तु, यदि {\qt चमसेनापः प्रणयति}  इति विहितं प्रकृतमपां प्रणयनमाश्रित्य गोदोहनं विधीयते तर्ह्यत्रापि ज्योतिष्टोमेन यजेतेति विहितं प्रकृतं ज्योतिष्टोममाश्रित्य खनित्रं विधीयतां तस्माद्गुणविधिरित्याशङ्क्य {\qt पशुकामो यजेत}  इत्यस्य पदद्वयस्यायमर्थः {\qt पशुरूपं फलं यागेन कुर्यादिति~}। तत्र केन यागेनेत्यपेक्षायां उद्भिदेति तृतीयान्तं पदं यागनामत्वेनान्वेति~। उद्भिद्यते  पशुफलमनेन यागेनेति निरुक्त्या नामत्वमुद्भित्पदस्योपपद्यते~। न चैवमपि  गुणविधिनामधेयत्वयोः शब्दनिर्वचनसाम्यान्न निर्णय इति वाच्यम्~। सामानाधिकरणस्य निर्णायकत्वात्~। तथा हि\textendash उद्भिन्नामकेन यागेन पशुरूपं फलं  कुर्यादित्युक्ते सामानाधिकरण्यं लभ्यते, गुणविधित्वे तु खनित्रेण साध्यो
यो यागस्तेन तादृशफलं कुर्यादित्येवं वैयधिकरण्यं स्यात् {\qt तच्चायुक्तम्}; किंच  नानेन वाक्येन ज्योतिष्टोमे खनित्ररूपो गुणो विधातुं शक्यते, तस्य सोमेन 
यजेतेत्युत्पत्तिशिष्टसोमरूपगुणावरुद्धत्वात्~। किंच यद्यस्योद्भिच्छब्दस्य खनित्ररूपगुणसमर्पकत्वं स्वीक्रियते तदा यत्र तेन गुणः समर्पणीयस्तादृशकर्मणोऽप्यप्राप्तत्वादनेन वाक्येन खनित्ररूपगुणविशिष्टकर्मविधानमेव वक्तव्यमन्यथा  वाक्यभेदप्रसङ्गात्~। ततश्चोद्भिच्छब्दार्थरूपखनित्रवता यागेनेति सामानाधिकरण्येनान्वयो भविष्यति~। तथा च मत्वर्थलक्षणापत्तिरिति परिहरति {\br उद्भि-}
\newpage
%%%%%%%%%%%%%%%%%%%%%%%%%%%%%%%%%%%%%%%%%%%%%%
\fancyhead[LO]{निमित्तचतुष्टयम् ]}
{\bl\noindent
वाच्यम्~। उद्भिद्वता यागेन पशुं भावयेदिति विशिष्टविधौ च  मत्वर्थलक्षणेत्युक्तमेव~।~}
\begin{center}
 \textbf{नामधेयत्वस्य वाक्यभेदप्रसङ्गरूपद्वितीयनिमित्तोदाहरणम् }   
\end{center}
 
{\bl{\qtl चित्रया यजेत पशुकामः}  इत्यत्र चित्राशब्दस्य कर्मनामधेयत्वं वाक्यभेदभयात्~। तथा हि न तावदत्र गुणविशिष्टयागविधानं संभवति~। {\qtl दधि मधु पयो घृतं धाना उदकं तण्डुलास्तत्संसृष्टं प्राजापत्यम्}  इत्यनेन गुणस्य विहितत्वात्तद्विशिष्टयागविध्यनुपपत्तेः~। यागस्य फलसंबन्धे गुणसंबन्धे च विधीयमाने }\\
\hrule
\vspace{3mm}
\noindent
{\br च्छब्दस्येत्यादिना~। उक्तमेवेति~।} {\qt सोमेन यजेत} इति विध्यर्थनिरूपणप्रस्ताव इति शेषः~।\\

 इदानीं द्वितीयं वाक्यभेदप्रसङ्गरूपं नामधेयत्वस्य निमित्तमुदाहरणद्वारा प्रदर्शयति {\br चित्रया यजेतेति~।} वाक्यभेदमेवोपपादयति {\br तथा हीत्यादिना~। अत्रेति~।} {\qt चित्रया यजेत पशुकामः} इत्यस्मिन्वाक्य इत्यर्थः~। {\br गुणविशिष्टेति~।} चित्राशब्दार्थभूतचित्रवर्णककिंचिद्गुणविशिष्टेत्यर्थः~। अत्र  विशिष्टविधानासंभवे हेतुमाह\textendash {\br दधीत्यादिना~। तत्संसृष्टं प्राजापत्यमितीति~।} तैर्दध्यादिभिर्व्यैर्युक्तं प्रजापतिदेवताकं कर्मेत्यर्थः~। {\br तद्विशिष्टेति~।} निरुक्तगुणविशिष्टेत्यर्थः~।
दधीत्यादिप्रकृतवाक्यस्यैतत्कर्मण उत्पति  वाक्यत्वादस्य कर्मण उत्पत्तिशिष्टदध्यादिगुणावरुद्धत्वान्न तत्र गुणान्तरं विधातुं  शक्यत इति भावः~। अत्र दध्यादिवाक्ये दध्यादीनि षडेव द्रव्याणि श्रुतावम्नातानि , उदकपदं तु प्रमादादायातम्, श्रुतितात्पर्यज्ञैर्माधवाचार्यैस्तथैवास्य  वाक्यस्य व्याख्यातत्वात्~। तथा च तद्वचनं {\qt दध्यादीनि विचित्राणि
देयद्रव्याणि  षडाम्नातानि} इति~। नन्वत्र वाक्ये ह्युत्पत्तिवाक्यसिद्धस्वरूपस्य यागस्य पशुरूपफलसंबन्धो विचित्रद्रव्यरूपगुणसंबन्धश्च विधीयते~। ततश्च न कर्मनामधेयत्वं चित्राशब्दस्येत्याशङ्क्याह\textendash {\br यागस्येति~।} तथा च यागेन पशुं भावयेद्यागं च  तादृशगुणेन भावयेदिति यागस्य गुणफलोभयसंबन्धे विधीयमाने सत्यावृत्तिलक्षणो वाक्यभेदो दुर्वार इति भावः~। उपपादितं वाक्यभेदप्रसङ्गमुपसंहरति-
\newpage
%%%%%%%%%%%%%%%%%%%%%%%%%%%%%%%%%%%
\fancyhead[RE]{[ नामधेयत्वे नि० चतु० ]}
{\bl\noindent
वाक्यभेदः~। तस्माच्चित्राशब्दः कर्मनामधेयम्~। तथा च चित्रायागेन पशुं भावयेदिति सामानाधिकरण्येनान्वयान्न वाक्यभेदः~।  प्रकृतेषटेरनेकद्रव्यत्वेन\footnotemark\ चित्राशब्दवाच्यत्वोपपत्तिः~।}\\
\hrule
\vspace{3mm}
\noindent
{\br तस्मादिति~।} सिद्धे चित्राशब्दस्य कमनामधेयत्वे वाक्यं योजयति {\br तथा चेति~। सामानाधिकरण्येनेति~।} यजिधात्वर्थयागसामानाधिकरण्येन  नामधेयस्यान्वयाच्चित्रानामकेन यागेन पशुं भावयेदित्याकारकान्न निरुक्तवाक्यभेदापत्तिरित्यर्थः~। {\br अनेकद्रव्यत्वेनेति~।} दध्यादिविचित्रानेकद्रव्यसाध्यत्वेनेत्यर्थः~। {\br ननु} चित्राशब्दाच्चित्रत्वस्त्रीत्वयोः प्रतीतेः स्त्रीरत्नस्य च स्वभावतः  प्राणिधर्मत्वात्प्रकृते दध्यादिद्रव्यके कर्मणि निवेशासंभवान्नानेन वाक्येन प्रकृते  कर्मणि तादृशगुणविधानं क्रियते , किंतु प्राणिद्रव्यके कर्मणि ; अनारभ्याधीतानां चाङ्गानां प्रकृतिमात्रे प्रवेशाङ्गीकारात्~। चित्रावाक्यस्याप्यनारभ्याधीतत्वात्सर्पशुयागप्रकृतिभूते प्राणिद्रव्यकेऽग्नीषोमीये कर्मणि तेन गुणो विधीयते~। तथा च  {\qt अग्नीषोमीयं पशुमालभेत}  इति विहितं पशुयागमत्र वाक्ये यजेतेति पदेनानूद्य  तत्र चित्रापदेन चित्रत्वस्त्रीत्वरूपौ गुणौ विधीयेते इति चेत् , न;
चित्रत्वेन स्त्रीत्वेन  च तं भावयेदिति द्वयोर्गुणयोर्विधाने वाक्यभेदप्रसङ्गात्~। प्राप्ते कर्मण्यनेकगुणविधाने वाक्यभेदप्रसङ्गस्य सर्वसंमतत्वात्~। तथा चोक्तम्\textendash {\qt प्राप्ते कर्मणि नानेको विधातुं शक्यते गुणः~। अप्राप्ते तु विधीयेरन्बहवोऽप्येकयत्नतः} इति~॥\\

 {\br नन्व}त्र वाक्यभेदपरिहाराय गुणद्वयविशिष्टं पशुद्रव्यरूपं कारकं विधीयत  इति चेत् ,{\qt न}; गौरवलक्षणवाक्यभेदप्रसङ्गात्~। किंच दध्यादिवाक्यं प्रकृतस्य 
चित्रानामकस्य यागस्योत्पत्तिवचनं भवति; यागस्वरूपभूतयोर्दध्यादिद्रव्यप्रजापतिदेवतयोरत्रोपदिश्यमानत्वात्, उत्पन्नस्य च तस्य यागस्य {\qt चित्रया यजेत पशुकामः} इत्येतत्फलवाक्यमत्र यागस्य फलसंबन्धबोधनात्~। एवं च  सति प्रकृतार्थो लभ्येत~। अग्नीषोमीयपश्वनुवादेन, तादृशगुणविधाने तु प्रकृतहानाप्रकृतप्रक्रिये प्रसज्येयातां, लिङ्प्रत्ययस्य चानुवादत्वाङ्गीकारान्मुख्यो विध्यर्थो बाध्येत, तस्माच्चित्रापदं नामधेयमेव न गुणविधिरिति ध्येयम्~।

\blfootnote{पाठा०\textemdash\ $^{१}$द्रव्यवत्त्वे चित्रा.}
\newpage
%%%%%%%%%%%%%%%%%%%%%%%%%%%%%%%%%
\fancyhead[LO]{[ देवतारूपेणा०प्रश्नः ]}
\begin{center}
    \textbf{तत्प्रख्यशारुत्रान्नामधेयत्वम्}
\end{center}

{\bl {\qtl अग्निहोत्रं जुहोति} इत्यत्राग्निहोत्रशब्दस्य कर्मनामधेयत्वं  तत्प्रख्यशास्त्रात्~। तस्य गुणस्य प्रख्यापकस्य प्रापकस्य शास्त्रस्य~। विद्यमानत्वात्, अग्निहोत्रशब्दः कर्मनामधेयमिति यावत्~। नन्वयं गुणविधिरेव कुतो न ? इति चेत् ,{\qtl  न} ; यद्यग्नौ होत्रमस्मिन्निति  सप्तमीसमासमाश्रित्य होमाधारत्वेनाग्निरूपो गुणो विधेयस्तदा 
{\qtl यदाहवनीये जुहोति}  इत्यनेनैवाग्नेः प्राप्तत्वात्तद्विधानानर्थक्यम्~। अग्नये होत्रमिति चतुर्थीसमासमाश्रित्य अग्निदेवतारूपगुणोऽनेन  विधीयत इति चेत्,{\qtl न}; तद्देवतायाः शास्त्रान्तरेण प्राप्तत्वात्~।}
\begin{center}
 \textbf{देवतारूपेणाग्चिप्रापकशास्त्रप्रश्नः }   
\end{center}
 
{\bl किं तच्छास्त्रान्तरमिति चेत्, {\qtl यदग्रये च प्रजापतये च सायं}}\\
\hrule
\vspace{3mm}
\noindent
 इदानीं तत्प्रख्यशास्त्ररूपात्तृतीयनिमित्तान्नामधेयत्वमग्निहोत्रशब्दस्य प्रदर्शयति {\br अग्निहोत्रमिति~।} {\qt तत्प्रख्यं चान्यशास्त्रम्}  इति हि तत्प्रख्यशास्त्रसूत्रम्~। तस्य फलितार्थमाह\textendash {\br तस्येत्यादिना~।} \\

 {\br नन्व}ग्निहोत्रं जुहोतीत्यत्राग्निरूपस्य गुणस्यैव विधिर्न नामधेयत्वमग्निहोत्रशब्दस्य स्वीकर्तव्यमित्याशङ्कते {\br नन्वयमिति~।} यद्यत्र सप्तमीसमासमाश्रित्य 
होमाधारत्वेनाग्निरूपस्य गुणस्य विधानं स्वीक्रियते तदा तदाधारत्वेनाग्निरूपस्य  गुणस्य वाक्यान्तरेण प्राप्तत्वात्तत्त्वेन तद्विधानस्यानर्थक्यमापद्येतेत्याह\textendash
{\br यद्यग्नावित्यादिना~।} चतुर्थीसमासमाश्रित्यात्राग्निदेवतारूपस्य गुणस्य विधानमाशङ्कते {\br चतुर्थीत्यादिना~।} नात्र देवतारूपेणाग्निरूपस्य गुणस्य विधानमुपपद्यत इति समाधत्ते\textendash {\br नेति~।} तत्र हेतुमाह\textendash {\br तद्देवताया इति~।} अग्निरत्र तच्छब्दार्थः~। \\

 देवतारूपेणाग्निप्रापकं शास्त्रं पृच्छति {\br किमिति~।} केषांचिन्मतानुसारेणोत्तरमाह\textendash {\br यदग्रये चेति~।} अत्राग्निर्ज्योतिरित्यादिमन्त्रवर्णप्राप्तमग्निमनूद्य 
तत्समुच्चितप्रजापतिमात्रविधाने लाघवं तदुभयसमुच्चितस्यैवात्र विधाने गौरव-
\newpage
%%%%%%%%%%%%%%%%%%%%%%%%%%%%%%%%%%%%%%%%
\fancyhead[RE]{[ देवतारूपेणा प्रश्नः ]}
{\bl {\qtl\noindent जुहोति}  इति केचित्~। अपरे तु  {\qtl अग्निर्ज्योतिर्ज्योतिरग्निः स्वाहा} इति मन्त्रवर्ण एवाग्निरूपदेवताप्रापकः~। नन्वग्नेर्मान्त्रवर्णिकत्वे प्रजाप्रतिदेवतया बाधः स्यात्; मन्त्रवर्णस्य चतुर्थीतो दुर्बलत्वात्~। यथाहुः {\qtl तद्धितेन चतुर्थ्या वा मन्त्रवर्णेन वा पुनः~। देवताया }}\\
\hrule
\vspace{3mm}
\noindent
मिति न समुच्चितोभयविधानं {\qt यदग्नये च प्रजापतये च सायं जुहोति}  इत्यत्र स्वीकर्तव्यमित्यस्वरसबीजं केचिदित्यनेन सूचितम्~।~अधुना सिद्धान्तमतेनोत्तरमाह\textendash {\br अपरे त्विति~।} किंच {\qt अग्निर्ज्योतिर्ज्योतिरग्निः स्वाहेति सायं जुहोति} इति विहितेन मन्त्रेण प्राप्तमग्निमनूद्य तत्समुच्चितस्यं प्रजापतेः {\qt यदग्रये च प्रजापतये च सायं  जुहोति}  इत्यत्र सायंकालेऽग्निहोत्रदेवतात्वं विधीयते~। {\qt सूर्यो ज्योतिर्ज्योतिः सूर्यः स्वाहेति प्रातर्जुहोति} इति विहितेन च मन्त्रेण प्राप्तं सूर्यमनूद्य तत्समुच्चितस्य च  तस्य {\qt यत्सूर्याय च प्रजापतये च प्रातर्जुहोति} इत्यत्र प्रातःकालेऽग्निहोत्रदेवतात्वं  विधीयते~। तेनाग्नेर्मान्त्रवर्णिकत्वे प्रजापतिविधेरेकेनैव वाक्येन सिद्धेः {\qt यदग्नये च  प्रजापतये च सायं जुहोति, यत्सूर्याय च प्रजापतये च प्रातर्जुहोति} इत्यत्र  वाक्यद्वयं  व्यर्थमिति निरस्तम्~। सायं होमेऽग्निसमुच्चितस्य प्रजापतेर्विधानं प्रातर्होमे सूर्यसमुच्चितस्य च तस्यैकेन वाक्येन कर्तुमशक्यत्वादित्यलं विस्तरेण , अधिकं तु न्यायप्रकाशे द्रष्टव्यम्~। नन्वग्नेर्मान्त्रवर्णिकत्वे प्रजापतिना तस्य बाधः
स्यात्, प्रजापतेश्चतुर्थ्या देवतारूपेण प्राप्तत्वेन प्रबलत्वात् , अग्नेस्तु मन्त्रवर्णप्राप्तत्वेन दुर्बलत्वाच्च~। न च  {\qt सास्य देवता}  इति तद्धितप्रत्ययस्य देवतात्वे स्मरणवच्चतुर्थी न
देवतात्वे स्मर्यते  {\qt संप्रदाने चतुर्थी}  इति संप्रदानमात्रे तस्याः स्मरणात्~। तस्मात्प्रजापतिना कथमग्नेर्बाधः स्यादिति वाच्यम् ; त्यज्यमानद्रव्योद्देशत्वे सति प्रतिग्रहीतृत्वस्य संप्रदानपदार्थत्वेन त्यज्यमानद्रव्योद्देश्यत्वरूपस्य देवतात्वस्य संप्रदानस्वरूपान्तर्गतत्वात्~। ततश्चतुर्थीतः संप्रदानैकदेशतया देवतात्वप्रत्ययो भवत्येव, मन्त्रवर्णात्तु न देवतात्वप्रतीतिरस्ति किंत्वधिष्ठानत्वमेव ततः प्रतीयते~। तस्मान्मन्त्रवर्णश्चतुर्थीतो दुर्बल एव~। तथा च प्रबलप्रमाणबोधितप्रजापतिदेवतया दुर्बलप्रमाणबधिताग्नेर्बाधो दुर्वार एवेत्याशयेन शङ्कते {\br नन्विति~।} तत्र संमतिमाह\textendash  {\br यथाहुरिति~। तत्रेति~।} तद्धितादिषु मध्य इत्यर्थः~। {\br परमिति~।} तद्धितापेक्षया  चतुर्थ्या दौर्बल्यं चतुर्थ्यपेक्षया च मन्त्रवर्णस्य दौर्बल्यं भवतीति परं परं दुर्बलं 
\newpage
%%%%%%%%%%%%%%%%%%%%%%%%%%%%%%%%%%%
\fancyhead[LO]{तद्य्वप०कर्म०धेयत्वम् ]}
{\bl\noindent {\qtl विधिस्तत्र दुर्बलं तु \footnotemark परं परम्} इति चेत् ; न; {\qtl यदग्रये च प्रजापतये  च सायं जुहोति}  इत्यत्र न केवलं प्रजापतिविधानम्, किंतु मन्त्रवर्णप्राप्तमग्निमनूद्य तत्समुच्चितप्रजापतेः~। एवं च न बाधः,  केवलप्रजापतिविधानाभावात्~। न चात्र समुच्चितोभयविधानमेव  कथं नेति वाच्यम्; समुच्चितोभयविधानापेक्षयान्यतः प्राप्तमग्निमनूद्य तत्समुच्चितप्रजापतिमात्रविधाने लाघवात्~। एवं प्रयाजेषु समिदादिदेवतानां {\qtl समिधः समिधो अग्न आज्यस्य व्यन्तु}  इत्यादिमन्त्रवर्णेभ्यः प्राप्तत्वात्~। {\qtl समिधो यजति}  इत्यादिषु समिदादिशब्दास्तत्प्रख्यशास्त्रात्कर्मनामधेयम्~।}
\begin{center}
 \textbf{तद्य्वयपदेशेन कर्मनामधेयत्वम् }   
\end{center}
 
{\bl {\qtl श्येनेनाभिचरन् यजेत}  इत्यत्र श्येनशब्दस्य कर्मनामधेयत्वं }\\
\hrule
\vspace{3mm}
\noindent
बोध्यमित्यर्थः~। ततः परमिति पाठे तु ततस्ततः परं दुर्बलप्रमाणमिति वीप्सापरत्वेन व्याख्येयम्~। यदि च {\qt प्रजापतये जुहोति} इति केवलप्रजापतिविधानं स्यात्तदा तु भवेदपि प्रजापतिनाग्नेर्बाधः, परंतु न तथा विधानं क्रियत इति परिहरति {\br यदग्नये चेत्यादिना~।} प्रजापतेरित्यत्र विधानमित्यस्यानुषङ्गः~।  {\br एवं चेति~।} मन्त्रवर्णप्राप्तमग्निमनूद्य तत्समुच्चितस्य प्रजापतेर्विधाने चेत्यर्थः~। {\br न बाध इति~।} न प्रजापतिनाग्नेर्बाध इत्यर्थः~। तत्र हेतुमाह\textendash {\br केवलेति~।} {\br ननु} यदग्नये च प्रजापतये चेत्यस्मिन्वाक्ये होमानुवादेन समुच्चितस्यैवोभयस्य  विधानं क्रियत इति कथं न स्वीक्रियत इत्याशङ्क्य तयोः समुच्चितयोर्विधानापेक्षया मन्त्रवर्णतः प्राप्तमग्निमनूद्य लाघवेन
तत्समुच्चितप्रजापतेर्विधानमेवोचितमिति परिहरति {\br न चेत्यादिना~।} तत्प्रख्यशास्त्रान्नामधेयत्वे उदाहरणान्तरमाह\textendash {\br एवमिति~।} समिदादिशब्दा इत्यत्र {\qt आदि}  पदेन तनूनपातादयः शब्दा गृह्यन्ते~।\\

 इदानीं चतुर्थनिमित्तेन तद्य्वपदेशरूपेण श्येनशब्दस्य कर्मनामधेयत्वं प्रदर्शयति {\br श्येनेति~। कर्मनामधेयत्वमिति~।} नन्वत्र श्येनशब्दस्य कर्मनाम
\blfootnote{पाठा०\textemdash\ $^{१}$ततः परम्. }
\newpage
%%%%%%%%%%%%%%%%%%%%%%%%%%%%%%%%%%%%%%%%%%%%%%
\fancyhead[RE]{[ तद्य्वपदे०कर्म०धेयत्वम् ]}
{\bl\noindent 
तद्य्वपदेशात्~। तेन व्यपदेशादुपमानात्तदन्यथानुपपत्तेरिति यावत्~। तथा हि\textendash यद्विधेयं तस्य स्तुतिर्भवति~। यद्यत्र श्येनो विधेयः स्यात्, तदार्थवादैस्तस्यैव स्तुतिः कार्या~। अत्र {\qtl यथा वै श्येनो निपत्यादत्ते एवमयं द्विषन्तं भ्रातृव्यं निपत्यादत्ते} इत्यनेनार्थवादेन श्येनः स्तोतुं न शक्यः, श्येनोपमानेनार्थान्तरस्तुतेः  क्रियमाणत्वात्~। न च श्येनोपमानत्वेन स एव स्तोतुं शक्यते,  उपमानोपमेयभावस्य भिन्ननिष्ठत्वात्~। यदा तु श्येनसंज्ञको यागो  विधीयते तदार्थवादेन श्येनोपमानेन तस्य स्तुतिः कर्तुं शक्यत
इति श्येनशब्दः कर्मनामधेयं तद्य्वपदेशादिति~।~}\\
\hrule
\vspace{3mm}
\noindent
धेयत्वं न भवति, किंतु सोमयागे नित्यं सोमद्रव्यं बाधित्वा तस्य स्थाने पक्षिद्रव्यरूपो गुणः काम्यो विधीयते, तथा सति श्येनशब्दस्य पक्षिणि लोकप्रसिद्धा 
रूढिरुपपन्ना भवतीत्याशङ्क्य श्येनशब्दस्य कर्मनामधेयत्वे हेतुमाह\textendash  {\br तद्व्यपदेशादिति~।} तद्य्वपदेशशब्दं व्याचष्टे {\br तेनेति~।} श्येनेनेत्यर्थः
~। {\br तदिति~।}  उपमानोपमेयभावस्य भेदघटितत्वादेवार्थवादवाक्ये श्येनोपमानेन विधेयस्तुतेः श्येननामककर्मविशेषं विनानुपपत्तर्नात्र पक्षिद्रव्यसपो गुणो विधातुं
शक्यत इति भावः~। तदन्यथानुपपत्तिमेवोपपादयति {\br तथा हीत्यादिना~। यदिति~।}  विधेयस्य स्तुतेः कर्तव्यत्वादित्यर्थः~। {\br अत्रेति~।} श्येनेनाभिचरेन्
यजेतेत्यत्रेत्यर्थः~। {\br श्येन इति~।} श्येननामकपक्षिविशेष इत्यर्थः~। {\br तस्यैवेति~।} श्येननामकपक्षिविशेषस्यैवेत्यर्थः~। {\br यथेति~।} यथा श्येनः पक्षिविशेषो निपत्य
मत्स्यादीञ्जन्तूनादत्ते, एवमयं श्येननामको यागो द्विषन्तं भ्रातृव्यं शत्रुं निपत्यादत्त इत्यर्थः~। यमभिचरति श्येनेनेति वाक्यशेषः~। {\br अवेति~।} अत्र प्रकृतेऽनेन
श्येनार्थवादेन  श्येनः पक्षिविशेष एव स्तोतुं न शक्यते इत्यर्थः~। तत्र हेतुमाह\textendash  {\br श्येनेति~।} ननु श्येनार्थवादोपमानेन श्येन एव पक्षिविशेषः कथं न स्तोतुं शक्यः
स्यादित्यत आह\textendash {\br न च श्येनोपमानत्वेनेति~।} तदशक्यत्वे हेतुमाह\textendash {\br उपमानेति~।} यद्यत्र श्येनसंज्ञकस्य यागस्य विधेयत्वं स्वीक्रियते तदा
तादृशार्थवादोपमानेन तस्य शयेनसंज्ञकस्य यागस्य स्तुतिः कर्तुं शक्या भवत्येवेत्याह\textendash {\br तदेत्यादिना~।} फलितमुपसंहरति {\br इतीति~।} एवमुक्तेन प्रकारेणेत्यर्थः~।
\newpage
%%%%%%%%%%%%%%%%%%%%%%%%%%%%%%%%%%%
\fancyhead[LO]{[ कर्म०उत्प०गुण० ]}
\begin{center}
 \textbf{कर्मनामधेयत्वे उत्पत्तिशिष्टगुणबलीयस्त्वम् }   
\end{center}
 
{\bl उत्पत्तिशिष्टगुणबलीयस्त्वमपि पञ्चमं नामधेयनिमित्तमिति केचित् । यथा {\qtl वैश्वदेवेन यजेत}  इत्यादौ~।~अत्रोत्पत्तिशिष्टाग्न्यादीनां बलीयस्त्वाद्वैश्वदेवशब्दस्य विश्वदेवदेवताभिधायकत्वं न }\\
\hrule
\vspace{3mm}

 अत्र कर्मनामधेयत्वे चोत्पत्तिशिष्टगुणबलीयस्त्वं पञ्चममपि निमित्तं भवतीति  केषांचिन्मतमाह {\br उत्पत्तिशिष्टेति~।} तत्रोदाहरणमाह\textendash {\br यथेति~। अत्रेति~।} 
अस्मिन्वाक्ये वैश्वदेवशब्दस्य विश्वदेवदेवताविधायकत्वं न संभवत्युत्पत्तिशिष्टाग्न्यादीनां बलीयस्त्वादित्यन्वयः~। फलितार्थमुपसंहरति {\br इतीति~।} तस्येति शेषः~। {\br अत्रेदं बोध्यम्} चातुर्मास्ये चत्वारि पर्वाणि {\qt वैश्वदेवो वरुणप्रघासः  साकमेधः शुनासीरीयश्चेति~}। तेषु प्रथमे पर्वण्यष्टौ यागा विहिताः {\qt आग्नेयमष्टाकपालं निर्वपति, सौम्यं चरुं, सावित्रं द्वादशकपालं, सारस्वतं चरुं, पौष्णं चरुं, मारुतं सप्तकपालं, वैश्वदेवीमामिक्षां, द्यावापृथिव्यमेककपालमिति~}। तेषामष्टानां यागानां संनिधाविदमाम्नायते {\br वैश्वदेवेन यजेतेति~।} अत्र चाग्नेयादीन्यागान्यजेतेत्यनूद्य वैश्वदेवशब्देन देवतारूपो गुणस्तेषु विधीयते~। यद्यपि वैश्वदेव्यामामिक्षायां विश्वेदेवाः प्राप्तास्तथाप्याग्नेयादिषु सप्तसु
यागेष्वप्राप्तत्वाद्विधीयन्ते~। तेष्वप्यग्न्यादिदेवताः सन्तीति चेत्तर्हि गत्यभावात्तेषु देवता विकल्प्यन्ताम्~। नामधेयत्वे तु नाममात्रस्याविधेयत्वाद्रव्यदेवतयोरभावेन
यागस्यात्र  स्वरूपासंभवाच्छ्रूयमाणो विधिरनर्थकः स्यात्तस्माद्गुणविधिरिति पूर्वपक्षः~। उत्पत्तिवाक्यैर्विहितानाग्नेयादीनष्टौ यागान्यजेतेत्यनूद्याष्टानां सङ्घे वैश्वदेवशब्दो नामत्वेनोपवर्ण्यते~। न च विधित्वाभावेऽपि नामोपदेशस्य वैयर्थ्यमिति वाच्यम्~। {\qt प्राचीनप्रवणे वैश्वदेवेन यजेत} इत्यादिषु वैश्वदेवशब्देनैकेनैवाष्टानां सङ्घस्य व्यवहर्तव्यत्वेनार्थवत्त्वोपपत्तेः~। नामप्रवृत्तिनिमित्तभूता निरुक्तिस्तु द्विधा~। आमिक्षायागे विश्वेषां देवानामिज्यमानतया तत्सहचरितानां {\br सर्वेषां}
छत्रिन्यायेन  वैश्वदेवत्वमिति, विश्वेदेवा अष्टानां कर्तार इति वा तेषां वैश्वदेवत्वम् ~। तथा च  ब्राह्मणं {\qt यद्विश्वेदेवाः समयजन्त तद्वैश्वदेवस्य वैश्वदेवत्वम्}  इति~। देवताविकल्पस्तु समानबलत्वाभावान्न युज्यते, अग्न्यादय उत्पत्तिशिष्टत्वात्प्रबलाः विश्वेदेवा उत्पन्नशिष्टत्वाद्दुर्बलाः तस्माद्वैश्वदेवशब्दः कर्मनामधेयमिति सिद्धान्तः~।~ वस्तुगतिमाश्रित्य तत्प्रख्यशास्त्रादेव वैश्वदेवशब्दस्य कर्मनामधेयत्वमाह\textendash 
\newpage
%%%%%%%%%%%%%%%%%%%%%%%%%%%%%%%%%%%%
\fancyhead[RE]{[ निषेधमीमांसा ]}
{\bl\noindent
संभवतीति कर्मनामधेयत्वम्~। वस्तुतस्तु तत्प्रख्यशास्त्रादेवास्य कर्मनामधेयं प्रकृतयागे विश्वदेवरूपगुणसंप्रतिपन्नशास्त्रस्यार्थवादरूपस्यैव सत्त्वात्~।{\qtl यद्विश्वेदेवाः समयजन्त तद्वैश्वदेवस्य वैश्वदेवत्वम्}  इति~।}
\begin{center}
 \textbf{अथ निषेधमीमांसा}   
\end{center}
 
{\bl पुरुषस्य निवर्तकं वाक्यं निषेधः, निषेधवाक्यानामनर्थहेतुक्रियानिवृत्तिजनकत्वेनैवार्थवत्तात्~। तथा हि यथा विधिः प्रवर्तनां  प्रतिपादयन्स्प्रवर्तकत्वनिर्वाहार्थं विधेयस्य यागादेरिष्टसाधनत्वमाक्षिपन्पुरूषं तत्र प्रवर्तयति, तथा {\qtl न कलञ्जं भक्षयेत}  इत्यादिनिषेधोऽपि निवर्तनां प्रतिपादयन्स्वनिवर्तकत्वनिर्वाहार्थं निषेध्यस्य  कलञ्जभक्षणस्य परानिष्टसाधनत्वमाक्षिपन्पुरुषं ततो निवर्तयति~।}\\
\hrule
\vspace{3mm}
\noindent
{\br वस्तुतस्त्विति~। प्रकृतयाग इति~।} वैश्वदेवनामकेऽष्टानामाग्नेयादीनां सङ्घात्मके प्रकृतयाग इत्यर्थः~। {\br विश्वदेवेति~।} प्रकृतयागे
विश्वदेवरूपगुणः संप्रतिपन्नः संप्राप्तो यस्मात्तादृशशास्त्रस्येत्यर्थः~। तत्र गुणप्रापकं शास्त्रमुदाहरति {\br यद्विश्वेदेवा इति~।} अस्य शास्त्रस्य कर्तृरूपेण प्रकृते यागे
विश्वदेवरूपगुणप्रापकत्वमिति भावः~। नामधेयस्य प्रयोजनं तु सर्वत्र व्यवहार एव~। न ह्यन्तरेण नामधेयमृत्विग्वरणादिष्वनेनाहं यक्ष्य इत्याख्यानोपायो लघुः कश्चिदस्ति, 
तस्माद्वैश्वदेवादिशब्दानां कर्मनामधेयत्वमेवेति सिद्धम्~।\\

 तदेवं मत्वर्थलक्षणादिनिमित्तचतुष्टयनिरूपणेन नामधेयस्य विधेयार्थपरिच्छेदकतयार्थवत्त्वं निरूपितम्~। अधुना निषेधवाक्यानामर्थवत्त्वनिरूपणाय निषेधवाक्यं लक्षयति\textendash {\br पुरूषस्य निवर्तकमिति~। निषेधेति~।} {\qt न कलञ्जं भक्षयेत्} इत्यादिनिषेधवाक्यानामनर्थहेतुभूतायाः कलञ्जभक्षणादिक्रियायाः
सकाशात्पुरुषस्य निवृत्तिजनकत्वेनैवार्थवत्त्वं न किंचित्कर्तव्यताप्रतिपादकत्वेनेति भावः,  यथा विधिवाक्यानां स्वप्रवर्तकत्वान्यथानुपपत्त्या विधेयार्थस्य
स्वर्गादिरूपेष्टसाधनत्वप्रत्यायनेन तत्र विधेयार्थे पुरुषप्रवृत्तिजनकत्वं तथा निषेधानामपि स्वनिवर्तकत्वान्यथानुपपत्त्या निषेध्यस्य पुरुषानिष्टसाधनत्वप्रत्यायनेन , ततः
कलञ्जभक्षणादेः पुरुषनिवृत्तिजनकत्वमिति दृष्टान्तदार्ष्टान्तिकाभ्यां निषेधवाक्यानामनर्थहेतुक्रियायाः पुरुषनिवृत्तिजनकत्वेनैवार्थवत्त्वमुपपादयति {\br तथा हीत्यादिना~।} 
\newpage
%%%%%%%%%%%%%%%%%%%%%%%%%%%%%%%%%%%%%%
\fancyhead[LO]{[लिङ०भा०नञर्थे० ]}
\begin{center}
 \textbf{लिङर्थशब्दभावनाया नञर्थेनान्वयः }   
\end{center}
 
{\doublespacing {\br {\br ननु} निषेधवाक्यस्य कथं निवर्तनाप्रतिपादकत्वमिति चेत्, उच्यते\textendash न तावदत्र धात्वर्थस्य नञर्थेनान्वयः, अव्यवधानेऽपि
तस्य प्रत्ययार्थभावनोपसर्जनत्वेनोपस्थितेः~। न ह्यन्योपसर्जनत्वे  नोपस्थितमन्यत्रान्वेतिः अन्यथा राजपुरुषमानयेत्यादावपि राज्ञः क्रियान्वयापत्तेः~। अतः प्रत्ययार्थस्यैव नञर्थेनान्वयः~। तत्रापि नाख्यातत्वांशवाच्यार्थभावनायास्तस्या लिङंशवाच्य-\\}}
\hrule
\vspace{3mm}
\noindent
 {\br ननु} निषेधवाक्यानां निवर्तनाप्रतिपादकत्वं न संभवति, {\qt न भक्षयेन्न  हन्तव्यः} इत्येवमादावव्यवधानेन नञर्थस्याभावस्य धात्वर्थेनान्वये सति धात्वर्थवर्जनकर्तव्यताया एव सर्वत्र वाक्यार्थत्वेन प्रतीयमानत्वात्~। तथा च यथा यजेतेत्यादौ यागकर्तव्यता वाक्यार्थो भवति तथा {\qt न कलञ्जं भक्षयेत् , ब्राह्मणो न हन्तव्यः}  इत्यादावपि तत्तद्धात्वर्थवर्जनकर्तव्यतैव वाक्यार्थो न निवर्तनेत्याशयेनाशङ्कते {\br नन्विति~।} नञर्थस्याभावस्य धात्वर्थेनाव्यवधानेऽपि धात्वर्थस्य प्रत्ययार्थभावनोपसर्जनत्वेनोपस्थितत्वान्न नञर्थेनान्वयः संभवति~। अन्यत्रोपसर्जनत्वेनान्वितस्यान्यत्रोपसर्जनत्वेनान्वयायोगादिति परिहरति {\br उच्यत इत्यादिना~। अत्रेति~।} {\qt न कलञ्जं भक्षयेत्} इत्यादावित्यर्थः~। अव्यवधानेऽपीत्यत्र धात्वर्थस्य नञर्थेनेत्यनुषङ्गः~। {\br तस्येति~।} धात्वर्थस्येत्यर्थः~। अन्योपसर्जनत्वेनोपस्थितस्याप्यन्यत्रोपसर्जनत्वेनान्वये को दोष इत्यत आह\textendash {\br न हीति~।}  तत्र बाधकं दोषमाह\textendash {\br अन्यथेति~।} अन्यविशेषणत्वेनोपस्थितस्याप्यन्यत्र विशेषणत्वेनान्वयस्वीकारे पुरुषोपसर्जनत्वेनोपस्थितस्य राज्ञोऽपि क्रियोपसर्जनत्वेनान्वयापत्तेरित्यर्थः~। धात्वर्थस्य नञर्थेनान्वयासंभवात्कलञ्जादिपदार्थस्यापि  कारकोपसर्जनत्वेनोपस्थितस्य तत्रान्वयासंभवाच्च परिशेषात्प्रत्ययार्थस्यैव नञर्थेनान्वयो भवतीत्याह\textendash {\br अत इति~।} किंच प्रत्ययार्थोऽपि द्विविधो भवत्याख्यात्तत्वांशवाच्यार्थभावना लिङंशवाच्या शब्दभावना चेति, तयोर्मध्येऽपि नाख्यातत्वांशवाच्यभूताया अर्थभावनाया नञर्थेऽन्वयः संभवतीत्याह\textendash {\br तत्रापीति~।}
\newpage
%%%%%%%%%%%%%%%%%%%%%%%%%%%%%%%%%%%%%%5
\fancyhead[RE]{[ नञ्स्वभावकथनम् ]}
{\bl\noindent
प्रवर्तनोपसर्जनत्वेनोपस्थितेः, किंतु लिङंशवाच्यशब्दभावनायाः,  तस्याः सर्वापेक्षया प्रधानत्वात्~।}
\begin{center}
 \textbf{नञ्स्वभावकथनम् }   
\end{center}
 
{\bl नञश्चैष स्वभावो यत्स्वसमभिव्याहृतपदार्थविरोधिबोधकत्वम्~। यथा घटो नास्तीत्यादौ अस्तीतिशब्दसमभिव्याहृतो नञ्  घटसत्त्वविरोधि घटासत्त्वं गमयति, तदिह लिङ्समभिव्याहृतो  नञ् लिङर्थप्रवर्तनाविरोधिनीं निवर्तनामेव बोधयति~। विधि-\\}
\hrule
\vspace{3mm}
\noindent
तत्र हेतुमाह\textendash {\br तस्या इति~। प्रवर्तनोपसर्जनत्वेनेति~।} शब्दभावनाविशेषणत्वेनेत्यर्थः~। लिङंशवाच्येत्यस्य लिङो योंऽशोलिङ्त्वरूपो धर्मस्तद्वाच्येत्यर्थो 
बोध्यः~। एवमग्रेऽपि~। {\br ननु} {\qt न कलञ्जं भक्षयेत्} इत्यादौ कलञ्जादिपदार्थस्य  धात्वर्थस्य च नञर्थेनान्वयो भवतैव निरस्तः, प्रत्ययार्थस्याप्यर्थभावनारूपस्य नञर्थेनानव्यानङ्गीकारे प्रत्ययार्थत्वाविशेषाच्छब्दभावनाया अपि तेनान्वयासंभवेनन्वितशब्दस्याप्रामाण्यापत्तिरित्याशङ्क्य परिहरति  {\br किंत्विति~।} तत्र 
हेतुमाह\textendash {\br तस्या इति~।} प्रत्ययार्थत्वं न नञर्थेनान्वये प्रयोजकं किंतु सर्वापेक्षया मुख्यत्वेनोपस्थितत्वं, तच्च शब्दभावनायामबाधितमिति भावः~।\\

 एवं शब्दभावनारूपस्य प्रत्ययार्थस्य प्रवर्तनाविशेषस्य नञर्थेनान्वये सिद्धे  नञः प्रत्ययार्थभूतप्रवर्तनाविरोधिनिवर्तनाबोधकत्वप्रदर्शनाय प्रथमं तत्स्वभावं प्रदर्शयति {\br नञश्चैष स्वभाव इति~।} अत्र स्वसमभिव्याहृतपदार्थविरोधिबोधकत्वं स्वान्वितपदार्थविरोधिबोधकत्वं बोध्यम्~। तेन कलञ्जादिपदार्थानामपि  विरोधिबोधकत्वं नञो दुर्वारं स्यात् , तेषामप्येकत्र सहपाठरूपनञ्समभिव्याहारस्य  सत्त्वादिति निरस्तम् ; तेषां नञन्वितपदार्थत्वाभावस्य दर्शितत्वेन नञस्तद्विरोधिबोधकत्वासंभवात्~। नञः स्वसमभिव्याहृतपदार्थविरोधिबोधकत्वं दृष्टान्तेन स्पष्टयति\textemdash\ {\br यथेत्यादिना~।~घटसत्त्वविरोधीति~।} नास्तीत्युक्ते कस्यासत्त्वमस्तीतिसत्त्वशब्दान्वयिना नञात्र बोध्यते इति सत्त्वनिरूपकाकाङ्क्षायाः  सत्त्वाद्धटनिरूपितसत्त्वविरोधीत्यर्थः~। इदानीं नञः स्वभावप्रदर्शनस्य फलं  दर्शयति {\br तदिति~।} तन्नञित्यन्वयः~। तत्प्रदर्शितस्वभावं नञित्यर्थः~।
\newpage
%%%%%%%%%%%%%%%%%%%%%%%%%%%%%%%%%%%
\fancyhead[LO]{[ बाधकं द्विविधम् ]}
{\bl\noindent
वाक्यश्रवणेऽयं मां प्रवर्तयतीति प्रतीतेः~। तस्मान्निषेधवाक्यस्थले निवर्तनैव वाक्यार्थः~। यदा तु प्रत्ययार्थस्य तत्रान्वये बाधकं  तदा धात्वर्थस्यैव तत्रान्वयः~।}
\begin{center}
 \textbf{बाधकं द्विविधम् }   
\end{center}
 
{\bl तच्च बाधकं द्विविधम् {\al तस्य व्रतमित्युपक्रमो विकल्पप्रसक्तिश्च~}। तत्राद्यं {\qtl नेक्षेतोद्यन्तमादित्यम्} इत्यादौ {\qtl तस्य व्रतम्} इत्यु-}\\
\hrule
\vspace{3mm}
\noindent
{\br इहेति~।} {\qt न कलञ्जं भक्षयेत्, ब्राह्मणो न हन्तव्यः} इत्यादौ वाक्य इत्यर्थः~। {\br ननु} लिङः प्रवर्तनाप्रतिपादकत्वे सिद्धे तत्संबधनञस्तदर्थप्रवर्तनाविरोधिनिवर्तनाप्रतिपादकत्वं सेत्स्यति तदेव कुत इत्याशङ्क्य तत्र हेतुमाह\textendash {\br विधीति~।  प्रतीतेरिति~।} प्रवर्तनाप्रतीतेरित्यर्थः~। यद्वा  {\qt यजेत स्वर्गकामः} इति विधिवाक्यश्रवणेऽयं मां प्रवर्तयतीति प्रवर्तनाप्रतीतिवत्, {\qt न कलञ्जं भक्षयेत्}  इत्यादिनिषेधवाक्यश्रवणेऽप्ययं मां निवर्तयतीति निवर्तनाप्रतीतेरित्यध्याहारेण  विधीत्यादिहेतुवाक्यं दृष्टान्तदार्ष्टान्तिकविधया योजनीयम्~। एवं सिद्धं निवर्तनारूपं निषेधवाक्यार्थमुपसंहरति {\br तस्मादिति~।}  {\qt न कलञ्जं भक्षयेत्} इत्यत्र  कलञ्जकर्मकभक्षणानुकूलपुरुषप्रवृत्तिजनकप्रवर्तनां प्रति लिर्ङ्थप्रवर्तनाविरोधिनिवर्तनैव वाक्यार्थः~। एवमन्यत्रापि निषेधवाक्येषु सर्वत्र निवर्तनाया एव 
वाक्यार्थत्वे विधिनिषेधयोर्भिन्नार्थत्वमप्युपपन्नं भवति, हननादिवर्जनकर्तव्यतावाक्यार्थत्वपक्षे तु कर्तव्यताया एवोभयत्र प्रतिपाद्यत्वात्तयोरेकार्थत्वं स्यात्तच्च न  युक्तम्~। यथाहुः {\qt अन्तरं यादृशं लोके ब्रह्महत्याश्वमेधयोः~। दृश्यते तादृगेवेह  विधानप्रतिषेधयोः}  इति~। तस्मान्निवर्तनैव प्रतिषेधेषु वाक्यार्थ इति सिद्धम्~। यदि तु प्रत्ययार्थस्य नञर्थेनान्वये किंचिद्बाधकं वर्तते तदा धात्वर्थस्यैव नञर्थेनान्वयो भवतीत्याह\textendash {\br यदा त्वित्यादिना~।} \\

 तत्र बाधकं विभजते {\br तच्चेति~।}  प्रत्ययार्थस्य नञर्थेनान्वये वर्तमानं चेत्यर्थः~। {\br तत्राद्यमिति~।} तयोरुक्तयोर्द्वयोर्मध्य आद्यबाधकमित्यर्थः~। {\br तस्य व्रतमितीति~।} तस्य स्नातकस्य ब्रह्मचारिविशेषस्य व्रतं प्रजापतिदेवताक-
\newpage
%%%%%%%%%%%%%%%%%%%%%%%%%%%%
\fancyhead[RE]{[ बाधकं\textemdash\ }
{\bl\noindent
पक्रम्यैतद्वाक्यपाठात्~। तथा चात्र पर्युदासाश्रयणम्~। तथा हि\textendash व्रतशब्दस्य कर्तव्यार्थे रूढत्वात्तस्य व्रतमित्यत्र स्नातकस्य व्रतानां कर्तव्यत्वेनोपक्रमात्~। किं तत्कर्तव्यमित्याकाङ्क्षायां {\qtl नेक्षेतोद्यन्तम्}  इत्यादिना कर्तव्यार्थ एव प्रतिपादनीयः~। अन्यथा पूर्वोत्तरवाक्ययोरेकवाक्यत्वं न स्यात्~। तथा च नञर्थेन न प्रत्यया-\\ }
\hrule
\vspace{3mm}
\noindent
मादित्यानीक्षणसंकल्पादिकं किंचिदनुष्ठेयमित्यर्थः~। {\br एतद्वाक्यपाठादिति~।} {\qt नेक्षेतोद्यन्तमादित्यं नास्तं यन्तं कदाचन} इत्येतद्वाक्यपाठादित्यर्थः~। {\br तथा चेति~।} 
तस्य व्रतमिति स्नातकस्यानुष्ठेयमुपक्रम्य {\qt नेक्षेतोद्यन्तम्} इत्यादिवाक्यपाठे चास्मिन्वाक्ये पर्युदास एव समाश्रीयत इत्यर्थः~। किंच {\qt नेक्षेतोद्यन्तमादित्यम्} इत्यत्र 
नञ्पदमभिधावृत्त्या प्रतिषेधं ब्रूते नतु पर्युदासं, लक्षणापत्तेः प्रतिषेधस्य च  प्राप्तिपूर्वकत्वाद्वैदिकस्य प्रतिषेधस्य वैदिक्येव प्राप्तिस्तु प्रत्यासन्ना भवेत्~। तथा च सति यत्र क्रतावादित्येक्षणं विहितं तत्रायं प्रतिषेध उदयास्तमयोद्देशेन प्रवर्तते~। एवं च सति नात्र फलं कल्पनीयं स्यात्~। पर्युदासमाश्रित्य पुरुषार्थत्वाङ्गीकारे त्वधिकारसिद्धये फलस्य कल्पनीयत्वमापद्येत , तस्मादत्र क्रत्वर्थः प्रतिषेध इत्याशङ्क्य, तस्य व्रतमित्युपक्रम्य {\qt नेक्षेतोद्यन्तम्} इत्याद्याम्नातत्वाद्व्रतशब्दस्य च कर्तव्यरूपार्थे रूढत्वादत्र किंचिदनुष्ठेयमेव प्रतिभाति~। तच्च पर्युदासाश्रयणे सत्येवोपपद्यते~। किंचोपक्रमवाक्ये प्रतिज्ञातस्यैवार्थस्यात्रापि वक्तव्यत्वात्, उपक्रमवाक्ये तु स्नातकानुष्ठेयव्रतानामेव प्रतिज्ञातत्वात् कानि तानि व्रतानीत्यपेक्षायां स्नातकव्रतप्रदर्शनायास्य वाक्यस्यावतारादत्र कर्तव्य एव  कश्चिदर्थो वक्तव्यः, स च पर्युदासपक्षे लभ्यते, निषेधपक्षे तु दुर्लभ
एवेत्यभिप्रायेण पर्युदासपक्षमुपपादयति {\br तथा हीत्यादिना~।} {\qt नेक्षेतोद्यन्तम्} इत्यादौ कर्तव्यरूपस्यार्थस्याप्रतिपादनीयत्वे बाधकमाह\textendash {\br अन्यथेति~। पूर्वोत्तरेति~।} तस्य व्रतमिति पूर्ववाक्यं {\qt नेक्षेतोद्यन्तम्} इत्याद्युत्तरवाक्यं च तयोरेकवाक्यत्वं बाध्येतेत्यर्थः~। {\br ननु} {\qt नेक्षेतोद्यन्तम्}  इत्यादौ भवतु कर्तव्यरूपार्थस्यैव प्रतिपादनं, ततोऽपि किं स्यादित्यत आह\textendash {\br तथा चेति~।} तथा च {\qt नेक्षेतोद्यन्तम्} इत्यादौ कर्तव्यरूपार्थस्य प्रतिपादनीयत्वे च नञर्थेन प्रत्ययार्थान्वयावकाशो न भवतीत्यर्थः~। {\br ननु} {\qt नेक्षेतोद्यन्तमादित्यम्}  इत्यत्र नञर्थेन प्रत्ययार्थस्यान्वये को दोषः 
\newpage
%%%%%%%%%%%%%%%%%%%%%%%%%%
\fancyhead[LO]{द्विविधम् ]}
{\bl\noindent र्थान्वयः कर्तव्यार्थानवबोधात्~। विध्यर्थप्रवर्तनाविरोधिनिवर्तनाया एव तादृशनञा बोधनात्, तस्याश्च कर्तव्यार्थत्वाभावात्~। तस्मात् {\qt नेक्षेत} इत्यत्र नञा धात्वर्थविरोध्यनीक्षणसंकल्प एव लक्षणया प्रतिपाद्यते ; तस्य कर्तव्यत्वसंभवात्~।}\\
\hrule
\vspace{3mm}
\noindent
इत्याशङ्क्य तत्र हेतुमाह\textendash {\br कर्तव्येति~।} नञर्थेन प्रत्ययार्थस्य \alfootnote{टिप्प०\textemdash\ $^{1}$प्रकृतवाक्य इत्यर्थः~।}\footnotemarkA[1]तत्रान्वये ततः कर्तव्यार्थानवबोधापत्तेरित्यर्थः~। {\br ननु} प्रत्ययान्वितस्य नञः कर्तव्यार्थावबोधकत्वानङ्गीकारे को वार्थस्तेनावबोध्यत इत्यत आह\textendash {\br विध्यर्थेति तादृशनञेति~।} प्रत्ययान्वितेन नञेत्यर्थः~। तादृशनिवर्तनैव कर्तव्यरूपार्थो भवतु को दोष इति मन्दाभिप्रायमाशङ्क्याह\textendash {\br तस्याश्चेति~।} निरुक्तनिवर्तनायाश्चेत्यर्थः~। किंचैवं नञः प्रत्ययेनान्वयासंभवे प्रत्ययादवतारितो नञ् धातुना सह संबध्यते,  धातुना नञः संबन्धे च न तस्य निषेधकत्वं संभवति; विधायकसंबद्धस्यैव नञो निषेधकत्वात्, निषेधकत्वस्य विधायकत्वप्रतिपक्षत्वात्, नामधातुभ्यां योगे तु नञो न निषेधकत्वं युक्तं; तयोरविधायकत्वात्~। यथाहुः\textendash {\qt नामधात्वर्थयोगे तु नैव नञ् प्रतिषेधकः~।}~\footnoteA[2]{अत्राब्राह्मणाधर्माविति कर्तृपदम् , शिष्टं कर्मपदम्~। वदतोऽब्राह्मणाधर्मावन्यमात्रविरोधिनौ}  इति~। तस्मान्नेक्षेतेत्यत्र नञो धातुना योगान्नञा  धात्वर्थक्षणविरोधी कश्चनार्थः प्रतिपाद्यते~। यद्यपि नञोऽभाव एव शक्तिः~। तथा चेक्षणस्याभाव एव नञः शक्यार्थो लाघवात् नतु तद्विरोधी तस्यां भावघटितत्वेन गौरवापत्तेः~। तदन्यतद्विरुद्धतदभावेषु नञिति स्मरणं तु प्रतीत्यभिप्रायं न शक्त्यभिप्रायं, तथापि
नेक्षेतेत्यत्र प्रत्ययस्य नञा संबन्धात् नञ्संबन्धशून्येन च तेन तावत्कश्चिदर्थो विधेयः स्वीकर्तव्यः~। तत्र न तावद्धात्वर्थो विधातुं शक्यते, नञा तदभावबोधनात्~। नापि तदभावो विधातुं शक्यते, अभावस्याविधेयत्वात्, तस्मात्पर्युदासाश्रयणेन  धात्वर्थेक्षणविरोधी कश्चनात्र विधानयोग्योऽर्थो नञा लक्षणया प्रतिपाद्यते, स च  विधानयोग्यः पदार्थोऽनीक्षणसंकल्प एव तस्येक्षणविरोधित्वात्कर्तव्यत्वसंभवाच्च,  ततश्च स एव संकल्पोऽत्रानुष्ठेयत्वेन विधीयत इत्यभिप्रायेण पर्युदासस्यावश्यकत्वात्प्रत्ययादवतारितस्य नञो धातुसंबन्धेन
तदर्थविरोध्यनीक्षणसंकल्पप्रतिपादकत्वमुपसंहरति {\br तस्मादिति~। लक्षणयेति~।} स्वसमभिव्याहृत-
\newpage
%%%%%%%%%%%%%%%%%%%%%%%%%%%%%%
\fancyhead[RE]{[ पर्यु०नेक्षे०वाक्यार्थः ]}
\begin{center}
 \textbf{पर्युदासपक्षे नेक्षेतेत्यस्य वाक्यार्थः }   
\end{center}
 
{\bl आदित्यविषयकानीक्षणसंकल्पेन भावयेदिति वाक्यार्थः~। तत्र भाव्याकाङ्क्षायाम् {\qtl एतवता हैनसा वियुक्तो भवति}  इति वाक्यशेषावगतः पापक्षयो भाव्यतयान्वेति~। एवं च पूर्वोत्तरयोरेकवाक्यत्वं निर्वहत्येव~। न चात्र धात्वर्थविरोधिनः पदार्थान्तरस्यापि संभवात्कथमनीक्षणसंकल्पस्यैव भावनान्वय इति वाच्यम् ; }\\
\hrule
\vspace{3mm}
\noindent
पदार्थाभाव एव नञः स्वशक्यार्थो भवति~। तथा च स्वसंबध्यमानधात्वर्थाभावं शक्त्या प्रतिपादयन् नञ् तदभावसंबन्धिनं तदर्थविरोधनं लक्षणया प्रतिपादयत्येव; तदभावतद्विरोधिनोः संबन्धस्य संभवात्~। दृश्यते हि तेजोऽभावतमसोः संबन्धः~। तथा च {\qt उद्यन्तमस्तं यन्तं चादित्यं नेक्षिष्ये}  इत्येवंरूपः संकल्पोऽत्रानुष्ठेयत्वेन विधीयमानो नञा लक्षणया प्रतिपाद्यत इति भावः~। {\br तस्येति~।} अनीक्षणसंकल्पस्येत्यर्थः~।~\\

 पर्युदासपक्षे {\qt नेक्षेतोद्यन्तमादित्यम्} इत्यस्य वाक्यस्यार्थं प्रदशयति {\br आदित्येति~।} यत्तु पर्युदासपक्षे फलस्यात्र कल्पनीयत्वमापादितं तदपि न, वाक्यशेषावगतस्य पापक्षयस्यैवात्र फलत्वसंभवादित्याशयेनात्र भाव्यान्वयं प्रदर्शयति {\br तत्रेति~।} अनीक्षणसंकल्पभावनायामित्यर्थः~। {\br एनसा वियुक्तो भवतीति~।} पापेन विरहितो भवतीत्यर्थः~। किंच {\qt नेक्षेतोद्यन्तमादित्यम्} इत्यत्रानीक्षणसंकल्परूपस्यानुष्ठेयस्य प्रतिपादने तस्य व्रतमिति स्नातकव्रतोपक्रमवाक्यस्य नेक्षेतेत्याद्युत्तरवाक्येनैकवाक्यत्वमुपपन्नं भवति, पूर्वप्रतिज्ञातस्यैवोत्तरप्रतिपादनादित्याह\textendash {\br एवमिति~।} {\br ननु} धात्वर्थेक्षणविरोधिनो बहवः पदार्था
अनुष्ठेयाः सन्ति, ततश्च कथमत्र वाक्येऽनीक्षणसंकल्पस्यैव भावनायां करणत्वेनान्वयः  स्वीक्रियत इत्याशङ्क्य परिहरति {\br न चेत्यादिना~।} यद्यपि पदार्थान्तराणां 
पटेन चक्षुषोः पिधानादिरूपाणां धात्वर्थेक्षणविरोधित्वं संभवति , तथापि  कायिकवाचिकव्यापारविशेषाणामत्राप्रतीयमानत्वान्मानसव्यापारस्य चाप्रतिषेधात्संकल्प एव मानसव्यापारविशेषोऽत्र परिशिष्यत इत्यभिप्रेत्य तत्र हेतुमाह\textendash
\newpage
%%%%%%%%%%%%%%%%%%%%%%%%%%%%%%%%%%%%
\fancyhead[LO]{[ वि० पर्युदासाश्रयणम् ]}
{\bl\noindent तस्य कर्तव्यताभावेन प्रकृते भावनान्वयायोग्यत्वात्~। द्वितीयं {\qtl यजतिषु येयजामहं करोति नानुयाजेषु}  इत्यादावत्र विकल्प प्रसक्तौ च पर्युदासाश्रयणात्~।}
\begin{center}
 \textbf{विकल्पप्रसक्तौ पर्युदासाश्रयणम्} 
\end{center}
 
{\bl तथा हि\textendash यद्यत्र वाक्ये नञर्थे प्रत्ययार्थान्वयः स्यात्तदा अनुयाजेषु {\qtl येयजामहे} इति मन्त्रस्य प्रतिषेधः स्यात्, अनुयाजेषु येयजामहं न कुर्यादिति~। स च प्राप्तिपूर्वक एव, प्राप्तस्यैव प्रतिषेधात्~। प्राप्तिश्च {\qtl यजतिषु येयजामहं करोति}  इति शास्त्रादेव  वाच्या~। शास्त्रप्राप्तस्य च प्रतिषेधे विकल्प एव, न तु बाधः~।}\\
\hrule
\vspace{3mm}
\noindent
{\br तस्येति~।} पटादिना चक्षुषोः पिधानादिरूपस्य पदार्थान्तरस्य कर्तव्यताभावेन  कर्तव्यत्वेन विवक्षाऽसंभवेन सर्वक्रियाऽविनाभूतस्यैव धात्वर्थेक्षणविरोधिपदार्थान्तरस्य 
नेक्षेतेत्यादौ प्रकृते भावनान्वययोग्यत्वमुपपद्यते~। तथा च धात्वर्थेक्षणविरोध्यनीक्षणसंकल्पस्यैवात्र भावनान्वययोग्यता संभवति, तस्य सर्वक्रियाऽविनाभूतत्वान्न तु 
तादृशपदार्थान्तरस्य तस्य सर्वक्रियाऽविनाभूतत्वासंभवादिति भावः~। प्रत्ययार्थस्य नञर्थेनान्वये विकल्पप्रसक्तिरूपस्य बाधकस्य प्रतिषेधविघटनेन पर्युदासगमकत्वप्रदर्शनाय तद्विषयमुदाहरति {\br द्वितीयमिति~।} \\

 अत्र पर्युदासाश्रयणाय विकल्पप्रसक्तिमेवोपपादयति {\br तथा हीत्यादिना~। स चेति~}। निषेधश्चेत्यर्थः~। तत्र हेतुमाह\textendash {\br प्राप्तस्यैवेति~।शास्त्रादेवेति~।} 
ब्राह्मणहननादेरिव रागतः प्राप्त्यसंभवात्तस्य शास्त्रादेव प्राप्तिर्वाच्येत्यर्थः~। उपपादितां विकल्पप्रसक्तिमुपसंहरति {\br शास्त्रप्राप्तस्येति~।ननु} यथा रागतो हननादौ प्रवृत्तं पुरुषं हननादिप्राप्तिमूलभूतरागस्य बाधेन ततो न हन्तव्य इत्यादिशास्त्रं निवर्तयति, तथा {\qt यजतिषु येयजामहं करोति}  इति शास्त्रादनुयाजेष्वप्यनुष्ठानकाले यजतित्वाविशेषादेव {\qt येयजामहे} इति मन्त्रस्य समुच्चारणे प्रवृत्तं पुरुषं तत्प्राप्तिमूलभूतस्य प्रदर्शितशास्त्रस्य यजतिसामान्ये {\qt येयजामहे} इति मन्त्रप्रापकस्यानुयाजेषु बाधेन ततो {\qt नानुयाजेषु येयजामहं करोति} इति 
\lfoot{८ अ०}
\newpage
%%%%%%%%%%%%%%%%%%%%%%%%%%%%%%%%%%%%%%%%%5
\lfoot{}
\fancyhead[RE]{[ वि० पर्युदासाश्रयणम् ]}
{\bl\noindent
प्राप्तिमूलरागस्येव तन्मूलशास्त्रस्य शास्त्रान्तरेण बाधायोगात्~। न  च {\qtl पदे जुहोति} इति विशेषशास्त्रेण {\qtl आहवनीये जुहोति} इति शास्त्रस्येव {\qtl नानुयाजेषु}  इत्यनेन {\qtl यजतिषु येयजामहं करोति} इत्यस्य  बाधः स्यादिति वाच्यम् ; परस्परनिरपेक्षयोरेव शास्त्रयोर्बाध्यबाधकभावात्~। पदशास्त्रस्य हि स्वार्थविधानार्थमाहवनीयशास्त्रान-\\}
\hrule
\vspace{3mm}
\noindent
शास्त्रं निवर्तयत्येवेति कथं न शास्त्रप्राप्त्य बाधः स्यादित्यत आह\textendash  {\br प्राप्तीति~।}  शास्त्रेण बाध इति \footnotemark शेषः ~। {\br तन्मूलशास्त्रस्येति~।} {\qt येयजामहे} इति मन्त्रप्राप्तिमूलशास्त्रात् {\qt यजतिषु येयजामहं करोति}  इत्येवंरूपाच्छास्त्रान्तरेण  {\qt नानुयाजेषु येयजामहं करोति}  इत्येवंरूपेण
प्रदर्शितमन्त्रप्राप्तिमूलभूतस्य प्रदर्शितशास्त्रस्य बाधायोगादित्यर्थः~। दृष्टान्तस्त्वत्र व्यतिरेकी बोध्यः~। तथा च यथा हननादिप्राप्तिमूलरागस्य भ्रान्तिनिमित्तकस्य शास्त्रेण बाधो भवति तथा मन्त्रप्राप्तिमूलशास्त्राच्छास्त्रान्तरेण मन्त्रप्राप्तिमूलशास्त्रस्य बाधो न युक्तो निषेधशास्त्रस्य निषेध्यप्राप्तिसापेक्षत्वेन तत्प्रापकस्याबाधकत्वादित्यनुपदं स्पष्टीभविष्यति मूल इति भावः~। यद्वा, {\qt तन्मूलशास्त्रादिति पश्चमी षष्ट्यर्थं बोध्येति, तेनानुषज्ञविभक्तिविपरिणामयोर्न श्रमः कर्तव्यः स्यात्~}।~केचित्तु {\qt तन्मूलशास्त्रादिति बाधमूलशास्त्राद्वाधकरूपात्प्राप्तिमूलरागस्य यथा बाधस्तथा शास्त्रान्तरेण प्राप्तिमूलशास्त्रस्य बाधायोगादिति व्याचक्षते~}। {\br ननु} यथा  पदाधिकरणकहोमविधायकविशेषशास्त्रेणाहवनीयाधिकरणकहोमविधायकस्य सामान्यशास्त्रस्य बाधः क्रियते, तथाऽनुयाजेषु येयजामहमन्त्रप्रतिषेधकरूपविशेषशास्त्रेण
यागसामान्ये तन्मन्त्रविधायकस्य सामान्यशास्त्रस्य बाधः कथं न क्रियत इत्याशङ्क्य परिहरति {\br न चेत्यादिना~।} न च वाच्यमित्यत्र हेतुमाह\textendash {\br परस्परेति~।} शास्त्रयोरेकविषये बाध्यबाधकभावे परस्परनिरपेक्षत्वं हेतुः पदशास्त्रस्य पदाधिकरणकहोमरूपस्वार्थविधानार्थमाहवनीयशास्त्रनिरपेक्षत्वादनुयाजेषु
येयजामहमन्त्रप्रतिषेधकशास्त्रस्य तत्र प्रतिषेध्यमन्त्रप्रसक्त्यर्थं  यजतिसामान्ये तादृशमन्त्रविधायकसामान्यशास्त्रसापेक्षत्वाच्चेति दृष्टान्तदार्ष्टान्तिक्योर्वैषम्यं
प्रदर्शयति {\br पदशास्त्रस्येत्यादिना~।} तथा च प्रतिषेधशास्त्रस्य विशेषविषयत्वेन प्रबलत्ववद्विधिशास्त्र-
\blfootnote{पाठा०\textemdash\ $^{१}$स चेवपदात्पूर्व द्रष्टव्यः~।} 
\newpage
%%%%%%%%%%%%%%%%%%%%%%%%%%%%%%%%%%%%%%%%%
\fancyhead[LO]{[ बाधायोगोपसंहारः ]}
{\bl\noindent पेक्षणान्निरपेक्षत्वम्~। प्रकृते तु निषेधशास्त्रस्य निषेध्यप्रसक्त्यर्थं  {\qtl यजतिषु येयजामह}  मित्यस्यापेक्षणान्न निरपेक्षत्वम्~।}
\begin{center}
 \textbf{बाधायोगोपसंहारः }   
\end{center}
 
{\bl तस्माच्छास्त्रविहितस्य शास्त्रान्तरेण प्रतिषेधे विकल्प एव ;  स च न युक्तः~। विकल्पे शास्त्रस्य पाक्षिकाप्रामाण्यापातात्~। न ह्यनुयाजेषु येयजामहमित्यस्यानुष्ठाने नानुयाजेष्वित्यस्य प्रामाण्यं संभवति, व्रीहियागानुष्ठाने यवशास्त्रस्येव~। द्विरदृष्टकल्पना च स्यात्, विधिप्रतिषेधयोरपि पुरुषार्थत्वात् , अतो }\\
\hrule
\vspace{3mm}
\noindent
स्याप्युपजीव्यत्वेन प्रबलत्वमस्तीति न प्रतिषेधशास्त्रेण विधिशास्त्रस्यात्यन्तबाधो युक्त इति विहितप्रतिषिद्धत्वाद्विकल्पः स्यात्, स च न युक्त इत्युपरिष्टात्स्पष्टीभविष्यतीति भावः~।\\

 बाधायोगमुपसंहरति {\br तस्मादिति~।} मन्त्रविधायकशास्त्रस्य तत्प्रतिषेधकशास्त्रेण बाधायोगाद्विहितस्यापि तेन प्रतिषेधे विकल्प एव स्यान्न बाध  इत्यर्थः~। {\br ननु} भवतु विकल्प एव, तेन किं हीयत इत्यत आह\textendash {\br स चेति~।} विकल्पस्यायुक्तत्वे हेतुमाह\textendash {\br विकल्प इति~।} {\br ननु} विकल्पस्वीकारे कथं
पक्षे  शास्त्रस्याप्रामाण्यापात इत्यत आह\textendash {\br न हीति~।} यथाव्रीहिभिर्यागानुष्ठानसमये  यवशास्त्रस्य न प्रामाण्यं भवति, तथा विधायकशास्त्रानुसारेण {\qt येयजामहे} इति मन्त्रस्यानुयाजेषूच्चारणानुष्ठानसमये न तत्प्रतिषेधकशास्त्रस्य प्रामाण्यं  संभवतीत्यर्थः~। किंच विकल्पपक्षे हि द्विरदृष्टकल्पनाप्रसङ्गोऽपि स्यात्,
विधिशास्त्राद्द्येवमवगम्यते यदनुयाजेषु {\qt येयजामह} इति मन्त्रस्य करणे कश्चनोपकारो भवतीति प्रतिषेधशास्त्रादपि तत्र तदकरणात्कश्चनोपकारो भवति~।दर्शपूर्णमासयोरनृतवदनाकरणादिजन्योपकारवदित्यवगम्यते, तच्चोपकारद्वयमदृष्टरूपमिति द्विरदृष्टकल्पनाप्रसङ्ग इत्याह\textendash {\br द्विरदृष्टकल्पना चेति~।} तत्र
हेतुमाह़\textendash  {\br विधीति~। पुरुषार्थत्वादिति~।} विधानवत्प्रतिषेधस्याप्यदृष्टपुरुषार्थसंपादकत्वादित्यर्थः~। तस्मान्नानुयाजेष्वित्यादौ प्रतिषेधस्वीकारे विकल्पादिदोषापत्तेर्नात्र  प्रतिषेधः समाश्रीयते किंतु नञो नामसंबन्धमाश्रित्य पर्युदासस्यैवाश्रयणं
\newpage
%%%%%%%%%%%%%%%%%%%%%%%%%%%
\fancyhead[RE]{[ बाधायोगोपसंहारः ]}
{\bl\noindent
नात्र प्रतिषेधस्याश्रयणम्, किंतु नञोऽनुयाजसंबन्धमाश्रित्य पर्युदासस्यैव । इत्थं चानुयाजव्यतिरिक्तेषु यजतिषु येयजामह  इति मन्त्रं कुर्यादिति वाक्यार्थबोधः, नञोऽनुयाजव्यतिरिक्ते लाक्षणिकत्वात्~। एवं च न विकल्पः~। अत्र च वाक्ये येयजामह इति न विधीयते, यजतिषु येयजामहमित्यनेनैव प्राप्तत्वात्~। किंतु सामान्यशास्त्रप्राप्त-येयजामह इत्यनुवादेन तस्यानुयाजव्यतिरिक्तविषयकत्वं विधीयते~। यद्यजतिषु येयजामहं करोति  तदनुयाजव्यतिरिक्तेष्विति~।~}\\
\hrule
\vspace{3mm}
\noindent
क्रियत इत्युपसंहरति {\br अत इति~।} एवं च {\qt नानुयाजेषु येयजामहं करोति} इत्यस्य वाक्यस्यार्थमाह\textendash {\br इत्थं चेति~।} अनेन पूर्वोक्तप्रकारेण
चेत्यर्थः~। अत्र पर्युदासस्य तात्पर्यविषयत्वेन करोतेरवतारितस्य नञोऽनुयाजशब्देन संबन्धसिद्धौ चेति यावत्~। तत्र हेतुमाह\textendash {\br नञ इति~।}
अनुयाजव्यतिरिक्तेऽनुयाजत्वाभावस्य सत्त्वाच्छक्यसंबन्धसंभवेन लक्षणोपपत्तिरिति भावः~। अत्र प्रतिषेधपक्षं परित्यज्य पर्युदासपक्षस्वीकारस्य फलमाह\textendash {\br एवं चेति~। ननु} {\qt नानुयाजेषु येयजामहं करोति} इत्यत्र {\qt येयजामहे}   इति मन्त्रविधौ तस्यानुयाजव्यतिरिक्तविषयत्वविधौ च वाक्यभेदः स्यादित्यत आह\textendash {\br अत्र चेति~।} {\br सामान्यशास्त्रेति~।} {\qt यजतिषु येयजामहं करोति}  इति यजिसामान्ये तद्विधायकशास्त्रप्राप्तमेव {\qt येयजामहे}  इति मन्त्रमनूद्य {\qt नानुयाजेषु येयजामहं करोति} इत्यत्र  तस्यानुयाजव्यतिरिक्तविषयता विधीयत इत्यर्थः~। तथा च यजतिष्वित्यादिसामान्यशास्त्रस्य विशेषापेक्षिणो
नानुयाजेष्वित्यादिविशेषशास्त्रेणानुयाजव्यतिरिक्तविषयसमर्पणादनुयाजव्यतिरिक्तयागेषु {\qt येयजामहे}  इति मन्त्रः कर्तव्यतया प्राप्तः~। अनुयाजेषु तु तस्य कर्तव्यत्वेनाप्राप्तत्वादप्रतिषिद्धत्वाच्च नात्र विकल्पप्रसङ्गोऽपि संभवति~। लक्षणयाऽनुयाजव्यतिरिक्तविषयसमर्पणाच्च न नानुयाजेष्वित्यादिवाक्यस्याप्रामाण्यमपि 
भवति ; तस्मात्पर्युदासाश्रयणे बाधकाभावात् स एव स्वीकर्तव्य इति भावः~।
\newpage
%%%%%%%%%%%%%%%%%%%%%%%%%%%%%%%%%
\fancyhead[LO]{पर्यु० भेदवर्णनम् ]}
\begin{center}
 \textbf{पर्युदासोपसंहारयोर्भेदवर्णनम् }   
\end{center}
 
{\bl नन्वेवं सामान्यशास्त्रप्राप्तस्य विशेषे संकोचनरूपादुपसंहारात्पर्युदासस्य भेदो न स्यादिति त,{\qtl  न}; उपसंहारो हि तन्मात्रसंकोचार्थः~। यथा पुरोडाशं चतुर्धा करोतीति सामान्यप्राप्तचतुर्धाकरणम् {\qtl आग्नेयं चतुर्धा करोति} इति विशेषादाग्नेयपुरोडाशमात्रे संकोच्यते~। पर्युदासस्तु तदन्यमात्रसंकोचार्थ इति ततो  भेदात्~। कुत्रचिद्विकल्पप्रसक्तावप्यनन्यगत्या प्रतिषेधाश्रयणम्~।}\\
\hrule
\vspace{3mm}

 {\br ननु} पर्युदासाश्रयणपक्षे {\qt यजतिषु येयजामहं करोति} इति सामान्यशास्त्रेण यागमात्रे प्राप्तस्य {\qt येयजामहे} इति मन्त्रस्य {\qt नानुयाजेषु येयजामहं करोति}  इति विशेषशास्त्रेणानुयाजव्यतिरिक्तविषये यागविशेषे संकोचनात्पर्युदासस्योपसंहारादभेदापत्तिः स्यात् ; सामान्ये प्राप्तस्य विशेषे संकोचस्यैवोपसंहारपदार्थत्वात्~। यथा {\qt पुरोडाशं चतुर्धा करोति} इति पुरोडाशसामान्ये प्राप्तस्य चतुर्धाकरणस्य आग्नेयं चतुर्धा करोतिइत्याग्नेये विशेषे संकोच इत्याशङ्कते {\br नन्वेवमिति~।} 
{\qt पुरोडाशं चतुर्धा करोति}  इत्यनेन पुरोडाशसामान्ये प्राप्तस्य चतुर्धाकरणस्य  {\qt आग्नेयं चतुर्धा करोति}  इत्यनेन च पुरोडाशविशेषमात्रै संकोचः क्रियते~। प्रकृते तु यजतिष्वित्यादिना यजतिसामान्ये प्राप्तस्य {\qt येयजामहे} इति मन्त्रस्य  नानुयाजेष्वित्यादिना यजतिविशेषादनुयाजरूपादन्यमात्रे यजतिविशेषे संकोच  इत्युपसंहारात्पर्युदासस्य भेदः स्पष्ट एवेति परिहरति {\br नेत्यादिना~। तन्मात्रसंकोचार्थ इति~।} सामान्यप्राप्तस्य विशेषमात्रसंकोचार्थ इत्यर्थः~। तत्रोदाहरणमाह\textendash {\br यथेति~। तदन्यमात्रसंकोचार्थ इति~।} सामान्यप्राप्तस्य विशेषादन्यमात्रसंकोचार्थ इत्यर्थः~।{\br इति ततो भेदादिति~।} एवमुक्तप्रकारेणोपसंहारात्पर्युदासस्य भेदान् तयोरभेदापत्तिरित्यर्थः~। अपरे तु, उपसंहारो  नाम सामान्यतः प्राप्तस्य विशेषे संकोचनरूपो विधेर्व्यापारविशेषः, पर्युदासस्तु {\qt पर्युदासः स विज्ञेयो यत्रोत्तरपदेन नञ्} इत्यमियुक्तोक्त्या प्रत्ययातिरिक्तेन नाम्ना धातुना च नञः संबन्धरूपः~। तस्मादनयोस्तावत्स्वरूपतः स्पष्ट एव भेद इति न तयोरभेदापत्तिरित्याहुः~। किंच यत्र तु पर्युदास आश्रयितुं न शक्यते तत्र विकल्पप्रसक्तावपि निषेध एवाश्रीयत इत्याह\textendash {\br कुत्रचिदिति~।अनन्यगत्येति~।} 
\newpage
%%%%%%%%%%%%%%%%%%%%%%%%%%%%%%%%%%%%%%%%
\fancyhead[RE]{[ पर्यु० भेदवर्णनम् ]}
{\bl\noindent
यथा {\qtl नातिरात्रे षोडशिनं गृह्णाति} इत्यादौ~। अत्र हि {\qtl अतिरात्रे  षोडशिनं गृह्णाति} इति शास्त्रप्राप्तषोडशिग्रहणस्य निषेधाद्विकल्पप्रसक्तावपि न पर्युदासाश्रयणम् , असंभवात्~। तथा हि\textendash यद्यत्र  षोडशिपदार्थेन नञर्थान्वयस्तदातिरात्रे षोडशिव्यतिरिक्तं गृहणातीति वाक्यार्थबोधः स्यात्, स च न संभवति, {\qtl अतिरात्रे षोडशिनं गृह्णाति}  इति प्रत्यक्षविधिविरोधात्~। यदि चातिरात्रेण  पदार्थेनान्वयस्तदातिरात्रव्यतिरिक्ते षोडशिनं गृह्णातीति वाक्यार्थबोधः स्यात् , सोऽपि न संभवति; तद्विधिविरोधात्~। अतोऽत्रानन्यगत्या शास्त्रप्राप्तषोडशिग्रहणस्यैव निषेधः~। न च विकल्पप्रसक्तिस्तस्याप्यपेक्षणीयत्वात्~।}\\
\hrule
\vspace{3mm}
\noindent
प्रतिषेधादन्यगतिविशेषस्य पर्युदासस्यासंभवेनेत्यर्थः~। तत्रोदाहरणमाह\textendash {\br यथेति~।} नातिरात्र इत्यादिवाक्ये प्रतिषेध एवाश्रीयते, नतु पर्युदासः~। अत्र
चातिरात्र  इत्यादिशास्त्रेण प्राप्तस्य षोडशिनामकपात्रविशेषग्रहणस्य प्रतिषेधाद्विकल्पप्रसङ्गेऽपि  पर्युदासाश्रयणासंभवादित्याह \textendash {\br अत्र हीत्यादिना~।} अत्र
पर्युदासाश्रयणासंभवमुपपादयति {\br तथा हीत्यादिना~। अत्रेति~।} नातिरात्रे षोडशिनं  गृह्णातीत्यत्रेत्यर्थः~। {\br षोडशिपदार्थेन नञर्थान्वय इति~।} अतिरात्रे न
षोडशिनं गृह्णातीत्यन्वय इत्यर्थः~। एतदन्वयपक्षे वाक्यार्थमाह\textendash {\br तदेति~।}  एतादृशवाक्यार्थासंभवे हेतुमाह\textendash {\br अतिरात्रे षोडशिनमिति~।} अत्र पर्युदासोपपत्तयेऽन्वयान्तरमाह\textendash {\br यदीति~। अन्वय इति~।} नञर्थस्येति शेषः~। निरुक्तान्वयानुरोधेन वाक्यार्थबोधासंभवेऽपि हेतुमाह\textendash {\br तद्विधिविरोधादिति~।} अतिरात्रे षोडशिनं गृह्णातीत्यतिरात्रयागे षोडशिग्रहणविधायकप्रत्यक्षविधिविरोधादित्यर्थः~।उपपादितमुपसंहरति {\br अत इति~।} उक्तदोषप्रसङ्गादस्मिन्वाक्ये प्रतिषेधादन्यासंभवेन प्रतिषेध एव स्वीक्रियत इत्यर्थः~। अत्र प्रतिषेधे सिद्धान्तिते विकल्पप्रसङ्गमाशङ्क्येष्टापत्त्या परिहरति {\br न चेत्यादिना~।} अत्रेदं बोध्यम्\textendash यत्र तस्य व्रतमित्याद्युपक्रमो नास्ति,
\footnotemarkA[1]विकल्पाप्रसक्तिश्च तत्र प्रतिषेधो 
\alfootnote{टिप्प०\textemdash\ $^{1}$अत्रास्तीति शेषः~।}
\newpage
%%%%%%%%%%%%%%%%%%%%%%%%%%%%%%%%%%%%%%%%%%%
\fancyhead[LO]{[ वि०प्र०हेतु०वर्णनम् ]}
 \begin{center}
 \textbf{विकल्पे प्रतिषिध्यमानस्यानर्थहेतुत्वाभाववर्णनम् }
 \end{center}
 
{\bl इयांस्तु विशेषः यद्विकल्पादेकप्रतिषेधेऽपि प्रतिषिध्यमानस्य नानर्थहेतुत्वम्, विधिनिषेधोभयस्यापि क्रत्वर्थत्वात्~। यत्र तु न विकल्पः, प्राप्तिश्च रागत एव, प्रतिषेधश्च पुरुषार्थः तत्र प्रतिषिध्यमानस्यानर्थहेतुत्वम् , यथा {\qtl न कलञ्जं भक्षयेत्}  इत्यादौ कलञ्जभक्षणादेः, तत्र भक्षणनिषेधस्यैव पुरुषार्थत्वात्~। न च {\qtl दीक्षितो न  ददाति न जुहोति} इत्यादौ शास्त्रप्राप्तदानहोमादीनां निषेधाद्विकल्पापत्तिरिति वाच्यम्~।स्वतःपुरुषार्थभूतदानहोमादीनां निषेधस्य }\\
\hrule
\vspace{3mm}
\noindent
भवति, यथा न कलञ्जं भक्षयेदिति~। यत्र तु विकल्पप्रसङ्गेऽपि पर्युदासो नाश्रयितुं  शक्यते तत्र प्रतिषेध एव स्वीक्रियते यथा {\qt नातिरात्रे षोडशिनं गृह्णाति} इति~।\\

 {\br ननु} {\qt नातिरात्रे षोडशिनं गृह्णाति} इत्यत्र विकल्पप्रसङ्गेऽपि यदि षोडशिग्रहणस्य प्रतिषेध एव स्वीक्रियते तदा प्रतिषिध्यमानस्यानर्थहेतुत्वस्वीकारात्षोडशिग्रहणस्यापि
कलञ्जभक्षणादिवत्प्रतिषिध्यमानत्वेनानर्थहेतुत्वादर्थत्वबाधापत्तिरित्याशङ्क्याह \textendash {\br इयांस्त्वित्यादिना ।} प्रतिषिध्यमानयोरनर्थहेतुत्वतदभावरूपो विशेष  इत्यर्थः~। {\br विकल्पादिति~।} यत्र प्रतिषेधस्य विकल्पापादकत्वं तत्रैकस्य प्रतिषेधेऽपि न प्रतिषिध्यमानस्यानर्थहेतुत्वमित्यर्थः~। तत्र हेतुमाह\textendash
{\br विधीति~।}  रागत एवेत्येवकारेण षोडशिग्रहणादिवच्छास्त्रेण प्राप्तिः प्रतिषिध्यते~। अत्र निषिध्यमानस्यानर्थहेतुत्वे हेतुमाह\textendash {\br तत्रेति~।}
षोडशिग्रहणविधितत्प्रतिषेधयोरुभयोरपि क्रत्वर्थत्वात्तत्र प्रतिषिध्यमानस्य नानर्थहेतुत्वम्, अत्र तु कलञ्जभक्षणादिप्रतिषेधस्यैव पुरुषार्थत्वं नतु तदनुष्ठानस्येति
प्रतिषिध्यमानस्य कलञ्जभक्षणब्राह्मणहननादेरनर्थहेतुत्वमेवेति भावः~। {\br ननु} यथा शास्त्रप्राप्तस्य षोडशिग्रहणस्य प्रतिषेधे विकल्पः स्वीकृतः, तथा शास्त्रप्राप्तदनादीनामपि प्रतिषेधे  विकल्पापत्तिः स्यादिति दृष्टान्ताभिप्रायेणातिप्रसङ्गमाशङ्क्य परिहरति {\br न चेत्यादिना~। स्वत इति~।} स्वतःपुरुषार्थभूतदानहोमादीनां पुरुषार्थत्वेनैव शास्त्रप्राप्तिः, क्रत्वर्थत्वेनैव च प्रतिषेध इति षोडशिग्रहणविधितत्प्रतिषेधयोरुभयोस्तुल्यार्थत्वाभावेन  न विकल्पापत्तिः~।किंच\footnotemarkA[1] रागतः प्राप्तस्य पुरुषार्थत्वेन 
\alfootnote{टिप्प०\textemdash\ $^{1}$ दानहोमादिनिषेधस्य क्रत्वर्थमात्रसाधनपरः
किंचेत्यादिग्रन्थः~।}
\newpage
%%%%%%%%%%%%%%%%%%%%%%%%%%%%%%%%%
\fancyhead[RE]{[ वि० प्र० हेतु० वर्णनम् ]}
{\bl\noindent
पुरुषार्थत्वाभावेऽपि निषिध्यमानस्यानर्थहेतुत्वात्, यथा क्रतौ स्वस्त्रीगमनादेः, तन्निषेधस्य क्रत्वर्थत्वेन तस्य क्रतुवैगुण्यसंपादकत्वात्~।}\\
\hrule
\vspace{3mm}
\noindent
प्रतिषेधे प्रतिषिध्यमानस्य कलञ्जभक्षणादिवदनर्थहेतुत्वस्वीकारान्न दानहोमादेः प्रतिषिध्यमानस्याप्यनर्थहेतुत्वं संभवति तस्य रागतः प्राप्तत्वाभावात्, कलञ्जभक्षणादिप्रतिषेधवत्तत्प्रतिषेधस्य पुरुषार्थत्वाभावाच्चेति भावः~॥ अत्रेदं बोध्यम्; ज्योतिष्टोमे श्रूयते {\qt दीक्षितो न ददाति न जुहोति न पचति}   इति~। तत्र
यद्दानादिकं पुरुषार्थ यच्च क्रत्वर्थं तत्राप्युपदिष्टमतिदिष्टं च तत्सर्वं प्रतिषिध्यते~। कुतः ? न  ददातीत्यादिवचनस्य सामान्यरूपत्वादित्याद्यः पक्षः~।
अग्निहोत्रदानादेरुपदिष्टस्यापि प्रतिषेधे सत्युपदेशो व्यर्थः स्यादिति ततोऽतिदिष्टं पुमर्थं चेत्येतदुभयमेव प्रतिषिध्यते क्रत्वर्थत्वेनोपदिष्टं दानादिकमनुष्ठेयमिति मध्यमः पक्षः~।
नित्याग्निहोत्रदानादिकं पुरुषार्थत्वेन यत्प्रत्यक्षश्रुतावुपदिष्टं ज्योतिष्टोमकालेऽपि प्राप्तं यच्चातिदिष्टं दानादिकं तयोरुभयोर्मध्ये प्रत्यक्षोपदिष्टनिषेधस्योपदिष्टमेव संनिहितमिति पुमर्थस्यैवात्र निषेधः~। न च नित्याग्निहोत्रदानादौ विधिनिषेधयोः प्रवृत्त्या  विकल्पः शङ्क्यः, भिन्नविषयत्वात्~। क्रत्वर्थो निषेधः क्रतुकाले तदनुष्ठानं वारयति~। पुरुषार्थस्तु विधिः कालान्तरे तदनुष्ठापयति तस्मात्पुमर्थस्य निषेध इति~। यदि प्रतिषेधपक्षे वाक्यभेदः शङ्क्येत, तर्हि  पुमर्थदानादिव्यतिरिक्तं क्रतुकालेऽनुष्ठेयमिति पर्युदासोऽस्तु~। {\br ननु} यत्र रागतः प्राप्तस्य प्रतिषेधः पुरुषार्थो भवति, तत्र प्रतिषिध्यमानस्यानर्थहेतुत्वं यथा कलञ्जभक्षणादेः, क्रतौ स्वस्त्रीगमनप्रतिषेधस्य तु पुरुषार्थत्वाभावात्तत्र निषिध्यमानस्य स्वस्त्रीगमनस्य कथमनर्थहेतुत्वमित्यत आह\textendash {\br निषेधस्येति~।} यद्वा, {\qt निषेधस्येत्यादिपूर्वमेवान्वेति तत्र प्रसिद्द्यर्थं तु यथा क्रतावित्यादिकमेव भवति}~। तथा च स्वतः पुरुषार्थेत्यादेरयमर्थः {\qt स्वत इति क्रत्वङ्गत्वोपाधिमन्तरेण केवलस्वरूपत एव पुमर्थभूतानां दानादीनां निषेधस्य पुरुषार्थत्वाभावेऽपि क्रत्वर्थत्वसंभवान्निषिध्यमानस्य दानादेः  क्रतुवैगुण्यसंपादकत्वेनानर्थहेतुत्वमिति योजना}~।~तथा च तेषामनुष्ठानस्य स्वतः  पुरुषार्थत्वेऽपि क्रत्वनुष्ठानकालेऽपुरुषार्थत्वात्तत्र तेषामनुष्ठानस्य भवत्यनर्थहेतुत्वमिति भावः~। {\br तस्येति~।} क्रतौ स्वस्त्रीगमनादेरित्यर्थः~। क्रतौ स्वस्त्रीगमनादेः कलञ्जभक्षणादिवदनर्थरूपनरकादिदुःखाहेतुत्वेऽपि
क्रतुवैगुण्यसंपादकत्वेनानर्थहेतुत्वसंभवाद्भवत्यत्रापि निषिध्यमानस्यानर्थहेतुत्वमिति भावः~। {\ab तस्मान्निषेधवाक्यानामनर्थहेतुक्रियानिवृत्तिजनकत्वेनैव पुरुषार्थानुबन्धित्वमिति सिद्धम्~॥}
\newpage
%%%%%%%%%%%%%%%%%%%%%%%%%%%%%%%%
\fancyhead[LO]{[ अर्थवादविभागः ]}
\begin{center}
 \textbf{अर्थवादमीमांसा }   
\end{center}
 
{\bl प्राशस्त्यनिन्दातरपरं वाक्यमर्थवादः~। तस्य च लक्षणया  प्रयोजनवदर्थपर्यवसानम्~। तथा हि\textendash अर्थवादवाक्यं हि स्वार्थप्रतिपादने प्रयोजनाभावाद्विधेयनिषेध्ययोः\footnotemark\ प्राशस्त्यनिन्दितत्वे लक्षणया प्रतिपादयति~। स्वार्थमात्रपरत्वे आनर्थक्यप्रसङ्गात्~। आम्नायस्य हि क्रियार्थत्वात्~। न चेष्टापत्तिः; {\qtl स्वाध्यायोऽध्येतव्यः} इत्यध्ययनविधिना सकलवेदाध्ययनं कर्तव्यमिति बोधयता सर्ववेदस्य प्रयोजनवदर्थपर्यवसायित्वं सूचयतोपात्तत्वेनानर्थक्यानुपपत्तेः~।~}
\begin{center}
 \textbf{अर्थवादविभागः }   
\end{center}
 
{\bl स द्विविधः {\al विधिशेषो निषेधशेषश्चेति}~। तत्र {\qtl वायव्यं }}\\
\hrule
\vspace{3mm}
\noindent
 इदानीं सर्वार्थवादसाधारण्येनार्थवादरूपवेदभागस्य लक्षणमाह\textendash  {\br प्राशस्त्येति~।} {\br ननु} {\qt वायुर्वै क्षेपिष्ठा देवता}
इत्यादेरर्थवादवाक्यात्प्राशस्त्यादेरप्रतीयमानत्वान्न तस्य प्राशस्त्यादिपरत्वं ततश्चासंभविलक्षणमिदमित्यत आह\textendash {\br तस्येति~।} तस्यार्थवादस्य लक्षणया प्राशस्त्यनिन्दान्यतरप्रतिपादकत्वसंभवेन नेदं लक्षणमसंभवीत्यर्थः~। अर्थवादस्य फलवदर्थपर्यवसानमेवोपपादयति {\br तथा हीत्यादिना~।}  अर्थवादवाक्यं कर्तृ प्राशस्त्यनिन्दितत्वे द्वे प्रतिपादयतीत्यन्वयः~। \textbf{नन्व}र्थवादस्य  स्वार्थमात्रपरत्वे कथमानर्थक्यप्रसङ्गो {\qt वायुर्वै क्षेपिष्ठा} इत्यादेरर्थवादवाक्यात्क्षिप्रगामिवायुदेवतारूपार्थस्य प्रतीयमानत्वादित्यत आह\textendash {\br आम्नायस्येति~।} समस्तवेदस्य क्रियाप्रतिपादकत्वस्वीकारेण नास्य सिद्धार्थप्रतिपादकस्य स्वार्थमात्रपरत्वे सार्थक्यमुपपद्यत इति भावः~।अर्थवादवाक्यानामनर्थकत्वस्येष्टत्वमाशङ्क्य परिहरति {\br न चेत्यादिना~।} \\

 अर्थवादं विभजते {\br स द्विविध इति~। विधिशेष इत्यादि~।} सोऽपि चतुर्विधः,\textendash प्रशंसा-निन्दा-परकृति-पुराकल्प-भेदात्~। तत्र प्रशंसार्थवादो हि  {\qt शोभतेऽस्य मुखं य एवं वेद} इत्यादिः {\qt वायुर्वै क्षेपिष्टा} इत्यादिर्मूलोदाहृतश्च~।
\blfootnote{पाठा०\textemdash\ $^{१}$निषिध्यमानयो~।}
\newpage
%%%%%%%%%%%%%%%%%%%%%%%%%%%%%%%%%%%%%%
\fancyhead[RE]{[ अर्थवादविभागः ]}
{\bl\noindent {\qtl श्वेतमालभेत भूतिकामः} इत्यादिविधिशेषस्य {\qtl वायुर्वै क्षेपिष्ठा देवता} इत्यादेर्विधेयार्थप्राशस्त्यबोधकतयार्थवत्त्वम्~। {\qtl बर्हिषि 
रजतं न देयम्} इत्यादिनिषेधशेषस्य, {\qtl सोऽरोदीद्यदरोदीत्तद्रुद्रस्य  रुद्रत्वम्} इत्यादेर्निषेध्यस्य निन्दितत्वबोधकतयार्थवत्त्वम्~। न च  प्राशस्त्यादिबोधस्य निष्प्रयोजनत्वेन नार्थवादस्यार्थवत्त्वमिति  वाच्यम् ; आलस्यादिवशादप्रवर्तमानस्य पुंसः प्रवृत्त्यादिजनकत्वेन तद्बोधस्योपयोगात्~।~}\\
\hrule
\vspace{3mm}
\noindent
निन्दार्थवादस्तु {\qt असत्रं वा एतद्यदच्छन्दोमम्, अश्रुजं हि रजतं यो बर्हिषि ददाति} इत्यादिः सोऽरोदीदित्यादिर्मूलोदाहृतश्च भवति~। परेणेदं महता पुरुषेण कर्म कृतमिति प्रतिपादकोऽर्थवादः परकृतिरित्युच्यते यथा {\qt अग्निर्वा अकामयतः} इत्यादिः~। परवक्तृकार्थादिप्रतिपादकोऽर्थवादः पुराकल्प इत्युच्यते~। यथा  {\qt तमशयद्धिया धियात्वावध्यासुः} इत्यादिरिति~। {\br विधेयार्थप्राशस्त्येति~।} वायुः क्षिप्रगामित्वादतीव प्रशस्ता देवता भवत्यतस्तद्दैवत्यं कर्म प्रशस्तमिति विधेयकर्मदेवतागतप्राशस्त्यप्रतिपादनद्वारा विध्येकवाक्यत्वेनार्थवत्त्वमित्यर्थः~। {\br निषेध्येति~।} निषेध्यरजतनिन्दाप्रतिपादनद्वारा रजतनिषेधप्राशस्त्यपरत्वेन निषेधैकवाक्यत्वादर्थवत्त्वमित्यर्थः~। एवम् {\qt अग्निर्वा अकामयत} इत्यादेरप्यग्निदैवत्यो यागः पूर्वकालेऽग्निना कृतत्वात्प्रशस्तोऽत आधिक्यादिदानींतनैरप्यन्यैर्यजमानैरवश्यं कर्तव्यमिति विधेयकर्मगतप्राशस्त्यप्रतिपादनद्वारा विध्येकवाक्यत्वेनार्थवत्त्वं बोध्यम्~। एवमन्यत्रापि तत्तद्विधेयकर्मप्राशस्त्यादिप्रतिपादनद्वारेण तत्तद्विध्याद्येकवाक्यत्वेनार्थवत्त्वं विभावनीयम्~। क्वचिदर्थवादस्य संदिग्धार्थनिर्णायकत्वेन प्रामाण्यं स्वीकृतम्~। यथा {\qt अक्ताः शर्करा उपदध्यात्}  इति विधावक्ता इति पदेन द्रवद्रव्यसामान्यं प्रतीयते~। तच्च द्रव्यं किमिति संदेहे {\qt तेजो वै घृतम्} श्मित्यर्थवादरूपाद्वाक्यशेषाद्धृतमिति निश्चीयत इति~। {\br ननु} विधेयकर्मप्राशस्त्यादिबोधस्य सुखदुःखाभावान्यतरविलक्षणत्वेन नार्थवादस्य तत्प्रतिपादनेनार्थवत्त्वमुपपद्यत इत्याशङ्क्य परिहरति {\br न चेत्यादिना~।तद्वोधस्योपयोगादिति~।} अर्थवादजन्यप्राशस्त्यादिबोधस्य विधेयकर्मानुष्ठानादावुपयोगादर्थवादस्यार्थवत्त्वमुपपद्यत इत्यर्थः~।
\newpage
%%%%%%%%%%%%%%%%%%%%%%%%%%%%%
\fancyhead[LO]{[ ग्रन्थोपसंहारः ]}
\begin{center}
 \textbf{अर्थवादस्य भेदत्रयम् }   
\end{center}
 
{\bl स पुनस्त्रेधा~। तदुक्तम्\textendash {\qtl विरोधे गुणवादःस्यादनुवादोऽवधारिते~। भूतार्थवादस्तद्धानादर्थवादस्त्रिधा मतः}  इति~। अस्यार्थः {\qtl प्रमाणान्तरविरोधे सत्यर्थवादो गुणवादः}, यथा {\qtl आदित्यो यूपः} इत्यादिः~। यूप आदित्याभेदस्य प्रत्यक्षबाधितत्वादादित्यवदुज्ज्वलत्वरूपगुणोऽनेन लक्षणया प्रतिपाद्यते~। {\qtl प्रमाणान्तरावगतार्थबोधकोऽर्थवादोऽनुवादः}, यथा {\qtl अग्निर्हिमस्य भेषजम्}  इति,  अत्र हिमविरोधित्वस्याग्नौ प्रत्यक्षावगतत्वात्~। प्रमाणान्तरविरोधतत्प्राप्तिरहितार्थबोधकोऽर्थवादो भूतार्थवादः, {\qtl यथा इन्द्रो वृत्राय वज्रमुदयच्छत्} इत्यादिः~।}
\begin{center}
 \textbf{अर्थवादस्य भेदत्रयम् }   
\end{center}
 
{\bl एवं च {\qtl यजेत स्वर्गकामः} इत्यादिनिखिलवेदस्य साक्षात्पर-\\}
\hrule
\vspace{3mm}

 स पुनरर्थवादो गुणवादोऽनुवादो भूतार्थवादश्चेति त्रिधा भवतीति पुनरपि तं त्रेधा विभजते {\br स पुनस्त्रेधेति~।} तस्य त्रिविधत्वे वृद्धसंमतिमुदाहरति {\br तदुक्तमिति~। तद्धानादिति~।} तयोर्हीनं तद्धानं तस्मादिति विग्रहः~। प्रमाणान्तरविरोधप्रमाणान्तरावधारणयोर्हानादित्यर्थः~। समुदाहृतवृद्धसंमतिं  व्याचष्टे {\br अस्यार्थ इत्यादिना~। इन्द्रो वृत्रायेति~।} अस्य वृत्रं प्रतीन्द्रवज्रोद्यमनाभावावगाहिप्रमाणान्तरस्यादर्शनान्न तद्बोधने प्रमाणान्तरेण विरोधः, नापि प्रमाणान्तरावगतार्थप्रतिपादकत्वं वृत्रं
प्रतीन्द्रवज्रोद्यमनप्रतिपादकप्रमाणान्तरस्यादर्शनादिति भावः~।\\

 एवमुपपादितं विध्यर्थवादादिरूपस्य पञ्चविधवेदस्य साक्षात्परम्परया यागादिधर्मप्रतिपादकत्वेनार्थवत्त्वमुपसंहरति {\br एवं चेति~।} उक्तेन प्रकारेण चेत्यर्थः~। योऽयं यागादिरूपो धर्मो यत्स्वर्गादिफलमुद्दिश्य विहितः सोऽयं तादृशफलोद्देशेनानुष्ठीयमान एव तस्य फलस्य हेतुर्भवति, परमेश्वरसमर्पणमनीषयाऽनुष्ठीयमानस्तु  चित्तशुद्धितत्त्वज्ञानप्राप्तिद्वारा परम्परया मोक्षहेतुरिति धर्मानुष्ठानस्य विविदिषा-
\newpage
%%%%%%%%%%%%%%%%%%%%%%%%%%%%%%%%%%
\fancyhead[RE]{[ ग्रन्थोप\textemdash\ }
{\bl\noindent 
म्परया वा यागादिधर्मप्रतिपादकत्वं सिद्धम्~। सोऽयं धर्मो  यदुद्दिश्य विहितस्तदुद्देशेन क्रियमाणस्तद्धेतुः~। ईश्वरार्पणबुद्ध्या  क्रियमाणस्तु निःश्रेयसहेतुः~। न च तदर्पणबुद्ध्यानुष्ठाने प्रमाणाभावः {\qt यत्करोषि यदश्नासि यज्जुहोषि ददासि यत्~। यत्तपस्यसि कौन्तेय तत्कुरुष्व मदर्पणम्}  इति भगवद्गीतास्मृतेरेव प्रमाणत्वात्~। स्मृतिचरणे तत्प्रामाण्यस्य श्रुतिमूलकत्वेन व्यवस्थापनादिति शिवम्~॥}
\begin{quote}
    \al  बालानां सुखबोधाय भास्करेण सुमेधसा~।\\
 रचितोऽयं समासेन जैमिनीयार्थसंग्रहः~॥ १~॥  
\end{quote}
 \begin{center}
     \bl
इति श्रीमहामहोपाध्यायलौगाक्षिभास्करविरचितपूर्वमीमांसार्थसंग्रहनामकं प्रकरणं समाप्तिमगात्~॥\\
\rule{.15\linewidth}{.5pt}
 \end{center}
\hrule
\vspace{3mm}
\noindent
श्रुतिप्रामाण्यात्संयोगपृथक्त्वन्यायमाश्रित्य परमपुरुषार्थपर्यवसायित्वमुपक्षिपति {\br सोऽयमित्यादि~।} धर्मस्य परमेश्वरसमर्पणबुद्ध्यानुष्ठाने प्रमाणाभावमाशङ्क्य तत्र भगवद्वाक्यं प्रमाणयति {\br न चेत्यादिना~।} न च तस्याः स्मृतेरप्रामाण्यमाशङ्क्यम्~। तन्त्रे स्मृतिपादे स्थितम् {\qt अष्टकाः कर्तव्याः} इत्यादिस्मृतिवाक्यं न धर्मे प्रमाणं, पौरुषेयवाक्यत्वे सति मूलप्रमाणरहितत्वाद्विप्रलम्भकवाक्यवत्~।
\noindent अथ\textendash  मूलप्रमाणवत्त्वाय वेदार्थ एव स्मृतिभिरुच्यत इति मन्येथाः तर्हि वेदेनैव तदर्थस्यावगतत्वादियं स्मृतिरनर्थिका स्यात्, तदानीमनुवादकत्वादप्रामाण्यमिति प्राप्ते ब्रूमः {\qt विमता स्मृतिर्वेदमूला , वैदिकमन्वादिप्रणीतस्मृतित्वात् , उपनयनाध्ययनादिस्मृतिवत्~}। न च वैयर्थ्यं शङ्कनीयम्~। अस्मदादीनां प्रत्यक्षेषु परोक्षेषु  नानावेदेषु विप्रकीर्णस्यानुष्ठेयार्थस्यैकत्र संक्षिप्यमाणत्वात्~। तस्मादियं स्मृतिर्धर्मे प्रमाणमिति~।~\\

  {\br ननु} किमनेन संग्रहनिरूपणेन विस्तृतभाष्यादिग्रन्थैरेव जैमिनीयन्यायार्थबोधसंभवादित्याशङ्क्य संग्रहनिरूपणे स्वप्रवृत्तिनिमित्तं प्रदर्शयति {\br बालानामिति~।} तेषां विस्तृतत्वेन दुःखग्राह्यत्वान्नतैर्बालशब्दितपूर्वमीमांसासंस्कार-
\newpage
%%%%%%%%%%%%%%%%%%%%%%%%%%%%%%%
\fancyhead[LO]{संहारः ]}
\hrule
\vspace{3mm}
\noindent
शून्यानां जैमिनीयन्यायार्थबोधः संभवति~। अनेन तु तत्तद्विस्तृतशास्त्रप्रवेशद्वारा संभवत्येव स इति भावः~॥~
\begin{quote}
    \ab 
 टीकाविहीने तु कृता हि टीका पूर्वे तु तन्त्रे खलु संग्रहेऽस्मिन्~।\\
 दुर्बोधशास्त्रे किमु मादृशानां दृष्ट्वापि दोषं न सहन्ति धीराः~॥ १~॥\\

 यदाज्ञया बन्धविमोक्षणं विना स्वधर्मसेवा फलहेतुतां गता~।\\
 प्रणौमि सोमं मृडमादिकारणं किमन्यदेवैर्मनफल्गुहेतुभिः~॥ २~॥~\\

मदीययत्नः शिवपादसेवया गुरोः कटाक्षैकलवेन लब्धया~। \\
 प्रयुज्यमानः शिवपादपङ्कजे स्वयं तु भूयान्मृडतोषकारणम्~॥ ३~॥ \\

सुजनपदविनीतो दुर्जनाद्दूरनिष्ठो गुरुतरशिवभक्तस्तेन लब्धागमेक्षः~।\\
 श्रुतिमणिपदनिष्ठो भिक्षुरामेश्वराख्यः सुजननयनिवेशाय प्रबन्धं चकार~॥ ४~॥~\\

गुणगणमणिसिन्धुः शम्भुपादैकभक्तो निगमशिरसि निष्ठो जातवैराग्यचित्तः~।~\\
 श्रुतिनलिनविकासे भानुभावो य ईशस्तमिह महिमपूज्यं नौमि \footnotemarkA[1]गोपं
यतीन्द्रम्~॥ ५~॥~\\

तस्मादेव गुरुवरादभीष्टलब्धं गोपालाश्रमपदगीयमानदैवात्~।\\
 येनोमाधवचरणाब्जसेविनात्र तं वन्दे महिमगुरुं विशालबुद्धिम्~॥ ६~॥\\

 या काशी निखिलगुरोर्महेश्वरस्य प्राणान्ते सकलशिवप्रदा प्रसिद्धा~। \\
 यत्राहं सकलसुरेशलब्धतत्त्वस्तत्रेयं सुजनहितप्रदा निबद्धा~॥ ७~॥    
\end{quote}
\begin{center}
 इति श्रीमत्परमहंसपरिव्राजकाचार्यगोपालेन्द्रसरस्वतीश्रीपूज्यपाद-\\
 शिष्यश्रीसदाशिवेन्द्रसरस्वतीशिष्येण श्रीरामेश्वरेण \\
 शिवयोगिभिक्षुणा विरचिता मीमांसार्थ-\\
 संग्रहकौमुदी चरमवर्ण-\\
 ध्वंसमगात्~।  \\
\vspace{5mm}
 \fbox{\bl ~~सप्तमोऽयं ग्रन्थः~।~~}
\end{center}

\alfootnote{टिप्प०\textemdash\ $^{1}$गोपालमित्यर्थः~।}
\newpage
%%%%%%%%%%%%%%%%%%%%%%%%%%%%%%%%%%%%%%%%
\pagestyle{empty}
\hfill
\newpage
%%%%%%%%%%%%%%%%%%%%%%%%%%%%%%%%%%%%%%%%%%5
 \begin{center}
\Large
 \textbf{कतिपयाः संग्राह्यग्रन्थाः}
 \end{center}
 
\begin{description}
\item[मीमांसापरिभाषा \textemdash\ ] श्रीमत्कृष्णयज्वप्रणीता~।\\
\hspace*{\fill} \textbf{मूल्यम् ९ आणकाः }\\
{\hspace*{\fill}\bld . \hspace*{\fill}}\\

\item[मृच्छकटिकम् \textemdash\ ] शूद्रककविप्रणीतम् ; मृच्छकटिकविवृति टीकया,
 विषमस्थलटिप्पण्या पाठान्तर-पात्रपरिचय-प्रत्यङ्ककथावस्तुस्थलनिर्देशादिविविधविषयैः समलङ्कृतम्~।~\\
\hspace*{\fill} \textbf{मूल्यम् ३~। रूप्यकाः}\\
{\hspace*{\fill}\bld . \hspace*{\fill}}\\

\item[अभिज्ञानशाकुन्तलम् \textemdash\ ]
 महाकविकालिदासविरचितं, राघवभट्टकृतार्थद्योतनिकाव्याख्यया वङ्गीयप्राच्यादर्शगतपाठान्तरपरिशिष्ट-सूच्यादिभिश्च सनाथीकृतम्~।\\
\hspace*{\fill} \textbf{मूल्यम् ३॥ रूप्यकाः}\\
{\hspace*{\fill}\bld . \hspace*{\fill}}\\

\item[रघुवंशम् \textemdash\ ] महाकविकालिदासविरचितं, मल्लिनाथकृतसंजीविनीटीकया, वल्लभ-हेमाद्रि-दिनकरमिश्र-चारित्रवर्धन -सुमतिविजयादिटीकाविशिष्टांशैः, रघुवंशसार-पाठान्तर-विविधपरिशिष्टादिभिः, \textbf{प्रो० हरि दामोदर वेलणकर एम्. ए.}
 इत्येतेषां सहृदयहृदयाह्लादकया भूमिकया च समुल्लसितम्~।\\
\hspace*{\fill} \textbf{मूल्यम् ४॥ रूप्यकाः}\\
{\hspace*{\fill}\bld . \hspace*{\fill}}\\

\item[कर्पूरमञ्जरी \textemdash\ ]
 महाकविश्रीराजशेखरकृता, वासुदेवकृतया व्याख्यया, पाठान्तर-परिशिष्टादिभिः समेता~।~\\
\hspace*{\fill} \textbf{मूल्यम् २॥ रूप्यकौ}\\
 {\hspace*{\fill}\bld . \hspace*{\fill}}\\
\end{description}
\newpage
%%%%%%%%%%%%%%%%%%%%%%%%%%%%
\pagestyle{fancy}
\cfoot{}
\chead{२}
\begin{description}
\item[उत्तररामचरितम् \textendash]  महाकविभवभूतिविरचितम् , वीरराघवकृतटीकया, पाठान्तर-टिप्पणी-पात्रपरिचय- परिशिष्टादिभिश्च 
 समेतम् , प्रत्यङ्कप्रारंभे तस्य तस्याङ्कस्य कथावस्तु- स्थलकालनिर्देशैश्च सम्यगलंकृतं वरीवर्ति~।~\\
\hspace*{\fill} मूल्यम् २~॥ रूप्यकौ \\
{\hspace*{\fill}\bld . \hspace*{\fill}}\\

\item[गीतगोविन्दम् \textemdash\ ] जयदेवकविविरचितम् , \textbf{रसिकप्रियारसमञ्जरी}व्याख्याभ्यां \textbf{दीपिका-पदद्योतनिका-सञ्जीविनी- बालवोधिनी} प्रभृत्यद्यावध्यप्रकाशितटीकागतविशिष्टांशैः, तत्तद्व्याख्यानसंमतपाठान्तरैः, उपोद्धातपरिशिष्टादिभिश्च समेतमभिनवं संस्करणम्~।
\hspace*{\fill} \textbf{मूल्यम् ३ रूप्यकाः}\\
{\hspace*{\fill}\bld . \hspace*{\fill}}\\

\item[स्वप्नवासवदत्तम् \textemdash\ ] दत्तवाडकरोपाह्वेन पुरुषोत्तमशास्त्रिणा प्रणीतया टीकयालङृतम् । व्याख्यानावसरेऽत्र साहित्यदृष्ट्या 
 ध्वन्यालङ्कारवृत्तानां यथावन्निर्देशं कृत्वा रूपप्रकरणान्तर्भाविनां पदार्थानामपि लक्षणादीनि स्थाने स्थाने वितीर्णानि, 
 अङ्कस्य प्रारम्भे च तस्य तस्य विषयः संक्षेपेण सुलभतया 
 प्रतिपादितोऽस्ति~।~\\
\hspace*{\fill} \textbf{मूल्यम् २॥ रूप्यकौ}
\end{description}
\begin{center}
\bl
  नि र्ण य सा ग र मु द्रणा ल य म्, मुंबई २ 
\end{center}
 \end{document}